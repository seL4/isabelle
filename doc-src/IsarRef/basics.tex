
\chapter{Basic Concepts}\label{ch:basics}

\section{Isabelle/Isar theories}

Isabelle/Isar offers two main improvements over classic Isabelle:
\begin{enumerate}
\item A new \emph{theory format}, occasionally referred to as ``new-style
  theories'', supporting interactive development and unlimited undo operation.
\item A \emph{formal proof document language} designed to support intelligible
  semi-automated reasoning.  Instead of putting together unreadable tactic
  scripts, the author is enabled to express the reasoning in way that is close
  to mathematical practice.
\end{enumerate}

The Isar proof language is embedded into the new theory format as a proper
sub-language.  Proof mode is entered by stating some $\THEOREMNAME$ or
$\LEMMANAME$ at the theory level, and left again with the final conclusion
(e.g.\ via $\QEDNAME$).  A few theory extension mechanisms require proof as
well, such as the HOL $\isarkeyword{typedef}$ which demands non-emptiness of
the representing sets.

New-style theory files may still be associated with an ML file consisting of
plain old tactic scripts.  There is no longer any ML binding generated for the
theory and theorems, though.  ML functions \texttt{theory}, \texttt{thm}, and
\texttt{thms} retrieve this information \cite{isabelle-ref}.  Nevertheless,
migration between classic Isabelle and Isabelle/Isar is relatively easy.  Thus
users may start to benefit from interactive theory development even before
they have any idea of the Isar proof language.

\begin{warn}
  Currently Proof~General does \emph{not} support mixed interactive
  development of classic Isabelle theory files and tactic scripts, together
  with Isar documents at the same time.  The ``\texttt{isa}'' and
  ``\texttt{isar}'' versions of Proof~General are handled as two different
  theorem proving systems, only one of these may be active.
\end{warn}

Porting of existing tactic scripts is best done by running two separate
Proof~General sessions, one for replaying the old script and the other for the
emerging Isabelle/Isar document.


\section{The Isar proof language}

Sorry, this important section has not been written yet!  Refer to
\cite{Wenzel:1999:TPHOL} for the time being.

\subsection{Commands}

\subsubsection{Isar primitives}

\subsubsection{Derived elements}


\subsection{Methods}

\subsection{Attributes}

%%% Local Variables: 
%%% mode: latex
%%% TeX-master: "isar-ref"
%%% End: 
