
\chapter{Basic Concepts}\label{ch:basics}

Isabelle/Isar offers two main improvements over classic Isabelle:
\begin{enumerate}
\item A new \emph{theory format}, often referred to as ``new-style theories'',
  supporting interactive development and unlimited undo operation.
\item A \emph{formal proof language} designed to support intelligible
  semi-automated reasoning.  Rather than putting together tactic scripts, the
  author is enabled to express the reasoning in way that is close to
  mathematical practice.
\end{enumerate}

The Isar proof language is embedded into the new theory format as a proper
sub-language.  Proof mode is entered by stating some $\THEOREMNAME$ or
$\LEMMANAME$ at the theory levels, and left with the final end of proof (e.g.\ 
via $\QEDNAME$).  Some theory extension mechanisms require proof as well, such
as the HOL $\isarkeyword{typedef}$ mechanism that only works for non-empty
representing sets.

New-style theory files may still be associated with an ML file consisting of
plain old tactic scripts.  There is no longer any ML binding generated for the
theory and theorems, though.  Functions \texttt{theory}, \texttt{thm}, and
\texttt{thms} may be used to retrieve this information from ML (see also
\cite{isabelle-ref}).  Nevertheless, migration between classic Isabelle and
Isabelle/Isar is relatively easy.  Thus users may start to benefit from
interactive theory development even before they have any idea of the Isar
proof language.

\begin{warn}
  Proof~General does \emph{not} support mixed interactive development of
  classic Isabelle theory files and tactic scripts together with Isar
  documents at the same time.  The \texttt{isa} and \texttt{isar} versions of
  Proof~General appear as two different theorem proving systems; only one
  prover may be active at any time.
\end{warn}


\section{The Isar proof language}

This rather important section has not been written yet!  Refer to
\cite{Wenzel:1999:TPHOL} for the time being.

\subsection{Commands}

\subsubsection{Isar primitives}

\subsubsection{Derived elements}


\subsection{Methods}

\subsection{Attributes}


%%% Local Variables: 
%%% mode: latex
%%% TeX-master: "isar-ref"
%%% End: 
