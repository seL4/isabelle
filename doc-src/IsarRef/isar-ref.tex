
%% $Id$

\documentclass[12pt,a4paper,fleqn]{report}
\usepackage{latexsym,graphicx,../iman,../extra,../proof,../rail,../railsetup,../isar,../pdfsetup}

\title{\includegraphics[scale=0.5]{isabelle_isar} \\[4ex] The Isabelle/Isar Reference Manual}
\author{\emph{Markus Wenzel} \\ TU M\"unchen}

\makeindex

\railterm{percent,ppercent,underscore,lbrace,rbrace,llbrace,rrbrace}
\railterm{ident,longident,symident,var,textvar,typefree,typevar,nat,string,verbatim,keyword}

\railalias{name}{\railqtoken{name}}
\railalias{nameref}{\railqtoken{nameref}}
\railalias{text}{\railqtoken{text}}
\railalias{type}{\railqtoken{type}}
\railalias{term}{\railqtoken{term}}
\railalias{prop}{\railqtoken{prop}}
\railalias{atom}{\railqtoken{atom}}

\newcommand{\drv}{\mathrel{\vdash}}
\newcommand{\edrv}{\mathop{\drv}\nolimits}
\newcommand{\Or}{\mathrel{\;|\;}}


\setcounter{secnumdepth}{2} \setcounter{tocdepth}{2}

\pagestyle{headings}
\sloppy
\binperiod     %%%treat . like a binary operator

\renewcommand{\phi}{\varphi}

%\includeonly{refcard}



\begin{document}

\underscoreoff

\maketitle 

\begin{abstract}
  \emph{Intelligible semi-automated reasoning} (\emph{Isar}) is a generic
  approach to readable formal proof documents.  It sets out to bridge the
  semantic gap between any internal notions of proof based on primitive
  inferences and tactics, and an appropriate level of abstraction for
  user-level work.  The Isar formal proof language has been designed to
  satisfy quite contradictory requirements, being both ``declarative'' and
  immediately ``executable'', by virtue of the \emph{Isar/VM} interpreter.
  
  The current version of Isabelle offers Isar as an alternative proof language
  interface layer.  The Isabelle/Isar system provides an interpreter for the
  Isar formal proof document language.  The input may consist either of proper
  document constructors, or improper auxiliary commands (for diagnostics,
  exploration etc.).  Proof texts consisting of proper document constructors
  only, admit a purely static reading, thus being intelligible later without
  requiring dynamic replay that is so typical for traditional proof scripts.
  Any of the Isabelle/Isar commands may be executed in single-steps, so
  basically the interpreter has a proof text debugger already built-in.
  
  Employing the Isar instantiation of \emph{Proof~General}, the generic Emacs
  interface for interactive proof assistants of LFCS Edinburgh, we arrive at a
  reasonable environment for \emph{live document editing}.  Thus proof texts
  may be developed incrementally by issuing proper document constructors,
  including forward and backward tracing of partial documents; intermediate
  states may be inspected by diagnostic commands.
  
  The Isar subsystem is tightly integrated into the Isabelle/Pure meta-logic
  implementation.  Theories, theorems, proof procedures etc.\ may be used
  interchangeably between classic Isabelle proof scripts and Isabelle/Isar
  documents.  Isar is as generic as Isabelle, able to support a wide range of
  object-logics.  Currently, the end-user working environment is most complete
  for Isabelle/HOL.
\end{abstract}

\pagenumbering{roman} \tableofcontents \clearfirst

%FIXME
\nocite{Rudnicki:1992:MizarOverview}
\nocite{Harrison:1996:MizarHOL}
\nocite{Rudnicki:1992:MizarOverview}
\nocite{Trybulec:1993:MizarFeatures}
\nocite{Syme:1997:DECLARE}
\nocite{Syme:1998:thesis}
\nocite{Syme:1999:TPHOL}


\chapter{Introduction}

\section{Quick start}

Isar is already part of Isabelle (as of version Isabelle99, or later).  The
\texttt{isabelle} binary provides option \texttt{-I} to run the Isar
interaction loop at startup, rather than the plain ML top-level.  Thus the
quickest way to do anything with Isabelle/Isar is as follows:
\begin{ttbox}
isabelle -I HOL\medskip
\out{> Welcome to Isabelle/HOL (Isabelle99)}\medskip
theory Foo = Main:
constdefs foo :: nat  "foo == 1";
lemma "0 < foo" by (simp add: foo_def);
end
\end{ttbox}
Note that any Isabelle/Isar command may be retracted by \texttt{undo}; the
\texttt{help} command prints a list of available language elements.

Plain TTY-based interaction like this used to be quite feasible with
traditional tactic based theorem proving, but developing Isar documents
demands some better user-interface support.  \emph{Proof~General}\index{Proof
  General} of LFCS Edinburgh \cite{proofgeneral} offers a generic Emacs-based
environment for interactive theorem provers that does all the cut-and-paste
and forward-backward walk through the text in a very neat way.  Note that in
Isabelle/Isar, the current position within a partial proof document is equally
important than the actual proof state.  Thus Proof~General provides the
canonical working environment for Isabelle/Isar, both for getting acquainted
(e.g.\ by replaying existing Isar documents) and real production work.

\medskip

The easiest way to use Proof~General is to make it the default Isabelle user
interface.  Just put something like this into your Isabelle settings file (see
also \cite{isabelle-sys}):
\begin{ttbox}
ISABELLE_INTERFACE=\$ISABELLE_HOME/contrib/ProofGeneral/isar/interface
PROOFGENERAL_OPTIONS="-u false"
\end{ttbox}
You may have to change \texttt{\$ISABELLE_HOME/contrib/ProofGeneral} to the
actual installation directory of Proof~General.  From now on, the capital
\texttt{Isabelle} executable refers to the \texttt{ProofGeneral/isar}
interface.\footnote{There is also a \texttt{ProofGeneral/isa} interface, for
  classic Isabelle tactic scripts.}  Its usage is as follows:
\begin{ttbox}
Usage: interface [OPTIONS] [FILES ...]

  Options are:
    -l NAME      logic image name (default $ISABELLE_LOGIC=HOL)
    -p NAME      Emacs program name (default xemacs)
    -u BOOL      use .emacs file (default true)
    -w BOOL      use window system (default true)

  Starts Proof General for Isabelle/Isar with proof documents FILES
  (default Scratch.thy).

  PROOFGENERAL_OPTIONS=
\end{ttbox} %$
Apart from the command line, the defaults for these options may be overridden
via the \texttt{PROOFGENERAL_OPTIONS} setting as well.  For example, plain GNU
Emacs may be configured as follows:
\begin{ttbox}
PROOFGENERAL_OPTIONS="-u false -p emacs"
\end{ttbox}

Occasionally, a user's \texttt{.emacs} file contains material that is
incompatible with the version of (X)Emacs that Proof~General prefers.  Then
proper startup may be still achieved by using the \texttt{-u false}
option.\footnote{Any Emacs lisp file \texttt{proofgeneral-settings.el}
  occurring in \texttt{\$ISABELLE_HOME/etc} or
  \texttt{\$ISABELLE_HOME_USER/etc} is automatically loaded by the
  Proof~General interface script as well.}

\medskip

With the proper Isabelle interface setup, Isar documents may now be edited by
visiting appropriate theory files, e.g.\ 
\begin{ttbox}
Isabelle \({\langle}isabellehome{\rangle}\)/src/HOL/Isar_examples/BasicLogic.thy
\end{ttbox}
Users of XEmacs may note the tool bar for navigating forward and backward
through the text.  Consult the Proof~General documentation \cite{proofgeneral}
for further basic command sequences, such as ``\texttt{c-c return}'' or
``\texttt{c-c u}''.


\section{Isabelle/Isar theories}

Isabelle/Isar offers two main improvements over classic Isabelle:
\begin{enumerate}
\item A new \emph{theory format}, occasionally referred to as ``new-style
  theories'', supporting interactive development and unlimited undo operation.
\item A \emph{formal proof document language} designed to support intelligible
  semi-automated reasoning.  Instead of putting together unreadable tactic
  scripts, the author is enabled to express the reasoning in way that is close
  to mathematical practice.
\end{enumerate}

The Isar proof language is embedded into the new theory format as a proper
sub-language.  Proof mode is entered by stating some $\THEOREMNAME$ or
$\LEMMANAME$ at the theory level, and left again with the final conclusion
(e.g.\ via $\QEDNAME$).  A few theory extension mechanisms require proof as
well, such as the HOL $\isarkeyword{typedef}$ which demands non-emptiness of
the representing sets.

New-style theory files may still be associated with separate ML files
consisting of plain old tactic scripts.  There is no longer any ML binding
generated for the theory and theorems, though.  ML functions \texttt{theory},
\texttt{thm}, and \texttt{thms} retrieve this information \cite{isabelle-ref}.
Nevertheless, migration between classic Isabelle and Isabelle/Isar is
relatively easy.  Thus users may start to benefit from interactive theory
development even before they have any idea of the Isar proof language at all.

\begin{warn}
  Currently Proof~General does \emph{not} support mixed interactive
  development of classic Isabelle theory files or tactic scripts, together
  with Isar documents.  The ``\texttt{isa}'' and ``\texttt{isar}'' versions of
  Proof~General are handled as two different theorem proving systems, only one
  of these may be active at the same time.
\end{warn}

Porting of existing tactic scripts is best done by running two separate
Proof~General sessions, one for replaying the old script and the other for the
emerging Isabelle/Isar document.


\section{How to write Isar proofs anyway?}

This is one of the key questions, of course.  Isar offers a rather different
approach to formal proof documents than plain old tactic scripts.  Experienced
users of existing interactive theorem proving systems may have to learn
thinking differently in order to make effective use of Isabelle/Isar.  On the
other hand, Isabelle/Isar comes much closer to existing mathematical practice
of formal proof, so users with less experience in old-style tactical proving,
but a good understanding of mathematical proof, might cope with Isar even
better.  See also \cite{Wenzel:1999:TPHOL} for further background information
on Isar.

\medskip This really is a \emph{reference manual}.  Nevertheless, we will also
give some clues of how the concepts introduced here may be put into practice.
Appendix~\ref{ap:refcard} provides a quick reference card of the most common
Isabelle/Isar language elements.  There are several examples distributed with
Isabelle, and available via the Isabelle WWW library:
\begin{center}\small
  \begin{tabular}{l}
    \url{http://www.cl.cam.ac.uk/Research/HVG/Isabelle/library/} \\
    \url{http://isabelle.in.tum.de/library/} \\
  \end{tabular}
\end{center}

See \texttt{HOL/Isar_examples} for a collection of introductory examples, and
\texttt{HOL/HOL-Real/HahnBanach} is a big mathematics application.  Apart from
browsable HTML sources, both sessions provide actual documents (in PDF).

%%% Local Variables: 
%%% mode: latex
%%% TeX-master: "isar-ref"
%%% End: 


%FIXME
%\chapter{Basic Concepts}\label{ch:basics}
%\section{The Isar proof language}

%%% Local Variables: 
%%% mode: latex
%%% TeX-master: "isar-ref"
%%% End: 


\chapter{Isar document syntax}

\section{Inner versus outer syntax}

\section{Lexical matters}

\section{Common syntax entities}

\subsection{Atoms}

\begin{rail}
  name : ident | symident | string
  ;

  nameref : name | longident
  ;

  text : nameref | verbatim
  ;
\end{rail}

\subsection{Comments}

\begin{rail}
  comment : (() | '--' text)
  ;
  interest : (() | '\%')
  ;
\end{rail}


\subsection{Sorts and arities}

\begin{rail}
  sort : nameref | lbrace (nameref * ',') rbrace
  ;
  arity : ( () | '(' (sort + ',') ')' ) sort
  ;
  simple\-arity : ( () | '(' (sort + ',') ')' ) nameref
  ;
\end{rail}


\subsection{Terms and Types}

\begin{rail}
  
\end{rail}

\subsection{Mixfix annotations}


\subsection{}

\subsection{}

\subsection{}


%%% Local Variables: 
%%% mode: latex
%%% TeX-master: "isar-ref"
%%% End: 


\chapter{Basic Isar Language Elements}\label{ch:pure-syntax}

Subsequently, we introduce the main part of Pure Isar theory and proof
commands, together with fundamental proof methods and attributes.
Chapter~\ref{ch:gen-tools} describes further Isar elements provided by generic
tools and packages (such as the Simplifier) that are either part of Pure
Isabelle or pre-installed by most object logics.  Chapter~\ref{ch:hol-tools}
refers to actual object-logic specific elements of Isabelle/HOL.

\medskip

Isar commands may be either \emph{proper} document constructors, or
\emph{improper commands}.  Some proof methods and attributes introduced later
are classified as improper as well.  Improper Isar language elements, which
are subsequently marked by $^*$, are often helpful when developing proof
documents, while their use is discouraged for the final outcome.  Typical
examples are diagnostic commands that print terms or theorems according to the
current context; other commands even emulate old-style tactical theorem
proving.


\section{Theory commands}

\subsection{Defining theories}\label{sec:begin-thy}

\indexisarcmd{header}\indexisarcmd{theory}\indexisarcmd{end}\indexisarcmd{context}
\begin{matharray}{rcl}
  \isarcmd{header} & : & \isarkeep{toplevel} \\
  \isarcmd{theory} & : & \isartrans{toplevel}{theory} \\
  \isarcmd{context}^* & : & \isartrans{toplevel}{theory} \\
  \isarcmd{end} & : & \isartrans{theory}{toplevel} \\
\end{matharray}

Isabelle/Isar ``new-style'' theories are either defined via theory files or
interactively.  Both theory-level specifications and proofs are handled
uniformly --- occasionally definitional mechanisms even require some explicit
proof as well.  In contrast, ``old-style'' Isabelle theories support batch
processing only, with the proof scripts collected in separate ML files.

The first actual command of any theory has to be $\THEORY$, starting a new
theory based on the merge of existing ones.  Just preceding $\THEORY$, there
may be an optional $\isarkeyword{header}$ declaration, which is relevant to
document preparation only; it acts very much like a special pre-theory markup
command (cf.\ \S\ref{sec:markup-thy} and \S\ref{sec:markup-thy}).  The theory
context may be also changed by $\CONTEXT$ without creating a new theory.  In
both cases, $\END$ concludes the theory development; it has to be the very
last command of any theory file.

\begin{rail}
  'header' text
  ;
  'theory' name '=' (name + '+') filespecs? ':'
  ;
  'context' name
  ;
  'end'
  ;;

  filespecs: 'files' ((name | parname) +);
\end{rail}

\begin{descr}
\item [$\isarkeyword{header}~text$] provides plain text markup just preceding
  the formal beginning of a theory.  In actual document preparation the
  corresponding {\LaTeX} macro \verb,\isamarkupheader, may be redefined to
  produce chapter or section headings.  See also \S\ref{sec:markup-thy} and
  \S\ref{sec:markup-prf} for further markup commands.
  
\item [$\THEORY~A = B@1 + \cdots + B@n\colon$] commences a new theory $A$
  based on existing ones $B@1 + \cdots + B@n$.  Isabelle's theory loader
  system ensures that any of the base theories are properly loaded (and fully
  up-to-date when $\THEORY$ is executed interactively).  The optional
  $\isarkeyword{files}$ specification declares additional dependencies on ML
  files.  Unless put in parentheses, any file will be loaded immediately via
  $\isarcmd{use}$ (see also \S\ref{sec:ML}).  The optional ML file
  \texttt{$A$.ML} that may be associated with any theory should \emph{not} be
  included in $\isarkeyword{files}$, though.
  
\item [$\CONTEXT~B$] enters an existing theory context, basically in read-only
  mode, so only a limited set of commands may be performed without destroying
  the theory.  Just as for $\THEORY$, the theory loader ensures that $B$ is
  loaded and up-to-date.
  
\item [$\END$] concludes the current theory definition or context switch.
Note that this command cannot be undone, but the whole theory definition has
to be retracted.
\end{descr}


\subsection{Theory markup commands}\label{sec:markup-thy}

\indexisarcmd{chapter}\indexisarcmd{section}\indexisarcmd{subsection}
\indexisarcmd{subsubsection}\indexisarcmd{text}\indexisarcmd{text-raw}
\begin{matharray}{rcl}
  \isarcmd{chapter} & : & \isartrans{theory}{theory} \\
  \isarcmd{section} & : & \isartrans{theory}{theory} \\
  \isarcmd{subsection} & : & \isartrans{theory}{theory} \\
  \isarcmd{subsubsection} & : & \isartrans{theory}{theory} \\
  \isarcmd{text} & : & \isartrans{theory}{theory} \\
  \isarcmd{text_raw} & : & \isartrans{theory}{theory} \\
\end{matharray}

Apart from formal comments (see \S\ref{sec:comments}), markup commands provide
a structured way to insert text into the document generated from a theory (see
\cite{isabelle-sys} for more information on Isabelle's document preparation
tools).

\railalias{textraw}{text\_raw}
\railterm{textraw}

\begin{rail}
  ('chapter' | 'section' | 'subsection' | 'subsubsection' | 'text' | textraw) text
  ;
\end{rail}

\begin{descr}
\item [$\isarkeyword{chapter}$, $\isarkeyword{section}$,
  $\isarkeyword{subsection}$, and $\isarkeyword{subsubsection}$] mark chapter
  and section headings.
\item [$\TEXT$] specifies paragraphs of plain text, including references to
  formal entities.\footnote{The latter feature is not yet supported.
    Nevertheless, any source text of the form
    ``\texttt{\at\ttlbrace$\dots$\ttrbrace}'' should be considered as reserved
    for future use.}
\item [$\isarkeyword{text_raw}$] inserts {\LaTeX} source into the output,
  without additional markup.  Thus the full range of document manipulations
  becomes available.  A typical application would be to emit
  \verb,\begin{comment}, and \verb,\end{comment}, commands to exclude certain
  parts from the final document.\footnote{This requires the \texttt{comment}
    package to be included in {\LaTeX}, of course.}
\end{descr}

Any markup command (except $\isarkeyword{text_raw}$) corresponds to a {\LaTeX}
macro with the name prefixed by \verb,\isamarkup, (e.g.\ 
\verb,\isamarkupchapter, for $\isarkeyword{chapter}$). The \railqtoken{text}
argument is passed to that macro unchanged, i.e.\ further {\LaTeX} commands
may be included here as well.

\medskip

Additional markup commands are available for proofs (see
\S\ref{sec:markup-prf}).  Also note that the $\isarkeyword{header}$
declaration (see \S\ref{sec:begin-thy}) admits to insert document markup
elements just preceding the actual theory definition.


\subsection{Type classes and sorts}\label{sec:classes}

\indexisarcmd{classes}\indexisarcmd{classrel}\indexisarcmd{defaultsort}
\begin{matharray}{rcl}
  \isarcmd{classes} & : & \isartrans{theory}{theory} \\
  \isarcmd{classrel} & : & \isartrans{theory}{theory} \\
  \isarcmd{defaultsort} & : & \isartrans{theory}{theory} \\
\end{matharray}

\begin{rail}
  'classes' (classdecl comment? +)
  ;
  'classrel' nameref '<' nameref comment?
  ;
  'defaultsort' sort comment?
  ;
\end{rail}

\begin{descr}
\item [$\isarkeyword{classes}~c<\vec c$] declares class $c$ to be a subclass
  of existing classes $\vec c$.  Cyclic class structures are ruled out.
\item [$\isarkeyword{classrel}~c@1<c@2$] states a subclass relation between
  existing classes $c@1$ and $c@2$.  This is done axiomatically!  The
  $\isarkeyword{instance}$ command (see \S\ref{sec:axclass}) provides a way to
  introduce proven class relations.
\item [$\isarkeyword{defaultsort}~s$] makes sort $s$ the new default sort for
  any type variables given without sort constraints.  Usually, the default
  sort would be only changed when defining new object-logics.
\end{descr}


\subsection{Primitive types and type abbreviations}\label{sec:types-pure}

\indexisarcmd{typedecl}\indexisarcmd{types}\indexisarcmd{nonterminals}\indexisarcmd{arities}
\begin{matharray}{rcl}
  \isarcmd{types} & : & \isartrans{theory}{theory} \\
  \isarcmd{typedecl} & : & \isartrans{theory}{theory} \\
  \isarcmd{nonterminals} & : & \isartrans{theory}{theory} \\
  \isarcmd{arities} & : & \isartrans{theory}{theory} \\
\end{matharray}

\begin{rail}
  'types' (typespec '=' type infix? comment? +)
  ;
  'typedecl' typespec infix? comment?
  ;
  'nonterminals' (name +) comment?
  ;
  'arities' (nameref '::' arity comment? +)
  ;
\end{rail}

\begin{descr}
\item [$\TYPES~(\vec\alpha)t = \tau$] introduces \emph{type synonym}
  $(\vec\alpha)t$ for existing type $\tau$.  Unlike actual type definitions,
  as are available in Isabelle/HOL for example, type synonyms are just purely
  syntactic abbreviations without any logical significance.  Internally, type
  synonyms are fully expanded.
\item [$\isarkeyword{typedecl}~(\vec\alpha)t$] declares a new type constructor
  $t$, intended as an actual logical type.  Note that object-logics such as
  Isabelle/HOL override $\isarkeyword{typedecl}$ by their own version.
\item [$\isarkeyword{nonterminals}~\vec c$] declares $0$-ary type constructors
  $\vec c$ to act as purely syntactic types, i.e.\ nonterminal symbols of
  Isabelle's inner syntax of terms or types.
\item [$\isarkeyword{arities}~t::(\vec s)s$] augments Isabelle's order-sorted
  signature of types by new type constructor arities.  This is done
  axiomatically!  The $\isarkeyword{instance}$ command (see
  \S\ref{sec:axclass}) provides a way to introduce proven type arities.
\end{descr}


\subsection{Constants and simple definitions}\label{sec:consts}

\indexisarcmd{consts}\indexisarcmd{defs}\indexisarcmd{constdefs}\indexoutertoken{constdecl}
\begin{matharray}{rcl}
  \isarcmd{consts} & : & \isartrans{theory}{theory} \\
  \isarcmd{defs} & : & \isartrans{theory}{theory} \\
  \isarcmd{constdefs} & : & \isartrans{theory}{theory} \\
\end{matharray}

\begin{rail}
  'consts' (constdecl +)
  ;
  'defs' (axmdecl prop comment? +)
  ;
  'constdefs' (constdecl prop comment? +)
  ;

  constdecl: name '::' type mixfix? comment?
  ;
\end{rail}

\begin{descr}
\item [$\CONSTS~c::\sigma$] declares constant $c$ to have any instance of type
  scheme $\sigma$.  The optional mixfix annotations may attach concrete syntax
  to the constants declared.
\item [$\DEFS~name: eqn$] introduces $eqn$ as a definitional axiom for some
  existing constant.  See \cite[\S6]{isabelle-ref} for more details on the
  form of equations admitted as constant definitions.
\item [$\isarkeyword{constdefs}~c::\sigma~eqn$] combines declarations and
  definitions of constants, using the canonical name $c_def$ for the
  definitional axiom.
\end{descr}


\subsection{Syntax and translations}\label{sec:syn-trans}

\indexisarcmd{syntax}\indexisarcmd{translations}
\begin{matharray}{rcl}
  \isarcmd{syntax} & : & \isartrans{theory}{theory} \\
  \isarcmd{translations} & : & \isartrans{theory}{theory} \\
\end{matharray}

\begin{rail}
  'syntax' ('(' name 'output'? ')')? (constdecl +)
  ;
  'translations' (transpat ('==' | '=>' | '<=') transpat comment? +)
  ;
  transpat: ('(' nameref ')')? string
  ;
\end{rail}

\begin{descr}
\item [$\isarkeyword{syntax}~(mode)~decls$] is similar to $\CONSTS~decls$,
  except that the actual logical signature extension is omitted.  Thus the
  context free grammar of Isabelle's inner syntax may be augmented in
  arbitrary ways, independently of the logic.  The $mode$ argument refers to
  the print mode that the grammar rules belong; unless the \texttt{output}
  flag is given, all productions are added both to the input and output
  grammar.
\item [$\isarkeyword{translations}~rules$] specifies syntactic translation
  rules (i.e.\ \emph{macros}): parse~/ print rules (\texttt{==}), parse rules
  (\texttt{=>}), or print rules (\texttt{<=}).  Translation patterns may be
  prefixed by the syntactic category to be used for parsing; the default is
  \texttt{logic}.
\end{descr}


\subsection{Axioms and theorems}

\indexisarcmd{axioms}\indexisarcmd{theorems}\indexisarcmd{lemmas}
\begin{matharray}{rcl}
  \isarcmd{axioms} & : & \isartrans{theory}{theory} \\
  \isarcmd{theorems} & : & \isartrans{theory}{theory} \\
  \isarcmd{lemmas} & : & \isartrans{theory}{theory} \\
\end{matharray}

\begin{rail}
  'axioms' (axmdecl prop comment? +)
  ;
  ('theorems' | 'lemmas') thmdef? thmrefs
  ;
\end{rail}

\begin{descr}
\item [$\isarkeyword{axioms}~a: \phi$] introduces arbitrary statements as
  axioms of the meta-logic.  In fact, axioms are ``axiomatic theorems'', and
  may be referred later just as any other theorem.
  
  Axioms are usually only introduced when declaring new logical systems.
  Everyday work is typically done the hard way, with proper definitions and
  actual proven theorems.
\item [$\isarkeyword{theorems}~a = \vec b$] stores lists of existing theorems.
  Typical applications would also involve attributes, to declare Simplifier
  rules, for example.
\item [$\isarkeyword{lemmas}$] is similar to $\isarkeyword{theorems}$, but
  tags the results as ``lemma''.
\end{descr}


\subsection{Name spaces}

\indexisarcmd{global}\indexisarcmd{local}
\begin{matharray}{rcl}
  \isarcmd{global} & : & \isartrans{theory}{theory} \\
  \isarcmd{local} & : & \isartrans{theory}{theory} \\
\end{matharray}

Isabelle organizes any kind of name declarations (of types, constants,
theorems etc.) by separate hierarchically structured name spaces.  Normally
the user never has to control the behavior of name space entry by hand, yet
the following commands provide some way to do so.

\begin{descr}
\item [$\isarkeyword{global}$ and $\isarkeyword{local}$] change the current
  name declaration mode.  Initially, theories start in $\isarkeyword{local}$
  mode, causing all names to be automatically qualified by the theory name.
  Changing this to $\isarkeyword{global}$ causes all names to be declared
  without the theory prefix, until $\isarkeyword{local}$ is declared again.
\end{descr}


\subsection{Incorporating ML code}\label{sec:ML}

\indexisarcmd{use}\indexisarcmd{ML}\indexisarcmd{ML-setup}\indexisarcmd{setup}
\begin{matharray}{rcl}
  \isarcmd{use} & : & \isartrans{\cdot}{\cdot} \\
  \isarcmd{ML} & : & \isartrans{\cdot}{\cdot} \\
  \isarcmd{ML_setup} & : & \isartrans{theory}{theory} \\
  \isarcmd{setup} & : & \isartrans{theory}{theory} \\
\end{matharray}

\railalias{MLsetup}{ML\_setup}
\railterm{MLsetup}

\begin{rail}
  'use' name
  ;
  ('ML' | MLsetup | 'setup') text
  ;
\end{rail}

\begin{descr}
\item [$\isarkeyword{use}~file$] reads and executes ML commands from $file$.
  The current theory context (if present) is passed down to the ML session,
  but may not be modified.  Furthermore, the file name is checked with the
  $\isarkeyword{files}$ dependency declaration given in the theory header (see
  also \S\ref{sec:begin-thy}).
  
\item [$\isarkeyword{ML}~text$] executes ML commands from $text$.  The theory
  context is passed in the same way as for $\isarkeyword{use}$.
  
\item [$\isarkeyword{ML_setup}~text$] executes ML commands from $text$.  The
  theory context is passed down to the ML session, and fetched back
  afterwards.  Thus $text$ may actually change the theory as a side effect.
  
\item [$\isarkeyword{setup}~text$] changes the current theory context by
  applying $text$, which refers to an ML expression of type
  \texttt{(theory~->~theory)~list}.  The $\isarkeyword{setup}$ command is the
  canonical way to initialize any object-logic specific tools and packages
  written in ML.
\end{descr}


\subsection{Syntax translation functions}

\indexisarcmd{parse-ast-translation}\indexisarcmd{parse-translation}
\indexisarcmd{print-translation}\indexisarcmd{typed-print-translation}
\indexisarcmd{print-ast-translation}\indexisarcmd{token-translation}
\begin{matharray}{rcl}
  \isarcmd{parse_ast_translation} & : & \isartrans{theory}{theory} \\
  \isarcmd{parse_translation} & : & \isartrans{theory}{theory} \\
  \isarcmd{print_translation} & : & \isartrans{theory}{theory} \\
  \isarcmd{typed_print_translation} & : & \isartrans{theory}{theory} \\
  \isarcmd{print_ast_translation} & : & \isartrans{theory}{theory} \\
  \isarcmd{token_translation} & : & \isartrans{theory}{theory} \\
\end{matharray}

Syntax translation functions written in ML admit almost arbitrary
manipulations of Isabelle's inner syntax.  Any of the above commands have a
single \railqtoken{text} argument that refers to an ML expression of
appropriate type.

\begin{ttbox}
val parse_ast_translation   : (string * (ast list -> ast)) list
val parse_translation       : (string * (term list -> term)) list
val print_translation       : (string * (term list -> term)) list
val typed_print_translation :
  (string * (bool -> typ -> term list -> term)) list
val print_ast_translation   : (string * (ast list -> ast)) list
val token_translation       :
  (string * string * (string -> string * real)) list
\end{ttbox}
See \cite[\S8]{isabelle-ref} for more information on syntax transformations.


\subsection{Oracles}

\indexisarcmd{oracle}
\begin{matharray}{rcl}
  \isarcmd{oracle} & : & \isartrans{theory}{theory} \\
\end{matharray}

Oracles provide an interface to external reasoning systems, without giving up
control completely --- each theorem carries a derivation object recording any
oracle invocation.  See \cite[\S6]{isabelle-ref} for more information.

\begin{rail}
  'oracle' name '=' text comment?
  ;
\end{rail}

\begin{descr}
\item [$\isarkeyword{oracle}~name=text$] declares oracle $name$ to be ML
  function $text$, which has to be of type
  \texttt{Sign.sg~*~Object.T~->~term}.
\end{descr}


\section{Proof commands}

Proof commands perform transitions of Isar/VM machine configurations, which
are block-structured, consisting of a stack of nodes with three main
components: logical proof context, current facts, and open goals.  Isar/VM
transitions are \emph{typed} according to the following three different modes
of operation:
\begin{descr}
\item [$proof(prove)$] means that a new goal has just been stated that is now
  to be \emph{proven}; the next command may refine it by some proof method,
  and enter a sub-proof to establish the actual result.
\item [$proof(state)$] is like an internal theory mode: the context may be
  augmented by \emph{stating} additional assumptions, intermediate results
  etc.
\item [$proof(chain)$] is intermediate between $proof(state)$ and
  $proof(prove)$: existing facts (i.e.\ the contents of the special ``$this$''
  register) have been just picked up in order to be used when refining the
  goal claimed next.
\end{descr}


\subsection{Proof markup commands}\label{sec:markup-prf}

\indexisarcmd{sect}\indexisarcmd{subsect}\indexisarcmd{subsubsect}
\indexisarcmd{txt}\indexisarcmd{txt-raw}
\begin{matharray}{rcl}
  \isarcmd{sect} & : & \isartrans{proof}{proof} \\
  \isarcmd{subsect} & : & \isartrans{proof}{proof} \\
  \isarcmd{subsubsect} & : & \isartrans{proof}{proof} \\
  \isarcmd{txt} & : & \isartrans{proof}{proof} \\
  \isarcmd{txt_raw} & : & \isartrans{proof}{proof} \\
\end{matharray}

These markup commands for proof mode closely correspond to the ones of theory
mode (see \S\ref{sec:markup-thy}).  Note that $\isarkeyword{txt_raw}$ is
special in the same way as $\isarkeyword{text_raw}$.

\railalias{txtraw}{txt\_raw}
\railterm{txtraw}

\begin{rail}
  ('sect' | 'subsect' | 'subsubsect' | 'txt' | txtraw) text
  ;
\end{rail}


\subsection{Proof context}\label{sec:proof-context}

\indexisarcmd{fix}\indexisarcmd{assume}\indexisarcmd{presume}\indexisarcmd{def}
\begin{matharray}{rcl}
  \isarcmd{fix} & : & \isartrans{proof(state)}{proof(state)} \\
  \isarcmd{assume} & : & \isartrans{proof(state)}{proof(state)} \\
  \isarcmd{presume} & : & \isartrans{proof(state)}{proof(state)} \\
  \isarcmd{def} & : & \isartrans{proof(state)}{proof(state)} \\
\end{matharray}

The logical proof context consists of fixed variables and assumptions.  The
former closely correspond to Skolem constants, or meta-level universal
quantification as provided by the Isabelle/Pure logical framework.
Introducing some \emph{arbitrary, but fixed} variable via $\FIX x$ results in
a local value that may be used in the subsequent proof as any other variable
or constant.  Furthermore, any result $\edrv \phi[x]$ exported from the
context will be universally closed wrt.\ $x$ at the outermost level: $\edrv
\All x \phi$ (this is expressed using Isabelle's meta-variables).

Similarly, introducing some assumption $\chi$ has two effects.  On the one
hand, a local theorem is created that may be used as a fact in subsequent
proof steps.  On the other hand, any result $\chi \drv \phi$ exported from the
context becomes conditional wrt.\ the assumption: $\edrv \chi \Imp \phi$.
Thus, solving an enclosing goal using such a result would basically introduce
a new subgoal stemming from the assumption.  How this situation is handled
depends on the actual version of assumption command used: while $\ASSUMENAME$
insists on solving the subgoal by unification with some premise of the goal,
$\PRESUMENAME$ leaves the subgoal unchanged in order to be proved later by the
user.

Local definitions, introduced by $\DEF{}{x \equiv t}$, are achieved by
combining $\FIX x$ with another version of assumption that causes any
hypothetical equation $x \equiv t$ to be eliminated by the reflexivity rule.
Thus, exporting some result $x \equiv t \drv \phi[x]$ yields $\edrv \phi[t]$.

\begin{rail}
  'fix' (vars + 'and') comment?
  ;
  ('assume' | 'presume') (assm comment? + 'and')
  ;
  'def' thmdecl? \\ var '==' term termpat? comment?
  ;

  var: name ('::' type)?
  ;
  vars: (name+) ('::' type)?
  ;
  assm: thmdecl? (prop proppat? +)
  ;
\end{rail}

\begin{descr}
\item [$\FIX{\vec x}$] introduces local \emph{arbitrary, but fixed} variables
  $\vec x$.
\item [$\ASSUME{a}{\vec\phi}$ and $\PRESUME{a}{\vec\phi}$] introduce local
  theorems $\vec\phi$ by assumption.  Subsequent results applied to an
  enclosing goal (e.g.\ by $\SHOWNAME$) are handled as follows: $\ASSUMENAME$
  expects to be able to unify with existing premises in the goal, while
  $\PRESUMENAME$ leaves $\vec\phi$ as new subgoals.
  
  Several lists of assumptions may be given (separated by
  $\isarkeyword{and}$); the resulting list of current facts consists of all of
  these concatenated.
\item [$\DEF{a}{x \equiv t}$] introduces a local (non-polymorphic) definition.
  In results exported from the context, $x$ is replaced by $t$.  Basically,
  $\DEF{}{x \equiv t}$ abbreviates $\FIX{x}~\ASSUME{}{x \equiv t}$, with the
  resulting hypothetical equation solved by reflexivity.
  
  The default name for the definitional equation is $x_def$.
\end{descr}

The special name $prems$\indexisarthm{prems} refers to all assumptions of the
current context as a list of theorems.


\subsection{Facts and forward chaining}

\indexisarcmd{note}\indexisarcmd{then}\indexisarcmd{from}\indexisarcmd{with}
\begin{matharray}{rcl}
  \isarcmd{note} & : & \isartrans{proof(state)}{proof(state)} \\
  \isarcmd{then} & : & \isartrans{proof(state)}{proof(chain)} \\
  \isarcmd{from} & : & \isartrans{proof(state)}{proof(chain)} \\
  \isarcmd{with} & : & \isartrans{proof(state)}{proof(chain)} \\
\end{matharray}

New facts are established either by assumption or proof of local statements.
Any fact will usually be involved in further proofs, either as explicit
arguments of proof methods, or when forward chaining towards the next goal via
$\THEN$ (and variants).  Note that the special theorem name
$this$\indexisarthm{this} refers to the most recently established facts.
\begin{rail}
  'note' thmdef? thmrefs comment?
  ;
  'then' comment?
  ;
  ('from' | 'with') thmrefs comment?
  ;
\end{rail}

\begin{descr}
\item [$\NOTE{a}{\vec b}$] recalls existing facts $\vec b$, binding the result
  as $a$.  Note that attributes may be involved as well, both on the left and
  right hand sides.
\item [$\THEN$] indicates forward chaining by the current facts in order to
  establish the goal to be claimed next.  The initial proof method invoked to
  refine that will be offered the facts to do ``anything appropriate'' (cf.\ 
  also \S\ref{sec:proof-steps}).  For example, method $rule$ (see
  \S\ref{sec:pure-meth-att}) would typically do an elimination rather than an
  introduction.  Automatic methods usually insert the facts into the goal
  state before operation.  This provides a simple scheme to control relevance
  of facts in automated proof search.
\item [$\FROM{\vec b}$] abbreviates $\NOTE{}{\vec b}~\THEN$; thus $\THEN$ is
  equivalent to $\FROM{this}$.
\item [$\WITH{\vec b}$] abbreviates $\FROM{\vec b~facts}$; thus the forward
  chaining is from earlier facts together with the current ones.
\end{descr}

Basic proof methods (such as $rule$, see \S\ref{sec:pure-meth-att}) expect
multiple facts to be given in their proper order, corresponding to a prefix of
the premises of the rule involved.  Note that positions may be easily skipped
using something like $\FROM{\text{\texttt{_}}~a~b}$, for example.  This
involves the trivial rule $\PROP\psi \Imp \PROP\psi$, which happens to be
bound in Isabelle/Pure as ``\texttt{_}''
(underscore).\indexisarthm{_@\texttt{_}}


\subsection{Goal statements}

\indexisarcmd{theorem}\indexisarcmd{lemma}
\indexisarcmd{have}\indexisarcmd{show}\indexisarcmd{hence}\indexisarcmd{thus}
\begin{matharray}{rcl}
  \isarcmd{theorem} & : & \isartrans{theory}{proof(prove)} \\
  \isarcmd{lemma} & : & \isartrans{theory}{proof(prove)} \\
  \isarcmd{have} & : & \isartrans{proof(state) ~|~ proof(chain)}{proof(prove)} \\
  \isarcmd{show} & : & \isartrans{proof(state) ~|~ proof(chain)}{proof(prove)} \\
  \isarcmd{hence} & : & \isartrans{proof(state)}{proof(prove)} \\
  \isarcmd{thus} & : & \isartrans{proof(state)}{proof(prove)} \\
\end{matharray}

Proof mode is entered from theory mode by initial goal commands $\THEOREMNAME$
and $\LEMMANAME$.  New local goals may be claimed within proof mode as well.
Four variants are available, indicating whether the result is meant to solve
some pending goal or whether forward chaining is indicated.

\begin{rail}
  ('theorem' | 'lemma') goal
  ;
  ('have' | 'show' | 'hence' | 'thus') goal
  ;

  goal: thmdecl? proppat comment?
  ;
\end{rail}

\begin{descr}
\item [$\THEOREM{a}{\phi}$] enters proof mode with $\phi$ as main goal,
  eventually resulting in some theorem $\turn \phi$ to be put back into the
  theory.
\item [$\LEMMA{a}{\phi}$] is similar to $\THEOREMNAME$, but tags the result as
  ``lemma''.
\item [$\HAVE{a}{\phi}$] claims a local goal, eventually resulting in a
  theorem with the current assumption context as hypotheses.
\item [$\SHOW{a}{\phi}$] is similar to $\HAVE{a}{\phi}$, but solves some
  pending goal with the result \emph{exported} into the corresponding context
  (cf.\ \S\ref{sec:proof-context}).
\item [$\HENCENAME$] abbreviates $\THEN~\HAVENAME$, i.e.\ claims a local goal
  to be proven by forward chaining the current facts.  Note that $\HENCENAME$
  is also equivalent to $\FROM{this}~\HAVENAME$.
\item [$\THUSNAME$] abbreviates $\THEN~\SHOWNAME$.  Note that $\THUSNAME$ is
  also equivalent to $\FROM{this}~\SHOWNAME$.
\end{descr}


\subsection{Initial and terminal proof steps}\label{sec:proof-steps}

\indexisarcmd{proof}\indexisarcmd{qed}\indexisarcmd{by}
\indexisarcmd{.}\indexisarcmd{..}\indexisarcmd{sorry}
\begin{matharray}{rcl}
  \isarcmd{proof} & : & \isartrans{proof(prove)}{proof(state)} \\
  \isarcmd{qed} & : & \isartrans{proof(state)}{proof(state) ~|~ theory} \\
  \isarcmd{by} & : & \isartrans{proof(prove)}{proof(state) ~|~ theory} \\
  \isarcmd{.\,.} & : & \isartrans{proof(prove)}{proof(state) ~|~ theory} \\
  \isarcmd{.} & : & \isartrans{proof(prove)}{proof(state) ~|~ theory} \\
  \isarcmd{sorry} & : & \isartrans{proof(prove)}{proof(state) ~|~ theory} \\
\end{matharray}

Arbitrary goal refinement via tactics is considered harmful.  Properly, the
Isar framework admits proof methods to be invoked in two places only.
\begin{enumerate}
\item An \emph{initial} refinement step $\PROOF{m@1}$ reduces a newly stated
  goal to a number of sub-goals that are to be solved later.  Facts are passed
  to $m@1$ for forward chaining, if so indicated by $proof(chain)$ mode.
  
\item A \emph{terminal} conclusion step $\QED{m@2}$ is intended to solve
  remaining goals.  No facts are passed to $m@2$.
\end{enumerate}

The only other proper way to affect pending goals is by $\SHOWNAME$, which
involves an explicit statement of what is to be solved.

\medskip

Also note that initial proof methods should either solve the goal completely,
or constitute some well-understood reduction to new sub-goals.  Arbitrary
automatic proof tools that are prone leave a large number of badly structured
sub-goals are no help in continuing the proof document in any intelligible
way.

\medskip

Unless given explicitly by the user, the default initial method is ``$rule$'',
which applies a single standard elimination or introduction rule according to
the topmost symbol involved.  There is no separate default terminal method.
Any remaining goals are always solved by assumption in the very last step.

\begin{rail}
  'proof' interest? meth? comment?
  ;
  'qed' meth? comment?
  ;
  'by' meth meth? comment?
  ;
  ('.' | '..' | 'sorry') comment?
  ;

  meth: method interest?
  ;
\end{rail}

\begin{descr}
\item [$\PROOF{m@1}$] refines the goal by proof method $m@1$; facts for
  forward chaining are passed if so indicated by $proof(chain)$ mode.
\item [$\QED{m@2}$] refines any remaining goals by proof method $m@2$ and
  concludes the sub-proof by assumption.  If the goal had been $\SHOWNAME$ (or
  $\THUSNAME$), some pending sub-goal is solved as well by the rule resulting
  from the result \emph{exported} into the enclosing goal context.  Thus
  $\QEDNAME$ may fail for two reasons: either $m@2$ fails, or the resulting
  rule does not fit to any pending goal\footnote{This includes any additional
    ``strong'' assumptions as introduced by $\ASSUMENAME$.} of the enclosing
  context.  Debugging such a situation might involve temporarily changing
  $\SHOWNAME$ into $\HAVENAME$, or weakening the local context by replacing
  some occurrences of $\ASSUMENAME$ by $\PRESUMENAME$.
\item [$\BYY{m@1}{m@2}$] is a \emph{terminal proof}\index{proof!terminal}; it
  abbreviates $\PROOF{m@1}~\QED{m@2}$, with backtracking across both methods,
  though.  Debugging an unsuccessful $\BYY{m@1}{m@2}$ commands might be done
  by expanding its definition; in many cases $\PROOF{m@1}$ is already
  sufficient to see what is going wrong.
\item [``$\DDOT$''] is a \emph{default proof}\index{proof!default}; it
  abbreviates $\BY{rule}$.
\item [``$\DOT$''] is a \emph{trivial proof}\index{proof!trivial}; it
  abbreviates $\BY{this}$.
\item [$\SORRY$] is a \emph{fake proof}\index{proof!fake}; provided that the
  \texttt{quick_and_dirty} flag is enabled, $\SORRY$ pretends to solve the
  goal without further ado.  Of course, the result would be a fake theorem
  only, involving some oracle in its internal derivation object (this is
  indicated as ``$[!]$'' in the printed result).  The main application of
  $\SORRY$ is to support experimentation and top-down proof development.
\end{descr}


\subsection{Fundamental methods and attributes}\label{sec:pure-meth-att}

The following proof methods and attributes refer to basic logical operations
of Isar.  Further methods and attributes are provided by several generic and
object-logic specific tools and packages (see chapters \ref{ch:gen-tools} and
\ref{ch:hol-tools}).

\indexisarmeth{assumption}\indexisarmeth{this}\indexisarmeth{rule}\indexisarmeth{$-$}
\indexisaratt{intro}\indexisaratt{elim}\indexisaratt{dest}
\indexisaratt{OF}\indexisaratt{of}
\begin{matharray}{rcl}
  assumption & : & \isarmeth \\
  this & : & \isarmeth \\
  rule & : & \isarmeth \\
  - & : & \isarmeth \\
  OF & : & \isaratt \\
  of & : & \isaratt \\
  intro & : & \isaratt \\
  elim & : & \isaratt \\
  dest & : & \isaratt \\
  delrule & : & \isaratt \\
\end{matharray}

\begin{rail}
  'rule' thmrefs?
  ;
  'OF' thmrefs
  ;
  'of' (inst * ) ('concl' ':' (inst * ))?
  ;

  inst: underscore | term
  ;
\end{rail}

\begin{descr}
\item [$assumption$] solves some goal by a single assumption step.  Any facts
  given (${} \le 1$) are guaranteed to participate in the refinement.  Recall
  that $\QEDNAME$ (see \S\ref{sec:proof-steps}) already concludes any
  remaining sub-goals by assumption.
\item [$this$] applies all of the current facts directly as rules.  Recall
  that ``$\DOT$'' (dot) abbreviates $\BY{this}$.
\item [$rule~\vec a$] applies some rule given as argument in backward manner;
  facts are used to reduce the rule before applying it to the goal.  Thus
  $rule$ without facts is plain \emph{introduction}, while with facts it
  becomes \emph{elimination}.
  
  When no arguments are given, the $rule$ method tries to pick appropriate
  rules automatically, as declared in the current context using the $intro$,
  $elim$, $dest$ attributes (see below).  This is the default behavior of
  $\PROOFNAME$ and ``$\DDOT$'' (double-dot) steps (see
  \S\ref{sec:proof-steps}).
\item [``$-$''] does nothing but insert the forward chaining facts as premises
  into the goal.  Note that command $\PROOFNAME$ without any method actually
  performs a single reduction step using the $rule$ method; thus a plain
  \emph{do-nothing} proof step would be $\PROOF{-}$ rather than $\PROOFNAME$
  alone.
\item [$OF~\vec a$] applies some theorem to given rules $\vec a$ (in
  parallel).  This corresponds to the \texttt{MRS} operator in ML
  \cite[\S5]{isabelle-ref}, but note the reversed order.  Positions may be
  skipped by including ``$\_$'' (underscore) as argument.
\item [$of~\vec t$] performs positional instantiation.  The terms $\vec t$ are
  substituted for any schematic variables occurring in a theorem from left to
  right; ``\texttt{_}'' (underscore) indicates to skip a position.  Arguments
  following a ``$concl\colon$'' specification refer to positions of the
  conclusion of a rule.
\item [$intro$, $elim$, and $dest$] declare introduction, elimination, and
  destruct rules, respectively.  Note that the classical reasoner (see
  \S\ref{sec:classical-basic}) introduces different versions of these
  attributes, and the $rule$ method, too.  In object-logics with classical
  reasoning enabled, the latter version should be used all the time to avoid
  confusion!
\item [$delrule$] undeclares introduction or elimination rules.
\end{descr}


\subsection{Term abbreviations}\label{sec:term-abbrev}

\indexisarcmd{let}
\begin{matharray}{rcl}
  \isarcmd{let} & : & \isartrans{proof(state)}{proof(state)} \\
  \isarkeyword{is} & : & syntax \\
\end{matharray}

Abbreviations may be either bound by explicit $\LET{p \equiv t}$ statements,
or by annotating assumptions or goal statements with a list of patterns
$\ISS{p@1\;\dots}{p@n}$.  In both cases, higher-order matching is invoked to
bind extra-logical term variables, which may be either named schematic
variables of the form $\Var{x}$, or nameless dummies ``\texttt{_}''
(underscore).\indexisarvar{_@\texttt{_}} Note that in the $\LETNAME$ form the
patterns occur on the left-hand side, while the $\ISNAME$ patterns are in
postfix position.

Polymorphism of term bindings is handled in Hindley-Milner style, as in ML.
Type variables referring to local assumptions or open goal statements are
\emph{fixed}, while those of finished results or bound by $\LETNAME$ may occur
in \emph{arbitrary} instances later.  Even though actual polymorphism should
be rarely used in practice, this mechanism is essential to achieve proper
incremental type-inference, as the user proceeds to build up the Isar proof
text.

\medskip

Term abbreviations are quite different from actual local definitions as
introduced via $\DEFNAME$ (see \S\ref{sec:proof-context}).  The latter are
visible within the logic as actual equations, while abbreviations disappear
during the input process just after type checking.  Also note that $\DEFNAME$
does not support polymorphism.

\begin{rail}
  'let' ((term + 'as') '=' term comment? + 'and')
  ;  
\end{rail}

The syntax of $\ISNAME$ patterns follows \railnonterm{termpat} or
\railnonterm{proppat} (see \S\ref{sec:term-pats}).

\begin{descr}
\item [$\LET{\vec p = \vec t}$] binds any text variables in patters $\vec p$
  by simultaneous higher-order matching against terms $\vec t$.
\item [$\IS{\vec p}$] resembles $\LETNAME$, but matches $\vec p$ against the
  preceding statement.  Also note that $\ISNAME$ is not a separate command,
  but part of others (such as $\ASSUMENAME$, $\HAVENAME$ etc.).
\end{descr}

A few \emph{automatic} term abbreviations\index{term abbreviations} for goals
and facts are available as well.  For any open goal,
$\Var{thesis_prop}$\indexisarvar{thesis-prop} refers to the full proposition
(which may be a rule), $\Var{thesis_concl}$\indexisarvar{thesis-concl} to its
(atomic) conclusion, and $\Var{thesis}$\indexisarvar{thesis} to its
object-level statement.  The latter two abstract over any meta-level
parameters.

Fact statements resulting from assumptions or finished goals are bound as
$\Var{this_prop}$\indexisarvar{this-prop},
$\Var{this_concl}$\indexisarvar{this-concl}, and
$\Var{this}$\indexisarvar{this}, similar to $\Var{thesis}$ above.  In case
$\Var{this}$ refers to an object-logic statement that is an application
$f(t)$, then $t$ is bound to the special text variable
``$\dots$''\indexisarvar{\dots} (three dots).  The canonical application of
the latter are calculational proofs (see \S\ref{sec:calculation}).


\subsection{Block structure}

\indexisarcmd{next}\indexisarcmd{\{\{}\indexisarcmd{\}\}}
\begin{matharray}{rcl}
  \NEXT & : & \isartrans{proof(state)}{proof(state)} \\
  \BG & : & \isartrans{proof(state)}{proof(state)} \\
  \EN & : & \isartrans{proof(state)}{proof(state)} \\
\end{matharray}

While Isar is inherently block-structured, opening and closing blocks is
mostly handled rather casually, with little explicit user-intervention.  Any
local goal statement automatically opens \emph{two} blocks, which are closed
again when concluding the sub-proof (by $\QEDNAME$ etc.).  Sections of
different context within a sub-proof may be switched via $\NEXT$, which is
just a single block-close followed by block-open again.  Thus the effect of
$\NEXT$ to reset the local proof context. There is no goal focus involved
here!

For slightly more advanced applications, there are explicit block parentheses
as well.  These typically achieve a stronger forward style of reasoning.

\begin{descr}
\item [$\NEXT$] switches to a fresh block within a sub-proof, resetting the
  local context to the initial one.
\item [$\isarkeyword{\{\{}$ and $\isarkeyword{\}\}}$] explicitly open and
  close blocks.  Any current facts pass through ``$\isarkeyword{\{\{}$''
  unchanged, while ``$\isarkeyword{\}\}}$'' causes any result to be
  \emph{exported} into the enclosing context.  Thus fixed variables are
  generalized, assumptions discharged, and local definitions unfolded (cf.\ 
  \S\ref{sec:proof-context}).  There is no difference of $\ASSUMENAME$ and
  $\PRESUMENAME$ in this mode of forward reasoning --- in contrast to plain
  backward reasoning with the result exported at $\SHOWNAME$ time.
\end{descr}


\subsection{Emulating tactic scripts}\label{sec:tactical-proof}

The following elements emulate unstructured tactic scripts to some extent.
While these are anathema for writing proper Isar proof documents, they might
come in handy for interactive exploration and debugging, or even actual
tactical proof within new-style theories (to benefit from document
preparation, for example).

\indexisarcmd{apply}\indexisarcmd{apply-end}
\indexisarcmd{defer}\indexisarcmd{prefer}\indexisarcmd{back}
\indexisarmeth{tactic}
\indexisarmeth{res-inst-tac}\indexisarmeth{eres-inst-tac}
\indexisarmeth{dres-inst-tac}\indexisarmeth{forw-inst-tac}
\indexisarmeth{subgoal-tac}
\begin{matharray}{rcl}
  \isarcmd{apply}^* & : & \isartrans{proof(prove)}{proof(prove)} \\
  \isarcmd{apply_end}^* & : & \isartrans{proof(state)}{proof(state)} \\
  \isarcmd{defer}^* & : & \isartrans{proof}{proof} \\
  \isarcmd{prefer}^* & : & \isartrans{proof}{proof} \\
  \isarcmd{back}^* & : & \isartrans{proof}{proof} \\
  tactic^* & : & \isarmeth \\
  res_inst_tac^* & : & \isarmeth \\
  eres_inst_tac^* & : & \isarmeth \\
  dres_inst_tac^* & : & \isarmeth \\
  forw_inst_tac^* & : & \isarmeth \\
  subgoal_tac^* & : & \isarmeth \\
\end{matharray}

\railalias{applyend}{apply\_end}
\railterm{applyend}

\railalias{resinsttac}{res\_inst\_tac}
\railterm{resinsttac}

\railalias{eresinsttac}{eres\_inst\_tac}
\railterm{eresinsttac}

\railalias{dresinsttac}{dres\_inst\_tac}
\railterm{dresinsttac}

\railalias{forwinsttac}{forw\_inst\_tac}
\railterm{forwinsttac}

\railalias{subgoaltac}{subgoal\_tac}
\railterm{subgoaltac}

\begin{rail}
  'apply' method
  ;
  applyend method
  ;
  'defer' nat?
  ;
  'prefer' nat
  ;
  'tactic' text
  ;
  ( resinsttac | eresinsttac | dresinsttac | forwinsttac ) goalspec? ((name '=' term) + 'and')
  ;
  subgoaltac goalspec? prop
  ;
\end{rail}

\begin{descr}
\item [$\isarkeyword{apply}~(m)$] applies proof method $m$ in initial
  position, but unlike $\PROOFNAME$ it retains ``$proof(prove)$'' mode.  Thus
  consecutive method applications may be given just as in tactic scripts.  In
  order to complete the proof properly, any of the actual structured proof
  commands (e.g.\ ``$\DOT$'') has to be given eventually.
  
  Facts are passed to $m$ as indicated by the goal's forward-chain mode.
  Common use of $\isarkeyword{apply}$ would be in a purely backward manner,
  though.
\item [$\isarkeyword{apply_end}~(m)$] applies proof method $m$ as if in
  terminal position.  Basically, this simulates a multi-step tactic script for
  $\QEDNAME$, but may be given anywhere within the proof body.
  
  No facts are passed to $m$.  Furthermore, the static context is that of the
  enclosing goal (as for actual $\QEDNAME$).  Thus the proof method may not
  refer to any assumptions introduced in the current body, for example.
\item [$\isarkeyword{defer}~n$ and $\isarkeyword{prefer}~n$] shuffle the list
  of pending goals: $defer$ puts off goal $n$ to the end of the list ($n = 1$
  by default), while $prefer$ brings goal $n$ to the top.
\item [$\isarkeyword{back}$] does back-tracking over the result sequence of
  the latest proof command.\footnote{Unlike the ML function \texttt{back}
    \cite{isabelle-ref}, the Isar command does not search upwards for further
    branch points.} Basically, any proof command may return multiple results.
\item [$tactic~text$] produces a proof method from any ML text of type
  \texttt{tactic}.  Apart from the usual ML environment and the current
  implicit theory context, the ML code may refer to the following locally
  bound values:
%%FIXME ttbox produces too much trailing space (why?)
{\footnotesize\begin{verbatim}
val ctxt  : Proof.context
val facts : thm list
val thm   : string -> thm
val thms  : string -> thm list
\end{verbatim}}
  Here \texttt{ctxt} refers to the current proof context, \texttt{facts}
  indicates any current facts for forward-chaining, and
  \texttt{thm}~/~\texttt{thms} retrieve named facts (including global
  theorems) from the context.
\item [$res_inst_tac$ etc.] do resolution of rules with explicit
  instantiation.  This works the same way as the corresponding ML tactics, see
  \cite[\S3]{isabelle-ref}.
  
  It is very important to note that the instantiations are read and
  type-checked according to the dynamic goal state, rather than the static
  proof context!  In particular, locally fixed variables and term
  abbreviations may not be included in the term specifications.
\item [$subgoal_tac~\phi$] emulates the ML tactic of the same name, see
  \cite[\S3]{isabelle-ref}.  Syntactically, the given proposition is handled
  as the instantiations in $res_inst_tac$ etc.
  
  Note that the proper Isar command $\PRESUMENAME$ achieves a similar effect
  as $subgoal_tac$.
\end{descr}


\subsection{Meta-linguistic features}

\indexisarcmd{oops}
\begin{matharray}{rcl}
  \isarcmd{oops} & : & \isartrans{proof}{theory} \\
\end{matharray}

The $\OOPS$ command discontinues the current proof attempt, while considering
the partial proof text as properly processed.  This is conceptually quite
different from ``faking'' actual proofs via $\SORRY$ (see
\S\ref{sec:proof-steps}): $\OOPS$ does not observe the proof structure at all,
but goes back right to the theory level.  Furthermore, $\OOPS$ does not
produce any result theorem --- there is no claim to be able to complete the
proof anyhow.

A typical application of $\OOPS$ is to explain Isar proofs \emph{within} the
system itself, in conjunction with the document preparation tools of Isabelle
described in \cite{isabelle-sys}.  Thus partial or even wrong proof attempts
can be discussed in a logically sound manner.  Note that the Isabelle {\LaTeX}
macros can be easily adapted to print something like ``$\dots$'' instead of an
``$\OOPS$'' keyword.

\medskip The $\OOPS$ command is undoable, unlike $\isarkeyword{kill}$ (see
\S\ref{sec:history}).  The effect is to get back to the theory \emph{before}
the opening of the proof.


\section{Other commands}

\subsection{Diagnostics}\label{sec:diag}

\indexisarcmd{pr}\indexisarcmd{thm}\indexisarcmd{term}\indexisarcmd{prop}\indexisarcmd{typ}
\indexisarcmd{print-facts}\indexisarcmd{print-binds}
\begin{matharray}{rcl}
  \isarcmd{help}^* & : & \isarkeep{\cdot} \\
  \isarcmd{pr}^* & : & \isarkeep{\cdot} \\
  \isarcmd{thm}^* & : & \isarkeep{theory~|~proof} \\
  \isarcmd{term}^* & : & \isarkeep{theory~|~proof} \\
  \isarcmd{prop}^* & : & \isarkeep{theory~|~proof} \\
  \isarcmd{typ}^* & : & \isarkeep{theory~|~proof} \\
  \isarcmd{print_facts}^* & : & \isarkeep{proof} \\
  \isarcmd{print_binds}^* & : & \isarkeep{proof} \\
\end{matharray}

These commands are not part of the actual Isabelle/Isar syntax, but assist
interactive development.  Also note that $undo$ does not apply here, since the
theory or proof configuration is not changed.

\begin{rail}
  'pr' modes? nat?
  ;
  'thm' modes? thmrefs
  ;
  'term' modes? term
  ;
  'prop' modes? prop
  ;
  'typ' modes? type
  ;

  modes: '(' (name + ) ')'
  ;
\end{rail}

\begin{descr}
\item [$\isarkeyword{help}$] prints a list of available language elements.
  Note that methods and attributes depend on the current theory context.
\item [$\isarkeyword{pr}~n$] prints the current top-level state, i.e.\ the
  theory identifier or proof state.  The latter includes the proof context,
  current facts and goals.  The optional argument $n$ affects the implicit
  limit of goals to be displayed, which is initially 10.  Omitting the limit
  leaves the value unchanged.
\item [$\isarkeyword{thm}~\vec a$] retrieves theorems from the current theory
  or proof context.  Note that any attributes included in the theorem
  specifications are applied to a temporary context derived from the current
  theory or proof; the result is discarded, i.e.\ attributes involved in $\vec
  a$ do not have any permanent effect.
\item [$\isarkeyword{term}~t$, $\isarkeyword{prop}~\phi$] read, type-check and
  print terms or propositions according to the current theory or proof
  context; the inferred type of $t$ is output as well.  Note that these
  commands are also useful in inspecting the current environment of term
  abbreviations.
\item [$\isarkeyword{typ}~\tau$] reads and prints types of the meta-logic
  according to the current theory or proof context.
\item [$\isarkeyword{print_facts}$] prints any named facts of the current
  context, including assumptions and local results.
\item [$\isarkeyword{print_binds}$] prints all term abbreviations present in
  the context.
\end{descr}

The basic diagnostic commands above admit a list of $modes$ to be specified,
which is appended to the current print mode (see also \cite{isabelle-ref}).
Thus the output behavior may be modified according particular print mode
features.

For example, $\isarkeyword{pr}~(latex~xsymbols~symbols)$ would print the
current proof state with mathematical symbols and special characters
represented in {\LaTeX} source, according to the Isabelle style
\cite{isabelle-sys}.  The resulting text can be directly pasted into a
\verb,\begin{isabelle},\dots\verb,\end{isabelle}, environment.  Note that
$\isarkeyword{pr}~(latex)$ is sufficient to achieve the same output, if the
current Isabelle session has the other modes already activated, say due to
some particular user interface configuration such as Proof~General
\cite{proofgeneral,Aspinall:TACAS:2000} with X-Symbol mode \cite{x-symbol}.


\subsection{History commands}\label{sec:history}

\indexisarcmd{undo}\indexisarcmd{redo}\indexisarcmd{kill}
\begin{matharray}{rcl}
  \isarcmd{undo}^{{*}{*}} & : & \isarkeep{\cdot} \\
  \isarcmd{redo}^{{*}{*}} & : & \isarkeep{\cdot} \\
  \isarcmd{kill}^{{*}{*}} & : & \isarkeep{\cdot} \\
\end{matharray}

The Isabelle/Isar top-level maintains a two-stage history, for theory and
proof state transformation.  Basically, any command can be undone using
$\isarkeyword{undo}$, excluding mere diagnostic elements.  Its effect may be
revoked via $\isarkeyword{redo}$, unless the corresponding the
$\isarkeyword{undo}$ step has crossed the beginning of a proof or theory.  The
$\isarkeyword{kill}$ command aborts the current history node altogether,
discontinuing a proof or even the whole theory.  This operation is \emph{not}
undoable.

\begin{warn}
  History commands should never be used with user interfaces such as
  Proof~General \cite{proofgeneral,Aspinall:TACAS:2000}, which takes care of
  stepping forth and back itself.  Interfering by manual $\isarkeyword{undo}$,
  $\isarkeyword{redo}$, or even $\isarkeyword{kill}$ commands would quickly
  result in utter confusion.
\end{warn}

%FIXME remove
% \begin{descr}
% \item [$\isarkeyword{undo}$] revokes the latest state-transforming command.
% \item [$\isarkeyword{redo}$] undos the latest $\isarkeyword{undo}$.
% \item [$\isarkeyword{kill}$] aborts the current history level.
% \end{descr}


\subsection{System operations}

\indexisarcmd{cd}\indexisarcmd{pwd}\indexisarcmd{use-thy}\indexisarcmd{use-thy-only}
\indexisarcmd{update-thy}\indexisarcmd{update-thy-only}
\begin{matharray}{rcl}
  \isarcmd{cd}^* & : & \isarkeep{\cdot} \\
  \isarcmd{pwd}^* & : & \isarkeep{\cdot} \\
  \isarcmd{use_thy}^* & : & \isarkeep{\cdot} \\
  \isarcmd{use_thy_only}^* & : & \isarkeep{\cdot} \\
  \isarcmd{update_thy}^* & : & \isarkeep{\cdot} \\
  \isarcmd{update_thy_only}^* & : & \isarkeep{\cdot} \\
\end{matharray}

\begin{descr}
\item [$\isarkeyword{cd}~name$] changes the current directory of the Isabelle
  process.
\item [$\isarkeyword{pwd}~$] prints the current working directory.
\item [$\isarkeyword{use_thy}$, $\isarkeyword{use_thy_only}$,
  $\isarkeyword{update_thy}$, $\isarkeyword{update_thy_only}$] load some
  theory given as $name$ argument.  These commands are basically the same as
  the corresponding ML functions\footnote{The ML versions also change the
    implicit theory context to that of the theory loaded.}  (see also
  \cite[\S1,\S6]{isabelle-ref}).  Note that both the ML and Isar versions may
  load new- and old-style theories alike.
\end{descr}

These system commands are scarcely used when working with the Proof~General
interface, since loading of theories is done fully transparently.


%%% Local Variables: 
%%% mode: latex
%%% TeX-master: "isar-ref"
%%% End: 


\chapter{Generic Tools and Packages}\label{ch:gen-tools}

\section{Basic proof methods}\label{sec:pure-meth}

\indexisarmeth{fail}\indexisarmeth{succeed}\indexisarmeth{$-$}\indexisarmeth{assumption}
\indexisarmeth{finish}\indexisarmeth{fold}\indexisarmeth{unfold}
\indexisarmeth{rule}\indexisarmeth{erule}
\begin{matharray}{rcl}
  fail & : & \isarmeth \\
  succeed & : & \isarmeth \\
  - & : & \isarmeth \\
  assumption & : & \isarmeth \\
  finish & : & \isarmeth \\
  fold & : & \isarmeth \\
  unfold & : & \isarmeth \\
  rule & : & \isarmeth \\
  erule^* & : & \isarmeth \\
\end{matharray}

\begin{rail}
  ('fold' | 'unfold' | 'rule' | 'erule') thmrefs
  ;
\end{rail}

\begin{descr}
\item [$ $]
\end{descr}

FIXME

%FIXME sort
%FIXME thmref (single)
%FIXME var vs. term


\section{Miscellaneous attributes}

\indexisaratt{tag}\indexisaratt{untag}\indexisaratt{COMP}\indexisaratt{RS}
\indexisaratt{OF}\indexisaratt{where}\indexisaratt{of}\indexisaratt{standard}
\indexisaratt{elimify}\indexisaratt{transfer}\indexisaratt{export}
\begin{matharray}{rcl}
  tag & : & \isaratt \\
  untag & : & \isaratt \\
  COMP & : & \isaratt \\
  RS & : & \isaratt \\
  OF & : & \isaratt \\
  where & : & \isaratt \\
  of & : & \isaratt \\
  standard & : & \isaratt \\
  elimify & : & \isaratt \\
  transfer & : & \isaratt \\
  export & : & \isaratt \\
\end{matharray}

\begin{rail}
  ('tag' | 'untag') (nameref+)
  ;
\end{rail}

\begin{rail}
  ('COMP' | 'RS') nat? thmref
  ;
  'OF' thmrefs
  ;
\end{rail}

\begin{rail}
  'where' (name '=' term * 'and')
  ;
  'of' (inst * ) ('concl' ':' (inst * ))?
  ;

  inst: underscore | term
  ;
\end{rail}

\begin{descr}
\item [$ $]
\end{descr}

FIXME


\section{Calculational proof}\label{sec:calculation}

\indexisarcmd{also}\indexisarcmd{finally}\indexisaratt{trans}
\begin{matharray}{rcl}
  \isarcmd{also} & : & \isartrans{proof(state)}{proof(state)} \\
  \isarcmd{finally} & : & \isartrans{proof(state)}{proof(chain)} \\
  trans & : & \isaratt \\
\end{matharray}

Calculational proof is forward reasoning with implicit application of
transitivity rules (such those of $=$, $\le$, $<$).  Isabelle/Isar maintains
an auxiliary register $calculation$\indexisarreg{calculation} for accumulating
results obtained by transitivity obtained together with the current facts.
Command $\ALSO$ updates $calculation$ from the most recent result, while
$\FINALLY$ exhibits the final result by forward chaining towards the next goal
statement.  Both commands require valid current facts, i.e.\ may occur only
after commands that produce theorems such as $\ASSUMENAME$, $\NOTENAME$, or
some finished $\HAVENAME$ or $\SHOWNAME$.

Also note that the automatic term abbreviation ``$\dots$'' has its canonical
application with calculational proofs.  It automatically refers to the
argument\footnote{The argument of a curried infix expression is its right-hand
  side.} of the preceding statement.

Isabelle/Isar calculations are implicitly subject to block structure in the
sense that new threads of calculational reasoning are commenced for any new
block (as opened by a local goal, for example).  This means that, apart from
being able to nest calculations, there is no separate \emph{begin-calculation}
command required.

\begin{rail}
  ('also' | 'finally') transrules? comment?
  ;
  'trans' (() | 'add' ':' | 'del' ':') thmrefs
  ;

  transrules: '(' thmrefs ')' interest?
  ;
\end{rail}

\begin{descr}
\item [$\ALSO~(thms)$] maintains the auxiliary $calculation$ register as
  follows.  The first occurrence of $\ALSO$ in some calculational thread
  initialises $calculation$ by $facts$. Any subsequent $\ALSO$ on the
  \emph{same} level of block-structure updates $calculation$ by some
  transitivity rule applied to $calculation$ and $facts$ (in that order).
  Transitivity rules are picked from the current context plus those given as
  $thms$ (the latter have precedence).
  
\item [$\FINALLY~(thms)$] maintaining $calculation$ in the same way as
  $\ALSO$, and concludes the current calculational thread.  The final result
  is exhibited as fact for forward chaining towards the next goal. Basically,
  $\FINALLY$ just abbreviates $\ALSO~\FROM{calculation}$.  A typical proof
  idiom is $\FINALLY~\SHOW~\VVar{thesis}~\DOT$.
  
\item [Attribute $trans$] maintains the set of transitivity rules of the
  theory or proof context, by adding or deleting the theorems provided as
  arguments.  The default is adding of rules.
\end{descr}

See theory \texttt{HOL/Isar_examples/Group} for a simple applications of
calculations for basic equational reasoning.
\texttt{HOL/Isar_examples/KnasterTarski} involves a few more advanced
calculational steps in combination with natural deduction.


\section{Axiomatic Type Classes}\label{sec:axclass}

\indexisarcmd{axclass}\indexisarcmd{instance}
\begin{matharray}{rcl}
  \isarcmd{axclass} & : & \isartrans{theory}{theory} \\
  \isarcmd{instance} & : & \isartrans{theory}{proof(prove)} \\
\end{matharray}

Axiomatic type classes are provided by Isabelle/Pure as a purely
\emph{definitional} interface to type classes (cf.~\S\ref{sec:classes}).  Thus
any object logic may make use of this light-weight mechanism for abstract
theories.  See \cite{Wenzel:1997:TPHOL} for more information.  There is also a
tutorial on \emph{Using Axiomatic Type Classes in Isabelle} that is part of
the standard Isabelle documentation.

\begin{rail}
  'axclass' classdecl (axmdecl prop comment? +)
  ;
  'instance' (nameref '<' nameref | nameref '::' simplearity) comment?
  ;
\end{rail}

\begin{descr}
\item [$\isarkeyword{axclass}~$] FIXME
\item [$\isarkeyword{instance}~c@1 < c@2$ and $\isarkeyword{instance}~c@1 <
  c@2$] setup up a goal stating the class relation or type arity.  The proof
  would usually proceed by the $expand_classes$ method, and then establish the
  characteristic theorems of the type classes involved.  After finishing the
  proof the theory will be augmented by a type signature declaration
  corresponding to the resulting theorem.
\end{descr}



\section{The Simplifier}

\subsection{Simplification methods}

\indexisarmeth{simp}\indexisarmeth{asm_simp}\indexisaratt{simp}
\begin{matharray}{rcl}
  simp & : & \isarmeth \\
  asm_simp & : & \isarmeth \\
  simp & : & \isaratt \\
\end{matharray}

\begin{rail}
  'simp' (simpmod+)?
  ;

  simpmod: ('add' | 'del' | 'only' | 'other') ':' thmrefs
  ;
\end{rail}


\subsection{Forward simplification}

\indexisaratt{simplify}\indexisaratt{asm_simplify}\indexisaratt{full_simplify}\indexisaratt{asm_full_simplify}
\begin{matharray}{rcl}
  simplify & : & \isaratt \\
  asm_simplify & : & \isaratt \\
  full_simplify & : & \isaratt \\
  asm_full_simplify & : & \isaratt \\
\end{matharray}

FIXME


\section{The Classical Reasoner}

\subsection{Single step methods}

\subsection{Automatic methods}

\subsection{Combined automatic methods}

\subsection{Modifying the context}



%\indexisarcmd{}
%\begin{matharray}{rcl}
%  \isarcmd{} & : & \isartrans{}{} \\
%\end{matharray}

%\begin{rail}
  
%\end{rail}

%\begin{descr}
%\item [$ $]
%\end{descr}


%%% Local Variables: 
%%% mode: latex
%%% TeX-master: "isar-ref"
%%% End: 


\chapter{Isabelle/HOL specific elements}\label{ch:hol-tools}

\section{Miscellaneous attributes}

\indexisarattof{HOL}{split-format}
\begin{matharray}{rcl}
  split_format^* & : & \isaratt \\
\end{matharray}

\railalias{splitformat}{split\_format}
\railterm{splitformat}
\railterm{complete}

\begin{rail}
  splitformat (((name * ) + 'and') | ('(' complete ')'))
  ;
\end{rail}

\begin{descr}
  
\item [$split_format~\vec p@1 \dots \vec p@n$] puts tuple objects into
  canonical form as specified by the arguments given; $\vec p@i$ refers to
  occurrences in premise $i$ of the rule.  The $split_format~(complete)$ form
  causes \emph{all} arguments in function applications to be represented
  canonically according to their tuple type structure.
  
  Note that these operations tend to invent funny names for new local
  parameters to be introduced.

\end{descr}


\section{Primitive types}\label{sec:typedef}

\indexisarcmdof{HOL}{typedecl}\indexisarcmdof{HOL}{typedef}
\begin{matharray}{rcl}
  \isarcmd{typedecl} & : & \isartrans{theory}{theory} \\
  \isarcmd{typedef} & : & \isartrans{theory}{proof(prove)} \\
\end{matharray}

\begin{rail}
  'typedecl' typespec infix? comment?
  ;
  'typedef' parname? typespec infix? \\ '=' term comment?
  ;
\end{rail}

\begin{descr}
\item [$\isarkeyword{typedecl}~(\vec\alpha)t$] is similar to the original
  $\isarkeyword{typedecl}$ of Isabelle/Pure (see \S\ref{sec:types-pure}), but
  also declares type arity $t :: (term, \dots, term) term$, making $t$ an
  actual HOL type constructor.
\item [$\isarkeyword{typedef}~(\vec\alpha)t = A$] sets up a goal stating
  non-emptiness of the set $A$.  After finishing the proof, the theory will be
  augmented by a Gordon/HOL-style type definition.  See \cite{isabelle-HOL}
  for more information.  Note that user-level theories usually do not directly
  refer to the HOL $\isarkeyword{typedef}$ primitive, but use more advanced
  packages such as $\isarkeyword{record}$ (see \S\ref{sec:hol-record}) and
  $\isarkeyword{datatype}$ (see \S\ref{sec:hol-datatype}).
\end{descr}


\section{Records}\label{sec:hol-record}

FIXME proof tools (simp, cases/induct; no split!?);

\indexisarcmdof{HOL}{record}
\begin{matharray}{rcl}
  \isarcmd{record} & : & \isartrans{theory}{theory} \\
\end{matharray}

\begin{rail}
  'record' typespec '=' (type '+')? (field +)
  ;

  field: name '::' type comment?
  ;
\end{rail}

\begin{descr}
\item [$\isarkeyword{record}~(\vec\alpha)t = \tau + \vec c :: \vec\sigma$]
  defines extensible record type $(\vec\alpha)t$, derived from the optional
  parent record $\tau$ by adding new field components $\vec c :: \vec\sigma$.
  See \cite{isabelle-HOL,NaraschewskiW-TPHOLs98} for more information on
  simply-typed extensible records.
\end{descr}


\section{Datatypes}\label{sec:hol-datatype}

\indexisarcmdof{HOL}{datatype}\indexisarcmdof{HOL}{rep-datatype}
\begin{matharray}{rcl}
  \isarcmd{datatype} & : & \isartrans{theory}{theory} \\
  \isarcmd{rep_datatype} & : & \isartrans{theory}{theory} \\
\end{matharray}

\railalias{repdatatype}{rep\_datatype}
\railterm{repdatatype}

\begin{rail}
  'datatype' (dtspec + 'and')
  ;
  repdatatype (name * ) dtrules
  ;

  dtspec: parname? typespec infix? '=' (cons + '|')
  ;
  cons: name (type * ) mixfix? comment?
  ;
  dtrules: 'distinct' thmrefs 'inject' thmrefs 'induction' thmrefs
\end{rail}

\begin{descr}
\item [$\isarkeyword{datatype}$] defines inductive datatypes in HOL.
\item [$\isarkeyword{rep_datatype}$] represents existing types as inductive
  ones, generating the standard infrastructure of derived concepts (primitive
  recursion etc.).
\end{descr}

The induction and exhaustion theorems generated provide case names according
to the constructors involved, while parameters are named after the types (see
also \S\ref{sec:cases-induct}).

See \cite{isabelle-HOL} for more details on datatypes.  Note that the theory
syntax above has been slightly simplified over the old version, usually
requiring more quotes and less parentheses.  Apart from proper proof methods
for case-analysis and induction, there are also emulations of ML tactics
\texttt{case_tac} and \texttt{induct_tac} available, see
\S\ref{sec:induct_tac}.


\section{Recursive functions}\label{sec:recursion}

\indexisarcmdof{HOL}{primrec}\indexisarcmdof{HOL}{recdef}\indexisarcmdof{HOL}{recdef-tc}
\begin{matharray}{rcl}
  \isarcmd{primrec} & : & \isartrans{theory}{theory} \\
  \isarcmd{recdef} & : & \isartrans{theory}{theory} \\
  \isarcmd{recdef_tc}^* & : & \isartrans{theory}{proof(prove)} \\
%FIXME
%  \isarcmd{defer_recdef} & : & \isartrans{theory}{theory} \\
\end{matharray}

\railalias{recdefsimp}{recdef\_simp}
\railterm{recdefsimp}

\railalias{recdefcong}{recdef\_cong}
\railterm{recdefcong}

\railalias{recdefwf}{recdef\_wf}
\railterm{recdefwf}

\railalias{recdeftc}{recdef\_tc}
\railterm{recdeftc}

\begin{rail}
  'primrec' parname? (equation + )
  ;
  'recdef' ('(' 'permissive' ')')? \\ name term (eqn + ) hints?
  ;
  recdeftc thmdecl? tc comment?
  ;

  equation: thmdecl? eqn
  ;
  eqn: prop comment?
  ;
  hints: '(' 'hints' (recdefmod * ) ')'
  ;
  recdefmod: ((recdefsimp | recdefcong | recdefwf) (() | 'add' | 'del') ':' thmrefs) | clasimpmod
  ;
  tc: nameref ('(' nat ')')?
  ;
\end{rail}

\begin{descr}
\item [$\isarkeyword{primrec}$] defines primitive recursive functions over
  datatypes, see also \cite{isabelle-HOL}.
\item [$\isarkeyword{recdef}$] defines general well-founded recursive
  functions (using the TFL package), see also \cite{isabelle-HOL}.  The
  $(permissive)$ option tells TFL to recover from failed proof attempts,
  returning unfinished results.  The $recdef_simp$, $recdef_cong$, and
  $recdef_wf$ hints refer to auxiliary rules to be used in the internal
  automated proof process of TFL.  Additional $clasimpmod$ declarations (cf.\ 
  \S\ref{sec:clasimp}) may be given to tune the context of the Simplifier
  (cf.\ \S\ref{sec:simplifier}) and Classical reasoner (cf.\ 
  \S\ref{sec:classical}).
\item [$\isarkeyword{recdef_tc}~c~(i)$] recommences the proof for leftover
  termination condition number $i$ (default $1$) as generated by a
  $\isarkeyword{recdef}$ definition of constant $c$.
  
  Note that in most cases, $\isarkeyword{recdef}$ is able to finish its
  internal proofs without manual intervention.
\end{descr}

Both kinds of recursive definitions accommodate reasoning by induction (cf.\ 
\S\ref{sec:cases-induct}): rule $c\mathord{.}induct$ (where $c$ is the name of
the function definition) refers to a specific induction rule, with parameters
named according to the user-specified equations.  Case names of
$\isarkeyword{primrec}$ are that of the datatypes involved, while those of
$\isarkeyword{recdef}$ are numbered (starting from $1$).

The equations provided by these packages may be referred later as theorem list
$f\mathord.simps$, where $f$ is the (collective) name of the functions
defined.  Individual equations may be named explicitly as well; note that for
$\isarkeyword{recdef}$ each specification given by the user may result in
several theorems.

\medskip Hints for $\isarkeyword{recdef}$ may be also declared globally, using
the following attributes.

\indexisarattof{HOL}{recdef-simp}\indexisarattof{HOL}{recdef-cong}\indexisarattof{HOL}{recdef-wf}
\begin{matharray}{rcl}
  recdef_simp & : & \isaratt \\
  recdef_cong & : & \isaratt \\
  recdef_wf & : & \isaratt \\
\end{matharray}

\railalias{recdefsimp}{recdef\_simp}
\railterm{recdefsimp}

\railalias{recdefcong}{recdef\_cong}
\railterm{recdefcong}

\railalias{recdefwf}{recdef\_wf}
\railterm{recdefwf}

\begin{rail}
  (recdefsimp | recdefcong | recdefwf) (() | 'add' | 'del')
  ;
\end{rail}


\section{(Co)Inductive sets}\label{sec:hol-inductive}

\indexisarcmdof{HOL}{inductive}\indexisarcmdof{HOL}{coinductive}\indexisarattof{HOL}{mono}
\begin{matharray}{rcl}
  \isarcmd{inductive} & : & \isartrans{theory}{theory} \\
  \isarcmd{coinductive} & : & \isartrans{theory}{theory} \\
  mono & : & \isaratt \\
\end{matharray}

\railalias{condefs}{con\_defs}
\railterm{condefs}

\begin{rail}
  ('inductive' | 'coinductive') sets intros monos?
  ;
  'mono' (() | 'add' | 'del')
  ;

  sets: (term comment? +)
  ;
  intros: 'intros' (thmdecl? prop comment? +)
  ;
  monos: 'monos' thmrefs comment?
  ;
\end{rail}

\begin{descr}
\item [$\isarkeyword{inductive}$ and $\isarkeyword{coinductive}$] define
  (co)inductive sets from the given introduction rules.
\item [$mono$] declares monotonicity rules.  These rule are involved in the
  automated monotonicity proof of $\isarkeyword{inductive}$.
\end{descr}

See \cite{isabelle-HOL} for further information on inductive definitions in
HOL.


\section{Arithmetic}

\indexisarmethof{HOL}{arith}\indexisarattof{HOL}{arith-split}
\begin{matharray}{rcl}
  arith & : & \isarmeth \\
  arith_split & : & \isaratt \\
\end{matharray}

\begin{rail}
  'arith' '!'?
  ;
\end{rail}

The $arith$ method decides linear arithmetic problems (on types $nat$, $int$,
$real$).  Any current facts are inserted into the goal before running the
procedure.  The ``!''~argument causes the full context of assumptions to be
included.  The $arith_split$ attribute declares case split rules to be
expanded before the arithmetic procedure is invoked.

Note that a simpler (but faster) version of arithmetic reasoning is already
performed by the Simplifier.


\section{Cases and induction: emulating tactic scripts}\label{sec:induct_tac}

The following important tactical tools of Isabelle/HOL have been ported to
Isar.  These should be never used in proper proof texts!

\indexisarmethof{HOL}{case-tac}\indexisarmethof{HOL}{induct-tac}
\indexisarmethof{HOL}{ind-cases}\indexisarcmdof{HOL}{inductive-cases}
\begin{matharray}{rcl}
  case_tac^* & : & \isarmeth \\
  induct_tac^* & : & \isarmeth \\
  ind_cases^* & : & \isarmeth \\
  \isarcmd{inductive_cases} & : & \isartrans{theory}{theory} \\
\end{matharray}

\railalias{casetac}{case\_tac}
\railterm{casetac}

\railalias{inducttac}{induct\_tac}
\railterm{inducttac}

\railalias{indcases}{ind\_cases}
\railterm{indcases}

\railalias{inductivecases}{inductive\_cases}
\railterm{inductivecases}

\begin{rail}
  casetac goalspec? term rule?
  ;
  inducttac goalspec? (insts * 'and') rule?
  ;
  indcases (prop +)
  ;
  inductivecases thmdecl? (prop +) comment?
  ;

  rule: ('rule' ':' thmref)
  ;
\end{rail}

\begin{descr}
\item [$case_tac$ and $induct_tac$] admit to reason about inductive datatypes
  only (unless an alternative rule is given explicitly).  Furthermore,
  $case_tac$ does a classical case split on booleans; $induct_tac$ allows only
  variables to be given as instantiation.  These tactic emulations feature
  both goal addressing and dynamic instantiation.  Note that named rule cases
  are \emph{not} provided as would be by the proper $induct$ and $cases$ proof
  methods (see \S\ref{sec:cases-induct}).
  
\item [$ind_cases$ and $\isarkeyword{inductive_cases}$] provide an interface
  to the \texttt{mk_cases} operation.  Rules are simplified in an unrestricted
  forward manner.
  
  While $ind_cases$ is a proof method to apply the result immediately as
  elimination rules, $\isarkeyword{inductive_cases}$ provides case split
  theorems at the theory level for later use,
\end{descr}


%%% Local Variables: 
%%% mode: latex
%%% TeX-master: "isar-ref"
%%% End: 


\appendix

\chapter{Isabelle/Isar quick reference}\label{ap:refcard}

\section{Proof commands}

\subsection{Primitives and basic syntax}

\begin{tabular}{ll}
  $\FIX{\vec x}$ & augment context by $\All {\vec x} \Box$ \\
  $\ASSUME{a}{\vec\phi}$ & augment context by $\vec\phi \Imp \Box$ \\
  $\THEN$ & indicate forward chaining of facts \\
  $\HAVE{a}{\phi}$ & prove local result \\
  $\SHOW{a}{\phi}$ & prove local result, establishing some goal \\
  $\USING{\vec a}$ & indicate use of additional facts \\
  $\PROOF{m@1}~\dots~\QED{m@2}$ & apply proof methods \\
  $\BG~\dots~\EN$ & declare explicit blocks \\
  $\NEXT$ & switch implicit blocks \\
  $\NOTE{a}{\vec b}$ & reconsider facts \\
  $\LET{p = t}$ & \Text{abbreviate terms by higher-order matching} \\
\end{tabular}

\begin{matharray}{rcl}
  theory{\dsh}stmt & = & \THEOREM{name}{prop} ~proof \\
  & \Or & \LEMMA{name}{prop}~proof \\
  & \Or & \TYPES~\dots \Or \CONSTS~\dots \Or \DEFS~\dots \Or \dots \\[1ex]
  proof & = & prfx^*~\PROOF{method}~stmt^*~\QED{method} \\[1ex]
  prfx & = & \APPLY{method} \\
  & \Or & \USING{name^+} \\
  stmt & = & \BG~stmt^*~\EN \\
  & \Or & \NEXT \\
  & \Or & \NOTE{name}{name^+} \\
  & \Or & \LET{term = term} \\[0.5ex]
  & \Or & \FIX{var^+} \\
  & \Or & \ASSUME{name}{prop^+}\\
  & \Or & \THEN~goal{\dsh}stmt \\
  & \Or & goal \\
  goal & = & \HAVE{name}{prop}~proof \\
  & \Or & \SHOW{name}{prop}~proof \\
\end{matharray}


\subsection{Abbreviations and synonyms}

\begin{matharray}{rcl}
  \BYY{m@1}{m@2} & \equiv & \PROOF{m@1}~\QED{m@2} \\
  \DDOT & \equiv & \BY{rule} \\
  \DOT & \equiv & \BY{this} \\
  \HENCENAME & \equiv & \THEN~\HAVENAME \\
  \THUSNAME & \equiv & \THEN~\SHOWNAME \\
  \FROM{\vec a} & \equiv & \NOTE{this}{\vec a}~\THEN \\
  \WITH{\vec a} & \equiv & \FROM{\vec a~\AND~this} \\[1ex]
  \FROM{this} & \equiv & \THEN \\
  \FROM{this}~\HAVENAME & \equiv & \HENCENAME \\
  \FROM{this}~\SHOWNAME & \equiv & \THUSNAME \\
\end{matharray}


\subsection{Derived elements}

\begin{matharray}{rcl}
  \ALSO@0 & \approx & \NOTE{calculation}{this} \\
  \ALSO@{n+1} & \approx & \NOTE{calculation}{trans~[OF~calculation~this]} \\
  \FINALLY & \approx & \ALSO~\FROM{calculation} \\[0.5ex]
  \MOREOVER & \approx & \NOTE{calculation}{calculation~this} \\
  \ULTIMATELY & \approx & \MOREOVER~\FROM{calculation} \\[0.5ex]
  \PRESUME{a}{\vec\phi} & \approx & \ASSUME{a}{\vec\phi} \\
%  & & \Text{(permissive assumption)} \\
  \DEF{a}{x \equiv t} & \approx & \FIX{x}~\ASSUME{a}{x \equiv t} \\
%  & & \Text{(definitional assumption)} \\
  \OBTAIN{\vec x}{a}{\vec\phi} & \approx & \dots~\FIX{\vec x}~\ASSUME{a}{\vec\phi} \\
%  & & \Text{(generalized existence)} \\
  \CASE{c} & \approx & \FIX{\vec x}~\ASSUME{c}{\vec\phi} \\
%  & & \Text{(named context)} \\[0.5ex]
  \SORRY & \approx & \BY{cheating} \\
\end{matharray}


\subsection{Diagnostic commands}

\begin{matharray}{ll}
  \isarkeyword{pr} & \Text{print current state} \\
  \isarkeyword{thm}~\vec a & \Text{print theorems} \\
  \isarkeyword{term}~t & \Text{print term} \\
  \isarkeyword{prop}~\phi & \Text{print meta-level proposition} \\
  \isarkeyword{typ}~\tau & \Text{print meta-level type} \\
\end{matharray}


\section{Proof methods}

\begin{tabular}{ll}
  \multicolumn{2}{l}{\textbf{Single steps (forward-chaining facts)}} \\[0.5ex]
  $assumption$ & apply some assumption \\
  $this$ & apply current facts \\
  $rule~\vec a$ & apply some rule  \\
  $rule$ & apply standard rule (default for $\PROOFNAME$) \\
  $contradiction$ & apply $\neg{}$ elimination rule (any order) \\
  $cases~t$ & case analysis (provides cases) \\
  $induct~\vec x$ & proof by induction (provides cases) \\[2ex]

  \multicolumn{2}{l}{\textbf{Repeated steps (inserting facts)}} \\[0.5ex]
  $-$ & \Text{no rules} \\
  $intro~\vec a$ & \Text{introduction rules} \\
  $intro_classes$ & \Text{class introduction rules} \\
  $elim~\vec a$ & \Text{elimination rules} \\
  $unfold~\vec a$ & \Text{definitions} \\[2ex]

  \multicolumn{2}{l}{\textbf{Automated proof tools (inserting facts, or even prems!)}} \\[0.5ex]
  $rules$ & \Text{intuitionistic proof search} \\
  $blast$, $fast$ & Classical Reasoner \\
  $simp$, $simp_all$ & Simplifier (+ Splitter) \\
  $auto$, $force$ & Simplifier + Classical Reasoner \\
  $arith$ & Arithmetic procedure \\
\end{tabular}


\section{Attributes}

\begin{tabular}{ll}
  \multicolumn{2}{l}{\textbf{Operations}} \\[0.5ex]
  $OF~\vec a$ & rule applied to facts (skipping ``$_$'') \\
  $of~\vec t$ & rule applied to terms (skipping ``$_$'') \\
  $symmetric$ & resolution with symmetry rule \\
  $THEN~b$ & resolution with another rule \\
  $rule_format$ & result put into standard rule format \\
  $elim_format$ & destruct rule turned into elimination rule format \\[1ex]

  \multicolumn{2}{l}{\textbf{Declarations}} \\[0.5ex]
  $simp$ & Simplifier rule \\
  $intro$, $elim$, $dest$ & Pure or Classical Reasoner rule \\
  $iff$ & Simplifier + Classical Reasoner rule \\
  $split$ & case split rule \\
  $trans$ & transitivity rule \\
  $sym$ & symmetry rule \\
\end{tabular}


\section{Rule declarations and methods}

\begin{tabular}{l|lllll}
                          & $rule$   & $rules$  & $blast$ etc. & $simp$ etc. & $auto$ etc. \\
  \hline
  $elim!$ $intro!$ (Pure) & $\times$ & $\times$ \\
  $elim$ $intro$ (Pure)   & $\times$ & $\times$ \\
  $elim!$ $intro!$        & $\times$ &          & $\times$     &             & $\times$ \\
  $elim$ $intro$          & $\times$ &          & $\times$     &             & $\times$ \\
  $iff$                   & $\times$ &          & $\times$     & $\times$    & $\times$ \\
  $iff?$                  & $\times$ \\
  $elim?$ $intro?$        & $\times$ \\
  $simp$                  &          &          &              & $\times$    & $\times$ \\
  $cong$                  &          &          &              & $\times$    & $\times$ \\
  $split$                 &          &          &              & $\times$    & $\times$ \\
\end{tabular}


\section{Emulating tactic scripts}

\subsection{Commands}

\begin{tabular}{ll}
  $\APPLY{m}$ & apply proof method at initial position \\
  $\isarkeyword{apply_end}~(m)$ & apply proof method near terminal position \\
  $\isarkeyword{done}$ & complete proof \\
  $\isarkeyword{defer}~n$ & move subgoal to end \\
  $\isarkeyword{prefer}~n$ & move subgoal to beginning \\
  $\isarkeyword{back}$ & backtrack last command \\
  $\isarkeyword{declare}$ & declare rules in current theory \\
\end{tabular}

\subsection{Methods}

\begin{tabular}{ll}
  $rule_tac~insts$ & resolution (with instantiation) \\
  $erule_tac~insts$ & elim-resolution (with instantiation) \\
  $drule_tac~insts$ & destruct-resolution (with instantiation) \\
  $frule_tac~insts$ & forward-resolution (with instantiation) \\
  $cut_tac~insts$ & insert facts (with instantiation) \\
  $thin_tac~\phi$ & delete assumptions \\
  $subgoal_tac~\phi$ & new claims \\
  $rename_tac~\vec x$ & rename suffix of goal parameters \\
  $rotate_tac~n$ & rotate assumptions of goal \\
  $tactic~text$ & arbitrary ML tactic \\
  $case_tac~t$ & exhaustion (datatypes) \\
  $induct_tac~\vec x$ & induction (datatypes) \\
  $ind_cases~t$ & exhaustion + simplification (inductive sets) \\
\end{tabular}


%%% Local Variables:
%%% mode: latex
%%% TeX-master: "isar-ref"
%%% End:


\begingroup
  \bibliographystyle{plain} \small\raggedright\frenchspacing
  \bibliography{../manual}
\endgroup


%% $Id$

\documentclass[12pt]{report}
\usepackage{graphicx,a4,../iman,../extra,../proof,../rail,../pdfsetup}

\title{\includegraphics[scale=0.5]{isabelle_isar} \\[4ex] The Isabelle/Isar Reference Manual}

\author{\emph{Markus Wenzel} \\ TU M\"unchen}

\setcounter{secnumdepth}{2} \setcounter{tocdepth}{2}

\pagestyle{headings}
\sloppy
\binperiod     %%%treat . like a binary operator

\railalias{lbrace}{\ttlbrace}
\railalias{rbrace}{\ttrbrace}
\railterm{lbrace,rbrace}

\railterm{ident,longident,symident,var,textvar,typefree,typevar,nat,string,verbatim}


\begin{document}

\underscoreoff

\maketitle 

\begin{abstract}
  FIXME
\end{abstract}

\pagenumbering{roman} \tableofcontents \clearfirst

%FIXME
\nocite{Rudnicki:1992:MizarOverview}
\nocite{Harrison:1996:MizarHOL}
\nocite{Rudnicki:1992:MizarOverview}
\nocite{Trybulec:1993:MizarFeatures}
\nocite{Syme:1997:DECLARE}
\nocite{Syme:1998:thesis}
\nocite{Syme:1999:TPHOL}
\nocite{Wenzel:1999:TPHOL}


\chapter{Introduction}

\section{Quick start}

Isar is already part of Isabelle (as of version Isabelle99, or later).  The
\texttt{isabelle} binary provides option \texttt{-I} to run the Isar
interaction loop at startup, rather than the plain ML top-level.  Thus the
quickest way to do anything with Isabelle/Isar is as follows:
\begin{ttbox}
isabelle -I HOL\medskip
\out{> Welcome to Isabelle/HOL (Isabelle99)}\medskip
theory Foo = Main:
constdefs foo :: nat  "foo == 1";
lemma "0 < foo" by (simp add: foo_def);
end
\end{ttbox}
Note that any Isabelle/Isar command may be retracted by \texttt{undo}; the
\texttt{help} command prints a list of available language elements.

Plain TTY-based interaction like this used to be quite feasible with
traditional tactic based theorem proving, but developing Isar documents
demands some better user-interface support.  \emph{Proof~General}\index{Proof
  General} of LFCS Edinburgh \cite{proofgeneral} offers a generic Emacs-based
environment for interactive theorem provers that does all the cut-and-paste
and forward-backward walk through the text in a very neat way.  Note that in
Isabelle/Isar, the current position within a partial proof document is equally
important than the actual proof state.  Thus Proof~General provides the
canonical working environment for Isabelle/Isar, both for getting acquainted
(e.g.\ by replaying existing Isar documents) and real production work.

\medskip

The easiest way to use Proof~General is to make it the default Isabelle user
interface.  Just put something like this into your Isabelle settings file (see
also \cite{isabelle-sys}):
\begin{ttbox}
ISABELLE_INTERFACE=\$ISABELLE_HOME/contrib/ProofGeneral/isar/interface
PROOFGENERAL_OPTIONS="-u false"
\end{ttbox}
You may have to change \texttt{\$ISABELLE_HOME/contrib/ProofGeneral} to the
actual installation directory of Proof~General.  From now on, the capital
\texttt{Isabelle} executable refers to the \texttt{ProofGeneral/isar}
interface.\footnote{There is also a \texttt{ProofGeneral/isa} interface, for
  classic Isabelle tactic scripts.}  Its usage is as follows:
\begin{ttbox}
Usage: interface [OPTIONS] [FILES ...]

  Options are:
    -l NAME      logic image name (default $ISABELLE_LOGIC=HOL)
    -p NAME      Emacs program name (default xemacs)
    -u BOOL      use .emacs file (default true)
    -w BOOL      use window system (default true)

  Starts Proof General for Isabelle/Isar with proof documents FILES
  (default Scratch.thy).

  PROOFGENERAL_OPTIONS=
\end{ttbox} %$
Apart from the command line, the defaults for these options may be overridden
via the \texttt{PROOFGENERAL_OPTIONS} setting as well.  For example, plain GNU
Emacs may be configured as follows:
\begin{ttbox}
PROOFGENERAL_OPTIONS="-u false -p emacs"
\end{ttbox}

Occasionally, a user's \texttt{.emacs} file contains material that is
incompatible with the version of (X)Emacs that Proof~General prefers.  Then
proper startup may be still achieved by using the \texttt{-u false}
option.\footnote{Any Emacs lisp file \texttt{proofgeneral-settings.el}
  occurring in \texttt{\$ISABELLE_HOME/etc} or
  \texttt{\$ISABELLE_HOME_USER/etc} is automatically loaded by the
  Proof~General interface script as well.}

\medskip

With the proper Isabelle interface setup, Isar documents may now be edited by
visiting appropriate theory files, e.g.\ 
\begin{ttbox}
Isabelle \({\langle}isabellehome{\rangle}\)/src/HOL/Isar_examples/BasicLogic.thy
\end{ttbox}
Users of XEmacs may note the tool bar for navigating forward and backward
through the text.  Consult the Proof~General documentation \cite{proofgeneral}
for further basic command sequences, such as ``\texttt{c-c return}'' or
``\texttt{c-c u}''.


\section{Isabelle/Isar theories}

Isabelle/Isar offers two main improvements over classic Isabelle:
\begin{enumerate}
\item A new \emph{theory format}, occasionally referred to as ``new-style
  theories'', supporting interactive development and unlimited undo operation.
\item A \emph{formal proof document language} designed to support intelligible
  semi-automated reasoning.  Instead of putting together unreadable tactic
  scripts, the author is enabled to express the reasoning in way that is close
  to mathematical practice.
\end{enumerate}

The Isar proof language is embedded into the new theory format as a proper
sub-language.  Proof mode is entered by stating some $\THEOREMNAME$ or
$\LEMMANAME$ at the theory level, and left again with the final conclusion
(e.g.\ via $\QEDNAME$).  A few theory extension mechanisms require proof as
well, such as the HOL $\isarkeyword{typedef}$ which demands non-emptiness of
the representing sets.

New-style theory files may still be associated with separate ML files
consisting of plain old tactic scripts.  There is no longer any ML binding
generated for the theory and theorems, though.  ML functions \texttt{theory},
\texttt{thm}, and \texttt{thms} retrieve this information \cite{isabelle-ref}.
Nevertheless, migration between classic Isabelle and Isabelle/Isar is
relatively easy.  Thus users may start to benefit from interactive theory
development even before they have any idea of the Isar proof language at all.

\begin{warn}
  Currently Proof~General does \emph{not} support mixed interactive
  development of classic Isabelle theory files or tactic scripts, together
  with Isar documents.  The ``\texttt{isa}'' and ``\texttt{isar}'' versions of
  Proof~General are handled as two different theorem proving systems, only one
  of these may be active at the same time.
\end{warn}

Porting of existing tactic scripts is best done by running two separate
Proof~General sessions, one for replaying the old script and the other for the
emerging Isabelle/Isar document.


\section{How to write Isar proofs anyway?}

This is one of the key questions, of course.  Isar offers a rather different
approach to formal proof documents than plain old tactic scripts.  Experienced
users of existing interactive theorem proving systems may have to learn
thinking differently in order to make effective use of Isabelle/Isar.  On the
other hand, Isabelle/Isar comes much closer to existing mathematical practice
of formal proof, so users with less experience in old-style tactical proving,
but a good understanding of mathematical proof, might cope with Isar even
better.  See also \cite{Wenzel:1999:TPHOL} for further background information
on Isar.

\medskip This really is a \emph{reference manual}.  Nevertheless, we will also
give some clues of how the concepts introduced here may be put into practice.
Appendix~\ref{ap:refcard} provides a quick reference card of the most common
Isabelle/Isar language elements.  There are several examples distributed with
Isabelle, and available via the Isabelle WWW library:
\begin{center}\small
  \begin{tabular}{l}
    \url{http://www.cl.cam.ac.uk/Research/HVG/Isabelle/library/} \\
    \url{http://isabelle.in.tum.de/library/} \\
  \end{tabular}
\end{center}

See \texttt{HOL/Isar_examples} for a collection of introductory examples, and
\texttt{HOL/HOL-Real/HahnBanach} is a big mathematics application.  Apart from
browsable HTML sources, both sessions provide actual documents (in PDF).

%%% Local Variables: 
%%% mode: latex
%%% TeX-master: "isar-ref"
%%% End: 


%FIXME
%\chapter{Basic Concepts}\label{ch:basics}
%\section{The Isar proof language}

%%% Local Variables: 
%%% mode: latex
%%% TeX-master: "isar-ref"
%%% End: 


\chapter{Isar document syntax}

\section{Inner versus outer syntax}

\section{Lexical matters}

\section{Common syntax entities}

\subsection{Atoms}

\begin{rail}
  name : ident | symident | string
  ;

  nameref : name | longident
  ;

  text : nameref | verbatim
  ;
\end{rail}

\subsection{Comments}

\begin{rail}
  comment : (() | '--' text)
  ;
  interest : (() | '\%')
  ;
\end{rail}


\subsection{Sorts and arities}

\begin{rail}
  sort : nameref | lbrace (nameref * ',') rbrace
  ;
  arity : ( () | '(' (sort + ',') ')' ) sort
  ;
  simple\-arity : ( () | '(' (sort + ',') ')' ) nameref
  ;
\end{rail}


\subsection{Terms and Types}

\begin{rail}
  
\end{rail}

\subsection{Mixfix annotations}


\subsection{}

\subsection{}

\subsection{}


%%% Local Variables: 
%%% mode: latex
%%% TeX-master: "isar-ref"
%%% End: 


\chapter{Basic Isar Language Elements}\label{ch:pure-syntax}

Subsequently, we introduce the main part of Pure Isar theory and proof
commands, together with fundamental proof methods and attributes.
Chapter~\ref{ch:gen-tools} describes further Isar elements provided by generic
tools and packages (such as the Simplifier) that are either part of Pure
Isabelle or pre-installed by most object logics.  Chapter~\ref{ch:hol-tools}
refers to actual object-logic specific elements of Isabelle/HOL.

\medskip

Isar commands may be either \emph{proper} document constructors, or
\emph{improper commands}.  Some proof methods and attributes introduced later
are classified as improper as well.  Improper Isar language elements, which
are subsequently marked by $^*$, are often helpful when developing proof
documents, while their use is discouraged for the final outcome.  Typical
examples are diagnostic commands that print terms or theorems according to the
current context; other commands even emulate old-style tactical theorem
proving.


\section{Theory commands}

\subsection{Defining theories}\label{sec:begin-thy}

\indexisarcmd{header}\indexisarcmd{theory}\indexisarcmd{end}\indexisarcmd{context}
\begin{matharray}{rcl}
  \isarcmd{header} & : & \isarkeep{toplevel} \\
  \isarcmd{theory} & : & \isartrans{toplevel}{theory} \\
  \isarcmd{context}^* & : & \isartrans{toplevel}{theory} \\
  \isarcmd{end} & : & \isartrans{theory}{toplevel} \\
\end{matharray}

Isabelle/Isar ``new-style'' theories are either defined via theory files or
interactively.  Both theory-level specifications and proofs are handled
uniformly --- occasionally definitional mechanisms even require some explicit
proof as well.  In contrast, ``old-style'' Isabelle theories support batch
processing only, with the proof scripts collected in separate ML files.

The first actual command of any theory has to be $\THEORY$, starting a new
theory based on the merge of existing ones.  Just preceding $\THEORY$, there
may be an optional $\isarkeyword{header}$ declaration, which is relevant to
document preparation only; it acts very much like a special pre-theory markup
command (cf.\ \S\ref{sec:markup-thy} and \S\ref{sec:markup-thy}).  The theory
context may be also changed by $\CONTEXT$ without creating a new theory.  In
both cases, $\END$ concludes the theory development; it has to be the very
last command of any theory file.

\begin{rail}
  'header' text
  ;
  'theory' name '=' (name + '+') filespecs? ':'
  ;
  'context' name
  ;
  'end'
  ;;

  filespecs: 'files' ((name | parname) +);
\end{rail}

\begin{descr}
\item [$\isarkeyword{header}~text$] provides plain text markup just preceding
  the formal beginning of a theory.  In actual document preparation the
  corresponding {\LaTeX} macro \verb,\isamarkupheader, may be redefined to
  produce chapter or section headings.  See also \S\ref{sec:markup-thy} and
  \S\ref{sec:markup-prf} for further markup commands.
  
\item [$\THEORY~A = B@1 + \cdots + B@n\colon$] commences a new theory $A$
  based on existing ones $B@1 + \cdots + B@n$.  Isabelle's theory loader
  system ensures that any of the base theories are properly loaded (and fully
  up-to-date when $\THEORY$ is executed interactively).  The optional
  $\isarkeyword{files}$ specification declares additional dependencies on ML
  files.  Unless put in parentheses, any file will be loaded immediately via
  $\isarcmd{use}$ (see also \S\ref{sec:ML}).  The optional ML file
  \texttt{$A$.ML} that may be associated with any theory should \emph{not} be
  included in $\isarkeyword{files}$, though.
  
\item [$\CONTEXT~B$] enters an existing theory context, basically in read-only
  mode, so only a limited set of commands may be performed without destroying
  the theory.  Just as for $\THEORY$, the theory loader ensures that $B$ is
  loaded and up-to-date.
  
\item [$\END$] concludes the current theory definition or context switch.
Note that this command cannot be undone, but the whole theory definition has
to be retracted.
\end{descr}


\subsection{Theory markup commands}\label{sec:markup-thy}

\indexisarcmd{chapter}\indexisarcmd{section}\indexisarcmd{subsection}
\indexisarcmd{subsubsection}\indexisarcmd{text}\indexisarcmd{text-raw}
\begin{matharray}{rcl}
  \isarcmd{chapter} & : & \isartrans{theory}{theory} \\
  \isarcmd{section} & : & \isartrans{theory}{theory} \\
  \isarcmd{subsection} & : & \isartrans{theory}{theory} \\
  \isarcmd{subsubsection} & : & \isartrans{theory}{theory} \\
  \isarcmd{text} & : & \isartrans{theory}{theory} \\
  \isarcmd{text_raw} & : & \isartrans{theory}{theory} \\
\end{matharray}

Apart from formal comments (see \S\ref{sec:comments}), markup commands provide
a structured way to insert text into the document generated from a theory (see
\cite{isabelle-sys} for more information on Isabelle's document preparation
tools).

\railalias{textraw}{text\_raw}
\railterm{textraw}

\begin{rail}
  ('chapter' | 'section' | 'subsection' | 'subsubsection' | 'text' | textraw) text
  ;
\end{rail}

\begin{descr}
\item [$\isarkeyword{chapter}$, $\isarkeyword{section}$,
  $\isarkeyword{subsection}$, and $\isarkeyword{subsubsection}$] mark chapter
  and section headings.
\item [$\TEXT$] specifies paragraphs of plain text, including references to
  formal entities.\footnote{The latter feature is not yet supported.
    Nevertheless, any source text of the form
    ``\texttt{\at\ttlbrace$\dots$\ttrbrace}'' should be considered as reserved
    for future use.}
\item [$\isarkeyword{text_raw}$] inserts {\LaTeX} source into the output,
  without additional markup.  Thus the full range of document manipulations
  becomes available.  A typical application would be to emit
  \verb,\begin{comment}, and \verb,\end{comment}, commands to exclude certain
  parts from the final document.\footnote{This requires the \texttt{comment}
    package to be included in {\LaTeX}, of course.}
\end{descr}

Any markup command (except $\isarkeyword{text_raw}$) corresponds to a {\LaTeX}
macro with the name prefixed by \verb,\isamarkup, (e.g.\ 
\verb,\isamarkupchapter, for $\isarkeyword{chapter}$). The \railqtoken{text}
argument is passed to that macro unchanged, i.e.\ further {\LaTeX} commands
may be included here as well.

\medskip

Additional markup commands are available for proofs (see
\S\ref{sec:markup-prf}).  Also note that the $\isarkeyword{header}$
declaration (see \S\ref{sec:begin-thy}) admits to insert document markup
elements just preceding the actual theory definition.


\subsection{Type classes and sorts}\label{sec:classes}

\indexisarcmd{classes}\indexisarcmd{classrel}\indexisarcmd{defaultsort}
\begin{matharray}{rcl}
  \isarcmd{classes} & : & \isartrans{theory}{theory} \\
  \isarcmd{classrel} & : & \isartrans{theory}{theory} \\
  \isarcmd{defaultsort} & : & \isartrans{theory}{theory} \\
\end{matharray}

\begin{rail}
  'classes' (classdecl comment? +)
  ;
  'classrel' nameref '<' nameref comment?
  ;
  'defaultsort' sort comment?
  ;
\end{rail}

\begin{descr}
\item [$\isarkeyword{classes}~c<\vec c$] declares class $c$ to be a subclass
  of existing classes $\vec c$.  Cyclic class structures are ruled out.
\item [$\isarkeyword{classrel}~c@1<c@2$] states a subclass relation between
  existing classes $c@1$ and $c@2$.  This is done axiomatically!  The
  $\isarkeyword{instance}$ command (see \S\ref{sec:axclass}) provides a way to
  introduce proven class relations.
\item [$\isarkeyword{defaultsort}~s$] makes sort $s$ the new default sort for
  any type variables given without sort constraints.  Usually, the default
  sort would be only changed when defining new object-logics.
\end{descr}


\subsection{Primitive types and type abbreviations}\label{sec:types-pure}

\indexisarcmd{typedecl}\indexisarcmd{types}\indexisarcmd{nonterminals}\indexisarcmd{arities}
\begin{matharray}{rcl}
  \isarcmd{types} & : & \isartrans{theory}{theory} \\
  \isarcmd{typedecl} & : & \isartrans{theory}{theory} \\
  \isarcmd{nonterminals} & : & \isartrans{theory}{theory} \\
  \isarcmd{arities} & : & \isartrans{theory}{theory} \\
\end{matharray}

\begin{rail}
  'types' (typespec '=' type infix? comment? +)
  ;
  'typedecl' typespec infix? comment?
  ;
  'nonterminals' (name +) comment?
  ;
  'arities' (nameref '::' arity comment? +)
  ;
\end{rail}

\begin{descr}
\item [$\TYPES~(\vec\alpha)t = \tau$] introduces \emph{type synonym}
  $(\vec\alpha)t$ for existing type $\tau$.  Unlike actual type definitions,
  as are available in Isabelle/HOL for example, type synonyms are just purely
  syntactic abbreviations without any logical significance.  Internally, type
  synonyms are fully expanded.
\item [$\isarkeyword{typedecl}~(\vec\alpha)t$] declares a new type constructor
  $t$, intended as an actual logical type.  Note that object-logics such as
  Isabelle/HOL override $\isarkeyword{typedecl}$ by their own version.
\item [$\isarkeyword{nonterminals}~\vec c$] declares $0$-ary type constructors
  $\vec c$ to act as purely syntactic types, i.e.\ nonterminal symbols of
  Isabelle's inner syntax of terms or types.
\item [$\isarkeyword{arities}~t::(\vec s)s$] augments Isabelle's order-sorted
  signature of types by new type constructor arities.  This is done
  axiomatically!  The $\isarkeyword{instance}$ command (see
  \S\ref{sec:axclass}) provides a way to introduce proven type arities.
\end{descr}


\subsection{Constants and simple definitions}\label{sec:consts}

\indexisarcmd{consts}\indexisarcmd{defs}\indexisarcmd{constdefs}\indexoutertoken{constdecl}
\begin{matharray}{rcl}
  \isarcmd{consts} & : & \isartrans{theory}{theory} \\
  \isarcmd{defs} & : & \isartrans{theory}{theory} \\
  \isarcmd{constdefs} & : & \isartrans{theory}{theory} \\
\end{matharray}

\begin{rail}
  'consts' (constdecl +)
  ;
  'defs' (axmdecl prop comment? +)
  ;
  'constdefs' (constdecl prop comment? +)
  ;

  constdecl: name '::' type mixfix? comment?
  ;
\end{rail}

\begin{descr}
\item [$\CONSTS~c::\sigma$] declares constant $c$ to have any instance of type
  scheme $\sigma$.  The optional mixfix annotations may attach concrete syntax
  to the constants declared.
\item [$\DEFS~name: eqn$] introduces $eqn$ as a definitional axiom for some
  existing constant.  See \cite[\S6]{isabelle-ref} for more details on the
  form of equations admitted as constant definitions.
\item [$\isarkeyword{constdefs}~c::\sigma~eqn$] combines declarations and
  definitions of constants, using the canonical name $c_def$ for the
  definitional axiom.
\end{descr}


\subsection{Syntax and translations}\label{sec:syn-trans}

\indexisarcmd{syntax}\indexisarcmd{translations}
\begin{matharray}{rcl}
  \isarcmd{syntax} & : & \isartrans{theory}{theory} \\
  \isarcmd{translations} & : & \isartrans{theory}{theory} \\
\end{matharray}

\begin{rail}
  'syntax' ('(' name 'output'? ')')? (constdecl +)
  ;
  'translations' (transpat ('==' | '=>' | '<=') transpat comment? +)
  ;
  transpat: ('(' nameref ')')? string
  ;
\end{rail}

\begin{descr}
\item [$\isarkeyword{syntax}~(mode)~decls$] is similar to $\CONSTS~decls$,
  except that the actual logical signature extension is omitted.  Thus the
  context free grammar of Isabelle's inner syntax may be augmented in
  arbitrary ways, independently of the logic.  The $mode$ argument refers to
  the print mode that the grammar rules belong; unless the \texttt{output}
  flag is given, all productions are added both to the input and output
  grammar.
\item [$\isarkeyword{translations}~rules$] specifies syntactic translation
  rules (i.e.\ \emph{macros}): parse~/ print rules (\texttt{==}), parse rules
  (\texttt{=>}), or print rules (\texttt{<=}).  Translation patterns may be
  prefixed by the syntactic category to be used for parsing; the default is
  \texttt{logic}.
\end{descr}


\subsection{Axioms and theorems}

\indexisarcmd{axioms}\indexisarcmd{theorems}\indexisarcmd{lemmas}
\begin{matharray}{rcl}
  \isarcmd{axioms} & : & \isartrans{theory}{theory} \\
  \isarcmd{theorems} & : & \isartrans{theory}{theory} \\
  \isarcmd{lemmas} & : & \isartrans{theory}{theory} \\
\end{matharray}

\begin{rail}
  'axioms' (axmdecl prop comment? +)
  ;
  ('theorems' | 'lemmas') thmdef? thmrefs
  ;
\end{rail}

\begin{descr}
\item [$\isarkeyword{axioms}~a: \phi$] introduces arbitrary statements as
  axioms of the meta-logic.  In fact, axioms are ``axiomatic theorems'', and
  may be referred later just as any other theorem.
  
  Axioms are usually only introduced when declaring new logical systems.
  Everyday work is typically done the hard way, with proper definitions and
  actual proven theorems.
\item [$\isarkeyword{theorems}~a = \vec b$] stores lists of existing theorems.
  Typical applications would also involve attributes, to declare Simplifier
  rules, for example.
\item [$\isarkeyword{lemmas}$] is similar to $\isarkeyword{theorems}$, but
  tags the results as ``lemma''.
\end{descr}


\subsection{Name spaces}

\indexisarcmd{global}\indexisarcmd{local}
\begin{matharray}{rcl}
  \isarcmd{global} & : & \isartrans{theory}{theory} \\
  \isarcmd{local} & : & \isartrans{theory}{theory} \\
\end{matharray}

Isabelle organizes any kind of name declarations (of types, constants,
theorems etc.) by separate hierarchically structured name spaces.  Normally
the user never has to control the behavior of name space entry by hand, yet
the following commands provide some way to do so.

\begin{descr}
\item [$\isarkeyword{global}$ and $\isarkeyword{local}$] change the current
  name declaration mode.  Initially, theories start in $\isarkeyword{local}$
  mode, causing all names to be automatically qualified by the theory name.
  Changing this to $\isarkeyword{global}$ causes all names to be declared
  without the theory prefix, until $\isarkeyword{local}$ is declared again.
\end{descr}


\subsection{Incorporating ML code}\label{sec:ML}

\indexisarcmd{use}\indexisarcmd{ML}\indexisarcmd{ML-setup}\indexisarcmd{setup}
\begin{matharray}{rcl}
  \isarcmd{use} & : & \isartrans{\cdot}{\cdot} \\
  \isarcmd{ML} & : & \isartrans{\cdot}{\cdot} \\
  \isarcmd{ML_setup} & : & \isartrans{theory}{theory} \\
  \isarcmd{setup} & : & \isartrans{theory}{theory} \\
\end{matharray}

\railalias{MLsetup}{ML\_setup}
\railterm{MLsetup}

\begin{rail}
  'use' name
  ;
  ('ML' | MLsetup | 'setup') text
  ;
\end{rail}

\begin{descr}
\item [$\isarkeyword{use}~file$] reads and executes ML commands from $file$.
  The current theory context (if present) is passed down to the ML session,
  but may not be modified.  Furthermore, the file name is checked with the
  $\isarkeyword{files}$ dependency declaration given in the theory header (see
  also \S\ref{sec:begin-thy}).
  
\item [$\isarkeyword{ML}~text$] executes ML commands from $text$.  The theory
  context is passed in the same way as for $\isarkeyword{use}$.
  
\item [$\isarkeyword{ML_setup}~text$] executes ML commands from $text$.  The
  theory context is passed down to the ML session, and fetched back
  afterwards.  Thus $text$ may actually change the theory as a side effect.
  
\item [$\isarkeyword{setup}~text$] changes the current theory context by
  applying $text$, which refers to an ML expression of type
  \texttt{(theory~->~theory)~list}.  The $\isarkeyword{setup}$ command is the
  canonical way to initialize any object-logic specific tools and packages
  written in ML.
\end{descr}


\subsection{Syntax translation functions}

\indexisarcmd{parse-ast-translation}\indexisarcmd{parse-translation}
\indexisarcmd{print-translation}\indexisarcmd{typed-print-translation}
\indexisarcmd{print-ast-translation}\indexisarcmd{token-translation}
\begin{matharray}{rcl}
  \isarcmd{parse_ast_translation} & : & \isartrans{theory}{theory} \\
  \isarcmd{parse_translation} & : & \isartrans{theory}{theory} \\
  \isarcmd{print_translation} & : & \isartrans{theory}{theory} \\
  \isarcmd{typed_print_translation} & : & \isartrans{theory}{theory} \\
  \isarcmd{print_ast_translation} & : & \isartrans{theory}{theory} \\
  \isarcmd{token_translation} & : & \isartrans{theory}{theory} \\
\end{matharray}

Syntax translation functions written in ML admit almost arbitrary
manipulations of Isabelle's inner syntax.  Any of the above commands have a
single \railqtoken{text} argument that refers to an ML expression of
appropriate type.

\begin{ttbox}
val parse_ast_translation   : (string * (ast list -> ast)) list
val parse_translation       : (string * (term list -> term)) list
val print_translation       : (string * (term list -> term)) list
val typed_print_translation :
  (string * (bool -> typ -> term list -> term)) list
val print_ast_translation   : (string * (ast list -> ast)) list
val token_translation       :
  (string * string * (string -> string * real)) list
\end{ttbox}
See \cite[\S8]{isabelle-ref} for more information on syntax transformations.


\subsection{Oracles}

\indexisarcmd{oracle}
\begin{matharray}{rcl}
  \isarcmd{oracle} & : & \isartrans{theory}{theory} \\
\end{matharray}

Oracles provide an interface to external reasoning systems, without giving up
control completely --- each theorem carries a derivation object recording any
oracle invocation.  See \cite[\S6]{isabelle-ref} for more information.

\begin{rail}
  'oracle' name '=' text comment?
  ;
\end{rail}

\begin{descr}
\item [$\isarkeyword{oracle}~name=text$] declares oracle $name$ to be ML
  function $text$, which has to be of type
  \texttt{Sign.sg~*~Object.T~->~term}.
\end{descr}


\section{Proof commands}

Proof commands perform transitions of Isar/VM machine configurations, which
are block-structured, consisting of a stack of nodes with three main
components: logical proof context, current facts, and open goals.  Isar/VM
transitions are \emph{typed} according to the following three different modes
of operation:
\begin{descr}
\item [$proof(prove)$] means that a new goal has just been stated that is now
  to be \emph{proven}; the next command may refine it by some proof method,
  and enter a sub-proof to establish the actual result.
\item [$proof(state)$] is like an internal theory mode: the context may be
  augmented by \emph{stating} additional assumptions, intermediate results
  etc.
\item [$proof(chain)$] is intermediate between $proof(state)$ and
  $proof(prove)$: existing facts (i.e.\ the contents of the special ``$this$''
  register) have been just picked up in order to be used when refining the
  goal claimed next.
\end{descr}


\subsection{Proof markup commands}\label{sec:markup-prf}

\indexisarcmd{sect}\indexisarcmd{subsect}\indexisarcmd{subsubsect}
\indexisarcmd{txt}\indexisarcmd{txt-raw}
\begin{matharray}{rcl}
  \isarcmd{sect} & : & \isartrans{proof}{proof} \\
  \isarcmd{subsect} & : & \isartrans{proof}{proof} \\
  \isarcmd{subsubsect} & : & \isartrans{proof}{proof} \\
  \isarcmd{txt} & : & \isartrans{proof}{proof} \\
  \isarcmd{txt_raw} & : & \isartrans{proof}{proof} \\
\end{matharray}

These markup commands for proof mode closely correspond to the ones of theory
mode (see \S\ref{sec:markup-thy}).  Note that $\isarkeyword{txt_raw}$ is
special in the same way as $\isarkeyword{text_raw}$.

\railalias{txtraw}{txt\_raw}
\railterm{txtraw}

\begin{rail}
  ('sect' | 'subsect' | 'subsubsect' | 'txt' | txtraw) text
  ;
\end{rail}


\subsection{Proof context}\label{sec:proof-context}

\indexisarcmd{fix}\indexisarcmd{assume}\indexisarcmd{presume}\indexisarcmd{def}
\begin{matharray}{rcl}
  \isarcmd{fix} & : & \isartrans{proof(state)}{proof(state)} \\
  \isarcmd{assume} & : & \isartrans{proof(state)}{proof(state)} \\
  \isarcmd{presume} & : & \isartrans{proof(state)}{proof(state)} \\
  \isarcmd{def} & : & \isartrans{proof(state)}{proof(state)} \\
\end{matharray}

The logical proof context consists of fixed variables and assumptions.  The
former closely correspond to Skolem constants, or meta-level universal
quantification as provided by the Isabelle/Pure logical framework.
Introducing some \emph{arbitrary, but fixed} variable via $\FIX x$ results in
a local value that may be used in the subsequent proof as any other variable
or constant.  Furthermore, any result $\edrv \phi[x]$ exported from the
context will be universally closed wrt.\ $x$ at the outermost level: $\edrv
\All x \phi$ (this is expressed using Isabelle's meta-variables).

Similarly, introducing some assumption $\chi$ has two effects.  On the one
hand, a local theorem is created that may be used as a fact in subsequent
proof steps.  On the other hand, any result $\chi \drv \phi$ exported from the
context becomes conditional wrt.\ the assumption: $\edrv \chi \Imp \phi$.
Thus, solving an enclosing goal using such a result would basically introduce
a new subgoal stemming from the assumption.  How this situation is handled
depends on the actual version of assumption command used: while $\ASSUMENAME$
insists on solving the subgoal by unification with some premise of the goal,
$\PRESUMENAME$ leaves the subgoal unchanged in order to be proved later by the
user.

Local definitions, introduced by $\DEF{}{x \equiv t}$, are achieved by
combining $\FIX x$ with another version of assumption that causes any
hypothetical equation $x \equiv t$ to be eliminated by the reflexivity rule.
Thus, exporting some result $x \equiv t \drv \phi[x]$ yields $\edrv \phi[t]$.

\begin{rail}
  'fix' (vars + 'and') comment?
  ;
  ('assume' | 'presume') (assm comment? + 'and')
  ;
  'def' thmdecl? \\ var '==' term termpat? comment?
  ;

  var: name ('::' type)?
  ;
  vars: (name+) ('::' type)?
  ;
  assm: thmdecl? (prop proppat? +)
  ;
\end{rail}

\begin{descr}
\item [$\FIX{\vec x}$] introduces local \emph{arbitrary, but fixed} variables
  $\vec x$.
\item [$\ASSUME{a}{\vec\phi}$ and $\PRESUME{a}{\vec\phi}$] introduce local
  theorems $\vec\phi$ by assumption.  Subsequent results applied to an
  enclosing goal (e.g.\ by $\SHOWNAME$) are handled as follows: $\ASSUMENAME$
  expects to be able to unify with existing premises in the goal, while
  $\PRESUMENAME$ leaves $\vec\phi$ as new subgoals.
  
  Several lists of assumptions may be given (separated by
  $\isarkeyword{and}$); the resulting list of current facts consists of all of
  these concatenated.
\item [$\DEF{a}{x \equiv t}$] introduces a local (non-polymorphic) definition.
  In results exported from the context, $x$ is replaced by $t$.  Basically,
  $\DEF{}{x \equiv t}$ abbreviates $\FIX{x}~\ASSUME{}{x \equiv t}$, with the
  resulting hypothetical equation solved by reflexivity.
  
  The default name for the definitional equation is $x_def$.
\end{descr}

The special name $prems$\indexisarthm{prems} refers to all assumptions of the
current context as a list of theorems.


\subsection{Facts and forward chaining}

\indexisarcmd{note}\indexisarcmd{then}\indexisarcmd{from}\indexisarcmd{with}
\begin{matharray}{rcl}
  \isarcmd{note} & : & \isartrans{proof(state)}{proof(state)} \\
  \isarcmd{then} & : & \isartrans{proof(state)}{proof(chain)} \\
  \isarcmd{from} & : & \isartrans{proof(state)}{proof(chain)} \\
  \isarcmd{with} & : & \isartrans{proof(state)}{proof(chain)} \\
\end{matharray}

New facts are established either by assumption or proof of local statements.
Any fact will usually be involved in further proofs, either as explicit
arguments of proof methods, or when forward chaining towards the next goal via
$\THEN$ (and variants).  Note that the special theorem name
$this$\indexisarthm{this} refers to the most recently established facts.
\begin{rail}
  'note' thmdef? thmrefs comment?
  ;
  'then' comment?
  ;
  ('from' | 'with') thmrefs comment?
  ;
\end{rail}

\begin{descr}
\item [$\NOTE{a}{\vec b}$] recalls existing facts $\vec b$, binding the result
  as $a$.  Note that attributes may be involved as well, both on the left and
  right hand sides.
\item [$\THEN$] indicates forward chaining by the current facts in order to
  establish the goal to be claimed next.  The initial proof method invoked to
  refine that will be offered the facts to do ``anything appropriate'' (cf.\ 
  also \S\ref{sec:proof-steps}).  For example, method $rule$ (see
  \S\ref{sec:pure-meth-att}) would typically do an elimination rather than an
  introduction.  Automatic methods usually insert the facts into the goal
  state before operation.  This provides a simple scheme to control relevance
  of facts in automated proof search.
\item [$\FROM{\vec b}$] abbreviates $\NOTE{}{\vec b}~\THEN$; thus $\THEN$ is
  equivalent to $\FROM{this}$.
\item [$\WITH{\vec b}$] abbreviates $\FROM{\vec b~facts}$; thus the forward
  chaining is from earlier facts together with the current ones.
\end{descr}

Basic proof methods (such as $rule$, see \S\ref{sec:pure-meth-att}) expect
multiple facts to be given in their proper order, corresponding to a prefix of
the premises of the rule involved.  Note that positions may be easily skipped
using something like $\FROM{\text{\texttt{_}}~a~b}$, for example.  This
involves the trivial rule $\PROP\psi \Imp \PROP\psi$, which happens to be
bound in Isabelle/Pure as ``\texttt{_}''
(underscore).\indexisarthm{_@\texttt{_}}


\subsection{Goal statements}

\indexisarcmd{theorem}\indexisarcmd{lemma}
\indexisarcmd{have}\indexisarcmd{show}\indexisarcmd{hence}\indexisarcmd{thus}
\begin{matharray}{rcl}
  \isarcmd{theorem} & : & \isartrans{theory}{proof(prove)} \\
  \isarcmd{lemma} & : & \isartrans{theory}{proof(prove)} \\
  \isarcmd{have} & : & \isartrans{proof(state) ~|~ proof(chain)}{proof(prove)} \\
  \isarcmd{show} & : & \isartrans{proof(state) ~|~ proof(chain)}{proof(prove)} \\
  \isarcmd{hence} & : & \isartrans{proof(state)}{proof(prove)} \\
  \isarcmd{thus} & : & \isartrans{proof(state)}{proof(prove)} \\
\end{matharray}

Proof mode is entered from theory mode by initial goal commands $\THEOREMNAME$
and $\LEMMANAME$.  New local goals may be claimed within proof mode as well.
Four variants are available, indicating whether the result is meant to solve
some pending goal or whether forward chaining is indicated.

\begin{rail}
  ('theorem' | 'lemma') goal
  ;
  ('have' | 'show' | 'hence' | 'thus') goal
  ;

  goal: thmdecl? proppat comment?
  ;
\end{rail}

\begin{descr}
\item [$\THEOREM{a}{\phi}$] enters proof mode with $\phi$ as main goal,
  eventually resulting in some theorem $\turn \phi$ to be put back into the
  theory.
\item [$\LEMMA{a}{\phi}$] is similar to $\THEOREMNAME$, but tags the result as
  ``lemma''.
\item [$\HAVE{a}{\phi}$] claims a local goal, eventually resulting in a
  theorem with the current assumption context as hypotheses.
\item [$\SHOW{a}{\phi}$] is similar to $\HAVE{a}{\phi}$, but solves some
  pending goal with the result \emph{exported} into the corresponding context
  (cf.\ \S\ref{sec:proof-context}).
\item [$\HENCENAME$] abbreviates $\THEN~\HAVENAME$, i.e.\ claims a local goal
  to be proven by forward chaining the current facts.  Note that $\HENCENAME$
  is also equivalent to $\FROM{this}~\HAVENAME$.
\item [$\THUSNAME$] abbreviates $\THEN~\SHOWNAME$.  Note that $\THUSNAME$ is
  also equivalent to $\FROM{this}~\SHOWNAME$.
\end{descr}


\subsection{Initial and terminal proof steps}\label{sec:proof-steps}

\indexisarcmd{proof}\indexisarcmd{qed}\indexisarcmd{by}
\indexisarcmd{.}\indexisarcmd{..}\indexisarcmd{sorry}
\begin{matharray}{rcl}
  \isarcmd{proof} & : & \isartrans{proof(prove)}{proof(state)} \\
  \isarcmd{qed} & : & \isartrans{proof(state)}{proof(state) ~|~ theory} \\
  \isarcmd{by} & : & \isartrans{proof(prove)}{proof(state) ~|~ theory} \\
  \isarcmd{.\,.} & : & \isartrans{proof(prove)}{proof(state) ~|~ theory} \\
  \isarcmd{.} & : & \isartrans{proof(prove)}{proof(state) ~|~ theory} \\
  \isarcmd{sorry} & : & \isartrans{proof(prove)}{proof(state) ~|~ theory} \\
\end{matharray}

Arbitrary goal refinement via tactics is considered harmful.  Properly, the
Isar framework admits proof methods to be invoked in two places only.
\begin{enumerate}
\item An \emph{initial} refinement step $\PROOF{m@1}$ reduces a newly stated
  goal to a number of sub-goals that are to be solved later.  Facts are passed
  to $m@1$ for forward chaining, if so indicated by $proof(chain)$ mode.
  
\item A \emph{terminal} conclusion step $\QED{m@2}$ is intended to solve
  remaining goals.  No facts are passed to $m@2$.
\end{enumerate}

The only other proper way to affect pending goals is by $\SHOWNAME$, which
involves an explicit statement of what is to be solved.

\medskip

Also note that initial proof methods should either solve the goal completely,
or constitute some well-understood reduction to new sub-goals.  Arbitrary
automatic proof tools that are prone leave a large number of badly structured
sub-goals are no help in continuing the proof document in any intelligible
way.

\medskip

Unless given explicitly by the user, the default initial method is ``$rule$'',
which applies a single standard elimination or introduction rule according to
the topmost symbol involved.  There is no separate default terminal method.
Any remaining goals are always solved by assumption in the very last step.

\begin{rail}
  'proof' interest? meth? comment?
  ;
  'qed' meth? comment?
  ;
  'by' meth meth? comment?
  ;
  ('.' | '..' | 'sorry') comment?
  ;

  meth: method interest?
  ;
\end{rail}

\begin{descr}
\item [$\PROOF{m@1}$] refines the goal by proof method $m@1$; facts for
  forward chaining are passed if so indicated by $proof(chain)$ mode.
\item [$\QED{m@2}$] refines any remaining goals by proof method $m@2$ and
  concludes the sub-proof by assumption.  If the goal had been $\SHOWNAME$ (or
  $\THUSNAME$), some pending sub-goal is solved as well by the rule resulting
  from the result \emph{exported} into the enclosing goal context.  Thus
  $\QEDNAME$ may fail for two reasons: either $m@2$ fails, or the resulting
  rule does not fit to any pending goal\footnote{This includes any additional
    ``strong'' assumptions as introduced by $\ASSUMENAME$.} of the enclosing
  context.  Debugging such a situation might involve temporarily changing
  $\SHOWNAME$ into $\HAVENAME$, or weakening the local context by replacing
  some occurrences of $\ASSUMENAME$ by $\PRESUMENAME$.
\item [$\BYY{m@1}{m@2}$] is a \emph{terminal proof}\index{proof!terminal}; it
  abbreviates $\PROOF{m@1}~\QED{m@2}$, with backtracking across both methods,
  though.  Debugging an unsuccessful $\BYY{m@1}{m@2}$ commands might be done
  by expanding its definition; in many cases $\PROOF{m@1}$ is already
  sufficient to see what is going wrong.
\item [``$\DDOT$''] is a \emph{default proof}\index{proof!default}; it
  abbreviates $\BY{rule}$.
\item [``$\DOT$''] is a \emph{trivial proof}\index{proof!trivial}; it
  abbreviates $\BY{this}$.
\item [$\SORRY$] is a \emph{fake proof}\index{proof!fake}; provided that the
  \texttt{quick_and_dirty} flag is enabled, $\SORRY$ pretends to solve the
  goal without further ado.  Of course, the result would be a fake theorem
  only, involving some oracle in its internal derivation object (this is
  indicated as ``$[!]$'' in the printed result).  The main application of
  $\SORRY$ is to support experimentation and top-down proof development.
\end{descr}


\subsection{Fundamental methods and attributes}\label{sec:pure-meth-att}

The following proof methods and attributes refer to basic logical operations
of Isar.  Further methods and attributes are provided by several generic and
object-logic specific tools and packages (see chapters \ref{ch:gen-tools} and
\ref{ch:hol-tools}).

\indexisarmeth{assumption}\indexisarmeth{this}\indexisarmeth{rule}\indexisarmeth{$-$}
\indexisaratt{intro}\indexisaratt{elim}\indexisaratt{dest}
\indexisaratt{OF}\indexisaratt{of}
\begin{matharray}{rcl}
  assumption & : & \isarmeth \\
  this & : & \isarmeth \\
  rule & : & \isarmeth \\
  - & : & \isarmeth \\
  OF & : & \isaratt \\
  of & : & \isaratt \\
  intro & : & \isaratt \\
  elim & : & \isaratt \\
  dest & : & \isaratt \\
  delrule & : & \isaratt \\
\end{matharray}

\begin{rail}
  'rule' thmrefs?
  ;
  'OF' thmrefs
  ;
  'of' (inst * ) ('concl' ':' (inst * ))?
  ;

  inst: underscore | term
  ;
\end{rail}

\begin{descr}
\item [$assumption$] solves some goal by a single assumption step.  Any facts
  given (${} \le 1$) are guaranteed to participate in the refinement.  Recall
  that $\QEDNAME$ (see \S\ref{sec:proof-steps}) already concludes any
  remaining sub-goals by assumption.
\item [$this$] applies all of the current facts directly as rules.  Recall
  that ``$\DOT$'' (dot) abbreviates $\BY{this}$.
\item [$rule~\vec a$] applies some rule given as argument in backward manner;
  facts are used to reduce the rule before applying it to the goal.  Thus
  $rule$ without facts is plain \emph{introduction}, while with facts it
  becomes \emph{elimination}.
  
  When no arguments are given, the $rule$ method tries to pick appropriate
  rules automatically, as declared in the current context using the $intro$,
  $elim$, $dest$ attributes (see below).  This is the default behavior of
  $\PROOFNAME$ and ``$\DDOT$'' (double-dot) steps (see
  \S\ref{sec:proof-steps}).
\item [``$-$''] does nothing but insert the forward chaining facts as premises
  into the goal.  Note that command $\PROOFNAME$ without any method actually
  performs a single reduction step using the $rule$ method; thus a plain
  \emph{do-nothing} proof step would be $\PROOF{-}$ rather than $\PROOFNAME$
  alone.
\item [$OF~\vec a$] applies some theorem to given rules $\vec a$ (in
  parallel).  This corresponds to the \texttt{MRS} operator in ML
  \cite[\S5]{isabelle-ref}, but note the reversed order.  Positions may be
  skipped by including ``$\_$'' (underscore) as argument.
\item [$of~\vec t$] performs positional instantiation.  The terms $\vec t$ are
  substituted for any schematic variables occurring in a theorem from left to
  right; ``\texttt{_}'' (underscore) indicates to skip a position.  Arguments
  following a ``$concl\colon$'' specification refer to positions of the
  conclusion of a rule.
\item [$intro$, $elim$, and $dest$] declare introduction, elimination, and
  destruct rules, respectively.  Note that the classical reasoner (see
  \S\ref{sec:classical-basic}) introduces different versions of these
  attributes, and the $rule$ method, too.  In object-logics with classical
  reasoning enabled, the latter version should be used all the time to avoid
  confusion!
\item [$delrule$] undeclares introduction or elimination rules.
\end{descr}


\subsection{Term abbreviations}\label{sec:term-abbrev}

\indexisarcmd{let}
\begin{matharray}{rcl}
  \isarcmd{let} & : & \isartrans{proof(state)}{proof(state)} \\
  \isarkeyword{is} & : & syntax \\
\end{matharray}

Abbreviations may be either bound by explicit $\LET{p \equiv t}$ statements,
or by annotating assumptions or goal statements with a list of patterns
$\ISS{p@1\;\dots}{p@n}$.  In both cases, higher-order matching is invoked to
bind extra-logical term variables, which may be either named schematic
variables of the form $\Var{x}$, or nameless dummies ``\texttt{_}''
(underscore).\indexisarvar{_@\texttt{_}} Note that in the $\LETNAME$ form the
patterns occur on the left-hand side, while the $\ISNAME$ patterns are in
postfix position.

Polymorphism of term bindings is handled in Hindley-Milner style, as in ML.
Type variables referring to local assumptions or open goal statements are
\emph{fixed}, while those of finished results or bound by $\LETNAME$ may occur
in \emph{arbitrary} instances later.  Even though actual polymorphism should
be rarely used in practice, this mechanism is essential to achieve proper
incremental type-inference, as the user proceeds to build up the Isar proof
text.

\medskip

Term abbreviations are quite different from actual local definitions as
introduced via $\DEFNAME$ (see \S\ref{sec:proof-context}).  The latter are
visible within the logic as actual equations, while abbreviations disappear
during the input process just after type checking.  Also note that $\DEFNAME$
does not support polymorphism.

\begin{rail}
  'let' ((term + 'as') '=' term comment? + 'and')
  ;  
\end{rail}

The syntax of $\ISNAME$ patterns follows \railnonterm{termpat} or
\railnonterm{proppat} (see \S\ref{sec:term-pats}).

\begin{descr}
\item [$\LET{\vec p = \vec t}$] binds any text variables in patters $\vec p$
  by simultaneous higher-order matching against terms $\vec t$.
\item [$\IS{\vec p}$] resembles $\LETNAME$, but matches $\vec p$ against the
  preceding statement.  Also note that $\ISNAME$ is not a separate command,
  but part of others (such as $\ASSUMENAME$, $\HAVENAME$ etc.).
\end{descr}

A few \emph{automatic} term abbreviations\index{term abbreviations} for goals
and facts are available as well.  For any open goal,
$\Var{thesis_prop}$\indexisarvar{thesis-prop} refers to the full proposition
(which may be a rule), $\Var{thesis_concl}$\indexisarvar{thesis-concl} to its
(atomic) conclusion, and $\Var{thesis}$\indexisarvar{thesis} to its
object-level statement.  The latter two abstract over any meta-level
parameters.

Fact statements resulting from assumptions or finished goals are bound as
$\Var{this_prop}$\indexisarvar{this-prop},
$\Var{this_concl}$\indexisarvar{this-concl}, and
$\Var{this}$\indexisarvar{this}, similar to $\Var{thesis}$ above.  In case
$\Var{this}$ refers to an object-logic statement that is an application
$f(t)$, then $t$ is bound to the special text variable
``$\dots$''\indexisarvar{\dots} (three dots).  The canonical application of
the latter are calculational proofs (see \S\ref{sec:calculation}).


\subsection{Block structure}

\indexisarcmd{next}\indexisarcmd{\{\{}\indexisarcmd{\}\}}
\begin{matharray}{rcl}
  \NEXT & : & \isartrans{proof(state)}{proof(state)} \\
  \BG & : & \isartrans{proof(state)}{proof(state)} \\
  \EN & : & \isartrans{proof(state)}{proof(state)} \\
\end{matharray}

While Isar is inherently block-structured, opening and closing blocks is
mostly handled rather casually, with little explicit user-intervention.  Any
local goal statement automatically opens \emph{two} blocks, which are closed
again when concluding the sub-proof (by $\QEDNAME$ etc.).  Sections of
different context within a sub-proof may be switched via $\NEXT$, which is
just a single block-close followed by block-open again.  Thus the effect of
$\NEXT$ to reset the local proof context. There is no goal focus involved
here!

For slightly more advanced applications, there are explicit block parentheses
as well.  These typically achieve a stronger forward style of reasoning.

\begin{descr}
\item [$\NEXT$] switches to a fresh block within a sub-proof, resetting the
  local context to the initial one.
\item [$\isarkeyword{\{\{}$ and $\isarkeyword{\}\}}$] explicitly open and
  close blocks.  Any current facts pass through ``$\isarkeyword{\{\{}$''
  unchanged, while ``$\isarkeyword{\}\}}$'' causes any result to be
  \emph{exported} into the enclosing context.  Thus fixed variables are
  generalized, assumptions discharged, and local definitions unfolded (cf.\ 
  \S\ref{sec:proof-context}).  There is no difference of $\ASSUMENAME$ and
  $\PRESUMENAME$ in this mode of forward reasoning --- in contrast to plain
  backward reasoning with the result exported at $\SHOWNAME$ time.
\end{descr}


\subsection{Emulating tactic scripts}\label{sec:tactical-proof}

The following elements emulate unstructured tactic scripts to some extent.
While these are anathema for writing proper Isar proof documents, they might
come in handy for interactive exploration and debugging, or even actual
tactical proof within new-style theories (to benefit from document
preparation, for example).

\indexisarcmd{apply}\indexisarcmd{apply-end}
\indexisarcmd{defer}\indexisarcmd{prefer}\indexisarcmd{back}
\indexisarmeth{tactic}
\indexisarmeth{res-inst-tac}\indexisarmeth{eres-inst-tac}
\indexisarmeth{dres-inst-tac}\indexisarmeth{forw-inst-tac}
\indexisarmeth{subgoal-tac}
\begin{matharray}{rcl}
  \isarcmd{apply}^* & : & \isartrans{proof(prove)}{proof(prove)} \\
  \isarcmd{apply_end}^* & : & \isartrans{proof(state)}{proof(state)} \\
  \isarcmd{defer}^* & : & \isartrans{proof}{proof} \\
  \isarcmd{prefer}^* & : & \isartrans{proof}{proof} \\
  \isarcmd{back}^* & : & \isartrans{proof}{proof} \\
  tactic^* & : & \isarmeth \\
  res_inst_tac^* & : & \isarmeth \\
  eres_inst_tac^* & : & \isarmeth \\
  dres_inst_tac^* & : & \isarmeth \\
  forw_inst_tac^* & : & \isarmeth \\
  subgoal_tac^* & : & \isarmeth \\
\end{matharray}

\railalias{applyend}{apply\_end}
\railterm{applyend}

\railalias{resinsttac}{res\_inst\_tac}
\railterm{resinsttac}

\railalias{eresinsttac}{eres\_inst\_tac}
\railterm{eresinsttac}

\railalias{dresinsttac}{dres\_inst\_tac}
\railterm{dresinsttac}

\railalias{forwinsttac}{forw\_inst\_tac}
\railterm{forwinsttac}

\railalias{subgoaltac}{subgoal\_tac}
\railterm{subgoaltac}

\begin{rail}
  'apply' method
  ;
  applyend method
  ;
  'defer' nat?
  ;
  'prefer' nat
  ;
  'tactic' text
  ;
  ( resinsttac | eresinsttac | dresinsttac | forwinsttac ) goalspec? ((name '=' term) + 'and')
  ;
  subgoaltac goalspec? prop
  ;
\end{rail}

\begin{descr}
\item [$\isarkeyword{apply}~(m)$] applies proof method $m$ in initial
  position, but unlike $\PROOFNAME$ it retains ``$proof(prove)$'' mode.  Thus
  consecutive method applications may be given just as in tactic scripts.  In
  order to complete the proof properly, any of the actual structured proof
  commands (e.g.\ ``$\DOT$'') has to be given eventually.
  
  Facts are passed to $m$ as indicated by the goal's forward-chain mode.
  Common use of $\isarkeyword{apply}$ would be in a purely backward manner,
  though.
\item [$\isarkeyword{apply_end}~(m)$] applies proof method $m$ as if in
  terminal position.  Basically, this simulates a multi-step tactic script for
  $\QEDNAME$, but may be given anywhere within the proof body.
  
  No facts are passed to $m$.  Furthermore, the static context is that of the
  enclosing goal (as for actual $\QEDNAME$).  Thus the proof method may not
  refer to any assumptions introduced in the current body, for example.
\item [$\isarkeyword{defer}~n$ and $\isarkeyword{prefer}~n$] shuffle the list
  of pending goals: $defer$ puts off goal $n$ to the end of the list ($n = 1$
  by default), while $prefer$ brings goal $n$ to the top.
\item [$\isarkeyword{back}$] does back-tracking over the result sequence of
  the latest proof command.\footnote{Unlike the ML function \texttt{back}
    \cite{isabelle-ref}, the Isar command does not search upwards for further
    branch points.} Basically, any proof command may return multiple results.
\item [$tactic~text$] produces a proof method from any ML text of type
  \texttt{tactic}.  Apart from the usual ML environment and the current
  implicit theory context, the ML code may refer to the following locally
  bound values:
%%FIXME ttbox produces too much trailing space (why?)
{\footnotesize\begin{verbatim}
val ctxt  : Proof.context
val facts : thm list
val thm   : string -> thm
val thms  : string -> thm list
\end{verbatim}}
  Here \texttt{ctxt} refers to the current proof context, \texttt{facts}
  indicates any current facts for forward-chaining, and
  \texttt{thm}~/~\texttt{thms} retrieve named facts (including global
  theorems) from the context.
\item [$res_inst_tac$ etc.] do resolution of rules with explicit
  instantiation.  This works the same way as the corresponding ML tactics, see
  \cite[\S3]{isabelle-ref}.
  
  It is very important to note that the instantiations are read and
  type-checked according to the dynamic goal state, rather than the static
  proof context!  In particular, locally fixed variables and term
  abbreviations may not be included in the term specifications.
\item [$subgoal_tac~\phi$] emulates the ML tactic of the same name, see
  \cite[\S3]{isabelle-ref}.  Syntactically, the given proposition is handled
  as the instantiations in $res_inst_tac$ etc.
  
  Note that the proper Isar command $\PRESUMENAME$ achieves a similar effect
  as $subgoal_tac$.
\end{descr}


\subsection{Meta-linguistic features}

\indexisarcmd{oops}
\begin{matharray}{rcl}
  \isarcmd{oops} & : & \isartrans{proof}{theory} \\
\end{matharray}

The $\OOPS$ command discontinues the current proof attempt, while considering
the partial proof text as properly processed.  This is conceptually quite
different from ``faking'' actual proofs via $\SORRY$ (see
\S\ref{sec:proof-steps}): $\OOPS$ does not observe the proof structure at all,
but goes back right to the theory level.  Furthermore, $\OOPS$ does not
produce any result theorem --- there is no claim to be able to complete the
proof anyhow.

A typical application of $\OOPS$ is to explain Isar proofs \emph{within} the
system itself, in conjunction with the document preparation tools of Isabelle
described in \cite{isabelle-sys}.  Thus partial or even wrong proof attempts
can be discussed in a logically sound manner.  Note that the Isabelle {\LaTeX}
macros can be easily adapted to print something like ``$\dots$'' instead of an
``$\OOPS$'' keyword.

\medskip The $\OOPS$ command is undoable, unlike $\isarkeyword{kill}$ (see
\S\ref{sec:history}).  The effect is to get back to the theory \emph{before}
the opening of the proof.


\section{Other commands}

\subsection{Diagnostics}\label{sec:diag}

\indexisarcmd{pr}\indexisarcmd{thm}\indexisarcmd{term}\indexisarcmd{prop}\indexisarcmd{typ}
\indexisarcmd{print-facts}\indexisarcmd{print-binds}
\begin{matharray}{rcl}
  \isarcmd{help}^* & : & \isarkeep{\cdot} \\
  \isarcmd{pr}^* & : & \isarkeep{\cdot} \\
  \isarcmd{thm}^* & : & \isarkeep{theory~|~proof} \\
  \isarcmd{term}^* & : & \isarkeep{theory~|~proof} \\
  \isarcmd{prop}^* & : & \isarkeep{theory~|~proof} \\
  \isarcmd{typ}^* & : & \isarkeep{theory~|~proof} \\
  \isarcmd{print_facts}^* & : & \isarkeep{proof} \\
  \isarcmd{print_binds}^* & : & \isarkeep{proof} \\
\end{matharray}

These commands are not part of the actual Isabelle/Isar syntax, but assist
interactive development.  Also note that $undo$ does not apply here, since the
theory or proof configuration is not changed.

\begin{rail}
  'pr' modes? nat?
  ;
  'thm' modes? thmrefs
  ;
  'term' modes? term
  ;
  'prop' modes? prop
  ;
  'typ' modes? type
  ;

  modes: '(' (name + ) ')'
  ;
\end{rail}

\begin{descr}
\item [$\isarkeyword{help}$] prints a list of available language elements.
  Note that methods and attributes depend on the current theory context.
\item [$\isarkeyword{pr}~n$] prints the current top-level state, i.e.\ the
  theory identifier or proof state.  The latter includes the proof context,
  current facts and goals.  The optional argument $n$ affects the implicit
  limit of goals to be displayed, which is initially 10.  Omitting the limit
  leaves the value unchanged.
\item [$\isarkeyword{thm}~\vec a$] retrieves theorems from the current theory
  or proof context.  Note that any attributes included in the theorem
  specifications are applied to a temporary context derived from the current
  theory or proof; the result is discarded, i.e.\ attributes involved in $\vec
  a$ do not have any permanent effect.
\item [$\isarkeyword{term}~t$, $\isarkeyword{prop}~\phi$] read, type-check and
  print terms or propositions according to the current theory or proof
  context; the inferred type of $t$ is output as well.  Note that these
  commands are also useful in inspecting the current environment of term
  abbreviations.
\item [$\isarkeyword{typ}~\tau$] reads and prints types of the meta-logic
  according to the current theory or proof context.
\item [$\isarkeyword{print_facts}$] prints any named facts of the current
  context, including assumptions and local results.
\item [$\isarkeyword{print_binds}$] prints all term abbreviations present in
  the context.
\end{descr}

The basic diagnostic commands above admit a list of $modes$ to be specified,
which is appended to the current print mode (see also \cite{isabelle-ref}).
Thus the output behavior may be modified according particular print mode
features.

For example, $\isarkeyword{pr}~(latex~xsymbols~symbols)$ would print the
current proof state with mathematical symbols and special characters
represented in {\LaTeX} source, according to the Isabelle style
\cite{isabelle-sys}.  The resulting text can be directly pasted into a
\verb,\begin{isabelle},\dots\verb,\end{isabelle}, environment.  Note that
$\isarkeyword{pr}~(latex)$ is sufficient to achieve the same output, if the
current Isabelle session has the other modes already activated, say due to
some particular user interface configuration such as Proof~General
\cite{proofgeneral,Aspinall:TACAS:2000} with X-Symbol mode \cite{x-symbol}.


\subsection{History commands}\label{sec:history}

\indexisarcmd{undo}\indexisarcmd{redo}\indexisarcmd{kill}
\begin{matharray}{rcl}
  \isarcmd{undo}^{{*}{*}} & : & \isarkeep{\cdot} \\
  \isarcmd{redo}^{{*}{*}} & : & \isarkeep{\cdot} \\
  \isarcmd{kill}^{{*}{*}} & : & \isarkeep{\cdot} \\
\end{matharray}

The Isabelle/Isar top-level maintains a two-stage history, for theory and
proof state transformation.  Basically, any command can be undone using
$\isarkeyword{undo}$, excluding mere diagnostic elements.  Its effect may be
revoked via $\isarkeyword{redo}$, unless the corresponding the
$\isarkeyword{undo}$ step has crossed the beginning of a proof or theory.  The
$\isarkeyword{kill}$ command aborts the current history node altogether,
discontinuing a proof or even the whole theory.  This operation is \emph{not}
undoable.

\begin{warn}
  History commands should never be used with user interfaces such as
  Proof~General \cite{proofgeneral,Aspinall:TACAS:2000}, which takes care of
  stepping forth and back itself.  Interfering by manual $\isarkeyword{undo}$,
  $\isarkeyword{redo}$, or even $\isarkeyword{kill}$ commands would quickly
  result in utter confusion.
\end{warn}

%FIXME remove
% \begin{descr}
% \item [$\isarkeyword{undo}$] revokes the latest state-transforming command.
% \item [$\isarkeyword{redo}$] undos the latest $\isarkeyword{undo}$.
% \item [$\isarkeyword{kill}$] aborts the current history level.
% \end{descr}


\subsection{System operations}

\indexisarcmd{cd}\indexisarcmd{pwd}\indexisarcmd{use-thy}\indexisarcmd{use-thy-only}
\indexisarcmd{update-thy}\indexisarcmd{update-thy-only}
\begin{matharray}{rcl}
  \isarcmd{cd}^* & : & \isarkeep{\cdot} \\
  \isarcmd{pwd}^* & : & \isarkeep{\cdot} \\
  \isarcmd{use_thy}^* & : & \isarkeep{\cdot} \\
  \isarcmd{use_thy_only}^* & : & \isarkeep{\cdot} \\
  \isarcmd{update_thy}^* & : & \isarkeep{\cdot} \\
  \isarcmd{update_thy_only}^* & : & \isarkeep{\cdot} \\
\end{matharray}

\begin{descr}
\item [$\isarkeyword{cd}~name$] changes the current directory of the Isabelle
  process.
\item [$\isarkeyword{pwd}~$] prints the current working directory.
\item [$\isarkeyword{use_thy}$, $\isarkeyword{use_thy_only}$,
  $\isarkeyword{update_thy}$, $\isarkeyword{update_thy_only}$] load some
  theory given as $name$ argument.  These commands are basically the same as
  the corresponding ML functions\footnote{The ML versions also change the
    implicit theory context to that of the theory loaded.}  (see also
  \cite[\S1,\S6]{isabelle-ref}).  Note that both the ML and Isar versions may
  load new- and old-style theories alike.
\end{descr}

These system commands are scarcely used when working with the Proof~General
interface, since loading of theories is done fully transparently.


%%% Local Variables: 
%%% mode: latex
%%% TeX-master: "isar-ref"
%%% End: 

%% $Id$
\chapter{Simplification} \label{simp-chap}
\index{simplification|(}

This chapter describes Isabelle's generic simplification package, which
provides a suite of simplification tactics.  This rewriting package is less
general than its predecessor --- it works only for the equality relation,
not arbitrary preorders --- but it is fast and flexible.  It performs
conditional and unconditional rewriting and uses contextual information
(``local assumptions'').  It provides a few general hooks, which can
provide automatic case splits during rewriting, for example.  The
simplifier is set up for many of Isabelle's logics: {\tt FOL}, {\tt ZF},
{\tt LCF} and {\tt HOL}.


\section{Simplification sets}\index{simplification sets} 
The simplification tactics are controlled by {\bf simpsets}.  These consist
of five components: rewrite rules, congruence rules, the subgoaler, the
solver and the looper.  Normally, the simplifier is set up with sensible
defaults, so that most simplifier calls specify only rewrite rules.
Sophisticated usage of the other components can be highly effective, but
most users should never worry about them.

\subsection{Rewrite rules}\index{rewrite rules}
Rewrite rules are theorems like $Suc(\Var{m}) + \Var{n} = \Var{m} +
Suc(\Var{n})$, $\Var{P}\conj\Var{P} \bimp \Var{P}$, or $\Var{A} \union \Var{B}
\equiv \{x.x\in A \disj x\in B\}$.  {\bf Conditional} rewrites such as
$\Var{m}<\Var{n} \Imp \Var{m}/\Var{n} = 0$ are permitted; the conditions
can be arbitrary terms.  The infix operation \ttindex{addsimps} adds new
rewrite rules, while \ttindex{delsimps} deletes rewrite rules from a
simpset.

Theorems added via \ttindex{addsimps} need not be equalities to start with.
Each simpset contains a (user-definable) function for extracting equalities
from arbitrary theorems.  For example $\neg(x\in \{\})$ could be turned
into $x\in \{\} \equiv False$.  This function can be set with
\ttindex{setmksimps} but only the definer of a logic should need to do
this.  Exceptionally, one may want to install a selective version of
\ttindex{mksimps} in order to filter out looping rewrite rules arising from
local assumptions (see below).

Internally, all rewrite rules are translated into meta-equalities:
theorems with conclusion $lhs \equiv rhs$.  To this end every simpset contains
a function of type \verb$thm -> thm list$ to extract a list
of meta-equalities from a given theorem.

\begin{warn}\index{rewrite rules}
  The left-hand side of a rewrite rule must look like a first-order term:
  after eta-contraction, none of its unknowns should have arguments.  Hence
  ${\Var{i}+(\Var{j}+\Var{k})} = {(\Var{i}+\Var{j})+\Var{k}}$ and $\neg(\forall
  x.\Var{P}(x)) \bimp (\exists x.\neg\Var{P}(x))$ are acceptable, whereas
  $\Var{f}(\Var{x})\in {\tt range}(\Var{f}) = True$ is not.  However, you can
  replace the offending subterms by new variables and conditions: $\Var{y} =
  \Var{f}(\Var{x}) \Imp \Var{y}\in {\tt range}(\Var{f}) = True$ is again
  acceptable.
\end{warn}

\subsection {Congruence rules}\index{congruence rules}
Congruence rules are meta-equalities of the form
\[ \List{\dots} \Imp
   f(\Var{x@1},\ldots,\Var{x@n}) \equiv f(\Var{y@1},\ldots,\Var{y@n}).
\]
They control the simplification of the arguments of certain constants.  For
example, some arguments can be simplified under additional assumptions:
\[ \List{\Var{P@1} \bimp \Var{Q@1};\; \Var{Q@1} \Imp \Var{P@2} \bimp \Var{Q@2}}
   \Imp (\Var{P@1} \imp \Var{P@2}) \equiv (\Var{Q@1} \imp \Var{Q@2})
\]
This rule assumes $Q@1$ and any rewrite rules it implies, while
simplifying~$P@2$.  Such ``local'' assumptions are effective for rewriting
formulae such as $x=0\imp y+x=y$.  The next example makes similar use of
such contextual information in bounded quantifiers:
\begin{eqnarray*}
  &&\List{\Var{A}=\Var{B};\; 
          \Forall x. x\in \Var{B} \Imp \Var{P}(x) = \Var{Q}(x)} \Imp{} \\
 &&\qquad\qquad
    (\forall x\in \Var{A}.\Var{P}(x)) = (\forall x\in \Var{B}.\Var{Q}(x))
\end{eqnarray*}
This congruence rule supplies contextual information for simplifying the
arms of a conditional expressions:
\[ \List{\Var{p}=\Var{q};~ \Var{q} \Imp \Var{a}=\Var{c};~
         \neg\Var{q} \Imp \Var{b}=\Var{d}} \Imp
   if(\Var{p},\Var{a},\Var{b}) \equiv if(\Var{q},\Var{c},\Var{d})
\]

A congruence rule can also suppress simplification of certain arguments.
Here is an alternative congruence rule for conditional expressions:
\[ \Var{p}=\Var{q} \Imp
   if(\Var{p},\Var{a},\Var{b}) \equiv if(\Var{q},\Var{a},\Var{b})
\]
Only the first argument is simplified; the others remain unchanged.
This can make simplification much faster, but may require an extra case split
to prove the goal.  

Congruence rules are added using \ttindexbold{addeqcongs}.  Their conclusion
must be a meta-equality, as in the examples above.  It is more
natural to derive the rules with object-logic equality, for example
\[ \List{\Var{P@1} \bimp \Var{Q@1};\; \Var{Q@1} \Imp \Var{P@2} \bimp \Var{Q@2}}
   \Imp (\Var{P@1} \imp \Var{P@2}) \bimp (\Var{Q@1} \imp \Var{Q@2}),
\]
Each object-logic should define an operator called \ttindex{addcongs} that
expects object-equalities and translates them into meta-equalities.

\subsection{The subgoaler}
The subgoaler is the tactic used to solve subgoals arising out of
conditional rewrite rules or congruence rules.  The default should be
simplification itself.  Occasionally this strategy needs to be changed.  For
example, if the premise of a conditional rule is an instance of its
conclusion, as in $Suc(\Var{m}) < \Var{n} \Imp \Var{m} < \Var{n}$, the
default strategy could loop.

The subgoaler can be set explicitly with \ttindex{setsubgoaler}.  For
example, the subgoaler
\begin{ttbox}
fun subgoal_tac ss = resolve_tac (prems_of_ss ss) ORELSE' 
                     asm_simp_tac ss;
\end{ttbox}
tries to solve the subgoal with one of the premises and calls
simplification only if that fails; here {\tt prems_of_ss} extracts the
current premises from a simpset.

\subsection{The solver}
The solver is a tactic that attempts to solve a subgoal after
simplification.  Typically it just proves trivial subgoals such as {\tt
  True} and $t=t$; it could use sophisticated means such as
\verb$fast_tac$.  The solver is set using \ttindex{setsolver}.

The tactic is presented with the full goal, including the asssumptions.
Hence it can use those assumptions (say by calling {\tt assume_tac}) even
inside {\tt simp_tac}, which otherwise does not use assumptions.  The
solver is also supplied a list of theorems, namely assumptions that hold in
the local context.

\begin{warn}
  Rewriting does not instantiate unknowns.  Trying to rewrite $a\in
  \Var{A}$ with the rule $\Var{x}\in \{\Var{x}\}$ leads nowhere.  The
  solver, however, is an arbitrary tactic and may instantiate unknowns as
  it pleases.  This is the only way the simplifier can handle a conditional
  rewrite rule whose condition contains extra variables.
\end{warn}

\begin{warn}
  If you want to supply your own subgoaler or solver, read on.  The subgoaler
  is also used to solve the premises of congruence rules, which are usually
  of the form $s = \Var{x}$, where $s$ needs to be simplified and $\Var{x}$
  needs to be instantiated with the result. Hence the subgoaler should call
  the simplifier at some point. The simplifier will then call the solver,
  which must therefore be prepared to solve goals of the form $t = \Var{x}$,
  usually by reflexivity. In particular, reflexivity should be tried before
  any of the fancy tactics like {\tt fast_tac}. It may even happen that, due
  to simplification, the subgoal is no longer an equality. For example $False
  \bimp \Var{Q}$ could be rewritten to $\neg\Var{Q}$, in which case the
  solver must also try resolving with the theorem $\neg False$.

  If the simplifier aborts with the message {\tt Failed congruence proof!},
  it is due to the subgoaler or solver that failed to prove a premise of a
  congruence rule.
\end{warn}

\subsection{The looper}
The looper is a tactic that is applied after simplification, in case the
solver failed to solve the simplified goal.  If the looper succeeds, the
simplification process is started all over again.  Each of the subgoals
generated by the looper is attacked in turn, in reverse order.  A
typical looper is case splitting: the expansion of a conditional.  Another
possibility is to apply an elimination rule on the assumptions.  More
adventurous loopers could start an induction.  The looper is set with 
\ttindex{setloop}.


\begin{figure}
\indexbold{*SIMPLIFIER}
\indexbold{*simpset}
\indexbold{*simp_tac}
\indexbold{*asm_simp_tac}
\indexbold{*asm_full_simp_tac}
\indexbold{*addeqcongs}
\indexbold{*addsimps}
\indexbold{*delsimps}
\indexbold{*empty_ss}
\indexbold{*merge_ss}
\indexbold{*setsubgoaler}
\indexbold{*setsolver}
\indexbold{*setloop}
\indexbold{*setmksimps}
\indexbold{*prems_of_ss}
\indexbold{*rep_ss}
\begin{ttbox}
infix addsimps addeqcongs delsimps
      setsubgoaler setsolver setloop setmksimps;

signature SIMPLIFIER =
sig
  type simpset
  val simp_tac:          simpset -> int -> tactic
  val asm_simp_tac:      simpset -> int -> tactic
  val asm_full_simp_tac: simpset -> int -> tactic\smallskip
  val addeqcongs:   simpset * thm list -> simpset
  val addsimps:     simpset * thm list -> simpset
  val delsimps:     simpset * thm list -> simpset
  val empty_ss:     simpset
  val merge_ss:     simpset * simpset -> simpset
  val setsubgoaler: simpset * (simpset -> int -> tactic) -> simpset
  val setsolver:    simpset * (thm list -> int -> tactic) -> simpset
  val setloop:      simpset * (int -> tactic) -> simpset
  val setmksimps:   simpset * (thm -> thm list) -> simpset
  val prems_of_ss:  simpset -> thm list
  val rep_ss:       simpset -> \{simps: thm list, congs: thm list\}
end;
\end{ttbox}
\caption{The signature \ttindex{SIMPLIFIER}} \label{SIMPLIFIER}
\end{figure}


\section{The simplification tactics} \label{simp-tactics}
\index{simplification!tactics|bold}
\index{tactics!simplification|bold}

The actual simplification work is performed by the following tactics.  The
rewriting strategy is strictly bottom up, except for congruence rules, which
are applied while descending into a term.  Conditions in conditional rewrite
rules are solved recursively before the rewrite rule is applied.

There are three basic simplification tactics:
\begin{description}
\item[\ttindexbold{simp_tac} $ss$ $i$] simplifies subgoal~$i$ using the rules
  in~$ss$.  It may solve the subgoal completely if it has become trivial,
  using the solver.
  
\item[\ttindexbold{asm_simp_tac}] is like \verb$simp_tac$, but also uses
  assumptions as additional rewrite rules.

\item[\ttindexbold{asm_full_simp_tac}] is like \verb$asm_simp_tac$, but also
  simplifies the assumptions one by one, using each assumption in the
  simplification of the following ones.
\end{description}
Using the simplifier effectively may take a bit of experimentation.  The
tactics can be traced with the ML command \verb$trace_simp := true$.  To
remind yourself of what is in a simpset, use the function \verb$rep_ss$ to
return its simplification and congruence rules.

\section{Examples using the simplifier}
\index{simplification!example}
Assume we are working within {\tt FOL} and that
\begin{description}
\item[\tt Nat.thy] is a theory including the constants $0$, $Suc$ and $+$,
\item[\tt add_0] is the rewrite rule $0+n = n$,
\item[\tt add_Suc] is the rewrite rule $Suc(m)+n = Suc(m+n)$,
\item[\tt induct] is the induction rule
$\List{P(0); \Forall x. P(x)\Imp P(Suc(x))} \Imp P(n)$.
\item[\tt FOL_ss] is a basic simpset for {\tt FOL}.\footnote
{These examples reside on the file {\tt FOL/ex/nat.ML}.} 
\end{description}

We create a simpset for natural numbers by extending~{\tt FOL_ss}:
\begin{ttbox}
val add_ss = FOL_ss addsimps [add_0, add_Suc];
\end{ttbox}
Proofs by induction typically involve simplification.  Here is a proof
that~0 is a right identity:
\begin{ttbox}
goal Nat.thy "m+0 = m";
{\out Level 0}
{\out m + 0 = m}
{\out  1. m + 0 = m}
\end{ttbox}
The first step is to perform induction on the variable~$m$.  This returns a
base case and inductive step as two subgoals:
\begin{ttbox}
by (res_inst_tac [("n","m")] induct 1);
{\out Level 1}
{\out m + 0 = m}
{\out  1. 0 + 0 = 0}
{\out  2. !!x. x + 0 = x ==> Suc(x) + 0 = Suc(x)}
\end{ttbox}
Simplification solves the first subgoal trivially:
\begin{ttbox}
by (simp_tac add_ss 1);
{\out Level 2}
{\out m + 0 = m}
{\out  1. !!x. x + 0 = x ==> Suc(x) + 0 = Suc(x)}
\end{ttbox}
The remaining subgoal requires \ttindex{asm_simp_tac} in order to use the
induction hypothesis as a rewrite rule:
\begin{ttbox}
by (asm_simp_tac add_ss 1);
{\out Level 3}
{\out m + 0 = m}
{\out No subgoals!}
\end{ttbox}

The next proof is similar.
\begin{ttbox}
goal Nat.thy "m+Suc(n) = Suc(m+n)";
{\out Level 0}
{\out m + Suc(n) = Suc(m + n)}
{\out  1. m + Suc(n) = Suc(m + n)}
\end{ttbox}
We again perform induction on~$m$ and get two subgoals:
\begin{ttbox}
by (res_inst_tac [("n","m")] induct 1);
{\out Level 1}
{\out m + Suc(n) = Suc(m + n)}
{\out  1. 0 + Suc(n) = Suc(0 + n)}
{\out  2. !!x. x + Suc(n) = Suc(x + n) ==>}
{\out          Suc(x) + Suc(n) = Suc(Suc(x) + n)}
\end{ttbox}
Simplification solves the first subgoal, this time rewriting two
occurrences of~0:
\begin{ttbox}
by (simp_tac add_ss 1);
{\out Level 2}
{\out m + Suc(n) = Suc(m + n)}
{\out  1. !!x. x + Suc(n) = Suc(x + n) ==>}
{\out          Suc(x) + Suc(n) = Suc(Suc(x) + n)}
\end{ttbox}
Switching tracing on illustrates how the simplifier solves the remaining
subgoal: 
\begin{ttbox}
trace_simp := true;
by (asm_simp_tac add_ss 1);
{\out Rewriting:}
{\out Suc(x) + Suc(n) == Suc(x + Suc(n))}
{\out Rewriting:}
{\out x + Suc(n) == Suc(x + n)}
{\out Rewriting:}
{\out Suc(x) + n == Suc(x + n)}
{\out Rewriting:}
{\out Suc(Suc(x + n)) = Suc(Suc(x + n)) == True}
{\out Level 3}
{\out m + Suc(n) = Suc(m + n)}
{\out No subgoals!}
\end{ttbox}
Many variations are possible.  At Level~1 (in either example) we could have
solved both subgoals at once using the tactical \ttindex{ALLGOALS}:
\begin{ttbox}
by (ALLGOALS (asm_simp_tac add_ss));
{\out Level 2}
{\out m + Suc(n) = Suc(m + n)}
{\out No subgoals!}
\end{ttbox}

\medskip
Here is a conjecture to be proved for an arbitrary function~$f$ satisfying
the law $f(Suc(n)) = Suc(f(n))$:
\begin{ttbox}
val [prem] = goal Nat.thy
    "(!!n. f(Suc(n)) = Suc(f(n))) ==> f(i+j) = i+f(j)";
{\out Level 0}
{\out f(i + j) = i + f(j)}
{\out  1. f(i + j) = i + f(j)}
{\out val prem = "f(Suc(?n)) = Suc(f(?n))}
{\out             [!!n. f(Suc(n)) = Suc(f(n))]" : thm}
\ttbreak
by (res_inst_tac [("n","i")] induct 1);
{\out Level 1}
{\out f(i + j) = i + f(j)}
{\out  1. f(0 + j) = 0 + f(j)}
{\out  2. !!x. f(x + j) = x + f(j) ==> f(Suc(x) + j) = Suc(x) + f(j)}
\end{ttbox}
We simplify each subgoal in turn.  The first one is trivial:
\begin{ttbox}
by (simp_tac add_ss 1);
{\out Level 2}
{\out f(i + j) = i + f(j)}
{\out  1. !!x. f(x + j) = x + f(j) ==> f(Suc(x) + j) = Suc(x) + f(j)}
\end{ttbox}
The remaining subgoal requires rewriting by the premise, so we add it to
{\tt add_ss}:\footnote{The previous
  simplifier required congruence rules for function variables like~$f$ in
  order to simplify their arguments.  The present simplifier can be given
  congruence rules to realize non-standard simplification of a function's
  arguments, but this is seldom necessary.}
\begin{ttbox}
by (asm_simp_tac (add_ss addsimps [prem]) 1);
{\out Level 3}
{\out f(i + j) = i + f(j)}
{\out No subgoals!}
\end{ttbox}


\section{Setting up the simplifier}
\index{simplification!setting up|bold}

Setting up the simplifier for new logics is complicated.  This section
describes how the simplifier is installed for first-order logic; the code
is largely taken from {\tt FOL/simpdata.ML}.

The simplifier and the case splitting tactic, which resides in a separate
file, are not part of Pure Isabelle.  They must be loaded explicitly:
\begin{ttbox}
use "../Provers/simplifier.ML";
use "../Provers/splitter.ML";
\end{ttbox}

Simplification works by reducing various object-equalities to
meta-equality.  It requires axioms stating that equal terms and equivalent
formulae are also equal at the meta-level.  The file {\tt FOL/ifol.thy}
contains the two lines
\begin{ttbox}\indexbold{*eq_reflection}\indexbold{*iff_reflection}
eq_reflection   "(x=y)   ==> (x==y)"
iff_reflection  "(P<->Q) ==> (P==Q)"
\end{ttbox}
Of course, you should only assert such axioms if they are true for your
particular logic.  In Constructive Type Theory, equality is a ternary
relation of the form $a=b\in A$; the type~$A$ determines the meaning of the
equality effectively as a partial equivalence relation.


\subsection{A collection of standard rewrite rules}
The file begins by proving lots of standard rewrite rules about the logical
connectives.  These include cancellation laws and associative laws but
certainly not commutative laws, which would case looping.  To prove the
laws easily, it defines a function that echoes the desired law and then
supplies it the theorem prover for intuitionistic \FOL:
\begin{ttbox}
fun int_prove_fun s = 
 (writeln s;  
  prove_goal IFOL.thy s
   (fn prems => [ (cut_facts_tac prems 1), 
                  (Int.fast_tac 1) ]));
\end{ttbox}
The following rewrite rules about conjunction are a selection of those
proved on {\tt FOL/simpdata.ML}.  Later, these will be supplied to the
standard simpset.
\begin{ttbox}
val conj_rews = map int_prove_fun
 ["P & True <-> P",      "True & P <-> P",
  "P & False <-> False", "False & P <-> False",
  "P & P <-> P",
  "P & ~P <-> False",    "~P & P <-> False",
  "(P & Q) & R <-> P & (Q & R)"];
\end{ttbox}
The file also proves some distributive laws.  As they can cause exponential
blowup, they will not be included in the standard simpset.  Instead they
are merely bound to an \ML{} identifier.
\begin{ttbox}
val distrib_rews  = map int_prove_fun
 ["P & (Q | R) <-> P&Q | P&R", 
  "(Q | R) & P <-> Q&P | R&P",
  "(P | Q --> R) <-> (P --> R) & (Q --> R)"];
\end{ttbox}


\subsection{Functions for preprocessing the rewrite rules}
The next step is to define the function for preprocessing rewrite rules.
This will be installed by calling {\tt setmksimps} below.  Preprocessing
occurs whenever rewrite rules are added, whether by user command or
automatically.  Preprocessing involves extracting atomic rewrites at the
object-level, then reflecting them to the meta-level.

To start, the function {\tt gen_all} strips any meta-level
quantifiers from the front of the given theorem.  Usually there are none
anyway.
\begin{ttbox}
fun gen_all th = forall_elim_vars (#maxidx(rep_thm th)+1) th;
\end{ttbox}
The function {\tt atomize} analyses a theorem in order to extract
atomic rewrite rules.  The head of all the patterns, matched by the
wildcard~{\tt _}, is the coercion function {\tt Trueprop}.
\begin{ttbox}
fun atomize th = case concl_of th of 
    _ $ (Const("op &",_) $ _ $ _)   => atomize(th RS conjunct1) \at
                                       atomize(th RS conjunct2)
  | _ $ (Const("op -->",_) $ _ $ _) => atomize(th RS mp)
  | _ $ (Const("All",_) $ _)        => atomize(th RS spec)
  | _ $ (Const("True",_))           => []
  | _ $ (Const("False",_))          => []
  | _                               => [th];
\end{ttbox}
There are several cases, depending upon the form of the conclusion:
\begin{itemize}
\item Conjunction: extract rewrites from both conjuncts.

\item Implication: convert $P\imp Q$ to the meta-implication $P\Imp Q$ and
  extract rewrites from~$Q$; these will be conditional rewrites with the
  condition~$P$.

\item Universal quantification: remove the quantifier, replacing the bound
  variable by a schematic variable, and extract rewrites from the body.

\item {\tt True} and {\tt False} contain no useful rewrites.

\item Anything else: return the theorem in a singleton list.
\end{itemize}
The resulting theorems are not literally atomic --- they could be
disjunctive, for example --- but are brokwn down as much as possible.  See
the file {\tt ZF/simpdata.ML} for a sophisticated translation of
set-theoretic formulae into rewrite rules.

The simplified rewrites must now be converted into meta-equalities.  The
axiom {\tt eq_reflection} converts equality rewrites, while {\tt
  iff_reflection} converts if-and-only-if rewrites.  The latter possibility
can arise in two other ways: the negative theorem~$\neg P$ is converted to
$P\equiv{\tt False}$, and any other theorem~$\neg P$ is converted to
$P\equiv{\tt True}$.  The rules {\tt iff_reflection_F} and {\tt
  iff_reflection_T} accomplish this conversion.
\begin{ttbox}
val P_iff_F = int_prove_fun "~P ==> (P <-> False)";
val iff_reflection_F = P_iff_F RS iff_reflection;
\ttbreak
val P_iff_T = int_prove_fun "P ==> (P <-> True)";
val iff_reflection_T = P_iff_T RS iff_reflection;
\end{ttbox}
The function {\tt mk_meta_eq} converts a theorem to a meta-equality
using the case analysis described above.
\begin{ttbox}
fun mk_meta_eq th = case concl_of th of
    _ $ (Const("op =",_)$_$_)   => th RS eq_reflection
  | _ $ (Const("op <->",_)$_$_) => th RS iff_reflection
  | _ $ (Const("Not",_)$_)      => th RS iff_reflection_F
  | _                           => th RS iff_reflection_T;
\end{ttbox}
The three functions {\tt gen_all}, {\tt atomize} and {\tt mk_meta_eq} will
be composed together and supplied below to {\tt setmksimps}.


\subsection{Making the initial simpset}
It is time to assemble these items.  We open module {\tt Simplifier} to
gain access to its components.  The infix operator \ttindexbold{addcongs}
handles congruence rules; given a list of theorems, it converts their
conclusions into meta-equalities and passes them to \ttindex{addeqcongs}.
\begin{ttbox}
open Simplifier;
\ttbreak
infix addcongs;
fun ss addcongs congs =
    ss addeqcongs (congs RL [eq_reflection,iff_reflection]);
\end{ttbox}
The list {\tt IFOL_rews} contains the default rewrite rules for first-order
logic.  The first of these is the reflexive law expressed as the
equivalence $(a=a)\bimp{\tt True}$; if we provided it as $a=a$ it would
cause looping.
\begin{ttbox}
val IFOL_rews =
   [refl RS P_iff_T] \at conj_rews \at disj_rews \at not_rews \at 
    imp_rews \at iff_rews \at quant_rews;
\end{ttbox}
The list {\tt triv_rls} contains trivial theorems for the solver.  Any
subgoal that is simplified to one of these will be removed.
\begin{ttbox}
val notFalseI = int_prove_fun "~False";
val triv_rls = [TrueI,refl,iff_refl,notFalseI];
\end{ttbox}

The basic simpset for intuitionistic \FOL{} starts with \ttindex{empty_ss}.
It preprocess rewrites using {\tt gen_all}, {\tt atomize} and {\tt
  mk_meta_eq}.  It solves simplified subgoals using {\tt triv_rls} and
assumptions.  It uses \ttindex{asm_simp_tac} to tackle subgoals of
conditional rewrites.  It takes {\tt IFOL_rews} as rewrite rules.  
Other simpsets built from {\tt IFOL_ss} will inherit these items.
\index{*setmksimps}\index{*setsolver}\index{*setsubgoaler}
\index{*addsimps}\index{*addcongs}
\begin{ttbox}
val IFOL_ss = 
  empty_ss 
  setmksimps (map mk_meta_eq o atomize o gen_all)
  setsolver  (fn prems => resolve_tac (triv_rls \at prems) ORELSE' 
                          assume_tac)
  setsubgoaler asm_simp_tac
  addsimps IFOL_rews
  addcongs [imp_cong];
\end{ttbox}
This simpset takes {\tt imp_cong} as a congruence rule in order to use
contextual information to simplify the conclusions of implications:
\[ \List{\Var{P}\bimp\Var{P'};\; \Var{P'} \Imp \Var{Q}\bimp\Var{Q'}} \Imp
   (\Var{P}\imp\Var{Q}) \bimp (\Var{P'}\imp\Var{Q'})
\]
By adding the congruence rule {\tt conj_cong}, we could obtain a similar
effect for conjunctions.


\subsection{Case splitting}
To set up case splitting, we must prove a theorem of the form shown below
and pass it to \ttindexbold{mk_case_split_tac}.  The tactic
\ttindexbold{split_tac} uses {\tt mk_meta_eq} to convert the splitting
rules to meta-equalities.

\begin{ttbox}
val meta_iffD = 
    prove_goal FOL.thy "[| P==Q; Q |] ==> P"
        (fn [prem1,prem2] => [rewtac prem1, rtac prem2 1])
\ttbreak
fun split_tac splits =
    mk_case_split_tac meta_iffD (map mk_meta_eq splits);
\end{ttbox}
%
The splitter is designed for rules roughly of the form
\[ \Var{P}(if(\Var{Q},\Var{x},\Var{y})) \bimp (\Var{Q} \imp \Var{P}(\Var{x}))
\conj (\lnot\Var{Q} \imp \Var{P}(\Var{y})) 
\] 
where the right-hand side can be anything.  Another example is the
elimination operator (which happens to be called~$split$) for Cartesian
products:
\[ \Var{P}(split(\Var{f},\Var{p})) \bimp (\forall a~b. \Var{p} =
\langle a,b\rangle \imp \Var{P}(\Var{f}(a,b))) 
\] 
Case splits should be allowed only when necessary; they are expensive
and hard to control.  Here is a typical example of use, where {\tt
  expand_if} is the first rule above:
\begin{ttbox}
by (simp_tac (prop_rec_ss setloop (split_tac [expand_if])) 1);
\end{ttbox}



\index{simplification|)}



%% $Id$
\chapter{The Classical Reasoner}\label{chap:classical}
\index{classical reasoner|(}
\newcommand\ainfer[2]{\begin{array}{r@{\,}l}#2\\ \hline#1\end{array}}

Although Isabelle is generic, many users will be working in some extension of
classical first-order logic.  Isabelle's set theory~ZF is built upon
theory~FOL, while HOL conceptually contains first-order logic as a fragment.
Theorem-proving in predicate logic is undecidable, but many researchers have
developed strategies to assist in this task.

Isabelle's classical reasoner is an \ML{} functor that accepts certain
information about a logic and delivers a suite of automatic tactics.  Each
tactic takes a collection of rules and executes a simple, non-clausal proof
procedure.  They are slow and simplistic compared with resolution theorem
provers, but they can save considerable time and effort.  They can prove
theorems such as Pelletier's~\cite{pelletier86} problems~40 and~41 in
seconds:
\[ (\exists y. \forall x. J(y,x) \bimp \neg J(x,x))  
   \imp  \neg (\forall x. \exists y. \forall z. J(z,y) \bimp \neg J(z,x)) \]
\[ (\forall z. \exists y. \forall x. F(x,y) \bimp F(x,z) \conj \neg F(x,x))
   \imp \neg (\exists z. \forall x. F(x,z))  
\]
%
The tactics are generic.  They are not restricted to first-order logic, and
have been heavily used in the development of Isabelle's set theory.  Few
interactive proof assistants provide this much automation.  The tactics can
be traced, and their components can be called directly; in this manner,
any proof can be viewed interactively.

The simplest way to apply the classical reasoner (to subgoal~$i$) is to type
\begin{ttbox}
by (Blast_tac \(i\));
\end{ttbox}
This command quickly proves most simple formulas of the predicate calculus or
set theory.  To attempt to prove subgoals using a combination of
rewriting and classical reasoning, try
\begin{ttbox}
auto();                         \emph{\textrm{applies to all subgoals}}
force i;                        \emph{\textrm{applies to one subgoal}}
\end{ttbox}
To do all obvious logical steps, even if they do not prove the
subgoal, type one of the following:
\begin{ttbox}
by Safe_tac;                   \emph{\textrm{applies to all subgoals}}
by (Clarify_tac \(i\));            \emph{\textrm{applies to one subgoal}}
\end{ttbox}


You need to know how the classical reasoner works in order to use it
effectively.  There are many tactics to choose from, including 
{\tt Fast_tac} and \texttt{Best_tac}.

We shall first discuss the underlying principles, then present the classical
reasoner.  Finally, we shall see how to instantiate it for new logics.  The
logics FOL, ZF, HOL and HOLCF have it already installed.


\section{The sequent calculus}
\index{sequent calculus}
Isabelle supports natural deduction, which is easy to use for interactive
proof.  But natural deduction does not easily lend itself to automation,
and has a bias towards intuitionism.  For certain proofs in classical
logic, it can not be called natural.  The {\bf sequent calculus}, a
generalization of natural deduction, is easier to automate.

A {\bf sequent} has the form $\Gamma\turn\Delta$, where $\Gamma$
and~$\Delta$ are sets of formulae.%
\footnote{For first-order logic, sequents can equivalently be made from
  lists or multisets of formulae.} The sequent
\[ P@1,\ldots,P@m\turn Q@1,\ldots,Q@n \]
is {\bf valid} if $P@1\conj\ldots\conj P@m$ implies $Q@1\disj\ldots\disj
Q@n$.  Thus $P@1,\ldots,P@m$ represent assumptions, each of which is true,
while $Q@1,\ldots,Q@n$ represent alternative goals.  A sequent is {\bf
basic} if its left and right sides have a common formula, as in $P,Q\turn
Q,R$; basic sequents are trivially valid.

Sequent rules are classified as {\bf right} or {\bf left}, indicating which
side of the $\turn$~symbol they operate on.  Rules that operate on the
right side are analogous to natural deduction's introduction rules, and
left rules are analogous to elimination rules.  
Recall the natural deduction rules for
  first-order logic, 
\iflabelundefined{fol-fig}{from {\it Introduction to Isabelle}}%
                          {Fig.\ts\ref{fol-fig}}.
The sequent calculus analogue of~$({\imp}I)$ is the rule
$$
\ainfer{\Gamma &\turn \Delta, P\imp Q}{P,\Gamma &\turn \Delta,Q}
\eqno({\imp}R)
$$
This breaks down some implication on the right side of a sequent; $\Gamma$
and $\Delta$ stand for the sets of formulae that are unaffected by the
inference.  The analogue of the pair~$({\disj}I1)$ and~$({\disj}I2)$ is the
single rule 
$$
\ainfer{\Gamma &\turn \Delta, P\disj Q}{\Gamma &\turn \Delta,P,Q}
\eqno({\disj}R)
$$
This breaks down some disjunction on the right side, replacing it by both
disjuncts.  Thus, the sequent calculus is a kind of multiple-conclusion logic.

To illustrate the use of multiple formulae on the right, let us prove
the classical theorem $(P\imp Q)\disj(Q\imp P)$.  Working backwards, we
reduce this formula to a basic sequent:
\[ \infer[(\disj)R]{\turn(P\imp Q)\disj(Q\imp P)}
   {\infer[(\imp)R]{\turn(P\imp Q), (Q\imp P)\;}
    {\infer[(\imp)R]{P \turn Q, (Q\imp P)\qquad}
                    {P, Q \turn Q, P\qquad\qquad}}}
\]
This example is typical of the sequent calculus: start with the desired
theorem and apply rules backwards in a fairly arbitrary manner.  This yields a
surprisingly effective proof procedure.  Quantifiers add few complications,
since Isabelle handles parameters and schematic variables.  See Chapter~10
of {\em ML for the Working Programmer}~\cite{paulson-ml2} for further
discussion.


\section{Simulating sequents by natural deduction}
Isabelle can represent sequents directly, as in the object-logic~\texttt{LK}\@.
But natural deduction is easier to work with, and most object-logics employ
it.  Fortunately, we can simulate the sequent $P@1,\ldots,P@m\turn
Q@1,\ldots,Q@n$ by the Isabelle formula
\[ \List{P@1;\ldots;P@m; \neg Q@2;\ldots; \neg Q@n}\Imp Q@1, \]
where the order of the assumptions and the choice of~$Q@1$ are arbitrary.
Elim-resolution plays a key role in simulating sequent proofs.

We can easily handle reasoning on the left.
As discussed in
\iflabelundefined{destruct}{{\it Introduction to Isabelle}}{{\S}\ref{destruct}}, 
elim-resolution with the rules $(\disj E)$, $(\bot E)$ and $(\exists E)$
achieves a similar effect as the corresponding sequent rules.  For the
other connectives, we use sequent-style elimination rules instead of
destruction rules such as $({\conj}E1,2)$ and $(\forall E)$.  But note that
the rule $(\neg L)$ has no effect under our representation of sequents!
$$
\ainfer{\neg P,\Gamma &\turn \Delta}{\Gamma &\turn \Delta,P}\eqno({\neg}L)
$$
What about reasoning on the right?  Introduction rules can only affect the
formula in the conclusion, namely~$Q@1$.  The other right-side formulae are
represented as negated assumptions, $\neg Q@2$, \ldots,~$\neg Q@n$.  
\index{assumptions!negated}
In order to operate on one of these, it must first be exchanged with~$Q@1$.
Elim-resolution with the {\bf swap} rule has this effect:
$$ \List{\neg P; \; \neg R\Imp P} \Imp R   \eqno(swap)  $$
To ensure that swaps occur only when necessary, each introduction rule is
converted into a swapped form: it is resolved with the second premise
of~$(swap)$.  The swapped form of~$({\conj}I)$, which might be
called~$({\neg\conj}E)$, is
\[ \List{\neg(P\conj Q); \; \neg R\Imp P; \; \neg R\Imp Q} \Imp R. \]
Similarly, the swapped form of~$({\imp}I)$ is
\[ \List{\neg(P\imp Q); \; \List{\neg R;P}\Imp Q} \Imp R  \]
Swapped introduction rules are applied using elim-resolution, which deletes
the negated formula.  Our representation of sequents also requires the use
of ordinary introduction rules.  If we had no regard for readability, we
could treat the right side more uniformly by representing sequents as
\[ \List{P@1;\ldots;P@m; \neg Q@1;\ldots; \neg Q@n}\Imp \bot. \]


\section{Extra rules for the sequent calculus}
As mentioned, destruction rules such as $({\conj}E1,2)$ and $(\forall E)$
must be replaced by sequent-style elimination rules.  In addition, we need
rules to embody the classical equivalence between $P\imp Q$ and $\neg P\disj
Q$.  The introduction rules~$({\disj}I1,2)$ are replaced by a rule that
simulates $({\disj}R)$:
\[ (\neg Q\Imp P) \Imp P\disj Q \]
The destruction rule $({\imp}E)$ is replaced by
\[ \List{P\imp Q;\; \neg P\Imp R;\; Q\Imp R} \Imp R. \]
Quantifier replication also requires special rules.  In classical logic,
$\exists x{.}P$ is equivalent to $\neg\forall x{.}\neg P$; the rules
$(\exists R)$ and $(\forall L)$ are dual:
\[ \ainfer{\Gamma &\turn \Delta, \exists x{.}P}
          {\Gamma &\turn \Delta, \exists x{.}P, P[t/x]} \; (\exists R)
   \qquad
   \ainfer{\forall x{.}P, \Gamma &\turn \Delta}
          {P[t/x], \forall x{.}P, \Gamma &\turn \Delta} \; (\forall L)
\]
Thus both kinds of quantifier may be replicated.  Theorems requiring
multiple uses of a universal formula are easy to invent; consider 
\[ (\forall x.P(x)\imp P(f(x))) \conj P(a) \imp P(f^n(a)), \]
for any~$n>1$.  Natural examples of the multiple use of an existential
formula are rare; a standard one is $\exists x.\forall y. P(x)\imp P(y)$.

Forgoing quantifier replication loses completeness, but gains decidability,
since the search space becomes finite.  Many useful theorems can be proved
without replication, and the search generally delivers its verdict in a
reasonable time.  To adopt this approach, represent the sequent rules
$(\exists R)$, $(\exists L)$ and $(\forall R)$ by $(\exists I)$, $(\exists
E)$ and $(\forall I)$, respectively, and put $(\forall E)$ into elimination
form:
$$ \List{\forall x{.}P(x); P(t)\Imp Q} \Imp Q    \eqno(\forall E@2) $$
Elim-resolution with this rule will delete the universal formula after a
single use.  To replicate universal quantifiers, replace the rule by
$$
\List{\forall x{.}P(x);\; \List{P(t); \forall x{.}P(x)}\Imp Q} \Imp Q.
\eqno(\forall E@3)
$$
To replicate existential quantifiers, replace $(\exists I)$ by
\[ \List{\neg(\exists x{.}P(x)) \Imp P(t)} \Imp \exists x{.}P(x). \]
All introduction rules mentioned above are also useful in swapped form.

Replication makes the search space infinite; we must apply the rules with
care.  The classical reasoner distinguishes between safe and unsafe
rules, applying the latter only when there is no alternative.  Depth-first
search may well go down a blind alley; best-first search is better behaved
in an infinite search space.  However, quantifier replication is too
expensive to prove any but the simplest theorems.


\section{Classical rule sets}
\index{classical sets}
Each automatic tactic takes a {\bf classical set} --- a collection of
rules, classified as introduction or elimination and as {\bf safe} or {\bf
unsafe}.  In general, safe rules can be attempted blindly, while unsafe
rules must be used with care.  A safe rule must never reduce a provable
goal to an unprovable set of subgoals.  

The rule~$({\disj}I1)$ is unsafe because it reduces $P\disj Q$ to~$P$.  Any
rule is unsafe whose premises contain new unknowns.  The elimination
rule~$(\forall E@2)$ is unsafe, since it is applied via elim-resolution,
which discards the assumption $\forall x{.}P(x)$ and replaces it by the
weaker assumption~$P(\Var{t})$.  The rule $({\exists}I)$ is unsafe for
similar reasons.  The rule~$(\forall E@3)$ is unsafe in a different sense:
since it keeps the assumption $\forall x{.}P(x)$, it is prone to looping.
In classical first-order logic, all rules are safe except those mentioned
above.

The safe/unsafe distinction is vague, and may be regarded merely as a way
of giving some rules priority over others.  One could argue that
$({\disj}E)$ is unsafe, because repeated application of it could generate
exponentially many subgoals.  Induction rules are unsafe because inductive
proofs are difficult to set up automatically.  Any inference is unsafe that
instantiates an unknown in the proof state --- thus \ttindex{match_tac}
must be used, rather than \ttindex{resolve_tac}.  Even proof by assumption
is unsafe if it instantiates unknowns shared with other subgoals --- thus
\ttindex{eq_assume_tac} must be used, rather than \ttindex{assume_tac}.

\subsection{Adding rules to classical sets}
Classical rule sets belong to the abstract type \mltydx{claset}, which
supports the following operations (provided the classical reasoner is
installed!):
\begin{ttbox} 
empty_cs : claset
print_cs : claset -> unit
rep_cs : claset -> \{safeEs: thm list, safeIs: thm list,
                    hazEs: thm list,  hazIs: thm list, 
                    swrappers: (string * wrapper) list, 
                    uwrappers: (string * wrapper) list,
                    safe0_netpair: netpair, safep_netpair: netpair,
                    haz_netpair: netpair, dup_netpair: netpair\}
addSIs   : claset * thm list -> claset                 \hfill{\bf infix 4}
addSEs   : claset * thm list -> claset                 \hfill{\bf infix 4}
addSDs   : claset * thm list -> claset                 \hfill{\bf infix 4}
addIs    : claset * thm list -> claset                 \hfill{\bf infix 4}
addEs    : claset * thm list -> claset                 \hfill{\bf infix 4}
addDs    : claset * thm list -> claset                 \hfill{\bf infix 4}
delrules : claset * thm list -> claset                 \hfill{\bf infix 4}
\end{ttbox}
The add operations ignore any rule already present in the claset with the same
classification (such as safe introduction).  They print a warning if the rule
has already been added with some other classification, but add the rule
anyway.  Calling \texttt{delrules} deletes all occurrences of a rule from the
claset, but see the warning below concerning destruction rules.
\begin{ttdescription}
\item[\ttindexbold{empty_cs}] is the empty classical set.

\item[\ttindexbold{print_cs} $cs$] displays the printable contents of~$cs$,
  which is the rules. All other parts are non-printable.

\item[\ttindexbold{rep_cs} $cs$] decomposes $cs$ as a record of its internal 
  components, namely the safe introduction and elimination rules, the unsafe
  introduction and elimination rules, the lists of safe and unsafe wrappers
  (see \ref{sec:modifying-search}), and the internalized forms of the rules.

\item[$cs$ addSIs $rules$] \indexbold{*addSIs}
adds safe introduction~$rules$ to~$cs$.

\item[$cs$ addSEs $rules$] \indexbold{*addSEs}
adds safe elimination~$rules$ to~$cs$.

\item[$cs$ addSDs $rules$] \indexbold{*addSDs}
adds safe destruction~$rules$ to~$cs$.

\item[$cs$ addIs $rules$] \indexbold{*addIs}
adds unsafe introduction~$rules$ to~$cs$.

\item[$cs$ addEs $rules$] \indexbold{*addEs}
adds unsafe elimination~$rules$ to~$cs$.

\item[$cs$ addDs $rules$] \indexbold{*addDs}
adds unsafe destruction~$rules$ to~$cs$.

\item[$cs$ delrules $rules$] \indexbold{*delrules}
deletes~$rules$ from~$cs$.  It prints a warning for those rules that are not
in~$cs$.
\end{ttdescription}

\begin{warn}
  If you added $rule$ using \texttt{addSDs} or \texttt{addDs}, then you must delete
  it as follows:
\begin{ttbox}
\(cs\) delrules [make_elim \(rule\)]
\end{ttbox}
\par\noindent
This is necessary because the operators \texttt{addSDs} and \texttt{addDs} convert
the destruction rules to elimination rules by applying \ttindex{make_elim},
and then insert them using \texttt{addSEs} and \texttt{addEs}, respectively.
\end{warn}

Introduction rules are those that can be applied using ordinary resolution.
The classical set automatically generates their swapped forms, which will
be applied using elim-resolution.  Elimination rules are applied using
elim-resolution.  In a classical set, rules are sorted by the number of new
subgoals they will yield; rules that generate the fewest subgoals will be
tried first (see {\S}\ref{biresolve_tac}).

For elimination and destruction rules there are variants of the add operations
adding a rule in a way such that it is applied only if also its second premise
can be unified with an assumption of the current proof state:
\indexbold{*addSE2}\indexbold{*addSD2}\indexbold{*addE2}\indexbold{*addD2}
\begin{ttbox}
addSE2      : claset * (string * thm) -> claset           \hfill{\bf infix 4}
addSD2      : claset * (string * thm) -> claset           \hfill{\bf infix 4}
addE2       : claset * (string * thm) -> claset           \hfill{\bf infix 4}
addD2       : claset * (string * thm) -> claset           \hfill{\bf infix 4}
\end{ttbox}
\begin{warn}
  A rule to be added in this special way must be given a name, which is used 
  to delete it again -- when desired -- using \texttt{delSWrappers} or 
  \texttt{delWrappers}, respectively. This is because these add operations
  are implemented as wrappers (see \ref{sec:modifying-search} below).
\end{warn}


\subsection{Modifying the search step}
\label{sec:modifying-search}
For a given classical set, the proof strategy is simple.  Perform as many safe
inferences as possible; or else, apply certain safe rules, allowing
instantiation of unknowns; or else, apply an unsafe rule.  The tactics also
eliminate assumptions of the form $x=t$ by substitution if they have been set
up to do so (see \texttt{hyp_subst_tacs} in~{\S}\ref{sec:classical-setup} below).
They may perform a form of Modus Ponens: if there are assumptions $P\imp Q$
and~$P$, then replace $P\imp Q$ by~$Q$.

The classical reasoning tactics --- except \texttt{blast_tac}! --- allow
you to modify this basic proof strategy by applying two lists of arbitrary 
{\bf wrapper tacticals} to it. 
The first wrapper list, which is considered to contain safe wrappers only, 
affects \ttindex{safe_step_tac} and all the tactics that call it.  
The second one, which may contain unsafe wrappers, affects the unsafe parts
of \ttindex{step_tac}, \ttindex{slow_step_tac}, and the tactics that call them.
A wrapper transforms each step of the search, for example 
by attempting other tactics before or after the original step tactic. 
All members of a wrapper list are applied in turn to the respective step tactic.

Initially the two wrapper lists are empty, which means no modification of the
step tactics. Safe and unsafe wrappers are added to a claset 
with the functions given below, supplying them with wrapper names. 
These names may be used to selectively delete wrappers.

\begin{ttbox} 
type wrapper = (int -> tactic) -> (int -> tactic);

addSWrapper  : claset * (string *  wrapper       ) -> claset \hfill{\bf infix 4}
addSbefore   : claset * (string * (int -> tactic)) -> claset \hfill{\bf infix 4}
addSafter    : claset * (string * (int -> tactic)) -> claset \hfill{\bf infix 4}
delSWrapper  : claset *  string                    -> claset \hfill{\bf infix 4}

addWrapper   : claset * (string *  wrapper       ) -> claset \hfill{\bf infix 4}
addbefore    : claset * (string * (int -> tactic)) -> claset \hfill{\bf infix 4}
addafter     : claset * (string * (int -> tactic)) -> claset \hfill{\bf infix 4}
delWrapper   : claset *  string                    -> claset \hfill{\bf infix 4}

addSss       : claset * simpset -> claset                 \hfill{\bf infix 4}
addss        : claset * simpset -> claset                 \hfill{\bf infix 4}
\end{ttbox}
%

\begin{ttdescription}
\item[$cs$ addSWrapper $(name,wrapper)$] \indexbold{*addSWrapper}
adds a new wrapper, which should yield a safe tactic, 
to modify the existing safe step tactic.

\item[$cs$ addSbefore $(name,tac)$] \indexbold{*addSbefore}
adds the given tactic as a safe wrapper, such that it is tried 
{\em before} each safe step of the search.

\item[$cs$ addSafter $(name,tac)$] \indexbold{*addSafter}
adds the given tactic as a safe wrapper, such that it is tried 
when a safe step of the search would fail.

\item[$cs$ delSWrapper $name$] \indexbold{*delSWrapper}
deletes the safe wrapper with the given name.

\item[$cs$ addWrapper $(name,wrapper)$] \indexbold{*addWrapper}
adds a new wrapper to modify the existing (unsafe) step tactic.

\item[$cs$ addbefore $(name,tac)$] \indexbold{*addbefore}
adds the given tactic as an unsafe wrapper, such that it its result is 
concatenated {\em before} the result of each unsafe step.

\item[$cs$ addafter $(name,tac)$] \indexbold{*addafter}
adds the given tactic as an unsafe wrapper, such that it its result is 
concatenated {\em after} the result of each unsafe step.

\item[$cs$ delWrapper $name$] \indexbold{*delWrapper}
deletes the unsafe wrapper with the given name.

\item[$cs$ addSss $ss$] \indexbold{*addss}
adds the simpset~$ss$ to the classical set.  The assumptions and goal will be
simplified, in a rather safe way, after each safe step of the search.

\item[$cs$ addss $ss$] \indexbold{*addss}
adds the simpset~$ss$ to the classical set.  The assumptions and goal will be
simplified, before the each unsafe step of the search.

\end{ttdescription}

\index{simplification!from classical reasoner} 
Strictly speaking, the operators \texttt{addss} and \texttt{addSss}
are not part of the classical reasoner.
, which are used as primitives 
for the automatic tactics described in {\S}\ref{sec:automatic-tactics}, are
implemented as wrapper tacticals.
they  
\begin{warn}
Being defined as wrappers, these operators are inappropriate for adding more 
than one simpset at a time: the simpset added last overwrites any earlier ones.
When a simpset combined with a claset is to be augmented, this should done 
{\em before} combining it with the claset.
\end{warn}


\section{The classical tactics}
\index{classical reasoner!tactics} If installed, the classical module provides
powerful theorem-proving tactics.  Most of them have capitalized analogues
that use the default claset; see {\S}\ref{sec:current-claset}.


\subsection{The tableau prover}
The tactic \texttt{blast_tac} searches for a proof using a fast tableau prover,
coded directly in \ML.  It then reconstructs the proof using Isabelle
tactics.  It is faster and more powerful than the other classical
reasoning tactics, but has major limitations too.
\begin{itemize}
\item It does not use the wrapper tacticals described above, such as
  \ttindex{addss}.
\item It ignores types, which can cause problems in HOL.  If it applies a rule
  whose types are inappropriate, then proof reconstruction will fail.
\item It does not perform higher-order unification, as needed by the rule {\tt
    rangeI} in HOL and \texttt{RepFunI} in ZF.  There are often alternatives
  to such rules, for example {\tt range_eqI} and \texttt{RepFun_eqI}.
\item Function variables may only be applied to parameters of the subgoal.
(This restriction arises because the prover does not use higher-order
unification.)  If other function variables are present then the prover will
fail with the message {\small\tt Function Var's argument not a bound variable}.
\item Its proof strategy is more general than \texttt{fast_tac}'s but can be
  slower.  If \texttt{blast_tac} fails or seems to be running forever, try {\tt
  fast_tac} and the other tactics described below.
\end{itemize}
%
\begin{ttbox} 
blast_tac        : claset -> int -> tactic
Blast.depth_tac  : claset -> int -> int -> tactic
Blast.trace      : bool ref \hfill{\bf initially false}
\end{ttbox}
The two tactics differ on how they bound the number of unsafe steps used in a
proof.  While \texttt{blast_tac} starts with a bound of zero and increases it
successively to~20, \texttt{Blast.depth_tac} applies a user-supplied search bound.
\begin{ttdescription}
\item[\ttindexbold{blast_tac} $cs$ $i$] tries to prove
  subgoal~$i$, increasing the search bound using iterative
  deepening~\cite{korf85}. 
  
\item[\ttindexbold{Blast.depth_tac} $cs$ $lim$ $i$] tries
  to prove subgoal~$i$ using a search bound of $lim$.  Sometimes a slow
  proof using \texttt{blast_tac} can be made much faster by supplying the
  successful search bound to this tactic instead.
  
\item[set \ttindexbold{Blast.trace};] \index{tracing!of classical prover}
  causes the tableau prover to print a trace of its search.  At each step it
  displays the formula currently being examined and reports whether the branch
  has been closed, extended or split.
\end{ttdescription}


\subsection{Automatic tactics}\label{sec:automatic-tactics}
\begin{ttbox} 
type clasimpset = claset * simpset;
auto_tac        : clasimpset ->        tactic
force_tac       : clasimpset -> int -> tactic
auto            : unit -> unit
force           : int  -> unit
\end{ttbox}
The automatic tactics attempt to prove goals using a combination of
simplification and classical reasoning. 
\begin{ttdescription}
\item[\ttindexbold{auto_tac $(cs,ss)$}] is intended for situations where 
there are a lot of mostly trivial subgoals; it proves all the easy ones, 
leaving the ones it cannot prove.
(Unfortunately, attempting to prove the hard ones may take a long time.)  
\item[\ttindexbold{force_tac} $(cs,ss)$ $i$] is intended to prove subgoal~$i$ 
completely. It tries to apply all fancy tactics it knows about, 
performing a rather exhaustive search.
\end{ttdescription}
They must be supplied both a simpset and a claset; therefore 
they are most easily called as \texttt{Auto_tac} and \texttt{Force_tac}, which 
use the default claset and simpset (see {\S}\ref{sec:current-claset} below). 
For interactive use, 
the shorthand \texttt{auto();} abbreviates \texttt{by Auto_tac;} 
while \texttt{force 1;} abbreviates \texttt{by (Force_tac 1);}


\subsection{Semi-automatic tactics}
\begin{ttbox} 
clarify_tac      : claset -> int -> tactic
clarify_step_tac : claset -> int -> tactic
clarsimp_tac     : clasimpset -> int -> tactic
\end{ttbox}
Use these when the automatic tactics fail.  They perform all the obvious
logical inferences that do not split the subgoal.  The result is a
simpler subgoal that can be tackled by other means, such as by
instantiating quantifiers yourself.
\begin{ttdescription}
\item[\ttindexbold{clarify_tac} $cs$ $i$] performs a series of safe steps on
subgoal~$i$ by repeatedly calling \texttt{clarify_step_tac}.
\item[\ttindexbold{clarify_step_tac} $cs$ $i$] performs a safe step on
  subgoal~$i$.  No splitting step is applied; for example, the subgoal $A\conj
  B$ is left as a conjunction.  Proof by assumption, Modus Ponens, etc., may be
  performed provided they do not instantiate unknowns.  Assumptions of the
  form $x=t$ may be eliminated.  The user-supplied safe wrapper tactical is
  applied.
\item[\ttindexbold{clarsimp_tac} $cs$ $i$] acts like \texttt{clarify_tac}, but
also does simplification with the given simpset. Note that if the simpset 
includes a splitter for the premises, the subgoal may still be split.
\end{ttdescription}


\subsection{Other classical tactics}
\begin{ttbox} 
fast_tac      : claset -> int -> tactic
best_tac      : claset -> int -> tactic
slow_tac      : claset -> int -> tactic
slow_best_tac : claset -> int -> tactic
\end{ttbox}
These tactics attempt to prove a subgoal using sequent-style reasoning.
Unlike \texttt{blast_tac}, they construct proofs directly in Isabelle.  Their
effect is restricted (by \texttt{SELECT_GOAL}) to one subgoal; they either prove
this subgoal or fail.  The \texttt{slow_} versions conduct a broader
search.%
\footnote{They may, when backtracking from a failed proof attempt, undo even
  the step of proving a subgoal by assumption.}

The best-first tactics are guided by a heuristic function: typically, the
total size of the proof state.  This function is supplied in the functor call
that sets up the classical reasoner.
\begin{ttdescription}
\item[\ttindexbold{fast_tac} $cs$ $i$] applies \texttt{step_tac} using
depth-first search to prove subgoal~$i$.

\item[\ttindexbold{best_tac} $cs$ $i$] applies \texttt{step_tac} using
best-first search to prove subgoal~$i$.

\item[\ttindexbold{slow_tac} $cs$ $i$] applies \texttt{slow_step_tac} using
depth-first search to prove subgoal~$i$.

\item[\ttindexbold{slow_best_tac} $cs$ $i$] applies \texttt{slow_step_tac} with
best-first search to prove subgoal~$i$.
\end{ttdescription}


\subsection{Depth-limited automatic tactics}
\begin{ttbox} 
depth_tac  : claset -> int -> int -> tactic
deepen_tac : claset -> int -> int -> tactic
\end{ttbox}
These work by exhaustive search up to a specified depth.  Unsafe rules are
modified to preserve the formula they act on, so that it be used repeatedly.
They can prove more goals than \texttt{fast_tac} can but are much
slower, for example if the assumptions have many universal quantifiers.

The depth limits the number of unsafe steps.  If you can estimate the minimum
number of unsafe steps needed, supply this value as~$m$ to save time.
\begin{ttdescription}
\item[\ttindexbold{depth_tac} $cs$ $m$ $i$] 
tries to prove subgoal~$i$ by exhaustive search up to depth~$m$.

\item[\ttindexbold{deepen_tac} $cs$ $m$ $i$] 
tries to prove subgoal~$i$ by iterative deepening.  It calls \texttt{depth_tac}
repeatedly with increasing depths, starting with~$m$.
\end{ttdescription}


\subsection{Single-step tactics}
\begin{ttbox} 
safe_step_tac : claset -> int -> tactic
safe_tac      : claset        -> tactic
inst_step_tac : claset -> int -> tactic
step_tac      : claset -> int -> tactic
slow_step_tac : claset -> int -> tactic
\end{ttbox}
The automatic proof procedures call these tactics.  By calling them
yourself, you can execute these procedures one step at a time.
\begin{ttdescription}
\item[\ttindexbold{safe_step_tac} $cs$ $i$] performs a safe step on
  subgoal~$i$.  The safe wrapper tacticals are applied to a tactic that may
  include proof by assumption or Modus Ponens (taking care not to instantiate
  unknowns), or substitution.

\item[\ttindexbold{safe_tac} $cs$] repeatedly performs safe steps on all 
subgoals.  It is deterministic, with at most one outcome.  

\item[\ttindexbold{inst_step_tac} $cs$ $i$] is like \texttt{safe_step_tac},
but allows unknowns to be instantiated.

\item[\ttindexbold{step_tac} $cs$ $i$] is the basic step of the proof
  procedure.  The unsafe wrapper tacticals are applied to a tactic that tries
  \texttt{safe_tac}, \texttt{inst_step_tac}, or applies an unsafe rule
  from~$cs$.

\item[\ttindexbold{slow_step_tac}] 
  resembles \texttt{step_tac}, but allows backtracking between using safe
  rules with instantiation (\texttt{inst_step_tac}) and using unsafe rules.
  The resulting search space is larger.
\end{ttdescription}


\subsection{The current claset}\label{sec:current-claset}

Each theory is equipped with an implicit \emph{current claset}
\index{claset!current}.  This is a default set of classical
rules.  The underlying idea is quite similar to that of a current
simpset described in {\S}\ref{sec:simp-for-dummies}; please read that
section, including its warnings.  

The tactics
\begin{ttbox}
Blast_tac        : int -> tactic
Auto_tac         :        tactic
Force_tac        : int -> tactic
Fast_tac         : int -> tactic
Best_tac         : int -> tactic
Deepen_tac       : int -> int -> tactic
Clarify_tac      : int -> tactic
Clarify_step_tac : int -> tactic
Clarsimp_tac     : int -> tactic
Safe_tac         :        tactic
Safe_step_tac    : int -> tactic
Step_tac         : int -> tactic
\end{ttbox}
\indexbold{*Blast_tac}\indexbold{*Auto_tac}\indexbold{*Force_tac}
\indexbold{*Best_tac}\indexbold{*Fast_tac}%
\indexbold{*Deepen_tac}
\indexbold{*Clarify_tac}\indexbold{*Clarify_step_tac}\indexbold{*Clarsimp_tac}
\indexbold{*Safe_tac}\indexbold{*Safe_step_tac}
\indexbold{*Step_tac}
make use of the current claset.  For example, \texttt{Blast_tac} is defined as 
\begin{ttbox}
fun Blast_tac i st = blast_tac (claset()) i st;
\end{ttbox}
and gets the current claset, only after it is applied to a proof state.  
The functions
\begin{ttbox}
AddSIs, AddSEs, AddSDs, AddIs, AddEs, AddDs: thm list -> unit
\end{ttbox}
\indexbold{*AddSIs} \indexbold{*AddSEs} \indexbold{*AddSDs}
\indexbold{*AddIs} \indexbold{*AddEs} \indexbold{*AddDs}
are used to add rules to the current claset.  They work exactly like their
lower case counterparts, such as \texttt{addSIs}.  Calling
\begin{ttbox}
Delrules : thm list -> unit
\end{ttbox}
deletes rules from the current claset. 


\subsection{Accessing the current claset}
\label{sec:access-current-claset}

the functions to access the current claset are analogous to the functions 
for the current simpset, so please see \ref{sec:access-current-simpset}
for a description.
\begin{ttbox}
claset        : unit   -> claset
claset_ref    : unit   -> claset ref
claset_of     : theory -> claset
claset_ref_of : theory -> claset ref
print_claset  : theory -> unit
CLASET        :(claset     ->       tactic) ->       tactic
CLASET'       :(claset     -> 'a -> tactic) -> 'a -> tactic
CLASIMPSET    :(clasimpset ->       tactic) ->       tactic
CLASIMPSET'   :(clasimpset -> 'a -> tactic) -> 'a -> tactic
\end{ttbox}


\subsection{Other useful tactics}
\index{tactics!for contradiction}
\index{tactics!for Modus Ponens}
\begin{ttbox} 
contr_tac    :             int -> tactic
mp_tac       :             int -> tactic
eq_mp_tac    :             int -> tactic
swap_res_tac : thm list -> int -> tactic
\end{ttbox}
These can be used in the body of a specialized search.
\begin{ttdescription}
\item[\ttindexbold{contr_tac} {\it i}]\index{assumptions!contradictory}
  solves subgoal~$i$ by detecting a contradiction among two assumptions of
  the form $P$ and~$\neg P$, or fail.  It may instantiate unknowns.  The
  tactic can produce multiple outcomes, enumerating all possible
  contradictions.

\item[\ttindexbold{mp_tac} {\it i}] 
is like \texttt{contr_tac}, but also attempts to perform Modus Ponens in
subgoal~$i$.  If there are assumptions $P\imp Q$ and~$P$, then it replaces
$P\imp Q$ by~$Q$.  It may instantiate unknowns.  It fails if it can do
nothing.

\item[\ttindexbold{eq_mp_tac} {\it i}] 
is like \texttt{mp_tac} {\it i}, but may not instantiate unknowns --- thus, it
is safe.

\item[\ttindexbold{swap_res_tac} {\it thms} {\it i}] refines subgoal~$i$ of
the proof state using {\it thms}, which should be a list of introduction
rules.  First, it attempts to prove the goal using \texttt{assume_tac} or
\texttt{contr_tac}.  It then attempts to apply each rule in turn, attempting
resolution and also elim-resolution with the swapped form.
\end{ttdescription}

\subsection{Creating swapped rules}
\begin{ttbox} 
swapify   : thm list -> thm list
joinrules : thm list * thm list -> (bool * thm) list
\end{ttbox}
\begin{ttdescription}
\item[\ttindexbold{swapify} {\it thms}] returns a list consisting of the
swapped versions of~{\it thms}, regarded as introduction rules.

\item[\ttindexbold{joinrules} ({\it intrs}, {\it elims})]
joins introduction rules, their swapped versions, and elimination rules for
use with \ttindex{biresolve_tac}.  Each rule is paired with~\texttt{false}
(indicating ordinary resolution) or~\texttt{true} (indicating
elim-resolution).
\end{ttdescription}


\section{Setting up the classical reasoner}\label{sec:classical-setup}
\index{classical reasoner!setting up}
Isabelle's classical object-logics, including \texttt{FOL} and \texttt{HOL}, 
have the classical reasoner already set up.  
When defining a new classical logic, you should set up the reasoner yourself.  
It consists of the \ML{} functor \ttindex{ClassicalFun}, which takes the 
argument signature \texttt{CLASSICAL_DATA}:
\begin{ttbox} 
signature CLASSICAL_DATA =
  sig
  val mp             : thm
  val not_elim       : thm
  val swap           : thm
  val sizef          : thm -> int
  val hyp_subst_tacs : (int -> tactic) list
  end;
\end{ttbox}
Thus, the functor requires the following items:
\begin{ttdescription}
\item[\tdxbold{mp}] should be the Modus Ponens rule
$\List{\Var{P}\imp\Var{Q};\; \Var{P}} \Imp \Var{Q}$.

\item[\tdxbold{not_elim}] should be the contradiction rule
$\List{\neg\Var{P};\; \Var{P}} \Imp \Var{R}$.

\item[\tdxbold{swap}] should be the swap rule
$\List{\neg \Var{P}; \; \neg \Var{R}\Imp \Var{P}} \Imp \Var{R}$.

\item[\ttindexbold{sizef}] is the heuristic function used for best-first
search.  It should estimate the size of the remaining subgoals.  A good
heuristic function is \ttindex{size_of_thm}, which measures the size of the
proof state.  Another size function might ignore certain subgoals (say,
those concerned with type-checking).  A heuristic function might simply
count the subgoals.

\item[\ttindexbold{hyp_subst_tacs}] is a list of tactics for substitution in
the hypotheses, typically created by \ttindex{HypsubstFun} (see
Chapter~\ref{substitution}).  This list can, of course, be empty.  The
tactics are assumed to be safe!
\end{ttdescription}
The functor is not at all sensitive to the formalization of the
object-logic.  It does not even examine the rules, but merely applies
them according to its fixed strategy.  The functor resides in {\tt
  Provers/classical.ML} in the Isabelle sources.

\index{classical reasoner|)}

\section{Setting up the combination with the simplifier}
\label{sec:clasimp-setup}

To combine the classical reasoner and the simplifier, we simply call the 
\ML{} functor \ttindex{ClasimpFun} that assembles the parts as required. 
It takes a structure (of signature \texttt{CLASIMP_DATA}) as
argment, which can be contructed on the fly:
\begin{ttbox}
structure Clasimp = ClasimpFun
 (structure Simplifier = Simplifier 
        and Classical  = Classical 
        and Blast      = Blast);
\end{ttbox}
%
%%% Local Variables: 
%%% mode: latex
%%% TeX-master: "ref"
%%% End: 


\chapter{Isabelle/HOL specific elements}\label{ch:hol-tools}

\section{Miscellaneous attributes}

\indexisarattof{HOL}{split-format}
\begin{matharray}{rcl}
  split_format^* & : & \isaratt \\
\end{matharray}

\railalias{splitformat}{split\_format}
\railterm{splitformat}
\railterm{complete}

\begin{rail}
  splitformat (((name * ) + 'and') | ('(' complete ')'))
  ;
\end{rail}

\begin{descr}
  
\item [$split_format~\vec p@1 \dots \vec p@n$] puts tuple objects into
  canonical form as specified by the arguments given; $\vec p@i$ refers to
  occurrences in premise $i$ of the rule.  The $split_format~(complete)$ form
  causes \emph{all} arguments in function applications to be represented
  canonically according to their tuple type structure.
  
  Note that these operations tend to invent funny names for new local
  parameters to be introduced.

\end{descr}


\section{Primitive types}\label{sec:typedef}

\indexisarcmdof{HOL}{typedecl}\indexisarcmdof{HOL}{typedef}
\begin{matharray}{rcl}
  \isarcmd{typedecl} & : & \isartrans{theory}{theory} \\
  \isarcmd{typedef} & : & \isartrans{theory}{proof(prove)} \\
\end{matharray}

\begin{rail}
  'typedecl' typespec infix? comment?
  ;
  'typedef' parname? typespec infix? \\ '=' term comment?
  ;
\end{rail}

\begin{descr}
\item [$\isarkeyword{typedecl}~(\vec\alpha)t$] is similar to the original
  $\isarkeyword{typedecl}$ of Isabelle/Pure (see \S\ref{sec:types-pure}), but
  also declares type arity $t :: (term, \dots, term) term$, making $t$ an
  actual HOL type constructor.
\item [$\isarkeyword{typedef}~(\vec\alpha)t = A$] sets up a goal stating
  non-emptiness of the set $A$.  After finishing the proof, the theory will be
  augmented by a Gordon/HOL-style type definition.  See \cite{isabelle-HOL}
  for more information.  Note that user-level theories usually do not directly
  refer to the HOL $\isarkeyword{typedef}$ primitive, but use more advanced
  packages such as $\isarkeyword{record}$ (see \S\ref{sec:hol-record}) and
  $\isarkeyword{datatype}$ (see \S\ref{sec:hol-datatype}).
\end{descr}


\section{Records}\label{sec:hol-record}

FIXME proof tools (simp, cases/induct; no split!?);

\indexisarcmdof{HOL}{record}
\begin{matharray}{rcl}
  \isarcmd{record} & : & \isartrans{theory}{theory} \\
\end{matharray}

\begin{rail}
  'record' typespec '=' (type '+')? (field +)
  ;

  field: name '::' type comment?
  ;
\end{rail}

\begin{descr}
\item [$\isarkeyword{record}~(\vec\alpha)t = \tau + \vec c :: \vec\sigma$]
  defines extensible record type $(\vec\alpha)t$, derived from the optional
  parent record $\tau$ by adding new field components $\vec c :: \vec\sigma$.
  See \cite{isabelle-HOL,NaraschewskiW-TPHOLs98} for more information on
  simply-typed extensible records.
\end{descr}


\section{Datatypes}\label{sec:hol-datatype}

\indexisarcmdof{HOL}{datatype}\indexisarcmdof{HOL}{rep-datatype}
\begin{matharray}{rcl}
  \isarcmd{datatype} & : & \isartrans{theory}{theory} \\
  \isarcmd{rep_datatype} & : & \isartrans{theory}{theory} \\
\end{matharray}

\railalias{repdatatype}{rep\_datatype}
\railterm{repdatatype}

\begin{rail}
  'datatype' (dtspec + 'and')
  ;
  repdatatype (name * ) dtrules
  ;

  dtspec: parname? typespec infix? '=' (cons + '|')
  ;
  cons: name (type * ) mixfix? comment?
  ;
  dtrules: 'distinct' thmrefs 'inject' thmrefs 'induction' thmrefs
\end{rail}

\begin{descr}
\item [$\isarkeyword{datatype}$] defines inductive datatypes in HOL.
\item [$\isarkeyword{rep_datatype}$] represents existing types as inductive
  ones, generating the standard infrastructure of derived concepts (primitive
  recursion etc.).
\end{descr}

The induction and exhaustion theorems generated provide case names according
to the constructors involved, while parameters are named after the types (see
also \S\ref{sec:cases-induct}).

See \cite{isabelle-HOL} for more details on datatypes.  Note that the theory
syntax above has been slightly simplified over the old version, usually
requiring more quotes and less parentheses.  Apart from proper proof methods
for case-analysis and induction, there are also emulations of ML tactics
\texttt{case_tac} and \texttt{induct_tac} available, see
\S\ref{sec:induct_tac}.


\section{Recursive functions}\label{sec:recursion}

\indexisarcmdof{HOL}{primrec}\indexisarcmdof{HOL}{recdef}\indexisarcmdof{HOL}{recdef-tc}
\begin{matharray}{rcl}
  \isarcmd{primrec} & : & \isartrans{theory}{theory} \\
  \isarcmd{recdef} & : & \isartrans{theory}{theory} \\
  \isarcmd{recdef_tc}^* & : & \isartrans{theory}{proof(prove)} \\
%FIXME
%  \isarcmd{defer_recdef} & : & \isartrans{theory}{theory} \\
\end{matharray}

\railalias{recdefsimp}{recdef\_simp}
\railterm{recdefsimp}

\railalias{recdefcong}{recdef\_cong}
\railterm{recdefcong}

\railalias{recdefwf}{recdef\_wf}
\railterm{recdefwf}

\railalias{recdeftc}{recdef\_tc}
\railterm{recdeftc}

\begin{rail}
  'primrec' parname? (equation + )
  ;
  'recdef' ('(' 'permissive' ')')? \\ name term (eqn + ) hints?
  ;
  recdeftc thmdecl? tc comment?
  ;

  equation: thmdecl? eqn
  ;
  eqn: prop comment?
  ;
  hints: '(' 'hints' (recdefmod * ) ')'
  ;
  recdefmod: ((recdefsimp | recdefcong | recdefwf) (() | 'add' | 'del') ':' thmrefs) | clasimpmod
  ;
  tc: nameref ('(' nat ')')?
  ;
\end{rail}

\begin{descr}
\item [$\isarkeyword{primrec}$] defines primitive recursive functions over
  datatypes, see also \cite{isabelle-HOL}.
\item [$\isarkeyword{recdef}$] defines general well-founded recursive
  functions (using the TFL package), see also \cite{isabelle-HOL}.  The
  $(permissive)$ option tells TFL to recover from failed proof attempts,
  returning unfinished results.  The $recdef_simp$, $recdef_cong$, and
  $recdef_wf$ hints refer to auxiliary rules to be used in the internal
  automated proof process of TFL.  Additional $clasimpmod$ declarations (cf.\ 
  \S\ref{sec:clasimp}) may be given to tune the context of the Simplifier
  (cf.\ \S\ref{sec:simplifier}) and Classical reasoner (cf.\ 
  \S\ref{sec:classical}).
\item [$\isarkeyword{recdef_tc}~c~(i)$] recommences the proof for leftover
  termination condition number $i$ (default $1$) as generated by a
  $\isarkeyword{recdef}$ definition of constant $c$.
  
  Note that in most cases, $\isarkeyword{recdef}$ is able to finish its
  internal proofs without manual intervention.
\end{descr}

Both kinds of recursive definitions accommodate reasoning by induction (cf.\ 
\S\ref{sec:cases-induct}): rule $c\mathord{.}induct$ (where $c$ is the name of
the function definition) refers to a specific induction rule, with parameters
named according to the user-specified equations.  Case names of
$\isarkeyword{primrec}$ are that of the datatypes involved, while those of
$\isarkeyword{recdef}$ are numbered (starting from $1$).

The equations provided by these packages may be referred later as theorem list
$f\mathord.simps$, where $f$ is the (collective) name of the functions
defined.  Individual equations may be named explicitly as well; note that for
$\isarkeyword{recdef}$ each specification given by the user may result in
several theorems.

\medskip Hints for $\isarkeyword{recdef}$ may be also declared globally, using
the following attributes.

\indexisarattof{HOL}{recdef-simp}\indexisarattof{HOL}{recdef-cong}\indexisarattof{HOL}{recdef-wf}
\begin{matharray}{rcl}
  recdef_simp & : & \isaratt \\
  recdef_cong & : & \isaratt \\
  recdef_wf & : & \isaratt \\
\end{matharray}

\railalias{recdefsimp}{recdef\_simp}
\railterm{recdefsimp}

\railalias{recdefcong}{recdef\_cong}
\railterm{recdefcong}

\railalias{recdefwf}{recdef\_wf}
\railterm{recdefwf}

\begin{rail}
  (recdefsimp | recdefcong | recdefwf) (() | 'add' | 'del')
  ;
\end{rail}


\section{(Co)Inductive sets}\label{sec:hol-inductive}

\indexisarcmdof{HOL}{inductive}\indexisarcmdof{HOL}{coinductive}\indexisarattof{HOL}{mono}
\begin{matharray}{rcl}
  \isarcmd{inductive} & : & \isartrans{theory}{theory} \\
  \isarcmd{coinductive} & : & \isartrans{theory}{theory} \\
  mono & : & \isaratt \\
\end{matharray}

\railalias{condefs}{con\_defs}
\railterm{condefs}

\begin{rail}
  ('inductive' | 'coinductive') sets intros monos?
  ;
  'mono' (() | 'add' | 'del')
  ;

  sets: (term comment? +)
  ;
  intros: 'intros' (thmdecl? prop comment? +)
  ;
  monos: 'monos' thmrefs comment?
  ;
\end{rail}

\begin{descr}
\item [$\isarkeyword{inductive}$ and $\isarkeyword{coinductive}$] define
  (co)inductive sets from the given introduction rules.
\item [$mono$] declares monotonicity rules.  These rule are involved in the
  automated monotonicity proof of $\isarkeyword{inductive}$.
\end{descr}

See \cite{isabelle-HOL} for further information on inductive definitions in
HOL.


\section{Arithmetic}

\indexisarmethof{HOL}{arith}\indexisarattof{HOL}{arith-split}
\begin{matharray}{rcl}
  arith & : & \isarmeth \\
  arith_split & : & \isaratt \\
\end{matharray}

\begin{rail}
  'arith' '!'?
  ;
\end{rail}

The $arith$ method decides linear arithmetic problems (on types $nat$, $int$,
$real$).  Any current facts are inserted into the goal before running the
procedure.  The ``!''~argument causes the full context of assumptions to be
included.  The $arith_split$ attribute declares case split rules to be
expanded before the arithmetic procedure is invoked.

Note that a simpler (but faster) version of arithmetic reasoning is already
performed by the Simplifier.


\section{Cases and induction: emulating tactic scripts}\label{sec:induct_tac}

The following important tactical tools of Isabelle/HOL have been ported to
Isar.  These should be never used in proper proof texts!

\indexisarmethof{HOL}{case-tac}\indexisarmethof{HOL}{induct-tac}
\indexisarmethof{HOL}{ind-cases}\indexisarcmdof{HOL}{inductive-cases}
\begin{matharray}{rcl}
  case_tac^* & : & \isarmeth \\
  induct_tac^* & : & \isarmeth \\
  ind_cases^* & : & \isarmeth \\
  \isarcmd{inductive_cases} & : & \isartrans{theory}{theory} \\
\end{matharray}

\railalias{casetac}{case\_tac}
\railterm{casetac}

\railalias{inducttac}{induct\_tac}
\railterm{inducttac}

\railalias{indcases}{ind\_cases}
\railterm{indcases}

\railalias{inductivecases}{inductive\_cases}
\railterm{inductivecases}

\begin{rail}
  casetac goalspec? term rule?
  ;
  inducttac goalspec? (insts * 'and') rule?
  ;
  indcases (prop +)
  ;
  inductivecases thmdecl? (prop +) comment?
  ;

  rule: ('rule' ':' thmref)
  ;
\end{rail}

\begin{descr}
\item [$case_tac$ and $induct_tac$] admit to reason about inductive datatypes
  only (unless an alternative rule is given explicitly).  Furthermore,
  $case_tac$ does a classical case split on booleans; $induct_tac$ allows only
  variables to be given as instantiation.  These tactic emulations feature
  both goal addressing and dynamic instantiation.  Note that named rule cases
  are \emph{not} provided as would be by the proper $induct$ and $cases$ proof
  methods (see \S\ref{sec:cases-induct}).
  
\item [$ind_cases$ and $\isarkeyword{inductive_cases}$] provide an interface
  to the \texttt{mk_cases} operation.  Rules are simplified in an unrestricted
  forward manner.
  
  While $ind_cases$ is a proof method to apply the result immediately as
  elimination rules, $\isarkeyword{inductive_cases}$ provides case split
  theorems at the theory level for later use,
\end{descr}


%%% Local Variables: 
%%% mode: latex
%%% TeX-master: "isar-ref"
%%% End: 


\begingroup
  \bibliographystyle{plain} \small\raggedright\frenchspacing
  \bibliography{../manual}
\endgroup


%% $Id$

\documentclass[12pt]{report}
\usepackage{graphicx,a4,../iman,../extra,../proof,../rail,../pdfsetup}

\title{\includegraphics[scale=0.5]{isabelle_isar} \\[4ex] The Isabelle/Isar Reference Manual}

\author{\emph{Markus Wenzel} \\ TU M\"unchen}

\setcounter{secnumdepth}{2} \setcounter{tocdepth}{2}

\pagestyle{headings}
\sloppy
\binperiod     %%%treat . like a binary operator

\railalias{lbrace}{\ttlbrace}
\railalias{rbrace}{\ttrbrace}
\railterm{lbrace,rbrace}

\railterm{ident,longident,symident,var,textvar,typefree,typevar,nat,string,verbatim}


\begin{document}

\underscoreoff

\maketitle 

\begin{abstract}
  FIXME
\end{abstract}

\pagenumbering{roman} \tableofcontents \clearfirst

%FIXME
\nocite{Rudnicki:1992:MizarOverview}
\nocite{Harrison:1996:MizarHOL}
\nocite{Rudnicki:1992:MizarOverview}
\nocite{Trybulec:1993:MizarFeatures}
\nocite{Syme:1997:DECLARE}
\nocite{Syme:1998:thesis}
\nocite{Syme:1999:TPHOL}
\nocite{Wenzel:1999:TPHOL}


\chapter{Introduction}

\section{Quick start}

Isar is already part of Isabelle (as of version Isabelle99, or later).  The
\texttt{isabelle} binary provides option \texttt{-I} to run the Isar
interaction loop at startup, rather than the plain ML top-level.  Thus the
quickest way to do anything with Isabelle/Isar is as follows:
\begin{ttbox}
isabelle -I HOL\medskip
\out{> Welcome to Isabelle/HOL (Isabelle99)}\medskip
theory Foo = Main:
constdefs foo :: nat  "foo == 1";
lemma "0 < foo" by (simp add: foo_def);
end
\end{ttbox}
Note that any Isabelle/Isar command may be retracted by \texttt{undo}; the
\texttt{help} command prints a list of available language elements.

Plain TTY-based interaction like this used to be quite feasible with
traditional tactic based theorem proving, but developing Isar documents
demands some better user-interface support.  \emph{Proof~General}\index{Proof
  General} of LFCS Edinburgh \cite{proofgeneral} offers a generic Emacs-based
environment for interactive theorem provers that does all the cut-and-paste
and forward-backward walk through the text in a very neat way.  Note that in
Isabelle/Isar, the current position within a partial proof document is equally
important than the actual proof state.  Thus Proof~General provides the
canonical working environment for Isabelle/Isar, both for getting acquainted
(e.g.\ by replaying existing Isar documents) and real production work.

\medskip

The easiest way to use Proof~General is to make it the default Isabelle user
interface.  Just put something like this into your Isabelle settings file (see
also \cite{isabelle-sys}):
\begin{ttbox}
ISABELLE_INTERFACE=\$ISABELLE_HOME/contrib/ProofGeneral/isar/interface
PROOFGENERAL_OPTIONS="-u false"
\end{ttbox}
You may have to change \texttt{\$ISABELLE_HOME/contrib/ProofGeneral} to the
actual installation directory of Proof~General.  From now on, the capital
\texttt{Isabelle} executable refers to the \texttt{ProofGeneral/isar}
interface.\footnote{There is also a \texttt{ProofGeneral/isa} interface, for
  classic Isabelle tactic scripts.}  Its usage is as follows:
\begin{ttbox}
Usage: interface [OPTIONS] [FILES ...]

  Options are:
    -l NAME      logic image name (default $ISABELLE_LOGIC=HOL)
    -p NAME      Emacs program name (default xemacs)
    -u BOOL      use .emacs file (default true)
    -w BOOL      use window system (default true)

  Starts Proof General for Isabelle/Isar with proof documents FILES
  (default Scratch.thy).

  PROOFGENERAL_OPTIONS=
\end{ttbox} %$
Apart from the command line, the defaults for these options may be overridden
via the \texttt{PROOFGENERAL_OPTIONS} setting as well.  For example, plain GNU
Emacs may be configured as follows:
\begin{ttbox}
PROOFGENERAL_OPTIONS="-u false -p emacs"
\end{ttbox}

Occasionally, a user's \texttt{.emacs} file contains material that is
incompatible with the version of (X)Emacs that Proof~General prefers.  Then
proper startup may be still achieved by using the \texttt{-u false}
option.\footnote{Any Emacs lisp file \texttt{proofgeneral-settings.el}
  occurring in \texttt{\$ISABELLE_HOME/etc} or
  \texttt{\$ISABELLE_HOME_USER/etc} is automatically loaded by the
  Proof~General interface script as well.}

\medskip

With the proper Isabelle interface setup, Isar documents may now be edited by
visiting appropriate theory files, e.g.\ 
\begin{ttbox}
Isabelle \({\langle}isabellehome{\rangle}\)/src/HOL/Isar_examples/BasicLogic.thy
\end{ttbox}
Users of XEmacs may note the tool bar for navigating forward and backward
through the text.  Consult the Proof~General documentation \cite{proofgeneral}
for further basic command sequences, such as ``\texttt{c-c return}'' or
``\texttt{c-c u}''.


\section{Isabelle/Isar theories}

Isabelle/Isar offers two main improvements over classic Isabelle:
\begin{enumerate}
\item A new \emph{theory format}, occasionally referred to as ``new-style
  theories'', supporting interactive development and unlimited undo operation.
\item A \emph{formal proof document language} designed to support intelligible
  semi-automated reasoning.  Instead of putting together unreadable tactic
  scripts, the author is enabled to express the reasoning in way that is close
  to mathematical practice.
\end{enumerate}

The Isar proof language is embedded into the new theory format as a proper
sub-language.  Proof mode is entered by stating some $\THEOREMNAME$ or
$\LEMMANAME$ at the theory level, and left again with the final conclusion
(e.g.\ via $\QEDNAME$).  A few theory extension mechanisms require proof as
well, such as the HOL $\isarkeyword{typedef}$ which demands non-emptiness of
the representing sets.

New-style theory files may still be associated with separate ML files
consisting of plain old tactic scripts.  There is no longer any ML binding
generated for the theory and theorems, though.  ML functions \texttt{theory},
\texttt{thm}, and \texttt{thms} retrieve this information \cite{isabelle-ref}.
Nevertheless, migration between classic Isabelle and Isabelle/Isar is
relatively easy.  Thus users may start to benefit from interactive theory
development even before they have any idea of the Isar proof language at all.

\begin{warn}
  Currently Proof~General does \emph{not} support mixed interactive
  development of classic Isabelle theory files or tactic scripts, together
  with Isar documents.  The ``\texttt{isa}'' and ``\texttt{isar}'' versions of
  Proof~General are handled as two different theorem proving systems, only one
  of these may be active at the same time.
\end{warn}

Porting of existing tactic scripts is best done by running two separate
Proof~General sessions, one for replaying the old script and the other for the
emerging Isabelle/Isar document.


\section{How to write Isar proofs anyway?}

This is one of the key questions, of course.  Isar offers a rather different
approach to formal proof documents than plain old tactic scripts.  Experienced
users of existing interactive theorem proving systems may have to learn
thinking differently in order to make effective use of Isabelle/Isar.  On the
other hand, Isabelle/Isar comes much closer to existing mathematical practice
of formal proof, so users with less experience in old-style tactical proving,
but a good understanding of mathematical proof, might cope with Isar even
better.  See also \cite{Wenzel:1999:TPHOL} for further background information
on Isar.

\medskip This really is a \emph{reference manual}.  Nevertheless, we will also
give some clues of how the concepts introduced here may be put into practice.
Appendix~\ref{ap:refcard} provides a quick reference card of the most common
Isabelle/Isar language elements.  There are several examples distributed with
Isabelle, and available via the Isabelle WWW library:
\begin{center}\small
  \begin{tabular}{l}
    \url{http://www.cl.cam.ac.uk/Research/HVG/Isabelle/library/} \\
    \url{http://isabelle.in.tum.de/library/} \\
  \end{tabular}
\end{center}

See \texttt{HOL/Isar_examples} for a collection of introductory examples, and
\texttt{HOL/HOL-Real/HahnBanach} is a big mathematics application.  Apart from
browsable HTML sources, both sessions provide actual documents (in PDF).

%%% Local Variables: 
%%% mode: latex
%%% TeX-master: "isar-ref"
%%% End: 


%FIXME
%\chapter{Basic Concepts}\label{ch:basics}
%\section{The Isar proof language}

%%% Local Variables: 
%%% mode: latex
%%% TeX-master: "isar-ref"
%%% End: 


\chapter{Isar document syntax}

\section{Inner versus outer syntax}

\section{Lexical matters}

\section{Common syntax entities}

\subsection{Atoms}

\begin{rail}
  name : ident | symident | string
  ;

  nameref : name | longident
  ;

  text : nameref | verbatim
  ;
\end{rail}

\subsection{Comments}

\begin{rail}
  comment : (() | '--' text)
  ;
  interest : (() | '\%')
  ;
\end{rail}


\subsection{Sorts and arities}

\begin{rail}
  sort : nameref | lbrace (nameref * ',') rbrace
  ;
  arity : ( () | '(' (sort + ',') ')' ) sort
  ;
  simple\-arity : ( () | '(' (sort + ',') ')' ) nameref
  ;
\end{rail}


\subsection{Terms and Types}

\begin{rail}
  
\end{rail}

\subsection{Mixfix annotations}


\subsection{}

\subsection{}

\subsection{}


%%% Local Variables: 
%%% mode: latex
%%% TeX-master: "isar-ref"
%%% End: 


\chapter{Basic Isar Language Elements}\label{ch:pure-syntax}

Subsequently, we introduce the main part of Pure Isar theory and proof
commands, together with fundamental proof methods and attributes.
Chapter~\ref{ch:gen-tools} describes further Isar elements provided by generic
tools and packages (such as the Simplifier) that are either part of Pure
Isabelle or pre-installed by most object logics.  Chapter~\ref{ch:hol-tools}
refers to actual object-logic specific elements of Isabelle/HOL.

\medskip

Isar commands may be either \emph{proper} document constructors, or
\emph{improper commands}.  Some proof methods and attributes introduced later
are classified as improper as well.  Improper Isar language elements, which
are subsequently marked by $^*$, are often helpful when developing proof
documents, while their use is discouraged for the final outcome.  Typical
examples are diagnostic commands that print terms or theorems according to the
current context; other commands even emulate old-style tactical theorem
proving.


\section{Theory commands}

\subsection{Defining theories}\label{sec:begin-thy}

\indexisarcmd{header}\indexisarcmd{theory}\indexisarcmd{end}\indexisarcmd{context}
\begin{matharray}{rcl}
  \isarcmd{header} & : & \isarkeep{toplevel} \\
  \isarcmd{theory} & : & \isartrans{toplevel}{theory} \\
  \isarcmd{context}^* & : & \isartrans{toplevel}{theory} \\
  \isarcmd{end} & : & \isartrans{theory}{toplevel} \\
\end{matharray}

Isabelle/Isar ``new-style'' theories are either defined via theory files or
interactively.  Both theory-level specifications and proofs are handled
uniformly --- occasionally definitional mechanisms even require some explicit
proof as well.  In contrast, ``old-style'' Isabelle theories support batch
processing only, with the proof scripts collected in separate ML files.

The first actual command of any theory has to be $\THEORY$, starting a new
theory based on the merge of existing ones.  Just preceding $\THEORY$, there
may be an optional $\isarkeyword{header}$ declaration, which is relevant to
document preparation only; it acts very much like a special pre-theory markup
command (cf.\ \S\ref{sec:markup-thy} and \S\ref{sec:markup-thy}).  The theory
context may be also changed by $\CONTEXT$ without creating a new theory.  In
both cases, $\END$ concludes the theory development; it has to be the very
last command of any theory file.

\begin{rail}
  'header' text
  ;
  'theory' name '=' (name + '+') filespecs? ':'
  ;
  'context' name
  ;
  'end'
  ;;

  filespecs: 'files' ((name | parname) +);
\end{rail}

\begin{descr}
\item [$\isarkeyword{header}~text$] provides plain text markup just preceding
  the formal beginning of a theory.  In actual document preparation the
  corresponding {\LaTeX} macro \verb,\isamarkupheader, may be redefined to
  produce chapter or section headings.  See also \S\ref{sec:markup-thy} and
  \S\ref{sec:markup-prf} for further markup commands.
  
\item [$\THEORY~A = B@1 + \cdots + B@n\colon$] commences a new theory $A$
  based on existing ones $B@1 + \cdots + B@n$.  Isabelle's theory loader
  system ensures that any of the base theories are properly loaded (and fully
  up-to-date when $\THEORY$ is executed interactively).  The optional
  $\isarkeyword{files}$ specification declares additional dependencies on ML
  files.  Unless put in parentheses, any file will be loaded immediately via
  $\isarcmd{use}$ (see also \S\ref{sec:ML}).  The optional ML file
  \texttt{$A$.ML} that may be associated with any theory should \emph{not} be
  included in $\isarkeyword{files}$, though.
  
\item [$\CONTEXT~B$] enters an existing theory context, basically in read-only
  mode, so only a limited set of commands may be performed without destroying
  the theory.  Just as for $\THEORY$, the theory loader ensures that $B$ is
  loaded and up-to-date.
  
\item [$\END$] concludes the current theory definition or context switch.
Note that this command cannot be undone, but the whole theory definition has
to be retracted.
\end{descr}


\subsection{Theory markup commands}\label{sec:markup-thy}

\indexisarcmd{chapter}\indexisarcmd{section}\indexisarcmd{subsection}
\indexisarcmd{subsubsection}\indexisarcmd{text}\indexisarcmd{text-raw}
\begin{matharray}{rcl}
  \isarcmd{chapter} & : & \isartrans{theory}{theory} \\
  \isarcmd{section} & : & \isartrans{theory}{theory} \\
  \isarcmd{subsection} & : & \isartrans{theory}{theory} \\
  \isarcmd{subsubsection} & : & \isartrans{theory}{theory} \\
  \isarcmd{text} & : & \isartrans{theory}{theory} \\
  \isarcmd{text_raw} & : & \isartrans{theory}{theory} \\
\end{matharray}

Apart from formal comments (see \S\ref{sec:comments}), markup commands provide
a structured way to insert text into the document generated from a theory (see
\cite{isabelle-sys} for more information on Isabelle's document preparation
tools).

\railalias{textraw}{text\_raw}
\railterm{textraw}

\begin{rail}
  ('chapter' | 'section' | 'subsection' | 'subsubsection' | 'text' | textraw) text
  ;
\end{rail}

\begin{descr}
\item [$\isarkeyword{chapter}$, $\isarkeyword{section}$,
  $\isarkeyword{subsection}$, and $\isarkeyword{subsubsection}$] mark chapter
  and section headings.
\item [$\TEXT$] specifies paragraphs of plain text, including references to
  formal entities.\footnote{The latter feature is not yet supported.
    Nevertheless, any source text of the form
    ``\texttt{\at\ttlbrace$\dots$\ttrbrace}'' should be considered as reserved
    for future use.}
\item [$\isarkeyword{text_raw}$] inserts {\LaTeX} source into the output,
  without additional markup.  Thus the full range of document manipulations
  becomes available.  A typical application would be to emit
  \verb,\begin{comment}, and \verb,\end{comment}, commands to exclude certain
  parts from the final document.\footnote{This requires the \texttt{comment}
    package to be included in {\LaTeX}, of course.}
\end{descr}

Any markup command (except $\isarkeyword{text_raw}$) corresponds to a {\LaTeX}
macro with the name prefixed by \verb,\isamarkup, (e.g.\ 
\verb,\isamarkupchapter, for $\isarkeyword{chapter}$). The \railqtoken{text}
argument is passed to that macro unchanged, i.e.\ further {\LaTeX} commands
may be included here as well.

\medskip

Additional markup commands are available for proofs (see
\S\ref{sec:markup-prf}).  Also note that the $\isarkeyword{header}$
declaration (see \S\ref{sec:begin-thy}) admits to insert document markup
elements just preceding the actual theory definition.


\subsection{Type classes and sorts}\label{sec:classes}

\indexisarcmd{classes}\indexisarcmd{classrel}\indexisarcmd{defaultsort}
\begin{matharray}{rcl}
  \isarcmd{classes} & : & \isartrans{theory}{theory} \\
  \isarcmd{classrel} & : & \isartrans{theory}{theory} \\
  \isarcmd{defaultsort} & : & \isartrans{theory}{theory} \\
\end{matharray}

\begin{rail}
  'classes' (classdecl comment? +)
  ;
  'classrel' nameref '<' nameref comment?
  ;
  'defaultsort' sort comment?
  ;
\end{rail}

\begin{descr}
\item [$\isarkeyword{classes}~c<\vec c$] declares class $c$ to be a subclass
  of existing classes $\vec c$.  Cyclic class structures are ruled out.
\item [$\isarkeyword{classrel}~c@1<c@2$] states a subclass relation between
  existing classes $c@1$ and $c@2$.  This is done axiomatically!  The
  $\isarkeyword{instance}$ command (see \S\ref{sec:axclass}) provides a way to
  introduce proven class relations.
\item [$\isarkeyword{defaultsort}~s$] makes sort $s$ the new default sort for
  any type variables given without sort constraints.  Usually, the default
  sort would be only changed when defining new object-logics.
\end{descr}


\subsection{Primitive types and type abbreviations}\label{sec:types-pure}

\indexisarcmd{typedecl}\indexisarcmd{types}\indexisarcmd{nonterminals}\indexisarcmd{arities}
\begin{matharray}{rcl}
  \isarcmd{types} & : & \isartrans{theory}{theory} \\
  \isarcmd{typedecl} & : & \isartrans{theory}{theory} \\
  \isarcmd{nonterminals} & : & \isartrans{theory}{theory} \\
  \isarcmd{arities} & : & \isartrans{theory}{theory} \\
\end{matharray}

\begin{rail}
  'types' (typespec '=' type infix? comment? +)
  ;
  'typedecl' typespec infix? comment?
  ;
  'nonterminals' (name +) comment?
  ;
  'arities' (nameref '::' arity comment? +)
  ;
\end{rail}

\begin{descr}
\item [$\TYPES~(\vec\alpha)t = \tau$] introduces \emph{type synonym}
  $(\vec\alpha)t$ for existing type $\tau$.  Unlike actual type definitions,
  as are available in Isabelle/HOL for example, type synonyms are just purely
  syntactic abbreviations without any logical significance.  Internally, type
  synonyms are fully expanded.
\item [$\isarkeyword{typedecl}~(\vec\alpha)t$] declares a new type constructor
  $t$, intended as an actual logical type.  Note that object-logics such as
  Isabelle/HOL override $\isarkeyword{typedecl}$ by their own version.
\item [$\isarkeyword{nonterminals}~\vec c$] declares $0$-ary type constructors
  $\vec c$ to act as purely syntactic types, i.e.\ nonterminal symbols of
  Isabelle's inner syntax of terms or types.
\item [$\isarkeyword{arities}~t::(\vec s)s$] augments Isabelle's order-sorted
  signature of types by new type constructor arities.  This is done
  axiomatically!  The $\isarkeyword{instance}$ command (see
  \S\ref{sec:axclass}) provides a way to introduce proven type arities.
\end{descr}


\subsection{Constants and simple definitions}\label{sec:consts}

\indexisarcmd{consts}\indexisarcmd{defs}\indexisarcmd{constdefs}\indexoutertoken{constdecl}
\begin{matharray}{rcl}
  \isarcmd{consts} & : & \isartrans{theory}{theory} \\
  \isarcmd{defs} & : & \isartrans{theory}{theory} \\
  \isarcmd{constdefs} & : & \isartrans{theory}{theory} \\
\end{matharray}

\begin{rail}
  'consts' (constdecl +)
  ;
  'defs' (axmdecl prop comment? +)
  ;
  'constdefs' (constdecl prop comment? +)
  ;

  constdecl: name '::' type mixfix? comment?
  ;
\end{rail}

\begin{descr}
\item [$\CONSTS~c::\sigma$] declares constant $c$ to have any instance of type
  scheme $\sigma$.  The optional mixfix annotations may attach concrete syntax
  to the constants declared.
\item [$\DEFS~name: eqn$] introduces $eqn$ as a definitional axiom for some
  existing constant.  See \cite[\S6]{isabelle-ref} for more details on the
  form of equations admitted as constant definitions.
\item [$\isarkeyword{constdefs}~c::\sigma~eqn$] combines declarations and
  definitions of constants, using the canonical name $c_def$ for the
  definitional axiom.
\end{descr}


\subsection{Syntax and translations}\label{sec:syn-trans}

\indexisarcmd{syntax}\indexisarcmd{translations}
\begin{matharray}{rcl}
  \isarcmd{syntax} & : & \isartrans{theory}{theory} \\
  \isarcmd{translations} & : & \isartrans{theory}{theory} \\
\end{matharray}

\begin{rail}
  'syntax' ('(' name 'output'? ')')? (constdecl +)
  ;
  'translations' (transpat ('==' | '=>' | '<=') transpat comment? +)
  ;
  transpat: ('(' nameref ')')? string
  ;
\end{rail}

\begin{descr}
\item [$\isarkeyword{syntax}~(mode)~decls$] is similar to $\CONSTS~decls$,
  except that the actual logical signature extension is omitted.  Thus the
  context free grammar of Isabelle's inner syntax may be augmented in
  arbitrary ways, independently of the logic.  The $mode$ argument refers to
  the print mode that the grammar rules belong; unless the \texttt{output}
  flag is given, all productions are added both to the input and output
  grammar.
\item [$\isarkeyword{translations}~rules$] specifies syntactic translation
  rules (i.e.\ \emph{macros}): parse~/ print rules (\texttt{==}), parse rules
  (\texttt{=>}), or print rules (\texttt{<=}).  Translation patterns may be
  prefixed by the syntactic category to be used for parsing; the default is
  \texttt{logic}.
\end{descr}


\subsection{Axioms and theorems}

\indexisarcmd{axioms}\indexisarcmd{theorems}\indexisarcmd{lemmas}
\begin{matharray}{rcl}
  \isarcmd{axioms} & : & \isartrans{theory}{theory} \\
  \isarcmd{theorems} & : & \isartrans{theory}{theory} \\
  \isarcmd{lemmas} & : & \isartrans{theory}{theory} \\
\end{matharray}

\begin{rail}
  'axioms' (axmdecl prop comment? +)
  ;
  ('theorems' | 'lemmas') thmdef? thmrefs
  ;
\end{rail}

\begin{descr}
\item [$\isarkeyword{axioms}~a: \phi$] introduces arbitrary statements as
  axioms of the meta-logic.  In fact, axioms are ``axiomatic theorems'', and
  may be referred later just as any other theorem.
  
  Axioms are usually only introduced when declaring new logical systems.
  Everyday work is typically done the hard way, with proper definitions and
  actual proven theorems.
\item [$\isarkeyword{theorems}~a = \vec b$] stores lists of existing theorems.
  Typical applications would also involve attributes, to declare Simplifier
  rules, for example.
\item [$\isarkeyword{lemmas}$] is similar to $\isarkeyword{theorems}$, but
  tags the results as ``lemma''.
\end{descr}


\subsection{Name spaces}

\indexisarcmd{global}\indexisarcmd{local}
\begin{matharray}{rcl}
  \isarcmd{global} & : & \isartrans{theory}{theory} \\
  \isarcmd{local} & : & \isartrans{theory}{theory} \\
\end{matharray}

Isabelle organizes any kind of name declarations (of types, constants,
theorems etc.) by separate hierarchically structured name spaces.  Normally
the user never has to control the behavior of name space entry by hand, yet
the following commands provide some way to do so.

\begin{descr}
\item [$\isarkeyword{global}$ and $\isarkeyword{local}$] change the current
  name declaration mode.  Initially, theories start in $\isarkeyword{local}$
  mode, causing all names to be automatically qualified by the theory name.
  Changing this to $\isarkeyword{global}$ causes all names to be declared
  without the theory prefix, until $\isarkeyword{local}$ is declared again.
\end{descr}


\subsection{Incorporating ML code}\label{sec:ML}

\indexisarcmd{use}\indexisarcmd{ML}\indexisarcmd{ML-setup}\indexisarcmd{setup}
\begin{matharray}{rcl}
  \isarcmd{use} & : & \isartrans{\cdot}{\cdot} \\
  \isarcmd{ML} & : & \isartrans{\cdot}{\cdot} \\
  \isarcmd{ML_setup} & : & \isartrans{theory}{theory} \\
  \isarcmd{setup} & : & \isartrans{theory}{theory} \\
\end{matharray}

\railalias{MLsetup}{ML\_setup}
\railterm{MLsetup}

\begin{rail}
  'use' name
  ;
  ('ML' | MLsetup | 'setup') text
  ;
\end{rail}

\begin{descr}
\item [$\isarkeyword{use}~file$] reads and executes ML commands from $file$.
  The current theory context (if present) is passed down to the ML session,
  but may not be modified.  Furthermore, the file name is checked with the
  $\isarkeyword{files}$ dependency declaration given in the theory header (see
  also \S\ref{sec:begin-thy}).
  
\item [$\isarkeyword{ML}~text$] executes ML commands from $text$.  The theory
  context is passed in the same way as for $\isarkeyword{use}$.
  
\item [$\isarkeyword{ML_setup}~text$] executes ML commands from $text$.  The
  theory context is passed down to the ML session, and fetched back
  afterwards.  Thus $text$ may actually change the theory as a side effect.
  
\item [$\isarkeyword{setup}~text$] changes the current theory context by
  applying $text$, which refers to an ML expression of type
  \texttt{(theory~->~theory)~list}.  The $\isarkeyword{setup}$ command is the
  canonical way to initialize any object-logic specific tools and packages
  written in ML.
\end{descr}


\subsection{Syntax translation functions}

\indexisarcmd{parse-ast-translation}\indexisarcmd{parse-translation}
\indexisarcmd{print-translation}\indexisarcmd{typed-print-translation}
\indexisarcmd{print-ast-translation}\indexisarcmd{token-translation}
\begin{matharray}{rcl}
  \isarcmd{parse_ast_translation} & : & \isartrans{theory}{theory} \\
  \isarcmd{parse_translation} & : & \isartrans{theory}{theory} \\
  \isarcmd{print_translation} & : & \isartrans{theory}{theory} \\
  \isarcmd{typed_print_translation} & : & \isartrans{theory}{theory} \\
  \isarcmd{print_ast_translation} & : & \isartrans{theory}{theory} \\
  \isarcmd{token_translation} & : & \isartrans{theory}{theory} \\
\end{matharray}

Syntax translation functions written in ML admit almost arbitrary
manipulations of Isabelle's inner syntax.  Any of the above commands have a
single \railqtoken{text} argument that refers to an ML expression of
appropriate type.

\begin{ttbox}
val parse_ast_translation   : (string * (ast list -> ast)) list
val parse_translation       : (string * (term list -> term)) list
val print_translation       : (string * (term list -> term)) list
val typed_print_translation :
  (string * (bool -> typ -> term list -> term)) list
val print_ast_translation   : (string * (ast list -> ast)) list
val token_translation       :
  (string * string * (string -> string * real)) list
\end{ttbox}
See \cite[\S8]{isabelle-ref} for more information on syntax transformations.


\subsection{Oracles}

\indexisarcmd{oracle}
\begin{matharray}{rcl}
  \isarcmd{oracle} & : & \isartrans{theory}{theory} \\
\end{matharray}

Oracles provide an interface to external reasoning systems, without giving up
control completely --- each theorem carries a derivation object recording any
oracle invocation.  See \cite[\S6]{isabelle-ref} for more information.

\begin{rail}
  'oracle' name '=' text comment?
  ;
\end{rail}

\begin{descr}
\item [$\isarkeyword{oracle}~name=text$] declares oracle $name$ to be ML
  function $text$, which has to be of type
  \texttt{Sign.sg~*~Object.T~->~term}.
\end{descr}


\section{Proof commands}

Proof commands perform transitions of Isar/VM machine configurations, which
are block-structured, consisting of a stack of nodes with three main
components: logical proof context, current facts, and open goals.  Isar/VM
transitions are \emph{typed} according to the following three different modes
of operation:
\begin{descr}
\item [$proof(prove)$] means that a new goal has just been stated that is now
  to be \emph{proven}; the next command may refine it by some proof method,
  and enter a sub-proof to establish the actual result.
\item [$proof(state)$] is like an internal theory mode: the context may be
  augmented by \emph{stating} additional assumptions, intermediate results
  etc.
\item [$proof(chain)$] is intermediate between $proof(state)$ and
  $proof(prove)$: existing facts (i.e.\ the contents of the special ``$this$''
  register) have been just picked up in order to be used when refining the
  goal claimed next.
\end{descr}


\subsection{Proof markup commands}\label{sec:markup-prf}

\indexisarcmd{sect}\indexisarcmd{subsect}\indexisarcmd{subsubsect}
\indexisarcmd{txt}\indexisarcmd{txt-raw}
\begin{matharray}{rcl}
  \isarcmd{sect} & : & \isartrans{proof}{proof} \\
  \isarcmd{subsect} & : & \isartrans{proof}{proof} \\
  \isarcmd{subsubsect} & : & \isartrans{proof}{proof} \\
  \isarcmd{txt} & : & \isartrans{proof}{proof} \\
  \isarcmd{txt_raw} & : & \isartrans{proof}{proof} \\
\end{matharray}

These markup commands for proof mode closely correspond to the ones of theory
mode (see \S\ref{sec:markup-thy}).  Note that $\isarkeyword{txt_raw}$ is
special in the same way as $\isarkeyword{text_raw}$.

\railalias{txtraw}{txt\_raw}
\railterm{txtraw}

\begin{rail}
  ('sect' | 'subsect' | 'subsubsect' | 'txt' | txtraw) text
  ;
\end{rail}


\subsection{Proof context}\label{sec:proof-context}

\indexisarcmd{fix}\indexisarcmd{assume}\indexisarcmd{presume}\indexisarcmd{def}
\begin{matharray}{rcl}
  \isarcmd{fix} & : & \isartrans{proof(state)}{proof(state)} \\
  \isarcmd{assume} & : & \isartrans{proof(state)}{proof(state)} \\
  \isarcmd{presume} & : & \isartrans{proof(state)}{proof(state)} \\
  \isarcmd{def} & : & \isartrans{proof(state)}{proof(state)} \\
\end{matharray}

The logical proof context consists of fixed variables and assumptions.  The
former closely correspond to Skolem constants, or meta-level universal
quantification as provided by the Isabelle/Pure logical framework.
Introducing some \emph{arbitrary, but fixed} variable via $\FIX x$ results in
a local value that may be used in the subsequent proof as any other variable
or constant.  Furthermore, any result $\edrv \phi[x]$ exported from the
context will be universally closed wrt.\ $x$ at the outermost level: $\edrv
\All x \phi$ (this is expressed using Isabelle's meta-variables).

Similarly, introducing some assumption $\chi$ has two effects.  On the one
hand, a local theorem is created that may be used as a fact in subsequent
proof steps.  On the other hand, any result $\chi \drv \phi$ exported from the
context becomes conditional wrt.\ the assumption: $\edrv \chi \Imp \phi$.
Thus, solving an enclosing goal using such a result would basically introduce
a new subgoal stemming from the assumption.  How this situation is handled
depends on the actual version of assumption command used: while $\ASSUMENAME$
insists on solving the subgoal by unification with some premise of the goal,
$\PRESUMENAME$ leaves the subgoal unchanged in order to be proved later by the
user.

Local definitions, introduced by $\DEF{}{x \equiv t}$, are achieved by
combining $\FIX x$ with another version of assumption that causes any
hypothetical equation $x \equiv t$ to be eliminated by the reflexivity rule.
Thus, exporting some result $x \equiv t \drv \phi[x]$ yields $\edrv \phi[t]$.

\begin{rail}
  'fix' (vars + 'and') comment?
  ;
  ('assume' | 'presume') (assm comment? + 'and')
  ;
  'def' thmdecl? \\ var '==' term termpat? comment?
  ;

  var: name ('::' type)?
  ;
  vars: (name+) ('::' type)?
  ;
  assm: thmdecl? (prop proppat? +)
  ;
\end{rail}

\begin{descr}
\item [$\FIX{\vec x}$] introduces local \emph{arbitrary, but fixed} variables
  $\vec x$.
\item [$\ASSUME{a}{\vec\phi}$ and $\PRESUME{a}{\vec\phi}$] introduce local
  theorems $\vec\phi$ by assumption.  Subsequent results applied to an
  enclosing goal (e.g.\ by $\SHOWNAME$) are handled as follows: $\ASSUMENAME$
  expects to be able to unify with existing premises in the goal, while
  $\PRESUMENAME$ leaves $\vec\phi$ as new subgoals.
  
  Several lists of assumptions may be given (separated by
  $\isarkeyword{and}$); the resulting list of current facts consists of all of
  these concatenated.
\item [$\DEF{a}{x \equiv t}$] introduces a local (non-polymorphic) definition.
  In results exported from the context, $x$ is replaced by $t$.  Basically,
  $\DEF{}{x \equiv t}$ abbreviates $\FIX{x}~\ASSUME{}{x \equiv t}$, with the
  resulting hypothetical equation solved by reflexivity.
  
  The default name for the definitional equation is $x_def$.
\end{descr}

The special name $prems$\indexisarthm{prems} refers to all assumptions of the
current context as a list of theorems.


\subsection{Facts and forward chaining}

\indexisarcmd{note}\indexisarcmd{then}\indexisarcmd{from}\indexisarcmd{with}
\begin{matharray}{rcl}
  \isarcmd{note} & : & \isartrans{proof(state)}{proof(state)} \\
  \isarcmd{then} & : & \isartrans{proof(state)}{proof(chain)} \\
  \isarcmd{from} & : & \isartrans{proof(state)}{proof(chain)} \\
  \isarcmd{with} & : & \isartrans{proof(state)}{proof(chain)} \\
\end{matharray}

New facts are established either by assumption or proof of local statements.
Any fact will usually be involved in further proofs, either as explicit
arguments of proof methods, or when forward chaining towards the next goal via
$\THEN$ (and variants).  Note that the special theorem name
$this$\indexisarthm{this} refers to the most recently established facts.
\begin{rail}
  'note' thmdef? thmrefs comment?
  ;
  'then' comment?
  ;
  ('from' | 'with') thmrefs comment?
  ;
\end{rail}

\begin{descr}
\item [$\NOTE{a}{\vec b}$] recalls existing facts $\vec b$, binding the result
  as $a$.  Note that attributes may be involved as well, both on the left and
  right hand sides.
\item [$\THEN$] indicates forward chaining by the current facts in order to
  establish the goal to be claimed next.  The initial proof method invoked to
  refine that will be offered the facts to do ``anything appropriate'' (cf.\ 
  also \S\ref{sec:proof-steps}).  For example, method $rule$ (see
  \S\ref{sec:pure-meth-att}) would typically do an elimination rather than an
  introduction.  Automatic methods usually insert the facts into the goal
  state before operation.  This provides a simple scheme to control relevance
  of facts in automated proof search.
\item [$\FROM{\vec b}$] abbreviates $\NOTE{}{\vec b}~\THEN$; thus $\THEN$ is
  equivalent to $\FROM{this}$.
\item [$\WITH{\vec b}$] abbreviates $\FROM{\vec b~facts}$; thus the forward
  chaining is from earlier facts together with the current ones.
\end{descr}

Basic proof methods (such as $rule$, see \S\ref{sec:pure-meth-att}) expect
multiple facts to be given in their proper order, corresponding to a prefix of
the premises of the rule involved.  Note that positions may be easily skipped
using something like $\FROM{\text{\texttt{_}}~a~b}$, for example.  This
involves the trivial rule $\PROP\psi \Imp \PROP\psi$, which happens to be
bound in Isabelle/Pure as ``\texttt{_}''
(underscore).\indexisarthm{_@\texttt{_}}


\subsection{Goal statements}

\indexisarcmd{theorem}\indexisarcmd{lemma}
\indexisarcmd{have}\indexisarcmd{show}\indexisarcmd{hence}\indexisarcmd{thus}
\begin{matharray}{rcl}
  \isarcmd{theorem} & : & \isartrans{theory}{proof(prove)} \\
  \isarcmd{lemma} & : & \isartrans{theory}{proof(prove)} \\
  \isarcmd{have} & : & \isartrans{proof(state) ~|~ proof(chain)}{proof(prove)} \\
  \isarcmd{show} & : & \isartrans{proof(state) ~|~ proof(chain)}{proof(prove)} \\
  \isarcmd{hence} & : & \isartrans{proof(state)}{proof(prove)} \\
  \isarcmd{thus} & : & \isartrans{proof(state)}{proof(prove)} \\
\end{matharray}

Proof mode is entered from theory mode by initial goal commands $\THEOREMNAME$
and $\LEMMANAME$.  New local goals may be claimed within proof mode as well.
Four variants are available, indicating whether the result is meant to solve
some pending goal or whether forward chaining is indicated.

\begin{rail}
  ('theorem' | 'lemma') goal
  ;
  ('have' | 'show' | 'hence' | 'thus') goal
  ;

  goal: thmdecl? proppat comment?
  ;
\end{rail}

\begin{descr}
\item [$\THEOREM{a}{\phi}$] enters proof mode with $\phi$ as main goal,
  eventually resulting in some theorem $\turn \phi$ to be put back into the
  theory.
\item [$\LEMMA{a}{\phi}$] is similar to $\THEOREMNAME$, but tags the result as
  ``lemma''.
\item [$\HAVE{a}{\phi}$] claims a local goal, eventually resulting in a
  theorem with the current assumption context as hypotheses.
\item [$\SHOW{a}{\phi}$] is similar to $\HAVE{a}{\phi}$, but solves some
  pending goal with the result \emph{exported} into the corresponding context
  (cf.\ \S\ref{sec:proof-context}).
\item [$\HENCENAME$] abbreviates $\THEN~\HAVENAME$, i.e.\ claims a local goal
  to be proven by forward chaining the current facts.  Note that $\HENCENAME$
  is also equivalent to $\FROM{this}~\HAVENAME$.
\item [$\THUSNAME$] abbreviates $\THEN~\SHOWNAME$.  Note that $\THUSNAME$ is
  also equivalent to $\FROM{this}~\SHOWNAME$.
\end{descr}


\subsection{Initial and terminal proof steps}\label{sec:proof-steps}

\indexisarcmd{proof}\indexisarcmd{qed}\indexisarcmd{by}
\indexisarcmd{.}\indexisarcmd{..}\indexisarcmd{sorry}
\begin{matharray}{rcl}
  \isarcmd{proof} & : & \isartrans{proof(prove)}{proof(state)} \\
  \isarcmd{qed} & : & \isartrans{proof(state)}{proof(state) ~|~ theory} \\
  \isarcmd{by} & : & \isartrans{proof(prove)}{proof(state) ~|~ theory} \\
  \isarcmd{.\,.} & : & \isartrans{proof(prove)}{proof(state) ~|~ theory} \\
  \isarcmd{.} & : & \isartrans{proof(prove)}{proof(state) ~|~ theory} \\
  \isarcmd{sorry} & : & \isartrans{proof(prove)}{proof(state) ~|~ theory} \\
\end{matharray}

Arbitrary goal refinement via tactics is considered harmful.  Properly, the
Isar framework admits proof methods to be invoked in two places only.
\begin{enumerate}
\item An \emph{initial} refinement step $\PROOF{m@1}$ reduces a newly stated
  goal to a number of sub-goals that are to be solved later.  Facts are passed
  to $m@1$ for forward chaining, if so indicated by $proof(chain)$ mode.
  
\item A \emph{terminal} conclusion step $\QED{m@2}$ is intended to solve
  remaining goals.  No facts are passed to $m@2$.
\end{enumerate}

The only other proper way to affect pending goals is by $\SHOWNAME$, which
involves an explicit statement of what is to be solved.

\medskip

Also note that initial proof methods should either solve the goal completely,
or constitute some well-understood reduction to new sub-goals.  Arbitrary
automatic proof tools that are prone leave a large number of badly structured
sub-goals are no help in continuing the proof document in any intelligible
way.

\medskip

Unless given explicitly by the user, the default initial method is ``$rule$'',
which applies a single standard elimination or introduction rule according to
the topmost symbol involved.  There is no separate default terminal method.
Any remaining goals are always solved by assumption in the very last step.

\begin{rail}
  'proof' interest? meth? comment?
  ;
  'qed' meth? comment?
  ;
  'by' meth meth? comment?
  ;
  ('.' | '..' | 'sorry') comment?
  ;

  meth: method interest?
  ;
\end{rail}

\begin{descr}
\item [$\PROOF{m@1}$] refines the goal by proof method $m@1$; facts for
  forward chaining are passed if so indicated by $proof(chain)$ mode.
\item [$\QED{m@2}$] refines any remaining goals by proof method $m@2$ and
  concludes the sub-proof by assumption.  If the goal had been $\SHOWNAME$ (or
  $\THUSNAME$), some pending sub-goal is solved as well by the rule resulting
  from the result \emph{exported} into the enclosing goal context.  Thus
  $\QEDNAME$ may fail for two reasons: either $m@2$ fails, or the resulting
  rule does not fit to any pending goal\footnote{This includes any additional
    ``strong'' assumptions as introduced by $\ASSUMENAME$.} of the enclosing
  context.  Debugging such a situation might involve temporarily changing
  $\SHOWNAME$ into $\HAVENAME$, or weakening the local context by replacing
  some occurrences of $\ASSUMENAME$ by $\PRESUMENAME$.
\item [$\BYY{m@1}{m@2}$] is a \emph{terminal proof}\index{proof!terminal}; it
  abbreviates $\PROOF{m@1}~\QED{m@2}$, with backtracking across both methods,
  though.  Debugging an unsuccessful $\BYY{m@1}{m@2}$ commands might be done
  by expanding its definition; in many cases $\PROOF{m@1}$ is already
  sufficient to see what is going wrong.
\item [``$\DDOT$''] is a \emph{default proof}\index{proof!default}; it
  abbreviates $\BY{rule}$.
\item [``$\DOT$''] is a \emph{trivial proof}\index{proof!trivial}; it
  abbreviates $\BY{this}$.
\item [$\SORRY$] is a \emph{fake proof}\index{proof!fake}; provided that the
  \texttt{quick_and_dirty} flag is enabled, $\SORRY$ pretends to solve the
  goal without further ado.  Of course, the result would be a fake theorem
  only, involving some oracle in its internal derivation object (this is
  indicated as ``$[!]$'' in the printed result).  The main application of
  $\SORRY$ is to support experimentation and top-down proof development.
\end{descr}


\subsection{Fundamental methods and attributes}\label{sec:pure-meth-att}

The following proof methods and attributes refer to basic logical operations
of Isar.  Further methods and attributes are provided by several generic and
object-logic specific tools and packages (see chapters \ref{ch:gen-tools} and
\ref{ch:hol-tools}).

\indexisarmeth{assumption}\indexisarmeth{this}\indexisarmeth{rule}\indexisarmeth{$-$}
\indexisaratt{intro}\indexisaratt{elim}\indexisaratt{dest}
\indexisaratt{OF}\indexisaratt{of}
\begin{matharray}{rcl}
  assumption & : & \isarmeth \\
  this & : & \isarmeth \\
  rule & : & \isarmeth \\
  - & : & \isarmeth \\
  OF & : & \isaratt \\
  of & : & \isaratt \\
  intro & : & \isaratt \\
  elim & : & \isaratt \\
  dest & : & \isaratt \\
  delrule & : & \isaratt \\
\end{matharray}

\begin{rail}
  'rule' thmrefs?
  ;
  'OF' thmrefs
  ;
  'of' (inst * ) ('concl' ':' (inst * ))?
  ;

  inst: underscore | term
  ;
\end{rail}

\begin{descr}
\item [$assumption$] solves some goal by a single assumption step.  Any facts
  given (${} \le 1$) are guaranteed to participate in the refinement.  Recall
  that $\QEDNAME$ (see \S\ref{sec:proof-steps}) already concludes any
  remaining sub-goals by assumption.
\item [$this$] applies all of the current facts directly as rules.  Recall
  that ``$\DOT$'' (dot) abbreviates $\BY{this}$.
\item [$rule~\vec a$] applies some rule given as argument in backward manner;
  facts are used to reduce the rule before applying it to the goal.  Thus
  $rule$ without facts is plain \emph{introduction}, while with facts it
  becomes \emph{elimination}.
  
  When no arguments are given, the $rule$ method tries to pick appropriate
  rules automatically, as declared in the current context using the $intro$,
  $elim$, $dest$ attributes (see below).  This is the default behavior of
  $\PROOFNAME$ and ``$\DDOT$'' (double-dot) steps (see
  \S\ref{sec:proof-steps}).
\item [``$-$''] does nothing but insert the forward chaining facts as premises
  into the goal.  Note that command $\PROOFNAME$ without any method actually
  performs a single reduction step using the $rule$ method; thus a plain
  \emph{do-nothing} proof step would be $\PROOF{-}$ rather than $\PROOFNAME$
  alone.
\item [$OF~\vec a$] applies some theorem to given rules $\vec a$ (in
  parallel).  This corresponds to the \texttt{MRS} operator in ML
  \cite[\S5]{isabelle-ref}, but note the reversed order.  Positions may be
  skipped by including ``$\_$'' (underscore) as argument.
\item [$of~\vec t$] performs positional instantiation.  The terms $\vec t$ are
  substituted for any schematic variables occurring in a theorem from left to
  right; ``\texttt{_}'' (underscore) indicates to skip a position.  Arguments
  following a ``$concl\colon$'' specification refer to positions of the
  conclusion of a rule.
\item [$intro$, $elim$, and $dest$] declare introduction, elimination, and
  destruct rules, respectively.  Note that the classical reasoner (see
  \S\ref{sec:classical-basic}) introduces different versions of these
  attributes, and the $rule$ method, too.  In object-logics with classical
  reasoning enabled, the latter version should be used all the time to avoid
  confusion!
\item [$delrule$] undeclares introduction or elimination rules.
\end{descr}


\subsection{Term abbreviations}\label{sec:term-abbrev}

\indexisarcmd{let}
\begin{matharray}{rcl}
  \isarcmd{let} & : & \isartrans{proof(state)}{proof(state)} \\
  \isarkeyword{is} & : & syntax \\
\end{matharray}

Abbreviations may be either bound by explicit $\LET{p \equiv t}$ statements,
or by annotating assumptions or goal statements with a list of patterns
$\ISS{p@1\;\dots}{p@n}$.  In both cases, higher-order matching is invoked to
bind extra-logical term variables, which may be either named schematic
variables of the form $\Var{x}$, or nameless dummies ``\texttt{_}''
(underscore).\indexisarvar{_@\texttt{_}} Note that in the $\LETNAME$ form the
patterns occur on the left-hand side, while the $\ISNAME$ patterns are in
postfix position.

Polymorphism of term bindings is handled in Hindley-Milner style, as in ML.
Type variables referring to local assumptions or open goal statements are
\emph{fixed}, while those of finished results or bound by $\LETNAME$ may occur
in \emph{arbitrary} instances later.  Even though actual polymorphism should
be rarely used in practice, this mechanism is essential to achieve proper
incremental type-inference, as the user proceeds to build up the Isar proof
text.

\medskip

Term abbreviations are quite different from actual local definitions as
introduced via $\DEFNAME$ (see \S\ref{sec:proof-context}).  The latter are
visible within the logic as actual equations, while abbreviations disappear
during the input process just after type checking.  Also note that $\DEFNAME$
does not support polymorphism.

\begin{rail}
  'let' ((term + 'as') '=' term comment? + 'and')
  ;  
\end{rail}

The syntax of $\ISNAME$ patterns follows \railnonterm{termpat} or
\railnonterm{proppat} (see \S\ref{sec:term-pats}).

\begin{descr}
\item [$\LET{\vec p = \vec t}$] binds any text variables in patters $\vec p$
  by simultaneous higher-order matching against terms $\vec t$.
\item [$\IS{\vec p}$] resembles $\LETNAME$, but matches $\vec p$ against the
  preceding statement.  Also note that $\ISNAME$ is not a separate command,
  but part of others (such as $\ASSUMENAME$, $\HAVENAME$ etc.).
\end{descr}

A few \emph{automatic} term abbreviations\index{term abbreviations} for goals
and facts are available as well.  For any open goal,
$\Var{thesis_prop}$\indexisarvar{thesis-prop} refers to the full proposition
(which may be a rule), $\Var{thesis_concl}$\indexisarvar{thesis-concl} to its
(atomic) conclusion, and $\Var{thesis}$\indexisarvar{thesis} to its
object-level statement.  The latter two abstract over any meta-level
parameters.

Fact statements resulting from assumptions or finished goals are bound as
$\Var{this_prop}$\indexisarvar{this-prop},
$\Var{this_concl}$\indexisarvar{this-concl}, and
$\Var{this}$\indexisarvar{this}, similar to $\Var{thesis}$ above.  In case
$\Var{this}$ refers to an object-logic statement that is an application
$f(t)$, then $t$ is bound to the special text variable
``$\dots$''\indexisarvar{\dots} (three dots).  The canonical application of
the latter are calculational proofs (see \S\ref{sec:calculation}).


\subsection{Block structure}

\indexisarcmd{next}\indexisarcmd{\{\{}\indexisarcmd{\}\}}
\begin{matharray}{rcl}
  \NEXT & : & \isartrans{proof(state)}{proof(state)} \\
  \BG & : & \isartrans{proof(state)}{proof(state)} \\
  \EN & : & \isartrans{proof(state)}{proof(state)} \\
\end{matharray}

While Isar is inherently block-structured, opening and closing blocks is
mostly handled rather casually, with little explicit user-intervention.  Any
local goal statement automatically opens \emph{two} blocks, which are closed
again when concluding the sub-proof (by $\QEDNAME$ etc.).  Sections of
different context within a sub-proof may be switched via $\NEXT$, which is
just a single block-close followed by block-open again.  Thus the effect of
$\NEXT$ to reset the local proof context. There is no goal focus involved
here!

For slightly more advanced applications, there are explicit block parentheses
as well.  These typically achieve a stronger forward style of reasoning.

\begin{descr}
\item [$\NEXT$] switches to a fresh block within a sub-proof, resetting the
  local context to the initial one.
\item [$\isarkeyword{\{\{}$ and $\isarkeyword{\}\}}$] explicitly open and
  close blocks.  Any current facts pass through ``$\isarkeyword{\{\{}$''
  unchanged, while ``$\isarkeyword{\}\}}$'' causes any result to be
  \emph{exported} into the enclosing context.  Thus fixed variables are
  generalized, assumptions discharged, and local definitions unfolded (cf.\ 
  \S\ref{sec:proof-context}).  There is no difference of $\ASSUMENAME$ and
  $\PRESUMENAME$ in this mode of forward reasoning --- in contrast to plain
  backward reasoning with the result exported at $\SHOWNAME$ time.
\end{descr}


\subsection{Emulating tactic scripts}\label{sec:tactical-proof}

The following elements emulate unstructured tactic scripts to some extent.
While these are anathema for writing proper Isar proof documents, they might
come in handy for interactive exploration and debugging, or even actual
tactical proof within new-style theories (to benefit from document
preparation, for example).

\indexisarcmd{apply}\indexisarcmd{apply-end}
\indexisarcmd{defer}\indexisarcmd{prefer}\indexisarcmd{back}
\indexisarmeth{tactic}
\indexisarmeth{res-inst-tac}\indexisarmeth{eres-inst-tac}
\indexisarmeth{dres-inst-tac}\indexisarmeth{forw-inst-tac}
\indexisarmeth{subgoal-tac}
\begin{matharray}{rcl}
  \isarcmd{apply}^* & : & \isartrans{proof(prove)}{proof(prove)} \\
  \isarcmd{apply_end}^* & : & \isartrans{proof(state)}{proof(state)} \\
  \isarcmd{defer}^* & : & \isartrans{proof}{proof} \\
  \isarcmd{prefer}^* & : & \isartrans{proof}{proof} \\
  \isarcmd{back}^* & : & \isartrans{proof}{proof} \\
  tactic^* & : & \isarmeth \\
  res_inst_tac^* & : & \isarmeth \\
  eres_inst_tac^* & : & \isarmeth \\
  dres_inst_tac^* & : & \isarmeth \\
  forw_inst_tac^* & : & \isarmeth \\
  subgoal_tac^* & : & \isarmeth \\
\end{matharray}

\railalias{applyend}{apply\_end}
\railterm{applyend}

\railalias{resinsttac}{res\_inst\_tac}
\railterm{resinsttac}

\railalias{eresinsttac}{eres\_inst\_tac}
\railterm{eresinsttac}

\railalias{dresinsttac}{dres\_inst\_tac}
\railterm{dresinsttac}

\railalias{forwinsttac}{forw\_inst\_tac}
\railterm{forwinsttac}

\railalias{subgoaltac}{subgoal\_tac}
\railterm{subgoaltac}

\begin{rail}
  'apply' method
  ;
  applyend method
  ;
  'defer' nat?
  ;
  'prefer' nat
  ;
  'tactic' text
  ;
  ( resinsttac | eresinsttac | dresinsttac | forwinsttac ) goalspec? ((name '=' term) + 'and')
  ;
  subgoaltac goalspec? prop
  ;
\end{rail}

\begin{descr}
\item [$\isarkeyword{apply}~(m)$] applies proof method $m$ in initial
  position, but unlike $\PROOFNAME$ it retains ``$proof(prove)$'' mode.  Thus
  consecutive method applications may be given just as in tactic scripts.  In
  order to complete the proof properly, any of the actual structured proof
  commands (e.g.\ ``$\DOT$'') has to be given eventually.
  
  Facts are passed to $m$ as indicated by the goal's forward-chain mode.
  Common use of $\isarkeyword{apply}$ would be in a purely backward manner,
  though.
\item [$\isarkeyword{apply_end}~(m)$] applies proof method $m$ as if in
  terminal position.  Basically, this simulates a multi-step tactic script for
  $\QEDNAME$, but may be given anywhere within the proof body.
  
  No facts are passed to $m$.  Furthermore, the static context is that of the
  enclosing goal (as for actual $\QEDNAME$).  Thus the proof method may not
  refer to any assumptions introduced in the current body, for example.
\item [$\isarkeyword{defer}~n$ and $\isarkeyword{prefer}~n$] shuffle the list
  of pending goals: $defer$ puts off goal $n$ to the end of the list ($n = 1$
  by default), while $prefer$ brings goal $n$ to the top.
\item [$\isarkeyword{back}$] does back-tracking over the result sequence of
  the latest proof command.\footnote{Unlike the ML function \texttt{back}
    \cite{isabelle-ref}, the Isar command does not search upwards for further
    branch points.} Basically, any proof command may return multiple results.
\item [$tactic~text$] produces a proof method from any ML text of type
  \texttt{tactic}.  Apart from the usual ML environment and the current
  implicit theory context, the ML code may refer to the following locally
  bound values:
%%FIXME ttbox produces too much trailing space (why?)
{\footnotesize\begin{verbatim}
val ctxt  : Proof.context
val facts : thm list
val thm   : string -> thm
val thms  : string -> thm list
\end{verbatim}}
  Here \texttt{ctxt} refers to the current proof context, \texttt{facts}
  indicates any current facts for forward-chaining, and
  \texttt{thm}~/~\texttt{thms} retrieve named facts (including global
  theorems) from the context.
\item [$res_inst_tac$ etc.] do resolution of rules with explicit
  instantiation.  This works the same way as the corresponding ML tactics, see
  \cite[\S3]{isabelle-ref}.
  
  It is very important to note that the instantiations are read and
  type-checked according to the dynamic goal state, rather than the static
  proof context!  In particular, locally fixed variables and term
  abbreviations may not be included in the term specifications.
\item [$subgoal_tac~\phi$] emulates the ML tactic of the same name, see
  \cite[\S3]{isabelle-ref}.  Syntactically, the given proposition is handled
  as the instantiations in $res_inst_tac$ etc.
  
  Note that the proper Isar command $\PRESUMENAME$ achieves a similar effect
  as $subgoal_tac$.
\end{descr}


\subsection{Meta-linguistic features}

\indexisarcmd{oops}
\begin{matharray}{rcl}
  \isarcmd{oops} & : & \isartrans{proof}{theory} \\
\end{matharray}

The $\OOPS$ command discontinues the current proof attempt, while considering
the partial proof text as properly processed.  This is conceptually quite
different from ``faking'' actual proofs via $\SORRY$ (see
\S\ref{sec:proof-steps}): $\OOPS$ does not observe the proof structure at all,
but goes back right to the theory level.  Furthermore, $\OOPS$ does not
produce any result theorem --- there is no claim to be able to complete the
proof anyhow.

A typical application of $\OOPS$ is to explain Isar proofs \emph{within} the
system itself, in conjunction with the document preparation tools of Isabelle
described in \cite{isabelle-sys}.  Thus partial or even wrong proof attempts
can be discussed in a logically sound manner.  Note that the Isabelle {\LaTeX}
macros can be easily adapted to print something like ``$\dots$'' instead of an
``$\OOPS$'' keyword.

\medskip The $\OOPS$ command is undoable, unlike $\isarkeyword{kill}$ (see
\S\ref{sec:history}).  The effect is to get back to the theory \emph{before}
the opening of the proof.


\section{Other commands}

\subsection{Diagnostics}\label{sec:diag}

\indexisarcmd{pr}\indexisarcmd{thm}\indexisarcmd{term}\indexisarcmd{prop}\indexisarcmd{typ}
\indexisarcmd{print-facts}\indexisarcmd{print-binds}
\begin{matharray}{rcl}
  \isarcmd{help}^* & : & \isarkeep{\cdot} \\
  \isarcmd{pr}^* & : & \isarkeep{\cdot} \\
  \isarcmd{thm}^* & : & \isarkeep{theory~|~proof} \\
  \isarcmd{term}^* & : & \isarkeep{theory~|~proof} \\
  \isarcmd{prop}^* & : & \isarkeep{theory~|~proof} \\
  \isarcmd{typ}^* & : & \isarkeep{theory~|~proof} \\
  \isarcmd{print_facts}^* & : & \isarkeep{proof} \\
  \isarcmd{print_binds}^* & : & \isarkeep{proof} \\
\end{matharray}

These commands are not part of the actual Isabelle/Isar syntax, but assist
interactive development.  Also note that $undo$ does not apply here, since the
theory or proof configuration is not changed.

\begin{rail}
  'pr' modes? nat?
  ;
  'thm' modes? thmrefs
  ;
  'term' modes? term
  ;
  'prop' modes? prop
  ;
  'typ' modes? type
  ;

  modes: '(' (name + ) ')'
  ;
\end{rail}

\begin{descr}
\item [$\isarkeyword{help}$] prints a list of available language elements.
  Note that methods and attributes depend on the current theory context.
\item [$\isarkeyword{pr}~n$] prints the current top-level state, i.e.\ the
  theory identifier or proof state.  The latter includes the proof context,
  current facts and goals.  The optional argument $n$ affects the implicit
  limit of goals to be displayed, which is initially 10.  Omitting the limit
  leaves the value unchanged.
\item [$\isarkeyword{thm}~\vec a$] retrieves theorems from the current theory
  or proof context.  Note that any attributes included in the theorem
  specifications are applied to a temporary context derived from the current
  theory or proof; the result is discarded, i.e.\ attributes involved in $\vec
  a$ do not have any permanent effect.
\item [$\isarkeyword{term}~t$, $\isarkeyword{prop}~\phi$] read, type-check and
  print terms or propositions according to the current theory or proof
  context; the inferred type of $t$ is output as well.  Note that these
  commands are also useful in inspecting the current environment of term
  abbreviations.
\item [$\isarkeyword{typ}~\tau$] reads and prints types of the meta-logic
  according to the current theory or proof context.
\item [$\isarkeyword{print_facts}$] prints any named facts of the current
  context, including assumptions and local results.
\item [$\isarkeyword{print_binds}$] prints all term abbreviations present in
  the context.
\end{descr}

The basic diagnostic commands above admit a list of $modes$ to be specified,
which is appended to the current print mode (see also \cite{isabelle-ref}).
Thus the output behavior may be modified according particular print mode
features.

For example, $\isarkeyword{pr}~(latex~xsymbols~symbols)$ would print the
current proof state with mathematical symbols and special characters
represented in {\LaTeX} source, according to the Isabelle style
\cite{isabelle-sys}.  The resulting text can be directly pasted into a
\verb,\begin{isabelle},\dots\verb,\end{isabelle}, environment.  Note that
$\isarkeyword{pr}~(latex)$ is sufficient to achieve the same output, if the
current Isabelle session has the other modes already activated, say due to
some particular user interface configuration such as Proof~General
\cite{proofgeneral,Aspinall:TACAS:2000} with X-Symbol mode \cite{x-symbol}.


\subsection{History commands}\label{sec:history}

\indexisarcmd{undo}\indexisarcmd{redo}\indexisarcmd{kill}
\begin{matharray}{rcl}
  \isarcmd{undo}^{{*}{*}} & : & \isarkeep{\cdot} \\
  \isarcmd{redo}^{{*}{*}} & : & \isarkeep{\cdot} \\
  \isarcmd{kill}^{{*}{*}} & : & \isarkeep{\cdot} \\
\end{matharray}

The Isabelle/Isar top-level maintains a two-stage history, for theory and
proof state transformation.  Basically, any command can be undone using
$\isarkeyword{undo}$, excluding mere diagnostic elements.  Its effect may be
revoked via $\isarkeyword{redo}$, unless the corresponding the
$\isarkeyword{undo}$ step has crossed the beginning of a proof or theory.  The
$\isarkeyword{kill}$ command aborts the current history node altogether,
discontinuing a proof or even the whole theory.  This operation is \emph{not}
undoable.

\begin{warn}
  History commands should never be used with user interfaces such as
  Proof~General \cite{proofgeneral,Aspinall:TACAS:2000}, which takes care of
  stepping forth and back itself.  Interfering by manual $\isarkeyword{undo}$,
  $\isarkeyword{redo}$, or even $\isarkeyword{kill}$ commands would quickly
  result in utter confusion.
\end{warn}

%FIXME remove
% \begin{descr}
% \item [$\isarkeyword{undo}$] revokes the latest state-transforming command.
% \item [$\isarkeyword{redo}$] undos the latest $\isarkeyword{undo}$.
% \item [$\isarkeyword{kill}$] aborts the current history level.
% \end{descr}


\subsection{System operations}

\indexisarcmd{cd}\indexisarcmd{pwd}\indexisarcmd{use-thy}\indexisarcmd{use-thy-only}
\indexisarcmd{update-thy}\indexisarcmd{update-thy-only}
\begin{matharray}{rcl}
  \isarcmd{cd}^* & : & \isarkeep{\cdot} \\
  \isarcmd{pwd}^* & : & \isarkeep{\cdot} \\
  \isarcmd{use_thy}^* & : & \isarkeep{\cdot} \\
  \isarcmd{use_thy_only}^* & : & \isarkeep{\cdot} \\
  \isarcmd{update_thy}^* & : & \isarkeep{\cdot} \\
  \isarcmd{update_thy_only}^* & : & \isarkeep{\cdot} \\
\end{matharray}

\begin{descr}
\item [$\isarkeyword{cd}~name$] changes the current directory of the Isabelle
  process.
\item [$\isarkeyword{pwd}~$] prints the current working directory.
\item [$\isarkeyword{use_thy}$, $\isarkeyword{use_thy_only}$,
  $\isarkeyword{update_thy}$, $\isarkeyword{update_thy_only}$] load some
  theory given as $name$ argument.  These commands are basically the same as
  the corresponding ML functions\footnote{The ML versions also change the
    implicit theory context to that of the theory loaded.}  (see also
  \cite[\S1,\S6]{isabelle-ref}).  Note that both the ML and Isar versions may
  load new- and old-style theories alike.
\end{descr}

These system commands are scarcely used when working with the Proof~General
interface, since loading of theories is done fully transparently.


%%% Local Variables: 
%%% mode: latex
%%% TeX-master: "isar-ref"
%%% End: 

%% $Id$
\chapter{Simplification} \label{simp-chap}
\index{simplification|(}

This chapter describes Isabelle's generic simplification package, which
provides a suite of simplification tactics.  This rewriting package is less
general than its predecessor --- it works only for the equality relation,
not arbitrary preorders --- but it is fast and flexible.  It performs
conditional and unconditional rewriting and uses contextual information
(``local assumptions'').  It provides a few general hooks, which can
provide automatic case splits during rewriting, for example.  The
simplifier is set up for many of Isabelle's logics: {\tt FOL}, {\tt ZF},
{\tt LCF} and {\tt HOL}.


\section{Simplification sets}\index{simplification sets} 
The simplification tactics are controlled by {\bf simpsets}.  These consist
of five components: rewrite rules, congruence rules, the subgoaler, the
solver and the looper.  Normally, the simplifier is set up with sensible
defaults, so that most simplifier calls specify only rewrite rules.
Sophisticated usage of the other components can be highly effective, but
most users should never worry about them.

\subsection{Rewrite rules}\index{rewrite rules}
Rewrite rules are theorems like $Suc(\Var{m}) + \Var{n} = \Var{m} +
Suc(\Var{n})$, $\Var{P}\conj\Var{P} \bimp \Var{P}$, or $\Var{A} \union \Var{B}
\equiv \{x.x\in A \disj x\in B\}$.  {\bf Conditional} rewrites such as
$\Var{m}<\Var{n} \Imp \Var{m}/\Var{n} = 0$ are permitted; the conditions
can be arbitrary terms.  The infix operation \ttindex{addsimps} adds new
rewrite rules, while \ttindex{delsimps} deletes rewrite rules from a
simpset.

Theorems added via \ttindex{addsimps} need not be equalities to start with.
Each simpset contains a (user-definable) function for extracting equalities
from arbitrary theorems.  For example $\neg(x\in \{\})$ could be turned
into $x\in \{\} \equiv False$.  This function can be set with
\ttindex{setmksimps} but only the definer of a logic should need to do
this.  Exceptionally, one may want to install a selective version of
\ttindex{mksimps} in order to filter out looping rewrite rules arising from
local assumptions (see below).

Internally, all rewrite rules are translated into meta-equalities:
theorems with conclusion $lhs \equiv rhs$.  To this end every simpset contains
a function of type \verb$thm -> thm list$ to extract a list
of meta-equalities from a given theorem.

\begin{warn}\index{rewrite rules}
  The left-hand side of a rewrite rule must look like a first-order term:
  after eta-contraction, none of its unknowns should have arguments.  Hence
  ${\Var{i}+(\Var{j}+\Var{k})} = {(\Var{i}+\Var{j})+\Var{k}}$ and $\neg(\forall
  x.\Var{P}(x)) \bimp (\exists x.\neg\Var{P}(x))$ are acceptable, whereas
  $\Var{f}(\Var{x})\in {\tt range}(\Var{f}) = True$ is not.  However, you can
  replace the offending subterms by new variables and conditions: $\Var{y} =
  \Var{f}(\Var{x}) \Imp \Var{y}\in {\tt range}(\Var{f}) = True$ is again
  acceptable.
\end{warn}

\subsection {Congruence rules}\index{congruence rules}
Congruence rules are meta-equalities of the form
\[ \List{\dots} \Imp
   f(\Var{x@1},\ldots,\Var{x@n}) \equiv f(\Var{y@1},\ldots,\Var{y@n}).
\]
They control the simplification of the arguments of certain constants.  For
example, some arguments can be simplified under additional assumptions:
\[ \List{\Var{P@1} \bimp \Var{Q@1};\; \Var{Q@1} \Imp \Var{P@2} \bimp \Var{Q@2}}
   \Imp (\Var{P@1} \imp \Var{P@2}) \equiv (\Var{Q@1} \imp \Var{Q@2})
\]
This rule assumes $Q@1$ and any rewrite rules it implies, while
simplifying~$P@2$.  Such ``local'' assumptions are effective for rewriting
formulae such as $x=0\imp y+x=y$.  The next example makes similar use of
such contextual information in bounded quantifiers:
\begin{eqnarray*}
  &&\List{\Var{A}=\Var{B};\; 
          \Forall x. x\in \Var{B} \Imp \Var{P}(x) = \Var{Q}(x)} \Imp{} \\
 &&\qquad\qquad
    (\forall x\in \Var{A}.\Var{P}(x)) = (\forall x\in \Var{B}.\Var{Q}(x))
\end{eqnarray*}
This congruence rule supplies contextual information for simplifying the
arms of a conditional expressions:
\[ \List{\Var{p}=\Var{q};~ \Var{q} \Imp \Var{a}=\Var{c};~
         \neg\Var{q} \Imp \Var{b}=\Var{d}} \Imp
   if(\Var{p},\Var{a},\Var{b}) \equiv if(\Var{q},\Var{c},\Var{d})
\]

A congruence rule can also suppress simplification of certain arguments.
Here is an alternative congruence rule for conditional expressions:
\[ \Var{p}=\Var{q} \Imp
   if(\Var{p},\Var{a},\Var{b}) \equiv if(\Var{q},\Var{a},\Var{b})
\]
Only the first argument is simplified; the others remain unchanged.
This can make simplification much faster, but may require an extra case split
to prove the goal.  

Congruence rules are added using \ttindexbold{addeqcongs}.  Their conclusion
must be a meta-equality, as in the examples above.  It is more
natural to derive the rules with object-logic equality, for example
\[ \List{\Var{P@1} \bimp \Var{Q@1};\; \Var{Q@1} \Imp \Var{P@2} \bimp \Var{Q@2}}
   \Imp (\Var{P@1} \imp \Var{P@2}) \bimp (\Var{Q@1} \imp \Var{Q@2}),
\]
Each object-logic should define an operator called \ttindex{addcongs} that
expects object-equalities and translates them into meta-equalities.

\subsection{The subgoaler}
The subgoaler is the tactic used to solve subgoals arising out of
conditional rewrite rules or congruence rules.  The default should be
simplification itself.  Occasionally this strategy needs to be changed.  For
example, if the premise of a conditional rule is an instance of its
conclusion, as in $Suc(\Var{m}) < \Var{n} \Imp \Var{m} < \Var{n}$, the
default strategy could loop.

The subgoaler can be set explicitly with \ttindex{setsubgoaler}.  For
example, the subgoaler
\begin{ttbox}
fun subgoal_tac ss = resolve_tac (prems_of_ss ss) ORELSE' 
                     asm_simp_tac ss;
\end{ttbox}
tries to solve the subgoal with one of the premises and calls
simplification only if that fails; here {\tt prems_of_ss} extracts the
current premises from a simpset.

\subsection{The solver}
The solver is a tactic that attempts to solve a subgoal after
simplification.  Typically it just proves trivial subgoals such as {\tt
  True} and $t=t$; it could use sophisticated means such as
\verb$fast_tac$.  The solver is set using \ttindex{setsolver}.

The tactic is presented with the full goal, including the asssumptions.
Hence it can use those assumptions (say by calling {\tt assume_tac}) even
inside {\tt simp_tac}, which otherwise does not use assumptions.  The
solver is also supplied a list of theorems, namely assumptions that hold in
the local context.

\begin{warn}
  Rewriting does not instantiate unknowns.  Trying to rewrite $a\in
  \Var{A}$ with the rule $\Var{x}\in \{\Var{x}\}$ leads nowhere.  The
  solver, however, is an arbitrary tactic and may instantiate unknowns as
  it pleases.  This is the only way the simplifier can handle a conditional
  rewrite rule whose condition contains extra variables.
\end{warn}

\begin{warn}
  If you want to supply your own subgoaler or solver, read on.  The subgoaler
  is also used to solve the premises of congruence rules, which are usually
  of the form $s = \Var{x}$, where $s$ needs to be simplified and $\Var{x}$
  needs to be instantiated with the result. Hence the subgoaler should call
  the simplifier at some point. The simplifier will then call the solver,
  which must therefore be prepared to solve goals of the form $t = \Var{x}$,
  usually by reflexivity. In particular, reflexivity should be tried before
  any of the fancy tactics like {\tt fast_tac}. It may even happen that, due
  to simplification, the subgoal is no longer an equality. For example $False
  \bimp \Var{Q}$ could be rewritten to $\neg\Var{Q}$, in which case the
  solver must also try resolving with the theorem $\neg False$.

  If the simplifier aborts with the message {\tt Failed congruence proof!},
  it is due to the subgoaler or solver that failed to prove a premise of a
  congruence rule.
\end{warn}

\subsection{The looper}
The looper is a tactic that is applied after simplification, in case the
solver failed to solve the simplified goal.  If the looper succeeds, the
simplification process is started all over again.  Each of the subgoals
generated by the looper is attacked in turn, in reverse order.  A
typical looper is case splitting: the expansion of a conditional.  Another
possibility is to apply an elimination rule on the assumptions.  More
adventurous loopers could start an induction.  The looper is set with 
\ttindex{setloop}.


\begin{figure}
\indexbold{*SIMPLIFIER}
\indexbold{*simpset}
\indexbold{*simp_tac}
\indexbold{*asm_simp_tac}
\indexbold{*asm_full_simp_tac}
\indexbold{*addeqcongs}
\indexbold{*addsimps}
\indexbold{*delsimps}
\indexbold{*empty_ss}
\indexbold{*merge_ss}
\indexbold{*setsubgoaler}
\indexbold{*setsolver}
\indexbold{*setloop}
\indexbold{*setmksimps}
\indexbold{*prems_of_ss}
\indexbold{*rep_ss}
\begin{ttbox}
infix addsimps addeqcongs delsimps
      setsubgoaler setsolver setloop setmksimps;

signature SIMPLIFIER =
sig
  type simpset
  val simp_tac:          simpset -> int -> tactic
  val asm_simp_tac:      simpset -> int -> tactic
  val asm_full_simp_tac: simpset -> int -> tactic\smallskip
  val addeqcongs:   simpset * thm list -> simpset
  val addsimps:     simpset * thm list -> simpset
  val delsimps:     simpset * thm list -> simpset
  val empty_ss:     simpset
  val merge_ss:     simpset * simpset -> simpset
  val setsubgoaler: simpset * (simpset -> int -> tactic) -> simpset
  val setsolver:    simpset * (thm list -> int -> tactic) -> simpset
  val setloop:      simpset * (int -> tactic) -> simpset
  val setmksimps:   simpset * (thm -> thm list) -> simpset
  val prems_of_ss:  simpset -> thm list
  val rep_ss:       simpset -> \{simps: thm list, congs: thm list\}
end;
\end{ttbox}
\caption{The signature \ttindex{SIMPLIFIER}} \label{SIMPLIFIER}
\end{figure}


\section{The simplification tactics} \label{simp-tactics}
\index{simplification!tactics|bold}
\index{tactics!simplification|bold}

The actual simplification work is performed by the following tactics.  The
rewriting strategy is strictly bottom up, except for congruence rules, which
are applied while descending into a term.  Conditions in conditional rewrite
rules are solved recursively before the rewrite rule is applied.

There are three basic simplification tactics:
\begin{description}
\item[\ttindexbold{simp_tac} $ss$ $i$] simplifies subgoal~$i$ using the rules
  in~$ss$.  It may solve the subgoal completely if it has become trivial,
  using the solver.
  
\item[\ttindexbold{asm_simp_tac}] is like \verb$simp_tac$, but also uses
  assumptions as additional rewrite rules.

\item[\ttindexbold{asm_full_simp_tac}] is like \verb$asm_simp_tac$, but also
  simplifies the assumptions one by one, using each assumption in the
  simplification of the following ones.
\end{description}
Using the simplifier effectively may take a bit of experimentation.  The
tactics can be traced with the ML command \verb$trace_simp := true$.  To
remind yourself of what is in a simpset, use the function \verb$rep_ss$ to
return its simplification and congruence rules.

\section{Examples using the simplifier}
\index{simplification!example}
Assume we are working within {\tt FOL} and that
\begin{description}
\item[\tt Nat.thy] is a theory including the constants $0$, $Suc$ and $+$,
\item[\tt add_0] is the rewrite rule $0+n = n$,
\item[\tt add_Suc] is the rewrite rule $Suc(m)+n = Suc(m+n)$,
\item[\tt induct] is the induction rule
$\List{P(0); \Forall x. P(x)\Imp P(Suc(x))} \Imp P(n)$.
\item[\tt FOL_ss] is a basic simpset for {\tt FOL}.\footnote
{These examples reside on the file {\tt FOL/ex/nat.ML}.} 
\end{description}

We create a simpset for natural numbers by extending~{\tt FOL_ss}:
\begin{ttbox}
val add_ss = FOL_ss addsimps [add_0, add_Suc];
\end{ttbox}
Proofs by induction typically involve simplification.  Here is a proof
that~0 is a right identity:
\begin{ttbox}
goal Nat.thy "m+0 = m";
{\out Level 0}
{\out m + 0 = m}
{\out  1. m + 0 = m}
\end{ttbox}
The first step is to perform induction on the variable~$m$.  This returns a
base case and inductive step as two subgoals:
\begin{ttbox}
by (res_inst_tac [("n","m")] induct 1);
{\out Level 1}
{\out m + 0 = m}
{\out  1. 0 + 0 = 0}
{\out  2. !!x. x + 0 = x ==> Suc(x) + 0 = Suc(x)}
\end{ttbox}
Simplification solves the first subgoal trivially:
\begin{ttbox}
by (simp_tac add_ss 1);
{\out Level 2}
{\out m + 0 = m}
{\out  1. !!x. x + 0 = x ==> Suc(x) + 0 = Suc(x)}
\end{ttbox}
The remaining subgoal requires \ttindex{asm_simp_tac} in order to use the
induction hypothesis as a rewrite rule:
\begin{ttbox}
by (asm_simp_tac add_ss 1);
{\out Level 3}
{\out m + 0 = m}
{\out No subgoals!}
\end{ttbox}

The next proof is similar.
\begin{ttbox}
goal Nat.thy "m+Suc(n) = Suc(m+n)";
{\out Level 0}
{\out m + Suc(n) = Suc(m + n)}
{\out  1. m + Suc(n) = Suc(m + n)}
\end{ttbox}
We again perform induction on~$m$ and get two subgoals:
\begin{ttbox}
by (res_inst_tac [("n","m")] induct 1);
{\out Level 1}
{\out m + Suc(n) = Suc(m + n)}
{\out  1. 0 + Suc(n) = Suc(0 + n)}
{\out  2. !!x. x + Suc(n) = Suc(x + n) ==>}
{\out          Suc(x) + Suc(n) = Suc(Suc(x) + n)}
\end{ttbox}
Simplification solves the first subgoal, this time rewriting two
occurrences of~0:
\begin{ttbox}
by (simp_tac add_ss 1);
{\out Level 2}
{\out m + Suc(n) = Suc(m + n)}
{\out  1. !!x. x + Suc(n) = Suc(x + n) ==>}
{\out          Suc(x) + Suc(n) = Suc(Suc(x) + n)}
\end{ttbox}
Switching tracing on illustrates how the simplifier solves the remaining
subgoal: 
\begin{ttbox}
trace_simp := true;
by (asm_simp_tac add_ss 1);
{\out Rewriting:}
{\out Suc(x) + Suc(n) == Suc(x + Suc(n))}
{\out Rewriting:}
{\out x + Suc(n) == Suc(x + n)}
{\out Rewriting:}
{\out Suc(x) + n == Suc(x + n)}
{\out Rewriting:}
{\out Suc(Suc(x + n)) = Suc(Suc(x + n)) == True}
{\out Level 3}
{\out m + Suc(n) = Suc(m + n)}
{\out No subgoals!}
\end{ttbox}
Many variations are possible.  At Level~1 (in either example) we could have
solved both subgoals at once using the tactical \ttindex{ALLGOALS}:
\begin{ttbox}
by (ALLGOALS (asm_simp_tac add_ss));
{\out Level 2}
{\out m + Suc(n) = Suc(m + n)}
{\out No subgoals!}
\end{ttbox}

\medskip
Here is a conjecture to be proved for an arbitrary function~$f$ satisfying
the law $f(Suc(n)) = Suc(f(n))$:
\begin{ttbox}
val [prem] = goal Nat.thy
    "(!!n. f(Suc(n)) = Suc(f(n))) ==> f(i+j) = i+f(j)";
{\out Level 0}
{\out f(i + j) = i + f(j)}
{\out  1. f(i + j) = i + f(j)}
{\out val prem = "f(Suc(?n)) = Suc(f(?n))}
{\out             [!!n. f(Suc(n)) = Suc(f(n))]" : thm}
\ttbreak
by (res_inst_tac [("n","i")] induct 1);
{\out Level 1}
{\out f(i + j) = i + f(j)}
{\out  1. f(0 + j) = 0 + f(j)}
{\out  2. !!x. f(x + j) = x + f(j) ==> f(Suc(x) + j) = Suc(x) + f(j)}
\end{ttbox}
We simplify each subgoal in turn.  The first one is trivial:
\begin{ttbox}
by (simp_tac add_ss 1);
{\out Level 2}
{\out f(i + j) = i + f(j)}
{\out  1. !!x. f(x + j) = x + f(j) ==> f(Suc(x) + j) = Suc(x) + f(j)}
\end{ttbox}
The remaining subgoal requires rewriting by the premise, so we add it to
{\tt add_ss}:\footnote{The previous
  simplifier required congruence rules for function variables like~$f$ in
  order to simplify their arguments.  The present simplifier can be given
  congruence rules to realize non-standard simplification of a function's
  arguments, but this is seldom necessary.}
\begin{ttbox}
by (asm_simp_tac (add_ss addsimps [prem]) 1);
{\out Level 3}
{\out f(i + j) = i + f(j)}
{\out No subgoals!}
\end{ttbox}


\section{Setting up the simplifier}
\index{simplification!setting up|bold}

Setting up the simplifier for new logics is complicated.  This section
describes how the simplifier is installed for first-order logic; the code
is largely taken from {\tt FOL/simpdata.ML}.

The simplifier and the case splitting tactic, which resides in a separate
file, are not part of Pure Isabelle.  They must be loaded explicitly:
\begin{ttbox}
use "../Provers/simplifier.ML";
use "../Provers/splitter.ML";
\end{ttbox}

Simplification works by reducing various object-equalities to
meta-equality.  It requires axioms stating that equal terms and equivalent
formulae are also equal at the meta-level.  The file {\tt FOL/ifol.thy}
contains the two lines
\begin{ttbox}\indexbold{*eq_reflection}\indexbold{*iff_reflection}
eq_reflection   "(x=y)   ==> (x==y)"
iff_reflection  "(P<->Q) ==> (P==Q)"
\end{ttbox}
Of course, you should only assert such axioms if they are true for your
particular logic.  In Constructive Type Theory, equality is a ternary
relation of the form $a=b\in A$; the type~$A$ determines the meaning of the
equality effectively as a partial equivalence relation.


\subsection{A collection of standard rewrite rules}
The file begins by proving lots of standard rewrite rules about the logical
connectives.  These include cancellation laws and associative laws but
certainly not commutative laws, which would case looping.  To prove the
laws easily, it defines a function that echoes the desired law and then
supplies it the theorem prover for intuitionistic \FOL:
\begin{ttbox}
fun int_prove_fun s = 
 (writeln s;  
  prove_goal IFOL.thy s
   (fn prems => [ (cut_facts_tac prems 1), 
                  (Int.fast_tac 1) ]));
\end{ttbox}
The following rewrite rules about conjunction are a selection of those
proved on {\tt FOL/simpdata.ML}.  Later, these will be supplied to the
standard simpset.
\begin{ttbox}
val conj_rews = map int_prove_fun
 ["P & True <-> P",      "True & P <-> P",
  "P & False <-> False", "False & P <-> False",
  "P & P <-> P",
  "P & ~P <-> False",    "~P & P <-> False",
  "(P & Q) & R <-> P & (Q & R)"];
\end{ttbox}
The file also proves some distributive laws.  As they can cause exponential
blowup, they will not be included in the standard simpset.  Instead they
are merely bound to an \ML{} identifier.
\begin{ttbox}
val distrib_rews  = map int_prove_fun
 ["P & (Q | R) <-> P&Q | P&R", 
  "(Q | R) & P <-> Q&P | R&P",
  "(P | Q --> R) <-> (P --> R) & (Q --> R)"];
\end{ttbox}


\subsection{Functions for preprocessing the rewrite rules}
The next step is to define the function for preprocessing rewrite rules.
This will be installed by calling {\tt setmksimps} below.  Preprocessing
occurs whenever rewrite rules are added, whether by user command or
automatically.  Preprocessing involves extracting atomic rewrites at the
object-level, then reflecting them to the meta-level.

To start, the function {\tt gen_all} strips any meta-level
quantifiers from the front of the given theorem.  Usually there are none
anyway.
\begin{ttbox}
fun gen_all th = forall_elim_vars (#maxidx(rep_thm th)+1) th;
\end{ttbox}
The function {\tt atomize} analyses a theorem in order to extract
atomic rewrite rules.  The head of all the patterns, matched by the
wildcard~{\tt _}, is the coercion function {\tt Trueprop}.
\begin{ttbox}
fun atomize th = case concl_of th of 
    _ $ (Const("op &",_) $ _ $ _)   => atomize(th RS conjunct1) \at
                                       atomize(th RS conjunct2)
  | _ $ (Const("op -->",_) $ _ $ _) => atomize(th RS mp)
  | _ $ (Const("All",_) $ _)        => atomize(th RS spec)
  | _ $ (Const("True",_))           => []
  | _ $ (Const("False",_))          => []
  | _                               => [th];
\end{ttbox}
There are several cases, depending upon the form of the conclusion:
\begin{itemize}
\item Conjunction: extract rewrites from both conjuncts.

\item Implication: convert $P\imp Q$ to the meta-implication $P\Imp Q$ and
  extract rewrites from~$Q$; these will be conditional rewrites with the
  condition~$P$.

\item Universal quantification: remove the quantifier, replacing the bound
  variable by a schematic variable, and extract rewrites from the body.

\item {\tt True} and {\tt False} contain no useful rewrites.

\item Anything else: return the theorem in a singleton list.
\end{itemize}
The resulting theorems are not literally atomic --- they could be
disjunctive, for example --- but are brokwn down as much as possible.  See
the file {\tt ZF/simpdata.ML} for a sophisticated translation of
set-theoretic formulae into rewrite rules.

The simplified rewrites must now be converted into meta-equalities.  The
axiom {\tt eq_reflection} converts equality rewrites, while {\tt
  iff_reflection} converts if-and-only-if rewrites.  The latter possibility
can arise in two other ways: the negative theorem~$\neg P$ is converted to
$P\equiv{\tt False}$, and any other theorem~$\neg P$ is converted to
$P\equiv{\tt True}$.  The rules {\tt iff_reflection_F} and {\tt
  iff_reflection_T} accomplish this conversion.
\begin{ttbox}
val P_iff_F = int_prove_fun "~P ==> (P <-> False)";
val iff_reflection_F = P_iff_F RS iff_reflection;
\ttbreak
val P_iff_T = int_prove_fun "P ==> (P <-> True)";
val iff_reflection_T = P_iff_T RS iff_reflection;
\end{ttbox}
The function {\tt mk_meta_eq} converts a theorem to a meta-equality
using the case analysis described above.
\begin{ttbox}
fun mk_meta_eq th = case concl_of th of
    _ $ (Const("op =",_)$_$_)   => th RS eq_reflection
  | _ $ (Const("op <->",_)$_$_) => th RS iff_reflection
  | _ $ (Const("Not",_)$_)      => th RS iff_reflection_F
  | _                           => th RS iff_reflection_T;
\end{ttbox}
The three functions {\tt gen_all}, {\tt atomize} and {\tt mk_meta_eq} will
be composed together and supplied below to {\tt setmksimps}.


\subsection{Making the initial simpset}
It is time to assemble these items.  We open module {\tt Simplifier} to
gain access to its components.  The infix operator \ttindexbold{addcongs}
handles congruence rules; given a list of theorems, it converts their
conclusions into meta-equalities and passes them to \ttindex{addeqcongs}.
\begin{ttbox}
open Simplifier;
\ttbreak
infix addcongs;
fun ss addcongs congs =
    ss addeqcongs (congs RL [eq_reflection,iff_reflection]);
\end{ttbox}
The list {\tt IFOL_rews} contains the default rewrite rules for first-order
logic.  The first of these is the reflexive law expressed as the
equivalence $(a=a)\bimp{\tt True}$; if we provided it as $a=a$ it would
cause looping.
\begin{ttbox}
val IFOL_rews =
   [refl RS P_iff_T] \at conj_rews \at disj_rews \at not_rews \at 
    imp_rews \at iff_rews \at quant_rews;
\end{ttbox}
The list {\tt triv_rls} contains trivial theorems for the solver.  Any
subgoal that is simplified to one of these will be removed.
\begin{ttbox}
val notFalseI = int_prove_fun "~False";
val triv_rls = [TrueI,refl,iff_refl,notFalseI];
\end{ttbox}

The basic simpset for intuitionistic \FOL{} starts with \ttindex{empty_ss}.
It preprocess rewrites using {\tt gen_all}, {\tt atomize} and {\tt
  mk_meta_eq}.  It solves simplified subgoals using {\tt triv_rls} and
assumptions.  It uses \ttindex{asm_simp_tac} to tackle subgoals of
conditional rewrites.  It takes {\tt IFOL_rews} as rewrite rules.  
Other simpsets built from {\tt IFOL_ss} will inherit these items.
\index{*setmksimps}\index{*setsolver}\index{*setsubgoaler}
\index{*addsimps}\index{*addcongs}
\begin{ttbox}
val IFOL_ss = 
  empty_ss 
  setmksimps (map mk_meta_eq o atomize o gen_all)
  setsolver  (fn prems => resolve_tac (triv_rls \at prems) ORELSE' 
                          assume_tac)
  setsubgoaler asm_simp_tac
  addsimps IFOL_rews
  addcongs [imp_cong];
\end{ttbox}
This simpset takes {\tt imp_cong} as a congruence rule in order to use
contextual information to simplify the conclusions of implications:
\[ \List{\Var{P}\bimp\Var{P'};\; \Var{P'} \Imp \Var{Q}\bimp\Var{Q'}} \Imp
   (\Var{P}\imp\Var{Q}) \bimp (\Var{P'}\imp\Var{Q'})
\]
By adding the congruence rule {\tt conj_cong}, we could obtain a similar
effect for conjunctions.


\subsection{Case splitting}
To set up case splitting, we must prove a theorem of the form shown below
and pass it to \ttindexbold{mk_case_split_tac}.  The tactic
\ttindexbold{split_tac} uses {\tt mk_meta_eq} to convert the splitting
rules to meta-equalities.

\begin{ttbox}
val meta_iffD = 
    prove_goal FOL.thy "[| P==Q; Q |] ==> P"
        (fn [prem1,prem2] => [rewtac prem1, rtac prem2 1])
\ttbreak
fun split_tac splits =
    mk_case_split_tac meta_iffD (map mk_meta_eq splits);
\end{ttbox}
%
The splitter is designed for rules roughly of the form
\[ \Var{P}(if(\Var{Q},\Var{x},\Var{y})) \bimp (\Var{Q} \imp \Var{P}(\Var{x}))
\conj (\lnot\Var{Q} \imp \Var{P}(\Var{y})) 
\] 
where the right-hand side can be anything.  Another example is the
elimination operator (which happens to be called~$split$) for Cartesian
products:
\[ \Var{P}(split(\Var{f},\Var{p})) \bimp (\forall a~b. \Var{p} =
\langle a,b\rangle \imp \Var{P}(\Var{f}(a,b))) 
\] 
Case splits should be allowed only when necessary; they are expensive
and hard to control.  Here is a typical example of use, where {\tt
  expand_if} is the first rule above:
\begin{ttbox}
by (simp_tac (prop_rec_ss setloop (split_tac [expand_if])) 1);
\end{ttbox}



\index{simplification|)}



%% $Id$
\chapter{The Classical Reasoner}\label{chap:classical}
\index{classical reasoner|(}
\newcommand\ainfer[2]{\begin{array}{r@{\,}l}#2\\ \hline#1\end{array}}

Although Isabelle is generic, many users will be working in some extension of
classical first-order logic.  Isabelle's set theory~ZF is built upon
theory~FOL, while HOL conceptually contains first-order logic as a fragment.
Theorem-proving in predicate logic is undecidable, but many researchers have
developed strategies to assist in this task.

Isabelle's classical reasoner is an \ML{} functor that accepts certain
information about a logic and delivers a suite of automatic tactics.  Each
tactic takes a collection of rules and executes a simple, non-clausal proof
procedure.  They are slow and simplistic compared with resolution theorem
provers, but they can save considerable time and effort.  They can prove
theorems such as Pelletier's~\cite{pelletier86} problems~40 and~41 in
seconds:
\[ (\exists y. \forall x. J(y,x) \bimp \neg J(x,x))  
   \imp  \neg (\forall x. \exists y. \forall z. J(z,y) \bimp \neg J(z,x)) \]
\[ (\forall z. \exists y. \forall x. F(x,y) \bimp F(x,z) \conj \neg F(x,x))
   \imp \neg (\exists z. \forall x. F(x,z))  
\]
%
The tactics are generic.  They are not restricted to first-order logic, and
have been heavily used in the development of Isabelle's set theory.  Few
interactive proof assistants provide this much automation.  The tactics can
be traced, and their components can be called directly; in this manner,
any proof can be viewed interactively.

The simplest way to apply the classical reasoner (to subgoal~$i$) is to type
\begin{ttbox}
by (Blast_tac \(i\));
\end{ttbox}
This command quickly proves most simple formulas of the predicate calculus or
set theory.  To attempt to prove subgoals using a combination of
rewriting and classical reasoning, try
\begin{ttbox}
auto();                         \emph{\textrm{applies to all subgoals}}
force i;                        \emph{\textrm{applies to one subgoal}}
\end{ttbox}
To do all obvious logical steps, even if they do not prove the
subgoal, type one of the following:
\begin{ttbox}
by Safe_tac;                   \emph{\textrm{applies to all subgoals}}
by (Clarify_tac \(i\));            \emph{\textrm{applies to one subgoal}}
\end{ttbox}


You need to know how the classical reasoner works in order to use it
effectively.  There are many tactics to choose from, including 
{\tt Fast_tac} and \texttt{Best_tac}.

We shall first discuss the underlying principles, then present the classical
reasoner.  Finally, we shall see how to instantiate it for new logics.  The
logics FOL, ZF, HOL and HOLCF have it already installed.


\section{The sequent calculus}
\index{sequent calculus}
Isabelle supports natural deduction, which is easy to use for interactive
proof.  But natural deduction does not easily lend itself to automation,
and has a bias towards intuitionism.  For certain proofs in classical
logic, it can not be called natural.  The {\bf sequent calculus}, a
generalization of natural deduction, is easier to automate.

A {\bf sequent} has the form $\Gamma\turn\Delta$, where $\Gamma$
and~$\Delta$ are sets of formulae.%
\footnote{For first-order logic, sequents can equivalently be made from
  lists or multisets of formulae.} The sequent
\[ P@1,\ldots,P@m\turn Q@1,\ldots,Q@n \]
is {\bf valid} if $P@1\conj\ldots\conj P@m$ implies $Q@1\disj\ldots\disj
Q@n$.  Thus $P@1,\ldots,P@m$ represent assumptions, each of which is true,
while $Q@1,\ldots,Q@n$ represent alternative goals.  A sequent is {\bf
basic} if its left and right sides have a common formula, as in $P,Q\turn
Q,R$; basic sequents are trivially valid.

Sequent rules are classified as {\bf right} or {\bf left}, indicating which
side of the $\turn$~symbol they operate on.  Rules that operate on the
right side are analogous to natural deduction's introduction rules, and
left rules are analogous to elimination rules.  
Recall the natural deduction rules for
  first-order logic, 
\iflabelundefined{fol-fig}{from {\it Introduction to Isabelle}}%
                          {Fig.\ts\ref{fol-fig}}.
The sequent calculus analogue of~$({\imp}I)$ is the rule
$$
\ainfer{\Gamma &\turn \Delta, P\imp Q}{P,\Gamma &\turn \Delta,Q}
\eqno({\imp}R)
$$
This breaks down some implication on the right side of a sequent; $\Gamma$
and $\Delta$ stand for the sets of formulae that are unaffected by the
inference.  The analogue of the pair~$({\disj}I1)$ and~$({\disj}I2)$ is the
single rule 
$$
\ainfer{\Gamma &\turn \Delta, P\disj Q}{\Gamma &\turn \Delta,P,Q}
\eqno({\disj}R)
$$
This breaks down some disjunction on the right side, replacing it by both
disjuncts.  Thus, the sequent calculus is a kind of multiple-conclusion logic.

To illustrate the use of multiple formulae on the right, let us prove
the classical theorem $(P\imp Q)\disj(Q\imp P)$.  Working backwards, we
reduce this formula to a basic sequent:
\[ \infer[(\disj)R]{\turn(P\imp Q)\disj(Q\imp P)}
   {\infer[(\imp)R]{\turn(P\imp Q), (Q\imp P)\;}
    {\infer[(\imp)R]{P \turn Q, (Q\imp P)\qquad}
                    {P, Q \turn Q, P\qquad\qquad}}}
\]
This example is typical of the sequent calculus: start with the desired
theorem and apply rules backwards in a fairly arbitrary manner.  This yields a
surprisingly effective proof procedure.  Quantifiers add few complications,
since Isabelle handles parameters and schematic variables.  See Chapter~10
of {\em ML for the Working Programmer}~\cite{paulson-ml2} for further
discussion.


\section{Simulating sequents by natural deduction}
Isabelle can represent sequents directly, as in the object-logic~\texttt{LK}\@.
But natural deduction is easier to work with, and most object-logics employ
it.  Fortunately, we can simulate the sequent $P@1,\ldots,P@m\turn
Q@1,\ldots,Q@n$ by the Isabelle formula
\[ \List{P@1;\ldots;P@m; \neg Q@2;\ldots; \neg Q@n}\Imp Q@1, \]
where the order of the assumptions and the choice of~$Q@1$ are arbitrary.
Elim-resolution plays a key role in simulating sequent proofs.

We can easily handle reasoning on the left.
As discussed in
\iflabelundefined{destruct}{{\it Introduction to Isabelle}}{{\S}\ref{destruct}}, 
elim-resolution with the rules $(\disj E)$, $(\bot E)$ and $(\exists E)$
achieves a similar effect as the corresponding sequent rules.  For the
other connectives, we use sequent-style elimination rules instead of
destruction rules such as $({\conj}E1,2)$ and $(\forall E)$.  But note that
the rule $(\neg L)$ has no effect under our representation of sequents!
$$
\ainfer{\neg P,\Gamma &\turn \Delta}{\Gamma &\turn \Delta,P}\eqno({\neg}L)
$$
What about reasoning on the right?  Introduction rules can only affect the
formula in the conclusion, namely~$Q@1$.  The other right-side formulae are
represented as negated assumptions, $\neg Q@2$, \ldots,~$\neg Q@n$.  
\index{assumptions!negated}
In order to operate on one of these, it must first be exchanged with~$Q@1$.
Elim-resolution with the {\bf swap} rule has this effect:
$$ \List{\neg P; \; \neg R\Imp P} \Imp R   \eqno(swap)  $$
To ensure that swaps occur only when necessary, each introduction rule is
converted into a swapped form: it is resolved with the second premise
of~$(swap)$.  The swapped form of~$({\conj}I)$, which might be
called~$({\neg\conj}E)$, is
\[ \List{\neg(P\conj Q); \; \neg R\Imp P; \; \neg R\Imp Q} \Imp R. \]
Similarly, the swapped form of~$({\imp}I)$ is
\[ \List{\neg(P\imp Q); \; \List{\neg R;P}\Imp Q} \Imp R  \]
Swapped introduction rules are applied using elim-resolution, which deletes
the negated formula.  Our representation of sequents also requires the use
of ordinary introduction rules.  If we had no regard for readability, we
could treat the right side more uniformly by representing sequents as
\[ \List{P@1;\ldots;P@m; \neg Q@1;\ldots; \neg Q@n}\Imp \bot. \]


\section{Extra rules for the sequent calculus}
As mentioned, destruction rules such as $({\conj}E1,2)$ and $(\forall E)$
must be replaced by sequent-style elimination rules.  In addition, we need
rules to embody the classical equivalence between $P\imp Q$ and $\neg P\disj
Q$.  The introduction rules~$({\disj}I1,2)$ are replaced by a rule that
simulates $({\disj}R)$:
\[ (\neg Q\Imp P) \Imp P\disj Q \]
The destruction rule $({\imp}E)$ is replaced by
\[ \List{P\imp Q;\; \neg P\Imp R;\; Q\Imp R} \Imp R. \]
Quantifier replication also requires special rules.  In classical logic,
$\exists x{.}P$ is equivalent to $\neg\forall x{.}\neg P$; the rules
$(\exists R)$ and $(\forall L)$ are dual:
\[ \ainfer{\Gamma &\turn \Delta, \exists x{.}P}
          {\Gamma &\turn \Delta, \exists x{.}P, P[t/x]} \; (\exists R)
   \qquad
   \ainfer{\forall x{.}P, \Gamma &\turn \Delta}
          {P[t/x], \forall x{.}P, \Gamma &\turn \Delta} \; (\forall L)
\]
Thus both kinds of quantifier may be replicated.  Theorems requiring
multiple uses of a universal formula are easy to invent; consider 
\[ (\forall x.P(x)\imp P(f(x))) \conj P(a) \imp P(f^n(a)), \]
for any~$n>1$.  Natural examples of the multiple use of an existential
formula are rare; a standard one is $\exists x.\forall y. P(x)\imp P(y)$.

Forgoing quantifier replication loses completeness, but gains decidability,
since the search space becomes finite.  Many useful theorems can be proved
without replication, and the search generally delivers its verdict in a
reasonable time.  To adopt this approach, represent the sequent rules
$(\exists R)$, $(\exists L)$ and $(\forall R)$ by $(\exists I)$, $(\exists
E)$ and $(\forall I)$, respectively, and put $(\forall E)$ into elimination
form:
$$ \List{\forall x{.}P(x); P(t)\Imp Q} \Imp Q    \eqno(\forall E@2) $$
Elim-resolution with this rule will delete the universal formula after a
single use.  To replicate universal quantifiers, replace the rule by
$$
\List{\forall x{.}P(x);\; \List{P(t); \forall x{.}P(x)}\Imp Q} \Imp Q.
\eqno(\forall E@3)
$$
To replicate existential quantifiers, replace $(\exists I)$ by
\[ \List{\neg(\exists x{.}P(x)) \Imp P(t)} \Imp \exists x{.}P(x). \]
All introduction rules mentioned above are also useful in swapped form.

Replication makes the search space infinite; we must apply the rules with
care.  The classical reasoner distinguishes between safe and unsafe
rules, applying the latter only when there is no alternative.  Depth-first
search may well go down a blind alley; best-first search is better behaved
in an infinite search space.  However, quantifier replication is too
expensive to prove any but the simplest theorems.


\section{Classical rule sets}
\index{classical sets}
Each automatic tactic takes a {\bf classical set} --- a collection of
rules, classified as introduction or elimination and as {\bf safe} or {\bf
unsafe}.  In general, safe rules can be attempted blindly, while unsafe
rules must be used with care.  A safe rule must never reduce a provable
goal to an unprovable set of subgoals.  

The rule~$({\disj}I1)$ is unsafe because it reduces $P\disj Q$ to~$P$.  Any
rule is unsafe whose premises contain new unknowns.  The elimination
rule~$(\forall E@2)$ is unsafe, since it is applied via elim-resolution,
which discards the assumption $\forall x{.}P(x)$ and replaces it by the
weaker assumption~$P(\Var{t})$.  The rule $({\exists}I)$ is unsafe for
similar reasons.  The rule~$(\forall E@3)$ is unsafe in a different sense:
since it keeps the assumption $\forall x{.}P(x)$, it is prone to looping.
In classical first-order logic, all rules are safe except those mentioned
above.

The safe/unsafe distinction is vague, and may be regarded merely as a way
of giving some rules priority over others.  One could argue that
$({\disj}E)$ is unsafe, because repeated application of it could generate
exponentially many subgoals.  Induction rules are unsafe because inductive
proofs are difficult to set up automatically.  Any inference is unsafe that
instantiates an unknown in the proof state --- thus \ttindex{match_tac}
must be used, rather than \ttindex{resolve_tac}.  Even proof by assumption
is unsafe if it instantiates unknowns shared with other subgoals --- thus
\ttindex{eq_assume_tac} must be used, rather than \ttindex{assume_tac}.

\subsection{Adding rules to classical sets}
Classical rule sets belong to the abstract type \mltydx{claset}, which
supports the following operations (provided the classical reasoner is
installed!):
\begin{ttbox} 
empty_cs : claset
print_cs : claset -> unit
rep_cs : claset -> \{safeEs: thm list, safeIs: thm list,
                    hazEs: thm list,  hazIs: thm list, 
                    swrappers: (string * wrapper) list, 
                    uwrappers: (string * wrapper) list,
                    safe0_netpair: netpair, safep_netpair: netpair,
                    haz_netpair: netpair, dup_netpair: netpair\}
addSIs   : claset * thm list -> claset                 \hfill{\bf infix 4}
addSEs   : claset * thm list -> claset                 \hfill{\bf infix 4}
addSDs   : claset * thm list -> claset                 \hfill{\bf infix 4}
addIs    : claset * thm list -> claset                 \hfill{\bf infix 4}
addEs    : claset * thm list -> claset                 \hfill{\bf infix 4}
addDs    : claset * thm list -> claset                 \hfill{\bf infix 4}
delrules : claset * thm list -> claset                 \hfill{\bf infix 4}
\end{ttbox}
The add operations ignore any rule already present in the claset with the same
classification (such as safe introduction).  They print a warning if the rule
has already been added with some other classification, but add the rule
anyway.  Calling \texttt{delrules} deletes all occurrences of a rule from the
claset, but see the warning below concerning destruction rules.
\begin{ttdescription}
\item[\ttindexbold{empty_cs}] is the empty classical set.

\item[\ttindexbold{print_cs} $cs$] displays the printable contents of~$cs$,
  which is the rules. All other parts are non-printable.

\item[\ttindexbold{rep_cs} $cs$] decomposes $cs$ as a record of its internal 
  components, namely the safe introduction and elimination rules, the unsafe
  introduction and elimination rules, the lists of safe and unsafe wrappers
  (see \ref{sec:modifying-search}), and the internalized forms of the rules.

\item[$cs$ addSIs $rules$] \indexbold{*addSIs}
adds safe introduction~$rules$ to~$cs$.

\item[$cs$ addSEs $rules$] \indexbold{*addSEs}
adds safe elimination~$rules$ to~$cs$.

\item[$cs$ addSDs $rules$] \indexbold{*addSDs}
adds safe destruction~$rules$ to~$cs$.

\item[$cs$ addIs $rules$] \indexbold{*addIs}
adds unsafe introduction~$rules$ to~$cs$.

\item[$cs$ addEs $rules$] \indexbold{*addEs}
adds unsafe elimination~$rules$ to~$cs$.

\item[$cs$ addDs $rules$] \indexbold{*addDs}
adds unsafe destruction~$rules$ to~$cs$.

\item[$cs$ delrules $rules$] \indexbold{*delrules}
deletes~$rules$ from~$cs$.  It prints a warning for those rules that are not
in~$cs$.
\end{ttdescription}

\begin{warn}
  If you added $rule$ using \texttt{addSDs} or \texttt{addDs}, then you must delete
  it as follows:
\begin{ttbox}
\(cs\) delrules [make_elim \(rule\)]
\end{ttbox}
\par\noindent
This is necessary because the operators \texttt{addSDs} and \texttt{addDs} convert
the destruction rules to elimination rules by applying \ttindex{make_elim},
and then insert them using \texttt{addSEs} and \texttt{addEs}, respectively.
\end{warn}

Introduction rules are those that can be applied using ordinary resolution.
The classical set automatically generates their swapped forms, which will
be applied using elim-resolution.  Elimination rules are applied using
elim-resolution.  In a classical set, rules are sorted by the number of new
subgoals they will yield; rules that generate the fewest subgoals will be
tried first (see {\S}\ref{biresolve_tac}).

For elimination and destruction rules there are variants of the add operations
adding a rule in a way such that it is applied only if also its second premise
can be unified with an assumption of the current proof state:
\indexbold{*addSE2}\indexbold{*addSD2}\indexbold{*addE2}\indexbold{*addD2}
\begin{ttbox}
addSE2      : claset * (string * thm) -> claset           \hfill{\bf infix 4}
addSD2      : claset * (string * thm) -> claset           \hfill{\bf infix 4}
addE2       : claset * (string * thm) -> claset           \hfill{\bf infix 4}
addD2       : claset * (string * thm) -> claset           \hfill{\bf infix 4}
\end{ttbox}
\begin{warn}
  A rule to be added in this special way must be given a name, which is used 
  to delete it again -- when desired -- using \texttt{delSWrappers} or 
  \texttt{delWrappers}, respectively. This is because these add operations
  are implemented as wrappers (see \ref{sec:modifying-search} below).
\end{warn}


\subsection{Modifying the search step}
\label{sec:modifying-search}
For a given classical set, the proof strategy is simple.  Perform as many safe
inferences as possible; or else, apply certain safe rules, allowing
instantiation of unknowns; or else, apply an unsafe rule.  The tactics also
eliminate assumptions of the form $x=t$ by substitution if they have been set
up to do so (see \texttt{hyp_subst_tacs} in~{\S}\ref{sec:classical-setup} below).
They may perform a form of Modus Ponens: if there are assumptions $P\imp Q$
and~$P$, then replace $P\imp Q$ by~$Q$.

The classical reasoning tactics --- except \texttt{blast_tac}! --- allow
you to modify this basic proof strategy by applying two lists of arbitrary 
{\bf wrapper tacticals} to it. 
The first wrapper list, which is considered to contain safe wrappers only, 
affects \ttindex{safe_step_tac} and all the tactics that call it.  
The second one, which may contain unsafe wrappers, affects the unsafe parts
of \ttindex{step_tac}, \ttindex{slow_step_tac}, and the tactics that call them.
A wrapper transforms each step of the search, for example 
by attempting other tactics before or after the original step tactic. 
All members of a wrapper list are applied in turn to the respective step tactic.

Initially the two wrapper lists are empty, which means no modification of the
step tactics. Safe and unsafe wrappers are added to a claset 
with the functions given below, supplying them with wrapper names. 
These names may be used to selectively delete wrappers.

\begin{ttbox} 
type wrapper = (int -> tactic) -> (int -> tactic);

addSWrapper  : claset * (string *  wrapper       ) -> claset \hfill{\bf infix 4}
addSbefore   : claset * (string * (int -> tactic)) -> claset \hfill{\bf infix 4}
addSafter    : claset * (string * (int -> tactic)) -> claset \hfill{\bf infix 4}
delSWrapper  : claset *  string                    -> claset \hfill{\bf infix 4}

addWrapper   : claset * (string *  wrapper       ) -> claset \hfill{\bf infix 4}
addbefore    : claset * (string * (int -> tactic)) -> claset \hfill{\bf infix 4}
addafter     : claset * (string * (int -> tactic)) -> claset \hfill{\bf infix 4}
delWrapper   : claset *  string                    -> claset \hfill{\bf infix 4}

addSss       : claset * simpset -> claset                 \hfill{\bf infix 4}
addss        : claset * simpset -> claset                 \hfill{\bf infix 4}
\end{ttbox}
%

\begin{ttdescription}
\item[$cs$ addSWrapper $(name,wrapper)$] \indexbold{*addSWrapper}
adds a new wrapper, which should yield a safe tactic, 
to modify the existing safe step tactic.

\item[$cs$ addSbefore $(name,tac)$] \indexbold{*addSbefore}
adds the given tactic as a safe wrapper, such that it is tried 
{\em before} each safe step of the search.

\item[$cs$ addSafter $(name,tac)$] \indexbold{*addSafter}
adds the given tactic as a safe wrapper, such that it is tried 
when a safe step of the search would fail.

\item[$cs$ delSWrapper $name$] \indexbold{*delSWrapper}
deletes the safe wrapper with the given name.

\item[$cs$ addWrapper $(name,wrapper)$] \indexbold{*addWrapper}
adds a new wrapper to modify the existing (unsafe) step tactic.

\item[$cs$ addbefore $(name,tac)$] \indexbold{*addbefore}
adds the given tactic as an unsafe wrapper, such that it its result is 
concatenated {\em before} the result of each unsafe step.

\item[$cs$ addafter $(name,tac)$] \indexbold{*addafter}
adds the given tactic as an unsafe wrapper, such that it its result is 
concatenated {\em after} the result of each unsafe step.

\item[$cs$ delWrapper $name$] \indexbold{*delWrapper}
deletes the unsafe wrapper with the given name.

\item[$cs$ addSss $ss$] \indexbold{*addss}
adds the simpset~$ss$ to the classical set.  The assumptions and goal will be
simplified, in a rather safe way, after each safe step of the search.

\item[$cs$ addss $ss$] \indexbold{*addss}
adds the simpset~$ss$ to the classical set.  The assumptions and goal will be
simplified, before the each unsafe step of the search.

\end{ttdescription}

\index{simplification!from classical reasoner} 
Strictly speaking, the operators \texttt{addss} and \texttt{addSss}
are not part of the classical reasoner.
, which are used as primitives 
for the automatic tactics described in {\S}\ref{sec:automatic-tactics}, are
implemented as wrapper tacticals.
they  
\begin{warn}
Being defined as wrappers, these operators are inappropriate for adding more 
than one simpset at a time: the simpset added last overwrites any earlier ones.
When a simpset combined with a claset is to be augmented, this should done 
{\em before} combining it with the claset.
\end{warn}


\section{The classical tactics}
\index{classical reasoner!tactics} If installed, the classical module provides
powerful theorem-proving tactics.  Most of them have capitalized analogues
that use the default claset; see {\S}\ref{sec:current-claset}.


\subsection{The tableau prover}
The tactic \texttt{blast_tac} searches for a proof using a fast tableau prover,
coded directly in \ML.  It then reconstructs the proof using Isabelle
tactics.  It is faster and more powerful than the other classical
reasoning tactics, but has major limitations too.
\begin{itemize}
\item It does not use the wrapper tacticals described above, such as
  \ttindex{addss}.
\item It ignores types, which can cause problems in HOL.  If it applies a rule
  whose types are inappropriate, then proof reconstruction will fail.
\item It does not perform higher-order unification, as needed by the rule {\tt
    rangeI} in HOL and \texttt{RepFunI} in ZF.  There are often alternatives
  to such rules, for example {\tt range_eqI} and \texttt{RepFun_eqI}.
\item Function variables may only be applied to parameters of the subgoal.
(This restriction arises because the prover does not use higher-order
unification.)  If other function variables are present then the prover will
fail with the message {\small\tt Function Var's argument not a bound variable}.
\item Its proof strategy is more general than \texttt{fast_tac}'s but can be
  slower.  If \texttt{blast_tac} fails or seems to be running forever, try {\tt
  fast_tac} and the other tactics described below.
\end{itemize}
%
\begin{ttbox} 
blast_tac        : claset -> int -> tactic
Blast.depth_tac  : claset -> int -> int -> tactic
Blast.trace      : bool ref \hfill{\bf initially false}
\end{ttbox}
The two tactics differ on how they bound the number of unsafe steps used in a
proof.  While \texttt{blast_tac} starts with a bound of zero and increases it
successively to~20, \texttt{Blast.depth_tac} applies a user-supplied search bound.
\begin{ttdescription}
\item[\ttindexbold{blast_tac} $cs$ $i$] tries to prove
  subgoal~$i$, increasing the search bound using iterative
  deepening~\cite{korf85}. 
  
\item[\ttindexbold{Blast.depth_tac} $cs$ $lim$ $i$] tries
  to prove subgoal~$i$ using a search bound of $lim$.  Sometimes a slow
  proof using \texttt{blast_tac} can be made much faster by supplying the
  successful search bound to this tactic instead.
  
\item[set \ttindexbold{Blast.trace};] \index{tracing!of classical prover}
  causes the tableau prover to print a trace of its search.  At each step it
  displays the formula currently being examined and reports whether the branch
  has been closed, extended or split.
\end{ttdescription}


\subsection{Automatic tactics}\label{sec:automatic-tactics}
\begin{ttbox} 
type clasimpset = claset * simpset;
auto_tac        : clasimpset ->        tactic
force_tac       : clasimpset -> int -> tactic
auto            : unit -> unit
force           : int  -> unit
\end{ttbox}
The automatic tactics attempt to prove goals using a combination of
simplification and classical reasoning. 
\begin{ttdescription}
\item[\ttindexbold{auto_tac $(cs,ss)$}] is intended for situations where 
there are a lot of mostly trivial subgoals; it proves all the easy ones, 
leaving the ones it cannot prove.
(Unfortunately, attempting to prove the hard ones may take a long time.)  
\item[\ttindexbold{force_tac} $(cs,ss)$ $i$] is intended to prove subgoal~$i$ 
completely. It tries to apply all fancy tactics it knows about, 
performing a rather exhaustive search.
\end{ttdescription}
They must be supplied both a simpset and a claset; therefore 
they are most easily called as \texttt{Auto_tac} and \texttt{Force_tac}, which 
use the default claset and simpset (see {\S}\ref{sec:current-claset} below). 
For interactive use, 
the shorthand \texttt{auto();} abbreviates \texttt{by Auto_tac;} 
while \texttt{force 1;} abbreviates \texttt{by (Force_tac 1);}


\subsection{Semi-automatic tactics}
\begin{ttbox} 
clarify_tac      : claset -> int -> tactic
clarify_step_tac : claset -> int -> tactic
clarsimp_tac     : clasimpset -> int -> tactic
\end{ttbox}
Use these when the automatic tactics fail.  They perform all the obvious
logical inferences that do not split the subgoal.  The result is a
simpler subgoal that can be tackled by other means, such as by
instantiating quantifiers yourself.
\begin{ttdescription}
\item[\ttindexbold{clarify_tac} $cs$ $i$] performs a series of safe steps on
subgoal~$i$ by repeatedly calling \texttt{clarify_step_tac}.
\item[\ttindexbold{clarify_step_tac} $cs$ $i$] performs a safe step on
  subgoal~$i$.  No splitting step is applied; for example, the subgoal $A\conj
  B$ is left as a conjunction.  Proof by assumption, Modus Ponens, etc., may be
  performed provided they do not instantiate unknowns.  Assumptions of the
  form $x=t$ may be eliminated.  The user-supplied safe wrapper tactical is
  applied.
\item[\ttindexbold{clarsimp_tac} $cs$ $i$] acts like \texttt{clarify_tac}, but
also does simplification with the given simpset. Note that if the simpset 
includes a splitter for the premises, the subgoal may still be split.
\end{ttdescription}


\subsection{Other classical tactics}
\begin{ttbox} 
fast_tac      : claset -> int -> tactic
best_tac      : claset -> int -> tactic
slow_tac      : claset -> int -> tactic
slow_best_tac : claset -> int -> tactic
\end{ttbox}
These tactics attempt to prove a subgoal using sequent-style reasoning.
Unlike \texttt{blast_tac}, they construct proofs directly in Isabelle.  Their
effect is restricted (by \texttt{SELECT_GOAL}) to one subgoal; they either prove
this subgoal or fail.  The \texttt{slow_} versions conduct a broader
search.%
\footnote{They may, when backtracking from a failed proof attempt, undo even
  the step of proving a subgoal by assumption.}

The best-first tactics are guided by a heuristic function: typically, the
total size of the proof state.  This function is supplied in the functor call
that sets up the classical reasoner.
\begin{ttdescription}
\item[\ttindexbold{fast_tac} $cs$ $i$] applies \texttt{step_tac} using
depth-first search to prove subgoal~$i$.

\item[\ttindexbold{best_tac} $cs$ $i$] applies \texttt{step_tac} using
best-first search to prove subgoal~$i$.

\item[\ttindexbold{slow_tac} $cs$ $i$] applies \texttt{slow_step_tac} using
depth-first search to prove subgoal~$i$.

\item[\ttindexbold{slow_best_tac} $cs$ $i$] applies \texttt{slow_step_tac} with
best-first search to prove subgoal~$i$.
\end{ttdescription}


\subsection{Depth-limited automatic tactics}
\begin{ttbox} 
depth_tac  : claset -> int -> int -> tactic
deepen_tac : claset -> int -> int -> tactic
\end{ttbox}
These work by exhaustive search up to a specified depth.  Unsafe rules are
modified to preserve the formula they act on, so that it be used repeatedly.
They can prove more goals than \texttt{fast_tac} can but are much
slower, for example if the assumptions have many universal quantifiers.

The depth limits the number of unsafe steps.  If you can estimate the minimum
number of unsafe steps needed, supply this value as~$m$ to save time.
\begin{ttdescription}
\item[\ttindexbold{depth_tac} $cs$ $m$ $i$] 
tries to prove subgoal~$i$ by exhaustive search up to depth~$m$.

\item[\ttindexbold{deepen_tac} $cs$ $m$ $i$] 
tries to prove subgoal~$i$ by iterative deepening.  It calls \texttt{depth_tac}
repeatedly with increasing depths, starting with~$m$.
\end{ttdescription}


\subsection{Single-step tactics}
\begin{ttbox} 
safe_step_tac : claset -> int -> tactic
safe_tac      : claset        -> tactic
inst_step_tac : claset -> int -> tactic
step_tac      : claset -> int -> tactic
slow_step_tac : claset -> int -> tactic
\end{ttbox}
The automatic proof procedures call these tactics.  By calling them
yourself, you can execute these procedures one step at a time.
\begin{ttdescription}
\item[\ttindexbold{safe_step_tac} $cs$ $i$] performs a safe step on
  subgoal~$i$.  The safe wrapper tacticals are applied to a tactic that may
  include proof by assumption or Modus Ponens (taking care not to instantiate
  unknowns), or substitution.

\item[\ttindexbold{safe_tac} $cs$] repeatedly performs safe steps on all 
subgoals.  It is deterministic, with at most one outcome.  

\item[\ttindexbold{inst_step_tac} $cs$ $i$] is like \texttt{safe_step_tac},
but allows unknowns to be instantiated.

\item[\ttindexbold{step_tac} $cs$ $i$] is the basic step of the proof
  procedure.  The unsafe wrapper tacticals are applied to a tactic that tries
  \texttt{safe_tac}, \texttt{inst_step_tac}, or applies an unsafe rule
  from~$cs$.

\item[\ttindexbold{slow_step_tac}] 
  resembles \texttt{step_tac}, but allows backtracking between using safe
  rules with instantiation (\texttt{inst_step_tac}) and using unsafe rules.
  The resulting search space is larger.
\end{ttdescription}


\subsection{The current claset}\label{sec:current-claset}

Each theory is equipped with an implicit \emph{current claset}
\index{claset!current}.  This is a default set of classical
rules.  The underlying idea is quite similar to that of a current
simpset described in {\S}\ref{sec:simp-for-dummies}; please read that
section, including its warnings.  

The tactics
\begin{ttbox}
Blast_tac        : int -> tactic
Auto_tac         :        tactic
Force_tac        : int -> tactic
Fast_tac         : int -> tactic
Best_tac         : int -> tactic
Deepen_tac       : int -> int -> tactic
Clarify_tac      : int -> tactic
Clarify_step_tac : int -> tactic
Clarsimp_tac     : int -> tactic
Safe_tac         :        tactic
Safe_step_tac    : int -> tactic
Step_tac         : int -> tactic
\end{ttbox}
\indexbold{*Blast_tac}\indexbold{*Auto_tac}\indexbold{*Force_tac}
\indexbold{*Best_tac}\indexbold{*Fast_tac}%
\indexbold{*Deepen_tac}
\indexbold{*Clarify_tac}\indexbold{*Clarify_step_tac}\indexbold{*Clarsimp_tac}
\indexbold{*Safe_tac}\indexbold{*Safe_step_tac}
\indexbold{*Step_tac}
make use of the current claset.  For example, \texttt{Blast_tac} is defined as 
\begin{ttbox}
fun Blast_tac i st = blast_tac (claset()) i st;
\end{ttbox}
and gets the current claset, only after it is applied to a proof state.  
The functions
\begin{ttbox}
AddSIs, AddSEs, AddSDs, AddIs, AddEs, AddDs: thm list -> unit
\end{ttbox}
\indexbold{*AddSIs} \indexbold{*AddSEs} \indexbold{*AddSDs}
\indexbold{*AddIs} \indexbold{*AddEs} \indexbold{*AddDs}
are used to add rules to the current claset.  They work exactly like their
lower case counterparts, such as \texttt{addSIs}.  Calling
\begin{ttbox}
Delrules : thm list -> unit
\end{ttbox}
deletes rules from the current claset. 


\subsection{Accessing the current claset}
\label{sec:access-current-claset}

the functions to access the current claset are analogous to the functions 
for the current simpset, so please see \ref{sec:access-current-simpset}
for a description.
\begin{ttbox}
claset        : unit   -> claset
claset_ref    : unit   -> claset ref
claset_of     : theory -> claset
claset_ref_of : theory -> claset ref
print_claset  : theory -> unit
CLASET        :(claset     ->       tactic) ->       tactic
CLASET'       :(claset     -> 'a -> tactic) -> 'a -> tactic
CLASIMPSET    :(clasimpset ->       tactic) ->       tactic
CLASIMPSET'   :(clasimpset -> 'a -> tactic) -> 'a -> tactic
\end{ttbox}


\subsection{Other useful tactics}
\index{tactics!for contradiction}
\index{tactics!for Modus Ponens}
\begin{ttbox} 
contr_tac    :             int -> tactic
mp_tac       :             int -> tactic
eq_mp_tac    :             int -> tactic
swap_res_tac : thm list -> int -> tactic
\end{ttbox}
These can be used in the body of a specialized search.
\begin{ttdescription}
\item[\ttindexbold{contr_tac} {\it i}]\index{assumptions!contradictory}
  solves subgoal~$i$ by detecting a contradiction among two assumptions of
  the form $P$ and~$\neg P$, or fail.  It may instantiate unknowns.  The
  tactic can produce multiple outcomes, enumerating all possible
  contradictions.

\item[\ttindexbold{mp_tac} {\it i}] 
is like \texttt{contr_tac}, but also attempts to perform Modus Ponens in
subgoal~$i$.  If there are assumptions $P\imp Q$ and~$P$, then it replaces
$P\imp Q$ by~$Q$.  It may instantiate unknowns.  It fails if it can do
nothing.

\item[\ttindexbold{eq_mp_tac} {\it i}] 
is like \texttt{mp_tac} {\it i}, but may not instantiate unknowns --- thus, it
is safe.

\item[\ttindexbold{swap_res_tac} {\it thms} {\it i}] refines subgoal~$i$ of
the proof state using {\it thms}, which should be a list of introduction
rules.  First, it attempts to prove the goal using \texttt{assume_tac} or
\texttt{contr_tac}.  It then attempts to apply each rule in turn, attempting
resolution and also elim-resolution with the swapped form.
\end{ttdescription}

\subsection{Creating swapped rules}
\begin{ttbox} 
swapify   : thm list -> thm list
joinrules : thm list * thm list -> (bool * thm) list
\end{ttbox}
\begin{ttdescription}
\item[\ttindexbold{swapify} {\it thms}] returns a list consisting of the
swapped versions of~{\it thms}, regarded as introduction rules.

\item[\ttindexbold{joinrules} ({\it intrs}, {\it elims})]
joins introduction rules, their swapped versions, and elimination rules for
use with \ttindex{biresolve_tac}.  Each rule is paired with~\texttt{false}
(indicating ordinary resolution) or~\texttt{true} (indicating
elim-resolution).
\end{ttdescription}


\section{Setting up the classical reasoner}\label{sec:classical-setup}
\index{classical reasoner!setting up}
Isabelle's classical object-logics, including \texttt{FOL} and \texttt{HOL}, 
have the classical reasoner already set up.  
When defining a new classical logic, you should set up the reasoner yourself.  
It consists of the \ML{} functor \ttindex{ClassicalFun}, which takes the 
argument signature \texttt{CLASSICAL_DATA}:
\begin{ttbox} 
signature CLASSICAL_DATA =
  sig
  val mp             : thm
  val not_elim       : thm
  val swap           : thm
  val sizef          : thm -> int
  val hyp_subst_tacs : (int -> tactic) list
  end;
\end{ttbox}
Thus, the functor requires the following items:
\begin{ttdescription}
\item[\tdxbold{mp}] should be the Modus Ponens rule
$\List{\Var{P}\imp\Var{Q};\; \Var{P}} \Imp \Var{Q}$.

\item[\tdxbold{not_elim}] should be the contradiction rule
$\List{\neg\Var{P};\; \Var{P}} \Imp \Var{R}$.

\item[\tdxbold{swap}] should be the swap rule
$\List{\neg \Var{P}; \; \neg \Var{R}\Imp \Var{P}} \Imp \Var{R}$.

\item[\ttindexbold{sizef}] is the heuristic function used for best-first
search.  It should estimate the size of the remaining subgoals.  A good
heuristic function is \ttindex{size_of_thm}, which measures the size of the
proof state.  Another size function might ignore certain subgoals (say,
those concerned with type-checking).  A heuristic function might simply
count the subgoals.

\item[\ttindexbold{hyp_subst_tacs}] is a list of tactics for substitution in
the hypotheses, typically created by \ttindex{HypsubstFun} (see
Chapter~\ref{substitution}).  This list can, of course, be empty.  The
tactics are assumed to be safe!
\end{ttdescription}
The functor is not at all sensitive to the formalization of the
object-logic.  It does not even examine the rules, but merely applies
them according to its fixed strategy.  The functor resides in {\tt
  Provers/classical.ML} in the Isabelle sources.

\index{classical reasoner|)}

\section{Setting up the combination with the simplifier}
\label{sec:clasimp-setup}

To combine the classical reasoner and the simplifier, we simply call the 
\ML{} functor \ttindex{ClasimpFun} that assembles the parts as required. 
It takes a structure (of signature \texttt{CLASIMP_DATA}) as
argment, which can be contructed on the fly:
\begin{ttbox}
structure Clasimp = ClasimpFun
 (structure Simplifier = Simplifier 
        and Classical  = Classical 
        and Blast      = Blast);
\end{ttbox}
%
%%% Local Variables: 
%%% mode: latex
%%% TeX-master: "ref"
%%% End: 


\chapter{Isabelle/HOL specific elements}\label{ch:hol-tools}

\section{Miscellaneous attributes}

\indexisarattof{HOL}{split-format}
\begin{matharray}{rcl}
  split_format^* & : & \isaratt \\
\end{matharray}

\railalias{splitformat}{split\_format}
\railterm{splitformat}
\railterm{complete}

\begin{rail}
  splitformat (((name * ) + 'and') | ('(' complete ')'))
  ;
\end{rail}

\begin{descr}
  
\item [$split_format~\vec p@1 \dots \vec p@n$] puts tuple objects into
  canonical form as specified by the arguments given; $\vec p@i$ refers to
  occurrences in premise $i$ of the rule.  The $split_format~(complete)$ form
  causes \emph{all} arguments in function applications to be represented
  canonically according to their tuple type structure.
  
  Note that these operations tend to invent funny names for new local
  parameters to be introduced.

\end{descr}


\section{Primitive types}\label{sec:typedef}

\indexisarcmdof{HOL}{typedecl}\indexisarcmdof{HOL}{typedef}
\begin{matharray}{rcl}
  \isarcmd{typedecl} & : & \isartrans{theory}{theory} \\
  \isarcmd{typedef} & : & \isartrans{theory}{proof(prove)} \\
\end{matharray}

\begin{rail}
  'typedecl' typespec infix? comment?
  ;
  'typedef' parname? typespec infix? \\ '=' term comment?
  ;
\end{rail}

\begin{descr}
\item [$\isarkeyword{typedecl}~(\vec\alpha)t$] is similar to the original
  $\isarkeyword{typedecl}$ of Isabelle/Pure (see \S\ref{sec:types-pure}), but
  also declares type arity $t :: (term, \dots, term) term$, making $t$ an
  actual HOL type constructor.
\item [$\isarkeyword{typedef}~(\vec\alpha)t = A$] sets up a goal stating
  non-emptiness of the set $A$.  After finishing the proof, the theory will be
  augmented by a Gordon/HOL-style type definition.  See \cite{isabelle-HOL}
  for more information.  Note that user-level theories usually do not directly
  refer to the HOL $\isarkeyword{typedef}$ primitive, but use more advanced
  packages such as $\isarkeyword{record}$ (see \S\ref{sec:hol-record}) and
  $\isarkeyword{datatype}$ (see \S\ref{sec:hol-datatype}).
\end{descr}


\section{Records}\label{sec:hol-record}

FIXME proof tools (simp, cases/induct; no split!?);

\indexisarcmdof{HOL}{record}
\begin{matharray}{rcl}
  \isarcmd{record} & : & \isartrans{theory}{theory} \\
\end{matharray}

\begin{rail}
  'record' typespec '=' (type '+')? (field +)
  ;

  field: name '::' type comment?
  ;
\end{rail}

\begin{descr}
\item [$\isarkeyword{record}~(\vec\alpha)t = \tau + \vec c :: \vec\sigma$]
  defines extensible record type $(\vec\alpha)t$, derived from the optional
  parent record $\tau$ by adding new field components $\vec c :: \vec\sigma$.
  See \cite{isabelle-HOL,NaraschewskiW-TPHOLs98} for more information on
  simply-typed extensible records.
\end{descr}


\section{Datatypes}\label{sec:hol-datatype}

\indexisarcmdof{HOL}{datatype}\indexisarcmdof{HOL}{rep-datatype}
\begin{matharray}{rcl}
  \isarcmd{datatype} & : & \isartrans{theory}{theory} \\
  \isarcmd{rep_datatype} & : & \isartrans{theory}{theory} \\
\end{matharray}

\railalias{repdatatype}{rep\_datatype}
\railterm{repdatatype}

\begin{rail}
  'datatype' (dtspec + 'and')
  ;
  repdatatype (name * ) dtrules
  ;

  dtspec: parname? typespec infix? '=' (cons + '|')
  ;
  cons: name (type * ) mixfix? comment?
  ;
  dtrules: 'distinct' thmrefs 'inject' thmrefs 'induction' thmrefs
\end{rail}

\begin{descr}
\item [$\isarkeyword{datatype}$] defines inductive datatypes in HOL.
\item [$\isarkeyword{rep_datatype}$] represents existing types as inductive
  ones, generating the standard infrastructure of derived concepts (primitive
  recursion etc.).
\end{descr}

The induction and exhaustion theorems generated provide case names according
to the constructors involved, while parameters are named after the types (see
also \S\ref{sec:cases-induct}).

See \cite{isabelle-HOL} for more details on datatypes.  Note that the theory
syntax above has been slightly simplified over the old version, usually
requiring more quotes and less parentheses.  Apart from proper proof methods
for case-analysis and induction, there are also emulations of ML tactics
\texttt{case_tac} and \texttt{induct_tac} available, see
\S\ref{sec:induct_tac}.


\section{Recursive functions}\label{sec:recursion}

\indexisarcmdof{HOL}{primrec}\indexisarcmdof{HOL}{recdef}\indexisarcmdof{HOL}{recdef-tc}
\begin{matharray}{rcl}
  \isarcmd{primrec} & : & \isartrans{theory}{theory} \\
  \isarcmd{recdef} & : & \isartrans{theory}{theory} \\
  \isarcmd{recdef_tc}^* & : & \isartrans{theory}{proof(prove)} \\
%FIXME
%  \isarcmd{defer_recdef} & : & \isartrans{theory}{theory} \\
\end{matharray}

\railalias{recdefsimp}{recdef\_simp}
\railterm{recdefsimp}

\railalias{recdefcong}{recdef\_cong}
\railterm{recdefcong}

\railalias{recdefwf}{recdef\_wf}
\railterm{recdefwf}

\railalias{recdeftc}{recdef\_tc}
\railterm{recdeftc}

\begin{rail}
  'primrec' parname? (equation + )
  ;
  'recdef' ('(' 'permissive' ')')? \\ name term (eqn + ) hints?
  ;
  recdeftc thmdecl? tc comment?
  ;

  equation: thmdecl? eqn
  ;
  eqn: prop comment?
  ;
  hints: '(' 'hints' (recdefmod * ) ')'
  ;
  recdefmod: ((recdefsimp | recdefcong | recdefwf) (() | 'add' | 'del') ':' thmrefs) | clasimpmod
  ;
  tc: nameref ('(' nat ')')?
  ;
\end{rail}

\begin{descr}
\item [$\isarkeyword{primrec}$] defines primitive recursive functions over
  datatypes, see also \cite{isabelle-HOL}.
\item [$\isarkeyword{recdef}$] defines general well-founded recursive
  functions (using the TFL package), see also \cite{isabelle-HOL}.  The
  $(permissive)$ option tells TFL to recover from failed proof attempts,
  returning unfinished results.  The $recdef_simp$, $recdef_cong$, and
  $recdef_wf$ hints refer to auxiliary rules to be used in the internal
  automated proof process of TFL.  Additional $clasimpmod$ declarations (cf.\ 
  \S\ref{sec:clasimp}) may be given to tune the context of the Simplifier
  (cf.\ \S\ref{sec:simplifier}) and Classical reasoner (cf.\ 
  \S\ref{sec:classical}).
\item [$\isarkeyword{recdef_tc}~c~(i)$] recommences the proof for leftover
  termination condition number $i$ (default $1$) as generated by a
  $\isarkeyword{recdef}$ definition of constant $c$.
  
  Note that in most cases, $\isarkeyword{recdef}$ is able to finish its
  internal proofs without manual intervention.
\end{descr}

Both kinds of recursive definitions accommodate reasoning by induction (cf.\ 
\S\ref{sec:cases-induct}): rule $c\mathord{.}induct$ (where $c$ is the name of
the function definition) refers to a specific induction rule, with parameters
named according to the user-specified equations.  Case names of
$\isarkeyword{primrec}$ are that of the datatypes involved, while those of
$\isarkeyword{recdef}$ are numbered (starting from $1$).

The equations provided by these packages may be referred later as theorem list
$f\mathord.simps$, where $f$ is the (collective) name of the functions
defined.  Individual equations may be named explicitly as well; note that for
$\isarkeyword{recdef}$ each specification given by the user may result in
several theorems.

\medskip Hints for $\isarkeyword{recdef}$ may be also declared globally, using
the following attributes.

\indexisarattof{HOL}{recdef-simp}\indexisarattof{HOL}{recdef-cong}\indexisarattof{HOL}{recdef-wf}
\begin{matharray}{rcl}
  recdef_simp & : & \isaratt \\
  recdef_cong & : & \isaratt \\
  recdef_wf & : & \isaratt \\
\end{matharray}

\railalias{recdefsimp}{recdef\_simp}
\railterm{recdefsimp}

\railalias{recdefcong}{recdef\_cong}
\railterm{recdefcong}

\railalias{recdefwf}{recdef\_wf}
\railterm{recdefwf}

\begin{rail}
  (recdefsimp | recdefcong | recdefwf) (() | 'add' | 'del')
  ;
\end{rail}


\section{(Co)Inductive sets}\label{sec:hol-inductive}

\indexisarcmdof{HOL}{inductive}\indexisarcmdof{HOL}{coinductive}\indexisarattof{HOL}{mono}
\begin{matharray}{rcl}
  \isarcmd{inductive} & : & \isartrans{theory}{theory} \\
  \isarcmd{coinductive} & : & \isartrans{theory}{theory} \\
  mono & : & \isaratt \\
\end{matharray}

\railalias{condefs}{con\_defs}
\railterm{condefs}

\begin{rail}
  ('inductive' | 'coinductive') sets intros monos?
  ;
  'mono' (() | 'add' | 'del')
  ;

  sets: (term comment? +)
  ;
  intros: 'intros' (thmdecl? prop comment? +)
  ;
  monos: 'monos' thmrefs comment?
  ;
\end{rail}

\begin{descr}
\item [$\isarkeyword{inductive}$ and $\isarkeyword{coinductive}$] define
  (co)inductive sets from the given introduction rules.
\item [$mono$] declares monotonicity rules.  These rule are involved in the
  automated monotonicity proof of $\isarkeyword{inductive}$.
\end{descr}

See \cite{isabelle-HOL} for further information on inductive definitions in
HOL.


\section{Arithmetic}

\indexisarmethof{HOL}{arith}\indexisarattof{HOL}{arith-split}
\begin{matharray}{rcl}
  arith & : & \isarmeth \\
  arith_split & : & \isaratt \\
\end{matharray}

\begin{rail}
  'arith' '!'?
  ;
\end{rail}

The $arith$ method decides linear arithmetic problems (on types $nat$, $int$,
$real$).  Any current facts are inserted into the goal before running the
procedure.  The ``!''~argument causes the full context of assumptions to be
included.  The $arith_split$ attribute declares case split rules to be
expanded before the arithmetic procedure is invoked.

Note that a simpler (but faster) version of arithmetic reasoning is already
performed by the Simplifier.


\section{Cases and induction: emulating tactic scripts}\label{sec:induct_tac}

The following important tactical tools of Isabelle/HOL have been ported to
Isar.  These should be never used in proper proof texts!

\indexisarmethof{HOL}{case-tac}\indexisarmethof{HOL}{induct-tac}
\indexisarmethof{HOL}{ind-cases}\indexisarcmdof{HOL}{inductive-cases}
\begin{matharray}{rcl}
  case_tac^* & : & \isarmeth \\
  induct_tac^* & : & \isarmeth \\
  ind_cases^* & : & \isarmeth \\
  \isarcmd{inductive_cases} & : & \isartrans{theory}{theory} \\
\end{matharray}

\railalias{casetac}{case\_tac}
\railterm{casetac}

\railalias{inducttac}{induct\_tac}
\railterm{inducttac}

\railalias{indcases}{ind\_cases}
\railterm{indcases}

\railalias{inductivecases}{inductive\_cases}
\railterm{inductivecases}

\begin{rail}
  casetac goalspec? term rule?
  ;
  inducttac goalspec? (insts * 'and') rule?
  ;
  indcases (prop +)
  ;
  inductivecases thmdecl? (prop +) comment?
  ;

  rule: ('rule' ':' thmref)
  ;
\end{rail}

\begin{descr}
\item [$case_tac$ and $induct_tac$] admit to reason about inductive datatypes
  only (unless an alternative rule is given explicitly).  Furthermore,
  $case_tac$ does a classical case split on booleans; $induct_tac$ allows only
  variables to be given as instantiation.  These tactic emulations feature
  both goal addressing and dynamic instantiation.  Note that named rule cases
  are \emph{not} provided as would be by the proper $induct$ and $cases$ proof
  methods (see \S\ref{sec:cases-induct}).
  
\item [$ind_cases$ and $\isarkeyword{inductive_cases}$] provide an interface
  to the \texttt{mk_cases} operation.  Rules are simplified in an unrestricted
  forward manner.
  
  While $ind_cases$ is a proof method to apply the result immediately as
  elimination rules, $\isarkeyword{inductive_cases}$ provides case split
  theorems at the theory level for later use,
\end{descr}


%%% Local Variables: 
%%% mode: latex
%%% TeX-master: "isar-ref"
%%% End: 


\begingroup
  \bibliographystyle{plain} \small\raggedright\frenchspacing
  \bibliography{../manual}
\endgroup


%% $Id$

\documentclass[12pt]{report}
\usepackage{graphicx,a4,../iman,../extra,../proof,../rail,../pdfsetup}

\title{\includegraphics[scale=0.5]{isabelle_isar} \\[4ex] The Isabelle/Isar Reference Manual}

\author{\emph{Markus Wenzel} \\ TU M\"unchen}

\setcounter{secnumdepth}{2} \setcounter{tocdepth}{2}

\pagestyle{headings}
\sloppy
\binperiod     %%%treat . like a binary operator

\railalias{lbrace}{\ttlbrace}
\railalias{rbrace}{\ttrbrace}
\railterm{lbrace,rbrace}

\railterm{ident,longident,symident,var,textvar,typefree,typevar,nat,string,verbatim}


\begin{document}

\underscoreoff

\maketitle 

\begin{abstract}
  FIXME
\end{abstract}

\pagenumbering{roman} \tableofcontents \clearfirst

%FIXME
\nocite{Rudnicki:1992:MizarOverview}
\nocite{Harrison:1996:MizarHOL}
\nocite{Rudnicki:1992:MizarOverview}
\nocite{Trybulec:1993:MizarFeatures}
\nocite{Syme:1997:DECLARE}
\nocite{Syme:1998:thesis}
\nocite{Syme:1999:TPHOL}
\nocite{Wenzel:1999:TPHOL}

\include{intro}
\include{basics}
\include{syntax}
\include{pure}
\include{simplifier}
\include{classical}
\include{hol}

\begingroup
  \bibliographystyle{plain} \small\raggedright\frenchspacing
  \bibliography{../manual}
\endgroup

\input{isar-ref.ind}

\end{document}


\end{document}


\end{document}


\end{document}
