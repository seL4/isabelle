%
\begin{isabellebody}%
\def\isabellecontext{termination}%
%
\begin{isamarkuptext}%
When a function~$f$ is defined via \isacommand{recdef}, Isabelle tries to prove
its termination with the help of the user-supplied measure.  Each of the examples
above is simple enough that Isabelle can automatically prove that the
argument's measure decreases in each recursive call. As a result,
$f$\isa{{\isachardot}simps} will contain the defining equations (or variants derived
from them) as theorems. For example, look (via \isacommand{thm}) at
\isa{sep{\isachardot}simps} and \isa{sep{\isadigit{1}}{\isachardot}simps} to see that they define
the same function. What is more, those equations are automatically declared as
simplification rules.

Isabelle may fail to prove the termination condition for some
recursive call.  Let us try the following artificial function:%
\end{isamarkuptext}%
\isacommand{consts}\ f\ {\isacharcolon}{\isacharcolon}\ {\isachardoublequote}nat{\isasymtimes}nat\ {\isasymRightarrow}\ nat{\isachardoublequote}\isanewline
\isacommand{recdef}\ \end{isabellebody}%
%%% Local Variables:
%%% mode: latex
%%% TeX-master: "root"
%%% End:
