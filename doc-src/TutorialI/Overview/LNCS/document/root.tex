\documentclass[11pt,a4paper]{article}
\usepackage{isabelle,isabellesym,pdfsetup}

%for best-style documents ...
\urlstyle{rm}
%\isabellestyle{it}

\newtheorem{Exercise}{Exercise}[section]
\newenvironment{exercise}{\begin{Exercise}\rm}{\end{Exercise}}

\begin{document}

\title{A Compact Overview of Isabelle/HOL}
\author{Tobias Nipkow\\Institut f{\"u}r Informatik, TU M{\"u}nchen\\
 \small\url{http://www.in.tum.de/~nipkow/}}
\date{}
\maketitle

\noindent
This document is a very compact introduction to the proof assistant
Isabelle/HOL and is based directly on the published tutorial~\cite{LNCS2283}
where full explanations and a wealth of additional material can be found.

While reading this document it is recommended to single-step through the
corresponding theories using Isabelle/HOL to follow the proofs.

\section{Functional Programming/Modelling}

\subsection{An Introductory Theory}
\input{FP0.tex}

\begin{exercise}
Define a datatype of binary trees and a function \isa{mirror}
that mirrors a binary tree by swapping subtrees recursively. Prove
\isa{mirror(mirror t) = t}.

Define a function \isa{flatten} that flattens a tree into a list
by traversing it in infix order. Prove
\isa{flatten(mirror t) = rev(flatten t)}.
\end{exercise}

\subsection{More Constructs}
\input{FP1.tex}

\input{RECDEF.tex}

\input{Rules.tex}

\input{Sets.tex}

\input{Ind.tex}

%%
\begin{isabellebody}%
\def\isabellecontext{Isar}%
%
\isadelimtheory
%
\endisadelimtheory
%
\isatagtheory
\isacommand{theory}\isamarkupfalse%
\ Isar\isanewline
\isakeyword{imports}\ Base\isanewline
\isakeyword{begin}%
\endisatagtheory
{\isafoldtheory}%
%
\isadelimtheory
%
\endisadelimtheory
%
\isamarkupchapter{Isar language elements%
}
\isamarkuptrue%
%
\begin{isamarkuptext}%
The primary Isar language consists of three main categories of
  language elements:

  \begin{enumerate}

  \item Proof commands

  \item Proof methods

  \item Attributes

  \end{enumerate}%
\end{isamarkuptext}%
\isamarkuptrue%
%
\isamarkupsection{Proof commands%
}
\isamarkuptrue%
%
\begin{isamarkuptext}%
FIXME%
\end{isamarkuptext}%
\isamarkuptrue%
%
\isamarkupsection{Proof methods%
}
\isamarkuptrue%
%
\begin{isamarkuptext}%
FIXME%
\end{isamarkuptext}%
\isamarkuptrue%
%
\isamarkupsection{Attributes%
}
\isamarkuptrue%
%
\begin{isamarkuptext}%
FIXME%
\end{isamarkuptext}%
\isamarkuptrue%
%
\isadelimtheory
%
\endisadelimtheory
%
\isatagtheory
\isacommand{end}\isamarkupfalse%
%
\endisatagtheory
{\isafoldtheory}%
%
\isadelimtheory
%
\endisadelimtheory
\isanewline
\end{isabellebody}%
%%% Local Variables:
%%% mode: latex
%%% TeX-master: "root"
%%% End:


%%% Local Variables:
%%% mode: latex
%%% TeX-master: "root"
%%% End:


\bibliographystyle{plain}
\bibliography{root}

\end{document}
