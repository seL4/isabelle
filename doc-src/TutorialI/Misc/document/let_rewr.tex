%
\begin{isabellebody}%
%
\isamarkupsubsubsection{Simplifying let-expressions}
%
\begin{isamarkuptext}%
\index{simplification!of let-expressions}
Proving a goal containing \isaindex{let}-expressions almost invariably
requires the \isa{let}-con\-structs to be expanded at some point. Since
\isa{let}-\isa{in} is just syntactic sugar for a predefined constant
(called \isa{Let}), expanding \isa{let}-constructs means rewriting with
\isa{Let{\isacharunderscore}def}:%
\end{isamarkuptext}%
\isacommand{lemma}\ {\isachardoublequote}{\isacharparenleft}let\ xs\ {\isacharequal}\ {\isacharbrackleft}{\isacharbrackright}\ in\ xs{\isacharat}ys{\isacharat}xs{\isacharparenright}\ {\isacharequal}\ ys{\isachardoublequote}\isanewline
\isacommand{by}{\isacharparenleft}simp\ add{\isacharcolon}\ Let{\isacharunderscore}def{\isacharparenright}%
\begin{isamarkuptext}%
If, in a particular context, there is no danger of a combinatorial explosion
of nested \isa{let}s one could even add \isa{Let{\isacharunderscore}def} permanently:%
\end{isamarkuptext}%
\isacommand{lemmas}\ {\isacharbrackleft}simp{\isacharbrackright}\ {\isacharequal}\ Let{\isacharunderscore}def\end{isabellebody}%
%%% Local Variables:
%%% mode: latex
%%% TeX-master: "root"
%%% End:
