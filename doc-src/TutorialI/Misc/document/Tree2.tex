%
\begin{isabellebody}%
\def\isabellecontext{Tree{\isadigit{2}}}%
\isamarkupfalse%
%
\begin{isamarkuptext}%
\noindent In Exercise~\ref{ex:Tree} we defined a function
\isa{flatten} from trees to lists. The straightforward version of
\isa{flatten} is based on \isa{{\isacharat}} and is thus, like \isa{rev},
quadratic. A linear time version of \isa{flatten} again reqires an extra
argument, the accumulator:%
\end{isamarkuptext}%
\isamarkuptrue%
\isacommand{consts}\ flatten{\isadigit{2}}\ {\isacharcolon}{\isacharcolon}\ {\isachardoublequote}{\isacharprime}a\ tree\ {\isacharequal}{\isachargreater}\ {\isacharprime}a\ list\ {\isacharequal}{\isachargreater}\ {\isacharprime}a\ list{\isachardoublequote}\isamarkupfalse%
\isamarkupfalse%
%
\begin{isamarkuptext}%
\noindent Define \isa{flatten{\isadigit{2}}} and prove%
\end{isamarkuptext}%
\isamarkuptrue%
\isamarkupfalse%
\isamarkupfalse%
\isamarkupfalse%
\isacommand{lemma}\ {\isachardoublequote}flatten{\isadigit{2}}\ t\ {\isacharbrackleft}{\isacharbrackright}\ {\isacharequal}\ flatten\ t{\isachardoublequote}\isamarkupfalse%
\isamarkupfalse%
\isamarkupfalse%
\end{isabellebody}%
%%% Local Variables:
%%% mode: latex
%%% TeX-master: "root"
%%% End:
