%
\begin{isabellebody}%
\def\isabellecontext{prime{\isacharunderscore}def}%
%
\begin{isamarkuptext}%
\begin{warn}
A common mistake when writing definitions is to introduce extra free
variables on the right-hand side as in the following fictitious definition:
\begin{isabelle}%
\ \ \ \ \ {\isachardoublequote}prime\ p\ {\isasymequiv}\ {\isadigit{1}}\ {\isacharless}\ p\ {\isasymand}\ {\isacharparenleft}m\ dvd\ p\ {\isasymlongrightarrow}\ m\ {\isacharequal}\ {\isadigit{1}}\ {\isasymor}\ m\ {\isacharequal}\ p{\isacharparenright}{\isachardoublequote}%
\end{isabelle}
where \isa{dvd} means ``divides''.
Isabelle rejects this ``definition'' because of the extra \isa{m} on the
right-hand side, which would introduce an inconsistency (why?). What you
should have written is
\begin{isabelle}%
\ \ \ \ \ {\isachardoublequote}prime\ p\ {\isasymequiv}\ {\isadigit{1}}\ {\isacharless}\ p\ {\isasymand}\ {\isacharparenleft}{\isasymforall}m{\isachardot}\ m\ dvd\ p\ {\isasymlongrightarrow}\ m\ {\isacharequal}\ {\isadigit{1}}\ {\isasymor}\ m\ {\isacharequal}\ p{\isacharparenright}{\isachardoublequote}%
\end{isabelle}
\end{warn}%
\end{isamarkuptext}%
\end{isabellebody}%
%%% Local Variables:
%%% mode: latex
%%% TeX-master: "root"
%%% End:
