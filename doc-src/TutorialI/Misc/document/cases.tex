\begin{isabelle}%
%
\isamarkupsubsection{Structural induction and case distinction}
%
\begin{isamarkuptext}%
\indexbold{structural induction}
\indexbold{induction!structural}
\indexbold{case distinction}
Almost all the basic laws about a datatype are applied automatically during
simplification. Only induction is invoked by hand via \isaindex{induct_tac},
which works for any datatype. In some cases, induction is overkill and a case
distinction over all constructors of the datatype suffices. This is performed
by \isaindexbold{case_tac}. A trivial example:%
\end{isamarkuptext}%
\isacommand{lemma}\ {\isachardoublequote}{\isacharparenleft}case\ xs\ of\ {\isacharbrackleft}{\isacharbrackright}\ {\isasymRightarrow}\ {\isacharbrackleft}{\isacharbrackright}\ {\isacharbar}\ y{\isacharhash}ys\ {\isasymRightarrow}\ xs{\isacharparenright}\ {\isacharequal}\ xs{\isachardoublequote}\isanewline
\isacommand{apply}{\isacharparenleft}case{\isacharunderscore}tac\ xs{\isacharparenright}%
\begin{isamarkuptxt}%
\noindent
results in the proof state
\begin{isabellepar}%
~1.~xs~=~[]~{\isasymLongrightarrow}~(case~xs~of~[]~{\isasymRightarrow}~[]~|~y~\#~ys~{\isasymRightarrow}~xs)~=~xs\isanewline
~2.~{\isasymAnd}a~list.~xs=a\#list~{\isasymLongrightarrow}~(case~xs~of~[]~{\isasymRightarrow}~[]~|~y\#ys~{\isasymRightarrow}~xs)~=~xs%
\end{isabellepar}%
which is solved automatically:%
\end{isamarkuptxt}%
\isacommand{by}{\isacharparenleft}auto{\isacharparenright}%
\begin{isamarkuptext}%
Note that we do not need to give a lemma a name if we do not intend to refer
to it explicitly in the future.%
\end{isamarkuptext}%
\end{isabelle}%
%%% Local Variables:
%%% mode: latex
%%% TeX-master: "root"
%%% End:
