%
\begin{isabellebody}%
\def\isabellecontext{Trie}%
%
\isadelimtheory
%
\endisadelimtheory
%
\isatagtheory
%
\endisatagtheory
{\isafoldtheory}%
%
\isadelimtheory
%
\endisadelimtheory
%
\begin{isamarkuptext}%
To minimize running time, each node of a trie should contain an array that maps
letters to subtries. We have chosen a
representation where the subtries are held in an association list, i.e.\ a
list of (letter,trie) pairs.  Abstracting over the alphabet \isa{{\isaliteral{27}{\isacharprime}}a} and the
values \isa{{\isaliteral{27}{\isacharprime}}v} we define a trie as follows:%
\end{isamarkuptext}%
\isamarkuptrue%
\isacommand{datatype}\isamarkupfalse%
\ {\isaliteral{28}{\isacharparenleft}}{\isaliteral{27}{\isacharprime}}a{\isaliteral{2C}{\isacharcomma}}{\isaliteral{27}{\isacharprime}}v{\isaliteral{29}{\isacharparenright}}trie\ {\isaliteral{3D}{\isacharequal}}\ Trie\ \ {\isaliteral{22}{\isachardoublequoteopen}}{\isaliteral{27}{\isacharprime}}v\ option{\isaliteral{22}{\isachardoublequoteclose}}\ \ {\isaliteral{22}{\isachardoublequoteopen}}{\isaliteral{28}{\isacharparenleft}}{\isaliteral{27}{\isacharprime}}a\ {\isaliteral{2A}{\isacharasterisk}}\ {\isaliteral{28}{\isacharparenleft}}{\isaliteral{27}{\isacharprime}}a{\isaliteral{2C}{\isacharcomma}}{\isaliteral{27}{\isacharprime}}v{\isaliteral{29}{\isacharparenright}}trie{\isaliteral{29}{\isacharparenright}}list{\isaliteral{22}{\isachardoublequoteclose}}%
\begin{isamarkuptext}%
\noindent
\index{datatypes!and nested recursion}%
The first component is the optional value, the second component the
association list of subtries.  This is an example of nested recursion involving products,
which is fine because products are datatypes as well.
We define two selector functions:%
\end{isamarkuptext}%
\isamarkuptrue%
\isacommand{primrec}\isamarkupfalse%
\ {\isaliteral{22}{\isachardoublequoteopen}}value{\isaliteral{22}{\isachardoublequoteclose}}\ {\isaliteral{3A}{\isacharcolon}}{\isaliteral{3A}{\isacharcolon}}\ {\isaliteral{22}{\isachardoublequoteopen}}{\isaliteral{28}{\isacharparenleft}}{\isaliteral{27}{\isacharprime}}a{\isaliteral{2C}{\isacharcomma}}{\isaliteral{27}{\isacharprime}}v{\isaliteral{29}{\isacharparenright}}trie\ {\isaliteral{5C3C52696768746172726F773E}{\isasymRightarrow}}\ {\isaliteral{27}{\isacharprime}}v\ option{\isaliteral{22}{\isachardoublequoteclose}}\ \isakeyword{where}\isanewline
{\isaliteral{22}{\isachardoublequoteopen}}value{\isaliteral{28}{\isacharparenleft}}Trie\ ov\ al{\isaliteral{29}{\isacharparenright}}\ {\isaliteral{3D}{\isacharequal}}\ ov{\isaliteral{22}{\isachardoublequoteclose}}\isanewline
\isacommand{primrec}\isamarkupfalse%
\ alist\ {\isaliteral{3A}{\isacharcolon}}{\isaliteral{3A}{\isacharcolon}}\ {\isaliteral{22}{\isachardoublequoteopen}}{\isaliteral{28}{\isacharparenleft}}{\isaliteral{27}{\isacharprime}}a{\isaliteral{2C}{\isacharcomma}}{\isaliteral{27}{\isacharprime}}v{\isaliteral{29}{\isacharparenright}}trie\ {\isaliteral{5C3C52696768746172726F773E}{\isasymRightarrow}}\ {\isaliteral{28}{\isacharparenleft}}{\isaliteral{27}{\isacharprime}}a\ {\isaliteral{2A}{\isacharasterisk}}\ {\isaliteral{28}{\isacharparenleft}}{\isaliteral{27}{\isacharprime}}a{\isaliteral{2C}{\isacharcomma}}{\isaliteral{27}{\isacharprime}}v{\isaliteral{29}{\isacharparenright}}trie{\isaliteral{29}{\isacharparenright}}list{\isaliteral{22}{\isachardoublequoteclose}}\ \isakeyword{where}\isanewline
{\isaliteral{22}{\isachardoublequoteopen}}alist{\isaliteral{28}{\isacharparenleft}}Trie\ ov\ al{\isaliteral{29}{\isacharparenright}}\ {\isaliteral{3D}{\isacharequal}}\ al{\isaliteral{22}{\isachardoublequoteclose}}%
\begin{isamarkuptext}%
\noindent
Association lists come with a generic lookup function.  Its result
involves type \isa{option} because a lookup can fail:%
\end{isamarkuptext}%
\isamarkuptrue%
\isacommand{primrec}\isamarkupfalse%
\ assoc\ {\isaliteral{3A}{\isacharcolon}}{\isaliteral{3A}{\isacharcolon}}\ {\isaliteral{22}{\isachardoublequoteopen}}{\isaliteral{28}{\isacharparenleft}}{\isaliteral{27}{\isacharprime}}key\ {\isaliteral{2A}{\isacharasterisk}}\ {\isaliteral{27}{\isacharprime}}val{\isaliteral{29}{\isacharparenright}}list\ {\isaliteral{5C3C52696768746172726F773E}{\isasymRightarrow}}\ {\isaliteral{27}{\isacharprime}}key\ {\isaliteral{5C3C52696768746172726F773E}{\isasymRightarrow}}\ {\isaliteral{27}{\isacharprime}}val\ option{\isaliteral{22}{\isachardoublequoteclose}}\ \isakeyword{where}\isanewline
{\isaliteral{22}{\isachardoublequoteopen}}assoc\ {\isaliteral{5B}{\isacharbrackleft}}{\isaliteral{5D}{\isacharbrackright}}\ x\ {\isaliteral{3D}{\isacharequal}}\ None{\isaliteral{22}{\isachardoublequoteclose}}\ {\isaliteral{7C}{\isacharbar}}\isanewline
{\isaliteral{22}{\isachardoublequoteopen}}assoc\ {\isaliteral{28}{\isacharparenleft}}p{\isaliteral{23}{\isacharhash}}ps{\isaliteral{29}{\isacharparenright}}\ x\ {\isaliteral{3D}{\isacharequal}}\isanewline
\ \ \ {\isaliteral{28}{\isacharparenleft}}let\ {\isaliteral{28}{\isacharparenleft}}a{\isaliteral{2C}{\isacharcomma}}b{\isaliteral{29}{\isacharparenright}}\ {\isaliteral{3D}{\isacharequal}}\ p\ in\ if\ a{\isaliteral{3D}{\isacharequal}}x\ then\ Some\ b\ else\ assoc\ ps\ x{\isaliteral{29}{\isacharparenright}}{\isaliteral{22}{\isachardoublequoteclose}}%
\begin{isamarkuptext}%
Now we can define the lookup function for tries. It descends into the trie
examining the letters of the search string one by one. As
recursion on lists is simpler than on tries, let us express this as primitive
recursion on the search string argument:%
\end{isamarkuptext}%
\isamarkuptrue%
\isacommand{primrec}\isamarkupfalse%
\ lookup\ {\isaliteral{3A}{\isacharcolon}}{\isaliteral{3A}{\isacharcolon}}\ {\isaliteral{22}{\isachardoublequoteopen}}{\isaliteral{28}{\isacharparenleft}}{\isaliteral{27}{\isacharprime}}a{\isaliteral{2C}{\isacharcomma}}{\isaliteral{27}{\isacharprime}}v{\isaliteral{29}{\isacharparenright}}trie\ {\isaliteral{5C3C52696768746172726F773E}{\isasymRightarrow}}\ {\isaliteral{27}{\isacharprime}}a\ list\ {\isaliteral{5C3C52696768746172726F773E}{\isasymRightarrow}}\ {\isaliteral{27}{\isacharprime}}v\ option{\isaliteral{22}{\isachardoublequoteclose}}\ \isakeyword{where}\isanewline
{\isaliteral{22}{\isachardoublequoteopen}}lookup\ t\ {\isaliteral{5B}{\isacharbrackleft}}{\isaliteral{5D}{\isacharbrackright}}\ {\isaliteral{3D}{\isacharequal}}\ value\ t{\isaliteral{22}{\isachardoublequoteclose}}\ {\isaliteral{7C}{\isacharbar}}\isanewline
{\isaliteral{22}{\isachardoublequoteopen}}lookup\ t\ {\isaliteral{28}{\isacharparenleft}}a{\isaliteral{23}{\isacharhash}}as{\isaliteral{29}{\isacharparenright}}\ {\isaliteral{3D}{\isacharequal}}\ {\isaliteral{28}{\isacharparenleft}}case\ assoc\ {\isaliteral{28}{\isacharparenleft}}alist\ t{\isaliteral{29}{\isacharparenright}}\ a\ of\isanewline
\ \ \ \ \ \ \ \ \ \ \ \ \ \ \ \ \ \ \ \ \ \ None\ {\isaliteral{5C3C52696768746172726F773E}{\isasymRightarrow}}\ None\isanewline
\ \ \ \ \ \ \ \ \ \ \ \ \ \ \ \ \ \ \ \ {\isaliteral{7C}{\isacharbar}}\ Some\ at\ {\isaliteral{5C3C52696768746172726F773E}{\isasymRightarrow}}\ lookup\ at\ as{\isaliteral{29}{\isacharparenright}}{\isaliteral{22}{\isachardoublequoteclose}}%
\begin{isamarkuptext}%
As a first simple property we prove that looking up a string in the empty
trie \isa{Trie\ None\ {\isaliteral{5B}{\isacharbrackleft}}{\isaliteral{5D}{\isacharbrackright}}} always returns \isa{None}. The proof merely
distinguishes the two cases whether the search string is empty or not:%
\end{isamarkuptext}%
\isamarkuptrue%
\isacommand{lemma}\isamarkupfalse%
\ {\isaliteral{5B}{\isacharbrackleft}}simp{\isaliteral{5D}{\isacharbrackright}}{\isaliteral{3A}{\isacharcolon}}\ {\isaliteral{22}{\isachardoublequoteopen}}lookup\ {\isaliteral{28}{\isacharparenleft}}Trie\ None\ {\isaliteral{5B}{\isacharbrackleft}}{\isaliteral{5D}{\isacharbrackright}}{\isaliteral{29}{\isacharparenright}}\ as\ {\isaliteral{3D}{\isacharequal}}\ None{\isaliteral{22}{\isachardoublequoteclose}}\isanewline
%
\isadelimproof
%
\endisadelimproof
%
\isatagproof
\isacommand{apply}\isamarkupfalse%
{\isaliteral{28}{\isacharparenleft}}case{\isaliteral{5F}{\isacharunderscore}}tac\ as{\isaliteral{2C}{\isacharcomma}}\ simp{\isaliteral{5F}{\isacharunderscore}}all{\isaliteral{29}{\isacharparenright}}\isanewline
\isacommand{done}\isamarkupfalse%
%
\endisatagproof
{\isafoldproof}%
%
\isadelimproof
%
\endisadelimproof
%
\begin{isamarkuptext}%
Things begin to get interesting with the definition of an update function
that adds a new (string, value) pair to a trie, overwriting the old value
associated with that string:%
\end{isamarkuptext}%
\isamarkuptrue%
\isacommand{primrec}\isamarkupfalse%
\ update{\isaliteral{3A}{\isacharcolon}}{\isaliteral{3A}{\isacharcolon}}\ {\isaliteral{22}{\isachardoublequoteopen}}{\isaliteral{28}{\isacharparenleft}}{\isaliteral{27}{\isacharprime}}a{\isaliteral{2C}{\isacharcomma}}{\isaliteral{27}{\isacharprime}}v{\isaliteral{29}{\isacharparenright}}trie\ {\isaliteral{5C3C52696768746172726F773E}{\isasymRightarrow}}\ {\isaliteral{27}{\isacharprime}}a\ list\ {\isaliteral{5C3C52696768746172726F773E}{\isasymRightarrow}}\ {\isaliteral{27}{\isacharprime}}v\ {\isaliteral{5C3C52696768746172726F773E}{\isasymRightarrow}}\ {\isaliteral{28}{\isacharparenleft}}{\isaliteral{27}{\isacharprime}}a{\isaliteral{2C}{\isacharcomma}}{\isaliteral{27}{\isacharprime}}v{\isaliteral{29}{\isacharparenright}}trie{\isaliteral{22}{\isachardoublequoteclose}}\ \isakeyword{where}\isanewline
{\isaliteral{22}{\isachardoublequoteopen}}update\ t\ {\isaliteral{5B}{\isacharbrackleft}}{\isaliteral{5D}{\isacharbrackright}}\ \ \ \ \ v\ {\isaliteral{3D}{\isacharequal}}\ Trie\ {\isaliteral{28}{\isacharparenleft}}Some\ v{\isaliteral{29}{\isacharparenright}}\ {\isaliteral{28}{\isacharparenleft}}alist\ t{\isaliteral{29}{\isacharparenright}}{\isaliteral{22}{\isachardoublequoteclose}}\ {\isaliteral{7C}{\isacharbar}}\isanewline
{\isaliteral{22}{\isachardoublequoteopen}}update\ t\ {\isaliteral{28}{\isacharparenleft}}a{\isaliteral{23}{\isacharhash}}as{\isaliteral{29}{\isacharparenright}}\ v\ {\isaliteral{3D}{\isacharequal}}\isanewline
\ \ \ {\isaliteral{28}{\isacharparenleft}}let\ tt\ {\isaliteral{3D}{\isacharequal}}\ {\isaliteral{28}{\isacharparenleft}}case\ assoc\ {\isaliteral{28}{\isacharparenleft}}alist\ t{\isaliteral{29}{\isacharparenright}}\ a\ of\isanewline
\ \ \ \ \ \ \ \ \ \ \ \ \ \ \ \ None\ {\isaliteral{5C3C52696768746172726F773E}{\isasymRightarrow}}\ Trie\ None\ {\isaliteral{5B}{\isacharbrackleft}}{\isaliteral{5D}{\isacharbrackright}}\ {\isaliteral{7C}{\isacharbar}}\ Some\ at\ {\isaliteral{5C3C52696768746172726F773E}{\isasymRightarrow}}\ at{\isaliteral{29}{\isacharparenright}}\isanewline
\ \ \ \ in\ Trie\ {\isaliteral{28}{\isacharparenleft}}value\ t{\isaliteral{29}{\isacharparenright}}\ {\isaliteral{28}{\isacharparenleft}}{\isaliteral{28}{\isacharparenleft}}a{\isaliteral{2C}{\isacharcomma}}update\ tt\ as\ v{\isaliteral{29}{\isacharparenright}}\ {\isaliteral{23}{\isacharhash}}\ alist\ t{\isaliteral{29}{\isacharparenright}}{\isaliteral{29}{\isacharparenright}}{\isaliteral{22}{\isachardoublequoteclose}}%
\begin{isamarkuptext}%
\noindent
The base case is obvious. In the recursive case the subtrie
\isa{tt} associated with the first letter \isa{a} is extracted,
recursively updated, and then placed in front of the association list.
The old subtrie associated with \isa{a} is still in the association list
but no longer accessible via \isa{assoc}. Clearly, there is room here for
optimizations!

Before we start on any proofs about \isa{update} we tell the simplifier to
expand all \isa{let}s and to split all \isa{case}-constructs over
options:%
\end{isamarkuptext}%
\isamarkuptrue%
\isacommand{declare}\isamarkupfalse%
\ Let{\isaliteral{5F}{\isacharunderscore}}def{\isaliteral{5B}{\isacharbrackleft}}simp{\isaliteral{5D}{\isacharbrackright}}\ option{\isaliteral{2E}{\isachardot}}split{\isaliteral{5B}{\isacharbrackleft}}split{\isaliteral{5D}{\isacharbrackright}}%
\begin{isamarkuptext}%
\noindent
The reason becomes clear when looking (probably after a failed proof
attempt) at the body of \isa{update}: it contains both
\isa{let} and a case distinction over type \isa{option}.

Our main goal is to prove the correct interaction of \isa{update} and
\isa{lookup}:%
\end{isamarkuptext}%
\isamarkuptrue%
\isacommand{theorem}\isamarkupfalse%
\ {\isaliteral{22}{\isachardoublequoteopen}}{\isaliteral{5C3C666F72616C6C3E}{\isasymforall}}t\ v\ bs{\isaliteral{2E}{\isachardot}}\ lookup\ {\isaliteral{28}{\isacharparenleft}}update\ t\ as\ v{\isaliteral{29}{\isacharparenright}}\ bs\ {\isaliteral{3D}{\isacharequal}}\isanewline
\ \ \ \ \ \ \ \ \ \ \ \ \ \ \ \ \ \ \ \ {\isaliteral{28}{\isacharparenleft}}if\ as{\isaliteral{3D}{\isacharequal}}bs\ then\ Some\ v\ else\ lookup\ t\ bs{\isaliteral{29}{\isacharparenright}}{\isaliteral{22}{\isachardoublequoteclose}}%
\isadelimproof
%
\endisadelimproof
%
\isatagproof
%
\begin{isamarkuptxt}%
\noindent
Our plan is to induct on \isa{as}; hence the remaining variables are
quantified. From the definitions it is clear that induction on either
\isa{as} or \isa{bs} is required. The choice of \isa{as} is 
guided by the intuition that simplification of \isa{lookup} might be easier
if \isa{update} has already been simplified, which can only happen if
\isa{as} is instantiated.
The start of the proof is conventional:%
\end{isamarkuptxt}%
\isamarkuptrue%
\isacommand{apply}\isamarkupfalse%
{\isaliteral{28}{\isacharparenleft}}induct{\isaliteral{5F}{\isacharunderscore}}tac\ as{\isaliteral{2C}{\isacharcomma}}\ auto{\isaliteral{29}{\isacharparenright}}%
\begin{isamarkuptxt}%
\noindent
Unfortunately, this time we are left with three intimidating looking subgoals:
\begin{isabelle}
~1.~\dots~{\isasymLongrightarrow}~lookup~\dots~bs~=~lookup~t~bs\isanewline
~2.~\dots~{\isasymLongrightarrow}~lookup~\dots~bs~=~lookup~t~bs\isanewline
~3.~\dots~{\isasymLongrightarrow}~lookup~\dots~bs~=~lookup~t~bs
\end{isabelle}
Clearly, if we want to make headway we have to instantiate \isa{bs} as
well now. It turns out that instead of induction, case distinction
suffices:%
\end{isamarkuptxt}%
\isamarkuptrue%
\isacommand{apply}\isamarkupfalse%
{\isaliteral{28}{\isacharparenleft}}case{\isaliteral{5F}{\isacharunderscore}}tac{\isaliteral{5B}{\isacharbrackleft}}{\isaliteral{21}{\isacharbang}}{\isaliteral{5D}{\isacharbrackright}}\ bs{\isaliteral{2C}{\isacharcomma}}\ auto{\isaliteral{29}{\isacharparenright}}\isanewline
\isacommand{done}\isamarkupfalse%
%
\endisatagproof
{\isafoldproof}%
%
\isadelimproof
%
\endisadelimproof
%
\begin{isamarkuptext}%
\noindent
\index{subgoal numbering}%
All methods ending in \isa{tac} take an optional first argument that
specifies the range of subgoals they are applied to, where \isa{{\isaliteral{5B}{\isacharbrackleft}}{\isaliteral{21}{\isacharbang}}{\isaliteral{5D}{\isacharbrackright}}} means
all subgoals, i.e.\ \isa{{\isaliteral{5B}{\isacharbrackleft}}{\isadigit{1}}{\isaliteral{2D}{\isacharminus}}{\isadigit{3}}{\isaliteral{5D}{\isacharbrackright}}} in our case. Individual subgoal numbers,
e.g. \isa{{\isaliteral{5B}{\isacharbrackleft}}{\isadigit{2}}{\isaliteral{5D}{\isacharbrackright}}} are also allowed.

This proof may look surprisingly straightforward. However, note that this
comes at a cost: the proof script is unreadable because the intermediate
proof states are invisible, and we rely on the (possibly brittle) magic of
\isa{auto} (\isa{simp{\isaliteral{5F}{\isacharunderscore}}all} will not do --- try it) to split the subgoals
of the induction up in such a way that case distinction on \isa{bs} makes
sense and solves the proof. 

\begin{exercise}
  Modify \isa{update} (and its type) such that it allows both insertion and
  deletion of entries with a single function.  Prove the corresponding version 
  of the main theorem above.
  Optimize your function such that it shrinks tries after
  deletion if possible.
\end{exercise}

\begin{exercise}
  Write an improved version of \isa{update} that does not suffer from the
  space leak (pointed out above) caused by not deleting overwritten entries
  from the association list. Prove the main theorem for your improved
  \isa{update}.
\end{exercise}

\begin{exercise}
  Conceptually, each node contains a mapping from letters to optional
  subtries. Above we have implemented this by means of an association
  list. Replay the development replacing \isa{{\isaliteral{28}{\isacharparenleft}}{\isaliteral{27}{\isacharprime}}a\ {\isaliteral{5C3C74696D65733E}{\isasymtimes}}\ {\isaliteral{28}{\isacharparenleft}}{\isaliteral{27}{\isacharprime}}a{\isaliteral{2C}{\isacharcomma}}\ {\isaliteral{27}{\isacharprime}}v{\isaliteral{29}{\isacharparenright}}\ trie{\isaliteral{29}{\isacharparenright}}\ list}
  with \isa{{\isaliteral{27}{\isacharprime}}a\ {\isaliteral{5C3C52696768746172726F773E}{\isasymRightarrow}}\ {\isaliteral{28}{\isacharparenleft}}{\isaliteral{27}{\isacharprime}}a{\isaliteral{2C}{\isacharcomma}}\ {\isaliteral{27}{\isacharprime}}v{\isaliteral{29}{\isacharparenright}}\ trie\ option}.
\end{exercise}%
\end{isamarkuptext}%
\isamarkuptrue%
%
\isadelimproof
%
\endisadelimproof
%
\isatagproof
%
\endisatagproof
{\isafoldproof}%
%
\isadelimproof
%
\endisadelimproof
%
\isadelimproof
%
\endisadelimproof
%
\isatagproof
%
\endisatagproof
{\isafoldproof}%
%
\isadelimproof
%
\endisadelimproof
%
\isadelimproof
%
\endisadelimproof
%
\isatagproof
%
\endisatagproof
{\isafoldproof}%
%
\isadelimproof
%
\endisadelimproof
%
\isadelimproof
%
\endisadelimproof
%
\isatagproof
%
\endisatagproof
{\isafoldproof}%
%
\isadelimproof
%
\endisadelimproof
%
\isadelimproof
%
\endisadelimproof
%
\isatagproof
%
\endisatagproof
{\isafoldproof}%
%
\isadelimproof
%
\endisadelimproof
%
\isadelimtheory
%
\endisadelimtheory
%
\isatagtheory
%
\endisatagtheory
{\isafoldtheory}%
%
\isadelimtheory
%
\endisadelimtheory
\end{isabellebody}%
%%% Local Variables:
%%% mode: latex
%%% TeX-master: "root"
%%% End:
