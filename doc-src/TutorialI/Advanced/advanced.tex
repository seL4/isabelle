\chapter{Advanced Simplification, Recursion and Induction}

Although we have already learned a lot about simplification, recursion and
induction, there are some advanced proof techniques that we have not covered
yet and which are worth knowing about if you intend to beome a serious
(human) theorem prover. The three sections of this chapter are almost
independent of each other and can be read in any order. Only the notion of
\emph{congruence rules}, introduced in the section on simplification, is
required for parts of the section on recursion.

%
\begin{isabellebody}%
\def\isabellecontext{simp}%
%
\isamarkupsubsubsection{Simplification rules%
}
%
\begin{isamarkuptext}%
\indexbold{simplification rule}
To facilitate simplification, theorems can be declared to be simplification
rules (with the help of the attribute \isa{{\isacharbrackleft}simp{\isacharbrackright}}\index{*simp
  (attribute)}), in which case proofs by simplification make use of these
rules automatically. In addition the constructs \isacommand{datatype} and
\isacommand{primrec} (and a few others) invisibly declare useful
simplification rules. Explicit definitions are \emph{not} declared
simplification rules automatically!

Not merely equations but pretty much any theorem can become a simplification
rule. The simplifier will try to make sense of it.  For example, a theorem
\isa{{\isasymnot}\ P} is automatically turned into \isa{P\ {\isacharequal}\ False}. The details
are explained in \S\ref{sec:SimpHow}.

The simplification attribute of theorems can be turned on and off as follows:
\begin{quote}
\isacommand{declare} \textit{theorem-name}\isa{{\isacharbrackleft}simp{\isacharbrackright}}\\
\isacommand{declare} \textit{theorem-name}\isa{{\isacharbrackleft}simp\ del{\isacharbrackright}}
\end{quote}
As a rule of thumb, equations that really simplify (like \isa{rev\ {\isacharparenleft}rev\ xs{\isacharparenright}\ {\isacharequal}\ xs} and \isa{xs\ {\isacharat}\ {\isacharbrackleft}{\isacharbrackright}\ {\isacharequal}\ xs}) should be made simplification
rules.  Those of a more specific nature (e.g.\ distributivity laws, which
alter the structure of terms considerably) should only be used selectively,
i.e.\ they should not be default simplification rules.  Conversely, it may
also happen that a simplification rule needs to be disabled in certain
proofs.  Frequent changes in the simplification status of a theorem may
indicate a badly designed theory.
\begin{warn}
  Simplification may not terminate, for example if both $f(x) = g(x)$ and
  $g(x) = f(x)$ are simplification rules. It is the user's responsibility not
  to include simplification rules that can lead to nontermination, either on
  their own or in combination with other simplification rules.
\end{warn}%
\end{isamarkuptext}%
%
\isamarkupsubsubsection{The simplification method%
}
%
\begin{isamarkuptext}%
\index{*simp (method)|bold}
The general format of the simplification method is
\begin{quote}
\isa{simp} \textit{list of modifiers}
\end{quote}
where the list of \emph{modifiers} helps to fine tune the behaviour and may
be empty. Most if not all of the proofs seen so far could have been performed
with \isa{simp} instead of \isa{auto}, except that \isa{simp} attacks
only the first subgoal and may thus need to be repeated---use
\isaindex{simp_all} to simplify all subgoals.
Note that \isa{simp} fails if nothing changes.%
\end{isamarkuptext}%
%
\isamarkupsubsubsection{Adding and deleting simplification rules%
}
%
\begin{isamarkuptext}%
If a certain theorem is merely needed in a few proofs by simplification,
we do not need to make it a global simplification rule. Instead we can modify
the set of simplification rules used in a simplification step by adding rules
to it and/or deleting rules from it. The two modifiers for this are
\begin{quote}
\isa{add{\isacharcolon}} \textit{list of theorem names}\\
\isa{del{\isacharcolon}} \textit{list of theorem names}
\end{quote}
In case you want to use only a specific list of theorems and ignore all
others:
\begin{quote}
\isa{only{\isacharcolon}} \textit{list of theorem names}
\end{quote}%
\end{isamarkuptext}%
%
\isamarkupsubsubsection{Assumptions%
}
%
\begin{isamarkuptext}%
\index{simplification!with/of assumptions}
By default, assumptions are part of the simplification process: they are used
as simplification rules and are simplified themselves. For example:%
\end{isamarkuptext}%
\isacommand{lemma}\ {\isachardoublequote}{\isasymlbrakk}\ xs\ {\isacharat}\ zs\ {\isacharequal}\ ys\ {\isacharat}\ xs{\isacharsemicolon}\ {\isacharbrackleft}{\isacharbrackright}\ {\isacharat}\ xs\ {\isacharequal}\ {\isacharbrackleft}{\isacharbrackright}\ {\isacharat}\ {\isacharbrackleft}{\isacharbrackright}\ {\isasymrbrakk}\ {\isasymLongrightarrow}\ ys\ {\isacharequal}\ zs{\isachardoublequote}\isanewline
\isacommand{apply}\ simp\isanewline
\isacommand{done}%
\begin{isamarkuptext}%
\noindent
The second assumption simplifies to \isa{xs\ {\isacharequal}\ {\isacharbrackleft}{\isacharbrackright}}, which in turn
simplifies the first assumption to \isa{zs\ {\isacharequal}\ ys}, thus reducing the
conclusion to \isa{ys\ {\isacharequal}\ ys} and hence to \isa{True}.

In some cases this may be too much of a good thing and may lead to
nontermination:%
\end{isamarkuptext}%
\isacommand{lemma}\ {\isachardoublequote}{\isasymforall}x{\isachardot}\ f\ x\ {\isacharequal}\ g\ {\isacharparenleft}f\ {\isacharparenleft}g\ x{\isacharparenright}{\isacharparenright}\ {\isasymLongrightarrow}\ f\ {\isacharbrackleft}{\isacharbrackright}\ {\isacharequal}\ f\ {\isacharbrackleft}{\isacharbrackright}\ {\isacharat}\ {\isacharbrackleft}{\isacharbrackright}{\isachardoublequote}%
\begin{isamarkuptxt}%
\noindent
cannot be solved by an unmodified application of \isa{simp} because the
simplification rule \isa{f\ x\ {\isacharequal}\ g\ {\isacharparenleft}f\ {\isacharparenleft}g\ x{\isacharparenright}{\isacharparenright}} extracted from the assumption
does not terminate. Isabelle notices certain simple forms of
nontermination but not this one. The problem can be circumvented by
explicitly telling the simplifier to ignore the assumptions:%
\end{isamarkuptxt}%
\isacommand{apply}{\isacharparenleft}simp\ {\isacharparenleft}no{\isacharunderscore}asm{\isacharparenright}{\isacharparenright}\isanewline
\isacommand{done}%
\begin{isamarkuptext}%
\noindent
There are three options that influence the treatment of assumptions:
\begin{description}
\item[\isa{{\isacharparenleft}no{\isacharunderscore}asm{\isacharparenright}}]\indexbold{*no_asm}
 means that assumptions are completely ignored.
\item[\isa{{\isacharparenleft}no{\isacharunderscore}asm{\isacharunderscore}simp{\isacharparenright}}]\indexbold{*no_asm_simp}
 means that the assumptions are not simplified but
  are used in the simplification of the conclusion.
\item[\isa{{\isacharparenleft}no{\isacharunderscore}asm{\isacharunderscore}use{\isacharparenright}}]\indexbold{*no_asm_use}
 means that the assumptions are simplified but are not
  used in the simplification of each other or the conclusion.
\end{description}
Neither \isa{{\isacharparenleft}no{\isacharunderscore}asm{\isacharunderscore}simp{\isacharparenright}} nor \isa{{\isacharparenleft}no{\isacharunderscore}asm{\isacharunderscore}use{\isacharparenright}} allow to simplify
the above problematic subgoal.

Note that only one of the above options is allowed, and it must precede all
other arguments.%
\end{isamarkuptext}%
%
\isamarkupsubsubsection{Rewriting with definitions%
}
%
\begin{isamarkuptext}%
\index{simplification!with definitions}
Constant definitions (\S\ref{sec:ConstDefinitions}) can
be used as simplification rules, but by default they are not.  Hence the
simplifier does not expand them automatically, just as it should be:
definitions are introduced for the purpose of abbreviating complex
concepts. Of course we need to expand the definitions initially to derive
enough lemmas that characterize the concept sufficiently for us to forget the
original definition. For example, given%
\end{isamarkuptext}%
\isacommand{constdefs}\ exor\ {\isacharcolon}{\isacharcolon}\ {\isachardoublequote}bool\ {\isasymRightarrow}\ bool\ {\isasymRightarrow}\ bool{\isachardoublequote}\isanewline
\ \ \ \ \ \ \ \ \ {\isachardoublequote}exor\ A\ B\ {\isasymequiv}\ {\isacharparenleft}A\ {\isasymand}\ {\isasymnot}B{\isacharparenright}\ {\isasymor}\ {\isacharparenleft}{\isasymnot}A\ {\isasymand}\ B{\isacharparenright}{\isachardoublequote}%
\begin{isamarkuptext}%
\noindent
we may want to prove%
\end{isamarkuptext}%
\isacommand{lemma}\ {\isachardoublequote}exor\ A\ {\isacharparenleft}{\isasymnot}A{\isacharparenright}{\isachardoublequote}%
\begin{isamarkuptxt}%
\noindent
Typically, the opening move consists in \emph{unfolding} the definition(s), which we need to
get started, but nothing else:\indexbold{*unfold}\indexbold{definition!unfolding}%
\end{isamarkuptxt}%
\isacommand{apply}{\isacharparenleft}simp\ only{\isacharcolon}exor{\isacharunderscore}def{\isacharparenright}%
\begin{isamarkuptxt}%
\noindent
In this particular case, the resulting goal
\begin{isabelle}%
\ {\isadigit{1}}{\isachardot}\ A\ {\isasymand}\ {\isasymnot}\ {\isasymnot}\ A\ {\isasymor}\ {\isasymnot}\ A\ {\isasymand}\ {\isasymnot}\ A%
\end{isabelle}
can be proved by simplification. Thus we could have proved the lemma outright by%
\end{isamarkuptxt}%
\isacommand{apply}{\isacharparenleft}simp\ add{\isacharcolon}\ exor{\isacharunderscore}def{\isacharparenright}%
\begin{isamarkuptext}%
\noindent
Of course we can also unfold definitions in the middle of a proof.

You should normally not turn a definition permanently into a simplification
rule because this defeats the whole purpose of an abbreviation.

\begin{warn}
  If you have defined $f\,x\,y~\isasymequiv~t$ then you can only expand
  occurrences of $f$ with at least two arguments. Thus it is safer to define
  $f$~\isasymequiv~\isasymlambda$x\,y.\;t$.
\end{warn}%
\end{isamarkuptext}%
%
\isamarkupsubsubsection{Simplifying let-expressions%
}
%
\begin{isamarkuptext}%
\index{simplification!of let-expressions}
Proving a goal containing \isaindex{let}-expressions almost invariably
requires the \isa{let}-con\-structs to be expanded at some point. Since
\isa{let}-\isa{in} is just syntactic sugar for a predefined constant
(called \isa{Let}), expanding \isa{let}-constructs means rewriting with
\isa{Let{\isacharunderscore}def}:%
\end{isamarkuptext}%
\isacommand{lemma}\ {\isachardoublequote}{\isacharparenleft}let\ xs\ {\isacharequal}\ {\isacharbrackleft}{\isacharbrackright}\ in\ xs{\isacharat}ys{\isacharat}xs{\isacharparenright}\ {\isacharequal}\ ys{\isachardoublequote}\isanewline
\isacommand{apply}{\isacharparenleft}simp\ add{\isacharcolon}\ Let{\isacharunderscore}def{\isacharparenright}\isanewline
\isacommand{done}%
\begin{isamarkuptext}%
If, in a particular context, there is no danger of a combinatorial explosion
of nested \isa{let}s one could even simlify with \isa{Let{\isacharunderscore}def} by
default:%
\end{isamarkuptext}%
\isacommand{declare}\ Let{\isacharunderscore}def\ {\isacharbrackleft}simp{\isacharbrackright}%
\isamarkupsubsubsection{Conditional equations%
}
%
\begin{isamarkuptext}%
So far all examples of rewrite rules were equations. The simplifier also
accepts \emph{conditional} equations, for example%
\end{isamarkuptext}%
\isacommand{lemma}\ hd{\isacharunderscore}Cons{\isacharunderscore}tl{\isacharbrackleft}simp{\isacharbrackright}{\isacharcolon}\ {\isachardoublequote}xs\ {\isasymnoteq}\ {\isacharbrackleft}{\isacharbrackright}\ \ {\isasymLongrightarrow}\ \ hd\ xs\ {\isacharhash}\ tl\ xs\ {\isacharequal}\ xs{\isachardoublequote}\isanewline
\isacommand{apply}{\isacharparenleft}case{\isacharunderscore}tac\ xs{\isacharcomma}\ simp{\isacharcomma}\ simp{\isacharparenright}\isanewline
\isacommand{done}%
\begin{isamarkuptext}%
\noindent
Note the use of ``\ttindexboldpos{,}{$Isar}'' to string together a
sequence of methods. Assuming that the simplification rule
\isa{{\isacharparenleft}rev\ xs\ {\isacharequal}\ {\isacharbrackleft}{\isacharbrackright}{\isacharparenright}\ {\isacharequal}\ {\isacharparenleft}xs\ {\isacharequal}\ {\isacharbrackleft}{\isacharbrackright}{\isacharparenright}}
is present as well,%
\end{isamarkuptext}%
\isacommand{lemma}\ {\isachardoublequote}xs\ {\isasymnoteq}\ {\isacharbrackleft}{\isacharbrackright}\ {\isasymLongrightarrow}\ hd{\isacharparenleft}rev\ xs{\isacharparenright}\ {\isacharhash}\ tl{\isacharparenleft}rev\ xs{\isacharparenright}\ {\isacharequal}\ rev\ xs{\isachardoublequote}%
\begin{isamarkuptext}%
\noindent
is proved by plain simplification:
the conditional equation \isa{hd{\isacharunderscore}Cons{\isacharunderscore}tl} above
can simplify \isa{hd\ {\isacharparenleft}rev\ xs{\isacharparenright}\ {\isacharhash}\ tl\ {\isacharparenleft}rev\ xs{\isacharparenright}} to \isa{rev\ xs}
because the corresponding precondition \isa{rev\ xs\ {\isasymnoteq}\ {\isacharbrackleft}{\isacharbrackright}}
simplifies to \isa{xs\ {\isasymnoteq}\ {\isacharbrackleft}{\isacharbrackright}}, which is exactly the local
assumption of the subgoal.%
\end{isamarkuptext}%
%
\isamarkupsubsubsection{Automatic case splits%
}
%
\begin{isamarkuptext}%
\indexbold{case splits}\index{*split|(}
Goals containing \isa{if}-expressions are usually proved by case
distinction on the condition of the \isa{if}. For example the goal%
\end{isamarkuptext}%
\isacommand{lemma}\ {\isachardoublequote}{\isasymforall}xs{\isachardot}\ if\ xs\ {\isacharequal}\ {\isacharbrackleft}{\isacharbrackright}\ then\ rev\ xs\ {\isacharequal}\ {\isacharbrackleft}{\isacharbrackright}\ else\ rev\ xs\ {\isasymnoteq}\ {\isacharbrackleft}{\isacharbrackright}{\isachardoublequote}%
\begin{isamarkuptxt}%
\noindent
can be split by a degenerate form of simplification%
\end{isamarkuptxt}%
\isacommand{apply}{\isacharparenleft}simp\ only{\isacharcolon}\ split{\isacharcolon}\ split{\isacharunderscore}if{\isacharparenright}%
\begin{isamarkuptxt}%
\noindent
\begin{isabelle}%
\ {\isadigit{1}}{\isachardot}\ {\isasymforall}xs{\isachardot}\ {\isacharparenleft}xs\ {\isacharequal}\ {\isacharbrackleft}{\isacharbrackright}\ {\isasymlongrightarrow}\ rev\ xs\ {\isacharequal}\ {\isacharbrackleft}{\isacharbrackright}{\isacharparenright}\ {\isasymand}\ {\isacharparenleft}xs\ {\isasymnoteq}\ {\isacharbrackleft}{\isacharbrackright}\ {\isasymlongrightarrow}\ rev\ xs\ {\isasymnoteq}\ {\isacharbrackleft}{\isacharbrackright}{\isacharparenright}%
\end{isabelle}
where no simplification rules are included (\isa{only{\isacharcolon}} is followed by the
empty list of theorems) but the rule \isaindexbold{split_if} for
splitting \isa{if}s is added (via the modifier \isa{split{\isacharcolon}}). Because
case-splitting on \isa{if}s is almost always the right proof strategy, the
simplifier performs it automatically. Try \isacommand{apply}\isa{{\isacharparenleft}simp{\isacharparenright}}
on the initial goal above.

This splitting idea generalizes from \isa{if} to \isaindex{case}:%
\end{isamarkuptxt}%
\isanewline
\isacommand{lemma}\ {\isachardoublequote}{\isacharparenleft}case\ xs\ of\ {\isacharbrackleft}{\isacharbrackright}\ {\isasymRightarrow}\ zs\ {\isacharbar}\ y{\isacharhash}ys\ {\isasymRightarrow}\ y{\isacharhash}{\isacharparenleft}ys{\isacharat}zs{\isacharparenright}{\isacharparenright}\ {\isacharequal}\ xs{\isacharat}zs{\isachardoublequote}\isanewline
\isacommand{apply}{\isacharparenleft}simp\ only{\isacharcolon}\ split{\isacharcolon}\ list{\isachardot}split{\isacharparenright}%
\begin{isamarkuptxt}%
\begin{isabelle}%
\ {\isadigit{1}}{\isachardot}\ {\isacharparenleft}xs\ {\isacharequal}\ {\isacharbrackleft}{\isacharbrackright}\ {\isasymlongrightarrow}\ zs\ {\isacharequal}\ xs\ {\isacharat}\ zs{\isacharparenright}\ {\isasymand}\isanewline
\ \ \ \ {\isacharparenleft}{\isasymforall}a\ list{\isachardot}\ xs\ {\isacharequal}\ a\ {\isacharhash}\ list\ {\isasymlongrightarrow}\ a\ {\isacharhash}\ list\ {\isacharat}\ zs\ {\isacharequal}\ xs\ {\isacharat}\ zs{\isacharparenright}%
\end{isabelle}
In contrast to \isa{if}-expressions, the simplifier does not split
\isa{case}-expressions by default because this can lead to nontermination
in case of recursive datatypes. Again, if the \isa{only{\isacharcolon}} modifier is
dropped, the above goal is solved,%
\end{isamarkuptxt}%
\isacommand{apply}{\isacharparenleft}simp\ split{\isacharcolon}\ list{\isachardot}split{\isacharparenright}%
\begin{isamarkuptext}%
\noindent%
which \isacommand{apply}\isa{{\isacharparenleft}simp{\isacharparenright}} alone will not do.

In general, every datatype $t$ comes with a theorem
$t$\isa{{\isachardot}split} which can be declared to be a \bfindex{split rule} either
locally as above, or by giving it the \isa{split} attribute globally:%
\end{isamarkuptext}%
\isacommand{declare}\ list{\isachardot}split\ {\isacharbrackleft}split{\isacharbrackright}%
\begin{isamarkuptext}%
\noindent
The \isa{split} attribute can be removed with the \isa{del} modifier,
either locally%
\end{isamarkuptext}%
\isacommand{apply}{\isacharparenleft}simp\ split\ del{\isacharcolon}\ split{\isacharunderscore}if{\isacharparenright}%
\begin{isamarkuptext}%
\noindent
or globally:%
\end{isamarkuptext}%
\isacommand{declare}\ list{\isachardot}split\ {\isacharbrackleft}split\ del{\isacharbrackright}%
\begin{isamarkuptext}%
The above split rules intentionally only affect the conclusion of a
subgoal.  If you want to split an \isa{if} or \isa{case}-expression in
the assumptions, you have to apply \isa{split{\isacharunderscore}if{\isacharunderscore}asm} or
$t$\isa{{\isachardot}split{\isacharunderscore}asm}:%
\end{isamarkuptext}%
\isacommand{lemma}\ {\isachardoublequote}if\ xs\ {\isacharequal}\ {\isacharbrackleft}{\isacharbrackright}\ then\ ys\ {\isachartilde}{\isacharequal}\ {\isacharbrackleft}{\isacharbrackright}\ else\ ys\ {\isacharequal}\ {\isacharbrackleft}{\isacharbrackright}\ {\isacharequal}{\isacharequal}{\isachargreater}\ xs\ {\isacharat}\ ys\ {\isachartilde}{\isacharequal}\ {\isacharbrackleft}{\isacharbrackright}{\isachardoublequote}\isanewline
\isacommand{apply}{\isacharparenleft}simp\ only{\isacharcolon}\ split{\isacharcolon}\ split{\isacharunderscore}if{\isacharunderscore}asm{\isacharparenright}%
\begin{isamarkuptxt}%
\noindent
In contrast to splitting the conclusion, this actually creates two
separate subgoals (which are solved by \isa{simp{\isacharunderscore}all}):
\begin{isabelle}%
\ {\isadigit{1}}{\isachardot}\ {\isasymlbrakk}xs\ {\isacharequal}\ {\isacharbrackleft}{\isacharbrackright}{\isacharsemicolon}\ ys\ {\isasymnoteq}\ {\isacharbrackleft}{\isacharbrackright}{\isasymrbrakk}\ {\isasymLongrightarrow}\ {\isacharbrackleft}{\isacharbrackright}\ {\isacharat}\ ys\ {\isasymnoteq}\ {\isacharbrackleft}{\isacharbrackright}\isanewline
\ {\isadigit{2}}{\isachardot}\ {\isasymlbrakk}xs\ {\isasymnoteq}\ {\isacharbrackleft}{\isacharbrackright}{\isacharsemicolon}\ ys\ {\isacharequal}\ {\isacharbrackleft}{\isacharbrackright}{\isasymrbrakk}\ {\isasymLongrightarrow}\ xs\ {\isacharat}\ {\isacharbrackleft}{\isacharbrackright}\ {\isasymnoteq}\ {\isacharbrackleft}{\isacharbrackright}%
\end{isabelle}
If you need to split both in the assumptions and the conclusion,
use $t$\isa{{\isachardot}splits} which subsumes $t$\isa{{\isachardot}split} and
$t$\isa{{\isachardot}split{\isacharunderscore}asm}. Analogously, there is \isa{if{\isacharunderscore}splits}.

\begin{warn}
  The simplifier merely simplifies the condition of an \isa{if} but not the
  \isa{then} or \isa{else} parts. The latter are simplified only after the
  condition reduces to \isa{True} or \isa{False}, or after splitting. The
  same is true for \isaindex{case}-expressions: only the selector is
  simplified at first, until either the expression reduces to one of the
  cases or it is split.
\end{warn}

\index{*split|)}%
\end{isamarkuptxt}%
%
\isamarkupsubsubsection{Arithmetic%
}
%
\begin{isamarkuptext}%
\index{arithmetic}
The simplifier routinely solves a small class of linear arithmetic formulae
(over type \isa{nat} and other numeric types): it only takes into account
assumptions and conclusions that are (possibly negated) (in)equalities
(\isa{{\isacharequal}}, \isasymle, \isa{{\isacharless}}) and it only knows about addition. Thus%
\end{isamarkuptext}%
\isacommand{lemma}\ {\isachardoublequote}{\isasymlbrakk}\ {\isasymnot}\ m\ {\isacharless}\ n{\isacharsemicolon}\ m\ {\isacharless}\ n{\isacharplus}{\isadigit{1}}\ {\isasymrbrakk}\ {\isasymLongrightarrow}\ m\ {\isacharequal}\ n{\isachardoublequote}%
\begin{isamarkuptext}%
\noindent
is proved by simplification, whereas the only slightly more complex%
\end{isamarkuptext}%
\isacommand{lemma}\ {\isachardoublequote}{\isasymnot}\ m\ {\isacharless}\ n\ {\isasymand}\ m\ {\isacharless}\ n{\isacharplus}{\isadigit{1}}\ {\isasymLongrightarrow}\ m\ {\isacharequal}\ n{\isachardoublequote}%
\begin{isamarkuptext}%
\noindent
is not proved by simplification and requires \isa{arith}.%
\end{isamarkuptext}%
%
\isamarkupsubsubsection{Tracing%
}
%
\begin{isamarkuptext}%
\indexbold{tracing the simplifier}
Using the simplifier effectively may take a bit of experimentation.  Set the
\isaindexbold{trace_simp} \rmindex{flag} to get a better idea of what is going
on:%
\end{isamarkuptext}%
\isacommand{ML}\ {\isachardoublequote}set\ trace{\isacharunderscore}simp{\isachardoublequote}\isanewline
\isacommand{lemma}\ {\isachardoublequote}rev\ {\isacharbrackleft}a{\isacharbrackright}\ {\isacharequal}\ {\isacharbrackleft}{\isacharbrackright}{\isachardoublequote}\isanewline
\isacommand{apply}{\isacharparenleft}simp{\isacharparenright}%
\begin{isamarkuptext}%
\noindent
produces the trace

\begin{ttbox}\makeatother
Applying instance of rewrite rule:
rev (?x1 \# ?xs1) == rev ?xs1 @ [?x1]
Rewriting:
rev [x] == rev [] @ [x]
Applying instance of rewrite rule:
rev [] == []
Rewriting:
rev [] == []
Applying instance of rewrite rule:
[] @ ?y == ?y
Rewriting:
[] @ [x] == [x]
Applying instance of rewrite rule:
?x3 \# ?t3 = ?t3 == False
Rewriting:
[x] = [] == False
\end{ttbox}

In more complicated cases, the trace can be quite lenghty, especially since
invocations of the simplifier are often nested (e.g.\ when solving conditions
of rewrite rules). Thus it is advisable to reset it:%
\end{isamarkuptext}%
\isacommand{ML}\ {\isachardoublequote}reset\ trace{\isacharunderscore}simp{\isachardoublequote}\isanewline
\end{isabellebody}%
%%% Local Variables:
%%% mode: latex
%%% TeX-master: "root"
%%% End:


\section{Advanced forms of recursion}
\index{*recdef|(}

The purpose of this section is to introduce advanced forms of \isacommand{recdef}. It
covers two topics: how to define recursive function over nested recursive datatypes
and how to establish termination by means other than measure functions.

If, after reading this section, you feel that the definition of recursive
functions is overly and maybe unnecessarily complicated by the requirement of
totality, you should ponder the alternative, a logic of partial functions,
where recursive definitions are always wellformed. For a start, there are many
such logics, and no clear winner has emerged. And in all of these logics you
are (more or less frequently) required to reason about the definedness of
terms explicitly. Thus one shifts definedness arguments from definition to
proof time. In HOL you may have to work hard to define a function, but proofs
can then proceed unencumbered by worries about undefinedness.

\subsection{Recursion over nested datatypes}
\label{sec:nested-recdef}
%
\begin{isabellebody}%
\def\isabellecontext{Nested{\isadigit{0}}}%
%
\begin{isamarkuptext}%
\index{datatypes!nested}%
In \S\ref{sec:nested-datatype} we defined the datatype of terms%
\end{isamarkuptext}%
\isacommand{datatype}\ {\isacharparenleft}{\isacharprime}a{\isacharcomma}{\isacharprime}b{\isacharparenright}{\isachardoublequote}term{\isachardoublequote}\ {\isacharequal}\ Var\ {\isacharprime}a\ {\isacharbar}\ App\ {\isacharprime}b\ {\isachardoublequote}{\isacharparenleft}{\isacharprime}a{\isacharcomma}{\isacharprime}b{\isacharparenright}term\ list{\isachardoublequote}%
\begin{isamarkuptext}%
\noindent
and closed with the observation that the associated schema for the definition
of primitive recursive functions leads to overly verbose definitions. Moreover,
if you have worked exercise~\ref{ex:trev-trev} you will have noticed that
you needed to declare essentially the same function as \isa{rev}
and prove many standard properties of list reversal all over again. 
We will now show you how \isacommand{recdef} can simplify
definitions and proofs about nested recursive datatypes. As an example we
choose exercise~\ref{ex:trev-trev}:%
\end{isamarkuptext}%
\isacommand{consts}\ trev\ \ {\isacharcolon}{\isacharcolon}\ {\isachardoublequote}{\isacharparenleft}{\isacharprime}a{\isacharcomma}{\isacharprime}b{\isacharparenright}term\ {\isasymRightarrow}\ {\isacharparenleft}{\isacharprime}a{\isacharcomma}{\isacharprime}b{\isacharparenright}term{\isachardoublequote}\end{isabellebody}%
%%% Local Variables:
%%% mode: latex
%%% TeX-master: "root"
%%% End:

\begin{isabelle}%
\isacommand{consts}\ trev\ \ {\isacharcolon}{\isacharcolon}\ {\isachardoublequote}{\isacharparenleft}{\isacharprime}a{\isacharcomma}{\isacharprime}b{\isacharparenright}term\ {\isacharequal}{\isachargreater}\ {\isacharparenleft}{\isacharprime}a{\isacharcomma}{\isacharprime}b{\isacharparenright}term{\isachardoublequote}%
\begin{isamarkuptext}%
\noindent
Although the definition of \isa{trev} is quite natural, we will have
overcome a minor difficulty in convincing Isabelle of is termination.
It is precisely this difficulty that is the \textit{rasion d'\^etre} of
this subsection.

Defining \isa{trev} by \isacommand{recdef} rather than \isacommand{primrec}
simplifies matters because we are now free to use the recursion equation
suggested at the end of \S\ref{sec:nested-datatype}:%
\end{isamarkuptext}%
\isacommand{recdef}\ trev\ {\isachardoublequote}measure\ size{\isachardoublequote}\isanewline
\ {\isachardoublequote}trev\ {\isacharparenleft}Var\ x{\isacharparenright}\ {\isacharequal}\ Var\ x{\isachardoublequote}\isanewline
\ {\isachardoublequote}trev\ {\isacharparenleft}App\ f\ ts{\isacharparenright}\ {\isacharequal}\ App\ f\ {\isacharparenleft}rev{\isacharparenleft}map\ trev\ ts{\isacharparenright}{\isacharparenright}{\isachardoublequote}%
\begin{isamarkuptext}%
FIXME: recdef should complain and generate unprovable termination condition!
moveto todo

Remember that function \isa{size} is defined for each \isacommand{datatype}.
However, the definition does not succeed. Isabelle complains about an unproved termination
condition
\begin{quote}

\begin{isabelle}%
\mbox{t}\ {\isasymin}\ set\ \mbox{ts}\ {\isasymlongrightarrow}\ size\ \mbox{t}\ {\isacharless}\ Suc\ {\isacharparenleft}term{\isacharunderscore}size\ \mbox{ts}{\isacharparenright}
\end{isabelle}%

\end{quote}
where \isa{set} returns the set of elements of a list---no special knowledge of sets is
required in the following.
First we have to understand why the recursive call of \isa{trev} underneath \isa{map} leads
to the above condition. The reason is that \isacommand{recdef} ``knows'' that \isa{map} will
apply \isa{trev} only to elements of \isa{\mbox{ts}}. Thus the above condition expresses that
the size of the argument \isa{\mbox{t}\ {\isasymin}\ set\ \mbox{ts}} of any recursive call of \isa{trev} is strictly
less than \isa{size\ {\isacharparenleft}App\ \mbox{f}\ \mbox{ts}{\isacharparenright}\ {\isacharequal}\ Suc\ {\isacharparenleft}term{\isacharunderscore}size\ \mbox{ts}{\isacharparenright}}.
We will now prove the termination condition and continue with our definition.
Below we return to the question of how \isacommand{recdef} ``knows'' about \isa{map}.%
\end{isamarkuptext}%
\end{isabelle}%
%%% Local Variables:
%%% mode: latex
%%% TeX-master: "root"
%%% End:

%
\begin{isabellebody}%
\def\isabellecontext{Nested{\isadigit{2}}}%
%
\begin{isamarkuptext}%
\noindent
The termintion condition is easily proved by induction:%
\end{isamarkuptext}%
\isacommand{lemma}\ {\isacharbrackleft}simp{\isacharbrackright}{\isacharcolon}\ {\isachardoublequote}t\ {\isasymin}\ set\ ts\ {\isasymlongrightarrow}\ size\ t\ {\isacharless}\ Suc{\isacharparenleft}term{\isacharunderscore}list{\isacharunderscore}size\ ts{\isacharparenright}{\isachardoublequote}\isanewline
\isacommand{by}{\isacharparenleft}induct{\isacharunderscore}tac\ ts{\isacharcomma}\ auto{\isacharparenright}%
\begin{isamarkuptext}%
\noindent
By making this theorem a simplification rule, \isacommand{recdef}
applies it automatically and the definition of \isa{trev}
succeeds now. As a reward for our effort, we can now prove the desired
lemma directly.  We no longer need the verbose
induction schema for type \isa{term} and can use the simpler one arising from
\isa{trev}:%
\end{isamarkuptext}%
\isacommand{lemma}\ {\isachardoublequote}trev{\isacharparenleft}trev\ t{\isacharparenright}\ {\isacharequal}\ t{\isachardoublequote}\isanewline
\isacommand{apply}{\isacharparenleft}induct{\isacharunderscore}tac\ t\ rule{\isacharcolon}trev{\isachardot}induct{\isacharparenright}%
\begin{isamarkuptxt}%
\noindent
This leaves us with a trivial base case \isa{trev\ {\isacharparenleft}trev\ {\isacharparenleft}Var\ x{\isacharparenright}{\isacharparenright}\ {\isacharequal}\ Var\ x} and the step case
\begin{isabelle}%
\ \ \ \ \ {\isasymforall}t{\isachardot}\ t\ {\isasymin}\ set\ ts\ {\isasymlongrightarrow}\ trev\ {\isacharparenleft}trev\ t{\isacharparenright}\ {\isacharequal}\ t\ {\isasymLongrightarrow}\isanewline
\isaindent{\ \ \ \ \ }trev\ {\isacharparenleft}trev\ {\isacharparenleft}App\ f\ ts{\isacharparenright}{\isacharparenright}\ {\isacharequal}\ App\ f\ ts%
\end{isabelle}
both of which are solved by simplification:%
\end{isamarkuptxt}%
\isacommand{by}{\isacharparenleft}simp{\isacharunderscore}all\ add{\isacharcolon}rev{\isacharunderscore}map\ sym{\isacharbrackleft}OF\ map{\isacharunderscore}compose{\isacharbrackright}\ cong{\isacharcolon}map{\isacharunderscore}cong{\isacharparenright}%
\begin{isamarkuptext}%
\noindent
If the proof of the induction step mystifies you, we recommend that you go through
the chain of simplification steps in detail; you will probably need the help of
\isa{trace{\isacharunderscore}simp}. Theorem \isa{map{\isacharunderscore}cong} is discussed below.
%\begin{quote}
%{term[display]"trev(trev(App f ts))"}\\
%{term[display]"App f (rev(map trev (rev(map trev ts))))"}\\
%{term[display]"App f (map trev (rev(rev(map trev ts))))"}\\
%{term[display]"App f (map trev (map trev ts))"}\\
%{term[display]"App f (map (trev o trev) ts)"}\\
%{term[display]"App f (map (%x. x) ts)"}\\
%{term[display]"App f ts"}
%\end{quote}

The definition of \isa{trev} above is superior to the one in
\S\ref{sec:nested-datatype} because it uses \isa{rev}
and lets us use existing facts such as \hbox{\isa{rev\ {\isacharparenleft}rev\ xs{\isacharparenright}\ {\isacharequal}\ xs}}.
Thus this proof is a good example of an important principle:
\begin{quote}
\emph{Chose your definitions carefully\\
because they determine the complexity of your proofs.}
\end{quote}

Let us now return to the question of how \isacommand{recdef} can come up with
sensible termination conditions in the presence of higher-order functions
like \isa{map}. For a start, if nothing were known about \isa{map},
\isa{map\ trev\ ts} might apply \isa{trev} to arbitrary terms, and thus
\isacommand{recdef} would try to prove the unprovable \isa{size\ t\ {\isacharless}\ Suc\ {\isacharparenleft}term{\isacharunderscore}list{\isacharunderscore}size\ ts{\isacharparenright}}, without any assumption about \isa{t}.  Therefore
\isacommand{recdef} has been supplied with the congruence theorem
\isa{map{\isacharunderscore}cong}:
\begin{isabelle}%
\ \ \ \ \ {\isasymlbrakk}xs\ {\isacharequal}\ ys{\isacharsemicolon}\ {\isasymAnd}x{\isachardot}\ x\ {\isasymin}\ set\ ys\ {\isasymLongrightarrow}\ f\ x\ {\isacharequal}\ g\ x{\isasymrbrakk}\isanewline
\isaindent{\ \ \ \ \ }{\isasymLongrightarrow}\ map\ f\ xs\ {\isacharequal}\ map\ g\ ys%
\end{isabelle}
Its second premise expresses (indirectly) that the second argument of
\isa{map} is only applied to elements of its third argument. Congruence
rules for other higher-order functions on lists look very similar. If you get
into a situation where you need to supply \isacommand{recdef} with new
congruence rules, you can either append a hint locally
to the specific occurrence of \isacommand{recdef}%
\end{isamarkuptext}%
{\isacharparenleft}\isakeyword{hints}\ recdef{\isacharunderscore}cong{\isacharcolon}\ map{\isacharunderscore}cong{\isacharparenright}%
\begin{isamarkuptext}%
\noindent
or declare them globally
by giving them the \isaindexbold{recdef_cong} attribute as in%
\end{isamarkuptext}%
\isacommand{declare}\ map{\isacharunderscore}cong{\isacharbrackleft}recdef{\isacharunderscore}cong{\isacharbrackright}%
\begin{isamarkuptext}%
Note that the \isa{cong} and \isa{recdef{\isacharunderscore}cong} attributes are
intentionally kept apart because they control different activities, namely
simplification and making recursive definitions.
% The local \isa{cong} in
% the hints section of \isacommand{recdef} is merely short for \isa{recdef{\isacharunderscore}cong}.
%The simplifier's congruence rules cannot be used by recdef.
%For example the weak congruence rules for if and case would prevent
%recdef from generating sensible termination conditions.%
\end{isamarkuptext}%
\end{isabellebody}%
%%% Local Variables:
%%% mode: latex
%%% TeX-master: "root"
%%% End:

\index{*recdef|)}

\subsection{Beyond measure}
\label{sec:beyond-measure}
%
\begin{isabellebody}%
\def\isabellecontext{WFrec}%
%
\begin{isamarkuptext}%
\noindent
So far, all recursive definitions where shown to terminate via measure
functions. Sometimes this can be quite inconvenient or even
impossible. Fortunately, \isacommand{recdef} supports much more
general definitions. For example, termination of Ackermann's function
can be shown by means of the lexicographic product \isa{{\isacharless}{\isacharasterisk}lex{\isacharasterisk}{\isachargreater}}:%
\end{isamarkuptext}%
\isacommand{consts}\ ack\ {\isacharcolon}{\isacharcolon}\ {\isachardoublequote}nat{\isasymtimes}nat\ {\isasymRightarrow}\ nat{\isachardoublequote}\isanewline
\isacommand{recdef}\ ack\ {\isachardoublequote}measure{\isacharparenleft}{\isasymlambda}m{\isachardot}\ m{\isacharparenright}\ {\isacharless}{\isacharasterisk}lex{\isacharasterisk}{\isachargreater}\ measure{\isacharparenleft}{\isasymlambda}n{\isachardot}\ n{\isacharparenright}{\isachardoublequote}\isanewline
\ \ {\isachardoublequote}ack{\isacharparenleft}{\isadigit{0}}{\isacharcomma}n{\isacharparenright}\ \ \ \ \ \ \ \ \ {\isacharequal}\ Suc\ n{\isachardoublequote}\isanewline
\ \ {\isachardoublequote}ack{\isacharparenleft}Suc\ m{\isacharcomma}{\isadigit{0}}{\isacharparenright}\ \ \ \ \ {\isacharequal}\ ack{\isacharparenleft}m{\isacharcomma}\ {\isadigit{1}}{\isacharparenright}{\isachardoublequote}\isanewline
\ \ {\isachardoublequote}ack{\isacharparenleft}Suc\ m{\isacharcomma}Suc\ n{\isacharparenright}\ {\isacharequal}\ ack{\isacharparenleft}m{\isacharcomma}ack{\isacharparenleft}Suc\ m{\isacharcomma}n{\isacharparenright}{\isacharparenright}{\isachardoublequote}%
\begin{isamarkuptext}%
\noindent
The lexicographic product decreases if either its first component
decreases (as in the second equation and in the outer call in the
third equation) or its first component stays the same and the second
component decreases (as in the inner call in the third equation).

In general, \isacommand{recdef} supports termination proofs based on
arbitrary \emph{wellfounded relations}, i.e.\ \emph{wellfounded
recursion}\indexbold{recursion!wellfounded}\index{wellfounded
recursion|see{recursion, wellfounded}}.  A relation $<$ is
\bfindex{wellfounded} if it has no infinite descending chain $\cdots <
a@2 < a@1 < a@0$. Clearly, a function definition is total iff the set
of all pairs $(r,l)$, where $l$ is the argument on the left-hand side
of an equation and $r$ the argument of some recursive call on the
corresponding right-hand side, induces a wellfounded relation.  For a
systematic account of termination proofs via wellfounded relations
see, for example, \cite{Baader-Nipkow}. The HOL library formalizes
some of the theory of wellfounded relations. For example
\isa{wf\ r}\index{*wf|bold} means that relation \isa{r{\isasymColon}{\isacharparenleft}{\isacharprime}a\ {\isasymtimes}\ {\isacharprime}a{\isacharparenright}\ set} is
wellfounded.

Each \isacommand{recdef} definition should be accompanied (after the
name of the function) by a wellfounded relation on the argument type
of the function. For example, \isa{measure} is defined by
\begin{isabelle}%
\ \ \ \ \ measure\ f\ {\isasymequiv}\ {\isacharbraceleft}{\isacharparenleft}y{\isacharcomma}\ x{\isacharparenright}{\isachardot}\ f\ y\ {\isacharless}\ f\ x{\isacharbraceright}%
\end{isabelle}
and it has been proved that \isa{measure\ f} is always wellfounded.

In addition to \isa{measure}, the library provides
a number of further constructions for obtaining wellfounded relations.
Above we have already met \isa{{\isacharless}{\isacharasterisk}lex{\isacharasterisk}{\isachargreater}} of type
\begin{isabelle}%
\ \ \ \ \ {\isachardoublequote}{\isacharparenleft}{\isacharprime}a\ {\isasymtimes}\ {\isacharprime}a{\isacharparenright}set\ {\isasymRightarrow}\ {\isacharparenleft}{\isacharprime}b\ {\isasymtimes}\ {\isacharprime}b{\isacharparenright}set\ {\isasymRightarrow}\ {\isacharparenleft}{\isacharparenleft}{\isacharprime}a\ {\isasymtimes}\ {\isacharprime}b{\isacharparenright}\ {\isasymtimes}\ {\isacharparenleft}{\isacharprime}a\ {\isasymtimes}\ {\isacharprime}b{\isacharparenright}{\isacharparenright}set{\isachardoublequote}%
\end{isabelle}
Of course the lexicographic product can also be interated, as in the following
function definition:%
\end{isamarkuptext}%
\isacommand{consts}\ contrived\ {\isacharcolon}{\isacharcolon}\ {\isachardoublequote}nat\ {\isasymtimes}\ nat\ {\isasymtimes}\ nat\ {\isasymRightarrow}\ nat{\isachardoublequote}\isanewline
\isacommand{recdef}\ contrived\isanewline
\ \ {\isachardoublequote}measure{\isacharparenleft}{\isasymlambda}i{\isachardot}\ i{\isacharparenright}\ {\isacharless}{\isacharasterisk}lex{\isacharasterisk}{\isachargreater}\ measure{\isacharparenleft}{\isasymlambda}j{\isachardot}\ j{\isacharparenright}\ {\isacharless}{\isacharasterisk}lex{\isacharasterisk}{\isachargreater}\ measure{\isacharparenleft}{\isasymlambda}k{\isachardot}\ k{\isacharparenright}{\isachardoublequote}\isanewline
{\isachardoublequote}contrived{\isacharparenleft}i{\isacharcomma}j{\isacharcomma}Suc\ k{\isacharparenright}\ {\isacharequal}\ contrived{\isacharparenleft}i{\isacharcomma}j{\isacharcomma}k{\isacharparenright}{\isachardoublequote}\isanewline
{\isachardoublequote}contrived{\isacharparenleft}i{\isacharcomma}Suc\ j{\isacharcomma}{\isadigit{0}}{\isacharparenright}\ {\isacharequal}\ contrived{\isacharparenleft}i{\isacharcomma}j{\isacharcomma}j{\isacharparenright}{\isachardoublequote}\isanewline
{\isachardoublequote}contrived{\isacharparenleft}Suc\ i{\isacharcomma}{\isadigit{0}}{\isacharcomma}{\isadigit{0}}{\isacharparenright}\ {\isacharequal}\ contrived{\isacharparenleft}i{\isacharcomma}i{\isacharcomma}i{\isacharparenright}{\isachardoublequote}\isanewline
{\isachardoublequote}contrived{\isacharparenleft}{\isadigit{0}}{\isacharcomma}{\isadigit{0}}{\isacharcomma}{\isadigit{0}}{\isacharparenright}\ \ \ \ \ {\isacharequal}\ {\isadigit{0}}{\isachardoublequote}%
\begin{isamarkuptext}%
Lexicographic products of measure functions already go a long way. A
further useful construction is the embedding of some type in an
existing wellfounded relation via the inverse image of a function:
\begin{isabelle}%
\ \ \ \ \ inv{\isacharunderscore}image\ {\isacharparenleft}r{\isasymColon}{\isacharparenleft}{\isacharprime}b\ {\isasymtimes}\ {\isacharprime}b{\isacharparenright}\ set{\isacharparenright}\ {\isacharparenleft}f{\isasymColon}{\isacharprime}a\ {\isasymRightarrow}\ {\isacharprime}b{\isacharparenright}\ {\isasymequiv}\isanewline
\ \ \ \ \ {\isacharbraceleft}{\isacharparenleft}x{\isasymColon}{\isacharprime}a{\isacharcomma}\ y{\isasymColon}{\isacharprime}a{\isacharparenright}{\isachardot}\ {\isacharparenleft}f\ x{\isacharcomma}\ f\ y{\isacharparenright}\ {\isasymin}\ r{\isacharbraceright}%
\end{isabelle}
\begin{sloppypar}
\noindent
For example, \isa{measure} is actually defined as \isa{inv{\isacharunderscore}mage\ less{\isacharunderscore}than}, where
\isa{less{\isacharunderscore}than} of type \isa{{\isacharparenleft}nat\ {\isasymtimes}\ nat{\isacharparenright}\ set} is the less-than relation on type \isa{nat}
(as opposed to \isa{op\ {\isacharless}}, which is of type \isa{{\isacharbrackleft}nat{\isacharcomma}\ nat{\isacharbrackright}\ {\isasymRightarrow}\ bool}).
\end{sloppypar}

%Finally there is also {finite_psubset} the proper subset relation on finite sets

All the above constructions are known to \isacommand{recdef}. Thus you
will never have to prove wellfoundedness of any relation composed
solely of these building blocks. But of course the proof of
termination of your function definition, i.e.\ that the arguments
decrease with every recursive call, may still require you to provide
additional lemmas.

It is also possible to use your own wellfounded relations with \isacommand{recdef}.
Here is a simplistic example:%
\end{isamarkuptext}%
\isacommand{consts}\ f\ {\isacharcolon}{\isacharcolon}\ {\isachardoublequote}nat\ {\isasymRightarrow}\ nat{\isachardoublequote}\isanewline
\isacommand{recdef}\ f\ {\isachardoublequote}id{\isacharparenleft}less{\isacharunderscore}than{\isacharparenright}{\isachardoublequote}\isanewline
{\isachardoublequote}f\ {\isadigit{0}}\ {\isacharequal}\ {\isadigit{0}}{\isachardoublequote}\isanewline
{\isachardoublequote}f\ {\isacharparenleft}Suc\ n{\isacharparenright}\ {\isacharequal}\ f\ n{\isachardoublequote}%
\begin{isamarkuptext}%
Since \isacommand{recdef} is not prepared for \isa{id}, the identity
function, this leads to the complaint that it could not prove
\isa{wf\ {\isacharparenleft}id\ less{\isacharunderscore}than{\isacharparenright}}, the wellfoundedness of \isa{id\ less{\isacharunderscore}than}. We should first have proved that \isa{id} preserves wellfoundedness%
\end{isamarkuptext}%
\isacommand{lemma}\ wf{\isacharunderscore}id{\isacharcolon}\ {\isachardoublequote}wf\ r\ {\isasymLongrightarrow}\ wf{\isacharparenleft}id\ r{\isacharparenright}{\isachardoublequote}\isanewline
\isacommand{by}\ simp%
\begin{isamarkuptext}%
\noindent
and should have added the following hint to our above definition:%
\end{isamarkuptext}%
{\isacharparenleft}\isakeyword{hints}\ recdef{\isacharunderscore}wf\ add{\isacharcolon}\ wf{\isacharunderscore}id{\isacharparenright}\end{isabellebody}%
%%% Local Variables:
%%% mode: latex
%%% TeX-master: "root"
%%% End:


\section{Advanced induction techniques}
\label{sec:advanced-ind}
\index{induction|(}
%
\begin{isabellebody}%
\def\isabellecontext{AdvancedInd}%
%
\begin{isamarkuptext}%
\noindent
Now that we have learned about rules and logic, we take another look at the
finer points of induction. The two questions we answer are: what to do if the
proposition to be proved is not directly amenable to induction
(\S\ref{sec:ind-var-in-prems}), and how to utilize (\S\ref{sec:complete-ind})
and even derive (\S\ref{sec:derive-ind}) new induction schemas. We conclude
with an extended example of induction (\S\ref{sec:CTL-revisited}).%
\end{isamarkuptext}%
%
\isamarkupsubsection{Massaging the Proposition%
}
%
\begin{isamarkuptext}%
\label{sec:ind-var-in-prems}
Often we have assumed that the theorem we want to prove is already in a form
that is amenable to induction, but sometimes it isn't.
Here is an example.
Since \isa{hd} and \isa{last} return the first and last element of a
non-empty list, this lemma looks easy to prove:%
\end{isamarkuptext}%
\isacommand{lemma}\ {\isachardoublequote}xs\ {\isasymnoteq}\ {\isacharbrackleft}{\isacharbrackright}\ {\isasymLongrightarrow}\ hd{\isacharparenleft}rev\ xs{\isacharparenright}\ {\isacharequal}\ last\ xs{\isachardoublequote}\isanewline
\isacommand{apply}{\isacharparenleft}induct{\isacharunderscore}tac\ xs{\isacharparenright}%
\begin{isamarkuptxt}%
\noindent
But induction produces the warning
\begin{quote}\tt
Induction variable occurs also among premises!
\end{quote}
and leads to the base case
\begin{isabelle}%
\ {\isadigit{1}}{\isachardot}\ xs\ {\isasymnoteq}\ {\isacharbrackleft}{\isacharbrackright}\ {\isasymLongrightarrow}\ hd\ {\isacharparenleft}rev\ {\isacharbrackleft}{\isacharbrackright}{\isacharparenright}\ {\isacharequal}\ last\ {\isacharbrackleft}{\isacharbrackright}%
\end{isabelle}
After simplification, it becomes
\begin{isabelle}
\ 1.\ xs\ {\isasymnoteq}\ []\ {\isasymLongrightarrow}\ hd\ []\ =\ last\ []
\end{isabelle}
We cannot prove this equality because we do not know what \isa{hd} and
\isa{last} return when applied to \isa{{\isacharbrackleft}{\isacharbrackright}}.

We should not have ignored the warning. Because the induction
formula is only the conclusion, induction does not affect the occurrence of \isa{xs} in the premises.  
Thus the case that should have been trivial
becomes unprovable. Fortunately, the solution is easy:\footnote{A very similar
heuristic applies to rule inductions; see \S\ref{sec:rtc}.}
\begin{quote}
\emph{Pull all occurrences of the induction variable into the conclusion
using \isa{{\isasymlongrightarrow}}.}
\end{quote}
Thus we should state the lemma as an ordinary 
implication~(\isa{{\isasymlongrightarrow}}), letting
\isa{rule{\isacharunderscore}format} (\S\ref{sec:forward}) convert the
result to the usual \isa{{\isasymLongrightarrow}} form:%
\end{isamarkuptxt}%
\isacommand{lemma}\ hd{\isacharunderscore}rev\ {\isacharbrackleft}rule{\isacharunderscore}format{\isacharbrackright}{\isacharcolon}\ {\isachardoublequote}xs\ {\isasymnoteq}\ {\isacharbrackleft}{\isacharbrackright}\ {\isasymlongrightarrow}\ hd{\isacharparenleft}rev\ xs{\isacharparenright}\ {\isacharequal}\ last\ xs{\isachardoublequote}%
\begin{isamarkuptxt}%
\noindent
This time, induction leaves us with a trivial base case:
\begin{isabelle}%
\ {\isadigit{1}}{\isachardot}\ {\isacharbrackleft}{\isacharbrackright}\ {\isasymnoteq}\ {\isacharbrackleft}{\isacharbrackright}\ {\isasymlongrightarrow}\ hd\ {\isacharparenleft}rev\ {\isacharbrackleft}{\isacharbrackright}{\isacharparenright}\ {\isacharequal}\ last\ {\isacharbrackleft}{\isacharbrackright}%
\end{isabelle}
And \isa{auto} completes the proof.

If there are multiple premises $A@1$, \dots, $A@n$ containing the
induction variable, you should turn the conclusion $C$ into
\[ A@1 \longrightarrow \cdots A@n \longrightarrow C \]
Additionally, you may also have to universally quantify some other variables,
which can yield a fairly complex conclusion.  However, \isa{rule{\isacharunderscore}format} 
can remove any number of occurrences of \isa{{\isasymforall}} and
\isa{{\isasymlongrightarrow}}.%
\end{isamarkuptxt}%
%
\begin{isamarkuptext}%
A second reason why your proposition may not be amenable to induction is that
you want to induct on a whole term, rather than an individual variable. In
general, when inducting on some term $t$ you must rephrase the conclusion $C$
as
\[ \forall y@1 \dots y@n.~ x = t \longrightarrow C \]
where $y@1 \dots y@n$ are the free variables in $t$ and $x$ is new, and
perform induction on $x$ afterwards. An example appears in
\S\ref{sec:complete-ind} below.

The very same problem may occur in connection with rule induction. Remember
that it requires a premise of the form $(x@1,\dots,x@k) \in R$, where $R$ is
some inductively defined set and the $x@i$ are variables.  If instead we have
a premise $t \in R$, where $t$ is not just an $n$-tuple of variables, we
replace it with $(x@1,\dots,x@k) \in R$, and rephrase the conclusion $C$ as
\[ \forall y@1 \dots y@n.~ (x@1,\dots,x@k) = t \longrightarrow C \]
For an example see \S\ref{sec:CTL-revisited} below.

Of course, all premises that share free variables with $t$ need to be pulled into
the conclusion as well, under the \isa{{\isasymforall}}, again using \isa{{\isasymlongrightarrow}} as shown above.%
\end{isamarkuptext}%
%
\isamarkupsubsection{Beyond Structural and Recursion Induction%
}
%
\begin{isamarkuptext}%
\label{sec:complete-ind}
So far, inductive proofs were by structural induction for
primitive recursive functions and recursion induction for total recursive
functions. But sometimes structural induction is awkward and there is no
recursive function that could furnish a more appropriate
induction schema. In such cases a general-purpose induction schema can
be helpful. We show how to apply such induction schemas by an example.

Structural induction on \isa{nat} is
usually known as mathematical induction. There is also \emph{complete}
induction, where you must prove $P(n)$ under the assumption that $P(m)$
holds for all $m<n$. In Isabelle, this is the theorem \isa{nat{\isacharunderscore}less{\isacharunderscore}induct}:
\begin{isabelle}%
\ \ \ \ \ {\isacharparenleft}{\isasymAnd}n{\isachardot}\ {\isasymforall}m{\isachardot}\ m\ {\isacharless}\ n\ {\isasymlongrightarrow}\ P\ m\ {\isasymLongrightarrow}\ P\ n{\isacharparenright}\ {\isasymLongrightarrow}\ P\ n%
\end{isabelle}
Here is an example of its application.%
\end{isamarkuptext}%
\isacommand{consts}\ f\ {\isacharcolon}{\isacharcolon}\ {\isachardoublequote}nat\ {\isasymRightarrow}\ nat{\isachardoublequote}\isanewline
\isacommand{axioms}\ f{\isacharunderscore}ax{\isacharcolon}\ {\isachardoublequote}f{\isacharparenleft}f{\isacharparenleft}n{\isacharparenright}{\isacharparenright}\ {\isacharless}\ f{\isacharparenleft}Suc{\isacharparenleft}n{\isacharparenright}{\isacharparenright}{\isachardoublequote}%
\begin{isamarkuptext}%
\noindent
The axiom for \isa{f} implies \isa{n\ {\isasymle}\ f\ n}, which can
be proved by induction on \mbox{\isa{f\ n}}. Following the recipe outlined
above, we have to phrase the proposition as follows to allow induction:%
\end{isamarkuptext}%
\isacommand{lemma}\ f{\isacharunderscore}incr{\isacharunderscore}lem{\isacharcolon}\ {\isachardoublequote}{\isasymforall}i{\isachardot}\ k\ {\isacharequal}\ f\ i\ {\isasymlongrightarrow}\ i\ {\isasymle}\ f\ i{\isachardoublequote}%
\begin{isamarkuptxt}%
\noindent
To perform induction on \isa{k} using \isa{nat{\isacharunderscore}less{\isacharunderscore}induct}, we use
the same general induction method as for recursion induction (see
\S\ref{sec:recdef-induction}):%
\end{isamarkuptxt}%
\isacommand{apply}{\isacharparenleft}induct{\isacharunderscore}tac\ k\ rule{\isacharcolon}\ nat{\isacharunderscore}less{\isacharunderscore}induct{\isacharparenright}%
\begin{isamarkuptxt}%
\noindent
which leaves us with the following proof state:
\begin{isabelle}%
\ {\isadigit{1}}{\isachardot}\ {\isasymAnd}n{\isachardot}\ {\isasymforall}m{\isachardot}\ m\ {\isacharless}\ n\ {\isasymlongrightarrow}\ {\isacharparenleft}{\isasymforall}i{\isachardot}\ m\ {\isacharequal}\ f\ i\ {\isasymlongrightarrow}\ i\ {\isasymle}\ f\ i{\isacharparenright}\ {\isasymLongrightarrow}\isanewline
\isaindent{\ {\isadigit{1}}{\isachardot}\ {\isasymAnd}n{\isachardot}\ }{\isasymforall}i{\isachardot}\ n\ {\isacharequal}\ f\ i\ {\isasymlongrightarrow}\ i\ {\isasymle}\ f\ i%
\end{isabelle}
After stripping the \isa{{\isasymforall}i}, the proof continues with a case
distinction on \isa{i}. The case \isa{i\ {\isacharequal}\ {\isadigit{0}}} is trivial and we focus on
the other case:%
\end{isamarkuptxt}%
\isacommand{apply}{\isacharparenleft}rule\ allI{\isacharparenright}\isanewline
\isacommand{apply}{\isacharparenleft}case{\isacharunderscore}tac\ i{\isacharparenright}\isanewline
\ \isacommand{apply}{\isacharparenleft}simp{\isacharparenright}%
\begin{isamarkuptxt}%
\begin{isabelle}%
\ {\isadigit{1}}{\isachardot}\ {\isasymAnd}n\ i\ nat{\isachardot}\isanewline
\isaindent{\ {\isadigit{1}}{\isachardot}\ \ \ \ }{\isasymlbrakk}{\isasymforall}m{\isachardot}\ m\ {\isacharless}\ n\ {\isasymlongrightarrow}\ {\isacharparenleft}{\isasymforall}i{\isachardot}\ m\ {\isacharequal}\ f\ i\ {\isasymlongrightarrow}\ i\ {\isasymle}\ f\ i{\isacharparenright}{\isacharsemicolon}\ i\ {\isacharequal}\ Suc\ nat{\isasymrbrakk}\isanewline
\isaindent{\ {\isadigit{1}}{\isachardot}\ \ \ \ }{\isasymLongrightarrow}\ n\ {\isacharequal}\ f\ i\ {\isasymlongrightarrow}\ i\ {\isasymle}\ f\ i%
\end{isabelle}%
\end{isamarkuptxt}%
\isacommand{by}{\isacharparenleft}blast\ intro{\isacharbang}{\isacharcolon}\ f{\isacharunderscore}ax\ Suc{\isacharunderscore}leI\ intro{\isacharcolon}\ le{\isacharunderscore}less{\isacharunderscore}trans{\isacharparenright}%
\begin{isamarkuptext}%
\noindent
If you find the last step puzzling, here are the two lemmas it employs:
\begin{isabelle}
\isa{m\ {\isacharless}\ n\ {\isasymLongrightarrow}\ Suc\ m\ {\isasymle}\ n}
\rulename{Suc_leI}\isanewline
\isa{{\isasymlbrakk}i\ {\isasymle}\ j{\isacharsemicolon}\ j\ {\isacharless}\ k{\isasymrbrakk}\ {\isasymLongrightarrow}\ i\ {\isacharless}\ k}
\rulename{le_less_trans}
\end{isabelle}
%
The proof goes like this (writing \isa{j} instead of \isa{nat}).
Since \isa{i\ {\isacharequal}\ Suc\ j} it suffices to show
\hbox{\isa{j\ {\isacharless}\ f\ {\isacharparenleft}Suc\ j{\isacharparenright}}},
by \isa{Suc{\isacharunderscore}leI}\@.  This is
proved as follows. From \isa{f{\isacharunderscore}ax} we have \isa{f\ {\isacharparenleft}f\ j{\isacharparenright}\ {\isacharless}\ f\ {\isacharparenleft}Suc\ j{\isacharparenright}}
(1) which implies \isa{f\ j\ {\isasymle}\ f\ {\isacharparenleft}f\ j{\isacharparenright}} by the induction hypothesis.
Using (1) once more we obtain \isa{f\ j\ {\isacharless}\ f\ {\isacharparenleft}Suc\ j{\isacharparenright}} (2) by the transitivity
rule \isa{le{\isacharunderscore}less{\isacharunderscore}trans}.
Using the induction hypothesis once more we obtain \isa{j\ {\isasymle}\ f\ j}
which, together with (2) yields \isa{j\ {\isacharless}\ f\ {\isacharparenleft}Suc\ j{\isacharparenright}} (again by
\isa{le{\isacharunderscore}less{\isacharunderscore}trans}).

This last step shows both the power and the danger of automatic proofs: they
will usually not tell you how the proof goes, because it can be very hard to
translate the internal proof into a human-readable format. Therefore
Chapter~\ref{sec:part2?} introduces a language for writing readable
proofs.

We can now derive the desired \isa{i\ {\isasymle}\ f\ i} from \isa{f{\isacharunderscore}incr{\isacharunderscore}lem}:%
\end{isamarkuptext}%
\isacommand{lemmas}\ f{\isacharunderscore}incr\ {\isacharequal}\ f{\isacharunderscore}incr{\isacharunderscore}lem{\isacharbrackleft}rule{\isacharunderscore}format{\isacharcomma}\ OF\ refl{\isacharbrackright}%
\begin{isamarkuptext}%
\noindent
The final \isa{refl} gets rid of the premise \isa{{\isacharquery}k\ {\isacharequal}\ f\ {\isacharquery}i}. 
We could have included this derivation in the original statement of the lemma:%
\end{isamarkuptext}%
\isacommand{lemma}\ f{\isacharunderscore}incr{\isacharbrackleft}rule{\isacharunderscore}format{\isacharcomma}\ OF\ refl{\isacharbrackright}{\isacharcolon}\ {\isachardoublequote}{\isasymforall}i{\isachardot}\ k\ {\isacharequal}\ f\ i\ {\isasymlongrightarrow}\ i\ {\isasymle}\ f\ i{\isachardoublequote}%
\begin{isamarkuptext}%
\begin{warn}
We discourage the use of axioms because of the danger of
inconsistencies.  Axiom \isa{f{\isacharunderscore}ax} does
not introduce an inconsistency because, for example, the identity function
satisfies it.  Axioms can be useful in exploratory developments, say when 
you assume some well-known theorems so that you can quickly demonstrate some
point about methodology.  If your example turns into a substantial proof
development, you should replace the axioms by proofs.
\end{warn}

\bigskip
In general, \isa{induct{\isacharunderscore}tac} can be applied with any rule $r$
whose conclusion is of the form ${?}P~?x@1 \dots ?x@n$, in which case the
format is
\begin{quote}
\isacommand{apply}\isa{{\isacharparenleft}induct{\isacharunderscore}tac} $y@1 \dots y@n$ \isa{rule{\isacharcolon}} $r$\isa{{\isacharparenright}}
\end{quote}\index{*induct_tac}%
where $y@1, \dots, y@n$ are variables in the first subgoal.
A further example of a useful induction rule is \isa{length{\isacharunderscore}induct},
induction on the length of a list:\indexbold{*length_induct}
\begin{isabelle}%
\ \ \ \ \ {\isacharparenleft}{\isasymAnd}xs{\isachardot}\ {\isasymforall}ys{\isachardot}\ length\ ys\ {\isacharless}\ length\ xs\ {\isasymlongrightarrow}\ P\ ys\ {\isasymLongrightarrow}\ P\ xs{\isacharparenright}\ {\isasymLongrightarrow}\ P\ xs%
\end{isabelle}

In fact, \isa{induct{\isacharunderscore}tac} even allows the conclusion of
$r$ to be an (iterated) conjunction of formulae of the above form, in
which case the application is
\begin{quote}
\isacommand{apply}\isa{{\isacharparenleft}induct{\isacharunderscore}tac} $y@1 \dots y@n$ \isa{and} \dots\ \isa{and} $z@1 \dots z@m$ \isa{rule{\isacharcolon}} $r$\isa{{\isacharparenright}}
\end{quote}

\begin{exercise}
From the axiom and lemma for \isa{f}, show that \isa{f} is the
identity function.
\end{exercise}%
\end{isamarkuptext}%
%
\isamarkupsubsection{Derivation of New Induction Schemas%
}
%
\begin{isamarkuptext}%
\label{sec:derive-ind}
Induction schemas are ordinary theorems and you can derive new ones
whenever you wish.  This section shows you how to, using the example
of \isa{nat{\isacharunderscore}less{\isacharunderscore}induct}. Assume we only have structural induction
available for \isa{nat} and want to derive complete induction. This
requires us to generalize the statement first:%
\end{isamarkuptext}%
\isacommand{lemma}\ induct{\isacharunderscore}lem{\isacharcolon}\ {\isachardoublequote}{\isacharparenleft}{\isasymAnd}n{\isacharcolon}{\isacharcolon}nat{\isachardot}\ {\isasymforall}m{\isacharless}n{\isachardot}\ P\ m\ {\isasymLongrightarrow}\ P\ n{\isacharparenright}\ {\isasymLongrightarrow}\ {\isasymforall}m{\isacharless}n{\isachardot}\ P\ m{\isachardoublequote}\isanewline
\isacommand{apply}{\isacharparenleft}induct{\isacharunderscore}tac\ n{\isacharparenright}%
\begin{isamarkuptxt}%
\noindent
The base case is vacuously true. For the induction step (\isa{m\ {\isacharless}\ Suc\ n}) we distinguish two cases: case \isa{m\ {\isacharless}\ n} is true by induction
hypothesis and case \isa{m\ {\isacharequal}\ n} follows from the assumption, again using
the induction hypothesis:%
\end{isamarkuptxt}%
\ \isacommand{apply}{\isacharparenleft}blast{\isacharparenright}\isanewline
\isacommand{by}{\isacharparenleft}blast\ elim{\isacharcolon}less{\isacharunderscore}SucE{\isacharparenright}%
\begin{isamarkuptext}%
\noindent
The elimination rule \isa{less{\isacharunderscore}SucE} expresses the case distinction:
\begin{isabelle}%
\ \ \ \ \ {\isasymlbrakk}m\ {\isacharless}\ Suc\ n{\isacharsemicolon}\ m\ {\isacharless}\ n\ {\isasymLongrightarrow}\ P{\isacharsemicolon}\ m\ {\isacharequal}\ n\ {\isasymLongrightarrow}\ P{\isasymrbrakk}\ {\isasymLongrightarrow}\ P%
\end{isabelle}

Now it is straightforward to derive the original version of
\isa{nat{\isacharunderscore}less{\isacharunderscore}induct} by manipulting the conclusion of the above lemma:
instantiate \isa{n} by \isa{Suc\ n} and \isa{m} by \isa{n} and
remove the trivial condition \isa{n\ {\isacharless}\ Suc\ n}. Fortunately, this
happens automatically when we add the lemma as a new premise to the
desired goal:%
\end{isamarkuptext}%
\isacommand{theorem}\ nat{\isacharunderscore}less{\isacharunderscore}induct{\isacharcolon}\ {\isachardoublequote}{\isacharparenleft}{\isasymAnd}n{\isacharcolon}{\isacharcolon}nat{\isachardot}\ {\isasymforall}m{\isacharless}n{\isachardot}\ P\ m\ {\isasymLongrightarrow}\ P\ n{\isacharparenright}\ {\isasymLongrightarrow}\ P\ n{\isachardoublequote}\isanewline
\isacommand{by}{\isacharparenleft}insert\ induct{\isacharunderscore}lem{\isacharcomma}\ blast{\isacharparenright}%
\begin{isamarkuptext}%
Finally we should remind the reader that HOL already provides the mother of
all inductions, well-founded induction (see \S\ref{sec:Well-founded}).  For
example theorem \isa{nat{\isacharunderscore}less{\isacharunderscore}induct} is
a special case of \isa{wf{\isacharunderscore}induct} where \isa{r} is \isa{{\isacharless}} on
\isa{nat}. The details can be found in theory \isa{Wellfounded_Recursion}.%
\end{isamarkuptext}%
\end{isabellebody}%
%%% Local Variables:
%%% mode: latex
%%% TeX-master: "root"
%%% End:

%
\begin{isabellebody}%
\def\isabellecontext{CTLind}%
%
\isamarkupsubsection{CTL revisited%
}
%
\begin{isamarkuptext}%
\label{sec:CTL-revisited}
The purpose of this section is twofold: we want to demonstrate
some of the induction principles and heuristics discussed above and we want to
show how inductive definitions can simplify proofs.
In \S\ref{sec:CTL} we gave a fairly involved proof of the correctness of a
model checker for CTL\@. In particular the proof of the
\isa{infinity{\isacharunderscore}lemma} on the way to \isa{AF{\isacharunderscore}lemma{\isadigit{2}}} is not as
simple as one might intuitively expect, due to the \isa{SOME} operator
involved. Below we give a simpler proof of \isa{AF{\isacharunderscore}lemma{\isadigit{2}}}
based on an auxiliary inductive definition.

Let us call a (finite or infinite) path \emph{\isa{A}-avoiding} if it does
not touch any node in the set \isa{A}. Then \isa{AF{\isacharunderscore}lemma{\isadigit{2}}} says
that if no infinite path from some state \isa{s} is \isa{A}-avoiding,
then \isa{s\ {\isasymin}\ lfp\ {\isacharparenleft}af\ A{\isacharparenright}}. We prove this by inductively defining the set
\isa{Avoid\ s\ A} of states reachable from \isa{s} by a finite \isa{A}-avoiding path:
% Second proof of opposite direction, directly by well-founded induction
% on the initial segment of M that avoids A.%
\end{isamarkuptext}%
\isacommand{consts}\ Avoid\ {\isacharcolon}{\isacharcolon}\ {\isachardoublequote}state\ {\isasymRightarrow}\ state\ set\ {\isasymRightarrow}\ state\ set{\isachardoublequote}\isanewline
\isacommand{inductive}\ {\isachardoublequote}Avoid\ s\ A{\isachardoublequote}\isanewline
\isakeyword{intros}\ {\isachardoublequote}s\ {\isasymin}\ Avoid\ s\ A{\isachardoublequote}\isanewline
\ \ \ \ \ \ \ {\isachardoublequote}{\isasymlbrakk}\ t\ {\isasymin}\ Avoid\ s\ A{\isacharsemicolon}\ t\ {\isasymnotin}\ A{\isacharsemicolon}\ {\isacharparenleft}t{\isacharcomma}u{\isacharparenright}\ {\isasymin}\ M\ {\isasymrbrakk}\ {\isasymLongrightarrow}\ u\ {\isasymin}\ Avoid\ s\ A{\isachardoublequote}%
\begin{isamarkuptext}%
It is easy to see that for any infinite \isa{A}-avoiding path \isa{f}
with \isa{f\ {\isadigit{0}}\ {\isasymin}\ Avoid\ s\ A} there is an infinite \isa{A}-avoiding path
starting with \isa{s} because (by definition of \isa{Avoid}) there is a
finite \isa{A}-avoiding path from \isa{s} to \isa{f\ {\isadigit{0}}}.
The proof is by induction on \isa{f\ {\isadigit{0}}\ {\isasymin}\ Avoid\ s\ A}. However,
this requires the following
reformulation, as explained in \S\ref{sec:ind-var-in-prems} above;
the \isa{rule{\isacharunderscore}format} directive undoes the reformulation after the proof.%
\end{isamarkuptext}%
\isacommand{lemma}\ ex{\isacharunderscore}infinite{\isacharunderscore}path{\isacharbrackleft}rule{\isacharunderscore}format{\isacharbrackright}{\isacharcolon}\isanewline
\ \ {\isachardoublequote}t\ {\isasymin}\ Avoid\ s\ A\ \ {\isasymLongrightarrow}\isanewline
\ \ \ {\isasymforall}f{\isasymin}Paths\ t{\isachardot}\ {\isacharparenleft}{\isasymforall}i{\isachardot}\ f\ i\ {\isasymnotin}\ A{\isacharparenright}\ {\isasymlongrightarrow}\ {\isacharparenleft}{\isasymexists}p{\isasymin}Paths\ s{\isachardot}\ {\isasymforall}i{\isachardot}\ p\ i\ {\isasymnotin}\ A{\isacharparenright}{\isachardoublequote}\isanewline
\isacommand{apply}{\isacharparenleft}erule\ Avoid{\isachardot}induct{\isacharparenright}\isanewline
\ \isacommand{apply}{\isacharparenleft}blast{\isacharparenright}\isanewline
\isacommand{apply}{\isacharparenleft}clarify{\isacharparenright}\isanewline
\isacommand{apply}{\isacharparenleft}drule{\isacharunderscore}tac\ x\ {\isacharequal}\ {\isachardoublequote}{\isasymlambda}i{\isachardot}\ case\ i\ of\ {\isadigit{0}}\ {\isasymRightarrow}\ t\ {\isacharbar}\ Suc\ i\ {\isasymRightarrow}\ f\ i{\isachardoublequote}\ \isakeyword{in}\ bspec{\isacharparenright}\isanewline
\isacommand{apply}{\isacharparenleft}simp{\isacharunderscore}all\ add{\isacharcolon}Paths{\isacharunderscore}def\ split{\isacharcolon}nat{\isachardot}split{\isacharparenright}\isanewline
\isacommand{done}%
\begin{isamarkuptext}%
\noindent
The base case (\isa{t\ {\isacharequal}\ s}) is trivial (\isa{blast}).
In the induction step, we have an infinite \isa{A}-avoiding path \isa{f}
starting from \isa{u}, a successor of \isa{t}. Now we simply instantiate
the \isa{{\isasymforall}f{\isasymin}Paths\ t} in the induction hypothesis by the path starting with
\isa{t} and continuing with \isa{f}. That is what the above $\lambda$-term
expresses. That fact that this is a path starting with \isa{t} and that
the instantiated induction hypothesis implies the conclusion is shown by
simplification.

Now we come to the key lemma. It says that if \isa{t} can be reached by a
finite \isa{A}-avoiding path from \isa{s}, then \isa{t\ {\isasymin}\ lfp\ {\isacharparenleft}af\ A{\isacharparenright}},
provided there is no infinite \isa{A}-avoiding path starting from \isa{s}.%
\end{isamarkuptext}%
\isacommand{lemma}\ Avoid{\isacharunderscore}in{\isacharunderscore}lfp{\isacharbrackleft}rule{\isacharunderscore}format{\isacharparenleft}no{\isacharunderscore}asm{\isacharparenright}{\isacharbrackright}{\isacharcolon}\isanewline
\ \ {\isachardoublequote}{\isasymforall}p{\isasymin}Paths\ s{\isachardot}\ {\isasymexists}i{\isachardot}\ p\ i\ {\isasymin}\ A\ {\isasymLongrightarrow}\ t\ {\isasymin}\ Avoid\ s\ A\ {\isasymlongrightarrow}\ t\ {\isasymin}\ lfp{\isacharparenleft}af\ A{\isacharparenright}{\isachardoublequote}%
\begin{isamarkuptxt}%
\noindent
The trick is not to induct on \isa{t\ {\isasymin}\ Avoid\ s\ A}, as already the base
case would be a problem, but to proceed by well-founded induction \isa{t}. Hence \isa{t\ {\isasymin}\ Avoid\ s\ A} needs to be brought into the conclusion as
well, which the directive \isa{rule{\isacharunderscore}format} undoes at the end (see below).
But induction with respect to which well-founded relation? The restriction
of \isa{M} to \isa{Avoid\ s\ A}:
\begin{isabelle}%
\ \ \ \ \ {\isacharbraceleft}{\isacharparenleft}y{\isacharcomma}\ x{\isacharparenright}{\isachardot}\ {\isacharparenleft}x{\isacharcomma}\ y{\isacharparenright}\ {\isasymin}\ M\ {\isasymand}\ x\ {\isasymin}\ Avoid\ s\ A\ {\isasymand}\ y\ {\isasymin}\ Avoid\ s\ A\ {\isasymand}\ x\ {\isasymnotin}\ A{\isacharbraceright}%
\end{isabelle}
As we shall see in a moment, the absence of infinite \isa{A}-avoiding paths
starting from \isa{s} implies well-foundedness of this relation. For the
moment we assume this and proceed with the induction:%
\end{isamarkuptxt}%
\isacommand{apply}{\isacharparenleft}subgoal{\isacharunderscore}tac\isanewline
\ \ {\isachardoublequote}wf{\isacharbraceleft}{\isacharparenleft}y{\isacharcomma}x{\isacharparenright}{\isachardot}\ {\isacharparenleft}x{\isacharcomma}y{\isacharparenright}{\isasymin}M\ {\isasymand}\ x\ {\isasymin}\ Avoid\ s\ A\ {\isasymand}\ y\ {\isasymin}\ Avoid\ s\ A\ {\isasymand}\ x\ {\isasymnotin}\ A{\isacharbraceright}{\isachardoublequote}{\isacharparenright}\isanewline
\ \isacommand{apply}{\isacharparenleft}erule{\isacharunderscore}tac\ a\ {\isacharequal}\ t\ \isakeyword{in}\ wf{\isacharunderscore}induct{\isacharparenright}\isanewline
\ \isacommand{apply}{\isacharparenleft}clarsimp{\isacharparenright}%
\begin{isamarkuptxt}%
\noindent
Now can assume additionally (induction hypothesis) that if \isa{t\ {\isasymnotin}\ A}
then all successors of \isa{t} that are in \isa{Avoid\ s\ A} are in
\isa{lfp\ {\isacharparenleft}af\ A{\isacharparenright}}. To prove the actual goal we unfold \isa{lfp} once. Now
we have to prove that \isa{t} is in \isa{A} or all successors of \isa{t} are in \isa{lfp\ {\isacharparenleft}af\ A{\isacharparenright}}. If \isa{t} is not in \isa{A}, the second
\isa{Avoid}-rule implies that all successors of \isa{t} are in
\isa{Avoid\ s\ A} (because we also assume \isa{t\ {\isasymin}\ Avoid\ s\ A}), and
hence, by the induction hypothesis, all successors of \isa{t} are indeed in
\isa{lfp\ {\isacharparenleft}af\ A{\isacharparenright}}. Mechanically:%
\end{isamarkuptxt}%
\ \isacommand{apply}{\isacharparenleft}rule\ ssubst\ {\isacharbrackleft}OF\ lfp{\isacharunderscore}unfold{\isacharbrackleft}OF\ mono{\isacharunderscore}af{\isacharbrackright}{\isacharbrackright}{\isacharparenright}\isanewline
\ \isacommand{apply}{\isacharparenleft}simp\ only{\isacharcolon}\ af{\isacharunderscore}def{\isacharparenright}\isanewline
\ \isacommand{apply}{\isacharparenleft}blast\ intro{\isacharcolon}Avoid{\isachardot}intros{\isacharparenright}%
\begin{isamarkuptxt}%
Having proved the main goal we return to the proof obligation that the above
relation is indeed well-founded. This is proved by contraposition: we assume
the relation is not well-founded. Thus there exists an infinite \isa{A}-avoiding path all in \isa{Avoid\ s\ A}, by theorem
\isa{wf{\isacharunderscore}iff{\isacharunderscore}no{\isacharunderscore}infinite{\isacharunderscore}down{\isacharunderscore}chain}:
\begin{isabelle}%
\ \ \ \ \ wf\ r\ {\isacharequal}\ {\isacharparenleft}{\isasymnot}\ {\isacharparenleft}{\isasymexists}f{\isachardot}\ {\isasymforall}i{\isachardot}\ {\isacharparenleft}f\ {\isacharparenleft}Suc\ i{\isacharparenright}{\isacharcomma}\ f\ i{\isacharparenright}\ {\isasymin}\ r{\isacharparenright}{\isacharparenright}%
\end{isabelle}
From lemma \isa{ex{\isacharunderscore}infinite{\isacharunderscore}path} the existence of an infinite
\isa{A}-avoiding path starting in \isa{s} follows, just as required for
the contraposition.%
\end{isamarkuptxt}%
\isacommand{apply}{\isacharparenleft}erule\ contrapos{\isacharunderscore}pp{\isacharparenright}\isanewline
\isacommand{apply}{\isacharparenleft}simp\ add{\isacharcolon}wf{\isacharunderscore}iff{\isacharunderscore}no{\isacharunderscore}infinite{\isacharunderscore}down{\isacharunderscore}chain{\isacharparenright}\isanewline
\isacommand{apply}{\isacharparenleft}erule\ exE{\isacharparenright}\isanewline
\isacommand{apply}{\isacharparenleft}rule\ ex{\isacharunderscore}infinite{\isacharunderscore}path{\isacharparenright}\isanewline
\isacommand{apply}{\isacharparenleft}auto\ simp\ add{\isacharcolon}Paths{\isacharunderscore}def{\isacharparenright}\isanewline
\isacommand{done}%
\begin{isamarkuptext}%
The \isa{{\isacharparenleft}no{\isacharunderscore}asm{\isacharparenright}} modifier of the \isa{rule{\isacharunderscore}format} directive means
that the assumption is left unchanged---otherwise the \isa{{\isasymforall}p} is turned
into a \isa{{\isasymAnd}p}, which would complicate matters below. As it is,
\isa{Avoid{\isacharunderscore}in{\isacharunderscore}lfp} is now
\begin{isabelle}%
\ \ \ \ \ {\isasymlbrakk}{\isasymforall}p{\isasymin}Paths\ s{\isachardot}\ {\isasymexists}i{\isachardot}\ p\ i\ {\isasymin}\ A{\isacharsemicolon}\ t\ {\isasymin}\ Avoid\ s\ A{\isasymrbrakk}\ {\isasymLongrightarrow}\ t\ {\isasymin}\ lfp\ {\isacharparenleft}af\ A{\isacharparenright}%
\end{isabelle}
The main theorem is simply the corollary where \isa{t\ {\isacharequal}\ s},
in which case the assumption \isa{t\ {\isasymin}\ Avoid\ s\ A} is trivially true
by the first \isa{Avoid}-rule). Isabelle confirms this:%
\end{isamarkuptext}%
\isacommand{theorem}\ AF{\isacharunderscore}lemma{\isadigit{2}}{\isacharcolon}\isanewline
\ \ {\isachardoublequote}{\isacharbraceleft}s{\isachardot}\ {\isasymforall}p\ {\isasymin}\ Paths\ s{\isachardot}\ {\isasymexists}\ i{\isachardot}\ p\ i\ {\isasymin}\ A{\isacharbraceright}\ {\isasymsubseteq}\ lfp{\isacharparenleft}af\ A{\isacharparenright}{\isachardoublequote}\isanewline
\isacommand{by}{\isacharparenleft}auto\ elim{\isacharcolon}Avoid{\isacharunderscore}in{\isacharunderscore}lfp\ intro{\isacharcolon}Avoid{\isachardot}intros{\isacharparenright}\isanewline
\isanewline
\end{isabellebody}%
%%% Local Variables:
%%% mode: latex
%%% TeX-master: "root"
%%% End:

\index{induction|)}
