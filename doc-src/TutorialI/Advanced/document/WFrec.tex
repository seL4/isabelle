%
\begin{isabellebody}%
\def\isabellecontext{WFrec}%
%
\begin{isamarkuptext}%
\noindent
So far, all recursive definitions where shown to terminate via measure
functions. Sometimes this can be quite inconvenient or even
impossible. Fortunately, \isacommand{recdef} supports much more
general definitions. For example, termination of Ackermann's function
can be shown by means of the lexicographic product \isa{{\isacharless}{\isacharasterisk}lex{\isacharasterisk}{\isachargreater}}:%
\end{isamarkuptext}%
\isacommand{consts}\ ack\ {\isacharcolon}{\isacharcolon}\ {\isachardoublequote}nat{\isasymtimes}nat\ {\isasymRightarrow}\ nat{\isachardoublequote}\isanewline
\isacommand{recdef}\ ack\ {\isachardoublequote}measure{\isacharparenleft}{\isasymlambda}m{\isachardot}\ m{\isacharparenright}\ {\isacharless}{\isacharasterisk}lex{\isacharasterisk}{\isachargreater}\ measure{\isacharparenleft}{\isasymlambda}n{\isachardot}\ n{\isacharparenright}{\isachardoublequote}\isanewline
\ \ {\isachardoublequote}ack{\isacharparenleft}{\isadigit{0}}{\isacharcomma}n{\isacharparenright}\ \ \ \ \ \ \ \ \ {\isacharequal}\ Suc\ n{\isachardoublequote}\isanewline
\ \ {\isachardoublequote}ack{\isacharparenleft}Suc\ m{\isacharcomma}{\isadigit{0}}{\isacharparenright}\ \ \ \ \ {\isacharequal}\ ack{\isacharparenleft}m{\isacharcomma}\ {\isadigit{1}}{\isacharparenright}{\isachardoublequote}\isanewline
\ \ {\isachardoublequote}ack{\isacharparenleft}Suc\ m{\isacharcomma}Suc\ n{\isacharparenright}\ {\isacharequal}\ ack{\isacharparenleft}m{\isacharcomma}ack{\isacharparenleft}Suc\ m{\isacharcomma}n{\isacharparenright}{\isacharparenright}{\isachardoublequote}%
\begin{isamarkuptext}%
\noindent
The lexicographic product decreases if either its first component
decreases (as in the second equation and in the outer call in the
third equation) or its first component stays the same and the second
component decreases (as in the inner call in the third equation).

In general, \isacommand{recdef} supports termination proofs based on
arbitrary \emph{wellfounded relations}, i.e.\ \emph{wellfounded
recursion}\indexbold{recursion!wellfounded}\index{wellfounded
recursion|see{recursion, wellfounded}}.  A relation $<$ is
\bfindex{wellfounded} if it has no infinite descending chain $\cdots <
a@2 < a@1 < a@0$. Clearly, a function definition is total iff the set
of all pairs $(r,l)$, where $l$ is the argument on the left-hand side
of an equation and $r$ the argument of some recursive call on the
corresponding right-hand side, induces a wellfounded relation.  For a
systematic account of termination proofs via wellfounded relations
see, for example, \cite{Baader-Nipkow}. The HOL library formalizes
some of the theory of wellfounded relations. For example
\isa{wf\ r}\index{*wf|bold} means that relation \isa{r{\isasymColon}{\isacharparenleft}{\isacharprime}a\ {\isasymtimes}\ {\isacharprime}a{\isacharparenright}\ set} is
wellfounded.

Each \isacommand{recdef} definition should be accompanied (after the
name of the function) by a wellfounded relation on the argument type
of the function. For example, \isa{measure} is defined by
\begin{isabelle}%
\ \ \ \ \ measure\ f\ {\isasymequiv}\ {\isacharbraceleft}{\isacharparenleft}y{\isacharcomma}\ x{\isacharparenright}{\isachardot}\ f\ y\ {\isacharless}\ f\ x{\isacharbraceright}%
\end{isabelle}
and it has been proved that \isa{measure\ f} is always wellfounded.

In addition to \isa{measure}, the library provides
a number of further constructions for obtaining wellfounded relations.
Above we have already met \isa{{\isacharless}{\isacharasterisk}lex{\isacharasterisk}{\isachargreater}} of type
\begin{isabelle}%
\ \ \ \ \ {\isachardoublequote}{\isacharparenleft}{\isacharprime}a\ {\isasymtimes}\ {\isacharprime}a{\isacharparenright}set\ {\isasymRightarrow}\ {\isacharparenleft}{\isacharprime}b\ {\isasymtimes}\ {\isacharprime}b{\isacharparenright}set\ {\isasymRightarrow}\ {\isacharparenleft}{\isacharparenleft}{\isacharprime}a\ {\isasymtimes}\ {\isacharprime}b{\isacharparenright}\ {\isasymtimes}\ {\isacharparenleft}{\isacharprime}a\ {\isasymtimes}\ {\isacharprime}b{\isacharparenright}{\isacharparenright}set{\isachardoublequote}%
\end{isabelle}
Of course the lexicographic product can also be interated, as in the following
function definition:%
\end{isamarkuptext}%
\isacommand{consts}\ contrived\ {\isacharcolon}{\isacharcolon}\ {\isachardoublequote}nat\ {\isasymtimes}\ nat\ {\isasymtimes}\ nat\ {\isasymRightarrow}\ nat{\isachardoublequote}\isanewline
\isacommand{recdef}\ contrived\isanewline
\ \ {\isachardoublequote}measure{\isacharparenleft}{\isasymlambda}i{\isachardot}\ i{\isacharparenright}\ {\isacharless}{\isacharasterisk}lex{\isacharasterisk}{\isachargreater}\ measure{\isacharparenleft}{\isasymlambda}j{\isachardot}\ j{\isacharparenright}\ {\isacharless}{\isacharasterisk}lex{\isacharasterisk}{\isachargreater}\ measure{\isacharparenleft}{\isasymlambda}k{\isachardot}\ k{\isacharparenright}{\isachardoublequote}\isanewline
{\isachardoublequote}contrived{\isacharparenleft}i{\isacharcomma}j{\isacharcomma}Suc\ k{\isacharparenright}\ {\isacharequal}\ contrived{\isacharparenleft}i{\isacharcomma}j{\isacharcomma}k{\isacharparenright}{\isachardoublequote}\isanewline
{\isachardoublequote}contrived{\isacharparenleft}i{\isacharcomma}Suc\ j{\isacharcomma}{\isadigit{0}}{\isacharparenright}\ {\isacharequal}\ contrived{\isacharparenleft}i{\isacharcomma}j{\isacharcomma}j{\isacharparenright}{\isachardoublequote}\isanewline
{\isachardoublequote}contrived{\isacharparenleft}Suc\ i{\isacharcomma}{\isadigit{0}}{\isacharcomma}{\isadigit{0}}{\isacharparenright}\ {\isacharequal}\ contrived{\isacharparenleft}i{\isacharcomma}i{\isacharcomma}i{\isacharparenright}{\isachardoublequote}\isanewline
{\isachardoublequote}contrived{\isacharparenleft}{\isadigit{0}}{\isacharcomma}{\isadigit{0}}{\isacharcomma}{\isadigit{0}}{\isacharparenright}\ \ \ \ \ {\isacharequal}\ {\isadigit{0}}{\isachardoublequote}%
\begin{isamarkuptext}%
Lexicographic products of measure functions already go a long way. A
further useful construction is the embedding of some type in an
existing wellfounded relation via the inverse image of a function:
\begin{isabelle}%
\ \ \ \ \ inv{\isacharunderscore}image\ {\isacharparenleft}r{\isasymColon}{\isacharparenleft}{\isacharprime}b\ {\isasymtimes}\ {\isacharprime}b{\isacharparenright}\ set{\isacharparenright}\ {\isacharparenleft}f{\isasymColon}{\isacharprime}a\ {\isasymRightarrow}\ {\isacharprime}b{\isacharparenright}\ {\isasymequiv}\isanewline
\ \ \ \ \ {\isacharbraceleft}{\isacharparenleft}x{\isasymColon}{\isacharprime}a{\isacharcomma}\ y{\isasymColon}{\isacharprime}a{\isacharparenright}{\isachardot}\ {\isacharparenleft}f\ x{\isacharcomma}\ f\ y{\isacharparenright}\ {\isasymin}\ r{\isacharbraceright}%
\end{isabelle}
\begin{sloppypar}
\noindent
For example, \isa{measure} is actually defined as \isa{inv{\isacharunderscore}mage\ less{\isacharunderscore}than}, where
\isa{less{\isacharunderscore}than} of type \isa{{\isacharparenleft}nat\ {\isasymtimes}\ nat{\isacharparenright}\ set} is the less-than relation on type \isa{nat}
(as opposed to \isa{op\ {\isacharless}}, which is of type \isa{{\isacharbrackleft}nat{\isacharcomma}\ nat{\isacharbrackright}\ {\isasymRightarrow}\ bool}).
\end{sloppypar}

%Finally there is also {finite_psubset} the proper subset relation on finite sets

All the above constructions are known to \isacommand{recdef}. Thus you
will never have to prove wellfoundedness of any relation composed
solely of these building blocks. But of course the proof of
termination of your function definition, i.e.\ that the arguments
decrease with every recursive call, may still require you to provide
additional lemmas.

It is also possible to use your own wellfounded relations with \isacommand{recdef}.
Here is a simplistic example:%
\end{isamarkuptext}%
\isacommand{consts}\ f\ {\isacharcolon}{\isacharcolon}\ {\isachardoublequote}nat\ {\isasymRightarrow}\ nat{\isachardoublequote}\isanewline
\isacommand{recdef}\ f\ {\isachardoublequote}id{\isacharparenleft}less{\isacharunderscore}than{\isacharparenright}{\isachardoublequote}\isanewline
{\isachardoublequote}f\ {\isadigit{0}}\ {\isacharequal}\ {\isadigit{0}}{\isachardoublequote}\isanewline
{\isachardoublequote}f\ {\isacharparenleft}Suc\ n{\isacharparenright}\ {\isacharequal}\ f\ n{\isachardoublequote}%
\begin{isamarkuptext}%
Since \isacommand{recdef} is not prepared for \isa{id}, the identity
function, this leads to the complaint that it could not prove
\isa{wf\ {\isacharparenleft}id\ less{\isacharunderscore}than{\isacharparenright}}, the wellfoundedness of \isa{id\ less{\isacharunderscore}than}. We should first have proved that \isa{id} preserves wellfoundedness%
\end{isamarkuptext}%
\isacommand{lemma}\ wf{\isacharunderscore}id{\isacharcolon}\ {\isachardoublequote}wf\ r\ {\isasymLongrightarrow}\ wf{\isacharparenleft}id\ r{\isacharparenright}{\isachardoublequote}\isanewline
\isacommand{by}\ simp%
\begin{isamarkuptext}%
\noindent
and should have added the following hint to our above definition:%
\end{isamarkuptext}%
{\isacharparenleft}\isakeyword{hints}\ recdef{\isacharunderscore}wf\ add{\isacharcolon}\ wf{\isacharunderscore}id{\isacharparenright}\end{isabellebody}%
%%% Local Variables:
%%% mode: latex
%%% TeX-master: "root"
%%% End:
