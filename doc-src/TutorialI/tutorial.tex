\documentclass{article}
\newif\ifremarks
%\remarkstrue          %TRUE causes remarks to be displayed (as marginal notes)
\remarksfalse
\usepackage{cl2emono-modified,isabelle,isabellesym}
\usepackage{../proof,amsmath,amsfonts}
\usepackage{latexsym,verbatim,graphicx,../iman,../extra,../ttbox,comment}
\usepackage{../pdfsetup}    %last package!

%\newtheorem{theorem}{Theorem}[section]
\newtheorem{Exercise}{Exercise}[section]
\newenvironment{exercise}{\begin{Exercise}\rm}{\end{Exercise}}
\newcommand{\ttlbr}{\texttt{[|}}
\newcommand{\ttrbr}{\texttt{|]}}
\newcommand{\ttor}{\texttt{|}}
\newcommand{\ttall}{\texttt{!}}
\newcommand{\ttuniquex}{\texttt{?!}}
\newcommand{\ttEXU}{\texttt{EX!}}
\newcommand{\ttAnd}{\texttt{!!}}

\newcommand{\isasymimp}{\isasymlongrightarrow}
\newcommand{\isasymImp}{\isasymLongrightarrow}
\newcommand{\isasymFun}{\isasymRightarrow}
\newcommand{\isasymuniqex}{\isamath{\exists!\,}}
\renewcommand{\S}{Sect.\ts}

\renewenvironment{isamarkuptxt}{\begin{isamarkuptext}}{\end{isamarkuptext}}

%% lcp's macros
\newcommand{\REMARK}[1]{\ifremarks\marginpar{\raggedright\footnotesize#1}\fi}
\newcommand{\rulename}[1]{\hfill$(\text{#1})$} %names of Isabelle rules
\let\bigisa=\isa
%% was previously
%% \newcommand{\bigisa}[1]{\texttt{\textsl{#1}}} 
%% because \isa is too small for variables, but does it really matter?


%%% to index derived rls:  ^\([a-zA-Z0-9][a-zA-Z0-9_]*\)        \\tdx{\1}  
%%% to index rulenames:   ^ *(\([a-zA-Z0-9][a-zA-Z0-9_]*\),     \\tdx{\1}  
%%% to index constants:   \\tt \([a-zA-Z0-9][a-zA-Z0-9_]*\)     \\cdx{\1}  
%%% to deverbify:         \\verb|\([^|]*\)|     \\ttindex{\1}  
%% run    ../sedindex logics    to prepare index file

\makeindex
\newcommand{\indexboldpos}[2]{#1\indexbold{#2@#1}}
\newcommand{\ttindexboldpos}[2]{\texttt{#1}\indexbold{#2@\texttt{#1}}}
\newcommand{\isaindexbold}[1]{\isa{#1}\index{*#1|bold}}
\newcommand{\isaindex}[1]{\isa{#1}\index{*#1}}

\index{termination|see{total function}}
\index{product type|see{pair}}
\index{tuple|see{pair}}
\index{*<*lex*>|see{lexicographic product}}

\underscoreoff

\setcounter{secnumdepth}{2} \setcounter{tocdepth}{2}  %% {secnumdepth}{2}???

\pagestyle{headings}
%\sloppy
%\binperiod     %%%treat . like a binary operator

\begin{document}
\title{\includegraphics[scale=.8]{isabelle_hol}
       \\ \vspace{0.5cm} The Tutorial
       \\ --- DRAFT ---}
\author{Tobias Nipkow \& Lawrence Paulson\\[1ex]
Technische Universit{\"a}t M{\"u}nchen \\
Institut f{\"u}r Informatik \\[1ex]
University of Cambridge\\
Computer Laboratory}
\maketitle

\pagenumbering{roman}
\tableofcontents

\subsubsection*{Acknowledgements}
This tutorial owes a lot to the constant discussions with and the valuable
feedback from the Isabelle group at Munich: Stefan Berghofer, Olaf M{\"u}ller,
Wolfgang Naraschewski, David von Oheimb, Leonor Prensa Nieto, Cornelia Pusch,
Martin Strecker and Markus Wenzel. Stephan Merz was also kind enough to
read and comment on a draft version.
\clearfirst


%FIXME
%\chapter{Basic Concepts}\label{ch:basics}
%\section{The Isar proof language}

%%% Local Variables: 
%%% mode: latex
%%% TeX-master: "isar-ref"
%%% End: 

\chapter{Functional Programming in HOL}

This chapter describes how to write
functional programs in HOL and how to verify them.  However, 
most of the constructs and
proof procedures introduced are general and recur in any specification
or verification task.  We really should speak of functional
\emph{modelling} rather than functional \emph{programming}: 
our primary aim is not
to write programs but to design abstract models of systems.  HOL is
a specification language that goes well beyond what can be expressed as a
program. However, for the time being we concentrate on the computable.

If you are a purist functional programmer, please note that all functions
in HOL must be total:
they must terminate for all inputs.  Lazy data structures are not
directly available.

\section{An Introductory Theory}
\label{sec:intro-theory}

Functional programming needs datatypes and functions. Both of them can be
defined in a theory with a syntax reminiscent of languages like ML or
Haskell. As an example consider the theory in figure~\ref{fig:ToyList}.
We will now examine it line by line.

\begin{figure}[htbp]
\begin{ttbox}\makeatother
\input{ToyList2/ToyList1}\end{ttbox}
\caption{A Theory of Lists}
\label{fig:ToyList}
\end{figure}

\index{*ToyList example|(}
{\makeatother\medskip%
\begin{isabellebody}%
\def\isabellecontext{ToyList}%
%
\isadelimtheory
%
\endisadelimtheory
%
\isatagtheory
\isacommand{theory}\isamarkupfalse%
\ ToyList\isanewline
\isakeyword{imports}\ Datatype\isanewline
\isakeyword{begin}%
\endisatagtheory
{\isafoldtheory}%
%
\isadelimtheory
%
\endisadelimtheory
%
\begin{isamarkuptext}%
\noindent
HOL already has a predefined theory of lists called \isa{List} ---
\isa{ToyList} is merely a small fragment of it chosen as an example. In
contrast to what is recommended in \S\ref{sec:Basic:Theories},
\isa{ToyList} is not based on \isa{Main} but on \isa{Datatype}, a
theory that contains pretty much everything but lists, thus avoiding
ambiguities caused by defining lists twice.%
\end{isamarkuptext}%
\isamarkuptrue%
\isacommand{datatype}\isamarkupfalse%
\ {\isaliteral{27}{\isacharprime}}a\ list\ {\isaliteral{3D}{\isacharequal}}\ Nil\ \ \ \ \ \ \ \ \ \ \ \ \ \ \ \ \ \ \ \ \ \ \ \ \ \ {\isaliteral{28}{\isacharparenleft}}{\isaliteral{22}{\isachardoublequoteopen}}{\isaliteral{5B}{\isacharbrackleft}}{\isaliteral{5D}{\isacharbrackright}}{\isaliteral{22}{\isachardoublequoteclose}}{\isaliteral{29}{\isacharparenright}}\isanewline
\ \ \ \ \ \ \ \ \ \ \ \ \ \ \ \ \ {\isaliteral{7C}{\isacharbar}}\ Cons\ {\isaliteral{27}{\isacharprime}}a\ {\isaliteral{22}{\isachardoublequoteopen}}{\isaliteral{27}{\isacharprime}}a\ list{\isaliteral{22}{\isachardoublequoteclose}}\ \ \ \ \ \ \ \ \ \ \ \ {\isaliteral{28}{\isacharparenleft}}\isakeyword{infixr}\ {\isaliteral{22}{\isachardoublequoteopen}}{\isaliteral{23}{\isacharhash}}{\isaliteral{22}{\isachardoublequoteclose}}\ {\isadigit{6}}{\isadigit{5}}{\isaliteral{29}{\isacharparenright}}%
\begin{isamarkuptext}%
\noindent
The datatype\index{datatype@\isacommand {datatype} (command)}
\tydx{list} introduces two
constructors \cdx{Nil} and \cdx{Cons}, the
empty~list and the operator that adds an element to the front of a list. For
example, the term \isa{Cons True (Cons False Nil)} is a value of
type \isa{bool\ list}, namely the list with the elements \isa{True} and
\isa{False}. Because this notation quickly becomes unwieldy, the
datatype declaration is annotated with an alternative syntax: instead of
\isa{Nil} and \isa{Cons x xs} we can write
\isa{{\isaliteral{5B}{\isacharbrackleft}}{\isaliteral{5D}{\isacharbrackright}}}\index{$HOL2list@\isa{[]}|bold} and
\isa{x\ {\isaliteral{23}{\isacharhash}}\ xs}\index{$HOL2list@\isa{\#}|bold}. In fact, this
alternative syntax is the familiar one.  Thus the list \isa{Cons True
(Cons False Nil)} becomes \isa{True\ {\isaliteral{23}{\isacharhash}}\ False\ {\isaliteral{23}{\isacharhash}}\ {\isaliteral{5B}{\isacharbrackleft}}{\isaliteral{5D}{\isacharbrackright}}}. The annotation
\isacommand{infixr}\index{infixr@\isacommand{infixr} (annotation)} 
means that \isa{{\isaliteral{23}{\isacharhash}}} associates to
the right: the term \isa{x\ {\isaliteral{23}{\isacharhash}}\ y\ {\isaliteral{23}{\isacharhash}}\ z} is read as \isa{x\ {\isaliteral{23}{\isacharhash}}\ {\isaliteral{28}{\isacharparenleft}}y\ {\isaliteral{23}{\isacharhash}}\ z{\isaliteral{29}{\isacharparenright}}}
and not as \isa{{\isaliteral{28}{\isacharparenleft}}x\ {\isaliteral{23}{\isacharhash}}\ y{\isaliteral{29}{\isacharparenright}}\ {\isaliteral{23}{\isacharhash}}\ z}.
The \isa{{\isadigit{6}}{\isadigit{5}}} is the priority of the infix \isa{{\isaliteral{23}{\isacharhash}}}.

\begin{warn}
  Syntax annotations can be powerful, but they are difficult to master and 
  are never necessary.  You
  could drop them from theory \isa{ToyList} and go back to the identifiers
  \isa{Nil} and \isa{Cons}.  Novices should avoid using
  syntax annotations in their own theories.
\end{warn}
Next, two functions \isa{app} and \cdx{rev} are defined recursively,
in this order, because Isabelle insists on definition before use:%
\end{isamarkuptext}%
\isamarkuptrue%
\isacommand{primrec}\isamarkupfalse%
\ app\ {\isaliteral{3A}{\isacharcolon}}{\isaliteral{3A}{\isacharcolon}}\ {\isaliteral{22}{\isachardoublequoteopen}}{\isaliteral{27}{\isacharprime}}a\ list\ {\isaliteral{5C3C52696768746172726F773E}{\isasymRightarrow}}\ {\isaliteral{27}{\isacharprime}}a\ list\ {\isaliteral{5C3C52696768746172726F773E}{\isasymRightarrow}}\ {\isaliteral{27}{\isacharprime}}a\ list{\isaliteral{22}{\isachardoublequoteclose}}\ {\isaliteral{28}{\isacharparenleft}}\isakeyword{infixr}\ {\isaliteral{22}{\isachardoublequoteopen}}{\isaliteral{40}{\isacharat}}{\isaliteral{22}{\isachardoublequoteclose}}\ {\isadigit{6}}{\isadigit{5}}{\isaliteral{29}{\isacharparenright}}\ \isakeyword{where}\isanewline
{\isaliteral{22}{\isachardoublequoteopen}}{\isaliteral{5B}{\isacharbrackleft}}{\isaliteral{5D}{\isacharbrackright}}\ {\isaliteral{40}{\isacharat}}\ ys\ \ \ \ \ \ \ {\isaliteral{3D}{\isacharequal}}\ ys{\isaliteral{22}{\isachardoublequoteclose}}\ {\isaliteral{7C}{\isacharbar}}\isanewline
{\isaliteral{22}{\isachardoublequoteopen}}{\isaliteral{28}{\isacharparenleft}}x\ {\isaliteral{23}{\isacharhash}}\ xs{\isaliteral{29}{\isacharparenright}}\ {\isaliteral{40}{\isacharat}}\ ys\ {\isaliteral{3D}{\isacharequal}}\ x\ {\isaliteral{23}{\isacharhash}}\ {\isaliteral{28}{\isacharparenleft}}xs\ {\isaliteral{40}{\isacharat}}\ ys{\isaliteral{29}{\isacharparenright}}{\isaliteral{22}{\isachardoublequoteclose}}\isanewline
\isanewline
\isacommand{primrec}\isamarkupfalse%
\ rev\ {\isaliteral{3A}{\isacharcolon}}{\isaliteral{3A}{\isacharcolon}}\ {\isaliteral{22}{\isachardoublequoteopen}}{\isaliteral{27}{\isacharprime}}a\ list\ {\isaliteral{5C3C52696768746172726F773E}{\isasymRightarrow}}\ {\isaliteral{27}{\isacharprime}}a\ list{\isaliteral{22}{\isachardoublequoteclose}}\ \isakeyword{where}\isanewline
{\isaliteral{22}{\isachardoublequoteopen}}rev\ {\isaliteral{5B}{\isacharbrackleft}}{\isaliteral{5D}{\isacharbrackright}}\ \ \ \ \ \ \ \ {\isaliteral{3D}{\isacharequal}}\ {\isaliteral{5B}{\isacharbrackleft}}{\isaliteral{5D}{\isacharbrackright}}{\isaliteral{22}{\isachardoublequoteclose}}\ {\isaliteral{7C}{\isacharbar}}\isanewline
{\isaliteral{22}{\isachardoublequoteopen}}rev\ {\isaliteral{28}{\isacharparenleft}}x\ {\isaliteral{23}{\isacharhash}}\ xs{\isaliteral{29}{\isacharparenright}}\ \ {\isaliteral{3D}{\isacharequal}}\ {\isaliteral{28}{\isacharparenleft}}rev\ xs{\isaliteral{29}{\isacharparenright}}\ {\isaliteral{40}{\isacharat}}\ {\isaliteral{28}{\isacharparenleft}}x\ {\isaliteral{23}{\isacharhash}}\ {\isaliteral{5B}{\isacharbrackleft}}{\isaliteral{5D}{\isacharbrackright}}{\isaliteral{29}{\isacharparenright}}{\isaliteral{22}{\isachardoublequoteclose}}%
\begin{isamarkuptext}%
\noindent
Each function definition is of the form
\begin{center}
\isacommand{primrec} \textit{name} \isa{{\isaliteral{3A}{\isacharcolon}}{\isaliteral{3A}{\isacharcolon}}} \textit{type} \textit{(optional syntax)} \isakeyword{where} \textit{equations}
\end{center}
The equations must be separated by \isa{{\isaliteral{7C}{\isacharbar}}}.
%
Function \isa{app} is annotated with concrete syntax. Instead of the
prefix syntax \isa{app\ xs\ ys} the infix
\isa{xs\ {\isaliteral{40}{\isacharat}}\ ys}\index{$HOL2list@\isa{\at}|bold} becomes the preferred
form.

\index{*rev (constant)|(}\index{append function|(}
The equations for \isa{app} and \isa{rev} hardly need comments:
\isa{app} appends two lists and \isa{rev} reverses a list.  The
keyword \commdx{primrec} indicates that the recursion is
of a particularly primitive kind where each recursive call peels off a datatype
constructor from one of the arguments.  Thus the
recursion always terminates, i.e.\ the function is \textbf{total}.
\index{functions!total}

The termination requirement is absolutely essential in HOL, a logic of total
functions. If we were to drop it, inconsistencies would quickly arise: the
``definition'' $f(n) = f(n)+1$ immediately leads to $0 = 1$ by subtracting
$f(n)$ on both sides.
% However, this is a subtle issue that we cannot discuss here further.

\begin{warn}
  As we have indicated, the requirement for total functions is an essential characteristic of HOL\@. It is only
  because of totality that reasoning in HOL is comparatively easy.  More
  generally, the philosophy in HOL is to refrain from asserting arbitrary axioms (such as
  function definitions whose totality has not been proved) because they
  quickly lead to inconsistencies. Instead, fixed constructs for introducing
  types and functions are offered (such as \isacommand{datatype} and
  \isacommand{primrec}) which are guaranteed to preserve consistency.
\end{warn}

\index{syntax}%
A remark about syntax.  The textual definition of a theory follows a fixed
syntax with keywords like \isacommand{datatype} and \isacommand{end}.
% (see Fig.~\ref{fig:keywords} in Appendix~\ref{sec:Appendix} for a full list).
Embedded in this syntax are the types and formulae of HOL, whose syntax is
extensible (see \S\ref{sec:concrete-syntax}), e.g.\ by new user-defined infix operators.
To distinguish the two levels, everything
HOL-specific (terms and types) should be enclosed in
\texttt{"}\dots\texttt{"}. 
To lessen this burden, quotation marks around a single identifier can be
dropped, unless the identifier happens to be a keyword, for example
\isa{"end"}.
When Isabelle prints a syntax error message, it refers to the HOL syntax as
the \textbf{inner syntax} and the enclosing theory language as the \textbf{outer syntax}.

Comments\index{comment} must be in enclosed in \texttt{(* }and\texttt{ *)}.

\section{Evaluation}
\index{evaluation}

Assuming you have processed the declarations and definitions of
\texttt{ToyList} presented so far, you may want to test your
functions by running them. For example, what is the value of
\isa{rev\ {\isaliteral{28}{\isacharparenleft}}True\ {\isaliteral{23}{\isacharhash}}\ False\ {\isaliteral{23}{\isacharhash}}\ {\isaliteral{5B}{\isacharbrackleft}}{\isaliteral{5D}{\isacharbrackright}}{\isaliteral{29}{\isacharparenright}}}? Command%
\end{isamarkuptext}%
\isamarkuptrue%
\isacommand{value}\isamarkupfalse%
\ {\isaliteral{22}{\isachardoublequoteopen}}rev\ {\isaliteral{28}{\isacharparenleft}}True\ {\isaliteral{23}{\isacharhash}}\ False\ {\isaliteral{23}{\isacharhash}}\ {\isaliteral{5B}{\isacharbrackleft}}{\isaliteral{5D}{\isacharbrackright}}{\isaliteral{29}{\isacharparenright}}{\isaliteral{22}{\isachardoublequoteclose}}%
\begin{isamarkuptext}%
\noindent yields the correct result \isa{False\ {\isaliteral{23}{\isacharhash}}\ True\ {\isaliteral{23}{\isacharhash}}\ {\isaliteral{5B}{\isacharbrackleft}}{\isaliteral{5D}{\isacharbrackright}}}.
But we can go beyond mere functional programming and evaluate terms with
variables in them, executing functions symbolically:%
\end{isamarkuptext}%
\isamarkuptrue%
\isacommand{value}\isamarkupfalse%
\ {\isaliteral{22}{\isachardoublequoteopen}}rev\ {\isaliteral{28}{\isacharparenleft}}a\ {\isaliteral{23}{\isacharhash}}\ b\ {\isaliteral{23}{\isacharhash}}\ c\ {\isaliteral{23}{\isacharhash}}\ {\isaliteral{5B}{\isacharbrackleft}}{\isaliteral{5D}{\isacharbrackright}}{\isaliteral{29}{\isacharparenright}}{\isaliteral{22}{\isachardoublequoteclose}}%
\begin{isamarkuptext}%
\noindent yields \isa{c\ {\isaliteral{23}{\isacharhash}}\ b\ {\isaliteral{23}{\isacharhash}}\ a\ {\isaliteral{23}{\isacharhash}}\ {\isaliteral{5B}{\isacharbrackleft}}{\isaliteral{5D}{\isacharbrackright}}}.

\section{An Introductory Proof}
\label{sec:intro-proof}

Having convinced ourselves (as well as one can by testing) that our
definitions capture our intentions, we are ready to prove a few simple
theorems. This will illustrate not just the basic proof commands but
also the typical proof process.

\subsubsection*{Main Goal.}

Our goal is to show that reversing a list twice produces the original
list.%
\end{isamarkuptext}%
\isamarkuptrue%
\isacommand{theorem}\isamarkupfalse%
\ rev{\isaliteral{5F}{\isacharunderscore}}rev\ {\isaliteral{5B}{\isacharbrackleft}}simp{\isaliteral{5D}{\isacharbrackright}}{\isaliteral{3A}{\isacharcolon}}\ {\isaliteral{22}{\isachardoublequoteopen}}rev{\isaliteral{28}{\isacharparenleft}}rev\ xs{\isaliteral{29}{\isacharparenright}}\ {\isaliteral{3D}{\isacharequal}}\ xs{\isaliteral{22}{\isachardoublequoteclose}}%
\isadelimproof
%
\endisadelimproof
%
\isatagproof
%
\begin{isamarkuptxt}%
\index{theorem@\isacommand {theorem} (command)|bold}%
\noindent
This \isacommand{theorem} command does several things:
\begin{itemize}
\item
It establishes a new theorem to be proved, namely \isa{rev\ {\isaliteral{28}{\isacharparenleft}}rev\ xs{\isaliteral{29}{\isacharparenright}}\ {\isaliteral{3D}{\isacharequal}}\ xs}.
\item
It gives that theorem the name \isa{rev{\isaliteral{5F}{\isacharunderscore}}rev}, for later reference.
\item
It tells Isabelle (via the bracketed attribute \attrdx{simp}) to take the eventual theorem as a simplification rule: future proofs involving
simplification will replace occurrences of \isa{rev\ {\isaliteral{28}{\isacharparenleft}}rev\ xs{\isaliteral{29}{\isacharparenright}}} by
\isa{xs}.
\end{itemize}
The name and the simplification attribute are optional.
Isabelle's response is to print the initial proof state consisting
of some header information (like how many subgoals there are) followed by
\begin{isabelle}%
\ {\isadigit{1}}{\isaliteral{2E}{\isachardot}}\ rev\ {\isaliteral{28}{\isacharparenleft}}rev\ xs{\isaliteral{29}{\isacharparenright}}\ {\isaliteral{3D}{\isacharequal}}\ xs%
\end{isabelle}
For compactness reasons we omit the header in this tutorial.
Until we have finished a proof, the \rmindex{proof state} proper
always looks like this:
\begin{isabelle}
~1.~$G\sb{1}$\isanewline
~~\vdots~~\isanewline
~$n$.~$G\sb{n}$
\end{isabelle}
The numbered lines contain the subgoals $G\sb{1}$, \dots, $G\sb{n}$
that we need to prove to establish the main goal.\index{subgoals}
Initially there is only one subgoal, which is identical with the
main goal. (If you always want to see the main goal as well,
set the flag \isa{Proof.show_main_goal}\index{*show_main_goal (flag)}
--- this flag used to be set by default.)

Let us now get back to \isa{rev\ {\isaliteral{28}{\isacharparenleft}}rev\ xs{\isaliteral{29}{\isacharparenright}}\ {\isaliteral{3D}{\isacharequal}}\ xs}. Properties of recursively
defined functions are best established by induction. In this case there is
nothing obvious except induction on \isa{xs}:%
\end{isamarkuptxt}%
\isamarkuptrue%
\isacommand{apply}\isamarkupfalse%
{\isaliteral{28}{\isacharparenleft}}induct{\isaliteral{5F}{\isacharunderscore}}tac\ xs{\isaliteral{29}{\isacharparenright}}%
\begin{isamarkuptxt}%
\noindent\index{*induct_tac (method)}%
This tells Isabelle to perform induction on variable \isa{xs}. The suffix
\isa{tac} stands for \textbf{tactic},\index{tactics}
a synonym for ``theorem proving function''.
By default, induction acts on the first subgoal. The new proof state contains
two subgoals, namely the base case (\isa{Nil}) and the induction step
(\isa{Cons}):
\begin{isabelle}%
\ {\isadigit{1}}{\isaliteral{2E}{\isachardot}}\ rev\ {\isaliteral{28}{\isacharparenleft}}rev\ {\isaliteral{5B}{\isacharbrackleft}}{\isaliteral{5D}{\isacharbrackright}}{\isaliteral{29}{\isacharparenright}}\ {\isaliteral{3D}{\isacharequal}}\ {\isaliteral{5B}{\isacharbrackleft}}{\isaliteral{5D}{\isacharbrackright}}\isanewline
\ {\isadigit{2}}{\isaliteral{2E}{\isachardot}}\ {\isaliteral{5C3C416E643E}{\isasymAnd}}a\ list{\isaliteral{2E}{\isachardot}}\isanewline
\isaindent{\ {\isadigit{2}}{\isaliteral{2E}{\isachardot}}\ \ \ \ }rev\ {\isaliteral{28}{\isacharparenleft}}rev\ list{\isaliteral{29}{\isacharparenright}}\ {\isaliteral{3D}{\isacharequal}}\ list\ {\isaliteral{5C3C4C6F6E6772696768746172726F773E}{\isasymLongrightarrow}}\ rev\ {\isaliteral{28}{\isacharparenleft}}rev\ {\isaliteral{28}{\isacharparenleft}}a\ {\isaliteral{23}{\isacharhash}}\ list{\isaliteral{29}{\isacharparenright}}{\isaliteral{29}{\isacharparenright}}\ {\isaliteral{3D}{\isacharequal}}\ a\ {\isaliteral{23}{\isacharhash}}\ list%
\end{isabelle}

The induction step is an example of the general format of a subgoal:\index{subgoals}
\begin{isabelle}
~$i$.~{\isasymAnd}$x\sb{1}$~\dots$x\sb{n}$.~{\it assumptions}~{\isasymLongrightarrow}~{\it conclusion}
\end{isabelle}\index{$IsaAnd@\isasymAnd|bold}
The prefix of bound variables \isasymAnd$x\sb{1}$~\dots~$x\sb{n}$ can be
ignored most of the time, or simply treated as a list of variables local to
this subgoal. Their deeper significance is explained in Chapter~\ref{chap:rules}.
The {\it assumptions}\index{assumptions!of subgoal}
are the local assumptions for this subgoal and {\it
  conclusion}\index{conclusion!of subgoal} is the actual proposition to be proved. 
Typical proof steps
that add new assumptions are induction and case distinction. In our example
the only assumption is the induction hypothesis \isa{rev\ {\isaliteral{28}{\isacharparenleft}}rev\ list{\isaliteral{29}{\isacharparenright}}\ {\isaliteral{3D}{\isacharequal}}\ list}, where \isa{list} is a variable name chosen by Isabelle. If there
are multiple assumptions, they are enclosed in the bracket pair
\indexboldpos{\isasymlbrakk}{$Isabrl} and
\indexboldpos{\isasymrbrakk}{$Isabrr} and separated by semicolons.

Let us try to solve both goals automatically:%
\end{isamarkuptxt}%
\isamarkuptrue%
\isacommand{apply}\isamarkupfalse%
{\isaliteral{28}{\isacharparenleft}}auto{\isaliteral{29}{\isacharparenright}}%
\begin{isamarkuptxt}%
\noindent
This command tells Isabelle to apply a proof strategy called
\isa{auto} to all subgoals. Essentially, \isa{auto} tries to
simplify the subgoals.  In our case, subgoal~1 is solved completely (thanks
to the equation \isa{rev\ {\isaliteral{5B}{\isacharbrackleft}}{\isaliteral{5D}{\isacharbrackright}}\ {\isaliteral{3D}{\isacharequal}}\ {\isaliteral{5B}{\isacharbrackleft}}{\isaliteral{5D}{\isacharbrackright}}}) and disappears; the simplified version
of subgoal~2 becomes the new subgoal~1:
\begin{isabelle}%
\ {\isadigit{1}}{\isaliteral{2E}{\isachardot}}\ {\isaliteral{5C3C416E643E}{\isasymAnd}}a\ list{\isaliteral{2E}{\isachardot}}\isanewline
\isaindent{\ {\isadigit{1}}{\isaliteral{2E}{\isachardot}}\ \ \ \ }rev\ {\isaliteral{28}{\isacharparenleft}}rev\ list{\isaliteral{29}{\isacharparenright}}\ {\isaliteral{3D}{\isacharequal}}\ list\ {\isaliteral{5C3C4C6F6E6772696768746172726F773E}{\isasymLongrightarrow}}\ rev\ {\isaliteral{28}{\isacharparenleft}}rev\ list\ {\isaliteral{40}{\isacharat}}\ a\ {\isaliteral{23}{\isacharhash}}\ {\isaliteral{5B}{\isacharbrackleft}}{\isaliteral{5D}{\isacharbrackright}}{\isaliteral{29}{\isacharparenright}}\ {\isaliteral{3D}{\isacharequal}}\ a\ {\isaliteral{23}{\isacharhash}}\ list%
\end{isabelle}
In order to simplify this subgoal further, a lemma suggests itself.%
\end{isamarkuptxt}%
\isamarkuptrue%
%
\endisatagproof
{\isafoldproof}%
%
\isadelimproof
%
\endisadelimproof
%
\isamarkupsubsubsection{First Lemma%
}
\isamarkuptrue%
%
\begin{isamarkuptext}%
\indexbold{abandoning a proof}\indexbold{proofs!abandoning}
After abandoning the above proof attempt (at the shell level type
\commdx{oops}) we start a new proof:%
\end{isamarkuptext}%
\isamarkuptrue%
\isacommand{lemma}\isamarkupfalse%
\ rev{\isaliteral{5F}{\isacharunderscore}}app\ {\isaliteral{5B}{\isacharbrackleft}}simp{\isaliteral{5D}{\isacharbrackright}}{\isaliteral{3A}{\isacharcolon}}\ {\isaliteral{22}{\isachardoublequoteopen}}rev{\isaliteral{28}{\isacharparenleft}}xs\ {\isaliteral{40}{\isacharat}}\ ys{\isaliteral{29}{\isacharparenright}}\ {\isaliteral{3D}{\isacharequal}}\ {\isaliteral{28}{\isacharparenleft}}rev\ ys{\isaliteral{29}{\isacharparenright}}\ {\isaliteral{40}{\isacharat}}\ {\isaliteral{28}{\isacharparenleft}}rev\ xs{\isaliteral{29}{\isacharparenright}}{\isaliteral{22}{\isachardoublequoteclose}}%
\isadelimproof
%
\endisadelimproof
%
\isatagproof
%
\begin{isamarkuptxt}%
\noindent The keywords \commdx{theorem} and
\commdx{lemma} are interchangeable and merely indicate
the importance we attach to a proposition.  Therefore we use the words
\emph{theorem} and \emph{lemma} pretty much interchangeably, too.

There are two variables that we could induct on: \isa{xs} and
\isa{ys}. Because \isa{{\isaliteral{40}{\isacharat}}} is defined by recursion on
the first argument, \isa{xs} is the correct one:%
\end{isamarkuptxt}%
\isamarkuptrue%
\isacommand{apply}\isamarkupfalse%
{\isaliteral{28}{\isacharparenleft}}induct{\isaliteral{5F}{\isacharunderscore}}tac\ xs{\isaliteral{29}{\isacharparenright}}%
\begin{isamarkuptxt}%
\noindent
This time not even the base case is solved automatically:%
\end{isamarkuptxt}%
\isamarkuptrue%
\isacommand{apply}\isamarkupfalse%
{\isaliteral{28}{\isacharparenleft}}auto{\isaliteral{29}{\isacharparenright}}%
\begin{isamarkuptxt}%
\begin{isabelle}%
\ {\isadigit{1}}{\isaliteral{2E}{\isachardot}}\ rev\ ys\ {\isaliteral{3D}{\isacharequal}}\ rev\ ys\ {\isaliteral{40}{\isacharat}}\ {\isaliteral{5B}{\isacharbrackleft}}{\isaliteral{5D}{\isacharbrackright}}%
\end{isabelle}
Again, we need to abandon this proof attempt and prove another simple lemma
first. In the future the step of abandoning an incomplete proof before
embarking on the proof of a lemma usually remains implicit.%
\end{isamarkuptxt}%
\isamarkuptrue%
%
\endisatagproof
{\isafoldproof}%
%
\isadelimproof
%
\endisadelimproof
%
\isamarkupsubsubsection{Second Lemma%
}
\isamarkuptrue%
%
\begin{isamarkuptext}%
We again try the canonical proof procedure:%
\end{isamarkuptext}%
\isamarkuptrue%
\isacommand{lemma}\isamarkupfalse%
\ app{\isaliteral{5F}{\isacharunderscore}}Nil{\isadigit{2}}\ {\isaliteral{5B}{\isacharbrackleft}}simp{\isaliteral{5D}{\isacharbrackright}}{\isaliteral{3A}{\isacharcolon}}\ {\isaliteral{22}{\isachardoublequoteopen}}xs\ {\isaliteral{40}{\isacharat}}\ {\isaliteral{5B}{\isacharbrackleft}}{\isaliteral{5D}{\isacharbrackright}}\ {\isaliteral{3D}{\isacharequal}}\ xs{\isaliteral{22}{\isachardoublequoteclose}}\isanewline
%
\isadelimproof
%
\endisadelimproof
%
\isatagproof
\isacommand{apply}\isamarkupfalse%
{\isaliteral{28}{\isacharparenleft}}induct{\isaliteral{5F}{\isacharunderscore}}tac\ xs{\isaliteral{29}{\isacharparenright}}\isanewline
\isacommand{apply}\isamarkupfalse%
{\isaliteral{28}{\isacharparenleft}}auto{\isaliteral{29}{\isacharparenright}}%
\begin{isamarkuptxt}%
\noindent
It works, yielding the desired message \isa{No\ subgoals{\isaliteral{21}{\isacharbang}}}:
\begin{isabelle}%
xs\ {\isaliteral{40}{\isacharat}}\ {\isaliteral{5B}{\isacharbrackleft}}{\isaliteral{5D}{\isacharbrackright}}\ {\isaliteral{3D}{\isacharequal}}\ xs\isanewline
No\ subgoals{\isaliteral{21}{\isacharbang}}%
\end{isabelle}
We still need to confirm that the proof is now finished:%
\end{isamarkuptxt}%
\isamarkuptrue%
\isacommand{done}\isamarkupfalse%
%
\endisatagproof
{\isafoldproof}%
%
\isadelimproof
%
\endisadelimproof
%
\begin{isamarkuptext}%
\noindent
As a result of that final \commdx{done}, Isabelle associates the lemma just proved
with its name. In this tutorial, we sometimes omit to show that final \isacommand{done}
if it is obvious from the context that the proof is finished.

% Instead of \isacommand{apply} followed by a dot, you can simply write
% \isacommand{by}\indexbold{by}, which we do most of the time.
Notice that in lemma \isa{app{\isaliteral{5F}{\isacharunderscore}}Nil{\isadigit{2}}},
as printed out after the final \isacommand{done}, the free variable \isa{xs} has been
replaced by the unknown \isa{{\isaliteral{3F}{\isacharquery}}xs}, just as explained in
\S\ref{sec:variables}.

Going back to the proof of the first lemma%
\end{isamarkuptext}%
\isamarkuptrue%
\isacommand{lemma}\isamarkupfalse%
\ rev{\isaliteral{5F}{\isacharunderscore}}app\ {\isaliteral{5B}{\isacharbrackleft}}simp{\isaliteral{5D}{\isacharbrackright}}{\isaliteral{3A}{\isacharcolon}}\ {\isaliteral{22}{\isachardoublequoteopen}}rev{\isaliteral{28}{\isacharparenleft}}xs\ {\isaliteral{40}{\isacharat}}\ ys{\isaliteral{29}{\isacharparenright}}\ {\isaliteral{3D}{\isacharequal}}\ {\isaliteral{28}{\isacharparenleft}}rev\ ys{\isaliteral{29}{\isacharparenright}}\ {\isaliteral{40}{\isacharat}}\ {\isaliteral{28}{\isacharparenleft}}rev\ xs{\isaliteral{29}{\isacharparenright}}{\isaliteral{22}{\isachardoublequoteclose}}\isanewline
%
\isadelimproof
%
\endisadelimproof
%
\isatagproof
\isacommand{apply}\isamarkupfalse%
{\isaliteral{28}{\isacharparenleft}}induct{\isaliteral{5F}{\isacharunderscore}}tac\ xs{\isaliteral{29}{\isacharparenright}}\isanewline
\isacommand{apply}\isamarkupfalse%
{\isaliteral{28}{\isacharparenleft}}auto{\isaliteral{29}{\isacharparenright}}%
\begin{isamarkuptxt}%
\noindent
we find that this time \isa{auto} solves the base case, but the
induction step merely simplifies to
\begin{isabelle}%
\ {\isadigit{1}}{\isaliteral{2E}{\isachardot}}\ {\isaliteral{5C3C416E643E}{\isasymAnd}}a\ list{\isaliteral{2E}{\isachardot}}\isanewline
\isaindent{\ {\isadigit{1}}{\isaliteral{2E}{\isachardot}}\ \ \ \ }rev\ {\isaliteral{28}{\isacharparenleft}}list\ {\isaliteral{40}{\isacharat}}\ ys{\isaliteral{29}{\isacharparenright}}\ {\isaliteral{3D}{\isacharequal}}\ rev\ ys\ {\isaliteral{40}{\isacharat}}\ rev\ list\ {\isaliteral{5C3C4C6F6E6772696768746172726F773E}{\isasymLongrightarrow}}\isanewline
\isaindent{\ {\isadigit{1}}{\isaliteral{2E}{\isachardot}}\ \ \ \ }{\isaliteral{28}{\isacharparenleft}}rev\ ys\ {\isaliteral{40}{\isacharat}}\ rev\ list{\isaliteral{29}{\isacharparenright}}\ {\isaliteral{40}{\isacharat}}\ a\ {\isaliteral{23}{\isacharhash}}\ {\isaliteral{5B}{\isacharbrackleft}}{\isaliteral{5D}{\isacharbrackright}}\ {\isaliteral{3D}{\isacharequal}}\ rev\ ys\ {\isaliteral{40}{\isacharat}}\ rev\ list\ {\isaliteral{40}{\isacharat}}\ a\ {\isaliteral{23}{\isacharhash}}\ {\isaliteral{5B}{\isacharbrackleft}}{\isaliteral{5D}{\isacharbrackright}}%
\end{isabelle}
Now we need to remember that \isa{{\isaliteral{40}{\isacharat}}} associates to the right, and that
\isa{{\isaliteral{23}{\isacharhash}}} and \isa{{\isaliteral{40}{\isacharat}}} have the same priority (namely the \isa{{\isadigit{6}}{\isadigit{5}}}
in their \isacommand{infixr} annotation). Thus the conclusion really is
\begin{isabelle}
~~~~~(rev~ys~@~rev~list)~@~(a~\#~[])~=~rev~ys~@~(rev~list~@~(a~\#~[]))
\end{isabelle}
and the missing lemma is associativity of \isa{{\isaliteral{40}{\isacharat}}}.%
\end{isamarkuptxt}%
\isamarkuptrue%
%
\endisatagproof
{\isafoldproof}%
%
\isadelimproof
%
\endisadelimproof
%
\isamarkupsubsubsection{Third Lemma%
}
\isamarkuptrue%
%
\begin{isamarkuptext}%
Abandoning the previous attempt, the canonical proof procedure
succeeds without further ado.%
\end{isamarkuptext}%
\isamarkuptrue%
\isacommand{lemma}\isamarkupfalse%
\ app{\isaliteral{5F}{\isacharunderscore}}assoc\ {\isaliteral{5B}{\isacharbrackleft}}simp{\isaliteral{5D}{\isacharbrackright}}{\isaliteral{3A}{\isacharcolon}}\ {\isaliteral{22}{\isachardoublequoteopen}}{\isaliteral{28}{\isacharparenleft}}xs\ {\isaliteral{40}{\isacharat}}\ ys{\isaliteral{29}{\isacharparenright}}\ {\isaliteral{40}{\isacharat}}\ zs\ {\isaliteral{3D}{\isacharequal}}\ xs\ {\isaliteral{40}{\isacharat}}\ {\isaliteral{28}{\isacharparenleft}}ys\ {\isaliteral{40}{\isacharat}}\ zs{\isaliteral{29}{\isacharparenright}}{\isaliteral{22}{\isachardoublequoteclose}}\isanewline
%
\isadelimproof
%
\endisadelimproof
%
\isatagproof
\isacommand{apply}\isamarkupfalse%
{\isaliteral{28}{\isacharparenleft}}induct{\isaliteral{5F}{\isacharunderscore}}tac\ xs{\isaliteral{29}{\isacharparenright}}\isanewline
\isacommand{apply}\isamarkupfalse%
{\isaliteral{28}{\isacharparenleft}}auto{\isaliteral{29}{\isacharparenright}}\isanewline
\isacommand{done}\isamarkupfalse%
%
\endisatagproof
{\isafoldproof}%
%
\isadelimproof
%
\endisadelimproof
%
\begin{isamarkuptext}%
\noindent
Now we can prove the first lemma:%
\end{isamarkuptext}%
\isamarkuptrue%
\isacommand{lemma}\isamarkupfalse%
\ rev{\isaliteral{5F}{\isacharunderscore}}app\ {\isaliteral{5B}{\isacharbrackleft}}simp{\isaliteral{5D}{\isacharbrackright}}{\isaliteral{3A}{\isacharcolon}}\ {\isaliteral{22}{\isachardoublequoteopen}}rev{\isaliteral{28}{\isacharparenleft}}xs\ {\isaliteral{40}{\isacharat}}\ ys{\isaliteral{29}{\isacharparenright}}\ {\isaliteral{3D}{\isacharequal}}\ {\isaliteral{28}{\isacharparenleft}}rev\ ys{\isaliteral{29}{\isacharparenright}}\ {\isaliteral{40}{\isacharat}}\ {\isaliteral{28}{\isacharparenleft}}rev\ xs{\isaliteral{29}{\isacharparenright}}{\isaliteral{22}{\isachardoublequoteclose}}\isanewline
%
\isadelimproof
%
\endisadelimproof
%
\isatagproof
\isacommand{apply}\isamarkupfalse%
{\isaliteral{28}{\isacharparenleft}}induct{\isaliteral{5F}{\isacharunderscore}}tac\ xs{\isaliteral{29}{\isacharparenright}}\isanewline
\isacommand{apply}\isamarkupfalse%
{\isaliteral{28}{\isacharparenleft}}auto{\isaliteral{29}{\isacharparenright}}\isanewline
\isacommand{done}\isamarkupfalse%
%
\endisatagproof
{\isafoldproof}%
%
\isadelimproof
%
\endisadelimproof
%
\begin{isamarkuptext}%
\noindent
Finally, we prove our main theorem:%
\end{isamarkuptext}%
\isamarkuptrue%
\isacommand{theorem}\isamarkupfalse%
\ rev{\isaliteral{5F}{\isacharunderscore}}rev\ {\isaliteral{5B}{\isacharbrackleft}}simp{\isaliteral{5D}{\isacharbrackright}}{\isaliteral{3A}{\isacharcolon}}\ {\isaliteral{22}{\isachardoublequoteopen}}rev{\isaliteral{28}{\isacharparenleft}}rev\ xs{\isaliteral{29}{\isacharparenright}}\ {\isaliteral{3D}{\isacharequal}}\ xs{\isaliteral{22}{\isachardoublequoteclose}}\isanewline
%
\isadelimproof
%
\endisadelimproof
%
\isatagproof
\isacommand{apply}\isamarkupfalse%
{\isaliteral{28}{\isacharparenleft}}induct{\isaliteral{5F}{\isacharunderscore}}tac\ xs{\isaliteral{29}{\isacharparenright}}\isanewline
\isacommand{apply}\isamarkupfalse%
{\isaliteral{28}{\isacharparenleft}}auto{\isaliteral{29}{\isacharparenright}}\isanewline
\isacommand{done}\isamarkupfalse%
%
\endisatagproof
{\isafoldproof}%
%
\isadelimproof
%
\endisadelimproof
%
\begin{isamarkuptext}%
\noindent
The final \commdx{end} tells Isabelle to close the current theory because
we are finished with its development:%
\index{*rev (constant)|)}\index{append function|)}%
\end{isamarkuptext}%
\isamarkuptrue%
%
\isadelimtheory
%
\endisadelimtheory
%
\isatagtheory
\isacommand{end}\isamarkupfalse%
%
\endisatagtheory
{\isafoldtheory}%
%
\isadelimtheory
%
\endisadelimtheory
\isanewline
\end{isabellebody}%
%%% Local Variables:
%%% mode: latex
%%% TeX-master: "root"
%%% End:
}

The complete proof script is shown in Fig.\ts\ref{fig:ToyList-proofs}. The
concatenation of Figs.\ts\ref{fig:ToyList} and~\ref{fig:ToyList-proofs}
constitutes the complete theory \texttt{ToyList} and should reside in file
\texttt{ToyList.thy}.
% It is good practice to present all declarations and
%definitions at the beginning of a theory to facilitate browsing.%
\index{*ToyList example|)}

\begin{figure}[htbp]
\begin{ttbox}\makeatother
\input{ToyList2/ToyList2}\end{ttbox}
\caption{Proofs about Lists}
\label{fig:ToyList-proofs}
\end{figure}

\subsubsection*{Review}

This is the end of our toy proof. It should have familiarized you with
\begin{itemize}
\item the standard theorem proving procedure:
state a goal (lemma or theorem); proceed with proof until a separate lemma is
required; prove that lemma; come back to the original goal.
\item a specific procedure that works well for functional programs:
induction followed by all-out simplification via \isa{auto}.
\item a basic repertoire of proof commands.
\end{itemize}

\begin{warn}
It is tempting to think that all lemmas should have the \isa{simp} attribute
just because this was the case in the example above. However, in that example
all lemmas were equations, and the right-hand side was simpler than the
left-hand side --- an ideal situation for simplification purposes. Unless
this is clearly the case, novices should refrain from awarding a lemma the
\isa{simp} attribute, which has a global effect. Instead, lemmas can be
applied locally where they are needed, which is discussed in the following
chapter.
\end{warn}

\section{Some Helpful Commands}
\label{sec:commands-and-hints}

This section discusses a few basic commands for manipulating the proof state
and can be skipped by casual readers.

There are two kinds of commands used during a proof: the actual proof
commands and auxiliary commands for examining the proof state and controlling
the display. Simple proof commands are of the form
\commdx{apply}(\textit{method}), where \textit{method} is typically 
\isa{induct_tac} or \isa{auto}.  All such theorem proving operations
are referred to as \bfindex{methods}, and further ones are
introduced throughout the tutorial.  Unless stated otherwise, you may
assume that a method attacks merely the first subgoal. An exception is
\isa{auto}, which tries to solve all subgoals.

The most useful auxiliary commands are as follows:
\begin{description}
\item[Modifying the order of subgoals:]
\commdx{defer} moves the first subgoal to the end and
\commdx{prefer}~$n$ moves subgoal $n$ to the front.
\item[Printing theorems:]
  \commdx{thm}~\textit{name}$@1$~\dots~\textit{name}$@n$
  prints the named theorems.
\item[Reading terms and types:] \commdx{term}
  \textit{string} reads, type-checks and prints the given string as a term in
  the current context; the inferred type is output as well.
  \commdx{typ} \textit{string} reads and prints the given
  string as a type in the current context.
\end{description}
Further commands are found in the Isabelle/Isar Reference
Manual~\cite{isabelle-isar-ref}.

\begin{pgnote}
Clicking on the \pgmenu{State} button redisplays the current proof state.
This is helpful in case commands like \isacommand{thm} have overwritten it.
\end{pgnote}

We now examine Isabelle's functional programming constructs systematically,
starting with inductive datatypes.


\section{Datatypes}
\label{sec:datatype}

\index{datatypes|(}%
Inductive datatypes are part of almost every non-trivial application of HOL.
First we take another look at an important example, the datatype of
lists, before we turn to datatypes in general. The section closes with a
case study.


\subsection{Lists}

\index{*list (type)}%
Lists are one of the essential datatypes in computing.  We expect that you
are already familiar with their basic operations.
Theory \isa{ToyList} is only a small fragment of HOL's predefined theory
\thydx{List}\footnote{\url{http://isabelle.in.tum.de/library/HOL/List.html}}.
The latter contains many further operations. For example, the functions
\cdx{hd} (``head'') and \cdx{tl} (``tail'') return the first
element and the remainder of a list. (However, pattern matching is usually
preferable to \isa{hd} and \isa{tl}.)  
Also available are higher-order functions like \isa{map} and \isa{filter}.
Theory \isa{List} also contains
more syntactic sugar: \isa{[}$x@1$\isa{,}\dots\isa{,}$x@n$\isa{]} abbreviates
$x@1$\isa{\#}\dots\isa{\#}$x@n$\isa{\#[]}.  In the rest of the tutorial we
always use HOL's predefined lists by building on theory \isa{Main}.
\begin{warn}
Looking ahead to sets and quanifiers in Part II:
The best way to express that some element \isa{x} is in a list \isa{xs} is
\isa{x $\in$ set xs}, where \isa{set} is a function that turns a list into the
set of its elements.
By the same device you can also write bounded quantifiers like
\isa{$\forall$x $\in$ set xs} or embed lists in other set expressions.
\end{warn}


\subsection{The General Format}
\label{sec:general-datatype}

The general HOL \isacommand{datatype} definition is of the form
\[
\isacommand{datatype}~(\alpha@1, \dots, \alpha@n) \, t ~=~
C@1~\tau@{11}~\dots~\tau@{1k@1} ~\mid~ \dots ~\mid~
C@m~\tau@{m1}~\dots~\tau@{mk@m}
\]
where $\alpha@i$ are distinct type variables (the parameters), $C@i$ are distinct
constructor names and $\tau@{ij}$ are types; it is customary to capitalize
the first letter in constructor names. There are a number of
restrictions (such as that the type should not be empty) detailed
elsewhere~\cite{isabelle-HOL}. Isabelle notifies you if you violate them.

Laws about datatypes, such as \isa{[] \isasymnoteq~x\#xs} and
\isa{(x\#xs = y\#ys) = (x=y \isasymand~xs=ys)}, are used automatically
during proofs by simplification.  The same is true for the equations in
primitive recursive function definitions.

Every\footnote{Except for advanced datatypes where the recursion involves
``\isasymRightarrow'' as in {\S}\ref{sec:nested-fun-datatype}.} datatype $t$
comes equipped with a \isa{size} function from $t$ into the natural numbers
(see~{\S}\ref{sec:nat} below). For lists, \isa{size} is just the length, i.e.\
\isa{size [] = 0} and \isa{size(x \# xs) = size xs + 1}.  In general,
\cdx{size} returns
\begin{itemize}
\item zero for all constructors that do not have an argument of type $t$,
\item one plus the sum of the sizes of all arguments of type~$t$,
for all other constructors.
\end{itemize}
Note that because
\isa{size} is defined on every datatype, it is overloaded; on lists
\isa{size} is also called \sdx{length}, which is not overloaded.
Isabelle will always show \isa{size} on lists as \isa{length}.


\subsection{Primitive Recursion}

\index{recursion!primitive}%
Functions on datatypes are usually defined by recursion. In fact, most of the
time they are defined by what is called \textbf{primitive recursion} over some
datatype $t$. This means that the recursion equations must be of the form
\[ f \, x@1 \, \dots \, (C \, y@1 \, \dots \, y@k)\, \dots \, x@n = r \]
such that $C$ is a constructor of $t$ and all recursive calls of
$f$ in $r$ are of the form $f \, \dots \, y@i \, \dots$ for some $i$. Thus
Isabelle immediately sees that $f$ terminates because one (fixed!) argument
becomes smaller with every recursive call. There must be at most one equation
for each constructor.  Their order is immaterial.
A more general method for defining total recursive functions is introduced in
{\S}\ref{sec:fun}.

\begin{exercise}\label{ex:Tree}
%
\begin{isabellebody}%
\def\isabellecontext{Tree}%
%
\isadelimtheory
%
\endisadelimtheory
%
\isatagtheory
%
\endisatagtheory
{\isafoldtheory}%
%
\isadelimtheory
%
\endisadelimtheory
\isamarkuptrue%
%
\begin{isamarkuptext}%
\noindent
Define the datatype of \rmindex{binary trees}:%
\end{isamarkuptext}%
\isamarkupfalse%
\isacommand{datatype}\ {\isacharprime}a\ tree\ {\isacharequal}\ Tip\ {\isacharbar}\ Node\ {\isachardoublequote}{\isacharprime}a\ tree{\isachardoublequote}\ {\isacharprime}a\ {\isachardoublequote}{\isacharprime}a\ tree{\isachardoublequote}\isamarkuptrue%
%
\begin{isamarkuptext}%
\noindent
Define a function \isa{mirror} that mirrors a binary tree
by swapping subtrees recursively. Prove%
\end{isamarkuptext}%
\isamarkupfalse%
\isacommand{lemma}\ mirror{\isacharunderscore}mirror{\isacharcolon}\ {\isachardoublequote}mirror{\isacharparenleft}mirror\ t{\isacharparenright}\ {\isacharequal}\ t{\isachardoublequote}%
\isadelimproof
%
\endisadelimproof
%
\isatagproof
%
\endisatagproof
{\isafoldproof}%
%
\isadelimproof
%
\endisadelimproof
\isamarkuptrue%
%
\begin{isamarkuptext}%
\noindent
Define a function \isa{flatten} that flattens a tree into a list
by traversing it in infix order. Prove%
\end{isamarkuptext}%
\isamarkupfalse%
\isacommand{lemma}\ {\isachardoublequote}flatten{\isacharparenleft}mirror\ t{\isacharparenright}\ {\isacharequal}\ rev{\isacharparenleft}flatten\ t{\isacharparenright}{\isachardoublequote}%
\isadelimproof
%
\endisadelimproof
%
\isatagproof
%
\endisatagproof
{\isafoldproof}%
%
\isadelimproof
%
\endisadelimproof
%
\isadelimtheory
%
\endisadelimtheory
%
\isatagtheory
%
\endisatagtheory
{\isafoldtheory}%
%
\isadelimtheory
%
\endisadelimtheory
\end{isabellebody}%
%%% Local Variables:
%%% mode: latex
%%% TeX-master: "root"
%%% End:
%
\end{exercise}

%
\begin{isabellebody}%
\def\isabellecontext{case{\isaliteral{5F}{\isacharunderscore}}exprs}%
%
\isadelimtheory
%
\endisadelimtheory
%
\isatagtheory
%
\endisatagtheory
{\isafoldtheory}%
%
\isadelimtheory
%
\endisadelimtheory
%
\begin{isamarkuptext}%
\subsection{Case Expressions}
\label{sec:case-expressions}\index{*case expressions}%
HOL also features \isa{case}-expressions for analyzing
elements of a datatype. For example,
\begin{isabelle}%
\ \ \ \ \ case\ xs\ of\ {\isaliteral{5B}{\isacharbrackleft}}{\isaliteral{5D}{\isacharbrackright}}\ {\isaliteral{5C3C52696768746172726F773E}{\isasymRightarrow}}\ {\isaliteral{5B}{\isacharbrackleft}}{\isaliteral{5D}{\isacharbrackright}}\ {\isaliteral{7C}{\isacharbar}}\ y\ {\isaliteral{23}{\isacharhash}}\ ys\ {\isaliteral{5C3C52696768746172726F773E}{\isasymRightarrow}}\ y%
\end{isabelle}
evaluates to \isa{{\isaliteral{5B}{\isacharbrackleft}}{\isaliteral{5D}{\isacharbrackright}}} if \isa{xs} is \isa{{\isaliteral{5B}{\isacharbrackleft}}{\isaliteral{5D}{\isacharbrackright}}} and to \isa{y} if 
\isa{xs} is \isa{y\ {\isaliteral{23}{\isacharhash}}\ ys}. (Since the result in both branches must be of
the same type, it follows that \isa{y} is of type \isa{{\isaliteral{27}{\isacharprime}}a\ list} and hence
that \isa{xs} is of type \isa{{\isaliteral{27}{\isacharprime}}a\ list\ list}.)

In general, case expressions are of the form
\[
\begin{array}{c}
\isa{case}~e~\isa{of}\ pattern@1~\isa{{\isaliteral{5C3C52696768746172726F773E}{\isasymRightarrow}}}~e@1\ \isa{{\isaliteral{7C}{\isacharbar}}}\ \dots\
 \isa{{\isaliteral{7C}{\isacharbar}}}~pattern@m~\isa{{\isaliteral{5C3C52696768746172726F773E}{\isasymRightarrow}}}~e@m
\end{array}
\]
Like in functional programming, patterns are expressions consisting of
datatype constructors (e.g. \isa{{\isaliteral{5B}{\isacharbrackleft}}{\isaliteral{5D}{\isacharbrackright}}} and \isa{{\isaliteral{23}{\isacharhash}}})
and variables, including the wildcard ``\verb$_$''.
Not all cases need to be covered and the order of cases matters.
However, one is well-advised not to wallow in complex patterns because
complex case distinctions tend to induce complex proofs.

\begin{warn}
Internally Isabelle only knows about exhaustive case expressions with
non-nested patterns: $pattern@i$ must be of the form
$C@i~x@ {i1}~\dots~x@ {ik@i}$ and $C@1, \dots, C@m$ must be exactly the
constructors of the type of $e$.
%
More complex case expressions are automatically
translated into the simpler form upon parsing but are not translated
back for printing. This may lead to surprising output.
\end{warn}

\begin{warn}
Like \isa{if}, \isa{case}-expressions may need to be enclosed in
parentheses to indicate their scope.
\end{warn}

\subsection{Structural Induction and Case Distinction}
\label{sec:struct-ind-case}
\index{case distinctions}\index{induction!structural}%
Induction is invoked by \methdx{induct_tac}, as we have seen above; 
it works for any datatype.  In some cases, induction is overkill and a case
distinction over all constructors of the datatype suffices.  This is performed
by \methdx{case_tac}.  Here is a trivial example:%
\end{isamarkuptext}%
\isamarkuptrue%
\isacommand{lemma}\isamarkupfalse%
\ {\isaliteral{22}{\isachardoublequoteopen}}{\isaliteral{28}{\isacharparenleft}}case\ xs\ of\ {\isaliteral{5B}{\isacharbrackleft}}{\isaliteral{5D}{\isacharbrackright}}\ {\isaliteral{5C3C52696768746172726F773E}{\isasymRightarrow}}\ {\isaliteral{5B}{\isacharbrackleft}}{\isaliteral{5D}{\isacharbrackright}}\ {\isaliteral{7C}{\isacharbar}}\ y{\isaliteral{23}{\isacharhash}}ys\ {\isaliteral{5C3C52696768746172726F773E}{\isasymRightarrow}}\ xs{\isaliteral{29}{\isacharparenright}}\ {\isaliteral{3D}{\isacharequal}}\ xs{\isaliteral{22}{\isachardoublequoteclose}}\isanewline
%
\isadelimproof
%
\endisadelimproof
%
\isatagproof
\isacommand{apply}\isamarkupfalse%
{\isaliteral{28}{\isacharparenleft}}case{\isaliteral{5F}{\isacharunderscore}}tac\ xs{\isaliteral{29}{\isacharparenright}}%
\begin{isamarkuptxt}%
\noindent
results in the proof state
\begin{isabelle}%
\ {\isadigit{1}}{\isaliteral{2E}{\isachardot}}\ xs\ {\isaliteral{3D}{\isacharequal}}\ {\isaliteral{5B}{\isacharbrackleft}}{\isaliteral{5D}{\isacharbrackright}}\ {\isaliteral{5C3C4C6F6E6772696768746172726F773E}{\isasymLongrightarrow}}\ {\isaliteral{28}{\isacharparenleft}}case\ xs\ of\ {\isaliteral{5B}{\isacharbrackleft}}{\isaliteral{5D}{\isacharbrackright}}\ {\isaliteral{5C3C52696768746172726F773E}{\isasymRightarrow}}\ {\isaliteral{5B}{\isacharbrackleft}}{\isaliteral{5D}{\isacharbrackright}}\ {\isaliteral{7C}{\isacharbar}}\ y\ {\isaliteral{23}{\isacharhash}}\ ys\ {\isaliteral{5C3C52696768746172726F773E}{\isasymRightarrow}}\ xs{\isaliteral{29}{\isacharparenright}}\ {\isaliteral{3D}{\isacharequal}}\ xs\isanewline
\ {\isadigit{2}}{\isaliteral{2E}{\isachardot}}\ {\isaliteral{5C3C416E643E}{\isasymAnd}}a\ list{\isaliteral{2E}{\isachardot}}\isanewline
\isaindent{\ {\isadigit{2}}{\isaliteral{2E}{\isachardot}}\ \ \ \ }xs\ {\isaliteral{3D}{\isacharequal}}\ a\ {\isaliteral{23}{\isacharhash}}\ list\ {\isaliteral{5C3C4C6F6E6772696768746172726F773E}{\isasymLongrightarrow}}\ {\isaliteral{28}{\isacharparenleft}}case\ xs\ of\ {\isaliteral{5B}{\isacharbrackleft}}{\isaliteral{5D}{\isacharbrackright}}\ {\isaliteral{5C3C52696768746172726F773E}{\isasymRightarrow}}\ {\isaliteral{5B}{\isacharbrackleft}}{\isaliteral{5D}{\isacharbrackright}}\ {\isaliteral{7C}{\isacharbar}}\ y\ {\isaliteral{23}{\isacharhash}}\ ys\ {\isaliteral{5C3C52696768746172726F773E}{\isasymRightarrow}}\ xs{\isaliteral{29}{\isacharparenright}}\ {\isaliteral{3D}{\isacharequal}}\ xs%
\end{isabelle}
which is solved automatically:%
\end{isamarkuptxt}%
\isamarkuptrue%
\isacommand{apply}\isamarkupfalse%
{\isaliteral{28}{\isacharparenleft}}auto{\isaliteral{29}{\isacharparenright}}%
\endisatagproof
{\isafoldproof}%
%
\isadelimproof
%
\endisadelimproof
%
\begin{isamarkuptext}%
Note that we do not need to give a lemma a name if we do not intend to refer
to it explicitly in the future.
Other basic laws about a datatype are applied automatically during
simplification, so no special methods are provided for them.

\begin{warn}
  Induction is only allowed on free (or \isasymAnd-bound) variables that
  should not occur among the assumptions of the subgoal; see
  \S\ref{sec:ind-var-in-prems} for details. Case distinction
  (\isa{case{\isaliteral{5F}{\isacharunderscore}}tac}) works for arbitrary terms, which need to be
  quoted if they are non-atomic. However, apart from \isa{{\isaliteral{5C3C416E643E}{\isasymAnd}}}-bound
  variables, the terms must not contain variables that are bound outside.
  For example, given the goal \isa{{\isaliteral{5C3C666F72616C6C3E}{\isasymforall}}xs{\isaliteral{2E}{\isachardot}}\ xs\ {\isaliteral{3D}{\isacharequal}}\ {\isaliteral{5B}{\isacharbrackleft}}{\isaliteral{5D}{\isacharbrackright}}\ {\isaliteral{5C3C6F723E}{\isasymor}}\ {\isaliteral{28}{\isacharparenleft}}{\isaliteral{5C3C6578697374733E}{\isasymexists}}y\ ys{\isaliteral{2E}{\isachardot}}\ xs\ {\isaliteral{3D}{\isacharequal}}\ y\ {\isaliteral{23}{\isacharhash}}\ ys{\isaliteral{29}{\isacharparenright}}},
  \isa{case{\isaliteral{5F}{\isacharunderscore}}tac\ xs} will not work as expected because Isabelle interprets
  the \isa{xs} as a new free variable distinct from the bound
  \isa{xs} in the goal.
\end{warn}%
\end{isamarkuptext}%
\isamarkuptrue%
%
\isadelimtheory
%
\endisadelimtheory
%
\isatagtheory
%
\endisatagtheory
{\isafoldtheory}%
%
\isadelimtheory
%
\endisadelimtheory
\end{isabellebody}%
%%% Local Variables:
%%% mode: latex
%%% TeX-master: "root"
%%% End:


%
\begin{isabellebody}%
\def\isabellecontext{Ifexpr}%
%
\isadelimtheory
%
\endisadelimtheory
%
\isatagtheory
%
\endisatagtheory
{\isafoldtheory}%
%
\isadelimtheory
%
\endisadelimtheory
%
\isamarkupsubsection{Case Study: Boolean Expressions%
}
\isamarkuptrue%
%
\begin{isamarkuptext}%
\label{sec:boolex}\index{boolean expressions example|(}
The aim of this case study is twofold: it shows how to model boolean
expressions and some algorithms for manipulating them, and it demonstrates
the constructs introduced above.%
\end{isamarkuptext}%
\isamarkuptrue%
%
\isamarkupsubsubsection{Modelling Boolean Expressions%
}
\isamarkuptrue%
%
\begin{isamarkuptext}%
We want to represent boolean expressions built up from variables and
constants by negation and conjunction. The following datatype serves exactly
that purpose:%
\end{isamarkuptext}%
\isamarkuptrue%
\isacommand{datatype}\isamarkupfalse%
\ boolex\ {\isaliteral{3D}{\isacharequal}}\ Const\ bool\ {\isaliteral{7C}{\isacharbar}}\ Var\ nat\ {\isaliteral{7C}{\isacharbar}}\ Neg\ boolex\isanewline
\ \ \ \ \ \ \ \ \ \ \ \ \ \ \ \ {\isaliteral{7C}{\isacharbar}}\ And\ boolex\ boolex%
\begin{isamarkuptext}%
\noindent
The two constants are represented by \isa{Const\ True} and
\isa{Const\ False}. Variables are represented by terms of the form
\isa{Var\ n}, where \isa{n} is a natural number (type \isa{nat}).
For example, the formula $P@0 \land \neg P@1$ is represented by the term
\isa{And\ {\isaliteral{28}{\isacharparenleft}}Var\ {\isadigit{0}}{\isaliteral{29}{\isacharparenright}}\ {\isaliteral{28}{\isacharparenleft}}Neg\ {\isaliteral{28}{\isacharparenleft}}Var\ {\isadigit{1}}{\isaliteral{29}{\isacharparenright}}{\isaliteral{29}{\isacharparenright}}}.

\subsubsection{The Value of a Boolean Expression}

The value of a boolean expression depends on the value of its variables.
Hence the function \isa{value} takes an additional parameter, an
\emph{environment} of type \isa{nat\ {\isaliteral{5C3C52696768746172726F773E}{\isasymRightarrow}}\ bool}, which maps variables to their
values:%
\end{isamarkuptext}%
\isamarkuptrue%
\isacommand{primrec}\isamarkupfalse%
\ {\isaliteral{22}{\isachardoublequoteopen}}value{\isaliteral{22}{\isachardoublequoteclose}}\ {\isaliteral{3A}{\isacharcolon}}{\isaliteral{3A}{\isacharcolon}}\ {\isaliteral{22}{\isachardoublequoteopen}}boolex\ {\isaliteral{5C3C52696768746172726F773E}{\isasymRightarrow}}\ {\isaliteral{28}{\isacharparenleft}}nat\ {\isaliteral{5C3C52696768746172726F773E}{\isasymRightarrow}}\ bool{\isaliteral{29}{\isacharparenright}}\ {\isaliteral{5C3C52696768746172726F773E}{\isasymRightarrow}}\ bool{\isaliteral{22}{\isachardoublequoteclose}}\ \isakeyword{where}\isanewline
{\isaliteral{22}{\isachardoublequoteopen}}value\ {\isaliteral{28}{\isacharparenleft}}Const\ b{\isaliteral{29}{\isacharparenright}}\ env\ {\isaliteral{3D}{\isacharequal}}\ b{\isaliteral{22}{\isachardoublequoteclose}}\ {\isaliteral{7C}{\isacharbar}}\isanewline
{\isaliteral{22}{\isachardoublequoteopen}}value\ {\isaliteral{28}{\isacharparenleft}}Var\ x{\isaliteral{29}{\isacharparenright}}\ \ \ env\ {\isaliteral{3D}{\isacharequal}}\ env\ x{\isaliteral{22}{\isachardoublequoteclose}}\ {\isaliteral{7C}{\isacharbar}}\isanewline
{\isaliteral{22}{\isachardoublequoteopen}}value\ {\isaliteral{28}{\isacharparenleft}}Neg\ b{\isaliteral{29}{\isacharparenright}}\ \ \ env\ {\isaliteral{3D}{\isacharequal}}\ {\isaliteral{28}{\isacharparenleft}}{\isaliteral{5C3C6E6F743E}{\isasymnot}}\ value\ b\ env{\isaliteral{29}{\isacharparenright}}{\isaliteral{22}{\isachardoublequoteclose}}\ {\isaliteral{7C}{\isacharbar}}\isanewline
{\isaliteral{22}{\isachardoublequoteopen}}value\ {\isaliteral{28}{\isacharparenleft}}And\ b\ c{\isaliteral{29}{\isacharparenright}}\ env\ {\isaliteral{3D}{\isacharequal}}\ {\isaliteral{28}{\isacharparenleft}}value\ b\ env\ {\isaliteral{5C3C616E643E}{\isasymand}}\ value\ c\ env{\isaliteral{29}{\isacharparenright}}{\isaliteral{22}{\isachardoublequoteclose}}%
\begin{isamarkuptext}%
\noindent
\subsubsection{If-Expressions}

An alternative and often more efficient (because in a certain sense
canonical) representation are so-called \emph{If-expressions} built up
from constants (\isa{CIF}), variables (\isa{VIF}) and conditionals
(\isa{IF}):%
\end{isamarkuptext}%
\isamarkuptrue%
\isacommand{datatype}\isamarkupfalse%
\ ifex\ {\isaliteral{3D}{\isacharequal}}\ CIF\ bool\ {\isaliteral{7C}{\isacharbar}}\ VIF\ nat\ {\isaliteral{7C}{\isacharbar}}\ IF\ ifex\ ifex\ ifex%
\begin{isamarkuptext}%
\noindent
The evaluation of If-expressions proceeds as for \isa{boolex}:%
\end{isamarkuptext}%
\isamarkuptrue%
\isacommand{primrec}\isamarkupfalse%
\ valif\ {\isaliteral{3A}{\isacharcolon}}{\isaliteral{3A}{\isacharcolon}}\ {\isaliteral{22}{\isachardoublequoteopen}}ifex\ {\isaliteral{5C3C52696768746172726F773E}{\isasymRightarrow}}\ {\isaliteral{28}{\isacharparenleft}}nat\ {\isaliteral{5C3C52696768746172726F773E}{\isasymRightarrow}}\ bool{\isaliteral{29}{\isacharparenright}}\ {\isaliteral{5C3C52696768746172726F773E}{\isasymRightarrow}}\ bool{\isaliteral{22}{\isachardoublequoteclose}}\ \isakeyword{where}\isanewline
{\isaliteral{22}{\isachardoublequoteopen}}valif\ {\isaliteral{28}{\isacharparenleft}}CIF\ b{\isaliteral{29}{\isacharparenright}}\ \ \ \ env\ {\isaliteral{3D}{\isacharequal}}\ b{\isaliteral{22}{\isachardoublequoteclose}}\ {\isaliteral{7C}{\isacharbar}}\isanewline
{\isaliteral{22}{\isachardoublequoteopen}}valif\ {\isaliteral{28}{\isacharparenleft}}VIF\ x{\isaliteral{29}{\isacharparenright}}\ \ \ \ env\ {\isaliteral{3D}{\isacharequal}}\ env\ x{\isaliteral{22}{\isachardoublequoteclose}}\ {\isaliteral{7C}{\isacharbar}}\isanewline
{\isaliteral{22}{\isachardoublequoteopen}}valif\ {\isaliteral{28}{\isacharparenleft}}IF\ b\ t\ e{\isaliteral{29}{\isacharparenright}}\ env\ {\isaliteral{3D}{\isacharequal}}\ {\isaliteral{28}{\isacharparenleft}}if\ valif\ b\ env\ then\ valif\ t\ env\isanewline
\ \ \ \ \ \ \ \ \ \ \ \ \ \ \ \ \ \ \ \ \ \ \ \ \ \ \ \ \ \ \ \ \ \ \ \ \ \ \ \ else\ valif\ e\ env{\isaliteral{29}{\isacharparenright}}{\isaliteral{22}{\isachardoublequoteclose}}%
\begin{isamarkuptext}%
\subsubsection{Converting Boolean and If-Expressions}

The type \isa{boolex} is close to the customary representation of logical
formulae, whereas \isa{ifex} is designed for efficiency. It is easy to
translate from \isa{boolex} into \isa{ifex}:%
\end{isamarkuptext}%
\isamarkuptrue%
\isacommand{primrec}\isamarkupfalse%
\ bool{\isadigit{2}}if\ {\isaliteral{3A}{\isacharcolon}}{\isaliteral{3A}{\isacharcolon}}\ {\isaliteral{22}{\isachardoublequoteopen}}boolex\ {\isaliteral{5C3C52696768746172726F773E}{\isasymRightarrow}}\ ifex{\isaliteral{22}{\isachardoublequoteclose}}\ \isakeyword{where}\isanewline
{\isaliteral{22}{\isachardoublequoteopen}}bool{\isadigit{2}}if\ {\isaliteral{28}{\isacharparenleft}}Const\ b{\isaliteral{29}{\isacharparenright}}\ {\isaliteral{3D}{\isacharequal}}\ CIF\ b{\isaliteral{22}{\isachardoublequoteclose}}\ {\isaliteral{7C}{\isacharbar}}\isanewline
{\isaliteral{22}{\isachardoublequoteopen}}bool{\isadigit{2}}if\ {\isaliteral{28}{\isacharparenleft}}Var\ x{\isaliteral{29}{\isacharparenright}}\ \ \ {\isaliteral{3D}{\isacharequal}}\ VIF\ x{\isaliteral{22}{\isachardoublequoteclose}}\ {\isaliteral{7C}{\isacharbar}}\isanewline
{\isaliteral{22}{\isachardoublequoteopen}}bool{\isadigit{2}}if\ {\isaliteral{28}{\isacharparenleft}}Neg\ b{\isaliteral{29}{\isacharparenright}}\ \ \ {\isaliteral{3D}{\isacharequal}}\ IF\ {\isaliteral{28}{\isacharparenleft}}bool{\isadigit{2}}if\ b{\isaliteral{29}{\isacharparenright}}\ {\isaliteral{28}{\isacharparenleft}}CIF\ False{\isaliteral{29}{\isacharparenright}}\ {\isaliteral{28}{\isacharparenleft}}CIF\ True{\isaliteral{29}{\isacharparenright}}{\isaliteral{22}{\isachardoublequoteclose}}\ {\isaliteral{7C}{\isacharbar}}\isanewline
{\isaliteral{22}{\isachardoublequoteopen}}bool{\isadigit{2}}if\ {\isaliteral{28}{\isacharparenleft}}And\ b\ c{\isaliteral{29}{\isacharparenright}}\ {\isaliteral{3D}{\isacharequal}}\ IF\ {\isaliteral{28}{\isacharparenleft}}bool{\isadigit{2}}if\ b{\isaliteral{29}{\isacharparenright}}\ {\isaliteral{28}{\isacharparenleft}}bool{\isadigit{2}}if\ c{\isaliteral{29}{\isacharparenright}}\ {\isaliteral{28}{\isacharparenleft}}CIF\ False{\isaliteral{29}{\isacharparenright}}{\isaliteral{22}{\isachardoublequoteclose}}%
\begin{isamarkuptext}%
\noindent
At last, we have something we can verify: that \isa{bool{\isadigit{2}}if} preserves the
value of its argument:%
\end{isamarkuptext}%
\isamarkuptrue%
\isacommand{lemma}\isamarkupfalse%
\ {\isaliteral{22}{\isachardoublequoteopen}}valif\ {\isaliteral{28}{\isacharparenleft}}bool{\isadigit{2}}if\ b{\isaliteral{29}{\isacharparenright}}\ env\ {\isaliteral{3D}{\isacharequal}}\ value\ b\ env{\isaliteral{22}{\isachardoublequoteclose}}%
\isadelimproof
%
\endisadelimproof
%
\isatagproof
%
\begin{isamarkuptxt}%
\noindent
The proof is canonical:%
\end{isamarkuptxt}%
\isamarkuptrue%
\isacommand{apply}\isamarkupfalse%
{\isaliteral{28}{\isacharparenleft}}induct{\isaliteral{5F}{\isacharunderscore}}tac\ b{\isaliteral{29}{\isacharparenright}}\isanewline
\isacommand{apply}\isamarkupfalse%
{\isaliteral{28}{\isacharparenleft}}auto{\isaliteral{29}{\isacharparenright}}\isanewline
\isacommand{done}\isamarkupfalse%
%
\endisatagproof
{\isafoldproof}%
%
\isadelimproof
%
\endisadelimproof
%
\begin{isamarkuptext}%
\noindent
In fact, all proofs in this case study look exactly like this. Hence we do
not show them below.

More interesting is the transformation of If-expressions into a normal form
where the first argument of \isa{IF} cannot be another \isa{IF} but
must be a constant or variable. Such a normal form can be computed by
repeatedly replacing a subterm of the form \isa{IF\ {\isaliteral{28}{\isacharparenleft}}IF\ b\ x\ y{\isaliteral{29}{\isacharparenright}}\ z\ u} by
\isa{IF\ b\ {\isaliteral{28}{\isacharparenleft}}IF\ x\ z\ u{\isaliteral{29}{\isacharparenright}}\ {\isaliteral{28}{\isacharparenleft}}IF\ y\ z\ u{\isaliteral{29}{\isacharparenright}}}, which has the same value. The following
primitive recursive functions perform this task:%
\end{isamarkuptext}%
\isamarkuptrue%
\isacommand{primrec}\isamarkupfalse%
\ normif\ {\isaliteral{3A}{\isacharcolon}}{\isaliteral{3A}{\isacharcolon}}\ {\isaliteral{22}{\isachardoublequoteopen}}ifex\ {\isaliteral{5C3C52696768746172726F773E}{\isasymRightarrow}}\ ifex\ {\isaliteral{5C3C52696768746172726F773E}{\isasymRightarrow}}\ ifex\ {\isaliteral{5C3C52696768746172726F773E}{\isasymRightarrow}}\ ifex{\isaliteral{22}{\isachardoublequoteclose}}\ \isakeyword{where}\isanewline
{\isaliteral{22}{\isachardoublequoteopen}}normif\ {\isaliteral{28}{\isacharparenleft}}CIF\ b{\isaliteral{29}{\isacharparenright}}\ \ \ \ t\ e\ {\isaliteral{3D}{\isacharequal}}\ IF\ {\isaliteral{28}{\isacharparenleft}}CIF\ b{\isaliteral{29}{\isacharparenright}}\ t\ e{\isaliteral{22}{\isachardoublequoteclose}}\ {\isaliteral{7C}{\isacharbar}}\isanewline
{\isaliteral{22}{\isachardoublequoteopen}}normif\ {\isaliteral{28}{\isacharparenleft}}VIF\ x{\isaliteral{29}{\isacharparenright}}\ \ \ \ t\ e\ {\isaliteral{3D}{\isacharequal}}\ IF\ {\isaliteral{28}{\isacharparenleft}}VIF\ x{\isaliteral{29}{\isacharparenright}}\ t\ e{\isaliteral{22}{\isachardoublequoteclose}}\ {\isaliteral{7C}{\isacharbar}}\isanewline
{\isaliteral{22}{\isachardoublequoteopen}}normif\ {\isaliteral{28}{\isacharparenleft}}IF\ b\ t\ e{\isaliteral{29}{\isacharparenright}}\ u\ f\ {\isaliteral{3D}{\isacharequal}}\ normif\ b\ {\isaliteral{28}{\isacharparenleft}}normif\ t\ u\ f{\isaliteral{29}{\isacharparenright}}\ {\isaliteral{28}{\isacharparenleft}}normif\ e\ u\ f{\isaliteral{29}{\isacharparenright}}{\isaliteral{22}{\isachardoublequoteclose}}\isanewline
\isanewline
\isacommand{primrec}\isamarkupfalse%
\ norm\ {\isaliteral{3A}{\isacharcolon}}{\isaliteral{3A}{\isacharcolon}}\ {\isaliteral{22}{\isachardoublequoteopen}}ifex\ {\isaliteral{5C3C52696768746172726F773E}{\isasymRightarrow}}\ ifex{\isaliteral{22}{\isachardoublequoteclose}}\ \isakeyword{where}\isanewline
{\isaliteral{22}{\isachardoublequoteopen}}norm\ {\isaliteral{28}{\isacharparenleft}}CIF\ b{\isaliteral{29}{\isacharparenright}}\ \ \ \ {\isaliteral{3D}{\isacharequal}}\ CIF\ b{\isaliteral{22}{\isachardoublequoteclose}}\ {\isaliteral{7C}{\isacharbar}}\isanewline
{\isaliteral{22}{\isachardoublequoteopen}}norm\ {\isaliteral{28}{\isacharparenleft}}VIF\ x{\isaliteral{29}{\isacharparenright}}\ \ \ \ {\isaliteral{3D}{\isacharequal}}\ VIF\ x{\isaliteral{22}{\isachardoublequoteclose}}\ {\isaliteral{7C}{\isacharbar}}\isanewline
{\isaliteral{22}{\isachardoublequoteopen}}norm\ {\isaliteral{28}{\isacharparenleft}}IF\ b\ t\ e{\isaliteral{29}{\isacharparenright}}\ {\isaliteral{3D}{\isacharequal}}\ normif\ b\ {\isaliteral{28}{\isacharparenleft}}norm\ t{\isaliteral{29}{\isacharparenright}}\ {\isaliteral{28}{\isacharparenleft}}norm\ e{\isaliteral{29}{\isacharparenright}}{\isaliteral{22}{\isachardoublequoteclose}}%
\begin{isamarkuptext}%
\noindent
Their interplay is tricky; we leave it to you to develop an
intuitive understanding. Fortunately, Isabelle can help us to verify that the
transformation preserves the value of the expression:%
\end{isamarkuptext}%
\isamarkuptrue%
\isacommand{theorem}\isamarkupfalse%
\ {\isaliteral{22}{\isachardoublequoteopen}}valif\ {\isaliteral{28}{\isacharparenleft}}norm\ b{\isaliteral{29}{\isacharparenright}}\ env\ {\isaliteral{3D}{\isacharequal}}\ valif\ b\ env{\isaliteral{22}{\isachardoublequoteclose}}%
\isadelimproof
%
\endisadelimproof
%
\isatagproof
%
\endisatagproof
{\isafoldproof}%
%
\isadelimproof
%
\endisadelimproof
%
\begin{isamarkuptext}%
\noindent
The proof is canonical, provided we first show the following simplification
lemma, which also helps to understand what \isa{normif} does:%
\end{isamarkuptext}%
\isamarkuptrue%
\isacommand{lemma}\isamarkupfalse%
\ {\isaliteral{5B}{\isacharbrackleft}}simp{\isaliteral{5D}{\isacharbrackright}}{\isaliteral{3A}{\isacharcolon}}\isanewline
\ \ {\isaliteral{22}{\isachardoublequoteopen}}{\isaliteral{5C3C666F72616C6C3E}{\isasymforall}}t\ e{\isaliteral{2E}{\isachardot}}\ valif\ {\isaliteral{28}{\isacharparenleft}}normif\ b\ t\ e{\isaliteral{29}{\isacharparenright}}\ env\ {\isaliteral{3D}{\isacharequal}}\ valif\ {\isaliteral{28}{\isacharparenleft}}IF\ b\ t\ e{\isaliteral{29}{\isacharparenright}}\ env{\isaliteral{22}{\isachardoublequoteclose}}%
\isadelimproof
%
\endisadelimproof
%
\isatagproof
%
\endisatagproof
{\isafoldproof}%
%
\isadelimproof
%
\endisadelimproof
%
\isadelimproof
%
\endisadelimproof
%
\isatagproof
%
\endisatagproof
{\isafoldproof}%
%
\isadelimproof
%
\endisadelimproof
%
\begin{isamarkuptext}%
\noindent
Note that the lemma does not have a name, but is implicitly used in the proof
of the theorem shown above because of the \isa{{\isaliteral{5B}{\isacharbrackleft}}simp{\isaliteral{5D}{\isacharbrackright}}} attribute.

But how can we be sure that \isa{norm} really produces a normal form in
the above sense? We define a function that tests If-expressions for normality:%
\end{isamarkuptext}%
\isamarkuptrue%
\isacommand{primrec}\isamarkupfalse%
\ normal\ {\isaliteral{3A}{\isacharcolon}}{\isaliteral{3A}{\isacharcolon}}\ {\isaliteral{22}{\isachardoublequoteopen}}ifex\ {\isaliteral{5C3C52696768746172726F773E}{\isasymRightarrow}}\ bool{\isaliteral{22}{\isachardoublequoteclose}}\ \isakeyword{where}\isanewline
{\isaliteral{22}{\isachardoublequoteopen}}normal{\isaliteral{28}{\isacharparenleft}}CIF\ b{\isaliteral{29}{\isacharparenright}}\ {\isaliteral{3D}{\isacharequal}}\ True{\isaliteral{22}{\isachardoublequoteclose}}\ {\isaliteral{7C}{\isacharbar}}\isanewline
{\isaliteral{22}{\isachardoublequoteopen}}normal{\isaliteral{28}{\isacharparenleft}}VIF\ x{\isaliteral{29}{\isacharparenright}}\ {\isaliteral{3D}{\isacharequal}}\ True{\isaliteral{22}{\isachardoublequoteclose}}\ {\isaliteral{7C}{\isacharbar}}\isanewline
{\isaliteral{22}{\isachardoublequoteopen}}normal{\isaliteral{28}{\isacharparenleft}}IF\ b\ t\ e{\isaliteral{29}{\isacharparenright}}\ {\isaliteral{3D}{\isacharequal}}\ {\isaliteral{28}{\isacharparenleft}}normal\ t\ {\isaliteral{5C3C616E643E}{\isasymand}}\ normal\ e\ {\isaliteral{5C3C616E643E}{\isasymand}}\isanewline
\ \ \ \ \ {\isaliteral{28}{\isacharparenleft}}case\ b\ of\ CIF\ b\ {\isaliteral{5C3C52696768746172726F773E}{\isasymRightarrow}}\ True\ {\isaliteral{7C}{\isacharbar}}\ VIF\ x\ {\isaliteral{5C3C52696768746172726F773E}{\isasymRightarrow}}\ True\ {\isaliteral{7C}{\isacharbar}}\ IF\ x\ y\ z\ {\isaliteral{5C3C52696768746172726F773E}{\isasymRightarrow}}\ False{\isaliteral{29}{\isacharparenright}}{\isaliteral{29}{\isacharparenright}}{\isaliteral{22}{\isachardoublequoteclose}}%
\begin{isamarkuptext}%
\noindent
Now we prove \isa{normal\ {\isaliteral{28}{\isacharparenleft}}norm\ b{\isaliteral{29}{\isacharparenright}}}. Of course, this requires a lemma about
normality of \isa{normif}:%
\end{isamarkuptext}%
\isamarkuptrue%
\isacommand{lemma}\isamarkupfalse%
\ {\isaliteral{5B}{\isacharbrackleft}}simp{\isaliteral{5D}{\isacharbrackright}}{\isaliteral{3A}{\isacharcolon}}\ {\isaliteral{22}{\isachardoublequoteopen}}{\isaliteral{5C3C666F72616C6C3E}{\isasymforall}}t\ e{\isaliteral{2E}{\isachardot}}\ normal{\isaliteral{28}{\isacharparenleft}}normif\ b\ t\ e{\isaliteral{29}{\isacharparenright}}\ {\isaliteral{3D}{\isacharequal}}\ {\isaliteral{28}{\isacharparenleft}}normal\ t\ {\isaliteral{5C3C616E643E}{\isasymand}}\ normal\ e{\isaliteral{29}{\isacharparenright}}{\isaliteral{22}{\isachardoublequoteclose}}%
\isadelimproof
%
\endisadelimproof
%
\isatagproof
%
\endisatagproof
{\isafoldproof}%
%
\isadelimproof
%
\endisadelimproof
%
\isadelimproof
%
\endisadelimproof
%
\isatagproof
%
\endisatagproof
{\isafoldproof}%
%
\isadelimproof
%
\endisadelimproof
%
\begin{isamarkuptext}%
\medskip
How do we come up with the required lemmas? Try to prove the main theorems
without them and study carefully what \isa{auto} leaves unproved. This 
can provide the clue.  The necessity of universal quantification
(\isa{{\isaliteral{5C3C666F72616C6C3E}{\isasymforall}}t\ e}) in the two lemmas is explained in
\S\ref{sec:InductionHeuristics}

\begin{exercise}
  We strengthen the definition of a \isa{normal} If-expression as follows:
  the first argument of all \isa{IF}s must be a variable. Adapt the above
  development to this changed requirement. (Hint: you may need to formulate
  some of the goals as implications (\isa{{\isaliteral{5C3C6C6F6E6772696768746172726F773E}{\isasymlongrightarrow}}}) rather than
  equalities (\isa{{\isaliteral{3D}{\isacharequal}}}).)
\end{exercise}
\index{boolean expressions example|)}%
\end{isamarkuptext}%
\isamarkuptrue%
%
\isadelimproof
%
\endisadelimproof
%
\isatagproof
%
\endisatagproof
{\isafoldproof}%
%
\isadelimproof
%
\endisadelimproof
%
\isadelimproof
%
\endisadelimproof
%
\isatagproof
%
\endisatagproof
{\isafoldproof}%
%
\isadelimproof
%
\endisadelimproof
%
\isadelimproof
%
\endisadelimproof
%
\isatagproof
%
\endisatagproof
{\isafoldproof}%
%
\isadelimproof
%
\endisadelimproof
%
\isadelimproof
%
\endisadelimproof
%
\isatagproof
%
\endisatagproof
{\isafoldproof}%
%
\isadelimproof
%
\endisadelimproof
%
\isadelimtheory
%
\endisadelimtheory
%
\isatagtheory
%
\endisatagtheory
{\isafoldtheory}%
%
\isadelimtheory
%
\endisadelimtheory
\end{isabellebody}%
%%% Local Variables:
%%% mode: latex
%%% TeX-master: "root"
%%% End:

\index{datatypes|)}


\section{Some Basic Types}

This section introduces the types of natural numbers and ordered pairs.  Also
described is type \isa{option}, which is useful for modelling exceptional
cases. 

\subsection{Natural Numbers}
\label{sec:nat}\index{natural numbers}%
\index{linear arithmetic|(}

\begin{isabelle}%
%
\begin{isamarkuptext}%
\noindent
The type \isaindexbold{nat}\index{*0|bold}\index{*Suc|bold} of natural
numbers is predefined and behaves like%
\end{isamarkuptext}%
\isacommand{datatype}\ nat\ {\isacharequal}\ \isadigit{0}\ {\isacharbar}\ Suc\ nat\end{isabelle}%
%%% Local Variables:
%%% mode: latex
%%% TeX-master: "root"
%%% End:
\medskip
%
\begin{isabellebody}%
\def\isabellecontext{natsum}%
%
\isadelimtheory
%
\endisadelimtheory
%
\isatagtheory
%
\endisatagtheory
{\isafoldtheory}%
%
\isadelimtheory
%
\endisadelimtheory
%
\begin{isamarkuptext}%
\noindent
In particular, there are \isa{case}-expressions, for example
\begin{isabelle}%
\ \ \ \ \ case\ n\ of\ {\isadigit{0}}\ {\isaliteral{5C3C52696768746172726F773E}{\isasymRightarrow}}\ {\isadigit{0}}\ {\isaliteral{7C}{\isacharbar}}\ Suc\ m\ {\isaliteral{5C3C52696768746172726F773E}{\isasymRightarrow}}\ m%
\end{isabelle}
primitive recursion, for example%
\end{isamarkuptext}%
\isamarkuptrue%
\isacommand{primrec}\isamarkupfalse%
\ sum\ {\isaliteral{3A}{\isacharcolon}}{\isaliteral{3A}{\isacharcolon}}\ {\isaliteral{22}{\isachardoublequoteopen}}nat\ {\isaliteral{5C3C52696768746172726F773E}{\isasymRightarrow}}\ nat{\isaliteral{22}{\isachardoublequoteclose}}\ \isakeyword{where}\isanewline
{\isaliteral{22}{\isachardoublequoteopen}}sum\ {\isadigit{0}}\ {\isaliteral{3D}{\isacharequal}}\ {\isadigit{0}}{\isaliteral{22}{\isachardoublequoteclose}}\ {\isaliteral{7C}{\isacharbar}}\isanewline
{\isaliteral{22}{\isachardoublequoteopen}}sum\ {\isaliteral{28}{\isacharparenleft}}Suc\ n{\isaliteral{29}{\isacharparenright}}\ {\isaliteral{3D}{\isacharequal}}\ Suc\ n\ {\isaliteral{2B}{\isacharplus}}\ sum\ n{\isaliteral{22}{\isachardoublequoteclose}}%
\begin{isamarkuptext}%
\noindent
and induction, for example%
\end{isamarkuptext}%
\isamarkuptrue%
\isacommand{lemma}\isamarkupfalse%
\ {\isaliteral{22}{\isachardoublequoteopen}}sum\ n\ {\isaliteral{2B}{\isacharplus}}\ sum\ n\ {\isaliteral{3D}{\isacharequal}}\ n{\isaliteral{2A}{\isacharasterisk}}{\isaliteral{28}{\isacharparenleft}}Suc\ n{\isaliteral{29}{\isacharparenright}}{\isaliteral{22}{\isachardoublequoteclose}}\isanewline
%
\isadelimproof
%
\endisadelimproof
%
\isatagproof
\isacommand{apply}\isamarkupfalse%
{\isaliteral{28}{\isacharparenleft}}induct{\isaliteral{5F}{\isacharunderscore}}tac\ n{\isaliteral{29}{\isacharparenright}}\isanewline
\isacommand{apply}\isamarkupfalse%
{\isaliteral{28}{\isacharparenleft}}auto{\isaliteral{29}{\isacharparenright}}\isanewline
\isacommand{done}\isamarkupfalse%
%
\endisatagproof
{\isafoldproof}%
%
\isadelimproof
%
\endisadelimproof
%
\begin{isamarkuptext}%
\newcommand{\mystar}{*%
}
\index{arithmetic operations!for \protect\isa{nat}}%
The arithmetic operations \isadxboldpos{+}{$HOL2arithfun},
\isadxboldpos{-}{$HOL2arithfun}, \isadxboldpos{\mystar}{$HOL2arithfun},
\sdx{div}, \sdx{mod}, \cdx{min} and
\cdx{max} are predefined, as are the relations
\isadxboldpos{\isasymle}{$HOL2arithrel} and
\isadxboldpos{<}{$HOL2arithrel}. As usual, \isa{m\ {\isaliteral{2D}{\isacharminus}}\ n\ {\isaliteral{3D}{\isacharequal}}\ {\isadigit{0}}} if
\isa{m\ {\isaliteral{3C}{\isacharless}}\ n}. There is even a least number operation
\sdx{LEAST}\@.  For example, \isa{{\isaliteral{28}{\isacharparenleft}}LEAST\ n{\isaliteral{2E}{\isachardot}}\ {\isadigit{0}}\ {\isaliteral{3C}{\isacharless}}\ n{\isaliteral{29}{\isacharparenright}}\ {\isaliteral{3D}{\isacharequal}}\ Suc\ {\isadigit{0}}}.
\begin{warn}\index{overloading}
  The constants \cdx{0} and \cdx{1} and the operations
  \isadxboldpos{+}{$HOL2arithfun}, \isadxboldpos{-}{$HOL2arithfun},
  \isadxboldpos{\mystar}{$HOL2arithfun}, \cdx{min},
  \cdx{max}, \isadxboldpos{\isasymle}{$HOL2arithrel} and
  \isadxboldpos{<}{$HOL2arithrel} are overloaded: they are available
  not just for natural numbers but for other types as well.
  For example, given the goal \isa{x\ {\isaliteral{2B}{\isacharplus}}\ {\isadigit{0}}\ {\isaliteral{3D}{\isacharequal}}\ x}, there is nothing to indicate
  that you are talking about natural numbers. Hence Isabelle can only infer
  that \isa{x} is of some arbitrary type where \isa{{\isadigit{0}}} and \isa{{\isaliteral{2B}{\isacharplus}}} are
  declared. As a consequence, you will be unable to prove the
  goal. To alert you to such pitfalls, Isabelle flags numerals without a
  fixed type in its output: \isa{x\ {\isaliteral{2B}{\isacharplus}}\ {\isaliteral{28}{\isacharparenleft}}{\isadigit{0}}{\isaliteral{5C3C436F6C6F6E3E}{\isasymColon}}{\isaliteral{27}{\isacharprime}}a{\isaliteral{29}{\isacharparenright}}\ {\isaliteral{3D}{\isacharequal}}\ x}. (In the absence of a numeral,
  it may take you some time to realize what has happened if \pgmenu{Show
  Types} is not set).  In this particular example, you need to include
  an explicit type constraint, for example \isa{x{\isaliteral{2B}{\isacharplus}}{\isadigit{0}}\ {\isaliteral{3D}{\isacharequal}}\ {\isaliteral{28}{\isacharparenleft}}x{\isaliteral{3A}{\isacharcolon}}{\isaliteral{3A}{\isacharcolon}}nat{\isaliteral{29}{\isacharparenright}}}. If there
  is enough contextual information this may not be necessary: \isa{Suc\ x\ {\isaliteral{3D}{\isacharequal}}\ x} automatically implies \isa{x{\isaliteral{3A}{\isacharcolon}}{\isaliteral{3A}{\isacharcolon}}nat} because \isa{Suc} is not
  overloaded.

  For details on overloading see \S\ref{sec:overloading}.
  Table~\ref{tab:overloading} in the appendix shows the most important
  overloaded operations.
\end{warn}
\begin{warn}
  The symbols \isadxboldpos{>}{$HOL2arithrel} and
  \isadxboldpos{\isasymge}{$HOL2arithrel} are merely syntax: \isa{x\ {\isaliteral{3E}{\isachargreater}}\ y}
  stands for \isa{y\ {\isaliteral{3C}{\isacharless}}\ x} and similary for \isa{{\isaliteral{5C3C67653E}{\isasymge}}} and
  \isa{{\isaliteral{5C3C6C653E}{\isasymle}}}.
\end{warn}
\begin{warn}
  Constant \isa{{\isadigit{1}}{\isaliteral{3A}{\isacharcolon}}{\isaliteral{3A}{\isacharcolon}}nat} is defined to equal \isa{Suc\ {\isadigit{0}}}. This definition
  (see \S\ref{sec:ConstDefinitions}) is unfolded automatically by some
  tactics (like \isa{auto}, \isa{simp} and \isa{arith}) but not by
  others (especially the single step tactics in Chapter~\ref{chap:rules}).
  If you need the full set of numerals, see~\S\ref{sec:numerals}.
  \emph{Novices are advised to stick to \isa{{\isadigit{0}}} and \isa{Suc}.}
\end{warn}

Both \isa{auto} and \isa{simp}
(a method introduced below, \S\ref{sec:Simplification}) prove 
simple arithmetic goals automatically:%
\end{isamarkuptext}%
\isamarkuptrue%
\isacommand{lemma}\isamarkupfalse%
\ {\isaliteral{22}{\isachardoublequoteopen}}{\isaliteral{5C3C6C6272616B6B3E}{\isasymlbrakk}}\ {\isaliteral{5C3C6E6F743E}{\isasymnot}}\ m\ {\isaliteral{3C}{\isacharless}}\ n{\isaliteral{3B}{\isacharsemicolon}}\ m\ {\isaliteral{3C}{\isacharless}}\ n\ {\isaliteral{2B}{\isacharplus}}\ {\isaliteral{28}{\isacharparenleft}}{\isadigit{1}}{\isaliteral{3A}{\isacharcolon}}{\isaliteral{3A}{\isacharcolon}}nat{\isaliteral{29}{\isacharparenright}}\ {\isaliteral{5C3C726272616B6B3E}{\isasymrbrakk}}\ {\isaliteral{5C3C4C6F6E6772696768746172726F773E}{\isasymLongrightarrow}}\ m\ {\isaliteral{3D}{\isacharequal}}\ n{\isaliteral{22}{\isachardoublequoteclose}}%
\isadelimproof
%
\endisadelimproof
%
\isatagproof
%
\endisatagproof
{\isafoldproof}%
%
\isadelimproof
%
\endisadelimproof
%
\begin{isamarkuptext}%
\noindent
For efficiency's sake, this built-in prover ignores quantified formulae,
many logical connectives, and all arithmetic operations apart from addition.
In consequence, \isa{auto} and \isa{simp} cannot prove this slightly more complex goal:%
\end{isamarkuptext}%
\isamarkuptrue%
\isacommand{lemma}\isamarkupfalse%
\ {\isaliteral{22}{\isachardoublequoteopen}}m\ {\isaliteral{5C3C6E6F7465713E}{\isasymnoteq}}\ {\isaliteral{28}{\isacharparenleft}}n{\isaliteral{3A}{\isacharcolon}}{\isaliteral{3A}{\isacharcolon}}nat{\isaliteral{29}{\isacharparenright}}\ {\isaliteral{5C3C4C6F6E6772696768746172726F773E}{\isasymLongrightarrow}}\ m\ {\isaliteral{3C}{\isacharless}}\ n\ {\isaliteral{5C3C6F723E}{\isasymor}}\ n\ {\isaliteral{3C}{\isacharless}}\ m{\isaliteral{22}{\isachardoublequoteclose}}%
\isadelimproof
%
\endisadelimproof
%
\isatagproof
%
\endisatagproof
{\isafoldproof}%
%
\isadelimproof
%
\endisadelimproof
%
\begin{isamarkuptext}%
\noindent The method \methdx{arith} is more general.  It attempts to
prove the first subgoal provided it is a \textbf{linear arithmetic} formula.
Such formulas may involve the usual logical connectives (\isa{{\isaliteral{5C3C6E6F743E}{\isasymnot}}},
\isa{{\isaliteral{5C3C616E643E}{\isasymand}}}, \isa{{\isaliteral{5C3C6F723E}{\isasymor}}}, \isa{{\isaliteral{5C3C6C6F6E6772696768746172726F773E}{\isasymlongrightarrow}}}, \isa{{\isaliteral{3D}{\isacharequal}}},
\isa{{\isaliteral{5C3C666F72616C6C3E}{\isasymforall}}}, \isa{{\isaliteral{5C3C6578697374733E}{\isasymexists}}}), the relations \isa{{\isaliteral{3D}{\isacharequal}}},
\isa{{\isaliteral{5C3C6C653E}{\isasymle}}} and \isa{{\isaliteral{3C}{\isacharless}}}, and the operations \isa{{\isaliteral{2B}{\isacharplus}}}, \isa{{\isaliteral{2D}{\isacharminus}}},
\isa{min} and \isa{max}.  For example,%
\end{isamarkuptext}%
\isamarkuptrue%
\isacommand{lemma}\isamarkupfalse%
\ {\isaliteral{22}{\isachardoublequoteopen}}min\ i\ {\isaliteral{28}{\isacharparenleft}}max\ j\ {\isaliteral{28}{\isacharparenleft}}k{\isaliteral{2A}{\isacharasterisk}}k{\isaliteral{29}{\isacharparenright}}{\isaliteral{29}{\isacharparenright}}\ {\isaliteral{3D}{\isacharequal}}\ max\ {\isaliteral{28}{\isacharparenleft}}min\ {\isaliteral{28}{\isacharparenleft}}k{\isaliteral{2A}{\isacharasterisk}}k{\isaliteral{29}{\isacharparenright}}\ i{\isaliteral{29}{\isacharparenright}}\ {\isaliteral{28}{\isacharparenleft}}min\ i\ {\isaliteral{28}{\isacharparenleft}}j{\isaliteral{3A}{\isacharcolon}}{\isaliteral{3A}{\isacharcolon}}nat{\isaliteral{29}{\isacharparenright}}{\isaliteral{29}{\isacharparenright}}{\isaliteral{22}{\isachardoublequoteclose}}\isanewline
%
\isadelimproof
%
\endisadelimproof
%
\isatagproof
\isacommand{apply}\isamarkupfalse%
{\isaliteral{28}{\isacharparenleft}}arith{\isaliteral{29}{\isacharparenright}}%
\endisatagproof
{\isafoldproof}%
%
\isadelimproof
%
\endisadelimproof
%
\begin{isamarkuptext}%
\noindent
succeeds because \isa{k\ {\isaliteral{2A}{\isacharasterisk}}\ k} can be treated as atomic. In contrast,%
\end{isamarkuptext}%
\isamarkuptrue%
\isacommand{lemma}\isamarkupfalse%
\ {\isaliteral{22}{\isachardoublequoteopen}}n{\isaliteral{2A}{\isacharasterisk}}n\ {\isaliteral{3D}{\isacharequal}}\ n{\isaliteral{2B}{\isacharplus}}{\isadigit{1}}\ {\isaliteral{5C3C4C6F6E6772696768746172726F773E}{\isasymLongrightarrow}}\ n{\isaliteral{3D}{\isacharequal}}{\isadigit{0}}{\isaliteral{22}{\isachardoublequoteclose}}%
\isadelimproof
%
\endisadelimproof
%
\isatagproof
%
\endisatagproof
{\isafoldproof}%
%
\isadelimproof
%
\endisadelimproof
%
\begin{isamarkuptext}%
\noindent
is not proved by \isa{arith} because the proof relies 
on properties of multiplication. Only multiplication by numerals (which is
the same as iterated addition) is taken into account.

\begin{warn} The running time of \isa{arith} is exponential in the number
  of occurrences of \ttindexboldpos{-}{$HOL2arithfun}, \cdx{min} and
  \cdx{max} because they are first eliminated by case distinctions.

If \isa{k} is a numeral, \sdx{div}~\isa{k}, \sdx{mod}~\isa{k} and
\isa{k}~\sdx{dvd} are also supported, where the former two are eliminated
by case distinctions, again blowing up the running time.

If the formula involves quantifiers, \isa{arith} may take
super-exponential time and space.
\end{warn}%
\end{isamarkuptext}%
\isamarkuptrue%
%
\isadelimtheory
%
\endisadelimtheory
%
\isatagtheory
%
\endisatagtheory
{\isafoldtheory}%
%
\isadelimtheory
%
\endisadelimtheory
\end{isabellebody}%
%%% Local Variables:
%%% mode: latex
%%% TeX-master: "root"
%%% End:


\index{linear arithmetic|)}


\subsection{Pairs}
\input{document/pairs2.tex}

\subsection{Datatype {\tt\slshape option}}
\label{sec:option}
\begin{isabelle}%
\isanewline
\isacommand{datatype}\ 'a\ option\ =\ None\ |\ Some\ 'a\end{isabelle}%
%%% Local Variables:
%%% mode: latex
%%% TeX-master: "root"
%%% End:


\section{Definitions}
\label{sec:Definitions}

A definition is simply an abbreviation, i.e.\ a new name for an existing
construction. In particular, definitions cannot be recursive. Isabelle offers
definitions on the level of types and terms. Those on the type level are
called \textbf{type synonyms}; those on the term level are simply called 
definitions.


\subsection{Type Synonyms}

\index{type synonyms}%
Type synonyms are similar to those found in ML\@. They are created by a 
\commdx{type\protect\_synonym} command:

\medskip
%
\begin{isabellebody}%
\def\isabellecontext{types}%
%
\isadelimtheory
%
\endisadelimtheory
%
\isatagtheory
%
\endisatagtheory
{\isafoldtheory}%
%
\isadelimtheory
%
\endisadelimtheory
\isacommand{type{\isaliteral{5F}{\isacharunderscore}}synonym}\isamarkupfalse%
\ number\ {\isaliteral{3D}{\isacharequal}}\ nat\isanewline
\isacommand{type{\isaliteral{5F}{\isacharunderscore}}synonym}\isamarkupfalse%
\ gate\ {\isaliteral{3D}{\isacharequal}}\ {\isaliteral{22}{\isachardoublequoteopen}}bool\ {\isaliteral{5C3C52696768746172726F773E}{\isasymRightarrow}}\ bool\ {\isaliteral{5C3C52696768746172726F773E}{\isasymRightarrow}}\ bool{\isaliteral{22}{\isachardoublequoteclose}}\isanewline
\isacommand{type{\isaliteral{5F}{\isacharunderscore}}synonym}\isamarkupfalse%
\ {\isaliteral{28}{\isacharparenleft}}{\isaliteral{27}{\isacharprime}}a{\isaliteral{2C}{\isacharcomma}}\ {\isaliteral{27}{\isacharprime}}b{\isaliteral{29}{\isacharparenright}}\ alist\ {\isaliteral{3D}{\isacharequal}}\ {\isaliteral{22}{\isachardoublequoteopen}}{\isaliteral{28}{\isacharparenleft}}{\isaliteral{27}{\isacharprime}}a\ {\isaliteral{5C3C74696D65733E}{\isasymtimes}}\ {\isaliteral{27}{\isacharprime}}b{\isaliteral{29}{\isacharparenright}}\ list{\isaliteral{22}{\isachardoublequoteclose}}%
\begin{isamarkuptext}%
\noindent
Internally all synonyms are fully expanded.  As a consequence Isabelle's
output never contains synonyms.  Their main purpose is to improve the
readability of theories.  Synonyms can be used just like any other
type.%
\end{isamarkuptext}%
\isamarkuptrue%
%
\isamarkupsubsection{Constant Definitions%
}
\isamarkuptrue%
%
\begin{isamarkuptext}%
\label{sec:ConstDefinitions}\indexbold{definitions}%
Nonrecursive definitions can be made with the \commdx{definition}
command, for example \isa{nand} and \isa{xor} gates
(based on type \isa{gate} above):%
\end{isamarkuptext}%
\isamarkuptrue%
\isacommand{definition}\isamarkupfalse%
\ nand\ {\isaliteral{3A}{\isacharcolon}}{\isaliteral{3A}{\isacharcolon}}\ gate\ \isakeyword{where}\ {\isaliteral{22}{\isachardoublequoteopen}}nand\ A\ B\ {\isaliteral{5C3C65717569763E}{\isasymequiv}}\ {\isaliteral{5C3C6E6F743E}{\isasymnot}}{\isaliteral{28}{\isacharparenleft}}A\ {\isaliteral{5C3C616E643E}{\isasymand}}\ B{\isaliteral{29}{\isacharparenright}}{\isaliteral{22}{\isachardoublequoteclose}}\isanewline
\isacommand{definition}\isamarkupfalse%
\ xor\ \ {\isaliteral{3A}{\isacharcolon}}{\isaliteral{3A}{\isacharcolon}}\ gate\ \isakeyword{where}\ {\isaliteral{22}{\isachardoublequoteopen}}xor\ \ A\ B\ {\isaliteral{5C3C65717569763E}{\isasymequiv}}\ A\ {\isaliteral{5C3C616E643E}{\isasymand}}\ {\isaliteral{5C3C6E6F743E}{\isasymnot}}B\ {\isaliteral{5C3C6F723E}{\isasymor}}\ {\isaliteral{5C3C6E6F743E}{\isasymnot}}A\ {\isaliteral{5C3C616E643E}{\isasymand}}\ B{\isaliteral{22}{\isachardoublequoteclose}}%
\begin{isamarkuptext}%
\noindent%
The symbol \indexboldpos{\isasymequiv}{$IsaEq} is a special form of equality
that must be used in constant definitions.
Pattern-matching is not allowed: each definition must be of
the form $f\,x@1\,\dots\,x@n~\isasymequiv~t$.
Section~\ref{sec:Simp-with-Defs} explains how definitions are used
in proofs. The default name of each definition is $f$\isa{{\isaliteral{5F}{\isacharunderscore}}def}, where
$f$ is the name of the defined constant.%
\end{isamarkuptext}%
\isamarkuptrue%
%
\isadelimtheory
%
\endisadelimtheory
%
\isatagtheory
%
\endisatagtheory
{\isafoldtheory}%
%
\isadelimtheory
%
\endisadelimtheory
\end{isabellebody}%
%%% Local Variables:
%%% mode: latex
%%% TeX-master: "root"
%%% End:


%
\begin{isabellebody}%
\def\isabellecontext{prime{\isaliteral{5F}{\isacharunderscore}}def}%
%
\isadelimtheory
%
\endisadelimtheory
%
\isatagtheory
%
\endisatagtheory
{\isafoldtheory}%
%
\isadelimtheory
%
\endisadelimtheory
%
\begin{isamarkuptext}%
\begin{warn}
A common mistake when writing definitions is to introduce extra free
variables on the right-hand side.  Consider the following, flawed definition
(where \isa{dvd} means ``divides''):
\begin{isabelle}%
\ \ \ \ \ {\isaliteral{22}{\isachardoublequote}}prime\ p\ {\isaliteral{5C3C65717569763E}{\isasymequiv}}\ {\isadigit{1}}\ {\isaliteral{3C}{\isacharless}}\ p\ {\isaliteral{5C3C616E643E}{\isasymand}}\ {\isaliteral{28}{\isacharparenleft}}m\ dvd\ p\ {\isaliteral{5C3C6C6F6E6772696768746172726F773E}{\isasymlongrightarrow}}\ m\ {\isaliteral{3D}{\isacharequal}}\ {\isadigit{1}}\ {\isaliteral{5C3C6F723E}{\isasymor}}\ m\ {\isaliteral{3D}{\isacharequal}}\ p{\isaliteral{29}{\isacharparenright}}{\isaliteral{22}{\isachardoublequote}}%
\end{isabelle}
\par\noindent\hangindent=0pt
Isabelle rejects this ``definition'' because of the extra \isa{m} on the
right-hand side, which would introduce an inconsistency (why?). 
The correct version is
\begin{isabelle}%
\ \ \ \ \ {\isaliteral{22}{\isachardoublequote}}prime\ p\ {\isaliteral{5C3C65717569763E}{\isasymequiv}}\ {\isadigit{1}}\ {\isaliteral{3C}{\isacharless}}\ p\ {\isaliteral{5C3C616E643E}{\isasymand}}\ {\isaliteral{28}{\isacharparenleft}}{\isaliteral{5C3C666F72616C6C3E}{\isasymforall}}m{\isaliteral{2E}{\isachardot}}\ m\ dvd\ p\ {\isaliteral{5C3C6C6F6E6772696768746172726F773E}{\isasymlongrightarrow}}\ m\ {\isaliteral{3D}{\isacharequal}}\ {\isadigit{1}}\ {\isaliteral{5C3C6F723E}{\isasymor}}\ m\ {\isaliteral{3D}{\isacharequal}}\ p{\isaliteral{29}{\isacharparenright}}{\isaliteral{22}{\isachardoublequote}}%
\end{isabelle}
\end{warn}%
\end{isamarkuptext}%
\isamarkuptrue%
%
\isadelimtheory
%
\endisadelimtheory
%
\isatagtheory
%
\endisatagtheory
{\isafoldtheory}%
%
\isadelimtheory
%
\endisadelimtheory
\end{isabellebody}%
%%% Local Variables:
%%% mode: latex
%%% TeX-master: "root"
%%% End:



\section{The Definitional Approach}
\label{sec:definitional}

\index{Definitional Approach}%
As we pointed out at the beginning of the chapter, asserting arbitrary
axioms such as $f(n) = f(n) + 1$ can easily lead to contradictions. In order
to avoid this danger, we advocate the definitional rather than
the axiomatic approach: introduce new concepts by definitions. However,  Isabelle/HOL seems to
support many richer definitional constructs, such as
\isacommand{primrec}. The point is that Isabelle reduces such constructs to first principles. For example, each
\isacommand{primrec} function definition is turned into a proper
(nonrecursive!) definition from which the user-supplied recursion equations are
automatically proved.  This process is
hidden from the user, who does not have to understand the details.  Other commands described
later, like \isacommand{fun} and \isacommand{inductive}, work similarly.  
This strict adherence to the definitional approach reduces the risk of 
soundness errors.

\chapter{More Functional Programming}

The purpose of this chapter is to deepen your understanding of the
concepts encountered so far and to introduce advanced forms of datatypes and
recursive functions. The first two sections give a structured presentation of
theorem proving by simplification ({\S}\ref{sec:Simplification}) and discuss
important heuristics for induction ({\S}\ref{sec:InductionHeuristics}).  You can
skip them if you are not planning to perform proofs yourself.
We then present a case
study: a compiler for expressions ({\S}\ref{sec:ExprCompiler}). Advanced
datatypes, including those involving function spaces, are covered in
{\S}\ref{sec:advanced-datatypes}; it closes with another case study, search
trees (``tries'').  Finally we introduce \isacommand{fun}, a general
form of recursive function definition that goes well beyond 
\isacommand{primrec} ({\S}\ref{sec:fun}).


\section{Simplification}
\label{sec:Simplification}
\index{simplification|(}

So far we have proved our theorems by \isa{auto}, which simplifies
all subgoals. In fact, \isa{auto} can do much more than that. 
To go beyond toy examples, you
need to understand the ingredients of \isa{auto}.  This section covers the
method that \isa{auto} always applies first, simplification.

Simplification is one of the central theorem proving tools in Isabelle and
many other systems. The tool itself is called the \textbf{simplifier}. 
This section introduces the many features of the simplifier
and is required reading if you intend to perform proofs.  Later on,
{\S}\ref{sec:simplification-II} explains some more advanced features and a
little bit of how the simplifier works. The serious student should read that
section as well, in particular to understand why the simplifier did
something unexpected.

\subsection{What is Simplification?}

In its most basic form, simplification means repeated application of
equations from left to right. For example, taking the rules for \isa{\at}
and applying them to the term \isa{[0,1] \at\ []} results in a sequence of
simplification steps:
\begin{ttbox}\makeatother
(0#1#[]) @ []  \(\leadsto\)  0#((1#[]) @ [])  \(\leadsto\)  0#(1#([] @ []))  \(\leadsto\)  0#1#[]
\end{ttbox}
This is also known as \bfindex{term rewriting}\indexbold{rewriting} and the
equations are referred to as \bfindex{rewrite rules}.
``Rewriting'' is more honest than ``simplification'' because the terms do not
necessarily become simpler in the process.

The simplifier proves arithmetic goals as described in
{\S}\ref{sec:nat} above.  Arithmetic expressions are simplified using built-in
procedures that go beyond mere rewrite rules.  New simplification procedures
can be coded and installed, but they are definitely not a matter for this
tutorial. 

%
\begin{isabellebody}%
\def\isabellecontext{simp}%
%
\isamarkupsubsubsection{Simplification rules%
}
%
\begin{isamarkuptext}%
\indexbold{simplification rule}
To facilitate simplification, theorems can be declared to be simplification
rules (with the help of the attribute \isa{{\isacharbrackleft}simp{\isacharbrackright}}\index{*simp
  (attribute)}), in which case proofs by simplification make use of these
rules automatically. In addition the constructs \isacommand{datatype} and
\isacommand{primrec} (and a few others) invisibly declare useful
simplification rules. Explicit definitions are \emph{not} declared
simplification rules automatically!

Not merely equations but pretty much any theorem can become a simplification
rule. The simplifier will try to make sense of it.  For example, a theorem
\isa{{\isasymnot}\ P} is automatically turned into \isa{P\ {\isacharequal}\ False}. The details
are explained in \S\ref{sec:SimpHow}.

The simplification attribute of theorems can be turned on and off as follows:
\begin{quote}
\isacommand{declare} \textit{theorem-name}\isa{{\isacharbrackleft}simp{\isacharbrackright}}\\
\isacommand{declare} \textit{theorem-name}\isa{{\isacharbrackleft}simp\ del{\isacharbrackright}}
\end{quote}
As a rule of thumb, equations that really simplify (like \isa{rev\ {\isacharparenleft}rev\ xs{\isacharparenright}\ {\isacharequal}\ xs} and \isa{xs\ {\isacharat}\ {\isacharbrackleft}{\isacharbrackright}\ {\isacharequal}\ xs}) should be made simplification
rules.  Those of a more specific nature (e.g.\ distributivity laws, which
alter the structure of terms considerably) should only be used selectively,
i.e.\ they should not be default simplification rules.  Conversely, it may
also happen that a simplification rule needs to be disabled in certain
proofs.  Frequent changes in the simplification status of a theorem may
indicate a badly designed theory.
\begin{warn}
  Simplification may not terminate, for example if both $f(x) = g(x)$ and
  $g(x) = f(x)$ are simplification rules. It is the user's responsibility not
  to include simplification rules that can lead to nontermination, either on
  their own or in combination with other simplification rules.
\end{warn}%
\end{isamarkuptext}%
%
\isamarkupsubsubsection{The simplification method%
}
%
\begin{isamarkuptext}%
\index{*simp (method)|bold}
The general format of the simplification method is
\begin{quote}
\isa{simp} \textit{list of modifiers}
\end{quote}
where the list of \emph{modifiers} helps to fine tune the behaviour and may
be empty. Most if not all of the proofs seen so far could have been performed
with \isa{simp} instead of \isa{auto}, except that \isa{simp} attacks
only the first subgoal and may thus need to be repeated---use
\isaindex{simp_all} to simplify all subgoals.
Note that \isa{simp} fails if nothing changes.%
\end{isamarkuptext}%
%
\isamarkupsubsubsection{Adding and deleting simplification rules%
}
%
\begin{isamarkuptext}%
If a certain theorem is merely needed in a few proofs by simplification,
we do not need to make it a global simplification rule. Instead we can modify
the set of simplification rules used in a simplification step by adding rules
to it and/or deleting rules from it. The two modifiers for this are
\begin{quote}
\isa{add{\isacharcolon}} \textit{list of theorem names}\\
\isa{del{\isacharcolon}} \textit{list of theorem names}
\end{quote}
In case you want to use only a specific list of theorems and ignore all
others:
\begin{quote}
\isa{only{\isacharcolon}} \textit{list of theorem names}
\end{quote}%
\end{isamarkuptext}%
%
\isamarkupsubsubsection{Assumptions%
}
%
\begin{isamarkuptext}%
\index{simplification!with/of assumptions}
By default, assumptions are part of the simplification process: they are used
as simplification rules and are simplified themselves. For example:%
\end{isamarkuptext}%
\isacommand{lemma}\ {\isachardoublequote}{\isasymlbrakk}\ xs\ {\isacharat}\ zs\ {\isacharequal}\ ys\ {\isacharat}\ xs{\isacharsemicolon}\ {\isacharbrackleft}{\isacharbrackright}\ {\isacharat}\ xs\ {\isacharequal}\ {\isacharbrackleft}{\isacharbrackright}\ {\isacharat}\ {\isacharbrackleft}{\isacharbrackright}\ {\isasymrbrakk}\ {\isasymLongrightarrow}\ ys\ {\isacharequal}\ zs{\isachardoublequote}\isanewline
\isacommand{apply}\ simp\isanewline
\isacommand{done}%
\begin{isamarkuptext}%
\noindent
The second assumption simplifies to \isa{xs\ {\isacharequal}\ {\isacharbrackleft}{\isacharbrackright}}, which in turn
simplifies the first assumption to \isa{zs\ {\isacharequal}\ ys}, thus reducing the
conclusion to \isa{ys\ {\isacharequal}\ ys} and hence to \isa{True}.

In some cases this may be too much of a good thing and may lead to
nontermination:%
\end{isamarkuptext}%
\isacommand{lemma}\ {\isachardoublequote}{\isasymforall}x{\isachardot}\ f\ x\ {\isacharequal}\ g\ {\isacharparenleft}f\ {\isacharparenleft}g\ x{\isacharparenright}{\isacharparenright}\ {\isasymLongrightarrow}\ f\ {\isacharbrackleft}{\isacharbrackright}\ {\isacharequal}\ f\ {\isacharbrackleft}{\isacharbrackright}\ {\isacharat}\ {\isacharbrackleft}{\isacharbrackright}{\isachardoublequote}%
\begin{isamarkuptxt}%
\noindent
cannot be solved by an unmodified application of \isa{simp} because the
simplification rule \isa{f\ x\ {\isacharequal}\ g\ {\isacharparenleft}f\ {\isacharparenleft}g\ x{\isacharparenright}{\isacharparenright}} extracted from the assumption
does not terminate. Isabelle notices certain simple forms of
nontermination but not this one. The problem can be circumvented by
explicitly telling the simplifier to ignore the assumptions:%
\end{isamarkuptxt}%
\isacommand{apply}{\isacharparenleft}simp\ {\isacharparenleft}no{\isacharunderscore}asm{\isacharparenright}{\isacharparenright}\isanewline
\isacommand{done}%
\begin{isamarkuptext}%
\noindent
There are three options that influence the treatment of assumptions:
\begin{description}
\item[\isa{{\isacharparenleft}no{\isacharunderscore}asm{\isacharparenright}}]\indexbold{*no_asm}
 means that assumptions are completely ignored.
\item[\isa{{\isacharparenleft}no{\isacharunderscore}asm{\isacharunderscore}simp{\isacharparenright}}]\indexbold{*no_asm_simp}
 means that the assumptions are not simplified but
  are used in the simplification of the conclusion.
\item[\isa{{\isacharparenleft}no{\isacharunderscore}asm{\isacharunderscore}use{\isacharparenright}}]\indexbold{*no_asm_use}
 means that the assumptions are simplified but are not
  used in the simplification of each other or the conclusion.
\end{description}
Neither \isa{{\isacharparenleft}no{\isacharunderscore}asm{\isacharunderscore}simp{\isacharparenright}} nor \isa{{\isacharparenleft}no{\isacharunderscore}asm{\isacharunderscore}use{\isacharparenright}} allow to simplify
the above problematic subgoal.

Note that only one of the above options is allowed, and it must precede all
other arguments.%
\end{isamarkuptext}%
%
\isamarkupsubsubsection{Rewriting with definitions%
}
%
\begin{isamarkuptext}%
\index{simplification!with definitions}
Constant definitions (\S\ref{sec:ConstDefinitions}) can
be used as simplification rules, but by default they are not.  Hence the
simplifier does not expand them automatically, just as it should be:
definitions are introduced for the purpose of abbreviating complex
concepts. Of course we need to expand the definitions initially to derive
enough lemmas that characterize the concept sufficiently for us to forget the
original definition. For example, given%
\end{isamarkuptext}%
\isacommand{constdefs}\ exor\ {\isacharcolon}{\isacharcolon}\ {\isachardoublequote}bool\ {\isasymRightarrow}\ bool\ {\isasymRightarrow}\ bool{\isachardoublequote}\isanewline
\ \ \ \ \ \ \ \ \ {\isachardoublequote}exor\ A\ B\ {\isasymequiv}\ {\isacharparenleft}A\ {\isasymand}\ {\isasymnot}B{\isacharparenright}\ {\isasymor}\ {\isacharparenleft}{\isasymnot}A\ {\isasymand}\ B{\isacharparenright}{\isachardoublequote}%
\begin{isamarkuptext}%
\noindent
we may want to prove%
\end{isamarkuptext}%
\isacommand{lemma}\ {\isachardoublequote}exor\ A\ {\isacharparenleft}{\isasymnot}A{\isacharparenright}{\isachardoublequote}%
\begin{isamarkuptxt}%
\noindent
Typically, the opening move consists in \emph{unfolding} the definition(s), which we need to
get started, but nothing else:\indexbold{*unfold}\indexbold{definition!unfolding}%
\end{isamarkuptxt}%
\isacommand{apply}{\isacharparenleft}simp\ only{\isacharcolon}exor{\isacharunderscore}def{\isacharparenright}%
\begin{isamarkuptxt}%
\noindent
In this particular case, the resulting goal
\begin{isabelle}%
\ {\isadigit{1}}{\isachardot}\ A\ {\isasymand}\ {\isasymnot}\ {\isasymnot}\ A\ {\isasymor}\ {\isasymnot}\ A\ {\isasymand}\ {\isasymnot}\ A%
\end{isabelle}
can be proved by simplification. Thus we could have proved the lemma outright by%
\end{isamarkuptxt}%
\isacommand{apply}{\isacharparenleft}simp\ add{\isacharcolon}\ exor{\isacharunderscore}def{\isacharparenright}%
\begin{isamarkuptext}%
\noindent
Of course we can also unfold definitions in the middle of a proof.

You should normally not turn a definition permanently into a simplification
rule because this defeats the whole purpose of an abbreviation.

\begin{warn}
  If you have defined $f\,x\,y~\isasymequiv~t$ then you can only expand
  occurrences of $f$ with at least two arguments. Thus it is safer to define
  $f$~\isasymequiv~\isasymlambda$x\,y.\;t$.
\end{warn}%
\end{isamarkuptext}%
%
\isamarkupsubsubsection{Simplifying let-expressions%
}
%
\begin{isamarkuptext}%
\index{simplification!of let-expressions}
Proving a goal containing \isaindex{let}-expressions almost invariably
requires the \isa{let}-con\-structs to be expanded at some point. Since
\isa{let}-\isa{in} is just syntactic sugar for a predefined constant
(called \isa{Let}), expanding \isa{let}-constructs means rewriting with
\isa{Let{\isacharunderscore}def}:%
\end{isamarkuptext}%
\isacommand{lemma}\ {\isachardoublequote}{\isacharparenleft}let\ xs\ {\isacharequal}\ {\isacharbrackleft}{\isacharbrackright}\ in\ xs{\isacharat}ys{\isacharat}xs{\isacharparenright}\ {\isacharequal}\ ys{\isachardoublequote}\isanewline
\isacommand{apply}{\isacharparenleft}simp\ add{\isacharcolon}\ Let{\isacharunderscore}def{\isacharparenright}\isanewline
\isacommand{done}%
\begin{isamarkuptext}%
If, in a particular context, there is no danger of a combinatorial explosion
of nested \isa{let}s one could even simlify with \isa{Let{\isacharunderscore}def} by
default:%
\end{isamarkuptext}%
\isacommand{declare}\ Let{\isacharunderscore}def\ {\isacharbrackleft}simp{\isacharbrackright}%
\isamarkupsubsubsection{Conditional equations%
}
%
\begin{isamarkuptext}%
So far all examples of rewrite rules were equations. The simplifier also
accepts \emph{conditional} equations, for example%
\end{isamarkuptext}%
\isacommand{lemma}\ hd{\isacharunderscore}Cons{\isacharunderscore}tl{\isacharbrackleft}simp{\isacharbrackright}{\isacharcolon}\ {\isachardoublequote}xs\ {\isasymnoteq}\ {\isacharbrackleft}{\isacharbrackright}\ \ {\isasymLongrightarrow}\ \ hd\ xs\ {\isacharhash}\ tl\ xs\ {\isacharequal}\ xs{\isachardoublequote}\isanewline
\isacommand{apply}{\isacharparenleft}case{\isacharunderscore}tac\ xs{\isacharcomma}\ simp{\isacharcomma}\ simp{\isacharparenright}\isanewline
\isacommand{done}%
\begin{isamarkuptext}%
\noindent
Note the use of ``\ttindexboldpos{,}{$Isar}'' to string together a
sequence of methods. Assuming that the simplification rule
\isa{{\isacharparenleft}rev\ xs\ {\isacharequal}\ {\isacharbrackleft}{\isacharbrackright}{\isacharparenright}\ {\isacharequal}\ {\isacharparenleft}xs\ {\isacharequal}\ {\isacharbrackleft}{\isacharbrackright}{\isacharparenright}}
is present as well,%
\end{isamarkuptext}%
\isacommand{lemma}\ {\isachardoublequote}xs\ {\isasymnoteq}\ {\isacharbrackleft}{\isacharbrackright}\ {\isasymLongrightarrow}\ hd{\isacharparenleft}rev\ xs{\isacharparenright}\ {\isacharhash}\ tl{\isacharparenleft}rev\ xs{\isacharparenright}\ {\isacharequal}\ rev\ xs{\isachardoublequote}%
\begin{isamarkuptext}%
\noindent
is proved by plain simplification:
the conditional equation \isa{hd{\isacharunderscore}Cons{\isacharunderscore}tl} above
can simplify \isa{hd\ {\isacharparenleft}rev\ xs{\isacharparenright}\ {\isacharhash}\ tl\ {\isacharparenleft}rev\ xs{\isacharparenright}} to \isa{rev\ xs}
because the corresponding precondition \isa{rev\ xs\ {\isasymnoteq}\ {\isacharbrackleft}{\isacharbrackright}}
simplifies to \isa{xs\ {\isasymnoteq}\ {\isacharbrackleft}{\isacharbrackright}}, which is exactly the local
assumption of the subgoal.%
\end{isamarkuptext}%
%
\isamarkupsubsubsection{Automatic case splits%
}
%
\begin{isamarkuptext}%
\indexbold{case splits}\index{*split|(}
Goals containing \isa{if}-expressions are usually proved by case
distinction on the condition of the \isa{if}. For example the goal%
\end{isamarkuptext}%
\isacommand{lemma}\ {\isachardoublequote}{\isasymforall}xs{\isachardot}\ if\ xs\ {\isacharequal}\ {\isacharbrackleft}{\isacharbrackright}\ then\ rev\ xs\ {\isacharequal}\ {\isacharbrackleft}{\isacharbrackright}\ else\ rev\ xs\ {\isasymnoteq}\ {\isacharbrackleft}{\isacharbrackright}{\isachardoublequote}%
\begin{isamarkuptxt}%
\noindent
can be split by a degenerate form of simplification%
\end{isamarkuptxt}%
\isacommand{apply}{\isacharparenleft}simp\ only{\isacharcolon}\ split{\isacharcolon}\ split{\isacharunderscore}if{\isacharparenright}%
\begin{isamarkuptxt}%
\noindent
\begin{isabelle}%
\ {\isadigit{1}}{\isachardot}\ {\isasymforall}xs{\isachardot}\ {\isacharparenleft}xs\ {\isacharequal}\ {\isacharbrackleft}{\isacharbrackright}\ {\isasymlongrightarrow}\ rev\ xs\ {\isacharequal}\ {\isacharbrackleft}{\isacharbrackright}{\isacharparenright}\ {\isasymand}\ {\isacharparenleft}xs\ {\isasymnoteq}\ {\isacharbrackleft}{\isacharbrackright}\ {\isasymlongrightarrow}\ rev\ xs\ {\isasymnoteq}\ {\isacharbrackleft}{\isacharbrackright}{\isacharparenright}%
\end{isabelle}
where no simplification rules are included (\isa{only{\isacharcolon}} is followed by the
empty list of theorems) but the rule \isaindexbold{split_if} for
splitting \isa{if}s is added (via the modifier \isa{split{\isacharcolon}}). Because
case-splitting on \isa{if}s is almost always the right proof strategy, the
simplifier performs it automatically. Try \isacommand{apply}\isa{{\isacharparenleft}simp{\isacharparenright}}
on the initial goal above.

This splitting idea generalizes from \isa{if} to \isaindex{case}:%
\end{isamarkuptxt}%
\isanewline
\isacommand{lemma}\ {\isachardoublequote}{\isacharparenleft}case\ xs\ of\ {\isacharbrackleft}{\isacharbrackright}\ {\isasymRightarrow}\ zs\ {\isacharbar}\ y{\isacharhash}ys\ {\isasymRightarrow}\ y{\isacharhash}{\isacharparenleft}ys{\isacharat}zs{\isacharparenright}{\isacharparenright}\ {\isacharequal}\ xs{\isacharat}zs{\isachardoublequote}\isanewline
\isacommand{apply}{\isacharparenleft}simp\ only{\isacharcolon}\ split{\isacharcolon}\ list{\isachardot}split{\isacharparenright}%
\begin{isamarkuptxt}%
\begin{isabelle}%
\ {\isadigit{1}}{\isachardot}\ {\isacharparenleft}xs\ {\isacharequal}\ {\isacharbrackleft}{\isacharbrackright}\ {\isasymlongrightarrow}\ zs\ {\isacharequal}\ xs\ {\isacharat}\ zs{\isacharparenright}\ {\isasymand}\isanewline
\ \ \ \ {\isacharparenleft}{\isasymforall}a\ list{\isachardot}\ xs\ {\isacharequal}\ a\ {\isacharhash}\ list\ {\isasymlongrightarrow}\ a\ {\isacharhash}\ list\ {\isacharat}\ zs\ {\isacharequal}\ xs\ {\isacharat}\ zs{\isacharparenright}%
\end{isabelle}
In contrast to \isa{if}-expressions, the simplifier does not split
\isa{case}-expressions by default because this can lead to nontermination
in case of recursive datatypes. Again, if the \isa{only{\isacharcolon}} modifier is
dropped, the above goal is solved,%
\end{isamarkuptxt}%
\isacommand{apply}{\isacharparenleft}simp\ split{\isacharcolon}\ list{\isachardot}split{\isacharparenright}%
\begin{isamarkuptext}%
\noindent%
which \isacommand{apply}\isa{{\isacharparenleft}simp{\isacharparenright}} alone will not do.

In general, every datatype $t$ comes with a theorem
$t$\isa{{\isachardot}split} which can be declared to be a \bfindex{split rule} either
locally as above, or by giving it the \isa{split} attribute globally:%
\end{isamarkuptext}%
\isacommand{declare}\ list{\isachardot}split\ {\isacharbrackleft}split{\isacharbrackright}%
\begin{isamarkuptext}%
\noindent
The \isa{split} attribute can be removed with the \isa{del} modifier,
either locally%
\end{isamarkuptext}%
\isacommand{apply}{\isacharparenleft}simp\ split\ del{\isacharcolon}\ split{\isacharunderscore}if{\isacharparenright}%
\begin{isamarkuptext}%
\noindent
or globally:%
\end{isamarkuptext}%
\isacommand{declare}\ list{\isachardot}split\ {\isacharbrackleft}split\ del{\isacharbrackright}%
\begin{isamarkuptext}%
The above split rules intentionally only affect the conclusion of a
subgoal.  If you want to split an \isa{if} or \isa{case}-expression in
the assumptions, you have to apply \isa{split{\isacharunderscore}if{\isacharunderscore}asm} or
$t$\isa{{\isachardot}split{\isacharunderscore}asm}:%
\end{isamarkuptext}%
\isacommand{lemma}\ {\isachardoublequote}if\ xs\ {\isacharequal}\ {\isacharbrackleft}{\isacharbrackright}\ then\ ys\ {\isachartilde}{\isacharequal}\ {\isacharbrackleft}{\isacharbrackright}\ else\ ys\ {\isacharequal}\ {\isacharbrackleft}{\isacharbrackright}\ {\isacharequal}{\isacharequal}{\isachargreater}\ xs\ {\isacharat}\ ys\ {\isachartilde}{\isacharequal}\ {\isacharbrackleft}{\isacharbrackright}{\isachardoublequote}\isanewline
\isacommand{apply}{\isacharparenleft}simp\ only{\isacharcolon}\ split{\isacharcolon}\ split{\isacharunderscore}if{\isacharunderscore}asm{\isacharparenright}%
\begin{isamarkuptxt}%
\noindent
In contrast to splitting the conclusion, this actually creates two
separate subgoals (which are solved by \isa{simp{\isacharunderscore}all}):
\begin{isabelle}%
\ {\isadigit{1}}{\isachardot}\ {\isasymlbrakk}xs\ {\isacharequal}\ {\isacharbrackleft}{\isacharbrackright}{\isacharsemicolon}\ ys\ {\isasymnoteq}\ {\isacharbrackleft}{\isacharbrackright}{\isasymrbrakk}\ {\isasymLongrightarrow}\ {\isacharbrackleft}{\isacharbrackright}\ {\isacharat}\ ys\ {\isasymnoteq}\ {\isacharbrackleft}{\isacharbrackright}\isanewline
\ {\isadigit{2}}{\isachardot}\ {\isasymlbrakk}xs\ {\isasymnoteq}\ {\isacharbrackleft}{\isacharbrackright}{\isacharsemicolon}\ ys\ {\isacharequal}\ {\isacharbrackleft}{\isacharbrackright}{\isasymrbrakk}\ {\isasymLongrightarrow}\ xs\ {\isacharat}\ {\isacharbrackleft}{\isacharbrackright}\ {\isasymnoteq}\ {\isacharbrackleft}{\isacharbrackright}%
\end{isabelle}
If you need to split both in the assumptions and the conclusion,
use $t$\isa{{\isachardot}splits} which subsumes $t$\isa{{\isachardot}split} and
$t$\isa{{\isachardot}split{\isacharunderscore}asm}. Analogously, there is \isa{if{\isacharunderscore}splits}.

\begin{warn}
  The simplifier merely simplifies the condition of an \isa{if} but not the
  \isa{then} or \isa{else} parts. The latter are simplified only after the
  condition reduces to \isa{True} or \isa{False}, or after splitting. The
  same is true for \isaindex{case}-expressions: only the selector is
  simplified at first, until either the expression reduces to one of the
  cases or it is split.
\end{warn}

\index{*split|)}%
\end{isamarkuptxt}%
%
\isamarkupsubsubsection{Arithmetic%
}
%
\begin{isamarkuptext}%
\index{arithmetic}
The simplifier routinely solves a small class of linear arithmetic formulae
(over type \isa{nat} and other numeric types): it only takes into account
assumptions and conclusions that are (possibly negated) (in)equalities
(\isa{{\isacharequal}}, \isasymle, \isa{{\isacharless}}) and it only knows about addition. Thus%
\end{isamarkuptext}%
\isacommand{lemma}\ {\isachardoublequote}{\isasymlbrakk}\ {\isasymnot}\ m\ {\isacharless}\ n{\isacharsemicolon}\ m\ {\isacharless}\ n{\isacharplus}{\isadigit{1}}\ {\isasymrbrakk}\ {\isasymLongrightarrow}\ m\ {\isacharequal}\ n{\isachardoublequote}%
\begin{isamarkuptext}%
\noindent
is proved by simplification, whereas the only slightly more complex%
\end{isamarkuptext}%
\isacommand{lemma}\ {\isachardoublequote}{\isasymnot}\ m\ {\isacharless}\ n\ {\isasymand}\ m\ {\isacharless}\ n{\isacharplus}{\isadigit{1}}\ {\isasymLongrightarrow}\ m\ {\isacharequal}\ n{\isachardoublequote}%
\begin{isamarkuptext}%
\noindent
is not proved by simplification and requires \isa{arith}.%
\end{isamarkuptext}%
%
\isamarkupsubsubsection{Tracing%
}
%
\begin{isamarkuptext}%
\indexbold{tracing the simplifier}
Using the simplifier effectively may take a bit of experimentation.  Set the
\isaindexbold{trace_simp} \rmindex{flag} to get a better idea of what is going
on:%
\end{isamarkuptext}%
\isacommand{ML}\ {\isachardoublequote}set\ trace{\isacharunderscore}simp{\isachardoublequote}\isanewline
\isacommand{lemma}\ {\isachardoublequote}rev\ {\isacharbrackleft}a{\isacharbrackright}\ {\isacharequal}\ {\isacharbrackleft}{\isacharbrackright}{\isachardoublequote}\isanewline
\isacommand{apply}{\isacharparenleft}simp{\isacharparenright}%
\begin{isamarkuptext}%
\noindent
produces the trace

\begin{ttbox}\makeatother
Applying instance of rewrite rule:
rev (?x1 \# ?xs1) == rev ?xs1 @ [?x1]
Rewriting:
rev [x] == rev [] @ [x]
Applying instance of rewrite rule:
rev [] == []
Rewriting:
rev [] == []
Applying instance of rewrite rule:
[] @ ?y == ?y
Rewriting:
[] @ [x] == [x]
Applying instance of rewrite rule:
?x3 \# ?t3 = ?t3 == False
Rewriting:
[x] = [] == False
\end{ttbox}

In more complicated cases, the trace can be quite lenghty, especially since
invocations of the simplifier are often nested (e.g.\ when solving conditions
of rewrite rules). Thus it is advisable to reset it:%
\end{isamarkuptext}%
\isacommand{ML}\ {\isachardoublequote}reset\ trace{\isacharunderscore}simp{\isachardoublequote}\isanewline
\end{isabellebody}%
%%% Local Variables:
%%% mode: latex
%%% TeX-master: "root"
%%% End:


\index{simplification|)}

\begin{isabelle}%
%
\begin{isamarkuptext}%
Function \isa{rev} has quadratic worst-case running time
because it calls function \isa{\at} for each element of the list and
\isa{\at} is linear in its first argument.  A linear time version of
\isa{rev} reqires an extra argument where the result is accumulated
gradually, using only \isa{\#}:%
\end{isamarkuptext}%
\isacommand{consts}\ itrev\ {\isacharcolon}{\isacharcolon}\ {\isachardoublequote}{\isacharprime}a\ list\ {\isasymRightarrow}\ {\isacharprime}a\ list\ {\isasymRightarrow}\ {\isacharprime}a\ list{\isachardoublequote}\isanewline
\isacommand{primrec}\isanewline
{\isachardoublequote}itrev\ {\isacharbrackleft}{\isacharbrackright}\ \ \ \ \ ys\ {\isacharequal}\ ys{\isachardoublequote}\isanewline
{\isachardoublequote}itrev\ {\isacharparenleft}x{\isacharhash}xs{\isacharparenright}\ ys\ {\isacharequal}\ itrev\ xs\ {\isacharparenleft}x{\isacharhash}ys{\isacharparenright}{\isachardoublequote}%
\begin{isamarkuptext}%
\noindent The behaviour of \isa{itrev} is simple: it reverses
its first argument by stacking its elements onto the second argument,
and returning that second argument when the first one becomes
empty. Note that \isa{itrev} is tail-recursive, i.e.\ it can be
compiled into a loop.

Naturally, we would like to show that \isa{itrev} does indeed reverse
its first argument provided the second one is empty:%
\end{isamarkuptext}%
\isacommand{lemma}\ {\isachardoublequote}itrev\ xs\ {\isacharbrackleft}{\isacharbrackright}\ {\isacharequal}\ rev\ xs{\isachardoublequote}%
\begin{isamarkuptxt}%
\noindent
There is no choice as to the induction variable, and we immediately simplify:%
\end{isamarkuptxt}%
\isacommand{apply}{\isacharparenleft}induct{\isacharunderscore}tac\ xs{\isacharcomma}\ auto{\isacharparenright}%
\begin{isamarkuptxt}%
\noindent
Unfortunately, this is not a complete success:
\begin{isabellepar}%
~1.~\dots~itrev~list~[]~=~rev~list~{\isasymLongrightarrow}~itrev~list~[a]~=~rev~list~@~[a]%
\end{isabellepar}%
Just as predicted above, the overall goal, and hence the induction
hypothesis, is too weak to solve the induction step because of the fixed
\isa{[]}. The corresponding heuristic:
\begin{quote}
{\em 3. Generalize goals for induction by replacing constants by variables.}
\end{quote}

Of course one cannot do this na\"{\i}vely: \isa{itrev xs ys = rev xs} is
just not true---the correct generalization is%
\end{isamarkuptxt}%
\isacommand{lemma}\ {\isachardoublequote}itrev\ xs\ ys\ {\isacharequal}\ rev\ xs\ {\isacharat}\ ys{\isachardoublequote}%
\begin{isamarkuptxt}%
\noindent
If \isa{ys} is replaced by \isa{[]}, the right-hand side simplifies to
\isa{rev\ \mbox{xs}}, just as required.

In this particular instance it was easy to guess the right generalization,
but in more complex situations a good deal of creativity is needed. This is
the main source of complications in inductive proofs.

Although we now have two variables, only \isa{xs} is suitable for
induction, and we repeat our above proof attempt. Unfortunately, we are still
not there:
\begin{isabellepar}%
~1.~{\isasymAnd}a~list.\isanewline
~~~~~~~itrev~list~ys~=~rev~list~@~ys~{\isasymLongrightarrow}\isanewline
~~~~~~~itrev~list~(a~\#~ys)~=~rev~list~@~a~\#~ys%
\end{isabellepar}%
The induction hypothesis is still too weak, but this time it takes no
intuition to generalize: the problem is that \isa{ys} is fixed throughout
the subgoal, but the induction hypothesis needs to be applied with
\isa{\mbox{a}\ {\isacharhash}\ \mbox{ys}} instead of \isa{ys}. Hence we prove the theorem
for all \isa{ys} instead of a fixed one:%
\end{isamarkuptxt}%
\isacommand{lemma}\ {\isachardoublequote}{\isasymforall}ys{\isachardot}\ itrev\ xs\ ys\ {\isacharequal}\ rev\ xs\ {\isacharat}\ ys{\isachardoublequote}%
\begin{isamarkuptxt}%
\noindent
This time induction on \isa{xs} followed by simplification succeeds. This
leads to another heuristic for generalization:
\begin{quote}
{\em 4. Generalize goals for induction by universally quantifying all free
variables {\em(except the induction variable itself!)}.}
\end{quote}
This prevents trivial failures like the above and does not change the
provability of the goal. Because it is not always required, and may even
complicate matters in some cases, this heuristic is often not
applied blindly.

In general, if you have tried the above heuristics and still find your
induction does not go through, and no obvious lemma suggests itself, you may
need to generalize your proposition even further. This requires insight into
the problem at hand and is beyond simple rules of thumb. In a nutshell: you
will need to be creative. Additionally, you can read \S\ref{sec:advanced-ind}
to learn about some advanced techniques for inductive proofs.%
\end{isamarkuptxt}%
\end{isabelle}%
%%% Local Variables:
%%% mode: latex
%%% TeX-master: "root"
%%% End:

\begin{exercise}
%
\begin{isabellebody}%
\def\isabellecontext{Plus}%
\isamarkupfalse%
%
\begin{isamarkuptext}%
\noindent Define the following addition function%
\end{isamarkuptext}%
\isamarkuptrue%
\isacommand{consts}\ plus\ {\isacharcolon}{\isacharcolon}\ {\isachardoublequote}nat\ {\isasymRightarrow}\ nat\ {\isasymRightarrow}\ nat{\isachardoublequote}\isanewline
\isamarkupfalse%
\isacommand{primrec}\isanewline
{\isachardoublequote}plus\ m\ {\isadigit{0}}\ {\isacharequal}\ m{\isachardoublequote}\isanewline
{\isachardoublequote}plus\ m\ {\isacharparenleft}Suc\ n{\isacharparenright}\ {\isacharequal}\ plus\ {\isacharparenleft}Suc\ m{\isacharparenright}\ n{\isachardoublequote}\isamarkupfalse%
%
\begin{isamarkuptext}%
\noindent and prove%
\end{isamarkuptext}%
\isamarkuptrue%
\isamarkupfalse%
\isamarkupfalse%
\isanewline
\isamarkupfalse%
\isacommand{lemma}\ {\isachardoublequote}plus\ m\ n\ {\isacharequal}\ m{\isacharplus}n{\isachardoublequote}\isamarkupfalse%
\isamarkupfalse%
\isanewline
\isamarkupfalse%
\end{isabellebody}%
%%% Local Variables:
%%% mode: latex
%%% TeX-master: "root"
%%% End:
%
\end{exercise}
\begin{exercise}
%
\begin{isabellebody}%
\def\isabellecontext{Tree{\isadigit{2}}}%
\isamarkupfalse%
%
\begin{isamarkuptext}%
\noindent In Exercise~\ref{ex:Tree} we defined a function
\isa{flatten} from trees to lists. The straightforward version of
\isa{flatten} is based on \isa{{\isacharat}} and is thus, like \isa{rev},
quadratic. A linear time version of \isa{flatten} again reqires an extra
argument, the accumulator:%
\end{isamarkuptext}%
\isamarkuptrue%
\isacommand{consts}\ flatten{\isadigit{2}}\ {\isacharcolon}{\isacharcolon}\ {\isachardoublequote}{\isacharprime}a\ tree\ {\isacharequal}{\isachargreater}\ {\isacharprime}a\ list\ {\isacharequal}{\isachargreater}\ {\isacharprime}a\ list{\isachardoublequote}\isamarkupfalse%
\isamarkupfalse%
%
\begin{isamarkuptext}%
\noindent Define \isa{flatten{\isadigit{2}}} and prove%
\end{isamarkuptext}%
\isamarkuptrue%
\isamarkupfalse%
\isamarkupfalse%
\isamarkupfalse%
\isacommand{lemma}\ {\isachardoublequote}flatten{\isadigit{2}}\ t\ {\isacharbrackleft}{\isacharbrackright}\ {\isacharequal}\ flatten\ t{\isachardoublequote}\isamarkupfalse%
\isamarkupfalse%
\isamarkupfalse%
\end{isabellebody}%
%%% Local Variables:
%%% mode: latex
%%% TeX-master: "root"
%%% End:
%
\end{exercise}

%
\begin{isabellebody}%
\def\isabellecontext{CodeGen}%
%
\isamarkupsection{Case Study: Compiling Expressions%
}
%
\begin{isamarkuptext}%
\label{sec:ExprCompiler}
\index{compiling expressions example|(}%
The task is to develop a compiler from a generic type of expressions (built
from variables, constants and binary operations) to a stack machine.  This
generic type of expressions is a generalization of the boolean expressions in
\S\ref{sec:boolex}.  This time we do not commit ourselves to a particular
type of variables or values but make them type parameters.  Neither is there
a fixed set of binary operations: instead the expression contains the
appropriate function itself.%
\end{isamarkuptext}%
\isacommand{types}\ {\isacharprime}v\ binop\ {\isacharequal}\ {\isachardoublequote}{\isacharprime}v\ {\isasymRightarrow}\ {\isacharprime}v\ {\isasymRightarrow}\ {\isacharprime}v{\isachardoublequote}\isanewline
\isacommand{datatype}\ {\isacharparenleft}{\isacharprime}a{\isacharcomma}{\isacharprime}v{\isacharparenright}expr\ {\isacharequal}\ Cex\ {\isacharprime}v\isanewline
\ \ \ \ \ \ \ \ \ \ \ \ \ \ \ \ \ \ \ \ \ {\isacharbar}\ Vex\ {\isacharprime}a\isanewline
\ \ \ \ \ \ \ \ \ \ \ \ \ \ \ \ \ \ \ \ \ {\isacharbar}\ Bex\ {\isachardoublequote}{\isacharprime}v\ binop{\isachardoublequote}\ \ {\isachardoublequote}{\isacharparenleft}{\isacharprime}a{\isacharcomma}{\isacharprime}v{\isacharparenright}expr{\isachardoublequote}\ \ {\isachardoublequote}{\isacharparenleft}{\isacharprime}a{\isacharcomma}{\isacharprime}v{\isacharparenright}expr{\isachardoublequote}%
\begin{isamarkuptext}%
\noindent
The three constructors represent constants, variables and the application of
a binary operation to two subexpressions.

The value of an expression with respect to an environment that maps variables to
values is easily defined:%
\end{isamarkuptext}%
\isacommand{consts}\ value\ {\isacharcolon}{\isacharcolon}\ {\isachardoublequote}{\isacharparenleft}{\isacharprime}a{\isacharcomma}{\isacharprime}v{\isacharparenright}expr\ {\isasymRightarrow}\ {\isacharparenleft}{\isacharprime}a\ {\isasymRightarrow}\ {\isacharprime}v{\isacharparenright}\ {\isasymRightarrow}\ {\isacharprime}v{\isachardoublequote}\isanewline
\isacommand{primrec}\isanewline
{\isachardoublequote}value\ {\isacharparenleft}Cex\ v{\isacharparenright}\ env\ {\isacharequal}\ v{\isachardoublequote}\isanewline
{\isachardoublequote}value\ {\isacharparenleft}Vex\ a{\isacharparenright}\ env\ {\isacharequal}\ env\ a{\isachardoublequote}\isanewline
{\isachardoublequote}value\ {\isacharparenleft}Bex\ f\ e{\isadigit{1}}\ e{\isadigit{2}}{\isacharparenright}\ env\ {\isacharequal}\ f\ {\isacharparenleft}value\ e{\isadigit{1}}\ env{\isacharparenright}\ {\isacharparenleft}value\ e{\isadigit{2}}\ env{\isacharparenright}{\isachardoublequote}%
\begin{isamarkuptext}%
The stack machine has three instructions: load a constant value onto the
stack, load the contents of an address onto the stack, and apply a
binary operation to the two topmost elements of the stack, replacing them by
the result. As for \isa{expr}, addresses and values are type parameters:%
\end{isamarkuptext}%
\isacommand{datatype}\ {\isacharparenleft}{\isacharprime}a{\isacharcomma}{\isacharprime}v{\isacharparenright}\ instr\ {\isacharequal}\ Const\ {\isacharprime}v\isanewline
\ \ \ \ \ \ \ \ \ \ \ \ \ \ \ \ \ \ \ \ \ \ \ {\isacharbar}\ Load\ {\isacharprime}a\isanewline
\ \ \ \ \ \ \ \ \ \ \ \ \ \ \ \ \ \ \ \ \ \ \ {\isacharbar}\ Apply\ {\isachardoublequote}{\isacharprime}v\ binop{\isachardoublequote}%
\begin{isamarkuptext}%
The execution of the stack machine is modelled by a function
\isa{exec} that takes a list of instructions, a store (modelled as a
function from addresses to values, just like the environment for
evaluating expressions), and a stack (modelled as a list) of values,
and returns the stack at the end of the execution --- the store remains
unchanged:%
\end{isamarkuptext}%
\isacommand{consts}\ exec\ {\isacharcolon}{\isacharcolon}\ {\isachardoublequote}{\isacharparenleft}{\isacharprime}a{\isacharcomma}{\isacharprime}v{\isacharparenright}instr\ list\ {\isasymRightarrow}\ {\isacharparenleft}{\isacharprime}a{\isasymRightarrow}{\isacharprime}v{\isacharparenright}\ {\isasymRightarrow}\ {\isacharprime}v\ list\ {\isasymRightarrow}\ {\isacharprime}v\ list{\isachardoublequote}\isanewline
\isacommand{primrec}\isanewline
{\isachardoublequote}exec\ {\isacharbrackleft}{\isacharbrackright}\ s\ vs\ {\isacharequal}\ vs{\isachardoublequote}\isanewline
{\isachardoublequote}exec\ {\isacharparenleft}i{\isacharhash}is{\isacharparenright}\ s\ vs\ {\isacharequal}\ {\isacharparenleft}case\ i\ of\isanewline
\ \ \ \ Const\ v\ \ {\isasymRightarrow}\ exec\ is\ s\ {\isacharparenleft}v{\isacharhash}vs{\isacharparenright}\isanewline
\ \ {\isacharbar}\ Load\ a\ \ \ {\isasymRightarrow}\ exec\ is\ s\ {\isacharparenleft}{\isacharparenleft}s\ a{\isacharparenright}{\isacharhash}vs{\isacharparenright}\isanewline
\ \ {\isacharbar}\ Apply\ f\ \ {\isasymRightarrow}\ exec\ is\ s\ {\isacharparenleft}{\isacharparenleft}f\ {\isacharparenleft}hd\ vs{\isacharparenright}\ {\isacharparenleft}hd{\isacharparenleft}tl\ vs{\isacharparenright}{\isacharparenright}{\isacharparenright}{\isacharhash}{\isacharparenleft}tl{\isacharparenleft}tl\ vs{\isacharparenright}{\isacharparenright}{\isacharparenright}{\isacharparenright}{\isachardoublequote}%
\begin{isamarkuptext}%
\noindent
Recall that \isa{hd} and \isa{tl}
return the first element and the remainder of a list.
Because all functions are total, \cdx{hd} is defined even for the empty
list, although we do not know what the result is. Thus our model of the
machine always terminates properly, although the definition above does not
tell us much about the result in situations where \isa{Apply} was executed
with fewer than two elements on the stack.

The compiler is a function from expressions to a list of instructions. Its
definition is obvious:%
\end{isamarkuptext}%
\isacommand{consts}\ comp\ {\isacharcolon}{\isacharcolon}\ {\isachardoublequote}{\isacharparenleft}{\isacharprime}a{\isacharcomma}{\isacharprime}v{\isacharparenright}expr\ {\isasymRightarrow}\ {\isacharparenleft}{\isacharprime}a{\isacharcomma}{\isacharprime}v{\isacharparenright}instr\ list{\isachardoublequote}\isanewline
\isacommand{primrec}\isanewline
{\isachardoublequote}comp\ {\isacharparenleft}Cex\ v{\isacharparenright}\ \ \ \ \ \ \ {\isacharequal}\ {\isacharbrackleft}Const\ v{\isacharbrackright}{\isachardoublequote}\isanewline
{\isachardoublequote}comp\ {\isacharparenleft}Vex\ a{\isacharparenright}\ \ \ \ \ \ \ {\isacharequal}\ {\isacharbrackleft}Load\ a{\isacharbrackright}{\isachardoublequote}\isanewline
{\isachardoublequote}comp\ {\isacharparenleft}Bex\ f\ e{\isadigit{1}}\ e{\isadigit{2}}{\isacharparenright}\ {\isacharequal}\ {\isacharparenleft}comp\ e{\isadigit{2}}{\isacharparenright}\ {\isacharat}\ {\isacharparenleft}comp\ e{\isadigit{1}}{\isacharparenright}\ {\isacharat}\ {\isacharbrackleft}Apply\ f{\isacharbrackright}{\isachardoublequote}%
\begin{isamarkuptext}%
Now we have to prove the correctness of the compiler, i.e.\ that the
execution of a compiled expression results in the value of the expression:%
\end{isamarkuptext}%
\isacommand{theorem}\ {\isachardoublequote}exec\ {\isacharparenleft}comp\ e{\isacharparenright}\ s\ {\isacharbrackleft}{\isacharbrackright}\ {\isacharequal}\ {\isacharbrackleft}value\ e\ s{\isacharbrackright}{\isachardoublequote}%
\begin{isamarkuptext}%
\noindent
This theorem needs to be generalized:%
\end{isamarkuptext}%
\isacommand{theorem}\ {\isachardoublequote}{\isasymforall}vs{\isachardot}\ exec\ {\isacharparenleft}comp\ e{\isacharparenright}\ s\ vs\ {\isacharequal}\ {\isacharparenleft}value\ e\ s{\isacharparenright}\ {\isacharhash}\ vs{\isachardoublequote}%
\begin{isamarkuptxt}%
\noindent
It will be proved by induction on \isa{e} followed by simplification.  
First, we must prove a lemma about executing the concatenation of two
instruction sequences:%
\end{isamarkuptxt}%
\isacommand{lemma}\ exec{\isacharunderscore}app{\isacharbrackleft}simp{\isacharbrackright}{\isacharcolon}\isanewline
\ \ {\isachardoublequote}{\isasymforall}vs{\isachardot}\ exec\ {\isacharparenleft}xs{\isacharat}ys{\isacharparenright}\ s\ vs\ {\isacharequal}\ exec\ ys\ s\ {\isacharparenleft}exec\ xs\ s\ vs{\isacharparenright}{\isachardoublequote}%
\begin{isamarkuptxt}%
\noindent
This requires induction on \isa{xs} and ordinary simplification for the
base cases. In the induction step, simplification leaves us with a formula
that contains two \isa{case}-expressions over instructions. Thus we add
automatic case splitting, which finishes the proof:%
\end{isamarkuptxt}%
\isacommand{apply}{\isacharparenleft}induct{\isacharunderscore}tac\ xs{\isacharcomma}\ simp{\isacharcomma}\ simp\ split{\isacharcolon}\ instr{\isachardot}split{\isacharparenright}%
\begin{isamarkuptext}%
\noindent
Note that because both \methdx{simp_all} and \methdx{auto} perform simplification, they can
be modified in the same way as \isa{simp}.  Thus the proof can be
rewritten as%
\end{isamarkuptext}%
\isacommand{apply}{\isacharparenleft}induct{\isacharunderscore}tac\ xs{\isacharcomma}\ simp{\isacharunderscore}all\ split{\isacharcolon}\ instr{\isachardot}split{\isacharparenright}%
\begin{isamarkuptext}%
\noindent
Although this is more compact, it is less clear for the reader of the proof.

We could now go back and prove \isa{exec (comp e) s [] = [value e s]}
merely by simplification with the generalized version we just proved.
However, this is unnecessary because the generalized version fully subsumes
its instance.%
\index{compiling expressions example|)}%
\end{isamarkuptext}%
\end{isabellebody}%
%%% Local Variables:
%%% mode: latex
%%% TeX-master: "root"
%%% End:



\section{Advanced Datatypes}
\label{sec:advanced-datatypes}
\index{datatype@\isacommand {datatype} (command)|(}
\index{primrec@\isacommand {primrec} (command)|(}
%|)

This section presents advanced forms of datatypes: mutual and nested
recursion.  A series of examples will culminate in a treatment of the trie
data structure.


\subsection{Mutual Recursion}
\label{sec:datatype-mut-rec}

%
\begin{isabellebody}%
\def\isabellecontext{ABexpr}%
%
\isadelimtheory
%
\endisadelimtheory
%
\isatagtheory
%
\endisatagtheory
{\isafoldtheory}%
%
\isadelimtheory
%
\endisadelimtheory
%
\begin{isamarkuptext}%
\index{datatypes!mutually recursive}%
Sometimes it is necessary to define two datatypes that depend on each
other. This is called \textbf{mutual recursion}. As an example consider a
language of arithmetic and boolean expressions where
\begin{itemize}
\item arithmetic expressions contain boolean expressions because there are
  conditional expressions like ``if $m<n$ then $n-m$ else $m-n$'',
  and
\item boolean expressions contain arithmetic expressions because of
  comparisons like ``$m<n$''.
\end{itemize}
In Isabelle this becomes%
\end{isamarkuptext}%
\isamarkuptrue%
\isacommand{datatype}\isamarkupfalse%
\ {\isaliteral{27}{\isacharprime}}a\ aexp\ {\isaliteral{3D}{\isacharequal}}\ IF\ \ \ {\isaliteral{22}{\isachardoublequoteopen}}{\isaliteral{27}{\isacharprime}}a\ bexp{\isaliteral{22}{\isachardoublequoteclose}}\ {\isaliteral{22}{\isachardoublequoteopen}}{\isaliteral{27}{\isacharprime}}a\ aexp{\isaliteral{22}{\isachardoublequoteclose}}\ {\isaliteral{22}{\isachardoublequoteopen}}{\isaliteral{27}{\isacharprime}}a\ aexp{\isaliteral{22}{\isachardoublequoteclose}}\isanewline
\ \ \ \ \ \ \ \ \ \ \ \ \ \ \ \ \ {\isaliteral{7C}{\isacharbar}}\ Sum\ \ {\isaliteral{22}{\isachardoublequoteopen}}{\isaliteral{27}{\isacharprime}}a\ aexp{\isaliteral{22}{\isachardoublequoteclose}}\ {\isaliteral{22}{\isachardoublequoteopen}}{\isaliteral{27}{\isacharprime}}a\ aexp{\isaliteral{22}{\isachardoublequoteclose}}\isanewline
\ \ \ \ \ \ \ \ \ \ \ \ \ \ \ \ \ {\isaliteral{7C}{\isacharbar}}\ Diff\ {\isaliteral{22}{\isachardoublequoteopen}}{\isaliteral{27}{\isacharprime}}a\ aexp{\isaliteral{22}{\isachardoublequoteclose}}\ {\isaliteral{22}{\isachardoublequoteopen}}{\isaliteral{27}{\isacharprime}}a\ aexp{\isaliteral{22}{\isachardoublequoteclose}}\isanewline
\ \ \ \ \ \ \ \ \ \ \ \ \ \ \ \ \ {\isaliteral{7C}{\isacharbar}}\ Var\ {\isaliteral{27}{\isacharprime}}a\isanewline
\ \ \ \ \ \ \ \ \ \ \ \ \ \ \ \ \ {\isaliteral{7C}{\isacharbar}}\ Num\ nat\isanewline
\isakeyword{and}\ \ \ \ \ \ {\isaliteral{27}{\isacharprime}}a\ bexp\ {\isaliteral{3D}{\isacharequal}}\ Less\ {\isaliteral{22}{\isachardoublequoteopen}}{\isaliteral{27}{\isacharprime}}a\ aexp{\isaliteral{22}{\isachardoublequoteclose}}\ {\isaliteral{22}{\isachardoublequoteopen}}{\isaliteral{27}{\isacharprime}}a\ aexp{\isaliteral{22}{\isachardoublequoteclose}}\isanewline
\ \ \ \ \ \ \ \ \ \ \ \ \ \ \ \ \ {\isaliteral{7C}{\isacharbar}}\ And\ \ {\isaliteral{22}{\isachardoublequoteopen}}{\isaliteral{27}{\isacharprime}}a\ bexp{\isaliteral{22}{\isachardoublequoteclose}}\ {\isaliteral{22}{\isachardoublequoteopen}}{\isaliteral{27}{\isacharprime}}a\ bexp{\isaliteral{22}{\isachardoublequoteclose}}\isanewline
\ \ \ \ \ \ \ \ \ \ \ \ \ \ \ \ \ {\isaliteral{7C}{\isacharbar}}\ Neg\ \ {\isaliteral{22}{\isachardoublequoteopen}}{\isaliteral{27}{\isacharprime}}a\ bexp{\isaliteral{22}{\isachardoublequoteclose}}%
\begin{isamarkuptext}%
\noindent
Type \isa{aexp} is similar to \isa{expr} in \S\ref{sec:ExprCompiler},
except that we have added an \isa{IF} constructor,
fixed the values to be of type \isa{nat} and declared the two binary
operations \isa{Sum} and \isa{Diff}.  Boolean
expressions can be arithmetic comparisons, conjunctions and negations.
The semantics is given by two evaluation functions:%
\end{isamarkuptext}%
\isamarkuptrue%
\isacommand{primrec}\isamarkupfalse%
\ evala\ {\isaliteral{3A}{\isacharcolon}}{\isaliteral{3A}{\isacharcolon}}\ {\isaliteral{22}{\isachardoublequoteopen}}{\isaliteral{27}{\isacharprime}}a\ aexp\ {\isaliteral{5C3C52696768746172726F773E}{\isasymRightarrow}}\ {\isaliteral{28}{\isacharparenleft}}{\isaliteral{27}{\isacharprime}}a\ {\isaliteral{5C3C52696768746172726F773E}{\isasymRightarrow}}\ nat{\isaliteral{29}{\isacharparenright}}\ {\isaliteral{5C3C52696768746172726F773E}{\isasymRightarrow}}\ nat{\isaliteral{22}{\isachardoublequoteclose}}\ \isakeyword{and}\isanewline
\ \ \ \ \ \ \ \ \ evalb\ {\isaliteral{3A}{\isacharcolon}}{\isaliteral{3A}{\isacharcolon}}\ {\isaliteral{22}{\isachardoublequoteopen}}{\isaliteral{27}{\isacharprime}}a\ bexp\ {\isaliteral{5C3C52696768746172726F773E}{\isasymRightarrow}}\ {\isaliteral{28}{\isacharparenleft}}{\isaliteral{27}{\isacharprime}}a\ {\isaliteral{5C3C52696768746172726F773E}{\isasymRightarrow}}\ nat{\isaliteral{29}{\isacharparenright}}\ {\isaliteral{5C3C52696768746172726F773E}{\isasymRightarrow}}\ bool{\isaliteral{22}{\isachardoublequoteclose}}\ \isakeyword{where}\isanewline
{\isaliteral{22}{\isachardoublequoteopen}}evala\ {\isaliteral{28}{\isacharparenleft}}IF\ b\ a{\isadigit{1}}\ a{\isadigit{2}}{\isaliteral{29}{\isacharparenright}}\ env\ {\isaliteral{3D}{\isacharequal}}\isanewline
\ \ \ {\isaliteral{28}{\isacharparenleft}}if\ evalb\ b\ env\ then\ evala\ a{\isadigit{1}}\ env\ else\ evala\ a{\isadigit{2}}\ env{\isaliteral{29}{\isacharparenright}}{\isaliteral{22}{\isachardoublequoteclose}}\ {\isaliteral{7C}{\isacharbar}}\isanewline
{\isaliteral{22}{\isachardoublequoteopen}}evala\ {\isaliteral{28}{\isacharparenleft}}Sum\ a{\isadigit{1}}\ a{\isadigit{2}}{\isaliteral{29}{\isacharparenright}}\ env\ {\isaliteral{3D}{\isacharequal}}\ evala\ a{\isadigit{1}}\ env\ {\isaliteral{2B}{\isacharplus}}\ evala\ a{\isadigit{2}}\ env{\isaliteral{22}{\isachardoublequoteclose}}\ {\isaliteral{7C}{\isacharbar}}\isanewline
{\isaliteral{22}{\isachardoublequoteopen}}evala\ {\isaliteral{28}{\isacharparenleft}}Diff\ a{\isadigit{1}}\ a{\isadigit{2}}{\isaliteral{29}{\isacharparenright}}\ env\ {\isaliteral{3D}{\isacharequal}}\ evala\ a{\isadigit{1}}\ env\ {\isaliteral{2D}{\isacharminus}}\ evala\ a{\isadigit{2}}\ env{\isaliteral{22}{\isachardoublequoteclose}}\ {\isaliteral{7C}{\isacharbar}}\isanewline
{\isaliteral{22}{\isachardoublequoteopen}}evala\ {\isaliteral{28}{\isacharparenleft}}Var\ v{\isaliteral{29}{\isacharparenright}}\ env\ {\isaliteral{3D}{\isacharequal}}\ env\ v{\isaliteral{22}{\isachardoublequoteclose}}\ {\isaliteral{7C}{\isacharbar}}\isanewline
{\isaliteral{22}{\isachardoublequoteopen}}evala\ {\isaliteral{28}{\isacharparenleft}}Num\ n{\isaliteral{29}{\isacharparenright}}\ env\ {\isaliteral{3D}{\isacharequal}}\ n{\isaliteral{22}{\isachardoublequoteclose}}\ {\isaliteral{7C}{\isacharbar}}\isanewline
\isanewline
{\isaliteral{22}{\isachardoublequoteopen}}evalb\ {\isaliteral{28}{\isacharparenleft}}Less\ a{\isadigit{1}}\ a{\isadigit{2}}{\isaliteral{29}{\isacharparenright}}\ env\ {\isaliteral{3D}{\isacharequal}}\ {\isaliteral{28}{\isacharparenleft}}evala\ a{\isadigit{1}}\ env\ {\isaliteral{3C}{\isacharless}}\ evala\ a{\isadigit{2}}\ env{\isaliteral{29}{\isacharparenright}}{\isaliteral{22}{\isachardoublequoteclose}}\ {\isaliteral{7C}{\isacharbar}}\isanewline
{\isaliteral{22}{\isachardoublequoteopen}}evalb\ {\isaliteral{28}{\isacharparenleft}}And\ b{\isadigit{1}}\ b{\isadigit{2}}{\isaliteral{29}{\isacharparenright}}\ env\ {\isaliteral{3D}{\isacharequal}}\ {\isaliteral{28}{\isacharparenleft}}evalb\ b{\isadigit{1}}\ env\ {\isaliteral{5C3C616E643E}{\isasymand}}\ evalb\ b{\isadigit{2}}\ env{\isaliteral{29}{\isacharparenright}}{\isaliteral{22}{\isachardoublequoteclose}}\ {\isaliteral{7C}{\isacharbar}}\isanewline
{\isaliteral{22}{\isachardoublequoteopen}}evalb\ {\isaliteral{28}{\isacharparenleft}}Neg\ b{\isaliteral{29}{\isacharparenright}}\ env\ {\isaliteral{3D}{\isacharequal}}\ {\isaliteral{28}{\isacharparenleft}}{\isaliteral{5C3C6E6F743E}{\isasymnot}}\ evalb\ b\ env{\isaliteral{29}{\isacharparenright}}{\isaliteral{22}{\isachardoublequoteclose}}%
\begin{isamarkuptext}%
\noindent

Both take an expression and an environment (a mapping from variables
\isa{{\isaliteral{27}{\isacharprime}}a} to values \isa{nat}) and return its arithmetic/boolean
value. Since the datatypes are mutually recursive, so are functions
that operate on them. Hence they need to be defined in a single
\isacommand{primrec} section. Notice the \isakeyword{and} separating
the declarations of \isa{evala} and \isa{evalb}. Their defining
equations need not be split into two groups;
the empty line is purely for readability.

In the same fashion we also define two functions that perform substitution:%
\end{isamarkuptext}%
\isamarkuptrue%
\isacommand{primrec}\isamarkupfalse%
\ substa\ {\isaliteral{3A}{\isacharcolon}}{\isaliteral{3A}{\isacharcolon}}\ {\isaliteral{22}{\isachardoublequoteopen}}{\isaliteral{28}{\isacharparenleft}}{\isaliteral{27}{\isacharprime}}a\ {\isaliteral{5C3C52696768746172726F773E}{\isasymRightarrow}}\ {\isaliteral{27}{\isacharprime}}b\ aexp{\isaliteral{29}{\isacharparenright}}\ {\isaliteral{5C3C52696768746172726F773E}{\isasymRightarrow}}\ {\isaliteral{27}{\isacharprime}}a\ aexp\ {\isaliteral{5C3C52696768746172726F773E}{\isasymRightarrow}}\ {\isaliteral{27}{\isacharprime}}b\ aexp{\isaliteral{22}{\isachardoublequoteclose}}\ \isakeyword{and}\isanewline
\ \ \ \ \ \ \ \ \ substb\ {\isaliteral{3A}{\isacharcolon}}{\isaliteral{3A}{\isacharcolon}}\ {\isaliteral{22}{\isachardoublequoteopen}}{\isaliteral{28}{\isacharparenleft}}{\isaliteral{27}{\isacharprime}}a\ {\isaliteral{5C3C52696768746172726F773E}{\isasymRightarrow}}\ {\isaliteral{27}{\isacharprime}}b\ aexp{\isaliteral{29}{\isacharparenright}}\ {\isaliteral{5C3C52696768746172726F773E}{\isasymRightarrow}}\ {\isaliteral{27}{\isacharprime}}a\ bexp\ {\isaliteral{5C3C52696768746172726F773E}{\isasymRightarrow}}\ {\isaliteral{27}{\isacharprime}}b\ bexp{\isaliteral{22}{\isachardoublequoteclose}}\ \isakeyword{where}\isanewline
{\isaliteral{22}{\isachardoublequoteopen}}substa\ s\ {\isaliteral{28}{\isacharparenleft}}IF\ b\ a{\isadigit{1}}\ a{\isadigit{2}}{\isaliteral{29}{\isacharparenright}}\ {\isaliteral{3D}{\isacharequal}}\isanewline
\ \ \ IF\ {\isaliteral{28}{\isacharparenleft}}substb\ s\ b{\isaliteral{29}{\isacharparenright}}\ {\isaliteral{28}{\isacharparenleft}}substa\ s\ a{\isadigit{1}}{\isaliteral{29}{\isacharparenright}}\ {\isaliteral{28}{\isacharparenleft}}substa\ s\ a{\isadigit{2}}{\isaliteral{29}{\isacharparenright}}{\isaliteral{22}{\isachardoublequoteclose}}\ {\isaliteral{7C}{\isacharbar}}\isanewline
{\isaliteral{22}{\isachardoublequoteopen}}substa\ s\ {\isaliteral{28}{\isacharparenleft}}Sum\ a{\isadigit{1}}\ a{\isadigit{2}}{\isaliteral{29}{\isacharparenright}}\ {\isaliteral{3D}{\isacharequal}}\ Sum\ {\isaliteral{28}{\isacharparenleft}}substa\ s\ a{\isadigit{1}}{\isaliteral{29}{\isacharparenright}}\ {\isaliteral{28}{\isacharparenleft}}substa\ s\ a{\isadigit{2}}{\isaliteral{29}{\isacharparenright}}{\isaliteral{22}{\isachardoublequoteclose}}\ {\isaliteral{7C}{\isacharbar}}\isanewline
{\isaliteral{22}{\isachardoublequoteopen}}substa\ s\ {\isaliteral{28}{\isacharparenleft}}Diff\ a{\isadigit{1}}\ a{\isadigit{2}}{\isaliteral{29}{\isacharparenright}}\ {\isaliteral{3D}{\isacharequal}}\ Diff\ {\isaliteral{28}{\isacharparenleft}}substa\ s\ a{\isadigit{1}}{\isaliteral{29}{\isacharparenright}}\ {\isaliteral{28}{\isacharparenleft}}substa\ s\ a{\isadigit{2}}{\isaliteral{29}{\isacharparenright}}{\isaliteral{22}{\isachardoublequoteclose}}\ {\isaliteral{7C}{\isacharbar}}\isanewline
{\isaliteral{22}{\isachardoublequoteopen}}substa\ s\ {\isaliteral{28}{\isacharparenleft}}Var\ v{\isaliteral{29}{\isacharparenright}}\ {\isaliteral{3D}{\isacharequal}}\ s\ v{\isaliteral{22}{\isachardoublequoteclose}}\ {\isaliteral{7C}{\isacharbar}}\isanewline
{\isaliteral{22}{\isachardoublequoteopen}}substa\ s\ {\isaliteral{28}{\isacharparenleft}}Num\ n{\isaliteral{29}{\isacharparenright}}\ {\isaliteral{3D}{\isacharequal}}\ Num\ n{\isaliteral{22}{\isachardoublequoteclose}}\ {\isaliteral{7C}{\isacharbar}}\isanewline
\isanewline
{\isaliteral{22}{\isachardoublequoteopen}}substb\ s\ {\isaliteral{28}{\isacharparenleft}}Less\ a{\isadigit{1}}\ a{\isadigit{2}}{\isaliteral{29}{\isacharparenright}}\ {\isaliteral{3D}{\isacharequal}}\ Less\ {\isaliteral{28}{\isacharparenleft}}substa\ s\ a{\isadigit{1}}{\isaliteral{29}{\isacharparenright}}\ {\isaliteral{28}{\isacharparenleft}}substa\ s\ a{\isadigit{2}}{\isaliteral{29}{\isacharparenright}}{\isaliteral{22}{\isachardoublequoteclose}}\ {\isaliteral{7C}{\isacharbar}}\isanewline
{\isaliteral{22}{\isachardoublequoteopen}}substb\ s\ {\isaliteral{28}{\isacharparenleft}}And\ b{\isadigit{1}}\ b{\isadigit{2}}{\isaliteral{29}{\isacharparenright}}\ {\isaliteral{3D}{\isacharequal}}\ And\ {\isaliteral{28}{\isacharparenleft}}substb\ s\ b{\isadigit{1}}{\isaliteral{29}{\isacharparenright}}\ {\isaliteral{28}{\isacharparenleft}}substb\ s\ b{\isadigit{2}}{\isaliteral{29}{\isacharparenright}}{\isaliteral{22}{\isachardoublequoteclose}}\ {\isaliteral{7C}{\isacharbar}}\isanewline
{\isaliteral{22}{\isachardoublequoteopen}}substb\ s\ {\isaliteral{28}{\isacharparenleft}}Neg\ b{\isaliteral{29}{\isacharparenright}}\ {\isaliteral{3D}{\isacharequal}}\ Neg\ {\isaliteral{28}{\isacharparenleft}}substb\ s\ b{\isaliteral{29}{\isacharparenright}}{\isaliteral{22}{\isachardoublequoteclose}}%
\begin{isamarkuptext}%
\noindent
Their first argument is a function mapping variables to expressions, the
substitution. It is applied to all variables in the second argument. As a
result, the type of variables in the expression may change from \isa{{\isaliteral{27}{\isacharprime}}a}
to \isa{{\isaliteral{27}{\isacharprime}}b}. Note that there are only arithmetic and no boolean variables.

Now we can prove a fundamental theorem about the interaction between
evaluation and substitution: applying a substitution $s$ to an expression $a$
and evaluating the result in an environment $env$ yields the same result as
evaluation $a$ in the environment that maps every variable $x$ to the value
of $s(x)$ under $env$. If you try to prove this separately for arithmetic or
boolean expressions (by induction), you find that you always need the other
theorem in the induction step. Therefore you need to state and prove both
theorems simultaneously:%
\end{isamarkuptext}%
\isamarkuptrue%
\isacommand{lemma}\isamarkupfalse%
\ {\isaliteral{22}{\isachardoublequoteopen}}evala\ {\isaliteral{28}{\isacharparenleft}}substa\ s\ a{\isaliteral{29}{\isacharparenright}}\ env\ {\isaliteral{3D}{\isacharequal}}\ evala\ a\ {\isaliteral{28}{\isacharparenleft}}{\isaliteral{5C3C6C616D6264613E}{\isasymlambda}}x{\isaliteral{2E}{\isachardot}}\ evala\ {\isaliteral{28}{\isacharparenleft}}s\ x{\isaliteral{29}{\isacharparenright}}\ env{\isaliteral{29}{\isacharparenright}}\ {\isaliteral{5C3C616E643E}{\isasymand}}\isanewline
\ \ \ \ \ \ \ \ evalb\ {\isaliteral{28}{\isacharparenleft}}substb\ s\ b{\isaliteral{29}{\isacharparenright}}\ env\ {\isaliteral{3D}{\isacharequal}}\ evalb\ b\ {\isaliteral{28}{\isacharparenleft}}{\isaliteral{5C3C6C616D6264613E}{\isasymlambda}}x{\isaliteral{2E}{\isachardot}}\ evala\ {\isaliteral{28}{\isacharparenleft}}s\ x{\isaliteral{29}{\isacharparenright}}\ env{\isaliteral{29}{\isacharparenright}}{\isaliteral{22}{\isachardoublequoteclose}}\isanewline
%
\isadelimproof
%
\endisadelimproof
%
\isatagproof
\isacommand{apply}\isamarkupfalse%
{\isaliteral{28}{\isacharparenleft}}induct{\isaliteral{5F}{\isacharunderscore}}tac\ a\ \isakeyword{and}\ b{\isaliteral{29}{\isacharparenright}}%
\begin{isamarkuptxt}%
\noindent The resulting 8 goals (one for each constructor) are proved in one fell swoop:%
\end{isamarkuptxt}%
\isamarkuptrue%
\isacommand{apply}\isamarkupfalse%
\ simp{\isaliteral{5F}{\isacharunderscore}}all%
\endisatagproof
{\isafoldproof}%
%
\isadelimproof
%
\endisadelimproof
%
\begin{isamarkuptext}%
In general, given $n$ mutually recursive datatypes $\tau@1$, \dots, $\tau@n$,
an inductive proof expects a goal of the form
\[ P@1(x@1)\ \land \dots \land P@n(x@n) \]
where each variable $x@i$ is of type $\tau@i$. Induction is started by
\begin{isabelle}
\isacommand{apply}\isa{{\isaliteral{28}{\isacharparenleft}}induct{\isaliteral{5F}{\isacharunderscore}}tac} $x@1$ \isacommand{and} \dots\ \isacommand{and} $x@n$\isa{{\isaliteral{29}{\isacharparenright}}}
\end{isabelle}

\begin{exercise}
  Define a function \isa{norma} of type \isa{{\isaliteral{27}{\isacharprime}}a\ aexp\ {\isaliteral{5C3C52696768746172726F773E}{\isasymRightarrow}}\ {\isaliteral{27}{\isacharprime}}a\ aexp} that
  replaces \isa{IF}s with complex boolean conditions by nested
  \isa{IF}s; it should eliminate the constructors
  \isa{And} and \isa{Neg}, leaving only \isa{Less}.
  Prove that \isa{norma}
  preserves the value of an expression and that the result of \isa{norma}
  is really normal, i.e.\ no more \isa{And}s and \isa{Neg}s occur in
  it.  ({\em Hint:} proceed as in \S\ref{sec:boolex} and read the discussion
  of type annotations following lemma \isa{subst{\isaliteral{5F}{\isacharunderscore}}id} below).
\end{exercise}%
\end{isamarkuptext}%
\isamarkuptrue%
%
\isadelimproof
%
\endisadelimproof
%
\isatagproof
%
\endisatagproof
{\isafoldproof}%
%
\isadelimproof
%
\endisadelimproof
%
\isadelimproof
%
\endisadelimproof
%
\isatagproof
%
\endisatagproof
{\isafoldproof}%
%
\isadelimproof
%
\endisadelimproof
%
\isadelimtheory
%
\endisadelimtheory
%
\isatagtheory
%
\endisatagtheory
{\isafoldtheory}%
%
\isadelimtheory
%
\endisadelimtheory
\end{isabellebody}%
%%% Local Variables:
%%% mode: latex
%%% TeX-master: "root"
%%% End:


\subsection{Nested Recursion}
\label{sec:nested-datatype}

{\makeatother%
\begin{isabellebody}%
\def\isabellecontext{Nested}%
%
\begin{isamarkuptext}%
So far, all datatypes had the property that on the right-hand side of their
definition they occurred only at the top-level, i.e.\ directly below a
constructor. This is not the case any longer for the following model of terms
where function symbols can be applied to a list of arguments:%
\end{isamarkuptext}%
\isacommand{datatype}\ {\isacharparenleft}{\isacharprime}a{\isacharcomma}{\isacharprime}b{\isacharparenright}{\isachardoublequote}term{\isachardoublequote}\ {\isacharequal}\ Var\ {\isacharprime}a\ {\isacharbar}\ App\ {\isacharprime}b\ {\isachardoublequote}{\isacharparenleft}{\isacharprime}a{\isacharcomma}{\isacharprime}b{\isacharparenright}term\ list{\isachardoublequote}%
\begin{isamarkuptext}%
\noindent
Note that we need to quote \isa{term} on the left to avoid confusion with
the Isabelle command \isacommand{term}.
Parameter \isa{{\isacharprime}a} is the type of variables and \isa{{\isacharprime}b} the type of
function symbols.
A mathematical term like $f(x,g(y))$ becomes \isa{App\ f\ {\isacharbrackleft}Var\ x{\isacharcomma}\ App\ g\ {\isacharbrackleft}Var\ y{\isacharbrackright}{\isacharbrackright}}, where \isa{f}, \isa{g}, \isa{x}, \isa{y} are
suitable values, e.g.\ numbers or strings.

What complicates the definition of \isa{term} is the nested occurrence of
\isa{term} inside \isa{list} on the right-hand side. In principle,
nested recursion can be eliminated in favour of mutual recursion by unfolding
the offending datatypes, here \isa{list}. The result for \isa{term}
would be something like
\medskip

%
\begin{isabellebody}%
\def\isabellecontext{unfoldnested}%
%
\isadelimtheory
%
\endisadelimtheory
%
\isatagtheory
%
\endisatagtheory
{\isafoldtheory}%
%
\isadelimtheory
%
\endisadelimtheory
\isacommand{datatype}\isamarkupfalse%
\ {\isaliteral{28}{\isacharparenleft}}{\isaliteral{27}{\isacharprime}}v{\isaliteral{2C}{\isacharcomma}}{\isaliteral{27}{\isacharprime}}f{\isaliteral{29}{\isacharparenright}}{\isaliteral{22}{\isachardoublequoteopen}}term{\isaliteral{22}{\isachardoublequoteclose}}\ {\isaliteral{3D}{\isacharequal}}\ Var\ {\isaliteral{27}{\isacharprime}}v\ {\isaliteral{7C}{\isacharbar}}\ App\ {\isaliteral{27}{\isacharprime}}f\ {\isaliteral{22}{\isachardoublequoteopen}}{\isaliteral{28}{\isacharparenleft}}{\isaliteral{27}{\isacharprime}}v{\isaliteral{2C}{\isacharcomma}}{\isaliteral{27}{\isacharprime}}f{\isaliteral{29}{\isacharparenright}}term{\isaliteral{5F}{\isacharunderscore}}list{\isaliteral{22}{\isachardoublequoteclose}}\isanewline
\isakeyword{and}\ {\isaliteral{28}{\isacharparenleft}}{\isaliteral{27}{\isacharprime}}v{\isaliteral{2C}{\isacharcomma}}{\isaliteral{27}{\isacharprime}}f{\isaliteral{29}{\isacharparenright}}term{\isaliteral{5F}{\isacharunderscore}}list\ {\isaliteral{3D}{\isacharequal}}\ Nil\ {\isaliteral{7C}{\isacharbar}}\ Cons\ {\isaliteral{22}{\isachardoublequoteopen}}{\isaliteral{28}{\isacharparenleft}}{\isaliteral{27}{\isacharprime}}v{\isaliteral{2C}{\isacharcomma}}{\isaliteral{27}{\isacharprime}}f{\isaliteral{29}{\isacharparenright}}term{\isaliteral{22}{\isachardoublequoteclose}}\ {\isaliteral{22}{\isachardoublequoteopen}}{\isaliteral{28}{\isacharparenleft}}{\isaliteral{27}{\isacharprime}}v{\isaliteral{2C}{\isacharcomma}}{\isaliteral{27}{\isacharprime}}f{\isaliteral{29}{\isacharparenright}}term{\isaliteral{5F}{\isacharunderscore}}list{\isaliteral{22}{\isachardoublequoteclose}}%
\isadelimtheory
%
\endisadelimtheory
%
\isatagtheory
%
\endisatagtheory
{\isafoldtheory}%
%
\isadelimtheory
%
\endisadelimtheory
\end{isabellebody}%
%%% Local Variables:
%%% mode: latex
%%% TeX-master: "root"
%%% End:

\medskip

\noindent
Although we do not recommend this unfolding to the user, it shows how to
simulate nested recursion by mutual recursion.
Now we return to the initial definition of \isa{term} using
nested recursion.

Let us define a substitution function on terms. Because terms involve term
lists, we need to define two substitution functions simultaneously:%
\end{isamarkuptext}%
\isacommand{consts}\isanewline
subst\ {\isacharcolon}{\isacharcolon}\ {\isachardoublequote}{\isacharparenleft}{\isacharprime}a{\isasymRightarrow}{\isacharparenleft}{\isacharprime}a{\isacharcomma}{\isacharprime}b{\isacharparenright}term{\isacharparenright}\ {\isasymRightarrow}\ {\isacharparenleft}{\isacharprime}a{\isacharcomma}{\isacharprime}b{\isacharparenright}term\ \ \ \ \ \ {\isasymRightarrow}\ {\isacharparenleft}{\isacharprime}a{\isacharcomma}{\isacharprime}b{\isacharparenright}term{\isachardoublequote}\isanewline
substs{\isacharcolon}{\isacharcolon}\ {\isachardoublequote}{\isacharparenleft}{\isacharprime}a{\isasymRightarrow}{\isacharparenleft}{\isacharprime}a{\isacharcomma}{\isacharprime}b{\isacharparenright}term{\isacharparenright}\ {\isasymRightarrow}\ {\isacharparenleft}{\isacharprime}a{\isacharcomma}{\isacharprime}b{\isacharparenright}term\ list\ {\isasymRightarrow}\ {\isacharparenleft}{\isacharprime}a{\isacharcomma}{\isacharprime}b{\isacharparenright}term\ list{\isachardoublequote}\isanewline
\isanewline
\isacommand{primrec}\isanewline
\ \ {\isachardoublequote}subst\ s\ {\isacharparenleft}Var\ x{\isacharparenright}\ {\isacharequal}\ s\ x{\isachardoublequote}\isanewline
\ \ subst{\isacharunderscore}App{\isacharcolon}\isanewline
\ \ {\isachardoublequote}subst\ s\ {\isacharparenleft}App\ f\ ts{\isacharparenright}\ {\isacharequal}\ App\ f\ {\isacharparenleft}substs\ s\ ts{\isacharparenright}{\isachardoublequote}\isanewline
\isanewline
\ \ {\isachardoublequote}substs\ s\ {\isacharbrackleft}{\isacharbrackright}\ {\isacharequal}\ {\isacharbrackleft}{\isacharbrackright}{\isachardoublequote}\isanewline
\ \ {\isachardoublequote}substs\ s\ {\isacharparenleft}t\ {\isacharhash}\ ts{\isacharparenright}\ {\isacharequal}\ subst\ s\ t\ {\isacharhash}\ substs\ s\ ts{\isachardoublequote}%
\begin{isamarkuptext}%
\noindent
Individual equations in a primrec definition may be named as shown for \isa{subst{\isacharunderscore}App}.
The significance of this device will become apparent below.

Similarly, when proving a statement about terms inductively, we need
to prove a related statement about term lists simultaneously. For example,
the fact that the identity substitution does not change a term needs to be
strengthened and proved as follows:%
\end{isamarkuptext}%
\isacommand{lemma}\ {\isachardoublequote}subst\ \ Var\ t\ \ {\isacharequal}\ {\isacharparenleft}t\ {\isacharcolon}{\isacharcolon}{\isacharparenleft}{\isacharprime}a{\isacharcomma}{\isacharprime}b{\isacharparenright}term{\isacharparenright}\ \ {\isasymand}\isanewline
\ \ \ \ \ \ \ \ substs\ Var\ ts\ {\isacharequal}\ {\isacharparenleft}ts{\isacharcolon}{\isacharcolon}{\isacharparenleft}{\isacharprime}a{\isacharcomma}{\isacharprime}b{\isacharparenright}term\ list{\isacharparenright}{\isachardoublequote}\isanewline
\isacommand{apply}{\isacharparenleft}induct{\isacharunderscore}tac\ t\ \isakeyword{and}\ ts{\isacharcomma}\ simp{\isacharunderscore}all{\isacharparenright}\isanewline
\isacommand{done}%
\begin{isamarkuptext}%
\noindent
Note that \isa{Var} is the identity substitution because by definition it
leaves variables unchanged: \isa{subst\ Var\ {\isacharparenleft}Var\ x{\isacharparenright}\ {\isacharequal}\ Var\ x}. Note also
that the type annotations are necessary because otherwise there is nothing in
the goal to enforce that both halves of the goal talk about the same type
parameters \isa{{\isacharparenleft}{\isacharprime}a{\isacharcomma}{\isacharprime}b{\isacharparenright}}. As a result, induction would fail
because the two halves of the goal would be unrelated.

\begin{exercise}
The fact that substitution distributes over composition can be expressed
roughly as follows:
\begin{isabelle}%
\ \ \ \ \ subst\ {\isacharparenleft}f\ {\isasymcirc}\ g{\isacharparenright}\ t\ {\isacharequal}\ subst\ f\ {\isacharparenleft}subst\ g\ t{\isacharparenright}%
\end{isabelle}
Correct this statement (you will find that it does not type-check),
strengthen it, and prove it. (Note: \isa{{\isasymcirc}} is function composition;
its definition is found in theorem \isa{o{\isacharunderscore}def}).
\end{exercise}
\begin{exercise}\label{ex:trev-trev}
  Define a function \isa{trev} of type \isa{{\isacharparenleft}{\isacharprime}a{\isacharcomma}\ {\isacharprime}b{\isacharparenright}\ term\ {\isasymRightarrow}\ {\isacharparenleft}{\isacharprime}a{\isacharcomma}\ {\isacharprime}b{\isacharparenright}\ term}
that recursively reverses the order of arguments of all function symbols in a
  term. Prove that \isa{trev\ {\isacharparenleft}trev\ t{\isacharparenright}\ {\isacharequal}\ t}.
\end{exercise}

The experienced functional programmer may feel that our above definition of
\isa{subst} is unnecessarily complicated in that \isa{substs} is
completely unnecessary. The \isa{App}-case can be defined directly as
\begin{isabelle}%
\ \ \ \ \ subst\ s\ {\isacharparenleft}App\ f\ ts{\isacharparenright}\ {\isacharequal}\ App\ f\ {\isacharparenleft}map\ {\isacharparenleft}subst\ s{\isacharparenright}\ ts{\isacharparenright}%
\end{isabelle}
where \isa{map} is the standard list function such that
\isa{map\ f\ {\isacharbrackleft}x\isadigit{1}{\isacharcomma}{\isachardot}{\isachardot}{\isachardot}{\isacharcomma}xn{\isacharbrackright}\ {\isacharequal}\ {\isacharbrackleft}f\ x\isadigit{1}{\isacharcomma}{\isachardot}{\isachardot}{\isachardot}{\isacharcomma}f\ xn{\isacharbrackright}}. This is true, but Isabelle
insists on the above fixed format. Fortunately, we can easily \emph{prove}
that the suggested equation holds:%
\end{isamarkuptext}%
\isacommand{lemma}\ {\isacharbrackleft}simp{\isacharbrackright}{\isacharcolon}\ {\isachardoublequote}subst\ s\ {\isacharparenleft}App\ f\ ts{\isacharparenright}\ {\isacharequal}\ App\ f\ {\isacharparenleft}map\ {\isacharparenleft}subst\ s{\isacharparenright}\ ts{\isacharparenright}{\isachardoublequote}\isanewline
\isacommand{apply}{\isacharparenleft}induct{\isacharunderscore}tac\ ts{\isacharcomma}\ simp{\isacharunderscore}all{\isacharparenright}\isanewline
\isacommand{done}%
\begin{isamarkuptext}%
\noindent
What is more, we can now disable the old defining equation as a
simplification rule:%
\end{isamarkuptext}%
\isacommand{declare}\ subst{\isacharunderscore}App\ {\isacharbrackleft}simp\ del{\isacharbrackright}%
\begin{isamarkuptext}%
\noindent
The advantage is that now we have replaced \isa{substs} by
\isa{map}, we can profit from the large number of pre-proved lemmas
about \isa{map}.  Unfortunately inductive proofs about type
\isa{term} are still awkward because they expect a conjunction. One
could derive a new induction principle as well (see
\S\ref{sec:derive-ind}), but turns out to be simpler to define
functions by \isacommand{recdef} instead of \isacommand{primrec}.
The details are explained in \S\ref{sec:advanced-recdef} below.

Of course, you may also combine mutual and nested recursion. For example,
constructor \isa{Sum} in \S\ref{sec:datatype-mut-rec} could take a list of
expressions as its argument: \isa{Sum}~\isa{{\isachardoublequote}{\isacharprime}a\ aexp\ list{\isachardoublequote}}.%
\end{isamarkuptext}%
\end{isabellebody}%
%%% Local Variables:
%%% mode: latex
%%% TeX-master: "root"
%%% End:
}


\subsection{The Limits of Nested Recursion}
\label{sec:nested-fun-datatype}

How far can we push nested recursion? By the unfolding argument above, we can
reduce nested to mutual recursion provided the nested recursion only involves
previously defined datatypes. This does not include functions:
\begin{isabelle}
\isacommand{datatype} t = C "t \isasymRightarrow\ bool"
\end{isabelle}
This declaration is a real can of worms.
In HOL it must be ruled out because it requires a type
\isa{t} such that \isa{t} and its power set \isa{t \isasymFun\ bool} have the
same cardinality --- an impossibility. For the same reason it is not possible
to allow recursion involving the type \isa{t set}, which is isomorphic to
\isa{t \isasymFun\ bool}.

Fortunately, a limited form of recursion
involving function spaces is permitted: the recursive type may occur on the
right of a function arrow, but never on the left. Hence the above can of worms
is ruled out but the following example of a potentially 
\index{infinitely branching trees}%
infinitely branching tree is accepted:
\smallskip

\begin{isabelle}%
\isacommand{datatype}\ {\isacharparenleft}{\isacharprime}a{\isacharcomma}{\isacharprime}i{\isacharparenright}bigtree\ {\isacharequal}\ Tip\ {\isacharbar}\ Branch\ {\isacharprime}a\ {\isachardoublequote}{\isacharprime}i\ {\isasymRightarrow}\ {\isacharparenleft}{\isacharprime}a{\isacharcomma}{\isacharprime}i{\isacharparenright}bigtree{\isachardoublequote}%
\begin{isamarkuptext}%
\noindent Parameter \isa{'a} is the type of values stored in
the \isa{Branch}es of the tree, whereas \isa{'i} is the index
type over which the tree branches. If \isa{'i} is instantiated to
\isa{bool}, the result is a binary tree; if it is instantiated to
\isa{nat}, we have an infinitely branching tree because each node
has as many subtrees as there are natural numbers. How can we possibly
write down such a tree? Using functional notation! For example, the term
\begin{quote}

\begin{isabelle}%
Branch\ \isadigit{0}\ {\isacharparenleft}{\isasymlambda}\mbox{i}{\isachardot}\ Branch\ \mbox{i}\ {\isacharparenleft}{\isasymlambda}\mbox{n}{\isachardot}\ Tip{\isacharparenright}{\isacharparenright}
\end{isabelle}%

\end{quote}
of type \isa{{\isacharparenleft}nat{\isacharcomma}\ nat{\isacharparenright}\ bigtree} is the tree whose
root is labeled with 0 and whose $i$th subtree is labeled with $i$ and
has merely \isa{Tip}s as further subtrees.

Function \isa{map_bt} applies a function to all labels in a \isa{bigtree}:%
\end{isamarkuptext}%
\isacommand{consts}\ map{\isacharunderscore}bt\ {\isacharcolon}{\isacharcolon}\ {\isachardoublequote}{\isacharparenleft}{\isacharprime}a\ {\isasymRightarrow}\ {\isacharprime}b{\isacharparenright}\ {\isasymRightarrow}\ {\isacharparenleft}{\isacharprime}a{\isacharcomma}{\isacharprime}i{\isacharparenright}bigtree\ {\isasymRightarrow}\ {\isacharparenleft}{\isacharprime}b{\isacharcomma}{\isacharprime}i{\isacharparenright}bigtree{\isachardoublequote}\isanewline
\isacommand{primrec}\isanewline
{\isachardoublequote}map{\isacharunderscore}bt\ f\ Tip\ \ \ \ \ \ \ \ \ \ {\isacharequal}\ Tip{\isachardoublequote}\isanewline
{\isachardoublequote}map{\isacharunderscore}bt\ f\ {\isacharparenleft}Branch\ a\ F{\isacharparenright}\ {\isacharequal}\ Branch\ {\isacharparenleft}f\ a{\isacharparenright}\ {\isacharparenleft}{\isasymlambda}i{\isachardot}\ map{\isacharunderscore}bt\ f\ {\isacharparenleft}F\ i{\isacharparenright}{\isacharparenright}{\isachardoublequote}%
\begin{isamarkuptext}%
\noindent This is a valid \isacommand{primrec} definition because the
recursive calls of \isa{map_bt} involve only subtrees obtained from
\isa{F}, i.e.\ the left-hand side. Thus termination is assured.  The
seasoned functional programmer might have written \isa{map{\isacharunderscore}bt\ \mbox{f}\ {\isasymcirc}\ \mbox{F}}
instead of \isa{{\isasymlambda}\mbox{i}{\isachardot}\ map{\isacharunderscore}bt\ \mbox{f}\ {\isacharparenleft}\mbox{F}\ \mbox{i}{\isacharparenright}}, but the former is not accepted by
Isabelle because the termination proof is not as obvious since
\isa{map_bt} is only partially applied.

The following lemma has a canonical proof%
\end{isamarkuptext}%
\isacommand{lemma}\ {\isachardoublequote}map{\isacharunderscore}bt\ {\isacharparenleft}g\ o\ f{\isacharparenright}\ T\ {\isacharequal}\ map{\isacharunderscore}bt\ g\ {\isacharparenleft}map{\isacharunderscore}bt\ f\ T{\isacharparenright}{\isachardoublequote}\isanewline
\isacommand{by}{\isacharparenleft}induct{\isacharunderscore}tac\ {\isachardoublequote}T{\isachardoublequote}{\isacharcomma}\ auto{\isacharparenright}%
\begin{isamarkuptext}%
\noindent
but it is worth taking a look at the proof state after the induction step
to understand what the presence of the function type entails:
\begin{isabellepar}%
~1.~map\_bt~g~(map\_bt~f~Tip)~=~map\_bt~(g~{\isasymcirc}~f)~Tip\isanewline
~2.~{\isasymAnd}a~F.\isanewline
~~~~~~{\isasymforall}x.~map\_bt~g~(map\_bt~f~(F~x))~=~map\_bt~(g~{\isasymcirc}~f)~(F~x)~{\isasymLongrightarrow}\isanewline
~~~~~~map\_bt~g~(map\_bt~f~(Branch~a~F))~=~map\_bt~(g~{\isasymcirc}~f)~(Branch~a~F)%
\end{isabellepar}%%
\end{isamarkuptext}%
\end{isabelle}%
%%% Local Variables:
%%% mode: latex
%%% TeX-master: "root"
%%% End:


If you need nested recursion on the left of a function arrow, there are
alternatives to pure HOL\@.  In the Logic for Computable Functions 
(\rmindex{LCF}), types like
\begin{isabelle}
\isacommand{datatype} lam = C "lam \isasymrightarrow\ lam"
\end{isabelle}
do indeed make sense~\cite{paulson87}.  Note the different arrow,
\isa{\isasymrightarrow} instead of \isa{\isasymRightarrow},
expressing the type of \emph{continuous} functions. 
There is even a version of LCF on top of HOL,
called \rmindex{HOLCF}~\cite{MuellerNvOS99}.
\index{datatype@\isacommand {datatype} (command)|)}
\index{primrec@\isacommand {primrec} (command)|)}


\subsection{Case Study: Tries}
\label{sec:Trie}

\index{tries|(}%
Tries are a classic search tree data structure~\cite{Knuth3-75} for fast
indexing with strings. Figure~\ref{fig:trie} gives a graphical example of a
trie containing the words ``all'', ``an'', ``ape'', ``can'', ``car'' and
``cat''.  When searching a string in a trie, the letters of the string are
examined sequentially. Each letter determines which subtrie to search next.
In this case study we model tries as a datatype, define a lookup and an
update function, and prove that they behave as expected.

\begin{figure}[htbp]
\begin{center}
\unitlength1mm
\begin{picture}(60,30)
\put( 5, 0){\makebox(0,0)[b]{l}}
\put(25, 0){\makebox(0,0)[b]{e}}
\put(35, 0){\makebox(0,0)[b]{n}}
\put(45, 0){\makebox(0,0)[b]{r}}
\put(55, 0){\makebox(0,0)[b]{t}}
%
\put( 5, 9){\line(0,-1){5}}
\put(25, 9){\line(0,-1){5}}
\put(44, 9){\line(-3,-2){9}}
\put(45, 9){\line(0,-1){5}}
\put(46, 9){\line(3,-2){9}}
%
\put( 5,10){\makebox(0,0)[b]{l}}
\put(15,10){\makebox(0,0)[b]{n}}
\put(25,10){\makebox(0,0)[b]{p}}
\put(45,10){\makebox(0,0)[b]{a}}
%
\put(14,19){\line(-3,-2){9}}
\put(15,19){\line(0,-1){5}}
\put(16,19){\line(3,-2){9}}
\put(45,19){\line(0,-1){5}}
%
\put(15,20){\makebox(0,0)[b]{a}}
\put(45,20){\makebox(0,0)[b]{c}}
%
\put(30,30){\line(-3,-2){13}}
\put(30,30){\line(3,-2){13}}
\end{picture}
\end{center}
\caption{A Sample Trie}
\label{fig:trie}
\end{figure}

Proper tries associate some value with each string. Since the
information is stored only in the final node associated with the string, many
nodes do not carry any value. This distinction is modeled with the help
of the predefined datatype \isa{option} (see {\S}\ref{sec:option}).
%
\begin{isabellebody}%
%
\begin{isamarkuptext}%
To minimize running time, each node of a trie should contain an array that maps
letters to subtries. We have chosen a (sometimes) more space efficient
representation where the subtries are held in an association list, i.e.\ a
list of (letter,trie) pairs.  Abstracting over the alphabet \isa{'a} and the
values \isa{'v} we define a trie as follows:%
\end{isamarkuptext}%
\isacommand{datatype}\ {\isacharparenleft}{\isacharprime}a{\isacharcomma}{\isacharprime}v{\isacharparenright}trie\ {\isacharequal}\ Trie\ \ {\isachardoublequote}{\isacharprime}v\ option{\isachardoublequote}\ \ {\isachardoublequote}{\isacharparenleft}{\isacharprime}a\ {\isacharasterisk}\ {\isacharparenleft}{\isacharprime}a{\isacharcomma}{\isacharprime}v{\isacharparenright}trie{\isacharparenright}list{\isachardoublequote}%
\begin{isamarkuptext}%
\noindent
The first component is the optional value, the second component the
association list of subtries.  This is an example of nested recursion involving products,
which is fine because products are datatypes as well.
We define two selector functions:%
\end{isamarkuptext}%
\isacommand{consts}\ value\ {\isacharcolon}{\isacharcolon}\ {\isachardoublequote}{\isacharparenleft}{\isacharprime}a{\isacharcomma}{\isacharprime}v{\isacharparenright}trie\ {\isasymRightarrow}\ {\isacharprime}v\ option{\isachardoublequote}\isanewline
\ \ \ \ \ \ \ alist\ {\isacharcolon}{\isacharcolon}\ {\isachardoublequote}{\isacharparenleft}{\isacharprime}a{\isacharcomma}{\isacharprime}v{\isacharparenright}trie\ {\isasymRightarrow}\ {\isacharparenleft}{\isacharprime}a\ {\isacharasterisk}\ {\isacharparenleft}{\isacharprime}a{\isacharcomma}{\isacharprime}v{\isacharparenright}trie{\isacharparenright}list{\isachardoublequote}\isanewline
\isacommand{primrec}\ {\isachardoublequote}value{\isacharparenleft}Trie\ ov\ al{\isacharparenright}\ {\isacharequal}\ ov{\isachardoublequote}\isanewline
\isacommand{primrec}\ {\isachardoublequote}alist{\isacharparenleft}Trie\ ov\ al{\isacharparenright}\ {\isacharequal}\ al{\isachardoublequote}%
\begin{isamarkuptext}%
\noindent
Association lists come with a generic lookup function:%
\end{isamarkuptext}%
\isacommand{consts}\ \ \ assoc\ {\isacharcolon}{\isacharcolon}\ {\isachardoublequote}{\isacharparenleft}{\isacharprime}key\ {\isacharasterisk}\ {\isacharprime}val{\isacharparenright}list\ {\isasymRightarrow}\ {\isacharprime}key\ {\isasymRightarrow}\ {\isacharprime}val\ option{\isachardoublequote}\isanewline
\isacommand{primrec}\ {\isachardoublequote}assoc\ {\isacharbrackleft}{\isacharbrackright}\ x\ {\isacharequal}\ None{\isachardoublequote}\isanewline
\ \ \ \ \ \ \ \ {\isachardoublequote}assoc\ {\isacharparenleft}p{\isacharhash}ps{\isacharparenright}\ x\ {\isacharequal}\isanewline
\ \ \ \ \ \ \ \ \ \ \ {\isacharparenleft}let\ {\isacharparenleft}a{\isacharcomma}b{\isacharparenright}\ {\isacharequal}\ p\ in\ if\ a{\isacharequal}x\ then\ Some\ b\ else\ assoc\ ps\ x{\isacharparenright}{\isachardoublequote}%
\begin{isamarkuptext}%
Now we can define the lookup function for tries. It descends into the trie
examining the letters of the search string one by one. As
recursion on lists is simpler than on tries, let us express this as primitive
recursion on the search string argument:%
\end{isamarkuptext}%
\isacommand{consts}\ \ \ lookup\ {\isacharcolon}{\isacharcolon}\ {\isachardoublequote}{\isacharparenleft}{\isacharprime}a{\isacharcomma}{\isacharprime}v{\isacharparenright}trie\ {\isasymRightarrow}\ {\isacharprime}a\ list\ {\isasymRightarrow}\ {\isacharprime}v\ option{\isachardoublequote}\isanewline
\isacommand{primrec}\ {\isachardoublequote}lookup\ t\ {\isacharbrackleft}{\isacharbrackright}\ {\isacharequal}\ value\ t{\isachardoublequote}\isanewline
\ \ \ \ \ \ \ \ {\isachardoublequote}lookup\ t\ {\isacharparenleft}a{\isacharhash}as{\isacharparenright}\ {\isacharequal}\ {\isacharparenleft}case\ assoc\ {\isacharparenleft}alist\ t{\isacharparenright}\ a\ of\isanewline
\ \ \ \ \ \ \ \ \ \ \ \ \ \ \ \ \ \ \ \ \ \ \ \ \ \ \ \ \ \ None\ {\isasymRightarrow}\ None\isanewline
\ \ \ \ \ \ \ \ \ \ \ \ \ \ \ \ \ \ \ \ \ \ \ \ \ \ \ \ {\isacharbar}\ Some\ at\ {\isasymRightarrow}\ lookup\ at\ as{\isacharparenright}{\isachardoublequote}%
\begin{isamarkuptext}%
As a first simple property we prove that looking up a string in the empty
trie \isa{Trie~None~[]} always returns \isa{None}. The proof merely
distinguishes the two cases whether the search string is empty or not:%
\end{isamarkuptext}%
\isacommand{lemma}\ {\isacharbrackleft}simp{\isacharbrackright}{\isacharcolon}\ {\isachardoublequote}lookup\ {\isacharparenleft}Trie\ None\ {\isacharbrackleft}{\isacharbrackright}{\isacharparenright}\ as\ {\isacharequal}\ None{\isachardoublequote}\isanewline
\isacommand{by}{\isacharparenleft}case{\isacharunderscore}tac\ as{\isacharcomma}\ simp{\isacharunderscore}all{\isacharparenright}%
\begin{isamarkuptext}%
Things begin to get interesting with the definition of an update function
that adds a new (string,value) pair to a trie, overwriting the old value
associated with that string:%
\end{isamarkuptext}%
\isacommand{consts}\ update\ {\isacharcolon}{\isacharcolon}\ {\isachardoublequote}{\isacharparenleft}{\isacharprime}a{\isacharcomma}{\isacharprime}v{\isacharparenright}trie\ {\isasymRightarrow}\ {\isacharprime}a\ list\ {\isasymRightarrow}\ {\isacharprime}v\ {\isasymRightarrow}\ {\isacharparenleft}{\isacharprime}a{\isacharcomma}{\isacharprime}v{\isacharparenright}trie{\isachardoublequote}\isanewline
\isacommand{primrec}\isanewline
\ \ {\isachardoublequote}update\ t\ {\isacharbrackleft}{\isacharbrackright}\ \ \ \ \ v\ {\isacharequal}\ Trie\ {\isacharparenleft}Some\ v{\isacharparenright}\ {\isacharparenleft}alist\ t{\isacharparenright}{\isachardoublequote}\isanewline
\ \ {\isachardoublequote}update\ t\ {\isacharparenleft}a{\isacharhash}as{\isacharparenright}\ v\ {\isacharequal}\isanewline
\ \ \ \ \ {\isacharparenleft}let\ tt\ {\isacharequal}\ {\isacharparenleft}case\ assoc\ {\isacharparenleft}alist\ t{\isacharparenright}\ a\ of\isanewline
\ \ \ \ \ \ \ \ \ \ \ \ \ \ \ \ \ \ None\ {\isasymRightarrow}\ Trie\ None\ {\isacharbrackleft}{\isacharbrackright}\ {\isacharbar}\ Some\ at\ {\isasymRightarrow}\ at{\isacharparenright}\isanewline
\ \ \ \ \ \ in\ Trie\ {\isacharparenleft}value\ t{\isacharparenright}\ {\isacharparenleft}{\isacharparenleft}a{\isacharcomma}update\ tt\ as\ v{\isacharparenright}{\isacharhash}alist\ t{\isacharparenright}{\isacharparenright}{\isachardoublequote}%
\begin{isamarkuptext}%
\noindent
The base case is obvious. In the recursive case the subtrie
\isa{tt} associated with the first letter \isa{a} is extracted,
recursively updated, and then placed in front of the association list.
The old subtrie associated with \isa{a} is still in the association list
but no longer accessible via \isa{assoc}. Clearly, there is room here for
optimizations!

Before we start on any proofs about \isa{update} we tell the simplifier to
expand all \isa{let}s and to split all \isa{case}-constructs over
options:%
\end{isamarkuptext}%
\isacommand{lemmas}\ {\isacharbrackleft}simp{\isacharbrackright}\ {\isacharequal}\ Let{\isacharunderscore}def\isanewline
\isacommand{lemmas}\ {\isacharbrackleft}split{\isacharbrackright}\ {\isacharequal}\ option{\isachardot}split%
\begin{isamarkuptext}%
\noindent
The reason becomes clear when looking (probably after a failed proof
attempt) at the body of \isa{update}: it contains both
\isa{let} and a case distinction over type \isa{option}.

Our main goal is to prove the correct interaction of \isa{update} and
\isa{lookup}:%
\end{isamarkuptext}%
\isacommand{theorem}\ {\isachardoublequote}{\isasymforall}t\ v\ bs{\isachardot}\ lookup\ {\isacharparenleft}update\ t\ as\ v{\isacharparenright}\ bs\ {\isacharequal}\isanewline
\ \ \ \ \ \ \ \ \ \ \ \ \ \ \ \ \ \ \ \ {\isacharparenleft}if\ as{\isacharequal}bs\ then\ Some\ v\ else\ lookup\ t\ bs{\isacharparenright}{\isachardoublequote}%
\begin{isamarkuptxt}%
\noindent
Our plan is to induct on \isa{as}; hence the remaining variables are
quantified. From the definitions it is clear that induction on either
\isa{as} or \isa{bs} is required. The choice of \isa{as} is merely
guided by the intuition that simplification of \isa{lookup} might be easier
if \isa{update} has already been simplified, which can only happen if
\isa{as} is instantiated.
The start of the proof is completely conventional:%
\end{isamarkuptxt}%
\isacommand{apply}{\isacharparenleft}induct{\isacharunderscore}tac\ as{\isacharcomma}\ auto{\isacharparenright}%
\begin{isamarkuptxt}%
\noindent
Unfortunately, this time we are left with three intimidating looking subgoals:
\begin{isabelle}
~1.~\dots~{\isasymLongrightarrow}~lookup~\dots~bs~=~lookup~t~bs\isanewline
~2.~\dots~{\isasymLongrightarrow}~lookup~\dots~bs~=~lookup~t~bs\isanewline
~3.~\dots~{\isasymLongrightarrow}~lookup~\dots~bs~=~lookup~t~bs%
\end{isabelle}
Clearly, if we want to make headway we have to instantiate \isa{bs} as
well now. It turns out that instead of induction, case distinction
suffices:%
\end{isamarkuptxt}%
\isacommand{by}{\isacharparenleft}case{\isacharunderscore}tac{\isacharbrackleft}{\isacharbang}{\isacharbrackright}\ bs{\isacharcomma}\ auto{\isacharparenright}%
\begin{isamarkuptext}%
\noindent
All methods ending in \isa{tac} take an optional first argument that
specifies the range of subgoals they are applied to, where \isa{[!]} means all
subgoals, i.e.\ \isa{[1-3]} in our case. Individual subgoal numbers,
e.g. \isa{[2]} are also allowed.

This proof may look surprisingly straightforward. However, note that this
comes at a cost: the proof script is unreadable because the
intermediate proof states are invisible, and we rely on the (possibly
brittle) magic of \isa{auto} (\isa{simp\_all} will not do---try it) to split the subgoals
of the induction up in such a way that case distinction on \isa{bs} makes sense and
solves the proof. Part~\ref{Isar} shows you how to write readable and stable
proofs.%
\end{isamarkuptext}%
\end{isabellebody}%
%%% Local Variables:
%%% mode: latex
%%% TeX-master: "root"
%%% End:

\index{tries|)}

\section{Total Recursive Functions: \isacommand{fun}}
\label{sec:fun}
\index{fun@\isacommand {fun} (command)|(}\index{functions!total|(}

Although many total functions have a natural primitive recursive definition,
this is not always the case. Arbitrary total recursive functions can be
defined by means of \isacommand{fun}: you can use full pattern matching,
recursion need not involve datatypes, and termination is proved by showing
that the arguments of all recursive calls are smaller in a suitable sense.
In this section we restrict ourselves to functions where Isabelle can prove
termination automatically. More advanced function definitions, including user
supplied termination proofs, nested recursion and partiality, are discussed
in a separate tutorial~\cite{isabelle-function}.

%
\begin{isabellebody}%
\def\isabellecontext{fun{\isadigit{0}}}%
%
\isadelimtheory
%
\endisadelimtheory
%
\isatagtheory
%
\endisatagtheory
{\isafoldtheory}%
%
\isadelimtheory
%
\endisadelimtheory
%
\begin{isamarkuptext}%
\subsection{Definition}
\label{sec:fun-examples}

Here is a simple example, the \rmindex{Fibonacci function}:%
\end{isamarkuptext}%
\isamarkuptrue%
\isacommand{fun}\isamarkupfalse%
\ fib\ {\isaliteral{3A}{\isacharcolon}}{\isaliteral{3A}{\isacharcolon}}\ {\isaliteral{22}{\isachardoublequoteopen}}nat\ {\isaliteral{5C3C52696768746172726F773E}{\isasymRightarrow}}\ nat{\isaliteral{22}{\isachardoublequoteclose}}\ \isakeyword{where}\isanewline
{\isaliteral{22}{\isachardoublequoteopen}}fib\ {\isadigit{0}}\ {\isaliteral{3D}{\isacharequal}}\ {\isadigit{0}}{\isaliteral{22}{\isachardoublequoteclose}}\ {\isaliteral{7C}{\isacharbar}}\isanewline
{\isaliteral{22}{\isachardoublequoteopen}}fib\ {\isaliteral{28}{\isacharparenleft}}Suc\ {\isadigit{0}}{\isaliteral{29}{\isacharparenright}}\ {\isaliteral{3D}{\isacharequal}}\ {\isadigit{1}}{\isaliteral{22}{\isachardoublequoteclose}}\ {\isaliteral{7C}{\isacharbar}}\isanewline
{\isaliteral{22}{\isachardoublequoteopen}}fib\ {\isaliteral{28}{\isacharparenleft}}Suc{\isaliteral{28}{\isacharparenleft}}Suc\ x{\isaliteral{29}{\isacharparenright}}{\isaliteral{29}{\isacharparenright}}\ {\isaliteral{3D}{\isacharequal}}\ fib\ x\ {\isaliteral{2B}{\isacharplus}}\ fib\ {\isaliteral{28}{\isacharparenleft}}Suc\ x{\isaliteral{29}{\isacharparenright}}{\isaliteral{22}{\isachardoublequoteclose}}%
\begin{isamarkuptext}%
\noindent
This resembles ordinary functional programming languages. Note the obligatory
\isacommand{where} and \isa{|}. Command \isacommand{fun} declares and
defines the function in one go. Isabelle establishes termination automatically
because \isa{fib}'s argument decreases in every recursive call.

Slightly more interesting is the insertion of a fixed element
between any two elements of a list:%
\end{isamarkuptext}%
\isamarkuptrue%
\isacommand{fun}\isamarkupfalse%
\ sep\ {\isaliteral{3A}{\isacharcolon}}{\isaliteral{3A}{\isacharcolon}}\ {\isaliteral{22}{\isachardoublequoteopen}}{\isaliteral{27}{\isacharprime}}a\ {\isaliteral{5C3C52696768746172726F773E}{\isasymRightarrow}}\ {\isaliteral{27}{\isacharprime}}a\ list\ {\isaliteral{5C3C52696768746172726F773E}{\isasymRightarrow}}\ {\isaliteral{27}{\isacharprime}}a\ list{\isaliteral{22}{\isachardoublequoteclose}}\ \isakeyword{where}\isanewline
{\isaliteral{22}{\isachardoublequoteopen}}sep\ a\ {\isaliteral{5B}{\isacharbrackleft}}{\isaliteral{5D}{\isacharbrackright}}\ \ \ \ \ {\isaliteral{3D}{\isacharequal}}\ {\isaliteral{5B}{\isacharbrackleft}}{\isaliteral{5D}{\isacharbrackright}}{\isaliteral{22}{\isachardoublequoteclose}}\ {\isaliteral{7C}{\isacharbar}}\isanewline
{\isaliteral{22}{\isachardoublequoteopen}}sep\ a\ {\isaliteral{5B}{\isacharbrackleft}}x{\isaliteral{5D}{\isacharbrackright}}\ \ \ \ {\isaliteral{3D}{\isacharequal}}\ {\isaliteral{5B}{\isacharbrackleft}}x{\isaliteral{5D}{\isacharbrackright}}{\isaliteral{22}{\isachardoublequoteclose}}\ {\isaliteral{7C}{\isacharbar}}\isanewline
{\isaliteral{22}{\isachardoublequoteopen}}sep\ a\ {\isaliteral{28}{\isacharparenleft}}x{\isaliteral{23}{\isacharhash}}y{\isaliteral{23}{\isacharhash}}zs{\isaliteral{29}{\isacharparenright}}\ {\isaliteral{3D}{\isacharequal}}\ x\ {\isaliteral{23}{\isacharhash}}\ a\ {\isaliteral{23}{\isacharhash}}\ sep\ a\ {\isaliteral{28}{\isacharparenleft}}y{\isaliteral{23}{\isacharhash}}zs{\isaliteral{29}{\isacharparenright}}{\isaliteral{22}{\isachardoublequoteclose}}%
\begin{isamarkuptext}%
\noindent
This time the length of the list decreases with the
recursive call; the first argument is irrelevant for termination.

Pattern matching\index{pattern matching!and \isacommand{fun}}
need not be exhaustive and may employ wildcards:%
\end{isamarkuptext}%
\isamarkuptrue%
\isacommand{fun}\isamarkupfalse%
\ last\ {\isaliteral{3A}{\isacharcolon}}{\isaliteral{3A}{\isacharcolon}}\ {\isaliteral{22}{\isachardoublequoteopen}}{\isaliteral{27}{\isacharprime}}a\ list\ {\isaliteral{5C3C52696768746172726F773E}{\isasymRightarrow}}\ {\isaliteral{27}{\isacharprime}}a{\isaliteral{22}{\isachardoublequoteclose}}\ \isakeyword{where}\isanewline
{\isaliteral{22}{\isachardoublequoteopen}}last\ {\isaliteral{5B}{\isacharbrackleft}}x{\isaliteral{5D}{\isacharbrackright}}\ \ \ \ \ \ {\isaliteral{3D}{\isacharequal}}\ x{\isaliteral{22}{\isachardoublequoteclose}}\ {\isaliteral{7C}{\isacharbar}}\isanewline
{\isaliteral{22}{\isachardoublequoteopen}}last\ {\isaliteral{28}{\isacharparenleft}}{\isaliteral{5F}{\isacharunderscore}}{\isaliteral{23}{\isacharhash}}y{\isaliteral{23}{\isacharhash}}zs{\isaliteral{29}{\isacharparenright}}\ {\isaliteral{3D}{\isacharequal}}\ last\ {\isaliteral{28}{\isacharparenleft}}y{\isaliteral{23}{\isacharhash}}zs{\isaliteral{29}{\isacharparenright}}{\isaliteral{22}{\isachardoublequoteclose}}%
\begin{isamarkuptext}%
Overlapping patterns are disambiguated by taking the order of equations into
account, just as in functional programming:%
\end{isamarkuptext}%
\isamarkuptrue%
\isacommand{fun}\isamarkupfalse%
\ sep{\isadigit{1}}\ {\isaliteral{3A}{\isacharcolon}}{\isaliteral{3A}{\isacharcolon}}\ {\isaliteral{22}{\isachardoublequoteopen}}{\isaliteral{27}{\isacharprime}}a\ {\isaliteral{5C3C52696768746172726F773E}{\isasymRightarrow}}\ {\isaliteral{27}{\isacharprime}}a\ list\ {\isaliteral{5C3C52696768746172726F773E}{\isasymRightarrow}}\ {\isaliteral{27}{\isacharprime}}a\ list{\isaliteral{22}{\isachardoublequoteclose}}\ \isakeyword{where}\isanewline
{\isaliteral{22}{\isachardoublequoteopen}}sep{\isadigit{1}}\ a\ {\isaliteral{28}{\isacharparenleft}}x{\isaliteral{23}{\isacharhash}}y{\isaliteral{23}{\isacharhash}}zs{\isaliteral{29}{\isacharparenright}}\ {\isaliteral{3D}{\isacharequal}}\ x\ {\isaliteral{23}{\isacharhash}}\ a\ {\isaliteral{23}{\isacharhash}}\ sep{\isadigit{1}}\ a\ {\isaliteral{28}{\isacharparenleft}}y{\isaliteral{23}{\isacharhash}}zs{\isaliteral{29}{\isacharparenright}}{\isaliteral{22}{\isachardoublequoteclose}}\ {\isaliteral{7C}{\isacharbar}}\isanewline
{\isaliteral{22}{\isachardoublequoteopen}}sep{\isadigit{1}}\ {\isaliteral{5F}{\isacharunderscore}}\ xs\ \ \ \ \ \ \ {\isaliteral{3D}{\isacharequal}}\ xs{\isaliteral{22}{\isachardoublequoteclose}}%
\begin{isamarkuptext}%
\noindent
To guarantee that the second equation can only be applied if the first
one does not match, Isabelle internally replaces the second equation
by the two possibilities that are left: \isa{sep{\isadigit{1}}\ a\ {\isaliteral{5B}{\isacharbrackleft}}{\isaliteral{5D}{\isacharbrackright}}\ {\isaliteral{3D}{\isacharequal}}\ {\isaliteral{5B}{\isacharbrackleft}}{\isaliteral{5D}{\isacharbrackright}}} and
\isa{sep{\isadigit{1}}\ a\ {\isaliteral{5B}{\isacharbrackleft}}x{\isaliteral{5D}{\isacharbrackright}}\ {\isaliteral{3D}{\isacharequal}}\ {\isaliteral{5B}{\isacharbrackleft}}x{\isaliteral{5D}{\isacharbrackright}}}.  Thus the functions \isa{sep} and
\isa{sep{\isadigit{1}}} are identical.

Because of its pattern matching syntax, \isacommand{fun} is also useful
for the definition of non-recursive functions:%
\end{isamarkuptext}%
\isamarkuptrue%
\isacommand{fun}\isamarkupfalse%
\ swap{\isadigit{1}}{\isadigit{2}}\ {\isaliteral{3A}{\isacharcolon}}{\isaliteral{3A}{\isacharcolon}}\ {\isaliteral{22}{\isachardoublequoteopen}}{\isaliteral{27}{\isacharprime}}a\ list\ {\isaliteral{5C3C52696768746172726F773E}{\isasymRightarrow}}\ {\isaliteral{27}{\isacharprime}}a\ list{\isaliteral{22}{\isachardoublequoteclose}}\ \isakeyword{where}\isanewline
{\isaliteral{22}{\isachardoublequoteopen}}swap{\isadigit{1}}{\isadigit{2}}\ {\isaliteral{28}{\isacharparenleft}}x{\isaliteral{23}{\isacharhash}}y{\isaliteral{23}{\isacharhash}}zs{\isaliteral{29}{\isacharparenright}}\ {\isaliteral{3D}{\isacharequal}}\ y{\isaliteral{23}{\isacharhash}}x{\isaliteral{23}{\isacharhash}}zs{\isaliteral{22}{\isachardoublequoteclose}}\ {\isaliteral{7C}{\isacharbar}}\isanewline
{\isaliteral{22}{\isachardoublequoteopen}}swap{\isadigit{1}}{\isadigit{2}}\ zs\ \ \ \ \ \ \ {\isaliteral{3D}{\isacharequal}}\ zs{\isaliteral{22}{\isachardoublequoteclose}}%
\begin{isamarkuptext}%
After a function~$f$ has been defined via \isacommand{fun},
its defining equations (or variants derived from them) are available
under the name $f$\isa{{\isaliteral{2E}{\isachardot}}simps} as theorems.
For example, look (via \isacommand{thm}) at
\isa{sep{\isaliteral{2E}{\isachardot}}simps} and \isa{sep{\isadigit{1}}{\isaliteral{2E}{\isachardot}}simps} to see that they define
the same function. What is more, those equations are automatically declared as
simplification rules.

\subsection{Termination}

Isabelle's automatic termination prover for \isacommand{fun} has a
fixed notion of the \emph{size} (of type \isa{nat}) of an
argument. The size of a natural number is the number itself. The size
of a list is its length. For the general case see \S\ref{sec:general-datatype}.
A recursive function is accepted if \isacommand{fun} can
show that the size of one fixed argument becomes smaller with each
recursive call.

More generally, \isacommand{fun} allows any \emph{lexicographic
combination} of size measures in case there are multiple
arguments. For example, the following version of \rmindex{Ackermann's
function} is accepted:%
\end{isamarkuptext}%
\isamarkuptrue%
\isacommand{fun}\isamarkupfalse%
\ ack{\isadigit{2}}\ {\isaliteral{3A}{\isacharcolon}}{\isaliteral{3A}{\isacharcolon}}\ {\isaliteral{22}{\isachardoublequoteopen}}nat\ {\isaliteral{5C3C52696768746172726F773E}{\isasymRightarrow}}\ nat\ {\isaliteral{5C3C52696768746172726F773E}{\isasymRightarrow}}\ nat{\isaliteral{22}{\isachardoublequoteclose}}\ \isakeyword{where}\isanewline
{\isaliteral{22}{\isachardoublequoteopen}}ack{\isadigit{2}}\ n\ {\isadigit{0}}\ {\isaliteral{3D}{\isacharequal}}\ Suc\ n{\isaliteral{22}{\isachardoublequoteclose}}\ {\isaliteral{7C}{\isacharbar}}\isanewline
{\isaliteral{22}{\isachardoublequoteopen}}ack{\isadigit{2}}\ {\isadigit{0}}\ {\isaliteral{28}{\isacharparenleft}}Suc\ m{\isaliteral{29}{\isacharparenright}}\ {\isaliteral{3D}{\isacharequal}}\ ack{\isadigit{2}}\ {\isaliteral{28}{\isacharparenleft}}Suc\ {\isadigit{0}}{\isaliteral{29}{\isacharparenright}}\ m{\isaliteral{22}{\isachardoublequoteclose}}\ {\isaliteral{7C}{\isacharbar}}\isanewline
{\isaliteral{22}{\isachardoublequoteopen}}ack{\isadigit{2}}\ {\isaliteral{28}{\isacharparenleft}}Suc\ n{\isaliteral{29}{\isacharparenright}}\ {\isaliteral{28}{\isacharparenleft}}Suc\ m{\isaliteral{29}{\isacharparenright}}\ {\isaliteral{3D}{\isacharequal}}\ ack{\isadigit{2}}\ {\isaliteral{28}{\isacharparenleft}}ack{\isadigit{2}}\ n\ {\isaliteral{28}{\isacharparenleft}}Suc\ m{\isaliteral{29}{\isacharparenright}}{\isaliteral{29}{\isacharparenright}}\ m{\isaliteral{22}{\isachardoublequoteclose}}%
\begin{isamarkuptext}%
The order of arguments has no influence on whether
\isacommand{fun} can prove termination of a function. For more details
see elsewhere~\cite{bulwahnKN07}.

\subsection{Simplification}
\label{sec:fun-simplification}

Upon a successful termination proof, the recursion equations become
simplification rules, just as with \isacommand{primrec}.
In most cases this works fine, but there is a subtle
problem that must be mentioned: simplification may not
terminate because of automatic splitting of \isa{if}.
\index{*if expressions!splitting of}
Let us look at an example:%
\end{isamarkuptext}%
\isamarkuptrue%
\isacommand{fun}\isamarkupfalse%
\ gcd\ {\isaliteral{3A}{\isacharcolon}}{\isaliteral{3A}{\isacharcolon}}\ {\isaliteral{22}{\isachardoublequoteopen}}nat\ {\isaliteral{5C3C52696768746172726F773E}{\isasymRightarrow}}\ nat\ {\isaliteral{5C3C52696768746172726F773E}{\isasymRightarrow}}\ nat{\isaliteral{22}{\isachardoublequoteclose}}\ \isakeyword{where}\isanewline
{\isaliteral{22}{\isachardoublequoteopen}}gcd\ m\ n\ {\isaliteral{3D}{\isacharequal}}\ {\isaliteral{28}{\isacharparenleft}}if\ n{\isaliteral{3D}{\isacharequal}}{\isadigit{0}}\ then\ m\ else\ gcd\ n\ {\isaliteral{28}{\isacharparenleft}}m\ mod\ n{\isaliteral{29}{\isacharparenright}}{\isaliteral{29}{\isacharparenright}}{\isaliteral{22}{\isachardoublequoteclose}}%
\begin{isamarkuptext}%
\noindent
The second argument decreases with each recursive call.
The termination condition
\begin{isabelle}%
\ \ \ \ \ n\ {\isaliteral{5C3C6E6F7465713E}{\isasymnoteq}}\ {\isadigit{0}}\ {\isaliteral{5C3C4C6F6E6772696768746172726F773E}{\isasymLongrightarrow}}\ m\ mod\ n\ {\isaliteral{3C}{\isacharless}}\ n%
\end{isabelle}
is proved automatically because it is already present as a lemma in
HOL\@.  Thus the recursion equation becomes a simplification
rule. Of course the equation is nonterminating if we are allowed to unfold
the recursive call inside the \isa{else} branch, which is why programming
languages and our simplifier don't do that. Unfortunately the simplifier does
something else that leads to the same problem: it splits 
each \isa{if}-expression unless its
condition simplifies to \isa{True} or \isa{False}.  For
example, simplification reduces
\begin{isabelle}%
\ \ \ \ \ gcd\ m\ n\ {\isaliteral{3D}{\isacharequal}}\ k%
\end{isabelle}
in one step to
\begin{isabelle}%
\ \ \ \ \ {\isaliteral{28}{\isacharparenleft}}if\ n\ {\isaliteral{3D}{\isacharequal}}\ {\isadigit{0}}\ then\ m\ else\ gcd\ n\ {\isaliteral{28}{\isacharparenleft}}m\ mod\ n{\isaliteral{29}{\isacharparenright}}{\isaliteral{29}{\isacharparenright}}\ {\isaliteral{3D}{\isacharequal}}\ k%
\end{isabelle}
where the condition cannot be reduced further, and splitting leads to
\begin{isabelle}%
\ \ \ \ \ {\isaliteral{28}{\isacharparenleft}}n\ {\isaliteral{3D}{\isacharequal}}\ {\isadigit{0}}\ {\isaliteral{5C3C6C6F6E6772696768746172726F773E}{\isasymlongrightarrow}}\ m\ {\isaliteral{3D}{\isacharequal}}\ k{\isaliteral{29}{\isacharparenright}}\ {\isaliteral{5C3C616E643E}{\isasymand}}\ {\isaliteral{28}{\isacharparenleft}}n\ {\isaliteral{5C3C6E6F7465713E}{\isasymnoteq}}\ {\isadigit{0}}\ {\isaliteral{5C3C6C6F6E6772696768746172726F773E}{\isasymlongrightarrow}}\ gcd\ n\ {\isaliteral{28}{\isacharparenleft}}m\ mod\ n{\isaliteral{29}{\isacharparenright}}\ {\isaliteral{3D}{\isacharequal}}\ k{\isaliteral{29}{\isacharparenright}}%
\end{isabelle}
Since the recursive call \isa{gcd\ n\ {\isaliteral{28}{\isacharparenleft}}m\ mod\ n{\isaliteral{29}{\isacharparenright}}} is no longer protected by
an \isa{if}, it is unfolded again, which leads to an infinite chain of
simplification steps. Fortunately, this problem can be avoided in many
different ways.

The most radical solution is to disable the offending theorem
\isa{split{\isaliteral{5F}{\isacharunderscore}}if},
as shown in \S\ref{sec:AutoCaseSplits}.  However, we do not recommend this
approach: you will often have to invoke the rule explicitly when
\isa{if} is involved.

If possible, the definition should be given by pattern matching on the left
rather than \isa{if} on the right. In the case of \isa{gcd} the
following alternative definition suggests itself:%
\end{isamarkuptext}%
\isamarkuptrue%
\isacommand{fun}\isamarkupfalse%
\ gcd{\isadigit{1}}\ {\isaliteral{3A}{\isacharcolon}}{\isaliteral{3A}{\isacharcolon}}\ {\isaliteral{22}{\isachardoublequoteopen}}nat\ {\isaliteral{5C3C52696768746172726F773E}{\isasymRightarrow}}\ nat\ {\isaliteral{5C3C52696768746172726F773E}{\isasymRightarrow}}\ nat{\isaliteral{22}{\isachardoublequoteclose}}\ \isakeyword{where}\isanewline
{\isaliteral{22}{\isachardoublequoteopen}}gcd{\isadigit{1}}\ m\ {\isadigit{0}}\ {\isaliteral{3D}{\isacharequal}}\ m{\isaliteral{22}{\isachardoublequoteclose}}\ {\isaliteral{7C}{\isacharbar}}\isanewline
{\isaliteral{22}{\isachardoublequoteopen}}gcd{\isadigit{1}}\ m\ n\ {\isaliteral{3D}{\isacharequal}}\ gcd{\isadigit{1}}\ n\ {\isaliteral{28}{\isacharparenleft}}m\ mod\ n{\isaliteral{29}{\isacharparenright}}{\isaliteral{22}{\isachardoublequoteclose}}%
\begin{isamarkuptext}%
\noindent
The order of equations is important: it hides the side condition
\isa{n\ {\isaliteral{5C3C6E6F7465713E}{\isasymnoteq}}\ {\isadigit{0}}}.  Unfortunately, not all conditionals can be
expressed by pattern matching.

A simple alternative is to replace \isa{if} by \isa{case}, 
which is also available for \isa{bool} and is not split automatically:%
\end{isamarkuptext}%
\isamarkuptrue%
\isacommand{fun}\isamarkupfalse%
\ gcd{\isadigit{2}}\ {\isaliteral{3A}{\isacharcolon}}{\isaliteral{3A}{\isacharcolon}}\ {\isaliteral{22}{\isachardoublequoteopen}}nat\ {\isaliteral{5C3C52696768746172726F773E}{\isasymRightarrow}}\ nat\ {\isaliteral{5C3C52696768746172726F773E}{\isasymRightarrow}}\ nat{\isaliteral{22}{\isachardoublequoteclose}}\ \isakeyword{where}\isanewline
{\isaliteral{22}{\isachardoublequoteopen}}gcd{\isadigit{2}}\ m\ n\ {\isaliteral{3D}{\isacharequal}}\ {\isaliteral{28}{\isacharparenleft}}case\ n{\isaliteral{3D}{\isacharequal}}{\isadigit{0}}\ of\ True\ {\isaliteral{5C3C52696768746172726F773E}{\isasymRightarrow}}\ m\ {\isaliteral{7C}{\isacharbar}}\ False\ {\isaliteral{5C3C52696768746172726F773E}{\isasymRightarrow}}\ gcd{\isadigit{2}}\ n\ {\isaliteral{28}{\isacharparenleft}}m\ mod\ n{\isaliteral{29}{\isacharparenright}}{\isaliteral{29}{\isacharparenright}}{\isaliteral{22}{\isachardoublequoteclose}}%
\begin{isamarkuptext}%
\noindent
This is probably the neatest solution next to pattern matching, and it is
always available.

A final alternative is to replace the offending simplification rules by
derived conditional ones. For \isa{gcd} it means we have to prove
these lemmas:%
\end{isamarkuptext}%
\isamarkuptrue%
\isacommand{lemma}\isamarkupfalse%
\ {\isaliteral{5B}{\isacharbrackleft}}simp{\isaliteral{5D}{\isacharbrackright}}{\isaliteral{3A}{\isacharcolon}}\ {\isaliteral{22}{\isachardoublequoteopen}}gcd\ m\ {\isadigit{0}}\ {\isaliteral{3D}{\isacharequal}}\ m{\isaliteral{22}{\isachardoublequoteclose}}\isanewline
%
\isadelimproof
%
\endisadelimproof
%
\isatagproof
\isacommand{apply}\isamarkupfalse%
{\isaliteral{28}{\isacharparenleft}}simp{\isaliteral{29}{\isacharparenright}}\isanewline
\isacommand{done}\isamarkupfalse%
%
\endisatagproof
{\isafoldproof}%
%
\isadelimproof
\isanewline
%
\endisadelimproof
\isanewline
\isacommand{lemma}\isamarkupfalse%
\ {\isaliteral{5B}{\isacharbrackleft}}simp{\isaliteral{5D}{\isacharbrackright}}{\isaliteral{3A}{\isacharcolon}}\ {\isaliteral{22}{\isachardoublequoteopen}}n\ {\isaliteral{5C3C6E6F7465713E}{\isasymnoteq}}\ {\isadigit{0}}\ {\isaliteral{5C3C4C6F6E6772696768746172726F773E}{\isasymLongrightarrow}}\ gcd\ m\ n\ {\isaliteral{3D}{\isacharequal}}\ gcd\ n\ {\isaliteral{28}{\isacharparenleft}}m\ mod\ n{\isaliteral{29}{\isacharparenright}}{\isaliteral{22}{\isachardoublequoteclose}}\isanewline
%
\isadelimproof
%
\endisadelimproof
%
\isatagproof
\isacommand{apply}\isamarkupfalse%
{\isaliteral{28}{\isacharparenleft}}simp{\isaliteral{29}{\isacharparenright}}\isanewline
\isacommand{done}\isamarkupfalse%
%
\endisatagproof
{\isafoldproof}%
%
\isadelimproof
%
\endisadelimproof
%
\begin{isamarkuptext}%
\noindent
Simplification terminates for these proofs because the condition of the \isa{if} simplifies to \isa{True} or \isa{False}.
Now we can disable the original simplification rule:%
\end{isamarkuptext}%
\isamarkuptrue%
\isacommand{declare}\isamarkupfalse%
\ gcd{\isaliteral{2E}{\isachardot}}simps\ {\isaliteral{5B}{\isacharbrackleft}}simp\ del{\isaliteral{5D}{\isacharbrackright}}%
\begin{isamarkuptext}%
\index{induction!recursion|(}
\index{recursion induction|(}

\subsection{Induction}
\label{sec:fun-induction}

Having defined a function we might like to prove something about it.
Since the function is recursive, the natural proof principle is
again induction. But this time the structural form of induction that comes
with datatypes is unlikely to work well --- otherwise we could have defined the
function by \isacommand{primrec}. Therefore \isacommand{fun} automatically
proves a suitable induction rule $f$\isa{{\isaliteral{2E}{\isachardot}}induct} that follows the
recursion pattern of the particular function $f$. We call this
\textbf{recursion induction}. Roughly speaking, it
requires you to prove for each \isacommand{fun} equation that the property
you are trying to establish holds for the left-hand side provided it holds
for all recursive calls on the right-hand side. Here is a simple example
involving the predefined \isa{map} functional on lists:%
\end{isamarkuptext}%
\isamarkuptrue%
\isacommand{lemma}\isamarkupfalse%
\ {\isaliteral{22}{\isachardoublequoteopen}}map\ f\ {\isaliteral{28}{\isacharparenleft}}sep\ x\ xs{\isaliteral{29}{\isacharparenright}}\ {\isaliteral{3D}{\isacharequal}}\ sep\ {\isaliteral{28}{\isacharparenleft}}f\ x{\isaliteral{29}{\isacharparenright}}\ {\isaliteral{28}{\isacharparenleft}}map\ f\ xs{\isaliteral{29}{\isacharparenright}}{\isaliteral{22}{\isachardoublequoteclose}}%
\isadelimproof
%
\endisadelimproof
%
\isatagproof
%
\begin{isamarkuptxt}%
\noindent
Note that \isa{map\ f\ xs}
is the result of applying \isa{f} to all elements of \isa{xs}. We prove
this lemma by recursion induction over \isa{sep}:%
\end{isamarkuptxt}%
\isamarkuptrue%
\isacommand{apply}\isamarkupfalse%
{\isaliteral{28}{\isacharparenleft}}induct{\isaliteral{5F}{\isacharunderscore}}tac\ x\ xs\ rule{\isaliteral{3A}{\isacharcolon}}\ sep{\isaliteral{2E}{\isachardot}}induct{\isaliteral{29}{\isacharparenright}}%
\begin{isamarkuptxt}%
\noindent
The resulting proof state has three subgoals corresponding to the three
clauses for \isa{sep}:
\begin{isabelle}%
\ {\isadigit{1}}{\isaliteral{2E}{\isachardot}}\ {\isaliteral{5C3C416E643E}{\isasymAnd}}a{\isaliteral{2E}{\isachardot}}\ map\ f\ {\isaliteral{28}{\isacharparenleft}}sep\ a\ {\isaliteral{5B}{\isacharbrackleft}}{\isaliteral{5D}{\isacharbrackright}}{\isaliteral{29}{\isacharparenright}}\ {\isaliteral{3D}{\isacharequal}}\ sep\ {\isaliteral{28}{\isacharparenleft}}f\ a{\isaliteral{29}{\isacharparenright}}\ {\isaliteral{28}{\isacharparenleft}}map\ f\ {\isaliteral{5B}{\isacharbrackleft}}{\isaliteral{5D}{\isacharbrackright}}{\isaliteral{29}{\isacharparenright}}\isanewline
\ {\isadigit{2}}{\isaliteral{2E}{\isachardot}}\ {\isaliteral{5C3C416E643E}{\isasymAnd}}a\ x{\isaliteral{2E}{\isachardot}}\ map\ f\ {\isaliteral{28}{\isacharparenleft}}sep\ a\ {\isaliteral{5B}{\isacharbrackleft}}x{\isaliteral{5D}{\isacharbrackright}}{\isaliteral{29}{\isacharparenright}}\ {\isaliteral{3D}{\isacharequal}}\ sep\ {\isaliteral{28}{\isacharparenleft}}f\ a{\isaliteral{29}{\isacharparenright}}\ {\isaliteral{28}{\isacharparenleft}}map\ f\ {\isaliteral{5B}{\isacharbrackleft}}x{\isaliteral{5D}{\isacharbrackright}}{\isaliteral{29}{\isacharparenright}}\isanewline
\ {\isadigit{3}}{\isaliteral{2E}{\isachardot}}\ {\isaliteral{5C3C416E643E}{\isasymAnd}}a\ x\ y\ zs{\isaliteral{2E}{\isachardot}}\isanewline
\isaindent{\ {\isadigit{3}}{\isaliteral{2E}{\isachardot}}\ \ \ \ }map\ f\ {\isaliteral{28}{\isacharparenleft}}sep\ a\ {\isaliteral{28}{\isacharparenleft}}y\ {\isaliteral{23}{\isacharhash}}\ zs{\isaliteral{29}{\isacharparenright}}{\isaliteral{29}{\isacharparenright}}\ {\isaliteral{3D}{\isacharequal}}\ sep\ {\isaliteral{28}{\isacharparenleft}}f\ a{\isaliteral{29}{\isacharparenright}}\ {\isaliteral{28}{\isacharparenleft}}map\ f\ {\isaliteral{28}{\isacharparenleft}}y\ {\isaliteral{23}{\isacharhash}}\ zs{\isaliteral{29}{\isacharparenright}}{\isaliteral{29}{\isacharparenright}}\ {\isaliteral{5C3C4C6F6E6772696768746172726F773E}{\isasymLongrightarrow}}\isanewline
\isaindent{\ {\isadigit{3}}{\isaliteral{2E}{\isachardot}}\ \ \ \ }map\ f\ {\isaliteral{28}{\isacharparenleft}}sep\ a\ {\isaliteral{28}{\isacharparenleft}}x\ {\isaliteral{23}{\isacharhash}}\ y\ {\isaliteral{23}{\isacharhash}}\ zs{\isaliteral{29}{\isacharparenright}}{\isaliteral{29}{\isacharparenright}}\ {\isaliteral{3D}{\isacharequal}}\ sep\ {\isaliteral{28}{\isacharparenleft}}f\ a{\isaliteral{29}{\isacharparenright}}\ {\isaliteral{28}{\isacharparenleft}}map\ f\ {\isaliteral{28}{\isacharparenleft}}x\ {\isaliteral{23}{\isacharhash}}\ y\ {\isaliteral{23}{\isacharhash}}\ zs{\isaliteral{29}{\isacharparenright}}{\isaliteral{29}{\isacharparenright}}%
\end{isabelle}
The rest is pure simplification:%
\end{isamarkuptxt}%
\isamarkuptrue%
\isacommand{apply}\isamarkupfalse%
\ simp{\isaliteral{5F}{\isacharunderscore}}all\isanewline
\isacommand{done}\isamarkupfalse%
%
\endisatagproof
{\isafoldproof}%
%
\isadelimproof
%
\endisadelimproof
%
\begin{isamarkuptext}%
\noindent The proof goes smoothly because the induction rule
follows the recursion of \isa{sep}.  Try proving the above lemma by
structural induction, and you find that you need an additional case
distinction.

In general, the format of invoking recursion induction is
\begin{quote}
\isacommand{apply}\isa{{\isaliteral{28}{\isacharparenleft}}induct{\isaliteral{5F}{\isacharunderscore}}tac} $x@1 \dots x@n$ \isa{rule{\isaliteral{3A}{\isacharcolon}}} $f$\isa{{\isaliteral{2E}{\isachardot}}induct{\isaliteral{29}{\isacharparenright}}}
\end{quote}\index{*induct_tac (method)}%
where $x@1~\dots~x@n$ is a list of free variables in the subgoal and $f$ the
name of a function that takes $n$ arguments. Usually the subgoal will
contain the term $f x@1 \dots x@n$ but this need not be the case. The
induction rules do not mention $f$ at all. Here is \isa{sep{\isaliteral{2E}{\isachardot}}induct}:
\begin{isabelle}
{\isasymlbrakk}~{\isasymAnd}a.~P~a~[];\isanewline
~~{\isasymAnd}a~x.~P~a~[x];\isanewline
~~{\isasymAnd}a~x~y~zs.~P~a~(y~\#~zs)~{\isasymLongrightarrow}~P~a~(x~\#~y~\#~zs){\isasymrbrakk}\isanewline
{\isasymLongrightarrow}~P~u~v%
\end{isabelle}
It merely says that in order to prove a property \isa{P} of \isa{u} and
\isa{v} you need to prove it for the three cases where \isa{v} is the
empty list, the singleton list, and the list with at least two elements.
The final case has an induction hypothesis:  you may assume that \isa{P}
holds for the tail of that list.
\index{induction!recursion|)}
\index{recursion induction|)}%
\end{isamarkuptext}%
\isamarkuptrue%
%
\isadelimtheory
%
\endisadelimtheory
%
\isatagtheory
%
\endisatagtheory
{\isafoldtheory}%
%
\isadelimtheory
%
\endisadelimtheory
\end{isabellebody}%
%%% Local Variables:
%%% mode: latex
%%% TeX-master: "root"
%%% End:


\index{fun@\isacommand {fun} (command)|)}\index{functions!total|)}

\chapter{The Rules of the Game}
\label{chap:rules}
 
Until now, we have proved everything using only induction and simplification.
Substantial proofs require more elaborate forms of inference.  This chapter
outlines the concepts and techniques that underlie reasoning in Isabelle. The examples
are mainly drawn from predicate logic.  The first examples in this
chapter will consist of detailed, low-level proof steps.  Later, we shall
see how to automate such reasoning using the methods \isa{blast},
\isa{auto} and others. 

\section{Natural deduction}

In Isabelle, proofs are constructed using inference rules. The 
most familiar inference rule is probably \emph{modus ponens}: 
\[ \infer{Q}{P\imp Q & P} \]
This rule says that from $P\imp Q$ and $P$  
we may infer~$Q$.  

%Early logical formalisms had this  
%rule and at most one or two others, along with many complicated 
%axioms. Any desired theorem could be obtained by applying \emph{modus 
%ponens} or other rules to the axioms, but proofs were 
%hard to find. For example, a standard inference system has 
%these two axioms (amongst others): 
%\begin{gather*}
%  P\imp(Q\imp P) \tag{K}\\
%  (P\imp(Q\imp R))\imp ((P\imp Q)\imp(P\imp R))  \tag{S}
%\end{gather*}
%Try proving the trivial fact $P\imp P$ using these axioms and \emph{modus
%ponens}!

\textbf{Natural deduction} is an attempt to formalize logic in a way 
that mirrors human reasoning patterns. 
%
%Instead of having a few 
%inference rules and many axioms, it has many inference rules 
%and few axioms. 
%
For each logical symbol (say, $\conj$), there 
are two kinds of rules: \textbf{introduction} and \textbf{elimination} rules. 
The introduction rules allow us to infer this symbol (say, to 
infer conjunctions). The elimination rules allow us to deduce 
consequences from this symbol. Ideally each rule should mention 
one symbol only.  For predicate logic this can be 
done, but when users define their own concepts they typically 
have to refer to other symbols as well.  It is best not be dogmatic.

Natural deduction generally deserves its name.  It is easy to use.  Each
proof step consists of identifying the outermost symbol of a formula and
applying the corresponding rule.  It creates new subgoals in
an obvious way from parts of the chosen formula.  Expanding the
definitions of constants can blow up the goal enormously.  Deriving natural
deduction rules for such constants lets us reason in terms of their key
properties, which might otherwise be obscured by the technicalities of its
definition.  Natural deduction rules also lend themselves to automation.
Isabelle's
\textbf{classical  reasoner} accepts any suitable  collection of natural deduction
rules and uses them to search for proofs automatically.  Isabelle is designed around
natural deduction and many of its  tools use the terminology of introduction and
elimination rules.


\section{Introduction rules}

An \textbf{introduction} rule tells us when we can infer a formula 
containing a specific logical symbol. For example, the conjunction 
introduction rule says that if we have $P$ and if we have $Q$ then 
we have $P\conj Q$. In a mathematics text, it is typically shown 
like this:
\[  \infer{P\conj Q}{P & Q} \]
The rule introduces the conjunction
symbol~($\conj$) in its conclusion.  Of course, in Isabelle proofs we
mainly  reason backwards.  When we apply this rule, the subgoal already has
the form of a conjunction; the proof step makes this conjunction symbol
disappear. 

In Isabelle notation, the rule looks like this:
\begin{isabelle}
\isasymlbrakk?P;\ ?Q\isasymrbrakk\ \isasymLongrightarrow\ ?P\ \isasymand\ ?Q\rulename{conjI}
\end{isabelle}
Carefully examine the syntax.  The premises appear to the
left of the arrow and the conclusion to the right.  The premises (if 
more than one) are grouped using the fat brackets.  The question marks
indicate \textbf{schematic variables} (also called \textbf{unknowns}): they may
be replaced by arbitrary formulas.  If we use the rule backwards, Isabelle
tries to unify the current subgoal with the conclusion of the rule, which
has the form \isa{?P\ \isasymand\ ?Q}.  (Unification is discussed below,
\S\ref{sec:unification}.)  If successful,
it yields new subgoals given by the formulas assigned to 
\isa{?P} and \isa{?Q}.

The following trivial proof illustrates this point. 
\begin{isabelle}
\isacommand{lemma}\ conj_rule:\ "{\isasymlbrakk}P;\
Q\isasymrbrakk\ \isasymLongrightarrow\ P\ \isasymand\
(Q\ \isasymand\ P)"\isanewline
\isacommand{apply}\ (rule\ conjI)\isanewline
\ \isacommand{apply}\ assumption\isanewline
\isacommand{apply}\ (rule\ conjI)\isanewline
\ \isacommand{apply}\ assumption\isanewline
\isacommand{apply}\ assumption
\end{isabelle}
At the start, Isabelle presents 
us with the assumptions (\isa{P} and~\isa{Q}) and with the goal to be proved,
\isa{P\ \isasymand\
(Q\ \isasymand\ P)}.  We are working backwards, so when we
apply conjunction introduction, the rule removes the outermost occurrence
of the \isa{\isasymand} symbol.  To apply a  rule to a subgoal, we apply
the proof method {\isa{rule}} --- here with {\isa{conjI}}, the  conjunction
introduction rule. 
\begin{isabelle}
%{\isasymlbrakk}P;\ Q\isasymrbrakk\ \isasymLongrightarrow\ P\ \isasymand\ Q\
%\isasymand\ P\isanewline
\ 1.\ {\isasymlbrakk}P;\ Q\isasymrbrakk\ \isasymLongrightarrow\ P\isanewline
\ 2.\ {\isasymlbrakk}P;\ Q\isasymrbrakk\ \isasymLongrightarrow\ Q\ \isasymand\ P
\end{isabelle}
Isabelle leaves two new subgoals: the two halves of the original conjunction. 
The first is simply \isa{P}, which is trivial, since \isa{P} is among 
the assumptions.  We can apply the {\isa{assumption}} 
method, which proves a subgoal by finding a matching assumption.
\begin{isabelle}
\ 1.\ {\isasymlbrakk}P;\ Q\isasymrbrakk\ \isasymLongrightarrow\ 
Q\ \isasymand\ P
\end{isabelle}
We are left with the subgoal of proving  
\isa{Q\ \isasymand\ P} from the assumptions \isa{P} and~\isa{Q}.  We apply
\isa{rule conjI} again. 
\begin{isabelle}
\ 1.\ {\isasymlbrakk}P;\ Q\isasymrbrakk\ \isasymLongrightarrow\ Q\isanewline
\ 2.\ {\isasymlbrakk}P;\ Q\isasymrbrakk\ \isasymLongrightarrow\ P
\end{isabelle}
We are left with two new subgoals, \isa{Q} and~\isa{P}, each of which can be proved
using the {\isa{assumption}} method. 


\section{Elimination rules}

\textbf{Elimination} rules work in the opposite direction from introduction 
rules. In the case of conjunction, there are two such rules. 
From $P\conj Q$ we infer $P$. also, from $P\conj Q$  
we infer $Q$:
\[ \infer{P}{P\conj Q} \qquad \infer{Q}{P\conj Q}  \]

Now consider disjunction. There are two introduction rules, which resemble inverted forms of the
conjunction elimination rules:
\[ \infer{P\disj Q}{P} \qquad \infer{P\disj Q}{Q}  \]

What is the disjunction elimination rule?  The situation is rather different from 
conjunction.  From $P\disj Q$ we cannot conclude  that $P$ is true and we
cannot conclude that $Q$ is true; there are no direct
elimination rules of the sort that we have seen for conjunction.  Instead,
there is an elimination  rule that works indirectly.  If we are trying  to prove
something else, say $R$, and we know that $P\disj Q$ holds,  then we have to consider
two cases.  We can assume that $P$ is true  and prove $R$ and then assume that $Q$ is
true and prove $R$ a second  time.  Here we see a fundamental concept used in natural
deduction:  that of the \textbf{assumptions}. We have to prove $R$ twice, under
different assumptions.  The assumptions are local to these subproofs and are visible 
nowhere else. 

In a logic text, the disjunction elimination rule might be shown 
like this:
\[ \infer{R}{P\disj Q & \infer*{R}{[P]} & \infer*{R}{[Q]}} \]
The assumptions $[P]$ and $[Q]$ are bracketed 
to emphasize that they are local to their subproofs.  In Isabelle 
notation, the already-familiar \isa\isasymLongrightarrow syntax serves the
same  purpose:
\begin{isabelle}
\isasymlbrakk?P\ \isasymor\ ?Q;\ ?P\ \isasymLongrightarrow\ ?R;\ ?Q\ \isasymLongrightarrow\ ?R\isasymrbrakk\ \isasymLongrightarrow\ ?R\rulename{disjE}
\end{isabelle}
When we use this sort of elimination rule backwards, it produces 
a case split.  (We have this before, in proofs by induction.)  The following  proof
illustrates the use of disjunction elimination.  
\begin{isabelle}
\isacommand{lemma}\ disj_swap:\ "P\ \isasymor\ Q\ 
\isasymLongrightarrow\ Q\ \isasymor\ P"\isanewline
\isacommand{apply}\ (erule\ disjE)\isanewline
\ \isacommand{apply}\ (rule\ disjI2)\isanewline
\ \isacommand{apply}\ assumption\isanewline
\isacommand{apply}\ (rule\ disjI1)\isanewline
\isacommand{apply}\ assumption
\end{isabelle}
We assume \isa{P\ \isasymor\ Q} and
must prove \isa{Q\ \isasymor\ P}\@.  Our first step uses the disjunction
elimination rule, \isa{disjE}.  The method {\isa{erule}}  applies an
elimination rule to the assumptions, searching for one that matches the
rule's first premise.  Deleting that assumption, it
return the subgoals for the remaining premises.  Most of the
time, this is  the best way to use elimination rules; only rarely is there
any  point in keeping the assumption.

\begin{isabelle}
%P\ \isasymor\ Q\ \isasymLongrightarrow\ Q\ \isasymor\ P\isanewline
\ 1.\ P\ \isasymLongrightarrow\ Q\ \isasymor\ P\isanewline
\ 2.\ Q\ \isasymLongrightarrow\ Q\ \isasymor\ P
\end{isabelle}
Here it leaves us with two subgoals.  The first assumes \isa{P} and the 
second assumes \isa{Q}.  Tackling the first subgoal, we need to 
show \isa{Q\ \isasymor\ P}\@.  The second introduction rule (\isa{disjI2})
can reduce this  to \isa{P}, which matches the assumption. So, we apply the
{\isa{rule}}  method with \isa{disjI2} \ldots
\begin{isabelle}
\ 1.\ P\ \isasymLongrightarrow\ P\isanewline
\ 2.\ Q\ \isasymLongrightarrow\ Q\ \isasymor\ P
\end{isabelle}
\ldots and finish off with the {\isa{assumption}} 
method.  We are left with the other subgoal, which 
assumes \isa{Q}.  
\begin{isabelle}
\ 1.\ Q\ \isasymLongrightarrow\ Q\ \isasymor\ P
\end{isabelle}
Its proof is similar, using the introduction 
rule \isa{disjI1}. 

The result of this proof is a new inference rule \isa{disj_swap}, which is neither 
an introduction nor an elimination rule, but which might 
be useful.  We can use it to replace any goal of the form $Q\disj P$
by a one of the form $P\disj Q$.



\section{Destruction rules: some examples}

Now let us examine the analogous proof for conjunction. 
\begin{isabelle}
\isacommand{lemma}\ conj_swap:\ "P\ \isasymand\ Q\ \isasymLongrightarrow\ Q\ \isasymand\ P"\isanewline
\isacommand{apply}\ (rule\ conjI)\isanewline
\ \isacommand{apply}\ (drule\ conjunct2)\isanewline
\ \isacommand{apply}\ assumption\isanewline
\isacommand{apply}\ (drule\ conjunct1)\isanewline
\isacommand{apply}\ assumption
\end{isabelle}
Recall that the conjunction elimination rules --- whose Isabelle names are 
\isa{conjunct1} and \isa{conjunct2} --- simply return the first or second half
of a conjunction.  Rules of this sort (where the conclusion is a subformula of a
premise) are called \textbf{destruction} rules, by analogy with the destructor
functions of functional programming.%
\footnote{This Isabelle terminology has no counterpart in standard logic texts, 
although the distinction between the two forms of elimination rule is well known. 
Girard \cite[page 74]{girard89}, for example, writes ``The elimination rules are very
bad.  What is catastrophic about them is the parasitic presence of a formula [$R$]
which has no structural link with the formula which is eliminated.''}

The first proof step applies conjunction introduction, leaving 
two subgoals: 
\begin{isabelle}
%P\ \isasymand\ Q\ \isasymLongrightarrow\ Q\ \isasymand\ P\isanewline
\ 1.\ P\ \isasymand\ Q\ \isasymLongrightarrow\ Q\isanewline
\ 2.\ P\ \isasymand\ Q\ \isasymLongrightarrow\ P
\end{isabelle}

To invoke the elimination rule, we apply a new method, \isa{drule}. 
Think of the \isa{d} as standing for \textbf{destruction} (or \textbf{direct}, if
you prefer).   Applying the 
second conjunction rule using \isa{drule} replaces the assumption 
\isa{P\ \isasymand\ Q} by \isa{Q}. 
\begin{isabelle}
\ 1.\ Q\ \isasymLongrightarrow\ Q\isanewline
\ 2.\ P\ \isasymand\ Q\ \isasymLongrightarrow\ P
\end{isabelle}
The resulting subgoal can be proved by applying \isa{assumption}.
The other subgoal is similarly proved, using the \isa{conjunct1} rule and the 
\isa{assumption} method.

Choosing among the methods \isa{rule}, \isa{erule} and \isa{drule} is up to 
you.  Isabelle does not attempt to work out whether a rule 
is an introduction rule or an elimination rule.  The 
method determines how the rule will be interpreted. Many rules 
can be used in more than one way.  For example, \isa{disj_swap} can 
be applied to assumptions as well as to goals; it replaces any
assumption of the form
$P\disj Q$ by a one of the form $Q\disj P$.

Destruction rules are simpler in form than indirect rules such as \isa{disjE},
but they can be inconvenient.  Each of the conjunction rules discards half 
of the formula, when usually we want to take both parts of the conjunction as new
assumptions.  The easiest way to do so is by using an 
alternative conjunction elimination rule that resembles \isa{disjE}.  It is seldom,
if ever, seen in logic books.  In Isabelle syntax it looks like this: 
\begin{isabelle}
\isasymlbrakk?P\ \isasymand\ ?Q;\ \isasymlbrakk?P;\ ?Q\isasymrbrakk\ \isasymLongrightarrow\ ?R\isasymrbrakk\ \isasymLongrightarrow\ ?R\rulename{conjE}
\end{isabelle}

\begin{exercise}
Use the rule {\isa{conjE}} to shorten the proof above. 
\end{exercise}


\section{Implication}

At the start of this chapter, we saw the rule \textit{modus ponens}.  It is, in fact,
a destruction rule. The matching introduction rule looks like this 
in Isabelle: 
\begin{isabelle}
(?P\ \isasymLongrightarrow\ ?Q)\ \isasymLongrightarrow\ ?P\
\isasymlongrightarrow\ ?Q\rulename{impI}
\end{isabelle}
And this is \textit{modus ponens}:
\begin{isabelle}
\isasymlbrakk?P\ \isasymlongrightarrow\ ?Q;\ ?P\isasymrbrakk\
\isasymLongrightarrow\ ?Q
\rulename{mp}
\end{isabelle}

Here is a proof using the rules for implication.  This 
lemma performs a sort of uncurrying, replacing the two antecedents 
of a nested implication by a conjunction. 
\begin{isabelle}
\isacommand{lemma}\ imp_uncurry:\
"P\ \isasymlongrightarrow\ (Q\
\isasymlongrightarrow\ R)\ \isasymLongrightarrow\ P\
\isasymand\ Q\ \isasymlongrightarrow\
R"\isanewline
\isacommand{apply}\ (rule\ impI)\isanewline
\isacommand{apply}\ (erule\ conjE)\isanewline
\isacommand{apply}\ (drule\ mp)\isanewline
\ \isacommand{apply}\ assumption\isanewline
\isacommand{apply}\ (drule\ mp)\isanewline
\ \ \isacommand{apply}\ assumption\isanewline
\ \isacommand{apply}\ assumption
\end{isabelle}
First, we state the lemma and apply implication introduction (\isa{rule impI}), 
which moves the conjunction to the assumptions. 
\begin{isabelle}
%P\ \isasymlongrightarrow\ Q\ \isasymlongrightarrow\ R\ \isasymLongrightarrow\ P\
%\isasymand\ Q\ \isasymlongrightarrow\ R\isanewline
\ 1.\ {\isasymlbrakk}P\ \isasymlongrightarrow\ Q\ \isasymlongrightarrow\ R;\ P\ \isasymand\ Q\isasymrbrakk\ \isasymLongrightarrow\ R
\end{isabelle}
Next, we apply conjunction elimination (\isa{erule conjE}), which splits this
conjunction into two  parts. 
\begin{isabelle}
\ 1.\ {\isasymlbrakk}P\ \isasymlongrightarrow\ Q\ \isasymlongrightarrow\ R;\ P;\
Q\isasymrbrakk\ \isasymLongrightarrow\ R
\end{isabelle}
Now, we work on the assumption \isa{P\ \isasymlongrightarrow\ (Q\
\isasymlongrightarrow\ R)}, where the parentheses have been inserted for
clarity.  The nested implication requires two applications of
\textit{modus ponens}: \isa{drule mp}.  The first use  yields the
implication \isa{Q\
\isasymlongrightarrow\ R}, but first we must prove the extra subgoal 
\isa{P}, which we do by assumption. 
\begin{isabelle}
\ 1.\ {\isasymlbrakk}P;\ Q\isasymrbrakk\ \isasymLongrightarrow\ P\isanewline
\ 2.\ {\isasymlbrakk}P;\ Q;\ Q\ \isasymlongrightarrow\ R\isasymrbrakk\ \isasymLongrightarrow\ R
\end{isabelle}
Repeating these steps for \isa{Q\
\isasymlongrightarrow\ R} yields the conclusion we seek, namely~\isa{R}.
\begin{isabelle}
\ 1.\ {\isasymlbrakk}P;\ Q;\ Q\ \isasymlongrightarrow\ R\isasymrbrakk\
\isasymLongrightarrow\ R
\end{isabelle}

The symbols \isa{\isasymLongrightarrow} and \isa{\isasymlongrightarrow}
both stand for implication, but they differ in many respects.  Isabelle
uses \isa{\isasymLongrightarrow} to express inference rules; the symbol is
built-in and Isabelle's inference mechanisms treat it specially.  On the
other hand, \isa{\isasymlongrightarrow} is just one of the many connectives
available in higher-order logic.  We reason about it using inference rules
such as \isa{impI} and \isa{mp}, just as we reason about the other
connectives.  You will have to use \isa{\isasymlongrightarrow} in any
context that requires a formula of higher-order logic.  Use
\isa{\isasymLongrightarrow} to separate a theorem's preconditions from its
conclusion.  

When using induction, often the desired theorem results in an induction
hypothesis that is too weak.  In such cases you may have to invent a more
complicated induction formula, typically involving
\isa{\isasymlongrightarrow} and \isa{\isasymforall}.  From this lemma you
derive the desired theorem , typically involving
\isa{\isasymLongrightarrow}.  We shall see an example below,
\S\ref{sec:proving-euclid}.


\section{Unification and substitution}\label{sec:unification}

As we have seen, Isabelle rules involve variables that begin  with a
question mark. These are called \textbf{schematic} variables  and act as
placeholders for terms. \textbf{Unification} refers to  the process of
making two terms identical, possibly by replacing  their variables by
terms. The simplest case is when the two terms  are already the same. Next
simplest is when the variables in only one of the term
 are replaced; this is called \textbf{pattern-matching}.  The
{\isa{rule}} method typically  matches the rule's conclusion
against the current subgoal.  In the most complex case,  variables in both
terms are replaced; the {\isa{rule}} method can do this the goal
itself contains schematic variables.  Other occurrences of the variables in
the rule or proof state are updated at the same time.

Schematic variables in goals are sometimes called \textbf{unknowns}.  They
are useful because they let us proceed with a proof even  when we do not
know what certain terms should be --- as when the goal is $\exists x.\,P$. 
They can be  filled in later, often automatically. 

 Unification is well known to Prolog programmers. Isabelle uses \textbf{higher-order} 
unification, which is unification in the
typed $\lambda$-calculus.  The general case is
undecidable, but for our purposes, the differences from ordinary
unification are straightforward.  It handles bound  variables
correctly, avoiding capture.  The two terms \isa{{\isasymlambda}x.\ ?P} and
\isa{{\isasymlambda}x.\ t x}  are not unifiable; replacing \isa{?P} by
\isa{t x} is forbidden because the free occurrence of~\isa{x} would become
bound.  The two terms
\isa{{\isasymlambda}x.\ f(x,z)} and \isa{{\isasymlambda}y.\ f(y,z)} are
trivially unifiable because they differ only by a bound variable renaming.

Higher-order unification sometimes must invent
$\lambda$-terms to replace function  variables,
which can lead to a combinatorial explosion. However,  Isabelle proofs tend
to involve easy cases where there are few possibilities for the
$\lambda$-term being constructed. In the easiest case, the
function variable is applied only to bound variables, 
as when we try to unify \isa{{\isasymlambda}x\ y.\ f(?h x y)} and
\isa{{\isasymlambda}x\ y.\ f(x+y+a)}.  The only solution is to replace
\isa{?h} by \isa{{\isasymlambda}x\ y.\ x+y+a}.  Such cases admit at most
one unifier, like ordinary unification.  A harder case is
unifying \isa{?h a} with~\isa{a+b}; it admits two solutions for \isa{?h},
namely \isa{{\isasymlambda}x.~a+b} and \isa{{\isasymlambda}x.~x+b}. 
Unifying \isa{?h a} with~\isa{a+a+b} admits four solutions; their number is
exponential in the number of occurrences of~\isa{a} in the second term.

Isabelle also uses function variables to express \textbf{substitution}. 
A typical substitution rule allows us to replace one term by 
another if we know that two terms are equal. 
\[ \infer{P[t/x]}{s=t & P[s/x]} \]
The conclusion uses a notation for substitution: $P[t/x]$ is the result of
replacing $x$ by~$t$ in~$P$.  The rule only substitutes in the positions
designated by~$x$, which gives it additional power. For example, it can
derive symmetry of equality from reflexivity.  Using $x=s$ for~$P$
replaces just the first $s$ in $s=s$ by~$t$.
\[ \infer{t=s}{s=t & \infer{s=s}{}} \]

The Isabelle version of the substitution rule looks like this: 
\begin{isabelle}
\isasymlbrakk?t\ =\ ?s;\ ?P\ ?s\isasymrbrakk\ \isasymLongrightarrow\ ?P\
?t
\rulename{ssubst}
\end{isabelle}
Crucially, \isa{?P} is a function 
variable: it can be replaced by a $\lambda$-expression 
involving one bound variable whose occurrences identify the places 
in which $s$ will be replaced by~$t$.  The proof above requires
\isa{{\isasymlambda}x.~x=s}.

The \isa{simp} method replaces equals by equals, but using the substitution
rule gives us more control. Consider this proof: 
\begin{isabelle}
\isacommand{lemma}\
"{\isasymlbrakk}\ x\
=\ f\ x;\ odd(f\
x)\ \isasymrbrakk\ \isasymLongrightarrow\ odd\
x"\isanewline
\isacommand{apply}\ (erule\ ssubst)\isanewline
\isacommand{apply}\ assumption\isanewline
\isacommand{done}\end{isabelle}
%
The simplifier might loop, replacing \isa{x} by \isa{f x} and then by
\isa{f(f x)} and so forth. (Actually, \isa{simp} 
sees the danger and re-orients this equality, but in more complicated cases
it can be fooled.) When we apply substitution,  Isabelle replaces every
\isa{x} in the subgoal by \isa{f x} just once: it cannot loop.  The
resulting subgoal is trivial by assumption. 

We are using the \isa{erule} method it in a novel way. Hitherto, 
the conclusion of the rule was just a variable such as~\isa{?R}, but it may
be any term. The conclusion is unified with the subgoal just as 
it would be with the \isa{rule} method. At the same time \isa{erule} looks 
for an assumption that matches the rule's first premise, as usual.  With
\isa{ssubst} the effect is to find, use and delete an equality 
assumption.


Higher-order unification can be tricky, as this example indicates: 
\begin{isabelle}
\isacommand{lemma}\ "{\isasymlbrakk}\ x\ =\
f\ x;\ triple\ (f\ x)\
(f\ x)\ x\ \isasymrbrakk\
\isasymLongrightarrow\ triple\ x\ x\ x"\isanewline
\isacommand{apply}\ (erule\ ssubst)\isanewline
\isacommand{back}\isanewline
\isacommand{back}\isanewline
\isacommand{back}\isanewline
\isacommand{back}\isanewline
\isacommand{apply}\ assumption\isanewline
\isacommand{done}
\end{isabelle}
%
By default, Isabelle tries to substitute for all the 
occurrences.  Applying \isa{erule\ ssubst} yields this subgoal:
\begin{isabelle}
\ 1.\ triple\ (f\ x)\ (f\ x)\ x\ \isasymLongrightarrow\ triple\ (f\ x)\ (f\ x)\ (f\ x)
\end{isabelle}
The substitution should have been done in the first two occurrences 
of~\isa{x} only. Isabelle has gone too far. The \isa{back} 
method allows us to reject this possibility and get a new one: 
\begin{isabelle}
\ 1.\ triple\ (f\ x)\ (f\ x)\ x\ \isasymLongrightarrow\ triple\ x\ (f\ x)\ (f\ x)
\end{isabelle}
%
Now Isabelle has left the first occurrence of~\isa{x} alone. That is 
promising but it is not the desired combination. So we use \isa{back} 
again:
\begin{isabelle}
\ 1.\ triple\ (f\ x)\ (f\ x)\ x\ \isasymLongrightarrow\ triple\ (f\ x)\ x\ (f\ x)
\end{isabelle}
%
This also is wrong, so we use \isa{back} again: 
\begin{isabelle}
\ 1.\ triple\ (f\ x)\ (f\ x)\ x\ \isasymLongrightarrow\ triple\ x\ x\ (f\ x)
\end{isabelle}
%
And this one is wrong too. Looking carefully at the series 
of alternatives, we see a binary countdown with reversed bits: 111,
011, 101, 001.  Invoke \isa{back} again: 
\begin{isabelle}
\ 1.\ triple\ (f\ x)\ (f\ x)\ x\ \isasymLongrightarrow\ triple\ (f\ x)\ (f\ x)\ x%
\end{isabelle}
At last, we have the right combination!  This goal follows by assumption.

Never use {\isa{back}} in the final version of a proof. 
It should only be used for exploration. One way to get rid of {\isa{back}} 
to combine two methods in a single \textbf{apply} command. Isabelle 
applies the first method and then the second. If the second method 
fails then Isabelle automatically backtracks. This process continues until 
the first method produces an output that the second method can 
use. We get a one-line proof of our example: 
\begin{isabelle}
\isacommand{lemma}\
"{\isasymlbrakk}\ x\
=\ f\ x;\ triple\ (f\
x)\ (f\ x)\ x\
\isasymrbrakk\
\isasymLongrightarrow\ triple\ x\ x\ x"\isanewline
\isacommand{apply}\ (erule\ ssubst,\ assumption)\isanewline
\isacommand{done}
\end{isabelle}

The most general way to get rid of the {\isa{back}} command is 
to instantiate variables in the rule.  The method {\isa{rule\_tac}} is
similar to \isa{rule}, but it
makes some of the rule's variables  denote specified terms.  
Also available are {\isa{drule\_tac}}  and \isa{erule\_tac}.  Here we need
\isa{erule\_tac} since above we used
\isa{erule}.
\begin{isabelle}
\isacommand{lemma}\ "{\isasymlbrakk}\ x\ =\ f\ x;\ triple\ (f\ x)\ (f\ x)\ x\ \isasymrbrakk\ \isasymLongrightarrow\ triple\ x\ x\ x"\isanewline
\isacommand{apply}\ (erule_tac\
P="{\isasymlambda}u.\ triple\ u\
u\ x"\ \isakeyword{in}\
ssubst)\isanewline
\isacommand{apply}\ assumption\isanewline
\isacommand{done}
\end{isabelle}
%
To specify a desired substitution 
requires instantiating the variable \isa{?P} with a $\lambda$-expression. 
The bound variable occurrences in \isa{{\isasymlambda}u.\ P\ u\
u\ x} indicate that the first two arguments have to be substituted, leaving
the third unchanged.

An alternative to {\isa{rule\_tac}} is to use \isa{rule} with the
{\isa{of}}  directive, described in \S\ref{sec:forward} below.   An
advantage  of {\isa{rule\_tac}} is that the instantiations may refer to 
variables bound in the current subgoal.


\section{Negation}
 
Negation causes surprising complexity in proofs.  Its natural 
deduction rules are straightforward, but additional rules seem 
necessary in order to handle negated assumptions gracefully. 

Negation introduction deduces $\neg P$ if assuming $P$ leads to a 
contradiction. Negation elimination deduces any formula in the 
presence of $\neg P$ together with~$P$: 
\begin{isabelle}
(?P\ \isasymLongrightarrow\ False)\ \isasymLongrightarrow\ \isasymnot\ ?P%
\rulename{notI}\isanewline
\isasymlbrakk{\isasymnot}\ ?P;\ ?P\isasymrbrakk\ \isasymLongrightarrow\ ?R%
\rulename{notE}
\end{isabelle}
%
Classical logic allows us to assume $\neg P$ 
when attempting to prove~$P$: 
\begin{isabelle}
(\isasymnot\ ?P\ \isasymLongrightarrow\ ?P)\ \isasymLongrightarrow\ ?P%
\rulename{classical}
\end{isabelle}
%
Three further rules are variations on the theme of contrapositive. 
They differ in the placement of the negation symbols: 
\begin{isabelle}
\isasymlbrakk?Q;\ \isasymnot\ ?P\ \isasymLongrightarrow\ \isasymnot\ ?Q\isasymrbrakk\ \isasymLongrightarrow\ ?P%
\rulename{contrapos_pp}\isanewline
\isasymlbrakk{\isasymnot}\ ?Q;\ \isasymnot\ ?P\ \isasymLongrightarrow\ ?Q\isasymrbrakk\ \isasymLongrightarrow\ ?P%
\rulename{contrapos_np}\isanewline
\isasymlbrakk{\isasymnot}\ ?Q;\ ?P\ \isasymLongrightarrow\ ?Q\isasymrbrakk\ \isasymLongrightarrow\ \isasymnot\ ?P%
\rulename{contrapos_nn}
\end{isabelle}
%
These rules are typically applied using the {\isa{erule}} method, where 
their effect is to form a contrapositive from an 
assumption and the goal's conclusion.  

The most important of these is \isa{contrapos_np}.  It is useful
for applying introduction rules to negated assumptions.  For instance, 
the assumption $\neg(P\imp Q)$ is equivalent to the conclusion $P\imp Q$ and we 
might want to use conjunction introduction on it. 
Before we can do so, we must move that assumption so that it 
becomes the conclusion. The following proof demonstrates this 
technique: 
\begin{isabelle}
\isacommand{lemma}\ "\isasymlbrakk{\isasymnot}(P{\isasymlongrightarrow}Q);\
\isasymnot(R{\isasymlongrightarrow}Q)\isasymrbrakk\ \isasymLongrightarrow\
R"\isanewline
\isacommand{apply}\ (erule_tac\ Q="R{\isasymlongrightarrow}Q"\ \isakeyword{in}\
contrapos_np)\isanewline
\isacommand{apply}\ intro\isanewline
\isacommand{apply}\ (erule\ notE,\ assumption)\isanewline
\isacommand{done}
\end{isabelle}
%
There are two negated assumptions and we need to exchange the conclusion with the
second one.  The method \isa{erule contrapos_np} would select the first assumption,
which we do not want.  So we specify the desired assumption explicitly, using
\isa{erule_tac}.  This is the resulting subgoal: 
\begin{isabelle}
\ 1.\ \isasymlbrakk{\isasymnot}\ (P\ \isasymlongrightarrow\ Q);\ \isasymnot\
R\isasymrbrakk\ \isasymLongrightarrow\ R\ \isasymlongrightarrow\ Q%
\end{isabelle}
The former conclusion, namely \isa{R}, now appears negated among the assumptions,
while the negated formula \isa{R\ \isasymlongrightarrow\ Q} becomes the new
conclusion.

We can now apply introduction rules.  We use the {\isa{intro}} method, which
repeatedly  applies built-in introduction rules.  Here its effect is equivalent
to \isa{rule impI}.\begin{isabelle}
\ 1.\ \isasymlbrakk{\isasymnot}\ (P\ \isasymlongrightarrow\ Q);\ \isasymnot\ R;\
R\isasymrbrakk\ \isasymLongrightarrow\ Q%
\end{isabelle}
We can see a contradiction in the form of assumptions \isa{\isasymnot\ R}
and~\isa{R}, which suggests using negation elimination.  If applied on its own,
however, it will select the first negated assumption, which is useless.   Instead,
we combine the rule with  the
\isa{assumption} method:
\begin{isabelle}
\ \ \ \ \ (erule\ notE,\ assumption)
\end{isabelle}
Now when Isabelle selects the first assumption, it tries to prove \isa{P\
\isasymlongrightarrow\ Q} and fails; it then backtracks, finds the 
assumption~\isa{\isasymnot\ R} and finally proves \isa{R} by assumption.  That
concludes the proof.

\medskip

Here is another example. 
\begin{isabelle}
\isacommand{lemma}\ "(P\ \isasymor\ Q)\ \isasymand\ R\
\isasymLongrightarrow\ P\ \isasymor\ Q\ \isasymand\ R"\isanewline
\isacommand{apply}\ intro%


\isacommand{apply}\ (elim\ conjE\ disjE)\isanewline
\ \isacommand{apply}\ assumption
\isanewline
\isacommand{apply}\ (erule\ contrapos_np,\ rule\ conjI)\isanewline
\ \ \isacommand{apply}\ assumption\isanewline
\ \isacommand{apply}\ assumption\isanewline
\isacommand{done}
\end{isabelle}
%
The first proof step applies the {\isa{intro}} method, which repeatedly 
uses built-in introduction rules.  Here it creates the negative assumption \isa{\isasymnot\ (Q\ \isasymand\
R)}.
\begin{isabelle}
\ 1.\ \isasymlbrakk(P\ \isasymor\ Q)\ \isasymand\ R;\ \isasymnot\ (Q\ \isasymand\
R)\isasymrbrakk\ \isasymLongrightarrow\ P%
\end{isabelle}
It comes from \isa{disjCI},  a disjunction introduction rule that is more
powerful than the separate rules  \isa{disjI1} and  \isa{disjI2}.

Next we apply the {\isa{elim}} method, which repeatedly applies 
elimination rules; here, the elimination rules given 
in the command.  One of the subgoals is trivial, leaving us with one other:
\begin{isabelle}
\ 1.\ \isasymlbrakk{\isasymnot}\ (Q\ \isasymand\ R);\ R;\ Q\isasymrbrakk\ \isasymLongrightarrow\ P%
\end{isabelle}
%
Now we must move the formula \isa{Q\ \isasymand\ R} to be the conclusion.  The
combination 
\begin{isabelle}
\ \ \ \ \ (erule\ contrapos_np,\ rule\ conjI)
\end{isabelle}
is robust: the \isa{conjI} forces the \isa{erule} to select a
conjunction.  The two subgoals are the ones we would expect from applying
conjunction introduction to
\isa{Q\
\isasymand\ R}:  
\begin{isabelle}
\ 1.\ {\isasymlbrakk}R;\ Q;\ \isasymnot\ P\isasymrbrakk\ \isasymLongrightarrow\
Q\isanewline
\ 2.\ {\isasymlbrakk}R;\ Q;\ \isasymnot\ P\isasymrbrakk\ \isasymLongrightarrow\ R%
\end{isabelle}
The rest of the proof is trivial.


\section{The universal quantifier}

Quantifiers require formalizing syntactic substitution and the notion of \textbf{arbitrary
value}.  Consider the universal quantifier.  In a logic book, its
introduction  rule looks like this: 
\[ \infer{\forall x.\,P}{P} \]
Typically, a proviso written in English says that $x$ must not
occur in the assumptions.  This proviso guarantees that $x$ can be regarded as
arbitrary, since it has not been assumed to satisfy any special conditions. 
Isabelle's  underlying formalism, called the
\textbf{meta-logic}, eliminates the  need for English.  It provides its own universal
quantifier (\isasymAnd) to express the notion of an arbitrary value.  We have
already seen  another symbol of the meta-logic, namely
\isa\isasymLongrightarrow, which expresses  inference rules and the treatment of
assumptions. The only other  symbol in the meta-logic is \isa\isasymequiv, which
can be used to define constants.

Returning to the universal quantifier, we find that having a similar quantifier
as part of the meta-logic makes the introduction rule trivial to express:
\begin{isabelle}
({\isasymAnd}x.\ ?P\ x)\ \isasymLongrightarrow\ {\isasymforall}x.\ ?P\ x\rulename{allI}
\end{isabelle}


The following trivial proof demonstrates how the universal introduction 
rule works. 
\begin{isabelle}
\isacommand{lemma}\ "{\isasymforall}x.\ P\ x\ \isasymlongrightarrow\ P\ x"\isanewline
\isacommand{apply}\ (rule\ allI)\isanewline
\isacommand{apply}\ (rule\ impI)\isanewline
\isacommand{apply}\ assumption
\end{isabelle}
The first step invokes the rule by applying the method \isa{rule allI}. 
\begin{isabelle}
%{\isasymforall}x.\ P\ x\ \isasymlongrightarrow\ P\ x\isanewline
\ 1.\ {\isasymAnd}x.\ P\ x\ \isasymlongrightarrow\ P\ x
\end{isabelle}
Note  that the resulting proof state has a bound variable,
namely~\bigisa{x}.  The rule has replaced the universal quantifier of
higher-order  logic by Isabelle's meta-level quantifier.  Our goal is to
prove
\isa{P\ x\ \isasymlongrightarrow\ P\ x} for arbitrary~\isa{x}; it is 
an implication, so we apply the corresponding introduction rule (\isa{impI}). 
\begin{isabelle}
\ 1.\ {\isasymAnd}x.\ P\ x\ \isasymLongrightarrow\ P\ x
\end{isabelle}
The {\isa{assumption}} method proves this last subgoal. 

\medskip
Now consider universal elimination. In a logic text, 
the rule looks like this: 
\[ \infer{P[t/x]}{\forall x.\,P} \]
The conclusion is $P$ with $t$ substituted for the variable~$x$.  
Isabelle expresses substitution using a function variable: 
\begin{isabelle}
{\isasymforall}x.\ ?P\ x\ \isasymLongrightarrow\ ?P\ ?x\rulename{spec}
\end{isabelle}
This destruction rule takes a 
universally quantified formula and removes the quantifier, replacing 
the bound variable \bigisa{x} by the schematic variable \bigisa{?x}.  Recall that a
schematic variable starts with a question mark and acts as a
placeholder: it can be replaced by any term. 

To see how this works, let us derive a rule about reducing 
the scope of a universal quantifier.  In mathematical notation we write
\[ \infer{P\imp\forall x.\,Q}{\forall x.\,P\imp Q} \]
with the proviso `$x$ not free in~$P$.'  Isabelle's treatment of
substitution makes the proviso unnecessary.  The conclusion is expressed as
\isa{P\
\isasymlongrightarrow\ ({\isasymforall}x.\ Q\ x)}. No substitution for the
variable \isa{P} can introduce a dependence upon~\isa{x}: that would be a
bound variable capture.  Here is the isabelle proof in full:
\begin{isabelle}
\isacommand{lemma}\ "({\isasymforall}x.\ P\
\isasymlongrightarrow\ Q\ x)\ \isasymLongrightarrow\ P\
\isasymlongrightarrow\ ({\isasymforall}x.\ Q\ x)"\isanewline
\isacommand{apply}\ (rule\ impI)\isanewline
\isacommand{apply}\ (rule\ allI)\isanewline
\isacommand{apply}\ (drule\ spec)\isanewline
\isacommand{apply}\ (drule\ mp)\isanewline
\ \ \isacommand{apply}\ assumption\isanewline
\ \isacommand{apply}\ assumption
\end{isabelle}
First we apply implies introduction (\isa{rule impI}), 
which moves the \isa{P} from the conclusion to the assumptions. Then 
we apply universal introduction (\isa{rule allI}).  
\begin{isabelle}
%{\isasymforall}x.\ P\ \isasymlongrightarrow\ Q\ x\ \isasymLongrightarrow\ P\
%\isasymlongrightarrow\ ({\isasymforall}x.\ Q\ x)\isanewline
\ 1.\ {\isasymAnd}x.\ \isasymlbrakk{\isasymforall}x.\ P\ \isasymlongrightarrow\ Q\ x;\ P\isasymrbrakk\ \isasymLongrightarrow\ Q\ x
\end{isabelle}
As before, it replaces the HOL 
quantifier by a meta-level quantifier, producing a subgoal that 
binds the variable~\bigisa{x}.  The leading bound variables
(here \isa{x}) and the assumptions (here \isa{{\isasymforall}x.\ P\
\isasymlongrightarrow\ Q\ x} and \isa{P}) form the \textbf{context} for the
conclusion, here \isa{Q\ x}.  At each proof step, the subgoals inherit the
previous context, though some context elements may be added or deleted. 
Applying \isa{erule} deletes an assumption, while many natural deduction
rules add bound variables or assumptions.

Now, to reason from the universally quantified 
assumption, we apply the elimination rule using the {\isa{drule}} 
method.  This rule is called \isa{spec} because it specializes a universal formula
to a particular term.
\begin{isabelle}
\ 1.\ {\isasymAnd}x.\ {\isasymlbrakk}P;\ P\ \isasymlongrightarrow\ Q\ (?x2\
x){\isasymrbrakk}\ \isasymLongrightarrow\ Q\ x
\end{isabelle}
Observe how the context has changed.  The quantified formula is gone,
replaced by a new assumption derived from its body.  Informally, we have
removed the quantifier.  The quantified variable
has been replaced by the curious term 
\bigisa{?x2~x}; it acts as a placeholder that may be replaced 
by any term that can be built up from~\bigisa{x}.  (Formally, \bigisa{?x2} is an
unknown of function type, applied to the argument~\bigisa{x}.)  This new assumption is
an implication, so we can  use \emph{modus ponens} on it. As before, it requires
proving the  antecedent (in this case \isa{P}) and leaves us with the consequent. 
\begin{isabelle}
\ 1.\ {\isasymAnd}x.\ {\isasymlbrakk}P;\ Q\ (?x2\ x){\isasymrbrakk}\
\isasymLongrightarrow\ Q\ x
\end{isabelle}
The consequent is \isa{Q} applied to that placeholder.  It may be replaced by any
term built from~\bigisa{x}, and here 
it should simply be~\bigisa{x}.  The \isa{assumption} method will do this.
The assumption need not be identical to the conclusion, provided the two formulas are
unifiable.  

\medskip
Note that \isa{drule spec} removes the universal quantifier and --- as
usual with elimination rules --- discards the original formula.  Sometimes, a
universal formula has to be kept so that it can be used again.  Then we use a new
method: \isa{frule}.  It acts like \isa{drule} but copies rather than replaces
the selected assumption.  The \isa{f} is for `forward.'

In this example, we intuitively see that to go from \isa{P\ a} to \isa{P(f\ (f\
a))} requires two uses of the quantified assumption, one for each
additional~\isa{f}.
\begin{isabelle}
\isacommand{lemma}\ "\isasymlbrakk{\isasymforall}x.\ P\ x\ \isasymlongrightarrow\ P\ (f\ x);
\ P\ a\isasymrbrakk\ \isasymLongrightarrow\ P(f\ (f\ a))"\isanewline
\isacommand{apply}\ (frule\ spec)\isanewline
\isacommand{apply}\ (drule\ mp,\ assumption)\isanewline
\isacommand{apply}\ (drule\ spec)\isanewline
\isacommand{apply}\ (drule\ mp,\ assumption,\ assumption)\isanewline
\isacommand{done}
\end{isabelle}
%
Applying \isa{frule\ spec} leaves this subgoal:
\begin{isabelle}
\ 1.\ \isasymlbrakk{\isasymforall}x.\ P\ x\ \isasymlongrightarrow\ P\ (f\ x);\ P\ a;\ P\ ?x\ \isasymlongrightarrow\ P\ (f\ ?x)\isasymrbrakk\ \isasymLongrightarrow\ P\ (f\ (f\ a))
\end{isabelle}
It is just what  \isa{drule} would have left except that the quantified
assumption is still present.  The next step is to apply \isa{mp} to the
implication and the assumption \isa{P\ a}, which leaves this subgoal:
\begin{isabelle}
\ 1.\ \isasymlbrakk{\isasymforall}x.\ P\ x\ \isasymlongrightarrow\ P\ (f\ x);\ P\ a;\ P\ (f\ a)\isasymrbrakk\ \isasymLongrightarrow\ P\ (f\ (f\ a))
\end{isabelle}
%
We have created the assumption \isa{P(f\ a)}, which is progress.  To finish the
proof, we apply \isa{spec} one last time, using \isa{drule}.  One final trick: if
we then apply
\begin{isabelle}
\ \ \ \ \ (drule\ mp,\ assumption)
\end{isabelle}
it will add a second copy of \isa{P(f\ a)} instead of the desired \isa{P(f\
(f\ a))}.  Bundling both \isa{assumption} calls with \isa{drule mp} causes
Isabelle to backtrack and find the correct one.


\section{The existential quantifier}

The concepts just presented also apply to the existential quantifier,
whose introduction rule looks like this in Isabelle: 
\begin{isabelle}
?P\ ?x\ \isasymLongrightarrow\ {\isasymexists}x.\ ?P\ x\rulename{exI}
\end{isabelle}
If we can exhibit some $x$ such that $P(x)$ is true, then $\exists x.
P(x)$ is also true. It is essentially a dual of the universal elimination rule, and
logic texts present it using the same notation for substitution.  The existential
elimination rule looks like this
in a logic text: 
\[ \infer{R}{\exists x.\,P & \infer*{R}{[P]}} \]
%
It looks like this in Isabelle: 
\begin{isabelle}
\isasymlbrakk{\isasymexists}x.\ ?P\ x;\ {\isasymAnd}x.\ ?P\ x\ \isasymLongrightarrow\ ?Q\isasymrbrakk\ \isasymLongrightarrow\ ?Q\rulename{exE}
\end{isabelle}
%
Given an existentially quantified theorem and some
formula $Q$ to prove, it creates a new assumption by removing the quantifier.  As with
the universal introduction  rule, the textbook version imposes a proviso on the
quantified variable, which Isabelle expresses using its meta-logic.  Note that it is
enough to have a universal quantifier in the meta-logic; we do not need an existential
quantifier to be built in as well.\REMARK{EX example needed?}
 
Isabelle/HOL also provides Hilbert's
$\epsilon$-operator.  The term $\epsilon x. P(x)$ denotes some $x$ such that $P(x)$ is
true, provided such a value exists.  Using this operator, we can express an
existential destruction rule:
\[ \infer{P[(\epsilon x. P) / \, x]}{\exists x.\,P} \]
This rule is seldom used, for it can cause exponential blow-up.  The
main use of $\epsilon x. P(x)$ is in definitions when $P(x)$ characterizes $x$
uniquely.  For instance, we can define the cardinality of a finite set~$A$ to be that
$n$ such that $A$ is in one-to-one correspondence with $\{1,\ldots,n\}$.  We can then
prove that the cardinality of the empty set is zero (since $n=0$ satisfies the
description) and proceed to prove other facts.\REMARK{SOME theorems
and example}

\begin{exercise}
Prove the lemma
\[ \exists x.\, P\conj Q(x)\Imp P\conj(\exists x.\, Q(x)). \]
\emph{Hint}: the proof is similar 
to the one just above for the universal quantifier. 
\end{exercise}


\section{Some proofs that fail}

Most of the examples in this tutorial involve proving theorems.  But not every 
conjecture is true, and it can be instructive to see how  
proofs fail. Here we attempt to prove a distributive law involving 
the existential quantifier and conjunction. 
\begin{isabelle}
\isacommand{lemma}\ "({\isasymexists}x.\ P\ x)\ \isasymand\ ({\isasymexists}x.\ Q\ x)\ \isasymLongrightarrow\ {\isasymexists}x.\ P\ x\ \isasymand\ Q\ x"\isanewline
\isacommand{apply}\ (erule\ conjE)\isanewline
\isacommand{apply}\ (erule\ exE)\isanewline
\isacommand{apply}\ (erule\ exE)\isanewline
\isacommand{apply}\ (rule\ exI)\isanewline
\isacommand{apply}\ (rule\ conjI)\isanewline
\ \isacommand{apply}\ assumption\isanewline
\isacommand{oops}
\end{isabelle}
The first steps are  routine.  We apply conjunction elimination (\isa{erule
conjE}) to split the assumption  in two, leaving two existentially quantified
assumptions.  Applying existential elimination  (\isa{erule exE}) removes one of
the quantifiers. 
\begin{isabelle}
%({\isasymexists}x.\ P\ x)\ \isasymand\ ({\isasymexists}x.\ Q\ x)\
%\isasymLongrightarrow\ {\isasymexists}x.\ P\ x\ \isasymand\ Q\ x\isanewline
\ 1.\ {\isasymAnd}x.\ \isasymlbrakk{\isasymexists}x.\ Q\ x;\ P\ x\isasymrbrakk\ \isasymLongrightarrow\ {\isasymexists}x.\ P\ x\ \isasymand\ Q\ x
\end{isabelle}
%
When we remove the other quantifier, we get a different bound 
variable in the subgoal.  (The name \isa{xa} is generated automatically.)
\begin{isabelle}
\ 1.\ {\isasymAnd}x\ xa.\ {\isasymlbrakk}P\ x;\ Q\ xa\isasymrbrakk\
\isasymLongrightarrow\ {\isasymexists}x.\ P\ x\ \isasymand\ Q\ x
\end{isabelle}
The proviso of the existential elimination rule has forced the variables to
differ: we can hardly expect two arbitrary values to be equal!  There is
no way to prove this subgoal.  Removing the
conclusion's existential quantifier yields two
identical placeholders, which can become  any term involving the variables \bigisa{x}
and~\bigisa{xa}.  We need one to become \bigisa{x}
and the other to become~\bigisa{xa}, but Isabelle requires all instances of a
placeholder to be identical. 
\begin{isabelle}
\ 1.\ {\isasymAnd}x\ xa.\ {\isasymlbrakk}P\ x;\ Q\ xa\isasymrbrakk\
\isasymLongrightarrow\ P\ (?x3\ x\ xa)\isanewline
\ 2.\ {\isasymAnd}x\ xa.\ {\isasymlbrakk}P\ x;\ Q\ xa\isasymrbrakk\ \isasymLongrightarrow\ Q\ (?x3\ x\ xa)
\end{isabelle}
We can prove either subgoal 
using the \isa{assumption} method.  If we prove the first one, the placeholder
changes  into~\bigisa{x}. 
\begin{isabelle}
\ 1.\ {\isasymAnd}x\ xa.\ {\isasymlbrakk}P\ x;\ Q\ xa\isasymrbrakk\
\isasymLongrightarrow\ Q\ x
\end{isabelle}
We are left with a subgoal that cannot be proved, 
because there is no way to prove that \bigisa{x}
equals~\bigisa{xa}.  Applying the \isa{assumption} method results in an
error message:
\begin{isabelle}
*** empty result sequence -- proof command failed
\end{isabelle}
We can tell Isabelle to abandon a failed proof using the \isacommand{oops} command.

\medskip 

Here is another abortive proof, illustrating the interaction between 
bound variables and unknowns.  
If $R$ is a reflexive relation, 
is there an $x$ such that $R\,x\,y$ holds for all $y$?  Let us see what happens when
we attempt to prove it. 
\begin{isabelle}
\isacommand{lemma}\ "{\isasymforall}z.\ R\ z\ z\ \isasymLongrightarrow\
{\isasymexists}x.\ {\isasymforall}y.\ R\ x\ y"\isanewline
\isacommand{apply}\ (rule\ exI)\isanewline
\isacommand{apply}\ (rule\ allI)\isanewline
\isacommand{apply}\ (drule\ spec)\isanewline
\isacommand{oops}
\end{isabelle}
First, 
we remove the existential quantifier. The new proof state has 
an unknown, namely~\bigisa{?x}. 
\begin{isabelle}
%{\isasymforall}z.\ R\ z\ z\ \isasymLongrightarrow\ {\isasymexists}x.\
%{\isasymforall}y.\ R\ x\ y\isanewline
\ 1.\ {\isasymforall}z.\ R\ z\ z\ \isasymLongrightarrow\ {\isasymforall}y.\ R\ ?x\ y
\end{isabelle}
Next, we remove the universal quantifier 
from the conclusion, putting the bound variable~\isa{y} into the subgoal. 
\begin{isabelle}
\ 1.\ {\isasymAnd}y.\ {\isasymforall}z.\ R\ z\ z\ \isasymLongrightarrow\ R\ ?x\ y
\end{isabelle}
Finally, we try to apply our reflexivity assumption.  We obtain a 
new assumption whose identical placeholders may be replaced by 
any term involving~\bigisa{y}. 
\begin{isabelle}
\ 1.\ {\isasymAnd}y.\ R\ (?z2\ y)\ (?z2\ y)\ \isasymLongrightarrow\ R\ ?x\ y
\end{isabelle}
This subgoal can only be proved by putting \bigisa{y} for all the placeholders,
making the assumption and conclusion become \isa{R\ y\ y}. 
But Isabelle refuses to substitute \bigisa{y}, a bound variable, for
\bigisa{?x}; that would be a bound variable capture.  The proof fails.
Note that Isabelle can replace \bigisa{?z2~y} by \bigisa{y}; this involves
instantiating
\bigisa{?z2} to the identity function.

This example is typical of how Isabelle enforces sound quantifier reasoning. 


\section{Proving theorems using the \emph{\texttt{blast}} method}

It is hard to prove substantial theorems using the methods 
described above. A proof may be dozens or hundreds of steps long.  You 
may need to search among different ways of proving certain 
subgoals. Often a choice that proves one subgoal renders another 
impossible to prove.  There are further complications that we have not
discussed, concerning negation and disjunction.  Isabelle's
\textbf{classical reasoner} is a family of tools that perform such
proofs automatically.  The most important of these is the 
{\isa{blast}} method. 

In this section, we shall first see how to use the classical 
reasoner in its default mode and then how to insert additional 
rules, enabling it to work in new problem domains. 

 We begin with examples from pure predicate logic. The following 
example is known as Andrew's challenge. Peter Andrews designed 
it to be hard to prove by automatic means.%
\footnote{Pelletier~\cite{pelletier86} describes it and many other
problems for automatic theorem provers.}
The nested biconditionals cause an exponential explosion: the formal
proof is  enormous.  However, the {\isa{blast}} method proves it in
a fraction  of a second. 
\begin{isabelle}
\isacommand{lemma}\
"(({\isasymexists}x.\
{\isasymforall}y.\
p(x){=}p(y))\
=\
(({\isasymexists}x.\
q(x))=({\isasymforall}y.\
p(y))))\
\ \ =\ \ \ \ \isanewline
\ \ \ \ \ \ \ \
(({\isasymexists}x.\
{\isasymforall}y.\
q(x){=}q(y))\
=\
(({\isasymexists}x.\
p(x))=({\isasymforall}y.\
q(y))))"\isanewline
\isacommand{apply}\ blast\isanewline
\isacommand{done}
\end{isabelle}
The next example is a logic problem composed by Lewis Carroll. 
The {\isa{blast}} method finds it trivial. Moreover, it turns out 
that not all of the assumptions are necessary. We can easily 
experiment with variations of this formula and see which ones 
can be proved. 
\begin{isabelle}
\isacommand{lemma}\
"({\isasymforall}x.\
honest(x)\ \isasymand\
industrious(x)\ \isasymlongrightarrow\
healthy(x))\
\isasymand\ \ \isanewline
\ \ \ \ \ \ \ \ \isasymnot\ ({\isasymexists}x.\
grocer(x)\ \isasymand\
healthy(x))\
\isasymand\ \isanewline
\ \ \ \ \ \ \ \ ({\isasymforall}x.\
industrious(x)\ \isasymand\
grocer(x)\ \isasymlongrightarrow\
honest(x))\
\isasymand\ \isanewline
\ \ \ \ \ \ \ \ ({\isasymforall}x.\
cyclist(x)\ \isasymlongrightarrow\
industrious(x))\
\isasymand\ \isanewline
\ \ \ \ \ \ \ \ ({\isasymforall}x.\
{\isasymnot}healthy(x)\ \isasymand\
cyclist(x)\ \isasymlongrightarrow\
{\isasymnot}honest(x))\
\ \isanewline
\ \ \ \ \ \ \ \ \isasymlongrightarrow\
({\isasymforall}x.\
grocer(x)\ \isasymlongrightarrow\
{\isasymnot}cyclist(x))"\isanewline
\isacommand{apply}\ blast\isanewline
\isacommand{done}
\end{isabelle}
The {\isa{blast}} method is also effective for set theory, which is
described in the next chapter.  This formula below may look horrible, but
the \isa{blast} method proves it easily. 
\begin{isabelle}
\isacommand{lemma}\ "({\isasymUnion}i{\isasymin}I.\ A(i))\ \isasyminter\ ({\isasymUnion}j{\isasymin}J.\ B(j))\ =\isanewline
\ \ \ \ \ \ \ \ ({\isasymUnion}i{\isasymin}I.\ {\isasymUnion}j{\isasymin}J.\ A(i)\ \isasyminter\ B(j))"\isanewline
\isacommand{apply}\ blast\isanewline
\isacommand{done}
\end{isabelle}

Few subgoals are couched purely in predicate logic and set theory.
We can extend the scope of the classical reasoner by giving it new rules. 
Extending it effectively requires understanding the notions of
introduction, elimination and destruction rules.  Moreover, there is a
distinction between  safe and unsafe rules. A \textbf{safe} rule is one
that can be applied  backwards without losing information; an
\textbf{unsafe} rule loses  information, perhaps transforming the subgoal
into one that cannot be proved.  The safe/unsafe
distinction affects the proof search: if a proof attempt fails, the
classical reasoner backtracks to the most recent unsafe rule application
and makes another choice. 

An important special case avoids all these complications.  A logical 
equivalence, which in higher-order logic is an equality between 
formulas, can be given to the classical 
reasoner and simplifier by using the attribute {\isa{iff}}.  You 
should do so if the right hand side of the equivalence is  
simpler than the left-hand side.  

For example, here is a simple fact about list concatenation. 
The result of appending two lists is empty if and only if both 
of the lists are themselves empty. Obviously, applying this equivalence 
will result in a simpler goal. When stating this lemma, we include 
the {\isa{iff}} attribute. Once we have proved the lemma, Isabelle 
will make it known to the classical reasoner (and to the simplifier). 
\begin{isabelle}
\isacommand{lemma}\
[iff]:\
"(xs{\isacharat}ys\ =\
\isacharbrackleft{]})\ =\
(xs=[]\
\isacharampersand\
ys=[])"\isanewline
\isacommand{apply}\ (induct_tac\
xs)\isanewline
\isacommand{apply}\ (simp_all)
\isanewline
\isacommand{done}
\end{isabelle}
%
This fact about multiplication is also appropriate for 
the {\isa{iff}} attribute:\REMARK{the ?s are ugly here but we need
them again when talking about \isa{of}; we need a consistent style}
\begin{isabelle}
(\mbox{?m}\ \isacharasterisk\ \mbox{?n}\ =\ 0)\ =\ (\mbox{?m}\ =\ 0\ \isasymor\ \mbox{?n}\ =\ 0)
\end{isabelle}
A product is zero if and only if one of the factors is zero.  The
reasoning  involves a logical \textsc{or}.  Proving new rules for
disjunctive reasoning  is hard, but translating to an actual disjunction
works:  the classical reasoner handles disjunction properly.

In more detail, this is how the {\isa{iff}} attribute works.  It converts
the equivalence $P=Q$ to a pair of rules: the introduction
rule $Q\Imp P$ and the destruction rule $P\Imp Q$.  It gives both to the
classical reasoner as safe rules, ensuring that all occurrences of $P$ in
a subgoal are replaced by~$Q$.  The simplifier performs the same
replacement, since \isa{iff} gives $P=Q$ to the
simplifier.  But classical reasoning is different from
simplification.  Simplification is deterministic: it applies rewrite rules
repeatedly, as long as possible, in order to \emph{transform} a goal.  Classical
reasoning uses search and backtracking in order to \emph{prove} a goal. 


\section{Proving the correctness of Euclid's algorithm}
\label{sec:proving-euclid}

A brief development will illustrate advanced use of  
\isa{blast}.  In \S\ref{sec:recdef-simplification}, we declared the
recursive function {\isa{gcd}}:
\begin{isabelle}
\isacommand{consts}\ gcd\ ::\ "nat{\isacharasterisk}nat\ \isasymRightarrow\ nat"\
\
\
\ \ \ \ \ \ \ \ \ \ \ \ \isanewline
\isacommand{recdef}\ gcd\ "measure\ ((\isasymlambda(m,n).n)\
::nat{\isacharasterisk}nat\ \isasymRightarrow\ nat)"\isanewline
\ \ \ \ "gcd\ (m,n)\ =\ (if\ n=0\ then\ m\ else\ gcd(n,\ m\ mod\ n))"
\end{isabelle}
Let us prove that it computes the greatest common
divisor of its two arguments.  
%
%The declaration yields a recursion
%equation  for {\isa{gcd}}.  Simplifying with this equation can 
%cause looping, expanding to ever-larger expressions of if-then-else 
%and {\isa{gcd}} calls.  To prevent this, we prove separate simplification rules
%for $n=0$\ldots
%\begin{isabelle}
%\isacommand{lemma}\ gcd_0\ [simp]:\ "gcd(m,0)\ =\ m"\isanewline
%\isacommand{apply}\ (simp)\isanewline
%\isacommand{done}
%\end{isabelle}
%\ldots{} and for $n>0$:
%\begin{isabelle}
%\isacommand{lemma}\ gcd_non_0:\ "0{\isacharless}n\ \isasymLongrightarrow\ gcd(m,n)\ =\ gcd\ (n,\ m\ mod\ n)"\isanewline
%\isacommand{apply}\ (simp)\isanewline
%\isacommand{done}
%\end{isabelle}
%This second rule is similar to the original equation but
%does not loop because it is conditional.  It can be applied only
%when the second argument is known to be non-zero.
%Armed with our two new simplification rules, we now delete the 
%original {\isa{gcd}} recursion equation. 
%\begin{isabelle}
%\isacommand{declare}\ gcd.simps\ [simp\ del]
%\end{isabelle}
%
%Now we can prove  some interesting facts about the {\isa{gcd}} function,
%for exampe, that it computes a common divisor of its arguments.  
%
The theorem is expressed in terms of the familiar
\textbf{divides} relation from number theory: 
\begin{isabelle}
?m\ dvd\ ?n\ \isasymequiv\ {\isasymexists}k.\ ?n\ =\ ?m\ \isacharasterisk\ k
\rulename{dvd_def}
\end{isabelle}
%
A simple induction proves the theorem.  Here \isa{gcd.induct} refers to the
induction rule returned by \isa{recdef}.  The proof relies on the simplification
rules proved in \S\ref{sec:recdef-simplification}, since rewriting by the
definition of \isa{gcd} can cause looping.
\begin{isabelle}
\isacommand{lemma}\ gcd_dvd_both:\ "(gcd(m,n)\ dvd\ m)\ \isasymand\ (gcd(m,n)\ dvd\ n)"\isanewline
\isacommand{apply}\ (induct_tac\ m\ n\ rule:\ gcd.induct)\isanewline
\isacommand{apply}\ (case_tac\ "n=0")\isanewline
\isacommand{apply}\ (simp_all)\isanewline
\isacommand{apply}\ (blast\ dest:\ dvd_mod_imp_dvd)\isanewline
\isacommand{done}%
\end{isabelle}
Notice that the induction formula 
is a conjunction.  This is necessary: in the inductive step, each 
half of the conjunction establishes the other. The first three proof steps 
are applying induction, performing a case analysis on \isa{n}, 
and simplifying.  Let us pass over these quickly and consider
the use of {\isa{blast}}.  We have reached the following 
subgoal: 
\begin{isabelle}
%gcd\ (m,\ n)\ dvd\ m\ \isasymand\ gcd\ (m,\ n)\ dvd\ n\isanewline
\ 1.\ {\isasymAnd}m\ n.\ \isasymlbrakk0\ \isacharless\ n;\isanewline
 \ \ \ \ \ \ \ \ \ \ \ \ gcd\ (n,\ m\ mod\ n)\ dvd\ n\ \isasymand\ gcd\ (n,\ m\ mod\ n)\ dvd\ (m\ mod\ n){\isasymrbrakk}\isanewline
\ \ \ \ \ \ \ \ \ \ \isasymLongrightarrow\ gcd\ (n,\ m\ mod\ n)\ dvd\ m
\end{isabelle}
%
One of the assumptions, the induction hypothesis, is a conjunction. 
The two divides relationships it asserts are enough to prove 
the conclusion, for we have the following theorem at our disposal: 
\begin{isabelle}
\isasymlbrakk?k\ dvd\ (?m\ mod\ ?n){;}\ ?k\ dvd\ ?n\isasymrbrakk\ \isasymLongrightarrow\ ?k\ dvd\ ?m%
\rulename{dvd_mod_imp_dvd}
\end{isabelle}
%
This theorem can be applied in various ways.  As an introduction rule, it
would cause backward chaining from  the conclusion (namely
\isa{?k\ dvd\ ?m}) to the two premises, which 
also involve the divides relation. This process does not look promising
and could easily loop.  More sensible is  to apply the rule in the forward
direction; each step would eliminate  the \isa{mod} symbol from an
assumption, so the process must terminate.  

So the final proof step applies the \isa{blast} method.
Attaching the {\isa{dest}} attribute to \isa{dvd_mod_imp_dvd} tells \isa{blast}
to use it as destruction rule: in the forward direction.

\medskip
We have proved a conjunction.  Now, let us give names to each of the
two halves:
\begin{isabelle}
\isacommand{lemmas}\ gcd_dvd1\ [iff]\ =\ gcd_dvd_both\ [THEN\ conjunct1]\isanewline
\isacommand{lemmas}\ gcd_dvd2\ [iff]\ =\ gcd_dvd_both\ [THEN\ conjunct2]%
\end{isabelle}

Several things are happening here. The keyword \isacommand{lemmas}
tells Isabelle to transform a theorem in some way and to
give a name to the resulting theorem.  Attributes can be given,
here \isa{iff}, which supplies the new theorems to the classical reasoner
and the simplifier.  The directive {\isa{THEN}}, which will be explained
below, supplies the lemma 
\isa{gcd_dvd_both} to the
destruction rule \isa{conjunct1} in order to extract the first part.
\begin{isabelle}
\ \ \ \ \ gcd\
(?m1,\
?n1)\ dvd\
?m1%
\end{isabelle}
The variable names \isa{?m1} and \isa{?n1} arise because
Isabelle renames schematic variables to prevent 
clashes.  The second \isacommand{lemmas} declaration yields
\begin{isabelle}
\ \ \ \ \ gcd\
(?m1,\
?n1)\ dvd\
?n1%
\end{isabelle}
Later, we shall explore this type of forward reasoning in detail. 

To complete the verification of the {\isa{gcd}} function, we must 
prove that it returns the greatest of all the common divisors 
of its arguments.  The proof is by induction and simplification.
\begin{isabelle}
\isacommand{lemma}\ gcd_greatest\
[rule_format]:\isanewline
\ \ \ \ \ \ \ "(k\ dvd\
m)\ \isasymlongrightarrow\ (k\ dvd\
n)\ \isasymlongrightarrow\ k\ dvd\
gcd(m,n)"\isanewline
\isacommand{apply}\ (induct_tac\ m\ n\
rule:\ gcd.induct)\isanewline
\isacommand{apply}\ (case_tac\ "n=0")\isanewline
\isacommand{apply}\ (simp_all\ add:\ gcd_non_0\ dvd_mod)\isanewline
\isacommand{done}
\end{isabelle}
%
Note that the theorem has been expressed using HOL implication,
\isa{\isasymlongrightarrow}, because the induction affects the two
preconditions.  The directive \isa{rule_format} tells Isabelle to replace
each \isa{\isasymlongrightarrow} by \isa{\isasymLongrightarrow} before
storing the theorem we have proved.  This directive also removes outer
universal quantifiers, converting a theorem into the usual format for
inference rules.

The facts proved above can be summarized as a single logical 
equivalence.  This step gives us a chance to see another application
of \isa{blast}, and it is worth doing for sound logical reasons.
\begin{isabelle}
\isacommand{theorem}\ gcd_greatest_iff\ [iff]:\isanewline
\ \ \ \ \ \ \ \ \ "k\ dvd\ gcd(m,n)\ =\ (k\ dvd\ m\ \isasymand\ k\ dvd\ n)"\isanewline
\isacommand{apply}\ (blast\ intro!:\ gcd_greatest\ intro:\ dvd_trans)\isanewline
\isacommand{done}
\end{isabelle}
This theorem concisely expresses the correctness of the {\isa{gcd}} 
function. 
We state it with the {\isa{iff}} attribute so that 
Isabelle can use it to remove some occurrences of {\isa{gcd}}. 
The theorem has a one-line 
proof using {\isa{blast}} supplied with four introduction 
rules: note the {\isa{intro}} attribute. The exclamation mark 
({\isa{intro}}{\isa{!}})\ signifies safe rules, which are 
applied aggressively.  Rules given without the exclamation mark 
are applied reluctantly and their uses can be undone if 
the search backtracks.  Here the unsafe rule expresses transitivity  
of the divides relation:
\begin{isabelle}
\isasymlbrakk?m\ dvd\ ?n;\ ?n\ dvd\ ?p\isasymrbrakk\ \isasymLongrightarrow\ ?m\ dvd\ ?p%
\rulename{dvd_trans}
\end{isabelle}
Applying \isa{dvd_trans} as 
an introduction rule entails a risk of looping, for it multiplies 
occurrences of the divides symbol. However, this proof relies 
on transitivity reasoning.  The rule {\isa{gcd\_greatest}} is safe to apply 
aggressively because it yields simpler subgoals.  The proof implicitly
uses \isa{gcd_dvd1} and \isa{gcd_dvd2} as safe rules, because they were
declared using \isa{iff}.


\section{Other classical reasoning methods}
 
The {\isa{blast}} method is our main workhorse for proving theorems 
automatically. Other components of the classical reasoner interact 
with the simplifier. Still others perform classical reasoning 
to a limited extent, giving the user fine control over the proof. 

Of the latter methods, the most useful is {\isa{clarify}}. It performs 
all obvious reasoning steps without splitting the goal into multiple 
parts. It does not apply rules that could render the 
goal unprovable (so-called unsafe rules). By performing the obvious 
steps, {\isa{clarify}} lays bare the difficult parts of the problem, 
where human intervention is necessary. 

For example, the following conjecture is false:
\begin{isabelle}
\isacommand{lemma}\ "({\isasymforall}x.\ P\ x)\ \isasymand\
({\isasymexists}x.\ Q\ x)\ \isasymlongrightarrow\ ({\isasymforall}x.\ P\ x\
\isasymand\ Q\ x)"\isanewline
\isacommand{apply}\ clarify
\end{isabelle}
The {\isa{blast}} method would simply fail, but {\isa{clarify}} presents 
a subgoal that helps us see why we cannot continue the proof. 
\begin{isabelle}
\ 1.\ {\isasymAnd}x\ xa.\ \isasymlbrakk{\isasymforall}x.\ P\ x;\ Q\
xa\isasymrbrakk\ \isasymLongrightarrow\ P\ x\ \isasymand\ Q\ x
\end{isabelle}
The proof must fail because the assumption \isa{Q\ xa} and conclusion \isa{Q\ x}
refer to distinct bound variables.  To reach this state, \isa{clarify} applied
the introduction rules for \isa{\isasymlongrightarrow} and \isa{\isasymforall}
and the elimination rule for ~\isa{\isasymand}.  It did not apply the introduction
rule for  \isa{\isasymand} because of its policy never to split goals.

Also available is {\isa{clarsimp}}, a method that interleaves {\isa{clarify}}
and {\isa{simp}}.  Also there is \isa{safe}, which like \isa{clarify} performs
obvious steps and even applies those that split goals.

The {\isa{force}} method applies the classical reasoner and simplifier 
to one goal. 
\REMARK{example needed? most
things done by blast, simp or auto can also be done by force, so why add a new
one?}
Unless it can prove the goal, it fails. Contrast 
that with the auto method, which also combines classical reasoning 
with simplification. The latter's purpose is to prove all the 
easy subgoals and parts of subgoals. Unfortunately, it can produce 
large numbers of new subgoals; also, since it proves some subgoals 
and splits others, it obscures the structure of the proof tree. 
The {\isa{force}} method does not have these drawbacks. Another 
difference: {\isa{force}} tries harder than {\isa{auto}} to prove 
its goal, so it can take much longer to terminate.

Older components of the classical reasoner have largely been 
superseded by {\isa{blast}}, but they still have niche applications. 
Most important among these are {\isa{fast}} and {\isa{best}}. While {\isa{blast}} 
searches for proofs using a built-in first-order reasoner, these 
earlier methods search for proofs using standard Isabelle inference. 
That makes them slower but enables them to work correctly in the 
presence of the more unusual features of Isabelle rules, such 
as type classes and function unknowns. For example, the introduction rule
for Hilbert's epsilon-operator has the following form: 
\begin{isabelle}
?P\ ?x\ \isasymLongrightarrow\ ?P\ (Eps\ ?P)
\rulename{someI}
\end{isabelle}

The repeated occurrence of the variable \isa{?P} makes this rule tricky 
to apply. Consider this contrived example: 
\begin{isabelle}
\isacommand{lemma}\ "{\isasymlbrakk}Q\ a;\ P\ a\isasymrbrakk\isanewline
\ \ \ \ \ \ \ \ \,\isasymLongrightarrow\ P\ (SOME\ x.\ P\ x\ \isasymand\ Q\ x)\
\isasymand\ Q\ (SOME\ x.\ P\ x\ \isasymand\ Q\ x)"\isanewline
\isacommand{apply}\ (rule\ someI)
\end{isabelle}
%
We can apply rule \isa{someI} explicitly.  It yields the 
following subgoal: 
\begin{isabelle}
\ 1.\ {\isasymlbrakk}Q\ a;\ P\ a\isasymrbrakk\ \isasymLongrightarrow\ P\ ?x\
\isasymand\ Q\ ?x%
\end{isabelle}
The proof from this point is trivial.  The question now arises, could we have
proved the theorem with a single command? Not using {\isa{blast}} method: it
cannot perform  the higher-order unification that is necessary here.  The
{\isa{fast}}  method succeeds: 
\begin{isabelle}
\isacommand{apply}\ (fast\ intro!:\ someI)
\end{isabelle}

The {\isa{best}} method is similar to {\isa{fast}} but it uses a 
best-first search instead of depth-first search. Accordingly, 
it is slower but is less susceptible to divergence. Transitivity 
rules usually cause {\isa{fast}} to loop where often {\isa{best}} 
can manage.

Here is a summary of the classical reasoning methods:
\begin{itemize}
\item \isa{blast} works automatically and is the fastest
\item \isa{clarify} and \isa{clarsimp} perform obvious steps without splitting the
goal; \isa{safe} even splits goals
\item \isa{force} uses classical reasoning and simplification to prove a goal;
 \isa{auto} is similar but leaves what it cannot prove
\item \isa{fast} and \isa{best} are legacy methods that work well with rules involving
unusual features
\end{itemize}
A table illustrates the relationships among four of these methods. 
\begin{center}
\begin{tabular}{r|l|l|}
           & no split   & split \\ \hline
  no simp  & \isa{clarify}    & \isa{safe} \\ \hline
     simp  & \isa{clarsimp}   & \isa{auto} \\ \hline
\end{tabular}
\end{center}




\section{Forward proof}\label{sec:forward}

Forward proof means deriving new facts from old ones.  It is  the
most fundamental type of proof.  Backward proof, by working  from goals to
subgoals, can help us find a difficult proof.  But it is
not always the best way of presenting the proof so found.  Forward
proof is particularly good for reasoning from the general
to the specific.  For example, consider the following distributive law for
the 
\isa{gcd} function:
\[ k\times\gcd(m,n) = \gcd(k\times m,k\times n)\]

Putting $m=1$ we get (since $\gcd(1,n)=1$ and $k\times1=k$) 
\[ k = \gcd(k,k\times n)\]
We have derived a new fact about \isa{gcd}; if re-oriented, it might be
useful for simplification.  After re-orienting it and putting $n=1$, we
derive another useful law: 
\[ \gcd(k,k)=k \]
Substituting values for variables --- instantiation --- is a forward step. 
Re-orientation works by applying the symmetry of equality to 
an equation, so it too is a forward step.  

Now let us reproduce our examples in Isabelle.  Here is the distributive
law:
\begin{isabelle}%
?k\ \isacharasterisk\ gcd\ (?m,\ ?n)\ =\ gcd\ (?k\ \isacharasterisk\ ?m,\ ?k\ \isacharasterisk\ ?n)
\rulename{gcd_mult_distrib2}
\end{isabelle}%
The first step is to replace \isa{?m} by~1 in this law.  We refer to the
variables not by name but by their position in the theorem, from left to
right.  In this case, the variables  are \isa{?k}, \isa{?m} and~\isa{?n}.
So, the expression
\hbox{\texttt{[of k 1]}} replaces \isa{?k} by~\isa{k} and \isa{?m}
by~\isa{1}.
\begin{isabelle}
\isacommand{lemmas}\ gcd_mult_0\ =\ gcd_mult_distrib2\ [of\ k\ 1]
\end{isabelle}
%
The command 
\isa{thm gcd_mult_0}
displays the resulting theorem:
\begin{isabelle}
\ \ \ \ \ k\ \isacharasterisk\ gcd\ (1,\ ?n)\ =\ gcd\ (k\ \isacharasterisk\ 1,\ k\ \isacharasterisk\ ?n)
\end{isabelle}
Something is odd: {\isa{k}} is an ordinary variable, while {\isa{?n}} 
is schematic.  We did not specify an instantiation 
for {\isa{?n}}.  In its present form, the theorem does not allow 
substitution for {\isa{k}}.  One solution is to avoid giving an instantiation for
\isa{?k}: instead of a term we can put an underscore~(\isa{_}).  For example,
\begin{isabelle}
\ \ \ \ \ gcd_mult_distrib2\ [of\ _\ 1]
\end{isabelle}
replaces \isa{?m} by~\isa{1} but leaves \isa{?k} unchanged.  Anyway, let us put
the theorem \isa{gcd_mult_0} into a simplified form: 
\begin{isabelle}
\isacommand{lemmas}\
gcd_mult_1\ =\ gcd_mult_0\
[simplified]%
\end{isabelle}
%
Again, we display the resulting theorem:
\begin{isabelle}
\ \ \ \ \ k\ =\ gcd\ (k,\ k\ \isacharasterisk\ ?n)
\end{isabelle}
%
To re-orient the equation requires the symmetry rule:
\begin{isabelle}
?s\ =\ ?t\
\isasymLongrightarrow\ ?t\ =\
?s%
\rulename{sym}
\end{isabelle}
The following declaration gives our equation to \isa{sym}:
\begin{isabelle}
\ \ \ \isacommand{lemmas}\ gcd_mult\ =\ gcd_mult_1\
[THEN\ sym]
\end{isabelle}
%
Here is the result:
\begin{isabelle}
\ \ \ \ \ gcd\ (k,\ k\ \isacharasterisk\
?n)\ =\ k%
\end{isabelle}
\isa{THEN~sym} gives the current theorem to the rule \isa{sym} and returns the
resulting conclusion.\REMARK{figure necessary?}  The effect is to exchange the
two operands of the equality. Typically {\isa{THEN}} is used with destruction
rules.  Above we have used
\isa{THEN~conjunct1} to extract the first part of the theorem
\isa{gcd_dvd_both}.  Also useful is \isa{THEN~spec}, which removes the quantifier
from a theorem of the form $\forall x.\,P$, and \isa{THEN~mp}, which converts the
implication $P\imp Q$ into the rule $\vcenter{\infer{Q}{P}}$.
Similar to \isa{mp} are the following two rules, which extract 
the two directions of reasoning about a boolean equivalence:
\begin{isabelle}
\isasymlbrakk?Q\ =\
?P;\ ?Q\isasymrbrakk\
\isasymLongrightarrow\ ?P%
\rulename{iffD1}%
\isanewline
\isasymlbrakk?P\ =\ ?Q;\ ?Q\isasymrbrakk\
\isasymLongrightarrow\ ?P%
\rulename{iffD2}
\end{isabelle}
%
Normally we would never name the intermediate theorems
such as \isa{gcd_mult_0} and
\isa{gcd_mult_1} but would combine
the three forward steps: 
\begin{isabelle}
\isacommand{lemmas}\ gcd_mult\ =\ gcd_mult_distrib2\ [of\ k\ 1,\ simplified,\ THEN\ sym]%
\end{isabelle}
The directives, or attributes, are processed from left to right.  This
declaration of \isa{gcd_mult} is equivalent to the
previous one.

Such declarations can make the proof script hard to read: 
what is being proved?  More legible   
is to state the new lemma explicitly and to prove it using a single
\isa{rule} method whose operand is expressed using forward reasoning:
\begin{isabelle}
\isacommand{lemma}\ gcd_mult\
[simp]:\
"gcd(k,\
k{\isacharasterisk}n)\ =\
k"\isanewline
\isacommand{apply}\ (rule\ gcd_mult_distrib2\ [of\ k\ 1,\ simplified,\ THEN\ sym])\isanewline
\isacommand{done}
\end{isabelle}
Compared with the previous proof of \isa{gcd_mult}, this
version shows the reader what has been proved.  Also, it receives
the usual Isabelle treatment.  In particular, Isabelle generalizes over all
variables: the resulting theorem will have {\isa{?k}} instead of {\isa{k}}.

At the start  of this section, we also saw a proof of $\gcd(k,k)=k$.  Here
is the Isabelle version: 
\begin{isabelle}
\isacommand{lemma}\ gcd_self\ [simp]:\ "gcd(k,k)\ =\ k"\isanewline
\isacommand{apply}\ (rule\ gcd_mult\ [of\ k\ 1,\ simplified])\isanewline
\isacommand{done}
\end{isabelle}

Recall that \isa{of} generates an instance of a rule by specifying
values for its variables.  Analogous is \isa{OF}, which generates an
instance of a rule by specifying facts for its premises.  Let us try
it with this rule:
\begin{isabelle}
{\isasymlbrakk}gcd(?k,?n){=}1;\ ?k\ dvd\ (?m * ?n){\isasymrbrakk}\
\isasymLongrightarrow\ ?k\ dvd\ ?m
\rulename{relprime_dvd_mult}
\end{isabelle}
First, we
prove an instance of its first premise:
\begin{isabelle}
\isacommand{lemma}\ relprime_20_81:\ "gcd(\#20,\#81)\ =\ 1"\isanewline
\isacommand{apply}\ (simp\ add:\ gcd.simps)\isanewline
\isacommand{done}%
\end{isabelle}
We have evaluated an application of the \isa{gcd} function by
simplification.  Expression evaluation  is not guaranteed to terminate, and
certainly is not  efficient; Isabelle performs arithmetic operations by 
rewriting symbolic bit strings.  Here, however, the simplification takes
less than one second.  We can specify this new lemma to {\isa{OF}},
generating an instance of \isa{relprime_dvd_mult}.  The expression
\begin{isabelle}
\ \ \ \ \ relprime_dvd_mult [OF relprime_20_81]
\end{isabelle}
yields the theorem
\begin{isabelle}
\ \ \ \ \ \isacharhash20\ dvd\ (?m\ \isacharasterisk\ \isacharhash81)\ \isasymLongrightarrow\ \isacharhash20\ dvd\ ?m%
\end{isabelle}
%
{\isa{OF}} takes any number of operands.  Consider 
the following facts about the divides relation: 
\begin{isabelle}
\isasymlbrakk?k\ dvd\ ?m;\
?k\ dvd\ ?n\isasymrbrakk\
\isasymLongrightarrow\ ?k\ dvd\
(?m\ \isacharplus\
?n)
\rulename{dvd_add}\isanewline
?m\ dvd\ ?m%
\rulename{dvd_refl}
\end{isabelle}
Let us supply \isa{dvd_refl} for each of the premises of \isa{dvd_add}:
\begin{isabelle}
\ \ \ \ \ dvd_add [OF dvd_refl dvd_refl]
\end{isabelle}
Here is the theorem that we have expressed: 
\begin{isabelle}
\ \ \ \ \ ?k\ dvd\ (?k\ \isacharplus\ ?k)
\end{isabelle}
As with \isa{of}, we can use the \isa{_} symbol to leave some positions
unspecified:
\begin{isabelle}
\ \ \ \ \ dvd_add [OF _ dvd_refl]
\end{isabelle}
The result is 
\begin{isabelle}
\ \ \ \ \ ?k\ dvd\ ?m\ \isasymLongrightarrow\ ?k\ dvd\ (?m\ \isacharplus\ ?k)
\end{isabelle}

You may have noticed that {\isa{THEN}} and {\isa{OF}} are based on 
the same idea, namely to combine two rules.  They differ in the
order of the combination and thus in their effect.  We use \isa{THEN}
typically with a destruction rule to extract a subformula of the current
theorem.  We use \isa{OF} with a list of facts to generate an instance of
the current theorem.


Here is a summary of the primitives for forward reasoning:
\begin{itemize}
\item {\isa{of}} instantiates the variables of a rule to a list of terms
\item {\isa{OF}} applies a rule to a list of theorems
\item {\isa{THEN}} gives a theorem to a named rule and returns the
conclusion 
\end{itemize}


\section{Methods for forward proof}

We have seen that forward proof works well within a backward 
proof.  Also in that spirit is the \isa{insert} method, which inserts a
given theorem as a new assumption of the current subgoal.  This already
is a forward step; moreover, we may (as always when using a theorem) apply
{\isa{of}}, {\isa{THEN}} and other directives.  The new assumption can then
be used to help prove the subgoal.

For example, consider this theorem about the divides relation. 
Only the first proof step is given; it inserts the distributive law for
\isa{gcd}.  We specify its variables as shown. 
\begin{isabelle}
\isacommand{lemma}\
relprime_dvd_mult:\isanewline
\ \ \ \ \ \ \ "{\isasymlbrakk}\ gcd(k,n){=}1;\
k\ dvd\ (m*n)\
{\isasymrbrakk}\
\isasymLongrightarrow\ k\ dvd\
m"\isanewline
\isacommand{apply}\ (insert\ gcd_mult_distrib2\ [of\ m\ k\
n])
\end{isabelle}
In the resulting subgoal, note how the equation has been 
inserted: 
\begin{isabelle}
{\isasymlbrakk}gcd\ (k,\ n)\ =\ 1;\ k\
dvd\ (m\ \isacharasterisk\
n){\isasymrbrakk}\ \isasymLongrightarrow\ k\ dvd\
m\isanewline
\ 1.\ {\isasymlbrakk}gcd\ (k,\ n)\ =\ 1;\ k\ dvd\ (m\ \isacharasterisk\ n){;}\isanewline
\ \ \ \ \ m\ \isacharasterisk\ gcd\
(k,\ n)\
=\ gcd\ (m\ \isacharasterisk\
k,\ m\ \isacharasterisk\
n){\isasymrbrakk}\isanewline
\ \ \ \ \isasymLongrightarrow\ k\ dvd\ m
\end{isabelle}
The next proof step, \isa{\isacommand{apply}(simp)}, 
utilizes the assumption \isa{gcd(k,n)\ =\
1}. Here is the result: 
\begin{isabelle}
{\isasymlbrakk}gcd\ (k,\
n)\ =\ 1;\ k\
dvd\ (m\ \isacharasterisk\
n){\isasymrbrakk}\ \isasymLongrightarrow\ k\ dvd\
m\isanewline
\ 1.\ {\isasymlbrakk}gcd\ (k,\ n)\ =\ 1;\ k\ dvd\ (m\ \isacharasterisk\ n){;}\isanewline
\ \ \ \ \ m\ =\ gcd\ (m\
\isacharasterisk\ k,\ m\ \isacharasterisk\
n){\isasymrbrakk}\isanewline
\ \ \ \ \isasymLongrightarrow\ k\ dvd\ m
\end{isabelle}
Simplification has yielded an equation for \isa{m} that will be used to
complete the proof. 

\medskip
Here is another proof using \isa{insert}.  \REMARK{Effect with unknowns?}
Division  and remainder obey a well-known law: 
\begin{isabelle}
(?m\ div\ ?n)\ \isacharasterisk\
?n\ \isacharplus\ ?m\ mod\ ?n\
=\ ?m
\rulename{mod_div_equality}
\end{isabelle}

We refer to this law explicitly in the following proof: 
\begin{isabelle}
\isacommand{lemma}\ div_mult_self_is_m:\ \isanewline
\ \ \ \ \ \ "0{\isacharless}n\ \isasymLongrightarrow\ (m{\isacharasterisk}n)\ div\ n\ =\ (m::nat)"\isanewline
\isacommand{apply}\ (insert\ mod_div_equality\ [of\ "m{\isacharasterisk}n"\ n])\isanewline
\isacommand{apply}\ (simp)\isanewline
\isacommand{done}
\end{isabelle}
The first step inserts the law, specifying \isa{m*n} and
\isa{n} for its variables.  Notice that non-trivial expressions must be
enclosed in quotation marks.  Here is the resulting 
subgoal, with its new assumption: 
\begin{isabelle}
%0\ \isacharless\ n\ \isasymLongrightarrow\ (m\
%\isacharasterisk\ n)\ div\ n\ =\ m\isanewline
\ 1.\ \isasymlbrakk0\ \isacharless\
n;\ \ (m\ \isacharasterisk\ n)\ div\ n\
\isacharasterisk\ n\ \isacharplus\ (m\ \isacharasterisk\ n)\ mod\ n\
=\ m\ \isacharasterisk\ n\isasymrbrakk\isanewline
\ \ \ \ \isasymLongrightarrow\ (m\ \isacharasterisk\ n)\ div\ n\
=\ m
\end{isabelle}
Simplification reduces \isa{(m\ \isacharasterisk\ n)\ mod\ n} to zero.
Then it cancels the factor~\isa{n} on both
sides of the equation, proving the theorem. 

\medskip
A similar method is {\isa{subgoal\_tac}}. Instead of inserting 
a theorem as an assumption, it inserts an arbitrary formula. 
This formula must be proved later as a separate subgoal. The 
idea is to claim that the formula holds on the basis of the current 
assumptions, to use this claim to complete the proof, and finally 
to justify the claim. It is a valuable means of giving the proof 
some structure. The explicit formula will be more readable than 
proof commands that yield that formula indirectly.

Look at the following example. 
\begin{isabelle}
\isacommand{lemma}\ "\isasymlbrakk(z::int)\ <\ \#37;\ \#66\ <\ \#2*z;\ z*z\
\isasymnoteq\ \#1225;\ Q(\#34);\ Q(\#36)\isasymrbrakk\isanewline
\ \ \ \ \ \ \ \ \,\isasymLongrightarrow\ Q(z)"\isanewline
\isacommand{apply}\ (subgoal_tac\ "z\ =\ \#34\ \isasymor\ z\ =\
\#36")\isanewline
\isacommand{apply}\ blast\isanewline
\isacommand{apply}\ (subgoal_tac\ "z\ \isasymnoteq\ \#35")\isanewline
\isacommand{apply}\ arith\isanewline
\isacommand{apply}\ force\isanewline
\isacommand{done}
\end{isabelle}
Let us prove it informally.  The first assumption tells us 
that \isa{z} is no greater than 36. The second tells us that \isa{z} 
is at least 34. The third assumption tells us that \isa{z} cannot be 35, since
$35\times35=1225$.  So \isa{z} is either 34 or 36, and since \isa{Q} holds for
both of those  values, we have the conclusion. 

The Isabelle proof closely follows this reasoning. The first 
step is to claim that \isa{z} is either 34 or 36. The resulting proof 
state gives us two subgoals: 
\begin{isabelle}
%{\isasymlbrakk}z\ <\ \#37;\ \#66\ <\ \#2\ *\ z;\ z\ *\ z\ \isasymnoteq\ \#1225;\
%Q\ \#34;\ Q\ \#36\isasymrbrakk\ \isasymLongrightarrow\ Q\ z\isanewline
\ 1.\ {\isasymlbrakk}z\ <\ \#37;\ \#66\ <\ \#2\ *\ z;\ z\ *\ z\ \isasymnoteq\ \#1225;\ Q\ \#34;\ Q\ \#36;\isanewline
\ \ \ \ \ z\ =\ \#34\ \isasymor\ z\ =\ \#36\isasymrbrakk\isanewline
\ \ \ \ \isasymLongrightarrow\ Q\ z\isanewline
\ 2.\ {\isasymlbrakk}z\ <\ \#37;\ \#66\ <\ \#2\ *\ z;\ z\ *\ z\ \isasymnoteq\ \#1225;\ Q\ \#34;\ Q\ \#36\isasymrbrakk\isanewline
\ \ \ \ \isasymLongrightarrow\ z\ =\ \#34\ \isasymor\ z\ =\ \#36
\end{isabelle}

The first subgoal is trivial, but for the second Isabelle needs help to eliminate
the case
\isa{z}=35.  The second invocation  of {\isa{subgoal\_tac}} leaves two
subgoals: 
\begin{isabelle}
\ 1.\ {\isasymlbrakk}z\ <\ \#37;\ \#66\ <\ \#2\ *\ z;\ z\ *\ z\ \isasymnoteq\
\#1225;\ Q\ \#34;\ Q\ \#36;\isanewline
\ \ \ \ \ z\ \isasymnoteq\ \#35\isasymrbrakk\isanewline
\ \ \ \ \isasymLongrightarrow\ z\ =\ \#34\ \isasymor\ z\ =\ \#36\isanewline
\ 2.\ {\isasymlbrakk}z\ <\ \#37;\ \#66\ <\ \#2\ *\ z;\ z\ *\ z\ \isasymnoteq\ \#1225;\ Q\ \#34;\ Q\ \#36\isasymrbrakk\isanewline
\ \ \ \ \isasymLongrightarrow\ z\ \isasymnoteq\ \#35
\end{isabelle}

Assuming that \isa{z} is not 35, the first subgoal follows by linear arithmetic:
the method {\isa{arith}}. For the second subgoal we apply the method {\isa{force}}, 
which proceeds by assuming that \isa{z}=35 and arriving at a contradiction.


\medskip
Summary of these methods:
\begin{itemize}
\item {\isa{insert}} adds a theorem as a new assumption
\item {\isa{subgoal_tac}} adds a formula as a new assumption and leaves the
subgoal of proving that formula
\end{itemize}

% $Id$
\chapter{Sets, Functions and Relations}

This chapter describes the formalization of typed set theory, which is
the basis of much else in HOL\@.  For example, an
inductive definition yields a set, and the abstract theories of relations
regard a relation as a set of pairs.  The chapter introduces the well-known
constants such as union and intersection, as well as the main operations on relations, such as converse, composition and
transitive closure.  Functions are also covered.  They are not sets in
HOL, but many of their properties concern sets: the range of a
function is a set, and the inverse image of a function maps sets to sets.

This chapter will be useful to anybody who plans to develop a substantial
proof.  sets are convenient for formalizing  computer science concepts such
as grammars, logical calculi and state transition systems.  Isabelle can
prove many statements involving sets automatically.

This chapter ends with a case study concerning model checking for the
temporal logic CTL\@.  Most of the other examples are simple.  The
chapter presents a small selection of built-in theorems in order to point
out some key properties of the various constants and to introduce you to
the notation. 

Natural deduction rules are provided for the set theory constants, but they
are seldom used directly, so only a few are presented here.  


\section{Sets}

\index{sets|(}%
HOL's set theory should not be confused with traditional,  untyped set
theory, in which everything is a set.  Our sets are typed. In a given set,
all elements have the same type, say~$\tau$,  and the set itself has type
\isa{$\tau$~set}. 

We begin with \bfindex{intersection}, \bfindex{union} and
\bfindex{complement}. In addition to the
\bfindex{membership relation}, there  is a symbol for its negation. These
points can be seen below.  

Here are the natural deduction rules for intersection.  Note the 
resemblance to those for conjunction.  
\begin{isabelle}
\isasymlbrakk c\ \isasymin\ A;\ c\ \isasymin\ B\isasymrbrakk\ 
\isasymLongrightarrow\ c\ \isasymin\ A\ \isasyminter\ B%
\rulename{IntI}\isanewline
c\ \isasymin\ A\ \isasyminter\ B\ \isasymLongrightarrow\ c\ \isasymin\ A
\rulename{IntD1}\isanewline
c\ \isasymin\ A\ \isasyminter\ B\ \isasymLongrightarrow\ c\ \isasymin\ B
\rulename{IntD2}
\end{isabelle}

Here are two of the many installed theorems concerning set complement.
Note that it is denoted by a minus sign.
\begin{isabelle}
(c\ \isasymin\ -\ A)\ =\ (c\ \isasymnotin\ A)
\rulename{Compl_iff}\isanewline
-\ (A\ \isasymunion\ B)\ =\ -\ A\ \isasyminter\ -\ B
\rulename{Compl_Un}
\end{isabelle}

Set \textbf{difference}\indexbold{difference!of sets} is the intersection
of a set with the  complement of another set. Here we also see the syntax
for the  empty set and for the universal set. 
\begin{isabelle}
A\ \isasyminter\ (B\ -\ A)\ =\ \isacharbraceleft\isacharbraceright
\rulename{Diff_disjoint}\isanewline
A\ \isasymunion\ -\ A\ =\ UNIV%
\rulename{Compl_partition}
\end{isabelle}

The \bfindex{subset relation} holds between two sets just if every element 
of one is also an element of the other. This relation is reflexive.  These
are its natural deduction rules:
\begin{isabelle}
({\isasymAnd}x.\ x\ \isasymin\ A\ \isasymLongrightarrow\ x\ \isasymin\ B)\ \isasymLongrightarrow\ A\ \isasymsubseteq\ B%
\rulename{subsetI}%
\par\smallskip%          \isanewline didn't leave enough space
\isasymlbrakk A\ \isasymsubseteq\ B;\ c\ \isasymin\
A\isasymrbrakk\ \isasymLongrightarrow\ c\
\isasymin\ B%
\rulename{subsetD}
\end{isabelle}
In harder proofs, you may need to apply \isa{subsetD} giving a specific term
for~\isa{c}.  However, \isa{blast} can instantly prove facts such as this
one: 
\begin{isabelle}
(A\ \isasymunion\ B\ \isasymsubseteq\ C)\ =\
(A\ \isasymsubseteq\ C\ \isasymand\ B\ \isasymsubseteq\ C)
\rulename{Un_subset_iff}
\end{isabelle}
Here is another example, also proved automatically:
\begin{isabelle}
\isacommand{lemma}\ "(A\
\isasymsubseteq\ -B)\ =\ (B\ \isasymsubseteq\ -A)"\isanewline
\isacommand{by}\ blast
\end{isabelle}
%
This is the same example using \textsc{ascii} syntax, illustrating a pitfall: 
\begin{isabelle}
\isacommand{lemma}\ "(A\ <=\ -B)\ =\ (B\ <=\ -A)"
\end{isabelle}
%
The proof fails.  It is not a statement about sets, due to overloading;
the relation symbol~\isa{<=} can be any relation, not just  
subset. 
In this general form, the statement is not valid.  Putting
in a type constraint forces the variables to denote sets, allowing the
proof to succeed:

\begin{isabelle}
\isacommand{lemma}\ "((A::\ {\isacharprime}a\ set)\ <=\ -B)\ =\ (B\ <=\
-A)"
\end{isabelle}
Section~\ref{sec:axclass} below describes overloading.  Incidentally,
\isa{A~\isasymsubseteq~-B} asserts that the sets \isa{A} and \isa{B} are
disjoint.

\medskip
Two sets are \textbf{equal}\indexbold{equality!of sets} if they contain the
same elements.   This is
the principle of \textbf{extensionality}\indexbold{extensionality!for sets}
for sets. 
\begin{isabelle}
({\isasymAnd}x.\ (x\ {\isasymin}\ A)\ =\ (x\ {\isasymin}\ B))\
{\isasymLongrightarrow}\ A\ =\ B
\rulename{set_ext}
\end{isabelle}
Extensionality is often expressed as 
$A=B\iff A\subseteq B\conj B\subseteq A$.  
The following rules express both
directions of this equivalence.  Proving a set equation using
\isa{equalityI} allows the two inclusions to be proved independently.
\begin{isabelle}
\isasymlbrakk A\ \isasymsubseteq\ B;\ B\ \isasymsubseteq\
A\isasymrbrakk\ \isasymLongrightarrow\ A\ =\ B
\rulename{equalityI}
\par\smallskip%          \isanewline didn't leave enough space
\isasymlbrakk A\ =\ B;\ \isasymlbrakk A\ \isasymsubseteq\ B;\ B\
\isasymsubseteq\ A\isasymrbrakk\ \isasymLongrightarrow\ P\isasymrbrakk\
\isasymLongrightarrow\ P%
\rulename{equalityE}
\end{isabelle}


\subsection{Finite Set Notation} 

\indexbold{sets!notation for finite}\index{*insert (constant)}
Finite sets are expressed using the constant \isa{insert}, which is
a form of union:
\begin{isabelle}
insert\ a\ A\ =\ \isacharbraceleft a\isacharbraceright\ \isasymunion\ A
\rulename{insert_is_Un}
\end{isabelle}
%
The finite set expression \isa{\isacharbraceleft
a,b\isacharbraceright} abbreviates
\isa{insert\ a\ (insert\ b\ \isacharbraceleft\isacharbraceright)}.
Many facts about finite sets can be proved automatically: 
\begin{isabelle}
\isacommand{lemma}\
"\isacharbraceleft a,b\isacharbraceright\
\isasymunion\ \isacharbraceleft c,d\isacharbraceright\ =\
\isacharbraceleft a,b,c,d\isacharbraceright"\isanewline
\isacommand{by}\ blast
\end{isabelle}


Not everything that we would like to prove is valid. 
Consider this attempt: 
\begin{isabelle}
\isacommand{lemma}\ "\isacharbraceleft a,b\isacharbraceright\ \isasyminter\ \isacharbraceleft b,c\isacharbraceright\ =\
\isacharbraceleft b\isacharbraceright"\isanewline
\isacommand{apply}\ auto
\end{isabelle}
%
The proof fails, leaving the subgoal \isa{b=c}. To see why it 
fails, consider a correct version: 
\begin{isabelle}
\isacommand{lemma}\ "\isacharbraceleft a,b\isacharbraceright\ \isasyminter\ 
\isacharbraceleft b,c\isacharbraceright\ =\ (if\ a=c\ then\
\isacharbraceleft a,b\isacharbraceright\ else\ \isacharbraceleft
b\isacharbraceright)"\isanewline
\isacommand{apply}\ simp\isanewline
\isacommand{by}\ blast
\end{isabelle}

Our mistake was to suppose that the various items were distinct.  Another
remark: this proof uses two methods, namely {\isa{simp}}  and
{\isa{blast}}. Calling {\isa{simp}} eliminates the
\isa{if}-\isa{then}-\isa{else} expression,  which {\isa{blast}}
cannot break down. The combined methods (namely {\isa{force}}  and
\isa{auto}) can prove this fact in one step. 


\subsection{Set Comprehension}

\index{set comprehensions|(}%
The set comprehension \isa{\isacharbraceleft x.\
P\isacharbraceright} expresses the set of all elements that satisfy the
predicate~\isa{P}.  Two laws describe the relationship between set 
comprehension and the membership relation:
\begin{isabelle}
(a\ \isasymin\
\isacharbraceleft x.\ P\ x\isacharbraceright)\ =\ P\ a
\rulename{mem_Collect_eq}\isanewline
\isacharbraceleft x.\ x\ \isasymin\ A\isacharbraceright\ =\ A
\rulename{Collect_mem_eq}
\end{isabelle}

\smallskip
Facts such as these have trivial proofs:
\begin{isabelle}
\isacommand{lemma}\ "\isacharbraceleft x.\ P\ x\ \isasymor\
x\ \isasymin\ A\isacharbraceright\ =\
\isacharbraceleft x.\ P\ x\isacharbraceright\ \isasymunion\ A"
\par\smallskip
\isacommand{lemma}\ "\isacharbraceleft x.\ P\ x\
\isasymlongrightarrow\ Q\ x\isacharbraceright\ =\
-\isacharbraceleft x.\ P\ x\isacharbraceright\
\isasymunion\ \isacharbraceleft x.\ Q\ x\isacharbraceright"
\end{isabelle}

\smallskip
Isabelle has a general syntax for comprehension, which is best 
described through an example: 
\begin{isabelle}
\isacommand{lemma}\ "\isacharbraceleft p*q\ \isacharbar\ p\ q.\ 
p{\isasymin}prime\ \isasymand\ q{\isasymin}prime\isacharbraceright\ =\ 
\isanewline
\ \ \ \ \ \ \ \ \isacharbraceleft z.\ \isasymexists p\ q.\ z\ =\ p*q\
\isasymand\ p{\isasymin}prime\ \isasymand\
q{\isasymin}prime\isacharbraceright"
\end{isabelle}
The proof is trivial because the left and right hand side 
of the expression are synonymous. The syntax appearing on the 
left-hand side abbreviates the right-hand side: in this case, all numbers
that are the product of two primes.  The syntax provides a neat
way of expressing any set given by an expression built up from variables
under specific constraints.  The drawback is that it hides the true form of
the expression, with its existential quantifiers. 

\smallskip
\emph{Remark}.  We do not need sets at all.  They are essentially equivalent
to predicate variables, which are allowed in  higher-order logic.  The main
benefit of sets is their notation;  we can write \isa{x{\isasymin}A}
and
\isa{\isacharbraceleft z.\ P\isacharbraceright} where predicates would
require writing
\isa{A(x)} and
\isa{{\isasymlambda}z.\ P}. 
\index{set comprehensions|)}


\subsection{Binding Operators}

\index{quantifiers!for sets|(}%
Universal and existential quantifications may range over sets, 
with the obvious meaning.  Here are the natural deduction rules for the
bounded universal quantifier.  Occasionally you will need to apply
\isa{bspec} with an explicit instantiation of the variable~\isa{x}:
%
\begin{isabelle}
({\isasymAnd}x.\ x\ \isasymin\ A\ \isasymLongrightarrow\ P\ x)\ \isasymLongrightarrow\ {\isasymforall}x\isasymin
A.\ P\ x%
\rulename{ballI}%
\isanewline
\isasymlbrakk{\isasymforall}x\isasymin A.\
P\ x;\ x\ \isasymin\
A\isasymrbrakk\ \isasymLongrightarrow\ P\
x%
\rulename{bspec}
\end{isabelle}
%
Dually, here are the natural deduction rules for the
bounded existential quantifier.  You may need to apply
\isa{bexI} with an explicit instantiation:
\begin{isabelle}
\isasymlbrakk P\ x;\
x\ \isasymin\ A\isasymrbrakk\
\isasymLongrightarrow\
\isasymexists x\isasymin A.\ P\
x%
\rulename{bexI}%
\isanewline
\isasymlbrakk\isasymexists x\isasymin A.\
P\ x;\ {\isasymAnd}x.\
{\isasymlbrakk}x\ \isasymin\ A;\
P\ x\isasymrbrakk\ \isasymLongrightarrow\
Q\isasymrbrakk\ \isasymLongrightarrow\ Q%
\rulename{bexE}
\end{isabelle}
\index{quantifiers!for sets|)}

\index{union!indexed}%
Unions can be formed over the values of a given  set.  The syntax is
\mbox{\isa{\isasymUnion x\isasymin A.\ B}} or \isa{UN
x\isasymin A.\ B} in \textsc{ascii}. Indexed union satisfies this basic law:
\begin{isabelle}
(b\ \isasymin\
(\isasymUnion x\isasymin A.\ B\ x))\ =\ (\isasymexists x\isasymin A.\
b\ \isasymin\ B\ x)
\rulename{UN_iff}
\end{isabelle}
It has two natural deduction rules similar to those for the existential
quantifier.  Sometimes \isa{UN_I} must be applied explicitly:
\begin{isabelle}
\isasymlbrakk a\ \isasymin\
A;\ b\ \isasymin\
B\ a\isasymrbrakk\ \isasymLongrightarrow\
b\ \isasymin\
({\isasymUnion}x\isasymin A.\
B\ x)
\rulename{UN_I}%
\isanewline
\isasymlbrakk b\ \isasymin\
({\isasymUnion}x\isasymin A.\
B\ x);\
{\isasymAnd}x.\ {\isasymlbrakk}x\ \isasymin\
A;\ b\ \isasymin\
B\ x\isasymrbrakk\ \isasymLongrightarrow\
R\isasymrbrakk\ \isasymLongrightarrow\ R%
\rulename{UN_E}
\end{isabelle}
%
The following built-in syntax translation (see \S\ref{sec:def-translations})
lets us express the union over a \emph{type}:
\begin{isabelle}
\ \ \ \ \
({\isasymUnion}x.\ B\ x)\ {\isasymrightleftharpoons}\
({\isasymUnion}x{\isasymin}UNIV.\ B\ x)
\end{isabelle}
%Abbreviations work as you might expect.  The term on the left-hand side of
%the \isasymrightleftharpoons\ symbol is automatically translated to the right-hand side when the
%term is parsed, the reverse translation being done when the term is
%displayed.

We may also express the union of a set of sets, written \isa{Union\ C} in
\textsc{ascii}: 
\begin{isabelle}
(A\ \isasymin\ \isasymUnion C)\ =\ (\isasymexists X\isasymin C.\ A\
\isasymin\ X)
\rulename{Union_iff}
\end{isabelle}

\index{intersection!indexed}%
Intersections are treated dually, although they seem to be used less often
than unions.  The syntax below would be \isa{INT
x:\ A.\ B} and \isa{Inter\ C} in \textsc{ascii}.  Among others, these
theorems are available:
\begin{isabelle}
(b\ \isasymin\
({\isasymInter}x\isasymin A.\
B\ x))\
=\
({\isasymforall}x\isasymin A.\
b\ \isasymin\ B\ x)
\rulename{INT_iff}%
\isanewline
(A\ \isasymin\
\isasymInter C)\ =\
({\isasymforall}X\isasymin C.\
A\ \isasymin\ X)
\rulename{Inter_iff}
\end{isabelle}

Isabelle uses logical equivalences such as those above in automatic proof. 
Unions, intersections and so forth are not simply replaced by their
definitions.  Instead, membership tests are simplified.  For example,  $x\in
A\cup B$ is replaced by $x\in A\vee x\in B$.

The internal form of a comprehension involves the constant  
\isa{Collect},\index{*Collect (constant)}
which occasionally appears when a goal or theorem
is displayed.  For example, \isa{Collect\ P}  is the same term as
\isa{\isacharbraceleft x.\ P\ x\isacharbraceright}.  The same thing can
happen with quantifiers:   \hbox{\isa{All\ P}}\index{*All (constant)}
is 
\isa{{\isasymforall}z.\ P\ x} and 
\hbox{\isa{Ex\ P}}\index{*Ex (constant)} is \isa{\isasymexists z.\ P\ x}; 
also \isa{Ball\ A\ P}\index{*Ball (constant)} is 
\isa{{\isasymforall}z\isasymin A.\ P\ x} and 
\isa{Bex\ A\ P}\index{*Bex (constant)} is 
\isa{\isasymexists z\isasymin A.\ P\ x}.  For indexed unions and
intersections, you may see the constants 
\isa{UNION}\index{*UNION (constant)} and 
\isa{INTER}\index{*INTER (constant)}\@.
The internal constant for $\varepsilon x.P(x)$ is 
\isa{Eps}\index{*Eps (constant)}.


We have only scratched the surface of Isabelle/HOL's set theory. 
One primitive not mentioned here is the powerset operator
{\isa{Pow}}.  Hundreds of theorems are proved in theory \isa{Set} and its
descendants.


\subsection{Finiteness and Cardinality}

\index{sets!finite}%
The predicate \isa{finite} holds of all finite sets.  Isabelle/HOL includes
many familiar theorems about finiteness and cardinality 
(\isa{card}). For example, we have theorems concerning the cardinalities
of unions, intersections and the powerset:\index{cardinality}
%
\begin{isabelle}
{\isasymlbrakk}finite\ A;\ finite\ B\isasymrbrakk\isanewline
\isasymLongrightarrow\ card\ A\ \isacharplus\ card\ B\ =\ card\ (A\ \isasymunion\ B)\ \isacharplus\ card\ (A\ \isasyminter\ B)
\rulename{card_Un_Int}%
\isanewline
\isanewline
finite\ A\ \isasymLongrightarrow\ card\
(Pow\ A)\  =\ 2\ \isacharcircum\ card\ A%
\rulename{card_Pow}%
\isanewline
\isanewline
finite\ A\ \isasymLongrightarrow\isanewline
card\ \isacharbraceleft B.\ B\ \isasymsubseteq\
A\ \isasymand\ card\ B\ =\
k\isacharbraceright\ =\ card\ A\ choose\ k%
\rulename{n_subsets}
\end{isabelle}
Writing $|A|$ as $n$, the last of these theorems says that the number of
$k$-element subsets of~$A$ is \index{binomial coefficients}
$\binom{n}{k}$.

\begin{warn}
The term \isa{finite\ A} is defined via a syntax translation as an
abbreviation for \isa{A \isasymin Finites}, where the constant \isa{Finites}
denotes the set of all finite sets of a given type.  There is no constant
\isa{finite}.
\end{warn}
\index{sets|)}


\section{Functions}

\index{functions|(}%
This section describes a few concepts that involve
functions.  Some of the more important theorems are given along with the 
names. A few sample proofs appear. Unlike with set theory, however, 
we cannot simply state lemmas and expect them to be proved using
\isa{blast}. 

\subsection{Function Basics}

Two functions are \textbf{equal}\indexbold{equality!of functions}
if they yield equal results given equal
arguments.  This is the principle of
\textbf{extensionality}\indexbold{extensionality!for functions} for
functions:
\begin{isabelle}
({\isasymAnd}x.\ f\ x\ =\ g\ x)\ {\isasymLongrightarrow}\ f\ =\ g
\rulename{ext}
\end{isabelle}

\indexbold{function updates}%
Function \textbf{update} is useful for modelling machine states. It has 
the obvious definition and many useful facts are proved about 
it.  In particular, the following equation is installed as a simplification
rule:
\begin{isabelle}
(f(x:=y))\ z\ =\ (if\ z\ =\ x\ then\ y\ else\ f\ z)
\rulename{fun_upd_apply}
\end{isabelle}
Two syntactic points must be noted.  In
\isa{(f(x:=y))\ z} we are applying an updated function to an
argument; the outer parentheses are essential.  A series of two or more
updates can be abbreviated as shown on the left-hand side of this theorem:
\begin{isabelle}
f(x:=y,\ x:=z)\ =\ f(x:=z)
\rulename{fun_upd_upd}
\end{isabelle}
Note also that we can write \isa{f(x:=z)} with only one pair of parentheses
when it is not being applied to an argument.

\medskip
The \bfindex{identity function} and function 
\textbf{composition}\indexbold{composition!of functions} are
defined:
\begin{isabelle}%
id\ \isasymequiv\ {\isasymlambda}x.\ x%
\rulename{id_def}\isanewline
f\ \isasymcirc\ g\ \isasymequiv\
{\isasymlambda}x.\ f\
(g\ x)%
\rulename{o_def}
\end{isabelle}
%
Many familiar theorems concerning the identity and composition 
are proved. For example, we have the associativity of composition: 
\begin{isabelle}
f\ \isasymcirc\ (g\ \isasymcirc\ h)\ =\ f\ \isasymcirc\ g\ \isasymcirc\ h
\rulename{o_assoc}
\end{isabelle}

\subsection{Injections, Surjections, Bijections}

\index{injections}\index{surjections}\index{bijections}%
A function may be \textbf{injective}, \textbf{surjective} or \textbf{bijective}: 
\begin{isabelle}
inj_on\ f\ A\ \isasymequiv\ {\isasymforall}x\isasymin A.\
{\isasymforall}y\isasymin  A.\ f\ x\ =\ f\ y\ \isasymlongrightarrow\ x\
=\ y%
\rulename{inj_on_def}\isanewline
surj\ f\ \isasymequiv\ {\isasymforall}y.\
\isasymexists x.\ y\ =\ f\ x%
\rulename{surj_def}\isanewline
bij\ f\ \isasymequiv\ inj\ f\ \isasymand\ surj\ f
\rulename{bij_def}
\end{isabelle}
The second argument
of \isa{inj_on} lets us express that a function is injective over a
given set. This refinement is useful in higher-order logic, where
functions are total; in some cases, a function's natural domain is a subset
of its domain type.  Writing \isa{inj\ f} abbreviates \isa{inj_on\ f\
UNIV}, for when \isa{f} is injective everywhere.

The operator \isa{inv} expresses the 
\textbf{inverse}\indexbold{inverse!of a function}
of a function. In 
general the inverse may not be well behaved.  We have the usual laws,
such as these: 
\begin{isabelle}
inj\ f\ \ \isasymLongrightarrow\ inv\ f\ (f\ x)\ =\ x%
\rulename{inv_f_f}\isanewline
surj\ f\ \isasymLongrightarrow\ f\ (inv\ f\ y)\ =\ y
\rulename{surj_f_inv_f}\isanewline
bij\ f\ \ \isasymLongrightarrow\ inv\ (inv\ f)\ =\ f
\rulename{inv_inv_eq}
\end{isabelle}
%
%Other useful facts are that the inverse of an injection 
%is a surjection and vice versa; the inverse of a bijection is 
%a bijection. 
%\begin{isabelle}
%inj\ f\ \isasymLongrightarrow\ surj\
%(inv\ f)
%\rulename{inj_imp_surj_inv}\isanewline
%surj\ f\ \isasymLongrightarrow\ inj\ (inv\ f)
%\rulename{surj_imp_inj_inv}\isanewline
%bij\ f\ \isasymLongrightarrow\ bij\ (inv\ f)
%\rulename{bij_imp_bij_inv}
%\end{isabelle}
%
%The converses of these results fail.  Unless a function is 
%well behaved, little can be said about its inverse. Here is another 
%law: 
%\begin{isabelle}
%{\isasymlbrakk}bij\ f;\ bij\ g\isasymrbrakk\ \isasymLongrightarrow\ inv\ (f\ \isasymcirc\ g)\ =\ inv\ g\ \isasymcirc\ inv\ f%
%\rulename{o_inv_distrib}
%\end{isabelle}

Theorems involving these concepts can be hard to prove. The following 
example is easy, but it cannot be proved automatically. To begin 
with, we need a law that relates the equality of functions to 
equality over all arguments: 
\begin{isabelle}
(f\ =\ g)\ =\ ({\isasymforall}x.\ f\ x\ =\ g\ x)
\rulename{expand_fun_eq}
\end{isabelle}
%
This is just a restatement of extensionality.  Our lemma states 
that an injection can be cancelled from the left 
side of function composition: 
\begin{isabelle}
\isacommand{lemma}\ "inj\ f\ \isasymLongrightarrow\ (f\ o\ g\ =\ f\ o\ h)\ =\ (g\ =\ h)"\isanewline
\isacommand{apply}\ (simp\ add:\ expand_fun_eq\ inj_on_def)\isanewline
\isacommand{apply}\ auto\isanewline
\isacommand{done}
\end{isabelle}

The first step of the proof invokes extensionality and the definitions 
of injectiveness and composition. It leaves one subgoal:
\begin{isabelle}
\ 1.\ {\isasymforall}x\ y.\ f\ x\ =\ f\ y\ \isasymlongrightarrow\ x\ =\ y\
\isasymLongrightarrow\isanewline
\ \ \ \ ({\isasymforall}x.\ f\ (g\ x)\ =\ f\ (h\ x))\ =\ ({\isasymforall}x.\ g\ x\ =\ h\ x)
\end{isabelle}
This can be proved using the \isa{auto} method. 


\subsection{Function Image}

The \textbf{image}\indexbold{image!under a function}
of a set under a function is a most useful notion.  It
has the obvious definition: 
\begin{isabelle}
f\ `\ A\ \isasymequiv\ \isacharbraceleft y.\ \isasymexists x\isasymin
A.\ y\ =\ f\ x\isacharbraceright
\rulename{image_def}
\end{isabelle}
%
Here are some of the many facts proved about image: 
\begin{isabelle}
(f\ \isasymcirc\ g)\ `\ r\ =\ f\ `\ g\ `\ r
\rulename{image_compose}\isanewline
f`(A\ \isasymunion\ B)\ =\ f`A\ \isasymunion\ f`B
\rulename{image_Un}\isanewline
inj\ f\ \isasymLongrightarrow\ f`(A\ \isasyminter\
B)\ =\ f`A\ \isasyminter\ f`B
\rulename{image_Int}
%\isanewline
%bij\ f\ \isasymLongrightarrow\ f\ `\ (-\ A)\ =\ -\ f\ `\ A%
%\rulename{bij_image_Compl_eq}
\end{isabelle}


Laws involving image can often be proved automatically. Here 
are two examples, illustrating connections with indexed union and with the
general syntax for comprehension:
\begin{isabelle}
\isacommand{lemma}\ "f`A\ \isasymunion\ g`A\ =\ ({\isasymUnion}x{\isasymin}A.\ \isacharbraceleft f\ x,\ g\
x\isacharbraceright)
\par\smallskip
\isacommand{lemma}\ "f\ `\ \isacharbraceleft(x,y){.}\ P\ x\ y\isacharbraceright\ =\ \isacharbraceleft f(x,y)\ \isacharbar\ x\ y.\ P\ x\
y\isacharbraceright"
\end{isabelle}

\medskip
 A function's \textbf{range} is the set of values that the function can 
take on. It is, in fact, the image of the universal set under 
that function. There is no constant {\isa{range}}.  Instead, {\isa{range}} 
abbreviates an application of image to {\isa{UNIV}}: 
\begin{isabelle}
\ \ \ \ \ range\ f\
{\isasymrightleftharpoons}\ f`UNIV
\end{isabelle}
%
Few theorems are proved specifically 
for {\isa{range}}; in most cases, you should look for a more general
theorem concerning images.

\medskip
\textbf{Inverse image}\index{inverse image!of a function} is also  useful.
It is defined as follows: 
\begin{isabelle}
f\ -`\ B\ \isasymequiv\ \isacharbraceleft x.\ f\ x\ \isasymin\ B\isacharbraceright
\rulename{vimage_def}
\end{isabelle}
%
This is one of the facts proved about it:
\begin{isabelle}
f\ -`\ (-\ A)\ =\ -\ f\ -`\ A%
\rulename{vimage_Compl}
\end{isabelle}
\index{functions|)}


\section{Relations}
\label{sec:Relations}

\index{relations|(}%
A \textbf{relation} is a set of pairs.  As such, the set operations apply
to them.  For instance, we may form the union of two relations.  Other
primitives are defined specifically for relations. 

\subsection{Relation Basics}

The \bfindex{identity relation}, also known as equality, has the obvious 
definition: 
\begin{isabelle}
Id\ \isasymequiv\ \isacharbraceleft p.\ \isasymexists x.\ p\ =\ (x,x){\isacharbraceright}%
\rulename{Id_def}
\end{isabelle}

\indexbold{composition!of relations}%
\textbf{Composition} of relations (the infix \isa{O}) is also available: 
\begin{isabelle}
r\ O\ s\ \isasymequiv\ \isacharbraceleft(x,z).\ \isasymexists y.\ (x,y)\ \isasymin\ s\ \isasymand\ (y,z)\ \isasymin\ r\isacharbraceright
\rulename{comp_def}
\end{isabelle}
%
This is one of the many lemmas proved about these concepts: 
\begin{isabelle}
R\ O\ Id\ =\ R
\rulename{R_O_Id}
\end{isabelle}
%
Composition is monotonic, as are most of the primitives appearing 
in this chapter.  We have many theorems similar to the following 
one: 
\begin{isabelle}
\isasymlbrakk r\isacharprime\ \isasymsubseteq\ r;\ s\isacharprime\
\isasymsubseteq\ s\isasymrbrakk\ \isasymLongrightarrow\ r\isacharprime\ O\
s\isacharprime\ \isasymsubseteq\ r\ O\ s%
\rulename{comp_mono}
\end{isabelle}

\indexbold{converse!of a relation}%
\indexbold{inverse!of a relation}%
The \textbf{converse} or inverse of a
relation exchanges the roles  of the two operands.  We use the postfix
notation~\isa{r\isasyminverse} or
\isa{r\isacharcircum-1} in ASCII\@.
\begin{isabelle}
((a,b)\ \isasymin\ r\isasyminverse)\ =\
((b,a)\ \isasymin\ r)
\rulename{converse_iff}
\end{isabelle}
%
Here is a typical law proved about converse and composition: 
\begin{isabelle}
(r\ O\ s)\isasyminverse\ =\ s\isasyminverse\ O\ r\isasyminverse
\rulename{converse_comp}
\end{isabelle}

\indexbold{image!under a relation}%
The \textbf{image} of a set under a relation is defined
analogously  to image under a function: 
\begin{isabelle}
(b\ \isasymin\ r\ ``\ A)\ =\ (\isasymexists x\isasymin
A.\ (x,b)\ \isasymin\ r)
\rulename{Image_iff}
\end{isabelle}
It satisfies many similar laws.

%Image under relations, like image under functions, distributes over unions: 
%\begin{isabelle}
%r\ ``\ 
%({\isasymUnion}x\isasyminA.\
%B\
%x)\ =\ 
%({\isasymUnion}x\isasyminA.\
%r\ ``\ B\
%x)
%\rulename{Image_UN}
%\end{isabelle}


The \bfindex{domain} and \bfindex{range} of a relation are defined in the 
standard way: 
\begin{isabelle}
(a\ \isasymin\ Domain\ r)\ =\ (\isasymexists y.\ (a,y)\ \isasymin\
r)
\rulename{Domain_iff}%
\isanewline
(a\ \isasymin\ Range\ r)\
\ =\ (\isasymexists y.\
(y,a)\
\isasymin\ r)
\rulename{Range_iff}
\end{isabelle}

Iterated composition of a relation is available.  The notation overloads 
that of exponentiation: 
\begin{isabelle}
R\ \isacharcircum\ \isadigit{0}\ =\ Id\isanewline
R\ \isacharcircum\ Suc\ n\ =\ R\ O\ R\isacharcircum n
\rulename{relpow.simps}
\end{isabelle}

\subsection{The Reflexive and Transitive Closure}

\index{closure!reflexive and transitive|(}%
The \textbf{reflexive and transitive closure} of the
relation~\isa{r} is written with a
postfix syntax.  In ASCII we write \isa{r\isacharcircum*} and in
X-symbol notation~\isa{r\isactrlsup *}.  It is the least solution of the
equation
\begin{isabelle}
r\isactrlsup *\ =\ Id\ \isasymunion \ (r\ O\ r\isactrlsup *)
\rulename{rtrancl_unfold}
\end{isabelle}
%
Among its basic properties are three that serve as introduction 
rules:
\begin{isabelle}
(a,\ a)\ \isasymin \ r\isactrlsup *
\rulename{rtrancl_refl}\isanewline
p\ \isasymin \ r\ \isasymLongrightarrow \ p\ \isasymin \ r\isactrlsup *
\rulename{r_into_rtrancl}\isanewline
\isasymlbrakk (a,b)\ \isasymin \ r\isactrlsup *;\ 
(b,c)\ \isasymin \ r\isactrlsup *\isasymrbrakk \ \isasymLongrightarrow \
(a,c)\ \isasymin \ r\isactrlsup *
\rulename{rtrancl_trans}
\end{isabelle}
%
Induction over the reflexive transitive closure is available: 
\begin{isabelle}
\isasymlbrakk (a,\ b)\ \isasymin \ r\isactrlsup *;\ P\ a;\ \isasymAnd y\ z.\ \isasymlbrakk (a,\ y)\ \isasymin \ r\isactrlsup *;\ (y,\ z)\ \isasymin \ r;\ P\ y\isasymrbrakk \ \isasymLongrightarrow \ P\ z\isasymrbrakk \isanewline
\isasymLongrightarrow \ P\ b%
\rulename{rtrancl_induct}
\end{isabelle}
%
Idempotence is one of the laws proved about the reflexive transitive 
closure: 
\begin{isabelle}
(r\isactrlsup *)\isactrlsup *\ =\ r\isactrlsup *
\rulename{rtrancl_idemp}
\end{isabelle}

\smallskip
The transitive closure is similar.  The ASCII syntax is
\isa{r\isacharcircum+}.  It has two  introduction rules: 
\begin{isabelle}
p\ \isasymin \ r\ \isasymLongrightarrow \ p\ \isasymin \ r\isactrlsup +
\rulename{r_into_trancl}\isanewline
\isasymlbrakk (a,\ b)\ \isasymin \ r\isactrlsup +;\ (b,\ c)\ \isasymin \ r\isactrlsup +\isasymrbrakk \ \isasymLongrightarrow \ (a,\ c)\ \isasymin \ r\isactrlsup +
\rulename{trancl_trans}
\end{isabelle}
%
The induction rule resembles the one shown above. 
A typical lemma states that transitive closure commutes with the converse
operator: 
\begin{isabelle}
(r\isasyminverse )\isactrlsup +\ =\ (r\isactrlsup +)\isasyminverse 
\rulename{trancl_converse}
\end{isabelle}

\subsection{A Sample Proof}

The reflexive transitive closure also commutes with the converse. 
Let us examine the proof. Each direction of the equivalence is 
proved separately. The two proofs are almost identical. Here 
is the first one: 
\begin{isabelle}
\isacommand{lemma}\ rtrancl_converseD:\ "(x,y)\ \isasymin \
(r\isasyminverse)\isactrlsup *\ \isasymLongrightarrow \ (y,x)\ \isasymin
\ r\isactrlsup *"\isanewline
\isacommand{apply}\ (erule\ rtrancl_induct)\isanewline
\ \isacommand{apply}\ (rule\ rtrancl_refl)\isanewline
\isacommand{apply}\ (blast\ intro:\ rtrancl_trans)\isanewline
\isacommand{done}
\end{isabelle}
%
The first step of the proof applies induction, leaving these subgoals: 
\begin{isabelle}
\ 1.\ (x,\ x)\ \isasymin \ r\isactrlsup *\isanewline
\ 2.\ \isasymAnd y\ z.\ \isasymlbrakk (x,y)\ \isasymin \
(r\isasyminverse)\isactrlsup *;\ (y,z)\ \isasymin \ r\isasyminverse ;\
(y,x)\ \isasymin \ r\isactrlsup *\isasymrbrakk \isanewline
\ \ \ \ \ \ \ \ \ \ \isasymLongrightarrow \ (z,x)\ \isasymin \ r\isactrlsup *
\end{isabelle}
%
The first subgoal is trivial by reflexivity. The second follows 
by first eliminating the converse operator, yielding the
assumption \isa{(z,y)\
\isasymin\ r}, and then
applying the introduction rules shown above.  The same proof script handles
the other direction: 
\begin{isabelle}
\isacommand{lemma}\ rtrancl_converseI:\ "(y,x)\ \isasymin \ r\isactrlsup *\ \isasymLongrightarrow \ (x,y)\ \isasymin \ (r\isasyminverse)\isactrlsup *"\isanewline
\isacommand{apply}\ (erule\ rtrancl_induct)\isanewline
\ \isacommand{apply}\ (rule\ rtrancl_refl)\isanewline
\isacommand{apply}\ (blast\ intro:\ rtrancl_trans)\isanewline
\isacommand{done}
\end{isabelle}


Finally, we combine the two lemmas to prove the desired equation: 
\begin{isabelle}
\isacommand{lemma}\ rtrancl_converse:\ "(r\isasyminverse)\isactrlsup *\ =\ (r\isactrlsup *)\isasyminverse"\isanewline
\isacommand{by}\ (auto\ intro:\ rtrancl_converseI\ dest:\
rtrancl_converseD)
\end{isabelle}

\begin{warn}
Note that \isa{blast} cannot prove this theorem.  Here is a subgoal that
arises internally after  the rules \isa{equalityI} and \isa{subsetI} have
been applied:
\begin{isabelle}
\ 1.\ \isasymAnd x.\ x\ \isasymin \ (r\isasyminverse )\isactrlsup *\ \isasymLongrightarrow \ x\ \isasymin \ (r\isactrlsup
*)\isasyminverse
%ignore subgoal 2
%\ 2.\ \isasymAnd x.\ x\ \isasymin \ (r\isactrlsup *)\isasyminverse \
%\isasymLongrightarrow \ x\ \isasymin \ (r\isasyminverse )\isactrlsup *
\end{isabelle}
\par\noindent
We cannot use \isa{rtrancl_converseD}\@.  It refers to
ordered pairs, while \isa{x} is a variable of product type.
The \isa{simp} and \isa{blast} methods can do nothing, so let us try
\isa{clarify}:
\begin{isabelle}
\ 1.\ \isasymAnd a\ b.\ (a,b)\ \isasymin \ (r\isasyminverse )\isactrlsup *\ \isasymLongrightarrow \ (b,a)\ \isasymin \ r\isactrlsup
*
\end{isabelle}
Now that \isa{x} has been replaced by the pair \isa{(a,b)}, we can
proceed.  Other methods that split variables in this way are \isa{force},
\isa{auto}, \isa{fast} and \isa{best}.  Section~\ref{sec:products} will discuss proof
techniques for ordered pairs in more detail.
\end{warn}
\index{relations|)}\index{closure!reflexive and transitive|)}


\section{Well-Founded Relations and Induction}
\label{sec:Well-founded}

\index{relations!well-founded|(}%
A well-founded relation captures the notion of a terminating process. 
Each \isacommand{recdef}\index{recdef@\isacommand{recdef}}
declaration must specify a well-founded relation that
justifies the termination of the desired recursive function.  Most of the
forms of induction found in mathematics are merely special cases of
induction over a well-founded relation.

Intuitively, the relation~$\prec$ is \textbf{well-founded} if it admits no
infinite  descending chains
\[ \cdots \prec a@2 \prec a@1 \prec a@0. \]
Well-foundedness can be hard to show. The various 
formulations are all complicated.  However,  often a relation
is well-founded by construction.  HOL provides
theorems concerning ways of constructing  a well-founded relation.  The
most familiar way is to specify a \bfindex{measure function}~\isa{f} into
the natural numbers, when $\isa{x}\prec \isa{y}\iff \isa{f x} < \isa{f y}$;
we write this particular relation as
\isa{measure~f}.

\begin{warn}
You may want to skip the rest of this section until you need to perform a
complex recursive function definition or induction.  The induction rule
returned by
\isacommand{recdef} is good enough for most purposes.  We use an explicit
well-founded induction only in \S\ref{sec:CTL-revisited}.
\end{warn}

Isabelle/HOL declares \isa{less_than} as a relation object, 
that is, a set of pairs of natural numbers. Two theorems tell us that this
relation  behaves as expected and that it is well-founded: 
\begin{isabelle}
((x,y)\ \isasymin\ less_than)\ =\ (x\ <\ y)
\rulename{less_than_iff}\isanewline
wf\ less_than
\rulename{wf_less_than}
\end{isabelle}

The notion of measure generalizes to the 
\index{inverse image!of a relation}\textbf{inverse image} of
a relation. Given a relation~\isa{r} and a function~\isa{f}, we express  a
new relation using \isa{f} as a measure.  An infinite descending chain on
this new relation would give rise to an infinite descending chain
on~\isa{r}.  Isabelle/HOL defines this concept and proves a
theorem stating that it preserves well-foundedness: 
\begin{isabelle}
inv_image\ r\ f\ \isasymequiv\ \isacharbraceleft(x,y).\ (f\ x,\ f\ y)\
\isasymin\ r\isacharbraceright
\rulename{inv_image_def}\isanewline
wf\ r\ \isasymLongrightarrow\ wf\ (inv_image\ r\ f)
\rulename{wf_inv_image}
\end{isabelle}

A measure function involves the natural numbers.  The relation \isa{measure
size} justifies primitive recursion and structural induction over a
datatype.  Isabelle/HOL defines
\isa{measure} as shown: 
\begin{isabelle}
measure\ \isasymequiv\ inv_image\ less_than%
\rulename{measure_def}\isanewline
wf\ (measure\ f)
\rulename{wf_measure}
\end{isabelle}

Of the other constructions, the most important is the \textbf{lexicographic 
product} of two relations. It expresses the standard dictionary 
ordering over pairs.  We write \isa{ra\ <*lex*>\ rb}, where \isa{ra}
and \isa{rb} are the two operands.  The lexicographic product satisfies the
usual  definition and it preserves well-foundedness: 
\begin{isabelle}
ra\ <*lex*>\ rb\ \isasymequiv \isanewline
\ \ \isacharbraceleft ((a,b),(a',b')).\ (a,a')\ \isasymin \ ra\
\isasymor\isanewline
\ \ \ \ \ \ \ \ \ \ \ \ \ \ \ \ \ \ \ \ \,a=a'\ \isasymand \ (b,b')\
\isasymin \ rb\isacharbraceright 
\rulename{lex_prod_def}%
\par\smallskip
\isasymlbrakk wf\ ra;\ wf\ rb\isasymrbrakk \ \isasymLongrightarrow \ wf\ (ra\ <*lex*>\ rb)
\rulename{wf_lex_prod}
\end{isabelle}

These constructions can be used in a
\textbf{recdef} declaration (\S\ref{sec:recdef-simplification}) to define
the well-founded relation used to prove termination.

The \bfindex{multiset ordering}, useful for hard termination proofs, is
available in the Library.  Baader and Nipkow \cite[\S2.5]{Baader-Nipkow}
discuss it. 

\medskip
Induction comes in many forms, including traditional mathematical 
induction, structural induction on lists and induction on size.  All are
instances of the following rule, for a suitable well-founded
relation~$\prec$: 
\[ \infer{P(a)}{\infer*{P(x)}{[\forall y.\, y\prec x \imp P(y)]}} \]
To show $P(a)$ for a particular term~$a$, it suffices to show $P(x)$ for
arbitrary~$x$ under the assumption that $P(y)$ holds for $y\prec x$. 
Intuitively, the well-foundedness of $\prec$ ensures that the chains of
reasoning are finite.

\smallskip
In Isabelle, the induction rule is expressed like this:
\begin{isabelle}
{\isasymlbrakk}wf\ r;\ 
  {\isasymAnd}x.\ {\isasymforall}y.\ (y,x)\ \isasymin\ r\
\isasymlongrightarrow\ P\ y\ \isasymLongrightarrow\ P\ x\isasymrbrakk\
\isasymLongrightarrow\ P\ a
\rulename{wf_induct}
\end{isabelle}
Here \isa{wf\ r} expresses that the relation~\isa{r} is well-founded.

Many familiar induction principles are instances of this rule. 
For example, the predecessor relation on the natural numbers 
is well-founded; induction over it is mathematical induction. 
The ``tail of'' relation on lists is well-founded; induction over 
it is structural induction. 
\index{relations!well-founded|)}


\section{Fixed Point Operators}

\index{fixed points|(}%
Fixed point operators define sets
recursively.  They are invoked implicitly when making an inductive
definition, as discussed in Chap.\ts\ref{chap:inductive} below.  However,
they can be used directly, too. The
\emph{least}  or \emph{strongest} fixed point yields an inductive
definition;  the \emph{greatest} or \emph{weakest} fixed point yields a
coinductive  definition.  Mathematicians may wish to note that the
existence  of these fixed points is guaranteed by the Knaster-Tarski
theorem. 

\begin{warn}
Casual readers should skip the rest of this section.  We use fixed point
operators only in \S\ref{sec:VMC}.
\end{warn}

The theory applies only to monotonic functions. Isabelle's 
definition of monotone is overloaded over all orderings:
\begin{isabelle}
mono\ f\ \isasymequiv\ {\isasymforall}A\ B.\ A\ \isasymle\ B\ \isasymlongrightarrow\ f\ A\ \isasymle\ f\ B%
\rulename{mono_def}
\end{isabelle}
%
For fixed point operators, the ordering will be the subset relation: if
$A\subseteq B$ then we expect $f(A)\subseteq f(B)$.  In addition to its
definition, monotonicity has the obvious introduction and destruction
rules:
\begin{isabelle}
({\isasymAnd}A\ B.\ A\ \isasymle\ B\ \isasymLongrightarrow\ f\ A\ \isasymle\ f\ B)\ \isasymLongrightarrow\ mono\ f%
\rulename{monoI}%
\par\smallskip%          \isanewline didn't leave enough space
{\isasymlbrakk}mono\ f;\ A\ \isasymle\ B\isasymrbrakk\
\isasymLongrightarrow\ f\ A\ \isasymle\ f\ B%
\rulename{monoD}
\end{isabelle}

The most important properties of the least fixed point are that 
it is a fixed point and that it enjoys an induction rule: 
\begin{isabelle}
mono\ f\ \isasymLongrightarrow\ lfp\ f\ =\ f\ (lfp\ f)
\rulename{lfp_unfold}%
\par\smallskip%          \isanewline didn't leave enough space
{\isasymlbrakk}a\ \isasymin\ lfp\ f;\ mono\ f;\isanewline
  \ {\isasymAnd}x.\ x\
\isasymin\ f\ (lfp\ f\ \isasyminter\ \isacharbraceleft x.\ P\
x\isacharbraceright)\ \isasymLongrightarrow\ P\ x\isasymrbrakk\
\isasymLongrightarrow\ P\ a%
\rulename{lfp_induct}
\end{isabelle}
%
The induction rule shown above is more convenient than the basic 
one derived from the minimality of {\isa{lfp}}.  Observe that both theorems
demand \isa{mono\ f} as a premise.

The greatest fixed point is similar, but it has a \bfindex{coinduction} rule: 
\begin{isabelle}
mono\ f\ \isasymLongrightarrow\ gfp\ f\ =\ f\ (gfp\ f)
\rulename{gfp_unfold}%
\isanewline
{\isasymlbrakk}mono\ f;\ a\ \isasymin\ X;\ X\ \isasymsubseteq\ f\ (X\
\isasymunion\ gfp\ f)\isasymrbrakk\ \isasymLongrightarrow\ a\ \isasymin\
gfp\ f%
\rulename{coinduct}
\end{isabelle}
A \bfindex{bisimulation} is perhaps the best-known concept defined as a
greatest fixed point.  Exhibiting a bisimulation to prove the equality of
two agents in a process algebra is an example of coinduction.
The coinduction rule can be strengthened in various ways; see 
theory \isa{Gfp} for details.  
\index{fixed points|)}
\index{model checking example|(}%
\index{lfp@{\texttt{lfp}}!applications of|see{CTL}}
%
\begin{isabellebody}%
\def\isabellecontext{Base}%
%
\isadelimtheory
%
\endisadelimtheory
%
\isatagtheory
\isacommand{theory}\isamarkupfalse%
\ Base\isanewline
\isakeyword{imports}\ Pure\isanewline
\isakeyword{begin}%
\endisatagtheory
{\isafoldtheory}%
%
\isadelimtheory
\isanewline
%
\endisadelimtheory
%
\isadelimML
\isanewline
%
\endisadelimML
%
\isatagML
\isacommand{ML{\isaliteral{5F}{\isacharunderscore}}file}\isamarkupfalse%
\ {\isaliteral{22}{\isachardoublequoteopen}}{\isaliteral{2E}{\isachardot}}{\isaliteral{2E}{\isachardot}}{\isaliteral{2F}{\isacharslash}}{\isaliteral{2E}{\isachardot}}{\isaliteral{2E}{\isachardot}}{\isaliteral{2F}{\isacharslash}}antiquote{\isaliteral{5F}{\isacharunderscore}}setup{\isaliteral{2E}{\isachardot}}ML{\isaliteral{22}{\isachardoublequoteclose}}\isanewline
\isacommand{setup}\isamarkupfalse%
\ Antiquote{\isaliteral{5F}{\isacharunderscore}}Setup{\isaliteral{2E}{\isachardot}}setup%
\endisatagML
{\isafoldML}%
%
\isadelimML
\isanewline
%
\endisadelimML
\isanewline
\isacommand{declare}\isamarkupfalse%
\ {\isaliteral{5B}{\isacharbrackleft}}{\isaliteral{5B}{\isacharbrackleft}}thy{\isaliteral{5F}{\isacharunderscore}}output{\isaliteral{5F}{\isacharunderscore}}source{\isaliteral{5D}{\isacharbrackright}}{\isaliteral{5D}{\isacharbrackright}}\isanewline
%
\isadelimtheory
\isanewline
%
\endisadelimtheory
%
\isatagtheory
\isacommand{end}\isamarkupfalse%
%
\endisatagtheory
{\isafoldtheory}%
%
\isadelimtheory
\isanewline
%
\endisadelimtheory
\end{isabellebody}%
%%% Local Variables:
%%% mode: latex
%%% TeX-master: "root"
%%% End:

%
\begin{isabellebody}%
\def\isabellecontext{PDL}%
\isamarkupfalse%
%
\isamarkupsubsection{Propositional Dynamic Logic --- PDL%
}
\isamarkuptrue%
%
\begin{isamarkuptext}%
\index{PDL|(}
The formulae of PDL are built up from atomic propositions via
negation and conjunction and the two temporal
connectives \isa{AX} and \isa{EF}\@. Since formulae are essentially
syntax trees, they are naturally modelled as a datatype:%
\footnote{The customary definition of PDL
\cite{HarelKT-DL} looks quite different from ours, but the two are easily
shown to be equivalent.}%
\end{isamarkuptext}%
\isamarkuptrue%
\isacommand{datatype}\ formula\ {\isacharequal}\ Atom\ atom\isanewline
\ \ \ \ \ \ \ \ \ \ \ \ \ \ \ \ \ \ {\isacharbar}\ Neg\ formula\isanewline
\ \ \ \ \ \ \ \ \ \ \ \ \ \ \ \ \ \ {\isacharbar}\ And\ formula\ formula\isanewline
\ \ \ \ \ \ \ \ \ \ \ \ \ \ \ \ \ \ {\isacharbar}\ AX\ formula\isanewline
\ \ \ \ \ \ \ \ \ \ \ \ \ \ \ \ \ \ {\isacharbar}\ EF\ formula\isamarkupfalse%
%
\begin{isamarkuptext}%
\noindent
This resembles the boolean expression case study in
\S\ref{sec:boolex}.
A validity relation between
states and formulae specifies the semantics:%
\end{isamarkuptext}%
\isamarkuptrue%
\isacommand{consts}\ valid\ {\isacharcolon}{\isacharcolon}\ {\isachardoublequote}state\ {\isasymRightarrow}\ formula\ {\isasymRightarrow}\ bool{\isachardoublequote}\ \ \ {\isacharparenleft}{\isachardoublequote}{\isacharparenleft}{\isacharunderscore}\ {\isasymTurnstile}\ {\isacharunderscore}{\isacharparenright}{\isachardoublequote}\ {\isacharbrackleft}{\isadigit{8}}{\isadigit{0}}{\isacharcomma}{\isadigit{8}}{\isadigit{0}}{\isacharbrackright}\ {\isadigit{8}}{\isadigit{0}}{\isacharparenright}\isamarkupfalse%
%
\begin{isamarkuptext}%
\noindent
The syntax annotation allows us to write \isa{s\ {\isasymTurnstile}\ f} instead of
\hbox{\isa{valid\ s\ f}}.
The definition of \isa{{\isasymTurnstile}} is by recursion over the syntax:%
\end{isamarkuptext}%
\isamarkuptrue%
\isacommand{primrec}\isanewline
{\isachardoublequote}s\ {\isasymTurnstile}\ Atom\ a\ \ {\isacharequal}\ {\isacharparenleft}a\ {\isasymin}\ L\ s{\isacharparenright}{\isachardoublequote}\isanewline
{\isachardoublequote}s\ {\isasymTurnstile}\ Neg\ f\ \ \ {\isacharequal}\ {\isacharparenleft}{\isasymnot}{\isacharparenleft}s\ {\isasymTurnstile}\ f{\isacharparenright}{\isacharparenright}{\isachardoublequote}\isanewline
{\isachardoublequote}s\ {\isasymTurnstile}\ And\ f\ g\ {\isacharequal}\ {\isacharparenleft}s\ {\isasymTurnstile}\ f\ {\isasymand}\ s\ {\isasymTurnstile}\ g{\isacharparenright}{\isachardoublequote}\isanewline
{\isachardoublequote}s\ {\isasymTurnstile}\ AX\ f\ \ \ \ {\isacharequal}\ {\isacharparenleft}{\isasymforall}t{\isachardot}\ {\isacharparenleft}s{\isacharcomma}t{\isacharparenright}\ {\isasymin}\ M\ {\isasymlongrightarrow}\ t\ {\isasymTurnstile}\ f{\isacharparenright}{\isachardoublequote}\isanewline
{\isachardoublequote}s\ {\isasymTurnstile}\ EF\ f\ \ \ \ {\isacharequal}\ {\isacharparenleft}{\isasymexists}t{\isachardot}\ {\isacharparenleft}s{\isacharcomma}t{\isacharparenright}\ {\isasymin}\ M\isactrlsup {\isacharasterisk}\ {\isasymand}\ t\ {\isasymTurnstile}\ f{\isacharparenright}{\isachardoublequote}\isamarkupfalse%
%
\begin{isamarkuptext}%
\noindent
The first three equations should be self-explanatory. The temporal formula
\isa{AX\ f} means that \isa{f} is true in \emph{A}ll ne\emph{X}t states whereas
\isa{EF\ f} means that there \emph{E}xists some \emph{F}uture state in which \isa{f} is
true. The future is expressed via \isa{\isactrlsup {\isacharasterisk}}, the reflexive transitive
closure. Because of reflexivity, the future includes the present.

Now we come to the model checker itself. It maps a formula into the set of
states where the formula is true.  It too is defined by recursion over the syntax:%
\end{isamarkuptext}%
\isamarkuptrue%
\isacommand{consts}\ mc\ {\isacharcolon}{\isacharcolon}\ {\isachardoublequote}formula\ {\isasymRightarrow}\ state\ set{\isachardoublequote}\isanewline
\isamarkupfalse%
\isacommand{primrec}\isanewline
{\isachardoublequote}mc{\isacharparenleft}Atom\ a{\isacharparenright}\ \ {\isacharequal}\ {\isacharbraceleft}s{\isachardot}\ a\ {\isasymin}\ L\ s{\isacharbraceright}{\isachardoublequote}\isanewline
{\isachardoublequote}mc{\isacharparenleft}Neg\ f{\isacharparenright}\ \ \ {\isacharequal}\ {\isacharminus}mc\ f{\isachardoublequote}\isanewline
{\isachardoublequote}mc{\isacharparenleft}And\ f\ g{\isacharparenright}\ {\isacharequal}\ mc\ f\ {\isasyminter}\ mc\ g{\isachardoublequote}\isanewline
{\isachardoublequote}mc{\isacharparenleft}AX\ f{\isacharparenright}\ \ \ \ {\isacharequal}\ {\isacharbraceleft}s{\isachardot}\ {\isasymforall}t{\isachardot}\ {\isacharparenleft}s{\isacharcomma}t{\isacharparenright}\ {\isasymin}\ M\ \ {\isasymlongrightarrow}\ t\ {\isasymin}\ mc\ f{\isacharbraceright}{\isachardoublequote}\isanewline
{\isachardoublequote}mc{\isacharparenleft}EF\ f{\isacharparenright}\ \ \ \ {\isacharequal}\ lfp{\isacharparenleft}{\isasymlambda}T{\isachardot}\ mc\ f\ {\isasymunion}\ {\isacharparenleft}M{\isasyminverse}\ {\isacharbackquote}{\isacharbackquote}\ T{\isacharparenright}{\isacharparenright}{\isachardoublequote}\isamarkupfalse%
%
\begin{isamarkuptext}%
\noindent
Only the equation for \isa{EF} deserves some comments. Remember that the
postfix \isa{{\isasyminverse}} and the infix \isa{{\isacharbackquote}{\isacharbackquote}} are predefined and denote the
converse of a relation and the image of a set under a relation.  Thus
\isa{M{\isasyminverse}\ {\isacharbackquote}{\isacharbackquote}\ T} is the set of all predecessors of \isa{T} and the least
fixed point (\isa{lfp}) of \isa{{\isasymlambda}T{\isachardot}\ mc\ f\ {\isasymunion}\ M{\isasyminverse}\ {\isacharbackquote}{\isacharbackquote}\ T} is the least set
\isa{T} containing \isa{mc\ f} and all predecessors of \isa{T}. If you
find it hard to see that \isa{mc\ {\isacharparenleft}EF\ f{\isacharparenright}} contains exactly those states from
which there is a path to a state where \isa{f} is true, do not worry --- this
will be proved in a moment.

First we prove monotonicity of the function inside \isa{lfp}
in order to make sure it really has a least fixed point.%
\end{isamarkuptext}%
\isamarkuptrue%
\isacommand{lemma}\ mono{\isacharunderscore}ef{\isacharcolon}\ {\isachardoublequote}mono{\isacharparenleft}{\isasymlambda}T{\isachardot}\ A\ {\isasymunion}\ {\isacharparenleft}M{\isasyminverse}\ {\isacharbackquote}{\isacharbackquote}\ T{\isacharparenright}{\isacharparenright}{\isachardoublequote}\isanewline
\isamarkupfalse%
\isamarkupfalse%
\isamarkupfalse%
\isamarkupfalse%
%
\begin{isamarkuptext}%
\noindent
Now we can relate model checking and semantics. For the \isa{EF} case we need
a separate lemma:%
\end{isamarkuptext}%
\isamarkuptrue%
\isacommand{lemma}\ EF{\isacharunderscore}lemma{\isacharcolon}\isanewline
\ \ {\isachardoublequote}lfp{\isacharparenleft}{\isasymlambda}T{\isachardot}\ A\ {\isasymunion}\ {\isacharparenleft}M{\isasyminverse}\ {\isacharbackquote}{\isacharbackquote}\ T{\isacharparenright}{\isacharparenright}\ {\isacharequal}\ {\isacharbraceleft}s{\isachardot}\ {\isasymexists}t{\isachardot}\ {\isacharparenleft}s{\isacharcomma}t{\isacharparenright}\ {\isasymin}\ M\isactrlsup {\isacharasterisk}\ {\isasymand}\ t\ {\isasymin}\ A{\isacharbraceright}{\isachardoublequote}\isamarkupfalse%
\isamarkuptrue%
\isamarkupfalse%
\isamarkupfalse%
\isamarkupfalse%
\isamarkupfalse%
\isamarkuptrue%
\isamarkupfalse%
\isamarkupfalse%
\isamarkupfalse%
\isamarkuptrue%
\isamarkupfalse%
\isamarkuptrue%
\isamarkupfalse%
\isamarkupfalse%
\isamarkuptrue%
\isamarkupfalse%
\isamarkuptrue%
\isamarkupfalse%
\isamarkuptrue%
\isamarkupfalse%
\isamarkuptrue%
\isamarkupfalse%
\isamarkupfalse%
\isamarkupfalse%
%
\begin{isamarkuptext}%
The main theorem is proved in the familiar manner: induction followed by
\isa{auto} augmented with the lemma as a simplification rule.%
\end{isamarkuptext}%
\isamarkuptrue%
\isacommand{theorem}\ {\isachardoublequote}mc\ f\ {\isacharequal}\ {\isacharbraceleft}s{\isachardot}\ s\ {\isasymTurnstile}\ f{\isacharbraceright}{\isachardoublequote}\isanewline
\isamarkupfalse%
\isamarkupfalse%
\isamarkupfalse%
\isamarkupfalse%
%
\begin{isamarkuptext}%
\begin{exercise}
\isa{AX} has a dual operator \isa{EN} 
(``there exists a next state such that'')%
\footnote{We cannot use the customary \isa{EX}: it is reserved
as the \textsc{ascii}-equivalent of \isa{{\isasymexists}}.}
with the intended semantics
\begin{isabelle}%
\ \ \ \ \ s\ {\isasymTurnstile}\ EN\ f\ {\isacharequal}\ {\isacharparenleft}{\isasymexists}t{\isachardot}\ {\isacharparenleft}s{\isacharcomma}\ t{\isacharparenright}\ {\isasymin}\ M\ {\isasymand}\ t\ {\isasymTurnstile}\ f{\isacharparenright}%
\end{isabelle}
Fortunately, \isa{EN\ f} can already be expressed as a PDL formula. How?

Show that the semantics for \isa{EF} satisfies the following recursion equation:
\begin{isabelle}%
\ \ \ \ \ s\ {\isasymTurnstile}\ EF\ f\ {\isacharequal}\ {\isacharparenleft}s\ {\isasymTurnstile}\ f\ {\isasymor}\ s\ {\isasymTurnstile}\ EN\ {\isacharparenleft}EF\ f{\isacharparenright}{\isacharparenright}%
\end{isabelle}
\end{exercise}
\index{PDL|)}%
\end{isamarkuptext}%
\isamarkuptrue%
\isamarkupfalse%
\isamarkupfalse%
\isamarkupfalse%
\isamarkupfalse%
\isamarkupfalse%
\isamarkupfalse%
\isamarkupfalse%
\isamarkupfalse%
\isamarkupfalse%
\isamarkupfalse%
\isamarkupfalse%
\isamarkupfalse%
\isamarkupfalse%
\end{isabellebody}%
%%% Local Variables:
%%% mode: latex
%%% TeX-master: "root"
%%% End:

%
\begin{isabellebody}%
\def\isabellecontext{CTL}%
%
\isadelimtheory
%
\endisadelimtheory
%
\isatagtheory
\isamarkupfalse%
%
\endisatagtheory
{\isafoldtheory}%
%
\isadelimtheory
%
\endisadelimtheory
%
\isamarkupsubsection{Computation Tree Logic --- CTL%
}
\isamarkuptrue%
%
\begin{isamarkuptext}%
\label{sec:CTL}
\index{CTL|(}%
The semantics of PDL only needs reflexive transitive closure.
Let us be adventurous and introduce a more expressive temporal operator.
We extend the datatype
\isa{formula} by a new constructor%
\end{isamarkuptext}%
\isamarkuptrue%
\isamarkupfalse%
\ \ \ \ \ \ \ \ \ \ \ \ \ \ \ \ \ \ {\isacharbar}\ AF\ formula%
\begin{isamarkuptext}%
\noindent
which stands for ``\emph{A}lways in the \emph{F}uture'':
on all infinite paths, at some point the formula holds.
Formalizing the notion of an infinite path is easy
in HOL: it is simply a function from \isa{nat} to \isa{state}.%
\end{isamarkuptext}%
\isamarkuptrue%
\isacommand{constdefs}\isamarkupfalse%
\ Paths\ {\isacharcolon}{\isacharcolon}\ {\isachardoublequoteopen}state\ {\isasymRightarrow}\ {\isacharparenleft}nat\ {\isasymRightarrow}\ state{\isacharparenright}set{\isachardoublequoteclose}\isanewline
\ \ \ \ \ \ \ \ \ {\isachardoublequoteopen}Paths\ s\ {\isasymequiv}\ {\isacharbraceleft}p{\isachardot}\ s\ {\isacharequal}\ p\ {\isadigit{0}}\ {\isasymand}\ {\isacharparenleft}{\isasymforall}i{\isachardot}\ {\isacharparenleft}p\ i{\isacharcomma}\ p{\isacharparenleft}i{\isacharplus}{\isadigit{1}}{\isacharparenright}{\isacharparenright}\ {\isasymin}\ M{\isacharparenright}{\isacharbraceright}{\isachardoublequoteclose}%
\begin{isamarkuptext}%
\noindent
This definition allows a succinct statement of the semantics of \isa{AF}:
\footnote{Do not be misled: neither datatypes nor recursive functions can be
extended by new constructors or equations. This is just a trick of the
presentation (see \S\ref{sec:doc-prep-suppress}). In reality one has to define
a new datatype and a new function.}%
\end{isamarkuptext}%
\isamarkuptrue%
\isamarkupfalse%
\isamarkupfalse%
{\isachardoublequoteopen}s\ {\isasymTurnstile}\ AF\ f\ \ \ \ {\isacharequal}\ {\isacharparenleft}{\isasymforall}p\ {\isasymin}\ Paths\ s{\isachardot}\ {\isasymexists}i{\isachardot}\ p\ i\ {\isasymTurnstile}\ f{\isacharparenright}{\isachardoublequoteclose}%
\begin{isamarkuptext}%
\noindent
Model checking \isa{AF} involves a function which
is just complicated enough to warrant a separate definition:%
\end{isamarkuptext}%
\isamarkuptrue%
\isacommand{constdefs}\isamarkupfalse%
\ af\ {\isacharcolon}{\isacharcolon}\ {\isachardoublequoteopen}state\ set\ {\isasymRightarrow}\ state\ set\ {\isasymRightarrow}\ state\ set{\isachardoublequoteclose}\isanewline
\ \ \ \ \ \ \ \ \ {\isachardoublequoteopen}af\ A\ T\ {\isasymequiv}\ A\ {\isasymunion}\ {\isacharbraceleft}s{\isachardot}\ {\isasymforall}t{\isachardot}\ {\isacharparenleft}s{\isacharcomma}\ t{\isacharparenright}\ {\isasymin}\ M\ {\isasymlongrightarrow}\ t\ {\isasymin}\ T{\isacharbraceright}{\isachardoublequoteclose}%
\begin{isamarkuptext}%
\noindent
Now we define \isa{mc\ {\isacharparenleft}AF\ f{\isacharparenright}} as the least set \isa{T} that includes
\isa{mc\ f} and all states all of whose direct successors are in \isa{T}:%
\end{isamarkuptext}%
\isamarkuptrue%
\isamarkupfalse%
\isamarkupfalse%
{\isachardoublequoteopen}mc{\isacharparenleft}AF\ f{\isacharparenright}\ \ \ \ {\isacharequal}\ lfp{\isacharparenleft}af{\isacharparenleft}mc\ f{\isacharparenright}{\isacharparenright}{\isachardoublequoteclose}%
\begin{isamarkuptext}%
\noindent
Because \isa{af} is monotone in its second argument (and also its first, but
that is irrelevant), \isa{af\ A} has a least fixed point:%
\end{isamarkuptext}%
\isamarkuptrue%
\isacommand{lemma}\isamarkupfalse%
\ mono{\isacharunderscore}af{\isacharcolon}\ {\isachardoublequoteopen}mono{\isacharparenleft}af\ A{\isacharparenright}{\isachardoublequoteclose}\isanewline
%
\isadelimproof
%
\endisadelimproof
%
\isatagproof
\isacommand{apply}\isamarkupfalse%
{\isacharparenleft}simp\ add{\isacharcolon}\ mono{\isacharunderscore}def\ af{\isacharunderscore}def{\isacharparenright}\isanewline
\isacommand{apply}\isamarkupfalse%
\ blast\isanewline
\isacommand{done}\isamarkupfalse%
%
\endisatagproof
{\isafoldproof}%
%
\isadelimproof
%
\endisadelimproof
\isamarkupfalse%
%
\isadelimproof
%
\endisadelimproof
%
\isatagproof
\isamarkupfalse%
\isamarkupfalse%
%
\endisatagproof
{\isafoldproof}%
%
\isadelimproof
%
\endisadelimproof
\isamarkupfalse%
%
\isadelimproof
%
\endisadelimproof
%
\isatagproof
\isamarkupfalse%
\isamarkupfalse%
\isamarkupfalse%
\isamarkupfalse%
\isamarkupfalse%
\isamarkupfalse%
\isamarkupfalse%
\isamarkupfalse%
\isamarkupfalse%
\isamarkupfalse%
\isamarkupfalse%
\isamarkupfalse%
\isamarkupfalse%
\isamarkupfalse%
%
\endisatagproof
{\isafoldproof}%
%
\isadelimproof
%
\endisadelimproof
%
\begin{isamarkuptext}%
All we need to prove now is  \isa{mc\ {\isacharparenleft}AF\ f{\isacharparenright}\ {\isacharequal}\ {\isacharbraceleft}s{\isachardot}\ s\ {\isasymTurnstile}\ AF\ f{\isacharbraceright}}, which states
that \isa{mc} and \isa{{\isasymTurnstile}} agree for \isa{AF}\@.
This time we prove the two inclusions separately, starting
with the easy one:%
\end{isamarkuptext}%
\isamarkuptrue%
\isacommand{theorem}\isamarkupfalse%
\ AF{\isacharunderscore}lemma{\isadigit{1}}{\isacharcolon}\isanewline
\ \ {\isachardoublequoteopen}lfp{\isacharparenleft}af\ A{\isacharparenright}\ {\isasymsubseteq}\ {\isacharbraceleft}s{\isachardot}\ {\isasymforall}p\ {\isasymin}\ Paths\ s{\isachardot}\ {\isasymexists}i{\isachardot}\ p\ i\ {\isasymin}\ A{\isacharbraceright}{\isachardoublequoteclose}%
\isadelimproof
%
\endisadelimproof
%
\isatagproof
%
\begin{isamarkuptxt}%
\noindent
In contrast to the analogous proof for \isa{EF}, and just
for a change, we do not use fixed point induction.  Park-induction,
named after David Park, is weaker but sufficient for this proof:
\begin{center}
\isa{f\ S\ {\isasymsubseteq}\ S\ {\isasymLongrightarrow}\ lfp\ f\ {\isasymsubseteq}\ S} \hfill (\isa{lfp{\isacharunderscore}lowerbound})
\end{center}
The instance of the premise \isa{f\ S\ {\isasymsubseteq}\ S} is proved pointwise,
a decision that \isa{auto} takes for us:%
\end{isamarkuptxt}%
\isamarkuptrue%
\isacommand{apply}\isamarkupfalse%
{\isacharparenleft}rule\ lfp{\isacharunderscore}lowerbound{\isacharparenright}\isanewline
\isacommand{apply}\isamarkupfalse%
{\isacharparenleft}auto\ simp\ add{\isacharcolon}\ af{\isacharunderscore}def\ Paths{\isacharunderscore}def{\isacharparenright}%
\begin{isamarkuptxt}%
\begin{isabelle}%
\ {\isadigit{1}}{\isachardot}\ {\isasymAnd}p{\isachardot}\ {\isasymlbrakk}{\isasymforall}t{\isachardot}\ {\isacharparenleft}p\ {\isadigit{0}}{\isacharcomma}\ t{\isacharparenright}\ {\isasymin}\ M\ {\isasymlongrightarrow}\isanewline
\isaindent{\ {\isadigit{1}}{\isachardot}\ {\isasymAnd}p{\isachardot}\ {\isasymlbrakk}{\isasymforall}t{\isachardot}\ }{\isacharparenleft}{\isasymforall}p{\isachardot}\ t\ {\isacharequal}\ p\ {\isadigit{0}}\ {\isasymand}\ {\isacharparenleft}{\isasymforall}i{\isachardot}\ {\isacharparenleft}p\ i{\isacharcomma}\ p\ {\isacharparenleft}Suc\ i{\isacharparenright}{\isacharparenright}\ {\isasymin}\ M{\isacharparenright}\ {\isasymlongrightarrow}\isanewline
\isaindent{\ {\isadigit{1}}{\isachardot}\ {\isasymAnd}p{\isachardot}\ {\isasymlbrakk}{\isasymforall}t{\isachardot}\ {\isacharparenleft}{\isasymforall}p{\isachardot}\ }{\isacharparenleft}{\isasymexists}i{\isachardot}\ p\ i\ {\isasymin}\ A{\isacharparenright}{\isacharparenright}{\isacharsemicolon}\isanewline
\isaindent{\ {\isadigit{1}}{\isachardot}\ {\isasymAnd}p{\isachardot}\ \ }{\isasymforall}i{\isachardot}\ {\isacharparenleft}p\ i{\isacharcomma}\ p\ {\isacharparenleft}Suc\ i{\isacharparenright}{\isacharparenright}\ {\isasymin}\ M{\isasymrbrakk}\isanewline
\isaindent{\ {\isadigit{1}}{\isachardot}\ {\isasymAnd}p{\isachardot}\ }{\isasymLongrightarrow}\ {\isasymexists}i{\isachardot}\ p\ i\ {\isasymin}\ A%
\end{isabelle}
In this remaining case, we set \isa{t} to \isa{p\ {\isadigit{1}}}.
The rest is automatic, which is surprising because it involves
finding the instantiation \isa{{\isasymlambda}i{\isachardot}\ p\ {\isacharparenleft}i\ {\isacharplus}\ {\isadigit{1}}{\isacharparenright}}
for \isa{{\isasymforall}p}.%
\end{isamarkuptxt}%
\isamarkuptrue%
\isacommand{apply}\isamarkupfalse%
{\isacharparenleft}erule{\isacharunderscore}tac\ x\ {\isacharequal}\ {\isachardoublequoteopen}p\ {\isadigit{1}}{\isachardoublequoteclose}\ \isakeyword{in}\ allE{\isacharparenright}\isanewline
\isacommand{apply}\isamarkupfalse%
{\isacharparenleft}auto{\isacharparenright}\isanewline
\isacommand{done}\isamarkupfalse%
%
\endisatagproof
{\isafoldproof}%
%
\isadelimproof
%
\endisadelimproof
%
\begin{isamarkuptext}%
The opposite inclusion is proved by contradiction: if some state
\isa{s} is not in \isa{lfp\ {\isacharparenleft}af\ A{\isacharparenright}}, then we can construct an
infinite \isa{A}-avoiding path starting from~\isa{s}. The reason is
that by unfolding \isa{lfp} we find that if \isa{s} is not in
\isa{lfp\ {\isacharparenleft}af\ A{\isacharparenright}}, then \isa{s} is not in \isa{A} and there is a
direct successor of \isa{s} that is again not in \mbox{\isa{lfp\ {\isacharparenleft}af\ A{\isacharparenright}}}. Iterating this argument yields the promised infinite
\isa{A}-avoiding path. Let us formalize this sketch.

The one-step argument in the sketch above
is proved by a variant of contraposition:%
\end{isamarkuptext}%
\isamarkuptrue%
\isacommand{lemma}\isamarkupfalse%
\ not{\isacharunderscore}in{\isacharunderscore}lfp{\isacharunderscore}afD{\isacharcolon}\isanewline
\ {\isachardoublequoteopen}s\ {\isasymnotin}\ lfp{\isacharparenleft}af\ A{\isacharparenright}\ {\isasymLongrightarrow}\ s\ {\isasymnotin}\ A\ {\isasymand}\ {\isacharparenleft}{\isasymexists}\ t{\isachardot}\ {\isacharparenleft}s{\isacharcomma}t{\isacharparenright}\ {\isasymin}\ M\ {\isasymand}\ t\ {\isasymnotin}\ lfp{\isacharparenleft}af\ A{\isacharparenright}{\isacharparenright}{\isachardoublequoteclose}\isanewline
%
\isadelimproof
%
\endisadelimproof
%
\isatagproof
\isacommand{apply}\isamarkupfalse%
{\isacharparenleft}erule\ contrapos{\isacharunderscore}np{\isacharparenright}\isanewline
\isacommand{apply}\isamarkupfalse%
{\isacharparenleft}subst\ lfp{\isacharunderscore}unfold{\isacharbrackleft}OF\ mono{\isacharunderscore}af{\isacharbrackright}{\isacharparenright}\isanewline
\isacommand{apply}\isamarkupfalse%
{\isacharparenleft}simp\ add{\isacharcolon}\ af{\isacharunderscore}def{\isacharparenright}\isanewline
\isacommand{done}\isamarkupfalse%
%
\endisatagproof
{\isafoldproof}%
%
\isadelimproof
%
\endisadelimproof
%
\begin{isamarkuptext}%
\noindent
We assume the negation of the conclusion and prove \isa{s\ {\isasymin}\ lfp\ {\isacharparenleft}af\ A{\isacharparenright}}.
Unfolding \isa{lfp} once and
simplifying with the definition of \isa{af} finishes the proof.

Now we iterate this process. The following construction of the desired
path is parameterized by a predicate \isa{Q} that should hold along the path:%
\end{isamarkuptext}%
\isamarkuptrue%
\isacommand{consts}\isamarkupfalse%
\ path\ {\isacharcolon}{\isacharcolon}\ {\isachardoublequoteopen}state\ {\isasymRightarrow}\ {\isacharparenleft}state\ {\isasymRightarrow}\ bool{\isacharparenright}\ {\isasymRightarrow}\ {\isacharparenleft}nat\ {\isasymRightarrow}\ state{\isacharparenright}{\isachardoublequoteclose}\isanewline
\isacommand{primrec}\isamarkupfalse%
\isanewline
{\isachardoublequoteopen}path\ s\ Q\ {\isadigit{0}}\ {\isacharequal}\ s{\isachardoublequoteclose}\isanewline
{\isachardoublequoteopen}path\ s\ Q\ {\isacharparenleft}Suc\ n{\isacharparenright}\ {\isacharequal}\ {\isacharparenleft}SOME\ t{\isachardot}\ {\isacharparenleft}path\ s\ Q\ n{\isacharcomma}t{\isacharparenright}\ {\isasymin}\ M\ {\isasymand}\ Q\ t{\isacharparenright}{\isachardoublequoteclose}%
\begin{isamarkuptext}%
\noindent
Element \isa{n\ {\isacharplus}\ {\isadigit{1}}} on this path is some arbitrary successor
\isa{t} of element \isa{n} such that \isa{Q\ t} holds.  Remember that \isa{SOME\ t{\isachardot}\ R\ t}
is some arbitrary but fixed \isa{t} such that \isa{R\ t} holds (see \S\ref{sec:SOME}). Of
course, such a \isa{t} need not exist, but that is of no
concern to us since we will only use \isa{path} when a
suitable \isa{t} does exist.

Let us show that if each state \isa{s} that satisfies \isa{Q}
has a successor that again satisfies \isa{Q}, then there exists an infinite \isa{Q}-path:%
\end{isamarkuptext}%
\isamarkuptrue%
\isacommand{lemma}\isamarkupfalse%
\ infinity{\isacharunderscore}lemma{\isacharcolon}\isanewline
\ \ {\isachardoublequoteopen}{\isasymlbrakk}\ Q\ s{\isacharsemicolon}\ {\isasymforall}s{\isachardot}\ Q\ s\ {\isasymlongrightarrow}\ {\isacharparenleft}{\isasymexists}\ t{\isachardot}\ {\isacharparenleft}s{\isacharcomma}t{\isacharparenright}\ {\isasymin}\ M\ {\isasymand}\ Q\ t{\isacharparenright}\ {\isasymrbrakk}\ {\isasymLongrightarrow}\isanewline
\ \ \ {\isasymexists}p{\isasymin}Paths\ s{\isachardot}\ {\isasymforall}i{\isachardot}\ Q{\isacharparenleft}p\ i{\isacharparenright}{\isachardoublequoteclose}%
\isadelimproof
%
\endisadelimproof
%
\isatagproof
%
\begin{isamarkuptxt}%
\noindent
First we rephrase the conclusion slightly because we need to prove simultaneously
both the path property and the fact that \isa{Q} holds:%
\end{isamarkuptxt}%
\isamarkuptrue%
\isacommand{apply}\isamarkupfalse%
{\isacharparenleft}subgoal{\isacharunderscore}tac\isanewline
\ \ {\isachardoublequoteopen}{\isasymexists}p{\isachardot}\ s\ {\isacharequal}\ p\ {\isadigit{0}}\ {\isasymand}\ {\isacharparenleft}{\isasymforall}i{\isacharcolon}{\isacharcolon}nat{\isachardot}\ {\isacharparenleft}p\ i{\isacharcomma}\ p{\isacharparenleft}i{\isacharplus}{\isadigit{1}}{\isacharparenright}{\isacharparenright}\ {\isasymin}\ M\ {\isasymand}\ Q{\isacharparenleft}p\ i{\isacharparenright}{\isacharparenright}{\isachardoublequoteclose}{\isacharparenright}%
\begin{isamarkuptxt}%
\noindent
From this proposition the original goal follows easily:%
\end{isamarkuptxt}%
\isamarkuptrue%
\ \isacommand{apply}\isamarkupfalse%
{\isacharparenleft}simp\ add{\isacharcolon}\ Paths{\isacharunderscore}def{\isacharcomma}\ blast{\isacharparenright}%
\begin{isamarkuptxt}%
\noindent
The new subgoal is proved by providing the witness \isa{path\ s\ Q} for \isa{p}:%
\end{isamarkuptxt}%
\isamarkuptrue%
\isacommand{apply}\isamarkupfalse%
{\isacharparenleft}rule{\isacharunderscore}tac\ x\ {\isacharequal}\ {\isachardoublequoteopen}path\ s\ Q{\isachardoublequoteclose}\ \isakeyword{in}\ exI{\isacharparenright}\isanewline
\isacommand{apply}\isamarkupfalse%
{\isacharparenleft}clarsimp{\isacharparenright}%
\begin{isamarkuptxt}%
\noindent
After simplification and clarification, the subgoal has the following form:
\begin{isabelle}%
\ {\isadigit{1}}{\isachardot}\ {\isasymAnd}i{\isachardot}\ {\isasymlbrakk}Q\ s{\isacharsemicolon}\ {\isasymforall}s{\isachardot}\ Q\ s\ {\isasymlongrightarrow}\ {\isacharparenleft}{\isasymexists}t{\isachardot}\ {\isacharparenleft}s{\isacharcomma}\ t{\isacharparenright}\ {\isasymin}\ M\ {\isasymand}\ Q\ t{\isacharparenright}{\isasymrbrakk}\isanewline
\isaindent{\ {\isadigit{1}}{\isachardot}\ {\isasymAnd}i{\isachardot}\ }{\isasymLongrightarrow}\ {\isacharparenleft}path\ s\ Q\ i{\isacharcomma}\ SOME\ t{\isachardot}\ {\isacharparenleft}path\ s\ Q\ i{\isacharcomma}\ t{\isacharparenright}\ {\isasymin}\ M\ {\isasymand}\ Q\ t{\isacharparenright}\ {\isasymin}\ M\ {\isasymand}\isanewline
\isaindent{\ {\isadigit{1}}{\isachardot}\ {\isasymAnd}i{\isachardot}\ {\isasymLongrightarrow}\ }Q\ {\isacharparenleft}path\ s\ Q\ i{\isacharparenright}%
\end{isabelle}
It invites a proof by induction on \isa{i}:%
\end{isamarkuptxt}%
\isamarkuptrue%
\isacommand{apply}\isamarkupfalse%
{\isacharparenleft}induct{\isacharunderscore}tac\ i{\isacharparenright}\isanewline
\ \isacommand{apply}\isamarkupfalse%
{\isacharparenleft}simp{\isacharparenright}%
\begin{isamarkuptxt}%
\noindent
After simplification, the base case boils down to
\begin{isabelle}%
\ {\isadigit{1}}{\isachardot}\ {\isasymlbrakk}Q\ s{\isacharsemicolon}\ {\isasymforall}s{\isachardot}\ Q\ s\ {\isasymlongrightarrow}\ {\isacharparenleft}{\isasymexists}t{\isachardot}\ {\isacharparenleft}s{\isacharcomma}\ t{\isacharparenright}\ {\isasymin}\ M\ {\isasymand}\ Q\ t{\isacharparenright}{\isasymrbrakk}\isanewline
\isaindent{\ {\isadigit{1}}{\isachardot}\ }{\isasymLongrightarrow}\ {\isacharparenleft}s{\isacharcomma}\ SOME\ t{\isachardot}\ {\isacharparenleft}s{\isacharcomma}\ t{\isacharparenright}\ {\isasymin}\ M\ {\isasymand}\ Q\ t{\isacharparenright}\ {\isasymin}\ M%
\end{isabelle}
The conclusion looks exceedingly trivial: after all, \isa{t} is chosen such that \isa{{\isacharparenleft}s{\isacharcomma}\ t{\isacharparenright}\ {\isasymin}\ M}
holds. However, we first have to show that such a \isa{t} actually exists! This reasoning
is embodied in the theorem \isa{someI{\isadigit{2}}{\isacharunderscore}ex}:
\begin{isabelle}%
\ \ \ \ \ {\isasymlbrakk}{\isasymexists}a{\isachardot}\ {\isacharquery}P\ a{\isacharsemicolon}\ {\isasymAnd}x{\isachardot}\ {\isacharquery}P\ x\ {\isasymLongrightarrow}\ {\isacharquery}Q\ x{\isasymrbrakk}\ {\isasymLongrightarrow}\ {\isacharquery}Q\ {\isacharparenleft}SOME\ x{\isachardot}\ {\isacharquery}P\ x{\isacharparenright}%
\end{isabelle}
When we apply this theorem as an introduction rule, \isa{{\isacharquery}P\ x} becomes
\isa{{\isacharparenleft}s{\isacharcomma}\ x{\isacharparenright}\ {\isasymin}\ M\ {\isasymand}\ Q\ x} and \isa{{\isacharquery}Q\ x} becomes \isa{{\isacharparenleft}s{\isacharcomma}\ x{\isacharparenright}\ {\isasymin}\ M} and we have to prove
two subgoals: \isa{{\isasymexists}a{\isachardot}\ {\isacharparenleft}s{\isacharcomma}\ a{\isacharparenright}\ {\isasymin}\ M\ {\isasymand}\ Q\ a}, which follows from the assumptions, and
\isa{{\isacharparenleft}s{\isacharcomma}\ x{\isacharparenright}\ {\isasymin}\ M\ {\isasymand}\ Q\ x\ {\isasymLongrightarrow}\ {\isacharparenleft}s{\isacharcomma}\ x{\isacharparenright}\ {\isasymin}\ M}, which is trivial. Thus it is not surprising that
\isa{fast} can prove the base case quickly:%
\end{isamarkuptxt}%
\isamarkuptrue%
\ \isacommand{apply}\isamarkupfalse%
{\isacharparenleft}fast\ intro{\isacharcolon}\ someI{\isadigit{2}}{\isacharunderscore}ex{\isacharparenright}%
\begin{isamarkuptxt}%
\noindent
What is worth noting here is that we have used \methdx{fast} rather than
\isa{blast}.  The reason is that \isa{blast} would fail because it cannot
cope with \isa{someI{\isadigit{2}}{\isacharunderscore}ex}: unifying its conclusion with the current
subgoal is non-trivial because of the nested schematic variables. For
efficiency reasons \isa{blast} does not even attempt such unifications.
Although \isa{fast} can in principle cope with complicated unification
problems, in practice the number of unifiers arising is often prohibitive and
the offending rule may need to be applied explicitly rather than
automatically. This is what happens in the step case.

The induction step is similar, but more involved, because now we face nested
occurrences of \isa{SOME}. As a result, \isa{fast} is no longer able to
solve the subgoal and we apply \isa{someI{\isadigit{2}}{\isacharunderscore}ex} by hand.  We merely
show the proof commands but do not describe the details:%
\end{isamarkuptxt}%
\isamarkuptrue%
\isacommand{apply}\isamarkupfalse%
{\isacharparenleft}simp{\isacharparenright}\isanewline
\isacommand{apply}\isamarkupfalse%
{\isacharparenleft}rule\ someI{\isadigit{2}}{\isacharunderscore}ex{\isacharparenright}\isanewline
\ \isacommand{apply}\isamarkupfalse%
{\isacharparenleft}blast{\isacharparenright}\isanewline
\isacommand{apply}\isamarkupfalse%
{\isacharparenleft}rule\ someI{\isadigit{2}}{\isacharunderscore}ex{\isacharparenright}\isanewline
\ \isacommand{apply}\isamarkupfalse%
{\isacharparenleft}blast{\isacharparenright}\isanewline
\isacommand{apply}\isamarkupfalse%
{\isacharparenleft}blast{\isacharparenright}\isanewline
\isacommand{done}\isamarkupfalse%
%
\endisatagproof
{\isafoldproof}%
%
\isadelimproof
%
\endisadelimproof
%
\begin{isamarkuptext}%
Function \isa{path} has fulfilled its purpose now and can be forgotten.
It was merely defined to provide the witness in the proof of the
\isa{infinity{\isacharunderscore}lemma}. Aficionados of minimal proofs might like to know
that we could have given the witness without having to define a new function:
the term
\begin{isabelle}%
\ \ \ \ \ nat{\isacharunderscore}rec\ s\ {\isacharparenleft}{\isasymlambda}n\ t{\isachardot}\ SOME\ u{\isachardot}\ {\isacharparenleft}t{\isacharcomma}\ u{\isacharparenright}\ {\isasymin}\ M\ {\isasymand}\ Q\ u{\isacharparenright}%
\end{isabelle}
is extensionally equal to \isa{path\ s\ Q},
where \isa{nat{\isacharunderscore}rec} is the predefined primitive recursor on \isa{nat}.%
\end{isamarkuptext}%
\isamarkuptrue%
\isamarkupfalse%
%
\isadelimproof
%
\endisadelimproof
%
\isatagproof
\isamarkupfalse%
\isamarkupfalse%
\isamarkupfalse%
\isamarkupfalse%
\isamarkupfalse%
\isamarkupfalse%
\isamarkupfalse%
\isamarkupfalse%
\isamarkupfalse%
\isamarkupfalse%
\isamarkupfalse%
\isamarkupfalse%
\isamarkupfalse%
\isamarkupfalse%
\isamarkupfalse%
%
\endisatagproof
{\isafoldproof}%
%
\isadelimproof
%
\endisadelimproof
%
\begin{isamarkuptext}%
At last we can prove the opposite direction of \isa{AF{\isacharunderscore}lemma{\isadigit{1}}}:%
\end{isamarkuptext}%
\isamarkuptrue%
\isacommand{theorem}\isamarkupfalse%
\ AF{\isacharunderscore}lemma{\isadigit{2}}{\isacharcolon}\ {\isachardoublequoteopen}{\isacharbraceleft}s{\isachardot}\ {\isasymforall}p\ {\isasymin}\ Paths\ s{\isachardot}\ {\isasymexists}i{\isachardot}\ p\ i\ {\isasymin}\ A{\isacharbraceright}\ {\isasymsubseteq}\ lfp{\isacharparenleft}af\ A{\isacharparenright}{\isachardoublequoteclose}%
\isadelimproof
%
\endisadelimproof
%
\isatagproof
%
\begin{isamarkuptxt}%
\noindent
The proof is again pointwise and then by contraposition:%
\end{isamarkuptxt}%
\isamarkuptrue%
\isacommand{apply}\isamarkupfalse%
{\isacharparenleft}rule\ subsetI{\isacharparenright}\isanewline
\isacommand{apply}\isamarkupfalse%
{\isacharparenleft}erule\ contrapos{\isacharunderscore}pp{\isacharparenright}\isanewline
\isacommand{apply}\isamarkupfalse%
\ simp%
\begin{isamarkuptxt}%
\begin{isabelle}%
\ {\isadigit{1}}{\isachardot}\ {\isasymAnd}x{\isachardot}\ x\ {\isasymnotin}\ lfp\ {\isacharparenleft}af\ A{\isacharparenright}\ {\isasymLongrightarrow}\ {\isasymexists}p{\isasymin}Paths\ x{\isachardot}\ {\isasymforall}i{\isachardot}\ p\ i\ {\isasymnotin}\ A%
\end{isabelle}
Applying the \isa{infinity{\isacharunderscore}lemma} as a destruction rule leaves two subgoals, the second
premise of \isa{infinity{\isacharunderscore}lemma} and the original subgoal:%
\end{isamarkuptxt}%
\isamarkuptrue%
\isacommand{apply}\isamarkupfalse%
{\isacharparenleft}drule\ infinity{\isacharunderscore}lemma{\isacharparenright}%
\begin{isamarkuptxt}%
\begin{isabelle}%
\ {\isadigit{1}}{\isachardot}\ {\isasymAnd}x{\isachardot}\ {\isasymforall}s{\isachardot}\ s\ {\isasymnotin}\ lfp\ {\isacharparenleft}af\ A{\isacharparenright}\ {\isasymlongrightarrow}\ {\isacharparenleft}{\isasymexists}t{\isachardot}\ {\isacharparenleft}s{\isacharcomma}\ t{\isacharparenright}\ {\isasymin}\ M\ {\isasymand}\ t\ {\isasymnotin}\ lfp\ {\isacharparenleft}af\ A{\isacharparenright}{\isacharparenright}\isanewline
\ {\isadigit{2}}{\isachardot}\ {\isasymAnd}x{\isachardot}\ {\isasymexists}p{\isasymin}Paths\ x{\isachardot}\ {\isasymforall}i{\isachardot}\ p\ i\ {\isasymnotin}\ lfp\ {\isacharparenleft}af\ A{\isacharparenright}\ {\isasymLongrightarrow}\isanewline
\isaindent{\ {\isadigit{2}}{\isachardot}\ {\isasymAnd}x{\isachardot}\ }{\isasymexists}p{\isasymin}Paths\ x{\isachardot}\ {\isasymforall}i{\isachardot}\ p\ i\ {\isasymnotin}\ A%
\end{isabelle}
Both are solved automatically:%
\end{isamarkuptxt}%
\isamarkuptrue%
\ \isacommand{apply}\isamarkupfalse%
{\isacharparenleft}auto\ dest{\isacharcolon}\ not{\isacharunderscore}in{\isacharunderscore}lfp{\isacharunderscore}afD{\isacharparenright}\isanewline
\isacommand{done}\isamarkupfalse%
%
\endisatagproof
{\isafoldproof}%
%
\isadelimproof
%
\endisadelimproof
%
\begin{isamarkuptext}%
If you find these proofs too complicated, we recommend that you read
\S\ref{sec:CTL-revisited}, where we show how inductive definitions lead to
simpler arguments.

The main theorem is proved as for PDL, except that we also derive the
necessary equality \isa{lfp{\isacharparenleft}af\ A{\isacharparenright}\ {\isacharequal}\ {\isachardot}{\isachardot}{\isachardot}} by combining
\isa{AF{\isacharunderscore}lemma{\isadigit{1}}} and \isa{AF{\isacharunderscore}lemma{\isadigit{2}}} on the spot:%
\end{isamarkuptext}%
\isamarkuptrue%
\isacommand{theorem}\isamarkupfalse%
\ {\isachardoublequoteopen}mc\ f\ {\isacharequal}\ {\isacharbraceleft}s{\isachardot}\ s\ {\isasymTurnstile}\ f{\isacharbraceright}{\isachardoublequoteclose}\isanewline
%
\isadelimproof
%
\endisadelimproof
%
\isatagproof
\isacommand{apply}\isamarkupfalse%
{\isacharparenleft}induct{\isacharunderscore}tac\ f{\isacharparenright}\isanewline
\isacommand{apply}\isamarkupfalse%
{\isacharparenleft}auto\ simp\ add{\isacharcolon}\ EF{\isacharunderscore}lemma\ equalityI{\isacharbrackleft}OF\ AF{\isacharunderscore}lemma{\isadigit{1}}\ AF{\isacharunderscore}lemma{\isadigit{2}}{\isacharbrackright}{\isacharparenright}\isanewline
\isacommand{done}\isamarkupfalse%
%
\endisatagproof
{\isafoldproof}%
%
\isadelimproof
%
\endisadelimproof
%
\begin{isamarkuptext}%
The language defined above is not quite CTL\@. The latter also includes an
until-operator \isa{EU\ f\ g} with semantics ``there \emph{E}xists a path
where \isa{f} is true \emph{U}ntil \isa{g} becomes true''.  We need
an auxiliary function:%
\end{isamarkuptext}%
\isamarkuptrue%
\isacommand{consts}\isamarkupfalse%
\ until{\isacharcolon}{\isacharcolon}\ {\isachardoublequoteopen}state\ set\ {\isasymRightarrow}\ state\ set\ {\isasymRightarrow}\ state\ {\isasymRightarrow}\ state\ list\ {\isasymRightarrow}\ bool{\isachardoublequoteclose}\isanewline
\isacommand{primrec}\isamarkupfalse%
\isanewline
{\isachardoublequoteopen}until\ A\ B\ s\ {\isacharbrackleft}{\isacharbrackright}\ \ \ \ {\isacharequal}\ {\isacharparenleft}s\ {\isasymin}\ B{\isacharparenright}{\isachardoublequoteclose}\isanewline
{\isachardoublequoteopen}until\ A\ B\ s\ {\isacharparenleft}t{\isacharhash}p{\isacharparenright}\ {\isacharequal}\ {\isacharparenleft}s\ {\isasymin}\ A\ {\isasymand}\ {\isacharparenleft}s{\isacharcomma}t{\isacharparenright}\ {\isasymin}\ M\ {\isasymand}\ until\ A\ B\ t\ p{\isacharparenright}{\isachardoublequoteclose}\isamarkupfalse%
%
\begin{isamarkuptext}%
\noindent
Expressing the semantics of \isa{EU} is now straightforward:
\begin{isabelle}%
\ \ \ \ \ s\ {\isasymTurnstile}\ EU\ f\ g\ {\isacharequal}\ {\isacharparenleft}{\isasymexists}p{\isachardot}\ until\ {\isacharbraceleft}t{\isachardot}\ t\ {\isasymTurnstile}\ f{\isacharbraceright}\ {\isacharbraceleft}t{\isachardot}\ t\ {\isasymTurnstile}\ g{\isacharbraceright}\ s\ p{\isacharparenright}%
\end{isabelle}
Note that \isa{EU} is not definable in terms of the other operators!

Model checking \isa{EU} is again a least fixed point construction:
\begin{isabelle}%
\ \ \ \ \ mc{\isacharparenleft}EU\ f\ g{\isacharparenright}\ {\isacharequal}\ lfp{\isacharparenleft}{\isasymlambda}T{\isachardot}\ mc\ g\ {\isasymunion}\ mc\ f\ {\isasyminter}\ {\isacharparenleft}M{\isasyminverse}\ {\isacharbackquote}{\isacharbackquote}\ T{\isacharparenright}{\isacharparenright}%
\end{isabelle}

\begin{exercise}
Extend the datatype of formulae by the above until operator
and prove the equivalence between semantics and model checking, i.e.\ that
\begin{isabelle}%
\ \ \ \ \ mc\ {\isacharparenleft}EU\ f\ g{\isacharparenright}\ {\isacharequal}\ {\isacharbraceleft}s{\isachardot}\ s\ {\isasymTurnstile}\ EU\ f\ g{\isacharbraceright}%
\end{isabelle}
%For readability you may want to annotate {term EU} with its customary syntax
%{text[display]"| EU formula formula    E[_ U _]"}
%which enables you to read and write {text"E[f U g]"} instead of {term"EU f g"}.
\end{exercise}
For more CTL exercises see, for example, Huth and Ryan \cite{Huth-Ryan-book}.%
\end{isamarkuptext}%
\isamarkuptrue%
\isamarkupfalse%
\isamarkupfalse%
%
\isadelimproof
%
\endisadelimproof
%
\isatagproof
\isamarkupfalse%
\isamarkupfalse%
\isamarkupfalse%
\isamarkupfalse%
\isamarkupfalse%
\isamarkupfalse%
\isamarkupfalse%
%
\endisatagproof
{\isafoldproof}%
%
\isadelimproof
%
\endisadelimproof
\isamarkupfalse%
%
\isadelimproof
%
\endisadelimproof
%
\isatagproof
\isamarkupfalse%
\isamarkupfalse%
\isamarkupfalse%
%
\endisatagproof
{\isafoldproof}%
%
\isadelimproof
%
\endisadelimproof
\isamarkupfalse%
%
\isadelimproof
%
\endisadelimproof
%
\isatagproof
\isamarkupfalse%
\isamarkupfalse%
\isamarkupfalse%
\isamarkupfalse%
\isamarkupfalse%
\isamarkupfalse%
\isamarkupfalse%
\isamarkupfalse%
\isamarkupfalse%
\isamarkupfalse%
\isamarkupfalse%
%
\endisatagproof
{\isafoldproof}%
%
\isadelimproof
%
\endisadelimproof
%
\begin{isamarkuptext}%
Let us close this section with a few words about the executability of
our model checkers.  It is clear that if all sets are finite, they can be
represented as lists and the usual set operations are easily
implemented. Only \isa{lfp} requires a little thought.  Fortunately, theory
\isa{While{\isacharunderscore}Combinator} in the Library~\cite{HOL-Library} provides a
theorem stating that in the case of finite sets and a monotone
function~\isa{F}, the value of \mbox{\isa{lfp\ F}} can be computed by
iterated application of \isa{F} to~\isa{{\isacharbraceleft}{\isacharbraceright}} until a fixed point is
reached. It is actually possible to generate executable functional programs
from HOL definitions, but that is beyond the scope of the tutorial.%
\index{CTL|)}%
\end{isamarkuptext}%
\isamarkuptrue%
%
\isadelimtheory
%
\endisadelimtheory
%
\isatagtheory
\isamarkupfalse%
%
\endisatagtheory
{\isafoldtheory}%
%
\isadelimtheory
%
\endisadelimtheory
\end{isabellebody}%
%%% Local Variables:
%%% mode: latex
%%% TeX-master: "root"
%%% End:

\index{model checking example|)}
  %these constitute ONE chapter
\chapter{Inductively Defined Sets}
\index{inductive definition|(}
\index{*inductive|(}

This chapter is dedicated to the most important definition principle after
recursive functions and datatypes: inductively defined sets.

We start with a simple example \ldots . A slightly more complicated example, the
reflexive transitive closure, is the subject of {\S}\ref{sec:rtc}. In particular,
some standard induction heuristics are discussed. To demonstrate the
versatility of inductive definitions, {\S}\ref{sec:CFG} presents a case study
from the realm of context-free grammars. The chapter closes with a discussion
of advanced forms of inductive definitions.

%
\begin{isabellebody}%
\def\isabellecontext{Star}%
%
\isadelimtheory
%
\endisadelimtheory
%
\isatagtheory
%
\endisatagtheory
{\isafoldtheory}%
%
\isadelimtheory
%
\endisadelimtheory
%
\isamarkupsection{The Reflexive Transitive Closure%
}
\isamarkuptrue%
%
\begin{isamarkuptext}%
\label{sec:rtc}
\index{reflexive transitive closure!defining inductively|(}%
An inductive definition may accept parameters, so it can express 
functions that yield sets.
Relations too can be defined inductively, since they are just sets of pairs.
A perfect example is the function that maps a relation to its
reflexive transitive closure.  This concept was already
introduced in \S\ref{sec:Relations}, where the operator \isa{\isaliteral{5C3C5E7375703E}{}\isactrlsup {\isaliteral{2A}{\isacharasterisk}}} was
defined as a least fixed point because inductive definitions were not yet
available. But now they are:%
\end{isamarkuptext}%
\isamarkuptrue%
\isacommand{inductive{\isaliteral{5F}{\isacharunderscore}}set}\isamarkupfalse%
\isanewline
\ \ rtc\ {\isaliteral{3A}{\isacharcolon}}{\isaliteral{3A}{\isacharcolon}}\ {\isaliteral{22}{\isachardoublequoteopen}}{\isaliteral{28}{\isacharparenleft}}{\isaliteral{27}{\isacharprime}}a\ {\isaliteral{5C3C74696D65733E}{\isasymtimes}}\ {\isaliteral{27}{\isacharprime}}a{\isaliteral{29}{\isacharparenright}}set\ {\isaliteral{5C3C52696768746172726F773E}{\isasymRightarrow}}\ {\isaliteral{28}{\isacharparenleft}}{\isaliteral{27}{\isacharprime}}a\ {\isaliteral{5C3C74696D65733E}{\isasymtimes}}\ {\isaliteral{27}{\isacharprime}}a{\isaliteral{29}{\isacharparenright}}set{\isaliteral{22}{\isachardoublequoteclose}}\ \ \ {\isaliteral{28}{\isacharparenleft}}{\isaliteral{22}{\isachardoublequoteopen}}{\isaliteral{5F}{\isacharunderscore}}{\isaliteral{2A}{\isacharasterisk}}{\isaliteral{22}{\isachardoublequoteclose}}\ {\isaliteral{5B}{\isacharbrackleft}}{\isadigit{1}}{\isadigit{0}}{\isadigit{0}}{\isadigit{0}}{\isaliteral{5D}{\isacharbrackright}}\ {\isadigit{9}}{\isadigit{9}}{\isadigit{9}}{\isaliteral{29}{\isacharparenright}}\isanewline
\ \ \isakeyword{for}\ r\ {\isaliteral{3A}{\isacharcolon}}{\isaliteral{3A}{\isacharcolon}}\ {\isaliteral{22}{\isachardoublequoteopen}}{\isaliteral{28}{\isacharparenleft}}{\isaliteral{27}{\isacharprime}}a\ {\isaliteral{5C3C74696D65733E}{\isasymtimes}}\ {\isaliteral{27}{\isacharprime}}a{\isaliteral{29}{\isacharparenright}}set{\isaliteral{22}{\isachardoublequoteclose}}\isanewline
\isakeyword{where}\isanewline
\ \ rtc{\isaliteral{5F}{\isacharunderscore}}refl{\isaliteral{5B}{\isacharbrackleft}}iff{\isaliteral{5D}{\isacharbrackright}}{\isaliteral{3A}{\isacharcolon}}\ \ {\isaliteral{22}{\isachardoublequoteopen}}{\isaliteral{28}{\isacharparenleft}}x{\isaliteral{2C}{\isacharcomma}}x{\isaliteral{29}{\isacharparenright}}\ {\isaliteral{5C3C696E3E}{\isasymin}}\ r{\isaliteral{2A}{\isacharasterisk}}{\isaliteral{22}{\isachardoublequoteclose}}\isanewline
{\isaliteral{7C}{\isacharbar}}\ rtc{\isaliteral{5F}{\isacharunderscore}}step{\isaliteral{3A}{\isacharcolon}}\ \ \ \ \ \ \ {\isaliteral{22}{\isachardoublequoteopen}}{\isaliteral{5C3C6C6272616B6B3E}{\isasymlbrakk}}\ {\isaliteral{28}{\isacharparenleft}}x{\isaliteral{2C}{\isacharcomma}}y{\isaliteral{29}{\isacharparenright}}\ {\isaliteral{5C3C696E3E}{\isasymin}}\ r{\isaliteral{3B}{\isacharsemicolon}}\ {\isaliteral{28}{\isacharparenleft}}y{\isaliteral{2C}{\isacharcomma}}z{\isaliteral{29}{\isacharparenright}}\ {\isaliteral{5C3C696E3E}{\isasymin}}\ r{\isaliteral{2A}{\isacharasterisk}}\ {\isaliteral{5C3C726272616B6B3E}{\isasymrbrakk}}\ {\isaliteral{5C3C4C6F6E6772696768746172726F773E}{\isasymLongrightarrow}}\ {\isaliteral{28}{\isacharparenleft}}x{\isaliteral{2C}{\isacharcomma}}z{\isaliteral{29}{\isacharparenright}}\ {\isaliteral{5C3C696E3E}{\isasymin}}\ r{\isaliteral{2A}{\isacharasterisk}}{\isaliteral{22}{\isachardoublequoteclose}}%
\begin{isamarkuptext}%
\noindent
The function \isa{rtc} is annotated with concrete syntax: instead of
\isa{rtc\ r} we can write \isa{r{\isaliteral{2A}{\isacharasterisk}}}. The actual definition
consists of two rules. Reflexivity is obvious and is immediately given the
\isa{iff} attribute to increase automation. The
second rule, \isa{rtc{\isaliteral{5F}{\isacharunderscore}}step}, says that we can always add one more
\isa{r}-step to the left. Although we could make \isa{rtc{\isaliteral{5F}{\isacharunderscore}}step} an
introduction rule, this is dangerous: the recursion in the second premise
slows down and may even kill the automatic tactics.

The above definition of the concept of reflexive transitive closure may
be sufficiently intuitive but it is certainly not the only possible one:
for a start, it does not even mention transitivity.
The rest of this section is devoted to proving that it is equivalent to
the standard definition. We start with a simple lemma:%
\end{isamarkuptext}%
\isamarkuptrue%
\isacommand{lemma}\isamarkupfalse%
\ {\isaliteral{5B}{\isacharbrackleft}}intro{\isaliteral{5D}{\isacharbrackright}}{\isaliteral{3A}{\isacharcolon}}\ {\isaliteral{22}{\isachardoublequoteopen}}{\isaliteral{28}{\isacharparenleft}}x{\isaliteral{2C}{\isacharcomma}}y{\isaliteral{29}{\isacharparenright}}\ {\isaliteral{5C3C696E3E}{\isasymin}}\ r\ {\isaliteral{5C3C4C6F6E6772696768746172726F773E}{\isasymLongrightarrow}}\ {\isaliteral{28}{\isacharparenleft}}x{\isaliteral{2C}{\isacharcomma}}y{\isaliteral{29}{\isacharparenright}}\ {\isaliteral{5C3C696E3E}{\isasymin}}\ r{\isaliteral{2A}{\isacharasterisk}}{\isaliteral{22}{\isachardoublequoteclose}}\isanewline
%
\isadelimproof
%
\endisadelimproof
%
\isatagproof
\isacommand{by}\isamarkupfalse%
{\isaliteral{28}{\isacharparenleft}}blast\ intro{\isaliteral{3A}{\isacharcolon}}\ rtc{\isaliteral{5F}{\isacharunderscore}}step{\isaliteral{29}{\isacharparenright}}%
\endisatagproof
{\isafoldproof}%
%
\isadelimproof
%
\endisadelimproof
%
\begin{isamarkuptext}%
\noindent
Although the lemma itself is an unremarkable consequence of the basic rules,
it has the advantage that it can be declared an introduction rule without the
danger of killing the automatic tactics because \isa{r{\isaliteral{2A}{\isacharasterisk}}} occurs only in
the conclusion and not in the premise. Thus some proofs that would otherwise
need \isa{rtc{\isaliteral{5F}{\isacharunderscore}}step} can now be found automatically. The proof also
shows that \isa{blast} is able to handle \isa{rtc{\isaliteral{5F}{\isacharunderscore}}step}. But
some of the other automatic tactics are more sensitive, and even \isa{blast} can be lead astray in the presence of large numbers of rules.

To prove transitivity, we need rule induction, i.e.\ theorem
\isa{rtc{\isaliteral{2E}{\isachardot}}induct}:
\begin{isabelle}%
\ \ \ \ \ {\isaliteral{5C3C6C6272616B6B3E}{\isasymlbrakk}}{\isaliteral{28}{\isacharparenleft}}{\isaliteral{3F}{\isacharquery}}x{\isadigit{1}}{\isaliteral{2E}{\isachardot}}{\isadigit{0}}{\isaliteral{2C}{\isacharcomma}}\ {\isaliteral{3F}{\isacharquery}}x{\isadigit{2}}{\isaliteral{2E}{\isachardot}}{\isadigit{0}}{\isaliteral{29}{\isacharparenright}}\ {\isaliteral{5C3C696E3E}{\isasymin}}\ {\isaliteral{3F}{\isacharquery}}r{\isaliteral{2A}{\isacharasterisk}}{\isaliteral{3B}{\isacharsemicolon}}\ {\isaliteral{5C3C416E643E}{\isasymAnd}}x{\isaliteral{2E}{\isachardot}}\ {\isaliteral{3F}{\isacharquery}}P\ x\ x{\isaliteral{3B}{\isacharsemicolon}}\isanewline
\isaindent{\ \ \ \ \ \ }{\isaliteral{5C3C416E643E}{\isasymAnd}}x\ y\ z{\isaliteral{2E}{\isachardot}}\ {\isaliteral{5C3C6C6272616B6B3E}{\isasymlbrakk}}{\isaliteral{28}{\isacharparenleft}}x{\isaliteral{2C}{\isacharcomma}}\ y{\isaliteral{29}{\isacharparenright}}\ {\isaliteral{5C3C696E3E}{\isasymin}}\ {\isaliteral{3F}{\isacharquery}}r{\isaliteral{3B}{\isacharsemicolon}}\ {\isaliteral{28}{\isacharparenleft}}y{\isaliteral{2C}{\isacharcomma}}\ z{\isaliteral{29}{\isacharparenright}}\ {\isaliteral{5C3C696E3E}{\isasymin}}\ {\isaliteral{3F}{\isacharquery}}r{\isaliteral{2A}{\isacharasterisk}}{\isaliteral{3B}{\isacharsemicolon}}\ {\isaliteral{3F}{\isacharquery}}P\ y\ z{\isaliteral{5C3C726272616B6B3E}{\isasymrbrakk}}\ {\isaliteral{5C3C4C6F6E6772696768746172726F773E}{\isasymLongrightarrow}}\ {\isaliteral{3F}{\isacharquery}}P\ x\ z{\isaliteral{5C3C726272616B6B3E}{\isasymrbrakk}}\isanewline
\isaindent{\ \ \ \ \ }{\isaliteral{5C3C4C6F6E6772696768746172726F773E}{\isasymLongrightarrow}}\ {\isaliteral{3F}{\isacharquery}}P\ {\isaliteral{3F}{\isacharquery}}x{\isadigit{1}}{\isaliteral{2E}{\isachardot}}{\isadigit{0}}\ {\isaliteral{3F}{\isacharquery}}x{\isadigit{2}}{\isaliteral{2E}{\isachardot}}{\isadigit{0}}%
\end{isabelle}
It says that \isa{{\isaliteral{3F}{\isacharquery}}P} holds for an arbitrary pair \isa{{\isaliteral{28}{\isacharparenleft}}{\isaliteral{3F}{\isacharquery}}x{\isadigit{1}}{\isaliteral{2E}{\isachardot}}{\isadigit{0}}{\isaliteral{2C}{\isacharcomma}}\ {\isaliteral{3F}{\isacharquery}}x{\isadigit{2}}{\isaliteral{2E}{\isachardot}}{\isadigit{0}}{\isaliteral{29}{\isacharparenright}}\ {\isaliteral{5C3C696E3E}{\isasymin}}\ {\isaliteral{3F}{\isacharquery}}r{\isaliteral{2A}{\isacharasterisk}}}
if \isa{{\isaliteral{3F}{\isacharquery}}P} is preserved by all rules of the inductive definition,
i.e.\ if \isa{{\isaliteral{3F}{\isacharquery}}P} holds for the conclusion provided it holds for the
premises. In general, rule induction for an $n$-ary inductive relation $R$
expects a premise of the form $(x@1,\dots,x@n) \in R$.

Now we turn to the inductive proof of transitivity:%
\end{isamarkuptext}%
\isamarkuptrue%
\isacommand{lemma}\isamarkupfalse%
\ rtc{\isaliteral{5F}{\isacharunderscore}}trans{\isaliteral{3A}{\isacharcolon}}\ {\isaliteral{22}{\isachardoublequoteopen}}{\isaliteral{5C3C6C6272616B6B3E}{\isasymlbrakk}}\ {\isaliteral{28}{\isacharparenleft}}x{\isaliteral{2C}{\isacharcomma}}y{\isaliteral{29}{\isacharparenright}}\ {\isaliteral{5C3C696E3E}{\isasymin}}\ r{\isaliteral{2A}{\isacharasterisk}}{\isaliteral{3B}{\isacharsemicolon}}\ {\isaliteral{28}{\isacharparenleft}}y{\isaliteral{2C}{\isacharcomma}}z{\isaliteral{29}{\isacharparenright}}\ {\isaliteral{5C3C696E3E}{\isasymin}}\ r{\isaliteral{2A}{\isacharasterisk}}\ {\isaliteral{5C3C726272616B6B3E}{\isasymrbrakk}}\ {\isaliteral{5C3C4C6F6E6772696768746172726F773E}{\isasymLongrightarrow}}\ {\isaliteral{28}{\isacharparenleft}}x{\isaliteral{2C}{\isacharcomma}}z{\isaliteral{29}{\isacharparenright}}\ {\isaliteral{5C3C696E3E}{\isasymin}}\ r{\isaliteral{2A}{\isacharasterisk}}{\isaliteral{22}{\isachardoublequoteclose}}\isanewline
%
\isadelimproof
%
\endisadelimproof
%
\isatagproof
\isacommand{apply}\isamarkupfalse%
{\isaliteral{28}{\isacharparenleft}}erule\ rtc{\isaliteral{2E}{\isachardot}}induct{\isaliteral{29}{\isacharparenright}}%
\begin{isamarkuptxt}%
\noindent
Unfortunately, even the base case is a problem:
\begin{isabelle}%
\ {\isadigit{1}}{\isaliteral{2E}{\isachardot}}\ {\isaliteral{5C3C416E643E}{\isasymAnd}}x{\isaliteral{2E}{\isachardot}}\ {\isaliteral{28}{\isacharparenleft}}y{\isaliteral{2C}{\isacharcomma}}\ z{\isaliteral{29}{\isacharparenright}}\ {\isaliteral{5C3C696E3E}{\isasymin}}\ r{\isaliteral{2A}{\isacharasterisk}}\ {\isaliteral{5C3C4C6F6E6772696768746172726F773E}{\isasymLongrightarrow}}\ {\isaliteral{28}{\isacharparenleft}}x{\isaliteral{2C}{\isacharcomma}}\ z{\isaliteral{29}{\isacharparenright}}\ {\isaliteral{5C3C696E3E}{\isasymin}}\ r{\isaliteral{2A}{\isacharasterisk}}%
\end{isabelle}
We have to abandon this proof attempt.
To understand what is going on, let us look again at \isa{rtc{\isaliteral{2E}{\isachardot}}induct}.
In the above application of \isa{erule}, the first premise of
\isa{rtc{\isaliteral{2E}{\isachardot}}induct} is unified with the first suitable assumption, which
is \isa{{\isaliteral{28}{\isacharparenleft}}x{\isaliteral{2C}{\isacharcomma}}\ y{\isaliteral{29}{\isacharparenright}}\ {\isaliteral{5C3C696E3E}{\isasymin}}\ r{\isaliteral{2A}{\isacharasterisk}}} rather than \isa{{\isaliteral{28}{\isacharparenleft}}y{\isaliteral{2C}{\isacharcomma}}\ z{\isaliteral{29}{\isacharparenright}}\ {\isaliteral{5C3C696E3E}{\isasymin}}\ r{\isaliteral{2A}{\isacharasterisk}}}. Although that
is what we want, it is merely due to the order in which the assumptions occur
in the subgoal, which it is not good practice to rely on. As a result,
\isa{{\isaliteral{3F}{\isacharquery}}xb} becomes \isa{x}, \isa{{\isaliteral{3F}{\isacharquery}}xa} becomes
\isa{y} and \isa{{\isaliteral{3F}{\isacharquery}}P} becomes \isa{{\isaliteral{5C3C6C616D6264613E}{\isasymlambda}}u\ v{\isaliteral{2E}{\isachardot}}\ {\isaliteral{28}{\isacharparenleft}}u{\isaliteral{2C}{\isacharcomma}}\ z{\isaliteral{29}{\isacharparenright}}\ {\isaliteral{5C3C696E3E}{\isasymin}}\ r{\isaliteral{2A}{\isacharasterisk}}}, thus
yielding the above subgoal. So what went wrong?

When looking at the instantiation of \isa{{\isaliteral{3F}{\isacharquery}}P} we see that it does not
depend on its second parameter at all. The reason is that in our original
goal, of the pair \isa{{\isaliteral{28}{\isacharparenleft}}x{\isaliteral{2C}{\isacharcomma}}\ y{\isaliteral{29}{\isacharparenright}}} only \isa{x} appears also in the
conclusion, but not \isa{y}. Thus our induction statement is too
general. Fortunately, it can easily be specialized:
transfer the additional premise \isa{{\isaliteral{28}{\isacharparenleft}}y{\isaliteral{2C}{\isacharcomma}}\ z{\isaliteral{29}{\isacharparenright}}\ {\isaliteral{5C3C696E3E}{\isasymin}}\ r{\isaliteral{2A}{\isacharasterisk}}} into the conclusion:%
\end{isamarkuptxt}%
\isamarkuptrue%
%
\endisatagproof
{\isafoldproof}%
%
\isadelimproof
%
\endisadelimproof
\isacommand{lemma}\isamarkupfalse%
\ rtc{\isaliteral{5F}{\isacharunderscore}}trans{\isaliteral{5B}{\isacharbrackleft}}rule{\isaliteral{5F}{\isacharunderscore}}format{\isaliteral{5D}{\isacharbrackright}}{\isaliteral{3A}{\isacharcolon}}\isanewline
\ \ {\isaliteral{22}{\isachardoublequoteopen}}{\isaliteral{28}{\isacharparenleft}}x{\isaliteral{2C}{\isacharcomma}}y{\isaliteral{29}{\isacharparenright}}\ {\isaliteral{5C3C696E3E}{\isasymin}}\ r{\isaliteral{2A}{\isacharasterisk}}\ {\isaliteral{5C3C4C6F6E6772696768746172726F773E}{\isasymLongrightarrow}}\ {\isaliteral{28}{\isacharparenleft}}y{\isaliteral{2C}{\isacharcomma}}z{\isaliteral{29}{\isacharparenright}}\ {\isaliteral{5C3C696E3E}{\isasymin}}\ r{\isaliteral{2A}{\isacharasterisk}}\ {\isaliteral{5C3C6C6F6E6772696768746172726F773E}{\isasymlongrightarrow}}\ {\isaliteral{28}{\isacharparenleft}}x{\isaliteral{2C}{\isacharcomma}}z{\isaliteral{29}{\isacharparenright}}\ {\isaliteral{5C3C696E3E}{\isasymin}}\ r{\isaliteral{2A}{\isacharasterisk}}{\isaliteral{22}{\isachardoublequoteclose}}%
\isadelimproof
%
\endisadelimproof
%
\isatagproof
%
\begin{isamarkuptxt}%
\noindent
This is not an obscure trick but a generally applicable heuristic:
\begin{quote}\em
When proving a statement by rule induction on $(x@1,\dots,x@n) \in R$,
pull all other premises containing any of the $x@i$ into the conclusion
using $\longrightarrow$.
\end{quote}
A similar heuristic for other kinds of inductions is formulated in
\S\ref{sec:ind-var-in-prems}. The \isa{rule{\isaliteral{5F}{\isacharunderscore}}format} directive turns
\isa{{\isaliteral{5C3C6C6F6E6772696768746172726F773E}{\isasymlongrightarrow}}} back into \isa{{\isaliteral{5C3C4C6F6E6772696768746172726F773E}{\isasymLongrightarrow}}}: in the end we obtain the original
statement of our lemma.%
\end{isamarkuptxt}%
\isamarkuptrue%
\isacommand{apply}\isamarkupfalse%
{\isaliteral{28}{\isacharparenleft}}erule\ rtc{\isaliteral{2E}{\isachardot}}induct{\isaliteral{29}{\isacharparenright}}%
\begin{isamarkuptxt}%
\noindent
Now induction produces two subgoals which are both proved automatically:
\begin{isabelle}%
\ {\isadigit{1}}{\isaliteral{2E}{\isachardot}}\ {\isaliteral{5C3C416E643E}{\isasymAnd}}x{\isaliteral{2E}{\isachardot}}\ {\isaliteral{28}{\isacharparenleft}}x{\isaliteral{2C}{\isacharcomma}}\ z{\isaliteral{29}{\isacharparenright}}\ {\isaliteral{5C3C696E3E}{\isasymin}}\ r{\isaliteral{2A}{\isacharasterisk}}\ {\isaliteral{5C3C6C6F6E6772696768746172726F773E}{\isasymlongrightarrow}}\ {\isaliteral{28}{\isacharparenleft}}x{\isaliteral{2C}{\isacharcomma}}\ z{\isaliteral{29}{\isacharparenright}}\ {\isaliteral{5C3C696E3E}{\isasymin}}\ r{\isaliteral{2A}{\isacharasterisk}}\isanewline
\ {\isadigit{2}}{\isaliteral{2E}{\isachardot}}\ {\isaliteral{5C3C416E643E}{\isasymAnd}}x\ y\ za{\isaliteral{2E}{\isachardot}}\isanewline
\isaindent{\ {\isadigit{2}}{\isaliteral{2E}{\isachardot}}\ \ \ \ }{\isaliteral{5C3C6C6272616B6B3E}{\isasymlbrakk}}{\isaliteral{28}{\isacharparenleft}}x{\isaliteral{2C}{\isacharcomma}}\ y{\isaliteral{29}{\isacharparenright}}\ {\isaliteral{5C3C696E3E}{\isasymin}}\ r{\isaliteral{3B}{\isacharsemicolon}}\ {\isaliteral{28}{\isacharparenleft}}y{\isaliteral{2C}{\isacharcomma}}\ za{\isaliteral{29}{\isacharparenright}}\ {\isaliteral{5C3C696E3E}{\isasymin}}\ r{\isaliteral{2A}{\isacharasterisk}}{\isaliteral{3B}{\isacharsemicolon}}\ {\isaliteral{28}{\isacharparenleft}}za{\isaliteral{2C}{\isacharcomma}}\ z{\isaliteral{29}{\isacharparenright}}\ {\isaliteral{5C3C696E3E}{\isasymin}}\ r{\isaliteral{2A}{\isacharasterisk}}\ {\isaliteral{5C3C6C6F6E6772696768746172726F773E}{\isasymlongrightarrow}}\ {\isaliteral{28}{\isacharparenleft}}y{\isaliteral{2C}{\isacharcomma}}\ z{\isaliteral{29}{\isacharparenright}}\ {\isaliteral{5C3C696E3E}{\isasymin}}\ r{\isaliteral{2A}{\isacharasterisk}}{\isaliteral{5C3C726272616B6B3E}{\isasymrbrakk}}\isanewline
\isaindent{\ {\isadigit{2}}{\isaliteral{2E}{\isachardot}}\ \ \ \ }{\isaliteral{5C3C4C6F6E6772696768746172726F773E}{\isasymLongrightarrow}}\ {\isaliteral{28}{\isacharparenleft}}za{\isaliteral{2C}{\isacharcomma}}\ z{\isaliteral{29}{\isacharparenright}}\ {\isaliteral{5C3C696E3E}{\isasymin}}\ r{\isaliteral{2A}{\isacharasterisk}}\ {\isaliteral{5C3C6C6F6E6772696768746172726F773E}{\isasymlongrightarrow}}\ {\isaliteral{28}{\isacharparenleft}}x{\isaliteral{2C}{\isacharcomma}}\ z{\isaliteral{29}{\isacharparenright}}\ {\isaliteral{5C3C696E3E}{\isasymin}}\ r{\isaliteral{2A}{\isacharasterisk}}%
\end{isabelle}%
\end{isamarkuptxt}%
\isamarkuptrue%
\ \isacommand{apply}\isamarkupfalse%
{\isaliteral{28}{\isacharparenleft}}blast{\isaliteral{29}{\isacharparenright}}\isanewline
\isacommand{apply}\isamarkupfalse%
{\isaliteral{28}{\isacharparenleft}}blast\ intro{\isaliteral{3A}{\isacharcolon}}\ rtc{\isaliteral{5F}{\isacharunderscore}}step{\isaliteral{29}{\isacharparenright}}\isanewline
\isacommand{done}\isamarkupfalse%
%
\endisatagproof
{\isafoldproof}%
%
\isadelimproof
%
\endisadelimproof
%
\begin{isamarkuptext}%
Let us now prove that \isa{r{\isaliteral{2A}{\isacharasterisk}}} is really the reflexive transitive closure
of \isa{r}, i.e.\ the least reflexive and transitive
relation containing \isa{r}. The latter is easily formalized%
\end{isamarkuptext}%
\isamarkuptrue%
\isacommand{inductive{\isaliteral{5F}{\isacharunderscore}}set}\isamarkupfalse%
\isanewline
\ \ rtc{\isadigit{2}}\ {\isaliteral{3A}{\isacharcolon}}{\isaliteral{3A}{\isacharcolon}}\ {\isaliteral{22}{\isachardoublequoteopen}}{\isaliteral{28}{\isacharparenleft}}{\isaliteral{27}{\isacharprime}}a\ {\isaliteral{5C3C74696D65733E}{\isasymtimes}}\ {\isaliteral{27}{\isacharprime}}a{\isaliteral{29}{\isacharparenright}}set\ {\isaliteral{5C3C52696768746172726F773E}{\isasymRightarrow}}\ {\isaliteral{28}{\isacharparenleft}}{\isaliteral{27}{\isacharprime}}a\ {\isaliteral{5C3C74696D65733E}{\isasymtimes}}\ {\isaliteral{27}{\isacharprime}}a{\isaliteral{29}{\isacharparenright}}set{\isaliteral{22}{\isachardoublequoteclose}}\isanewline
\ \ \isakeyword{for}\ r\ {\isaliteral{3A}{\isacharcolon}}{\isaliteral{3A}{\isacharcolon}}\ {\isaliteral{22}{\isachardoublequoteopen}}{\isaliteral{28}{\isacharparenleft}}{\isaliteral{27}{\isacharprime}}a\ {\isaliteral{5C3C74696D65733E}{\isasymtimes}}\ {\isaliteral{27}{\isacharprime}}a{\isaliteral{29}{\isacharparenright}}set{\isaliteral{22}{\isachardoublequoteclose}}\isanewline
\isakeyword{where}\isanewline
\ \ {\isaliteral{22}{\isachardoublequoteopen}}{\isaliteral{28}{\isacharparenleft}}x{\isaliteral{2C}{\isacharcomma}}y{\isaliteral{29}{\isacharparenright}}\ {\isaliteral{5C3C696E3E}{\isasymin}}\ r\ {\isaliteral{5C3C4C6F6E6772696768746172726F773E}{\isasymLongrightarrow}}\ {\isaliteral{28}{\isacharparenleft}}x{\isaliteral{2C}{\isacharcomma}}y{\isaliteral{29}{\isacharparenright}}\ {\isaliteral{5C3C696E3E}{\isasymin}}\ rtc{\isadigit{2}}\ r{\isaliteral{22}{\isachardoublequoteclose}}\isanewline
{\isaliteral{7C}{\isacharbar}}\ {\isaliteral{22}{\isachardoublequoteopen}}{\isaliteral{28}{\isacharparenleft}}x{\isaliteral{2C}{\isacharcomma}}x{\isaliteral{29}{\isacharparenright}}\ {\isaliteral{5C3C696E3E}{\isasymin}}\ rtc{\isadigit{2}}\ r{\isaliteral{22}{\isachardoublequoteclose}}\isanewline
{\isaliteral{7C}{\isacharbar}}\ {\isaliteral{22}{\isachardoublequoteopen}}{\isaliteral{5C3C6C6272616B6B3E}{\isasymlbrakk}}\ {\isaliteral{28}{\isacharparenleft}}x{\isaliteral{2C}{\isacharcomma}}y{\isaliteral{29}{\isacharparenright}}\ {\isaliteral{5C3C696E3E}{\isasymin}}\ rtc{\isadigit{2}}\ r{\isaliteral{3B}{\isacharsemicolon}}\ {\isaliteral{28}{\isacharparenleft}}y{\isaliteral{2C}{\isacharcomma}}z{\isaliteral{29}{\isacharparenright}}\ {\isaliteral{5C3C696E3E}{\isasymin}}\ rtc{\isadigit{2}}\ r\ {\isaliteral{5C3C726272616B6B3E}{\isasymrbrakk}}\ {\isaliteral{5C3C4C6F6E6772696768746172726F773E}{\isasymLongrightarrow}}\ {\isaliteral{28}{\isacharparenleft}}x{\isaliteral{2C}{\isacharcomma}}z{\isaliteral{29}{\isacharparenright}}\ {\isaliteral{5C3C696E3E}{\isasymin}}\ rtc{\isadigit{2}}\ r{\isaliteral{22}{\isachardoublequoteclose}}%
\begin{isamarkuptext}%
\noindent
and the equivalence of the two definitions is easily shown by the obvious rule
inductions:%
\end{isamarkuptext}%
\isamarkuptrue%
\isacommand{lemma}\isamarkupfalse%
\ {\isaliteral{22}{\isachardoublequoteopen}}{\isaliteral{28}{\isacharparenleft}}x{\isaliteral{2C}{\isacharcomma}}y{\isaliteral{29}{\isacharparenright}}\ {\isaliteral{5C3C696E3E}{\isasymin}}\ rtc{\isadigit{2}}\ r\ {\isaliteral{5C3C4C6F6E6772696768746172726F773E}{\isasymLongrightarrow}}\ {\isaliteral{28}{\isacharparenleft}}x{\isaliteral{2C}{\isacharcomma}}y{\isaliteral{29}{\isacharparenright}}\ {\isaliteral{5C3C696E3E}{\isasymin}}\ r{\isaliteral{2A}{\isacharasterisk}}{\isaliteral{22}{\isachardoublequoteclose}}\isanewline
%
\isadelimproof
%
\endisadelimproof
%
\isatagproof
\isacommand{apply}\isamarkupfalse%
{\isaliteral{28}{\isacharparenleft}}erule\ rtc{\isadigit{2}}{\isaliteral{2E}{\isachardot}}induct{\isaliteral{29}{\isacharparenright}}\isanewline
\ \ \isacommand{apply}\isamarkupfalse%
{\isaliteral{28}{\isacharparenleft}}blast{\isaliteral{29}{\isacharparenright}}\isanewline
\ \isacommand{apply}\isamarkupfalse%
{\isaliteral{28}{\isacharparenleft}}blast{\isaliteral{29}{\isacharparenright}}\isanewline
\isacommand{apply}\isamarkupfalse%
{\isaliteral{28}{\isacharparenleft}}blast\ intro{\isaliteral{3A}{\isacharcolon}}\ rtc{\isaliteral{5F}{\isacharunderscore}}trans{\isaliteral{29}{\isacharparenright}}\isanewline
\isacommand{done}\isamarkupfalse%
%
\endisatagproof
{\isafoldproof}%
%
\isadelimproof
\isanewline
%
\endisadelimproof
\isanewline
\isacommand{lemma}\isamarkupfalse%
\ {\isaliteral{22}{\isachardoublequoteopen}}{\isaliteral{28}{\isacharparenleft}}x{\isaliteral{2C}{\isacharcomma}}y{\isaliteral{29}{\isacharparenright}}\ {\isaliteral{5C3C696E3E}{\isasymin}}\ r{\isaliteral{2A}{\isacharasterisk}}\ {\isaliteral{5C3C4C6F6E6772696768746172726F773E}{\isasymLongrightarrow}}\ {\isaliteral{28}{\isacharparenleft}}x{\isaliteral{2C}{\isacharcomma}}y{\isaliteral{29}{\isacharparenright}}\ {\isaliteral{5C3C696E3E}{\isasymin}}\ rtc{\isadigit{2}}\ r{\isaliteral{22}{\isachardoublequoteclose}}\isanewline
%
\isadelimproof
%
\endisadelimproof
%
\isatagproof
\isacommand{apply}\isamarkupfalse%
{\isaliteral{28}{\isacharparenleft}}erule\ rtc{\isaliteral{2E}{\isachardot}}induct{\isaliteral{29}{\isacharparenright}}\isanewline
\ \isacommand{apply}\isamarkupfalse%
{\isaliteral{28}{\isacharparenleft}}blast\ intro{\isaliteral{3A}{\isacharcolon}}\ rtc{\isadigit{2}}{\isaliteral{2E}{\isachardot}}intros{\isaliteral{29}{\isacharparenright}}\isanewline
\isacommand{apply}\isamarkupfalse%
{\isaliteral{28}{\isacharparenleft}}blast\ intro{\isaliteral{3A}{\isacharcolon}}\ rtc{\isadigit{2}}{\isaliteral{2E}{\isachardot}}intros{\isaliteral{29}{\isacharparenright}}\isanewline
\isacommand{done}\isamarkupfalse%
%
\endisatagproof
{\isafoldproof}%
%
\isadelimproof
%
\endisadelimproof
%
\begin{isamarkuptext}%
So why did we start with the first definition? Because it is simpler. It
contains only two rules, and the single step rule is simpler than
transitivity.  As a consequence, \isa{rtc{\isaliteral{2E}{\isachardot}}induct} is simpler than
\isa{rtc{\isadigit{2}}{\isaliteral{2E}{\isachardot}}induct}. Since inductive proofs are hard enough
anyway, we should always pick the simplest induction schema available.
Hence \isa{rtc} is the definition of choice.
\index{reflexive transitive closure!defining inductively|)}

\begin{exercise}\label{ex:converse-rtc-step}
Show that the converse of \isa{rtc{\isaliteral{5F}{\isacharunderscore}}step} also holds:
\begin{isabelle}%
\ \ \ \ \ {\isaliteral{5C3C6C6272616B6B3E}{\isasymlbrakk}}{\isaliteral{28}{\isacharparenleft}}x{\isaliteral{2C}{\isacharcomma}}\ y{\isaliteral{29}{\isacharparenright}}\ {\isaliteral{5C3C696E3E}{\isasymin}}\ r{\isaliteral{2A}{\isacharasterisk}}{\isaliteral{3B}{\isacharsemicolon}}\ {\isaliteral{28}{\isacharparenleft}}y{\isaliteral{2C}{\isacharcomma}}\ z{\isaliteral{29}{\isacharparenright}}\ {\isaliteral{5C3C696E3E}{\isasymin}}\ r{\isaliteral{5C3C726272616B6B3E}{\isasymrbrakk}}\ {\isaliteral{5C3C4C6F6E6772696768746172726F773E}{\isasymLongrightarrow}}\ {\isaliteral{28}{\isacharparenleft}}x{\isaliteral{2C}{\isacharcomma}}\ z{\isaliteral{29}{\isacharparenright}}\ {\isaliteral{5C3C696E3E}{\isasymin}}\ r{\isaliteral{2A}{\isacharasterisk}}%
\end{isabelle}
\end{exercise}
\begin{exercise}
Repeat the development of this section, but starting with a definition of
\isa{rtc} where \isa{rtc{\isaliteral{5F}{\isacharunderscore}}step} is replaced by its converse as shown
in exercise~\ref{ex:converse-rtc-step}.
\end{exercise}%
\end{isamarkuptext}%
\isamarkuptrue%
%
\isadelimproof
%
\endisadelimproof
%
\isatagproof
%
\endisatagproof
{\isafoldproof}%
%
\isadelimproof
%
\endisadelimproof
%
\isadelimtheory
%
\endisadelimtheory
%
\isatagtheory
%
\endisatagtheory
{\isafoldtheory}%
%
\isadelimtheory
%
\endisadelimtheory
\end{isabellebody}%
%%% Local Variables:
%%% mode: latex
%%% TeX-master: "root"
%%% End:

%
\begin{isabellebody}%
\def\isabellecontext{AB}%
%
\isamarkupsection{Case Study: A Context Free Grammar%
}
%
\begin{isamarkuptext}%
\label{sec:CFG}
\index{grammars!defining inductively|(}%
Grammars are nothing but shorthands for inductive definitions of nonterminals
which represent sets of strings. For example, the production
$A \to B c$ is short for
\[ w \in B \Longrightarrow wc \in A \]
This section demonstrates this idea with an example
due to Hopcroft and Ullman, a grammar for generating all words with an
equal number of $a$'s and~$b$'s:
\begin{eqnarray}
S &\to& \epsilon \mid b A \mid a B \nonumber\\
A &\to& a S \mid b A A \nonumber\\
B &\to& b S \mid a B B \nonumber
\end{eqnarray}
At the end we say a few words about the relationship between
the original proof \cite[p.\ts81]{HopcroftUllman} and our formal version.

We start by fixing the alphabet, which consists only of \isa{a}'s
and~\isa{b}'s:%
\end{isamarkuptext}%
\isacommand{datatype}\ alfa\ {\isacharequal}\ a\ {\isacharbar}\ b%
\begin{isamarkuptext}%
\noindent
For convenience we include the following easy lemmas as simplification rules:%
\end{isamarkuptext}%
\isacommand{lemma}\ {\isacharbrackleft}simp{\isacharbrackright}{\isacharcolon}\ {\isachardoublequote}{\isacharparenleft}x\ {\isasymnoteq}\ a{\isacharparenright}\ {\isacharequal}\ {\isacharparenleft}x\ {\isacharequal}\ b{\isacharparenright}\ {\isasymand}\ {\isacharparenleft}x\ {\isasymnoteq}\ b{\isacharparenright}\ {\isacharequal}\ {\isacharparenleft}x\ {\isacharequal}\ a{\isacharparenright}{\isachardoublequote}\isanewline
\isacommand{by}\ {\isacharparenleft}case{\isacharunderscore}tac\ x{\isacharcomma}\ auto{\isacharparenright}%
\begin{isamarkuptext}%
\noindent
Words over this alphabet are of type \isa{alfa\ list}, and
the three nonterminals are declared as sets of such words:%
\end{isamarkuptext}%
\isacommand{consts}\ S\ {\isacharcolon}{\isacharcolon}\ {\isachardoublequote}alfa\ list\ set{\isachardoublequote}\isanewline
\ \ \ \ \ \ \ A\ {\isacharcolon}{\isacharcolon}\ {\isachardoublequote}alfa\ list\ set{\isachardoublequote}\isanewline
\ \ \ \ \ \ \ B\ {\isacharcolon}{\isacharcolon}\ {\isachardoublequote}alfa\ list\ set{\isachardoublequote}%
\begin{isamarkuptext}%
\noindent
The productions above are recast as a \emph{mutual} inductive
definition\index{inductive definition!simultaneous}
of \isa{S}, \isa{A} and~\isa{B}:%
\end{isamarkuptext}%
\isacommand{inductive}\ S\ A\ B\isanewline
\isakeyword{intros}\isanewline
\ \ {\isachardoublequote}{\isacharbrackleft}{\isacharbrackright}\ {\isasymin}\ S{\isachardoublequote}\isanewline
\ \ {\isachardoublequote}w\ {\isasymin}\ A\ {\isasymLongrightarrow}\ b{\isacharhash}w\ {\isasymin}\ S{\isachardoublequote}\isanewline
\ \ {\isachardoublequote}w\ {\isasymin}\ B\ {\isasymLongrightarrow}\ a{\isacharhash}w\ {\isasymin}\ S{\isachardoublequote}\isanewline
\isanewline
\ \ {\isachardoublequote}w\ {\isasymin}\ S\ \ \ \ \ \ \ \ {\isasymLongrightarrow}\ a{\isacharhash}w\ \ \ {\isasymin}\ A{\isachardoublequote}\isanewline
\ \ {\isachardoublequote}{\isasymlbrakk}\ v{\isasymin}A{\isacharsemicolon}\ w{\isasymin}A\ {\isasymrbrakk}\ {\isasymLongrightarrow}\ b{\isacharhash}v{\isacharat}w\ {\isasymin}\ A{\isachardoublequote}\isanewline
\isanewline
\ \ {\isachardoublequote}w\ {\isasymin}\ S\ \ \ \ \ \ \ \ \ \ \ \ {\isasymLongrightarrow}\ b{\isacharhash}w\ \ \ {\isasymin}\ B{\isachardoublequote}\isanewline
\ \ {\isachardoublequote}{\isasymlbrakk}\ v\ {\isasymin}\ B{\isacharsemicolon}\ w\ {\isasymin}\ B\ {\isasymrbrakk}\ {\isasymLongrightarrow}\ a{\isacharhash}v{\isacharat}w\ {\isasymin}\ B{\isachardoublequote}%
\begin{isamarkuptext}%
\noindent
First we show that all words in \isa{S} contain the same number of \isa{a}'s and \isa{b}'s. Since the definition of \isa{S} is by mutual
induction, so is the proof: we show at the same time that all words in
\isa{A} contain one more \isa{a} than \isa{b} and all words in \isa{B} contains one more \isa{b} than \isa{a}.%
\end{isamarkuptext}%
\isacommand{lemma}\ correctness{\isacharcolon}\isanewline
\ \ {\isachardoublequote}{\isacharparenleft}w\ {\isasymin}\ S\ {\isasymlongrightarrow}\ size{\isacharbrackleft}x{\isasymin}w{\isachardot}\ x{\isacharequal}a{\isacharbrackright}\ {\isacharequal}\ size{\isacharbrackleft}x{\isasymin}w{\isachardot}\ x{\isacharequal}b{\isacharbrackright}{\isacharparenright}\ \ \ \ \ {\isasymand}\isanewline
\ \ \ {\isacharparenleft}w\ {\isasymin}\ A\ {\isasymlongrightarrow}\ size{\isacharbrackleft}x{\isasymin}w{\isachardot}\ x{\isacharequal}a{\isacharbrackright}\ {\isacharequal}\ size{\isacharbrackleft}x{\isasymin}w{\isachardot}\ x{\isacharequal}b{\isacharbrackright}\ {\isacharplus}\ {\isadigit{1}}{\isacharparenright}\ {\isasymand}\isanewline
\ \ \ {\isacharparenleft}w\ {\isasymin}\ B\ {\isasymlongrightarrow}\ size{\isacharbrackleft}x{\isasymin}w{\isachardot}\ x{\isacharequal}b{\isacharbrackright}\ {\isacharequal}\ size{\isacharbrackleft}x{\isasymin}w{\isachardot}\ x{\isacharequal}a{\isacharbrackright}\ {\isacharplus}\ {\isadigit{1}}{\isacharparenright}{\isachardoublequote}%
\begin{isamarkuptxt}%
\noindent
These propositions are expressed with the help of the predefined \isa{filter} function on lists, which has the convenient syntax \isa{{\isacharbrackleft}x{\isasymin}xs{\isachardot}\ P\ x{\isacharbrackright}}, the list of all elements \isa{x} in \isa{xs} such that \isa{P\ x}
holds. Remember that on lists \isa{size} and \isa{length} are synonymous.

The proof itself is by rule induction and afterwards automatic:%
\end{isamarkuptxt}%
\isacommand{by}\ {\isacharparenleft}rule\ S{\isacharunderscore}A{\isacharunderscore}B{\isachardot}induct{\isacharcomma}\ auto{\isacharparenright}%
\begin{isamarkuptext}%
\noindent
This may seem surprising at first, and is indeed an indication of the power
of inductive definitions. But it is also quite straightforward. For example,
consider the production $A \to b A A$: if $v,w \in A$ and the elements of $A$
contain one more $a$ than~$b$'s, then $bvw$ must again contain one more $a$
than~$b$'s.

As usual, the correctness of syntactic descriptions is easy, but completeness
is hard: does \isa{S} contain \emph{all} words with an equal number of
\isa{a}'s and \isa{b}'s? It turns out that this proof requires the
following lemma: every string with two more \isa{a}'s than \isa{b}'s can be cut somewhere such that each half has one more \isa{a} than
\isa{b}. This is best seen by imagining counting the difference between the
number of \isa{a}'s and \isa{b}'s starting at the left end of the
word. We start with 0 and end (at the right end) with 2. Since each move to the
right increases or decreases the difference by 1, we must have passed through
1 on our way from 0 to 2. Formally, we appeal to the following discrete
intermediate value theorem \isa{nat{\isadigit{0}}{\isacharunderscore}intermed{\isacharunderscore}int{\isacharunderscore}val}
\begin{isabelle}%
\ \ \ \ \ {\isasymlbrakk}{\isasymforall}i{\isachardot}\ i\ {\isacharless}\ n\ {\isasymlongrightarrow}\ {\isasymbar}f\ {\isacharparenleft}i\ {\isacharplus}\ {\isadigit{1}}{\isacharparenright}\ {\isacharminus}\ f\ i{\isasymbar}\ {\isasymle}\ {\isacharhash}{\isadigit{1}}{\isacharsemicolon}\ f\ {\isadigit{0}}\ {\isasymle}\ k{\isacharsemicolon}\ k\ {\isasymle}\ f\ n{\isasymrbrakk}\isanewline
\isaindent{\ \ \ \ \ }{\isasymLongrightarrow}\ {\isasymexists}i{\isachardot}\ i\ {\isasymle}\ n\ {\isasymand}\ f\ i\ {\isacharequal}\ k%
\end{isabelle}
where \isa{f} is of type \isa{nat\ {\isasymRightarrow}\ int}, \isa{int} are the integers,
\isa{{\isasymbar}{\isachardot}{\isasymbar}} is the absolute value function\footnote{See
Table~\ref{tab:ascii} in the Appendix for the correct \textsc{ascii}
syntax.}, and \isa{{\isacharhash}{\isadigit{1}}} is the integer 1 (see \S\ref{sec:numbers}).

First we show that our specific function, the difference between the
numbers of \isa{a}'s and \isa{b}'s, does indeed only change by 1 in every
move to the right. At this point we also start generalizing from \isa{a}'s
and \isa{b}'s to an arbitrary property \isa{P}. Otherwise we would have
to prove the desired lemma twice, once as stated above and once with the
roles of \isa{a}'s and \isa{b}'s interchanged.%
\end{isamarkuptext}%
\isacommand{lemma}\ step{\isadigit{1}}{\isacharcolon}\ {\isachardoublequote}{\isasymforall}i\ {\isacharless}\ size\ w{\isachardot}\isanewline
\ \ {\isasymbar}{\isacharparenleft}int{\isacharparenleft}size{\isacharbrackleft}x{\isasymin}take\ {\isacharparenleft}i{\isacharplus}{\isadigit{1}}{\isacharparenright}\ w{\isachardot}\ P\ x{\isacharbrackright}{\isacharparenright}{\isacharminus}int{\isacharparenleft}size{\isacharbrackleft}x{\isasymin}take\ {\isacharparenleft}i{\isacharplus}{\isadigit{1}}{\isacharparenright}\ w{\isachardot}\ {\isasymnot}P\ x{\isacharbrackright}{\isacharparenright}{\isacharparenright}\isanewline
\ \ \ {\isacharminus}\ {\isacharparenleft}int{\isacharparenleft}size{\isacharbrackleft}x{\isasymin}take\ i\ w{\isachardot}\ P\ x{\isacharbrackright}{\isacharparenright}{\isacharminus}int{\isacharparenleft}size{\isacharbrackleft}x{\isasymin}take\ i\ w{\isachardot}\ {\isasymnot}P\ x{\isacharbrackright}{\isacharparenright}{\isacharparenright}{\isasymbar}\ {\isasymle}\ {\isacharhash}{\isadigit{1}}{\isachardoublequote}%
\begin{isamarkuptxt}%
\noindent
The lemma is a bit hard to read because of the coercion function
\isa{int\ {\isacharcolon}{\isacharcolon}\ nat\ {\isasymRightarrow}\ int}. It is required because \isa{size} returns
a natural number, but subtraction on type~\isa{nat} will do the wrong thing.
Function \isa{take} is predefined and \isa{take\ i\ xs} is the prefix of
length \isa{i} of \isa{xs}; below we also need \isa{drop\ i\ xs}, which
is what remains after that prefix has been dropped from \isa{xs}.

The proof is by induction on \isa{w}, with a trivial base case, and a not
so trivial induction step. Since it is essentially just arithmetic, we do not
discuss it.%
\end{isamarkuptxt}%
\isacommand{apply}{\isacharparenleft}induct\ w{\isacharparenright}\isanewline
\ \isacommand{apply}{\isacharparenleft}simp{\isacharparenright}\isanewline
\isacommand{by}{\isacharparenleft}force\ simp\ add{\isacharcolon}zabs{\isacharunderscore}def\ take{\isacharunderscore}Cons\ split{\isacharcolon}nat{\isachardot}split\ if{\isacharunderscore}splits{\isacharparenright}%
\begin{isamarkuptext}%
Finally we come to the above-mentioned lemma about cutting in half a word with two more elements of one sort than of the other sort:%
\end{isamarkuptext}%
\isacommand{lemma}\ part{\isadigit{1}}{\isacharcolon}\isanewline
\ {\isachardoublequote}size{\isacharbrackleft}x{\isasymin}w{\isachardot}\ P\ x{\isacharbrackright}\ {\isacharequal}\ size{\isacharbrackleft}x{\isasymin}w{\isachardot}\ {\isasymnot}P\ x{\isacharbrackright}{\isacharplus}{\isadigit{2}}\ {\isasymLongrightarrow}\isanewline
\ \ {\isasymexists}i{\isasymle}size\ w{\isachardot}\ size{\isacharbrackleft}x{\isasymin}take\ i\ w{\isachardot}\ P\ x{\isacharbrackright}\ {\isacharequal}\ size{\isacharbrackleft}x{\isasymin}take\ i\ w{\isachardot}\ {\isasymnot}P\ x{\isacharbrackright}{\isacharplus}{\isadigit{1}}{\isachardoublequote}%
\begin{isamarkuptxt}%
\noindent
This is proved by \isa{force} with the help of the intermediate value theorem,
instantiated appropriately and with its first premise disposed of by lemma
\isa{step{\isadigit{1}}}:%
\end{isamarkuptxt}%
\isacommand{apply}{\isacharparenleft}insert\ nat{\isadigit{0}}{\isacharunderscore}intermed{\isacharunderscore}int{\isacharunderscore}val{\isacharbrackleft}OF\ step{\isadigit{1}}{\isacharcomma}\ of\ {\isachardoublequote}P{\isachardoublequote}\ {\isachardoublequote}w{\isachardoublequote}\ {\isachardoublequote}{\isacharhash}{\isadigit{1}}{\isachardoublequote}{\isacharbrackright}{\isacharparenright}\isanewline
\isacommand{by}\ force%
\begin{isamarkuptext}%
\noindent

Lemma \isa{part{\isadigit{1}}} tells us only about the prefix \isa{take\ i\ w}.
An easy lemma deals with the suffix \isa{drop\ i\ w}:%
\end{isamarkuptext}%
\isacommand{lemma}\ part{\isadigit{2}}{\isacharcolon}\isanewline
\ \ {\isachardoublequote}{\isasymlbrakk}size{\isacharbrackleft}x{\isasymin}take\ i\ w\ {\isacharat}\ drop\ i\ w{\isachardot}\ P\ x{\isacharbrackright}\ {\isacharequal}\isanewline
\ \ \ \ size{\isacharbrackleft}x{\isasymin}take\ i\ w\ {\isacharat}\ drop\ i\ w{\isachardot}\ {\isasymnot}P\ x{\isacharbrackright}{\isacharplus}{\isadigit{2}}{\isacharsemicolon}\isanewline
\ \ \ \ size{\isacharbrackleft}x{\isasymin}take\ i\ w{\isachardot}\ P\ x{\isacharbrackright}\ {\isacharequal}\ size{\isacharbrackleft}x{\isasymin}take\ i\ w{\isachardot}\ {\isasymnot}P\ x{\isacharbrackright}{\isacharplus}{\isadigit{1}}{\isasymrbrakk}\isanewline
\ \ \ {\isasymLongrightarrow}\ size{\isacharbrackleft}x{\isasymin}drop\ i\ w{\isachardot}\ P\ x{\isacharbrackright}\ {\isacharequal}\ size{\isacharbrackleft}x{\isasymin}drop\ i\ w{\isachardot}\ {\isasymnot}P\ x{\isacharbrackright}{\isacharplus}{\isadigit{1}}{\isachardoublequote}\isanewline
\isacommand{by}{\isacharparenleft}simp\ del{\isacharcolon}append{\isacharunderscore}take{\isacharunderscore}drop{\isacharunderscore}id{\isacharparenright}%
\begin{isamarkuptext}%
\noindent
In the proof we have disabled the normally useful lemma
\begin{isabelle}
\isa{take\ n\ xs\ {\isacharat}\ drop\ n\ xs\ {\isacharequal}\ xs}
\rulename{append_take_drop_id}
\end{isabelle}
to allow the simplifier to apply the following lemma instead:
\begin{isabelle}%
\ \ \ \ \ {\isacharbrackleft}x{\isasymin}xs{\isacharat}ys{\isachardot}\ P\ x{\isacharbrackright}\ {\isacharequal}\ {\isacharbrackleft}x{\isasymin}xs{\isachardot}\ P\ x{\isacharbrackright}\ {\isacharat}\ {\isacharbrackleft}x{\isasymin}ys{\isachardot}\ P\ x{\isacharbrackright}%
\end{isabelle}

To dispose of trivial cases automatically, the rules of the inductive
definition are declared simplification rules:%
\end{isamarkuptext}%
\isacommand{declare}\ S{\isacharunderscore}A{\isacharunderscore}B{\isachardot}intros{\isacharbrackleft}simp{\isacharbrackright}%
\begin{isamarkuptext}%
\noindent
This could have been done earlier but was not necessary so far.

The completeness theorem tells us that if a word has the same number of
\isa{a}'s and \isa{b}'s, then it is in \isa{S}, and similarly 
for \isa{A} and \isa{B}:%
\end{isamarkuptext}%
\isacommand{theorem}\ completeness{\isacharcolon}\isanewline
\ \ {\isachardoublequote}{\isacharparenleft}size{\isacharbrackleft}x{\isasymin}w{\isachardot}\ x{\isacharequal}a{\isacharbrackright}\ {\isacharequal}\ size{\isacharbrackleft}x{\isasymin}w{\isachardot}\ x{\isacharequal}b{\isacharbrackright}\ \ \ \ \ {\isasymlongrightarrow}\ w\ {\isasymin}\ S{\isacharparenright}\ {\isasymand}\isanewline
\ \ \ {\isacharparenleft}size{\isacharbrackleft}x{\isasymin}w{\isachardot}\ x{\isacharequal}a{\isacharbrackright}\ {\isacharequal}\ size{\isacharbrackleft}x{\isasymin}w{\isachardot}\ x{\isacharequal}b{\isacharbrackright}\ {\isacharplus}\ {\isadigit{1}}\ {\isasymlongrightarrow}\ w\ {\isasymin}\ A{\isacharparenright}\ {\isasymand}\isanewline
\ \ \ {\isacharparenleft}size{\isacharbrackleft}x{\isasymin}w{\isachardot}\ x{\isacharequal}b{\isacharbrackright}\ {\isacharequal}\ size{\isacharbrackleft}x{\isasymin}w{\isachardot}\ x{\isacharequal}a{\isacharbrackright}\ {\isacharplus}\ {\isadigit{1}}\ {\isasymlongrightarrow}\ w\ {\isasymin}\ B{\isacharparenright}{\isachardoublequote}%
\begin{isamarkuptxt}%
\noindent
The proof is by induction on \isa{w}. Structural induction would fail here
because, as we can see from the grammar, we need to make bigger steps than
merely appending a single letter at the front. Hence we induct on the length
of \isa{w}, using the induction rule \isa{length{\isacharunderscore}induct}:%
\end{isamarkuptxt}%
\isacommand{apply}{\isacharparenleft}induct{\isacharunderscore}tac\ w\ rule{\isacharcolon}\ length{\isacharunderscore}induct{\isacharparenright}%
\begin{isamarkuptxt}%
\noindent
The \isa{rule} parameter tells \isa{induct{\isacharunderscore}tac} explicitly which induction
rule to use. For details see \S\ref{sec:complete-ind} below.
In this case the result is that we may assume the lemma already
holds for all words shorter than \isa{w}.

The proof continues with a case distinction on \isa{w},
on whether \isa{w} is empty or not.%
\end{isamarkuptxt}%
\isacommand{apply}{\isacharparenleft}case{\isacharunderscore}tac\ w{\isacharparenright}\isanewline
\ \isacommand{apply}{\isacharparenleft}simp{\isacharunderscore}all{\isacharparenright}%
\begin{isamarkuptxt}%
\noindent
Simplification disposes of the base case and leaves only a conjunction
of two step cases to be proved:
if \isa{w\ {\isacharequal}\ a\ {\isacharhash}\ v} and \begin{isabelle}%
\ \ \ \ \ length\ {\isacharbrackleft}x{\isasymin}v\ {\isachardot}\ x\ {\isacharequal}\ a{\isacharbrackright}\ {\isacharequal}\ length\ {\isacharbrackleft}x{\isasymin}v\ {\isachardot}\ x\ {\isacharequal}\ b{\isacharbrackright}\ {\isacharplus}\ {\isadigit{2}}%
\end{isabelle} then
\isa{b\ {\isacharhash}\ v\ {\isasymin}\ A}, and similarly for \isa{w\ {\isacharequal}\ b\ {\isacharhash}\ v}.
We only consider the first case in detail.

After breaking the conjunction up into two cases, we can apply
\isa{part{\isadigit{1}}} to the assumption that \isa{w} contains two more \isa{a}'s than \isa{b}'s.%
\end{isamarkuptxt}%
\isacommand{apply}{\isacharparenleft}rule\ conjI{\isacharparenright}\isanewline
\ \isacommand{apply}{\isacharparenleft}clarify{\isacharparenright}\isanewline
\ \isacommand{apply}{\isacharparenleft}frule\ part{\isadigit{1}}{\isacharbrackleft}of\ {\isachardoublequote}{\isasymlambda}x{\isachardot}\ x{\isacharequal}a{\isachardoublequote}{\isacharcomma}\ simplified{\isacharbrackright}{\isacharparenright}\isanewline
\ \isacommand{apply}{\isacharparenleft}clarify{\isacharparenright}%
\begin{isamarkuptxt}%
\noindent
This yields an index \isa{i\ {\isasymle}\ length\ v} such that
\begin{isabelle}%
\ \ \ \ \ length\ {\isacharbrackleft}x{\isasymin}take\ i\ v\ {\isachardot}\ x\ {\isacharequal}\ a{\isacharbrackright}\ {\isacharequal}\ length\ {\isacharbrackleft}x{\isasymin}take\ i\ v\ {\isachardot}\ x\ {\isacharequal}\ b{\isacharbrackright}\ {\isacharplus}\ {\isadigit{1}}%
\end{isabelle}
With the help of \isa{part{\isadigit{2}}} it follows that
\begin{isabelle}%
\ \ \ \ \ length\ {\isacharbrackleft}x{\isasymin}drop\ i\ v\ {\isachardot}\ x\ {\isacharequal}\ a{\isacharbrackright}\ {\isacharequal}\ length\ {\isacharbrackleft}x{\isasymin}drop\ i\ v\ {\isachardot}\ x\ {\isacharequal}\ b{\isacharbrackright}\ {\isacharplus}\ {\isadigit{1}}%
\end{isabelle}%
\end{isamarkuptxt}%
\ \isacommand{apply}{\isacharparenleft}drule\ part{\isadigit{2}}{\isacharbrackleft}of\ {\isachardoublequote}{\isasymlambda}x{\isachardot}\ x{\isacharequal}a{\isachardoublequote}{\isacharcomma}\ simplified{\isacharbrackright}{\isacharparenright}\isanewline
\ \ \isacommand{apply}{\isacharparenleft}assumption{\isacharparenright}%
\begin{isamarkuptxt}%
\noindent
Now it is time to decompose \isa{v} in the conclusion \isa{b\ {\isacharhash}\ v\ {\isasymin}\ A}
into \isa{take\ i\ v\ {\isacharat}\ drop\ i\ v},%
\end{isamarkuptxt}%
\ \isacommand{apply}{\isacharparenleft}rule{\isacharunderscore}tac\ n{\isadigit{1}}{\isacharequal}i\ \isakeyword{and}\ t{\isacharequal}v\ \isakeyword{in}\ subst{\isacharbrackleft}OF\ append{\isacharunderscore}take{\isacharunderscore}drop{\isacharunderscore}id{\isacharbrackright}{\isacharparenright}%
\begin{isamarkuptxt}%
\noindent
(the variables \isa{n{\isadigit{1}}} and \isa{t} are the result of composing the
theorems \isa{subst} and \isa{append{\isacharunderscore}take{\isacharunderscore}drop{\isacharunderscore}id})
after which the appropriate rule of the grammar reduces the goal
to the two subgoals \isa{take\ i\ v\ {\isasymin}\ A} and \isa{drop\ i\ v\ {\isasymin}\ A}:%
\end{isamarkuptxt}%
\ \isacommand{apply}{\isacharparenleft}rule\ S{\isacharunderscore}A{\isacharunderscore}B{\isachardot}intros{\isacharparenright}%
\begin{isamarkuptxt}%
Both subgoals follow from the induction hypothesis because both \isa{take\ i\ v} and \isa{drop\ i\ v} are shorter than \isa{w}:%
\end{isamarkuptxt}%
\ \ \isacommand{apply}{\isacharparenleft}force\ simp\ add{\isacharcolon}\ min{\isacharunderscore}less{\isacharunderscore}iff{\isacharunderscore}disj{\isacharparenright}\isanewline
\ \isacommand{apply}{\isacharparenleft}force\ split\ add{\isacharcolon}\ nat{\isacharunderscore}diff{\isacharunderscore}split{\isacharparenright}%
\begin{isamarkuptxt}%
The case \isa{w\ {\isacharequal}\ b\ {\isacharhash}\ v} is proved analogously:%
\end{isamarkuptxt}%
\isacommand{apply}{\isacharparenleft}clarify{\isacharparenright}\isanewline
\isacommand{apply}{\isacharparenleft}frule\ part{\isadigit{1}}{\isacharbrackleft}of\ {\isachardoublequote}{\isasymlambda}x{\isachardot}\ x{\isacharequal}b{\isachardoublequote}{\isacharcomma}\ simplified{\isacharbrackright}{\isacharparenright}\isanewline
\isacommand{apply}{\isacharparenleft}clarify{\isacharparenright}\isanewline
\isacommand{apply}{\isacharparenleft}drule\ part{\isadigit{2}}{\isacharbrackleft}of\ {\isachardoublequote}{\isasymlambda}x{\isachardot}\ x{\isacharequal}b{\isachardoublequote}{\isacharcomma}\ simplified{\isacharbrackright}{\isacharparenright}\isanewline
\ \isacommand{apply}{\isacharparenleft}assumption{\isacharparenright}\isanewline
\isacommand{apply}{\isacharparenleft}rule{\isacharunderscore}tac\ n{\isadigit{1}}{\isacharequal}i\ \isakeyword{and}\ t{\isacharequal}v\ \isakeyword{in}\ subst{\isacharbrackleft}OF\ append{\isacharunderscore}take{\isacharunderscore}drop{\isacharunderscore}id{\isacharbrackright}{\isacharparenright}\isanewline
\isacommand{apply}{\isacharparenleft}rule\ S{\isacharunderscore}A{\isacharunderscore}B{\isachardot}intros{\isacharparenright}\isanewline
\ \isacommand{apply}{\isacharparenleft}force\ simp\ add{\isacharcolon}min{\isacharunderscore}less{\isacharunderscore}iff{\isacharunderscore}disj{\isacharparenright}\isanewline
\isacommand{by}{\isacharparenleft}force\ simp\ add{\isacharcolon}min{\isacharunderscore}less{\isacharunderscore}iff{\isacharunderscore}disj\ split\ add{\isacharcolon}\ nat{\isacharunderscore}diff{\isacharunderscore}split{\isacharparenright}%
\begin{isamarkuptext}%
We conclude this section with a comparison of our proof with 
Hopcroft\index{Hopcroft, J. E.} and Ullman's\index{Ullman, J. D.}
\cite[p.\ts81]{HopcroftUllman}.
For a start, the textbook
grammar, for no good reason, excludes the empty word, thus complicating
matters just a little bit: they have 8 instead of our 7 productions.

More importantly, the proof itself is different: rather than
separating the two directions, they perform one induction on the
length of a word. This deprives them of the beauty of rule induction,
and in the easy direction (correctness) their reasoning is more
detailed than our \isa{auto}. For the hard part (completeness), they
consider just one of the cases that our \isa{simp{\isacharunderscore}all} disposes of
automatically. Then they conclude the proof by saying about the
remaining cases: ``We do this in a manner similar to our method of
proof for part (1); this part is left to the reader''. But this is
precisely the part that requires the intermediate value theorem and
thus is not at all similar to the other cases (which are automatic in
Isabelle). The authors are at least cavalier about this point and may
even have overlooked the slight difficulty lurking in the omitted
cases.  Such errors are found in many pen-and-paper proofs when they
are scrutinized formally.%
\index{grammars!defining inductively|)}%
\end{isamarkuptext}%
\end{isabellebody}%
%%% Local Variables:
%%% mode: latex
%%% TeX-master: "root"
%%% End:


\index{inductive definition|)}
\index{*inductive|)}

\section{Advanced inductive definitions}

%
\begin{isabellebody}%
\def\isabellecontext{types}%
%
\isadelimtheory
%
\endisadelimtheory
%
\isatagtheory
%
\endisatagtheory
{\isafoldtheory}%
%
\isadelimtheory
%
\endisadelimtheory
\isacommand{type{\isaliteral{5F}{\isacharunderscore}}synonym}\isamarkupfalse%
\ number\ {\isaliteral{3D}{\isacharequal}}\ nat\isanewline
\isacommand{type{\isaliteral{5F}{\isacharunderscore}}synonym}\isamarkupfalse%
\ gate\ {\isaliteral{3D}{\isacharequal}}\ {\isaliteral{22}{\isachardoublequoteopen}}bool\ {\isaliteral{5C3C52696768746172726F773E}{\isasymRightarrow}}\ bool\ {\isaliteral{5C3C52696768746172726F773E}{\isasymRightarrow}}\ bool{\isaliteral{22}{\isachardoublequoteclose}}\isanewline
\isacommand{type{\isaliteral{5F}{\isacharunderscore}}synonym}\isamarkupfalse%
\ {\isaliteral{28}{\isacharparenleft}}{\isaliteral{27}{\isacharprime}}a{\isaliteral{2C}{\isacharcomma}}\ {\isaliteral{27}{\isacharprime}}b{\isaliteral{29}{\isacharparenright}}\ alist\ {\isaliteral{3D}{\isacharequal}}\ {\isaliteral{22}{\isachardoublequoteopen}}{\isaliteral{28}{\isacharparenleft}}{\isaliteral{27}{\isacharprime}}a\ {\isaliteral{5C3C74696D65733E}{\isasymtimes}}\ {\isaliteral{27}{\isacharprime}}b{\isaliteral{29}{\isacharparenright}}\ list{\isaliteral{22}{\isachardoublequoteclose}}%
\begin{isamarkuptext}%
\noindent
Internally all synonyms are fully expanded.  As a consequence Isabelle's
output never contains synonyms.  Their main purpose is to improve the
readability of theories.  Synonyms can be used just like any other
type.%
\end{isamarkuptext}%
\isamarkuptrue%
%
\isamarkupsubsection{Constant Definitions%
}
\isamarkuptrue%
%
\begin{isamarkuptext}%
\label{sec:ConstDefinitions}\indexbold{definitions}%
Nonrecursive definitions can be made with the \commdx{definition}
command, for example \isa{nand} and \isa{xor} gates
(based on type \isa{gate} above):%
\end{isamarkuptext}%
\isamarkuptrue%
\isacommand{definition}\isamarkupfalse%
\ nand\ {\isaliteral{3A}{\isacharcolon}}{\isaliteral{3A}{\isacharcolon}}\ gate\ \isakeyword{where}\ {\isaliteral{22}{\isachardoublequoteopen}}nand\ A\ B\ {\isaliteral{5C3C65717569763E}{\isasymequiv}}\ {\isaliteral{5C3C6E6F743E}{\isasymnot}}{\isaliteral{28}{\isacharparenleft}}A\ {\isaliteral{5C3C616E643E}{\isasymand}}\ B{\isaliteral{29}{\isacharparenright}}{\isaliteral{22}{\isachardoublequoteclose}}\isanewline
\isacommand{definition}\isamarkupfalse%
\ xor\ \ {\isaliteral{3A}{\isacharcolon}}{\isaliteral{3A}{\isacharcolon}}\ gate\ \isakeyword{where}\ {\isaliteral{22}{\isachardoublequoteopen}}xor\ \ A\ B\ {\isaliteral{5C3C65717569763E}{\isasymequiv}}\ A\ {\isaliteral{5C3C616E643E}{\isasymand}}\ {\isaliteral{5C3C6E6F743E}{\isasymnot}}B\ {\isaliteral{5C3C6F723E}{\isasymor}}\ {\isaliteral{5C3C6E6F743E}{\isasymnot}}A\ {\isaliteral{5C3C616E643E}{\isasymand}}\ B{\isaliteral{22}{\isachardoublequoteclose}}%
\begin{isamarkuptext}%
\noindent%
The symbol \indexboldpos{\isasymequiv}{$IsaEq} is a special form of equality
that must be used in constant definitions.
Pattern-matching is not allowed: each definition must be of
the form $f\,x@1\,\dots\,x@n~\isasymequiv~t$.
Section~\ref{sec:Simp-with-Defs} explains how definitions are used
in proofs. The default name of each definition is $f$\isa{{\isaliteral{5F}{\isacharunderscore}}def}, where
$f$ is the name of the defined constant.%
\end{isamarkuptext}%
\isamarkuptrue%
%
\isadelimtheory
%
\endisadelimtheory
%
\isatagtheory
%
\endisatagtheory
{\isafoldtheory}%
%
\isadelimtheory
%
\endisadelimtheory
\end{isabellebody}%
%%% Local Variables:
%%% mode: latex
%%% TeX-master: "root"
%%% End:

\chapter{Advanced Simplification, Recursion and Induction}

Although we have already learned a lot about simplification, recursion and
induction, there are some advanced proof techniques that we have not covered
yet and which are worth learning. The three sections of this chapter are almost
independent of each other and can be read in any order. Only the notion of
\emph{congruence rules}, introduced in the section on simplification, is
required for parts of the section on recursion.

%
\begin{isabellebody}%
\def\isabellecontext{simp}%
%
\isamarkupsubsubsection{Simplification rules%
}
%
\begin{isamarkuptext}%
\indexbold{simplification rule}
To facilitate simplification, theorems can be declared to be simplification
rules (with the help of the attribute \isa{{\isacharbrackleft}simp{\isacharbrackright}}\index{*simp
  (attribute)}), in which case proofs by simplification make use of these
rules automatically. In addition the constructs \isacommand{datatype} and
\isacommand{primrec} (and a few others) invisibly declare useful
simplification rules. Explicit definitions are \emph{not} declared
simplification rules automatically!

Not merely equations but pretty much any theorem can become a simplification
rule. The simplifier will try to make sense of it.  For example, a theorem
\isa{{\isasymnot}\ P} is automatically turned into \isa{P\ {\isacharequal}\ False}. The details
are explained in \S\ref{sec:SimpHow}.

The simplification attribute of theorems can be turned on and off as follows:
\begin{quote}
\isacommand{declare} \textit{theorem-name}\isa{{\isacharbrackleft}simp{\isacharbrackright}}\\
\isacommand{declare} \textit{theorem-name}\isa{{\isacharbrackleft}simp\ del{\isacharbrackright}}
\end{quote}
As a rule of thumb, equations that really simplify (like \isa{rev\ {\isacharparenleft}rev\ xs{\isacharparenright}\ {\isacharequal}\ xs} and \isa{xs\ {\isacharat}\ {\isacharbrackleft}{\isacharbrackright}\ {\isacharequal}\ xs}) should be made simplification
rules.  Those of a more specific nature (e.g.\ distributivity laws, which
alter the structure of terms considerably) should only be used selectively,
i.e.\ they should not be default simplification rules.  Conversely, it may
also happen that a simplification rule needs to be disabled in certain
proofs.  Frequent changes in the simplification status of a theorem may
indicate a badly designed theory.
\begin{warn}
  Simplification may not terminate, for example if both $f(x) = g(x)$ and
  $g(x) = f(x)$ are simplification rules. It is the user's responsibility not
  to include simplification rules that can lead to nontermination, either on
  their own or in combination with other simplification rules.
\end{warn}%
\end{isamarkuptext}%
%
\isamarkupsubsubsection{The simplification method%
}
%
\begin{isamarkuptext}%
\index{*simp (method)|bold}
The general format of the simplification method is
\begin{quote}
\isa{simp} \textit{list of modifiers}
\end{quote}
where the list of \emph{modifiers} helps to fine tune the behaviour and may
be empty. Most if not all of the proofs seen so far could have been performed
with \isa{simp} instead of \isa{auto}, except that \isa{simp} attacks
only the first subgoal and may thus need to be repeated---use
\isaindex{simp_all} to simplify all subgoals.
Note that \isa{simp} fails if nothing changes.%
\end{isamarkuptext}%
%
\isamarkupsubsubsection{Adding and deleting simplification rules%
}
%
\begin{isamarkuptext}%
If a certain theorem is merely needed in a few proofs by simplification,
we do not need to make it a global simplification rule. Instead we can modify
the set of simplification rules used in a simplification step by adding rules
to it and/or deleting rules from it. The two modifiers for this are
\begin{quote}
\isa{add{\isacharcolon}} \textit{list of theorem names}\\
\isa{del{\isacharcolon}} \textit{list of theorem names}
\end{quote}
In case you want to use only a specific list of theorems and ignore all
others:
\begin{quote}
\isa{only{\isacharcolon}} \textit{list of theorem names}
\end{quote}%
\end{isamarkuptext}%
%
\isamarkupsubsubsection{Assumptions%
}
%
\begin{isamarkuptext}%
\index{simplification!with/of assumptions}
By default, assumptions are part of the simplification process: they are used
as simplification rules and are simplified themselves. For example:%
\end{isamarkuptext}%
\isacommand{lemma}\ {\isachardoublequote}{\isasymlbrakk}\ xs\ {\isacharat}\ zs\ {\isacharequal}\ ys\ {\isacharat}\ xs{\isacharsemicolon}\ {\isacharbrackleft}{\isacharbrackright}\ {\isacharat}\ xs\ {\isacharequal}\ {\isacharbrackleft}{\isacharbrackright}\ {\isacharat}\ {\isacharbrackleft}{\isacharbrackright}\ {\isasymrbrakk}\ {\isasymLongrightarrow}\ ys\ {\isacharequal}\ zs{\isachardoublequote}\isanewline
\isacommand{apply}\ simp\isanewline
\isacommand{done}%
\begin{isamarkuptext}%
\noindent
The second assumption simplifies to \isa{xs\ {\isacharequal}\ {\isacharbrackleft}{\isacharbrackright}}, which in turn
simplifies the first assumption to \isa{zs\ {\isacharequal}\ ys}, thus reducing the
conclusion to \isa{ys\ {\isacharequal}\ ys} and hence to \isa{True}.

In some cases this may be too much of a good thing and may lead to
nontermination:%
\end{isamarkuptext}%
\isacommand{lemma}\ {\isachardoublequote}{\isasymforall}x{\isachardot}\ f\ x\ {\isacharequal}\ g\ {\isacharparenleft}f\ {\isacharparenleft}g\ x{\isacharparenright}{\isacharparenright}\ {\isasymLongrightarrow}\ f\ {\isacharbrackleft}{\isacharbrackright}\ {\isacharequal}\ f\ {\isacharbrackleft}{\isacharbrackright}\ {\isacharat}\ {\isacharbrackleft}{\isacharbrackright}{\isachardoublequote}%
\begin{isamarkuptxt}%
\noindent
cannot be solved by an unmodified application of \isa{simp} because the
simplification rule \isa{f\ x\ {\isacharequal}\ g\ {\isacharparenleft}f\ {\isacharparenleft}g\ x{\isacharparenright}{\isacharparenright}} extracted from the assumption
does not terminate. Isabelle notices certain simple forms of
nontermination but not this one. The problem can be circumvented by
explicitly telling the simplifier to ignore the assumptions:%
\end{isamarkuptxt}%
\isacommand{apply}{\isacharparenleft}simp\ {\isacharparenleft}no{\isacharunderscore}asm{\isacharparenright}{\isacharparenright}\isanewline
\isacommand{done}%
\begin{isamarkuptext}%
\noindent
There are three options that influence the treatment of assumptions:
\begin{description}
\item[\isa{{\isacharparenleft}no{\isacharunderscore}asm{\isacharparenright}}]\indexbold{*no_asm}
 means that assumptions are completely ignored.
\item[\isa{{\isacharparenleft}no{\isacharunderscore}asm{\isacharunderscore}simp{\isacharparenright}}]\indexbold{*no_asm_simp}
 means that the assumptions are not simplified but
  are used in the simplification of the conclusion.
\item[\isa{{\isacharparenleft}no{\isacharunderscore}asm{\isacharunderscore}use{\isacharparenright}}]\indexbold{*no_asm_use}
 means that the assumptions are simplified but are not
  used in the simplification of each other or the conclusion.
\end{description}
Neither \isa{{\isacharparenleft}no{\isacharunderscore}asm{\isacharunderscore}simp{\isacharparenright}} nor \isa{{\isacharparenleft}no{\isacharunderscore}asm{\isacharunderscore}use{\isacharparenright}} allow to simplify
the above problematic subgoal.

Note that only one of the above options is allowed, and it must precede all
other arguments.%
\end{isamarkuptext}%
%
\isamarkupsubsubsection{Rewriting with definitions%
}
%
\begin{isamarkuptext}%
\index{simplification!with definitions}
Constant definitions (\S\ref{sec:ConstDefinitions}) can
be used as simplification rules, but by default they are not.  Hence the
simplifier does not expand them automatically, just as it should be:
definitions are introduced for the purpose of abbreviating complex
concepts. Of course we need to expand the definitions initially to derive
enough lemmas that characterize the concept sufficiently for us to forget the
original definition. For example, given%
\end{isamarkuptext}%
\isacommand{constdefs}\ exor\ {\isacharcolon}{\isacharcolon}\ {\isachardoublequote}bool\ {\isasymRightarrow}\ bool\ {\isasymRightarrow}\ bool{\isachardoublequote}\isanewline
\ \ \ \ \ \ \ \ \ {\isachardoublequote}exor\ A\ B\ {\isasymequiv}\ {\isacharparenleft}A\ {\isasymand}\ {\isasymnot}B{\isacharparenright}\ {\isasymor}\ {\isacharparenleft}{\isasymnot}A\ {\isasymand}\ B{\isacharparenright}{\isachardoublequote}%
\begin{isamarkuptext}%
\noindent
we may want to prove%
\end{isamarkuptext}%
\isacommand{lemma}\ {\isachardoublequote}exor\ A\ {\isacharparenleft}{\isasymnot}A{\isacharparenright}{\isachardoublequote}%
\begin{isamarkuptxt}%
\noindent
Typically, the opening move consists in \emph{unfolding} the definition(s), which we need to
get started, but nothing else:\indexbold{*unfold}\indexbold{definition!unfolding}%
\end{isamarkuptxt}%
\isacommand{apply}{\isacharparenleft}simp\ only{\isacharcolon}exor{\isacharunderscore}def{\isacharparenright}%
\begin{isamarkuptxt}%
\noindent
In this particular case, the resulting goal
\begin{isabelle}%
\ {\isadigit{1}}{\isachardot}\ A\ {\isasymand}\ {\isasymnot}\ {\isasymnot}\ A\ {\isasymor}\ {\isasymnot}\ A\ {\isasymand}\ {\isasymnot}\ A%
\end{isabelle}
can be proved by simplification. Thus we could have proved the lemma outright by%
\end{isamarkuptxt}%
\isacommand{apply}{\isacharparenleft}simp\ add{\isacharcolon}\ exor{\isacharunderscore}def{\isacharparenright}%
\begin{isamarkuptext}%
\noindent
Of course we can also unfold definitions in the middle of a proof.

You should normally not turn a definition permanently into a simplification
rule because this defeats the whole purpose of an abbreviation.

\begin{warn}
  If you have defined $f\,x\,y~\isasymequiv~t$ then you can only expand
  occurrences of $f$ with at least two arguments. Thus it is safer to define
  $f$~\isasymequiv~\isasymlambda$x\,y.\;t$.
\end{warn}%
\end{isamarkuptext}%
%
\isamarkupsubsubsection{Simplifying let-expressions%
}
%
\begin{isamarkuptext}%
\index{simplification!of let-expressions}
Proving a goal containing \isaindex{let}-expressions almost invariably
requires the \isa{let}-con\-structs to be expanded at some point. Since
\isa{let}-\isa{in} is just syntactic sugar for a predefined constant
(called \isa{Let}), expanding \isa{let}-constructs means rewriting with
\isa{Let{\isacharunderscore}def}:%
\end{isamarkuptext}%
\isacommand{lemma}\ {\isachardoublequote}{\isacharparenleft}let\ xs\ {\isacharequal}\ {\isacharbrackleft}{\isacharbrackright}\ in\ xs{\isacharat}ys{\isacharat}xs{\isacharparenright}\ {\isacharequal}\ ys{\isachardoublequote}\isanewline
\isacommand{apply}{\isacharparenleft}simp\ add{\isacharcolon}\ Let{\isacharunderscore}def{\isacharparenright}\isanewline
\isacommand{done}%
\begin{isamarkuptext}%
If, in a particular context, there is no danger of a combinatorial explosion
of nested \isa{let}s one could even simlify with \isa{Let{\isacharunderscore}def} by
default:%
\end{isamarkuptext}%
\isacommand{declare}\ Let{\isacharunderscore}def\ {\isacharbrackleft}simp{\isacharbrackright}%
\isamarkupsubsubsection{Conditional equations%
}
%
\begin{isamarkuptext}%
So far all examples of rewrite rules were equations. The simplifier also
accepts \emph{conditional} equations, for example%
\end{isamarkuptext}%
\isacommand{lemma}\ hd{\isacharunderscore}Cons{\isacharunderscore}tl{\isacharbrackleft}simp{\isacharbrackright}{\isacharcolon}\ {\isachardoublequote}xs\ {\isasymnoteq}\ {\isacharbrackleft}{\isacharbrackright}\ \ {\isasymLongrightarrow}\ \ hd\ xs\ {\isacharhash}\ tl\ xs\ {\isacharequal}\ xs{\isachardoublequote}\isanewline
\isacommand{apply}{\isacharparenleft}case{\isacharunderscore}tac\ xs{\isacharcomma}\ simp{\isacharcomma}\ simp{\isacharparenright}\isanewline
\isacommand{done}%
\begin{isamarkuptext}%
\noindent
Note the use of ``\ttindexboldpos{,}{$Isar}'' to string together a
sequence of methods. Assuming that the simplification rule
\isa{{\isacharparenleft}rev\ xs\ {\isacharequal}\ {\isacharbrackleft}{\isacharbrackright}{\isacharparenright}\ {\isacharequal}\ {\isacharparenleft}xs\ {\isacharequal}\ {\isacharbrackleft}{\isacharbrackright}{\isacharparenright}}
is present as well,%
\end{isamarkuptext}%
\isacommand{lemma}\ {\isachardoublequote}xs\ {\isasymnoteq}\ {\isacharbrackleft}{\isacharbrackright}\ {\isasymLongrightarrow}\ hd{\isacharparenleft}rev\ xs{\isacharparenright}\ {\isacharhash}\ tl{\isacharparenleft}rev\ xs{\isacharparenright}\ {\isacharequal}\ rev\ xs{\isachardoublequote}%
\begin{isamarkuptext}%
\noindent
is proved by plain simplification:
the conditional equation \isa{hd{\isacharunderscore}Cons{\isacharunderscore}tl} above
can simplify \isa{hd\ {\isacharparenleft}rev\ xs{\isacharparenright}\ {\isacharhash}\ tl\ {\isacharparenleft}rev\ xs{\isacharparenright}} to \isa{rev\ xs}
because the corresponding precondition \isa{rev\ xs\ {\isasymnoteq}\ {\isacharbrackleft}{\isacharbrackright}}
simplifies to \isa{xs\ {\isasymnoteq}\ {\isacharbrackleft}{\isacharbrackright}}, which is exactly the local
assumption of the subgoal.%
\end{isamarkuptext}%
%
\isamarkupsubsubsection{Automatic case splits%
}
%
\begin{isamarkuptext}%
\indexbold{case splits}\index{*split|(}
Goals containing \isa{if}-expressions are usually proved by case
distinction on the condition of the \isa{if}. For example the goal%
\end{isamarkuptext}%
\isacommand{lemma}\ {\isachardoublequote}{\isasymforall}xs{\isachardot}\ if\ xs\ {\isacharequal}\ {\isacharbrackleft}{\isacharbrackright}\ then\ rev\ xs\ {\isacharequal}\ {\isacharbrackleft}{\isacharbrackright}\ else\ rev\ xs\ {\isasymnoteq}\ {\isacharbrackleft}{\isacharbrackright}{\isachardoublequote}%
\begin{isamarkuptxt}%
\noindent
can be split by a degenerate form of simplification%
\end{isamarkuptxt}%
\isacommand{apply}{\isacharparenleft}simp\ only{\isacharcolon}\ split{\isacharcolon}\ split{\isacharunderscore}if{\isacharparenright}%
\begin{isamarkuptxt}%
\noindent
\begin{isabelle}%
\ {\isadigit{1}}{\isachardot}\ {\isasymforall}xs{\isachardot}\ {\isacharparenleft}xs\ {\isacharequal}\ {\isacharbrackleft}{\isacharbrackright}\ {\isasymlongrightarrow}\ rev\ xs\ {\isacharequal}\ {\isacharbrackleft}{\isacharbrackright}{\isacharparenright}\ {\isasymand}\ {\isacharparenleft}xs\ {\isasymnoteq}\ {\isacharbrackleft}{\isacharbrackright}\ {\isasymlongrightarrow}\ rev\ xs\ {\isasymnoteq}\ {\isacharbrackleft}{\isacharbrackright}{\isacharparenright}%
\end{isabelle}
where no simplification rules are included (\isa{only{\isacharcolon}} is followed by the
empty list of theorems) but the rule \isaindexbold{split_if} for
splitting \isa{if}s is added (via the modifier \isa{split{\isacharcolon}}). Because
case-splitting on \isa{if}s is almost always the right proof strategy, the
simplifier performs it automatically. Try \isacommand{apply}\isa{{\isacharparenleft}simp{\isacharparenright}}
on the initial goal above.

This splitting idea generalizes from \isa{if} to \isaindex{case}:%
\end{isamarkuptxt}%
\isanewline
\isacommand{lemma}\ {\isachardoublequote}{\isacharparenleft}case\ xs\ of\ {\isacharbrackleft}{\isacharbrackright}\ {\isasymRightarrow}\ zs\ {\isacharbar}\ y{\isacharhash}ys\ {\isasymRightarrow}\ y{\isacharhash}{\isacharparenleft}ys{\isacharat}zs{\isacharparenright}{\isacharparenright}\ {\isacharequal}\ xs{\isacharat}zs{\isachardoublequote}\isanewline
\isacommand{apply}{\isacharparenleft}simp\ only{\isacharcolon}\ split{\isacharcolon}\ list{\isachardot}split{\isacharparenright}%
\begin{isamarkuptxt}%
\begin{isabelle}%
\ {\isadigit{1}}{\isachardot}\ {\isacharparenleft}xs\ {\isacharequal}\ {\isacharbrackleft}{\isacharbrackright}\ {\isasymlongrightarrow}\ zs\ {\isacharequal}\ xs\ {\isacharat}\ zs{\isacharparenright}\ {\isasymand}\isanewline
\ \ \ \ {\isacharparenleft}{\isasymforall}a\ list{\isachardot}\ xs\ {\isacharequal}\ a\ {\isacharhash}\ list\ {\isasymlongrightarrow}\ a\ {\isacharhash}\ list\ {\isacharat}\ zs\ {\isacharequal}\ xs\ {\isacharat}\ zs{\isacharparenright}%
\end{isabelle}
In contrast to \isa{if}-expressions, the simplifier does not split
\isa{case}-expressions by default because this can lead to nontermination
in case of recursive datatypes. Again, if the \isa{only{\isacharcolon}} modifier is
dropped, the above goal is solved,%
\end{isamarkuptxt}%
\isacommand{apply}{\isacharparenleft}simp\ split{\isacharcolon}\ list{\isachardot}split{\isacharparenright}%
\begin{isamarkuptext}%
\noindent%
which \isacommand{apply}\isa{{\isacharparenleft}simp{\isacharparenright}} alone will not do.

In general, every datatype $t$ comes with a theorem
$t$\isa{{\isachardot}split} which can be declared to be a \bfindex{split rule} either
locally as above, or by giving it the \isa{split} attribute globally:%
\end{isamarkuptext}%
\isacommand{declare}\ list{\isachardot}split\ {\isacharbrackleft}split{\isacharbrackright}%
\begin{isamarkuptext}%
\noindent
The \isa{split} attribute can be removed with the \isa{del} modifier,
either locally%
\end{isamarkuptext}%
\isacommand{apply}{\isacharparenleft}simp\ split\ del{\isacharcolon}\ split{\isacharunderscore}if{\isacharparenright}%
\begin{isamarkuptext}%
\noindent
or globally:%
\end{isamarkuptext}%
\isacommand{declare}\ list{\isachardot}split\ {\isacharbrackleft}split\ del{\isacharbrackright}%
\begin{isamarkuptext}%
The above split rules intentionally only affect the conclusion of a
subgoal.  If you want to split an \isa{if} or \isa{case}-expression in
the assumptions, you have to apply \isa{split{\isacharunderscore}if{\isacharunderscore}asm} or
$t$\isa{{\isachardot}split{\isacharunderscore}asm}:%
\end{isamarkuptext}%
\isacommand{lemma}\ {\isachardoublequote}if\ xs\ {\isacharequal}\ {\isacharbrackleft}{\isacharbrackright}\ then\ ys\ {\isachartilde}{\isacharequal}\ {\isacharbrackleft}{\isacharbrackright}\ else\ ys\ {\isacharequal}\ {\isacharbrackleft}{\isacharbrackright}\ {\isacharequal}{\isacharequal}{\isachargreater}\ xs\ {\isacharat}\ ys\ {\isachartilde}{\isacharequal}\ {\isacharbrackleft}{\isacharbrackright}{\isachardoublequote}\isanewline
\isacommand{apply}{\isacharparenleft}simp\ only{\isacharcolon}\ split{\isacharcolon}\ split{\isacharunderscore}if{\isacharunderscore}asm{\isacharparenright}%
\begin{isamarkuptxt}%
\noindent
In contrast to splitting the conclusion, this actually creates two
separate subgoals (which are solved by \isa{simp{\isacharunderscore}all}):
\begin{isabelle}%
\ {\isadigit{1}}{\isachardot}\ {\isasymlbrakk}xs\ {\isacharequal}\ {\isacharbrackleft}{\isacharbrackright}{\isacharsemicolon}\ ys\ {\isasymnoteq}\ {\isacharbrackleft}{\isacharbrackright}{\isasymrbrakk}\ {\isasymLongrightarrow}\ {\isacharbrackleft}{\isacharbrackright}\ {\isacharat}\ ys\ {\isasymnoteq}\ {\isacharbrackleft}{\isacharbrackright}\isanewline
\ {\isadigit{2}}{\isachardot}\ {\isasymlbrakk}xs\ {\isasymnoteq}\ {\isacharbrackleft}{\isacharbrackright}{\isacharsemicolon}\ ys\ {\isacharequal}\ {\isacharbrackleft}{\isacharbrackright}{\isasymrbrakk}\ {\isasymLongrightarrow}\ xs\ {\isacharat}\ {\isacharbrackleft}{\isacharbrackright}\ {\isasymnoteq}\ {\isacharbrackleft}{\isacharbrackright}%
\end{isabelle}
If you need to split both in the assumptions and the conclusion,
use $t$\isa{{\isachardot}splits} which subsumes $t$\isa{{\isachardot}split} and
$t$\isa{{\isachardot}split{\isacharunderscore}asm}. Analogously, there is \isa{if{\isacharunderscore}splits}.

\begin{warn}
  The simplifier merely simplifies the condition of an \isa{if} but not the
  \isa{then} or \isa{else} parts. The latter are simplified only after the
  condition reduces to \isa{True} or \isa{False}, or after splitting. The
  same is true for \isaindex{case}-expressions: only the selector is
  simplified at first, until either the expression reduces to one of the
  cases or it is split.
\end{warn}

\index{*split|)}%
\end{isamarkuptxt}%
%
\isamarkupsubsubsection{Arithmetic%
}
%
\begin{isamarkuptext}%
\index{arithmetic}
The simplifier routinely solves a small class of linear arithmetic formulae
(over type \isa{nat} and other numeric types): it only takes into account
assumptions and conclusions that are (possibly negated) (in)equalities
(\isa{{\isacharequal}}, \isasymle, \isa{{\isacharless}}) and it only knows about addition. Thus%
\end{isamarkuptext}%
\isacommand{lemma}\ {\isachardoublequote}{\isasymlbrakk}\ {\isasymnot}\ m\ {\isacharless}\ n{\isacharsemicolon}\ m\ {\isacharless}\ n{\isacharplus}{\isadigit{1}}\ {\isasymrbrakk}\ {\isasymLongrightarrow}\ m\ {\isacharequal}\ n{\isachardoublequote}%
\begin{isamarkuptext}%
\noindent
is proved by simplification, whereas the only slightly more complex%
\end{isamarkuptext}%
\isacommand{lemma}\ {\isachardoublequote}{\isasymnot}\ m\ {\isacharless}\ n\ {\isasymand}\ m\ {\isacharless}\ n{\isacharplus}{\isadigit{1}}\ {\isasymLongrightarrow}\ m\ {\isacharequal}\ n{\isachardoublequote}%
\begin{isamarkuptext}%
\noindent
is not proved by simplification and requires \isa{arith}.%
\end{isamarkuptext}%
%
\isamarkupsubsubsection{Tracing%
}
%
\begin{isamarkuptext}%
\indexbold{tracing the simplifier}
Using the simplifier effectively may take a bit of experimentation.  Set the
\isaindexbold{trace_simp} \rmindex{flag} to get a better idea of what is going
on:%
\end{isamarkuptext}%
\isacommand{ML}\ {\isachardoublequote}set\ trace{\isacharunderscore}simp{\isachardoublequote}\isanewline
\isacommand{lemma}\ {\isachardoublequote}rev\ {\isacharbrackleft}a{\isacharbrackright}\ {\isacharequal}\ {\isacharbrackleft}{\isacharbrackright}{\isachardoublequote}\isanewline
\isacommand{apply}{\isacharparenleft}simp{\isacharparenright}%
\begin{isamarkuptext}%
\noindent
produces the trace

\begin{ttbox}\makeatother
Applying instance of rewrite rule:
rev (?x1 \# ?xs1) == rev ?xs1 @ [?x1]
Rewriting:
rev [x] == rev [] @ [x]
Applying instance of rewrite rule:
rev [] == []
Rewriting:
rev [] == []
Applying instance of rewrite rule:
[] @ ?y == ?y
Rewriting:
[] @ [x] == [x]
Applying instance of rewrite rule:
?x3 \# ?t3 = ?t3 == False
Rewriting:
[x] = [] == False
\end{ttbox}

In more complicated cases, the trace can be quite lenghty, especially since
invocations of the simplifier are often nested (e.g.\ when solving conditions
of rewrite rules). Thus it is advisable to reset it:%
\end{isamarkuptext}%
\isacommand{ML}\ {\isachardoublequote}reset\ trace{\isacharunderscore}simp{\isachardoublequote}\isanewline
\end{isabellebody}%
%%% Local Variables:
%%% mode: latex
%%% TeX-master: "root"
%%% End:


\section{Advanced forms of recursion}
\index{*recdef|(}

The purpose of this section is to introduce advanced forms of
\isacommand{recdef}: how to establish termination by means other than measure
functions, how to define recursive function over nested recursive datatypes,
and how to deal with partial functions.

If, after reading this section, you feel that the definition of recursive
functions is overly complicated by the requirement of
totality, you should ponder the alternative, a logic of partial functions,
where recursive definitions are always wellformed. For a start, there are many
such logics, and no clear winner has emerged. And in all of these logics you
are (more or less frequently) required to reason about the definedness of
terms explicitly. Thus one shifts definedness arguments from definition time to
proof time. In HOL you may have to work hard to define a function, but proofs
can then proceed unencumbered by worries about undefinedness.

\subsection{Beyond measure}
\label{sec:beyond-measure}
%
\begin{isabellebody}%
\def\isabellecontext{WFrec}%
%
\begin{isamarkuptext}%
\noindent
So far, all recursive definitions where shown to terminate via measure
functions. Sometimes this can be quite inconvenient or even
impossible. Fortunately, \isacommand{recdef} supports much more
general definitions. For example, termination of Ackermann's function
can be shown by means of the lexicographic product \isa{{\isacharless}{\isacharasterisk}lex{\isacharasterisk}{\isachargreater}}:%
\end{isamarkuptext}%
\isacommand{consts}\ ack\ {\isacharcolon}{\isacharcolon}\ {\isachardoublequote}nat{\isasymtimes}nat\ {\isasymRightarrow}\ nat{\isachardoublequote}\isanewline
\isacommand{recdef}\ ack\ {\isachardoublequote}measure{\isacharparenleft}{\isasymlambda}m{\isachardot}\ m{\isacharparenright}\ {\isacharless}{\isacharasterisk}lex{\isacharasterisk}{\isachargreater}\ measure{\isacharparenleft}{\isasymlambda}n{\isachardot}\ n{\isacharparenright}{\isachardoublequote}\isanewline
\ \ {\isachardoublequote}ack{\isacharparenleft}{\isadigit{0}}{\isacharcomma}n{\isacharparenright}\ \ \ \ \ \ \ \ \ {\isacharequal}\ Suc\ n{\isachardoublequote}\isanewline
\ \ {\isachardoublequote}ack{\isacharparenleft}Suc\ m{\isacharcomma}{\isadigit{0}}{\isacharparenright}\ \ \ \ \ {\isacharequal}\ ack{\isacharparenleft}m{\isacharcomma}\ {\isadigit{1}}{\isacharparenright}{\isachardoublequote}\isanewline
\ \ {\isachardoublequote}ack{\isacharparenleft}Suc\ m{\isacharcomma}Suc\ n{\isacharparenright}\ {\isacharequal}\ ack{\isacharparenleft}m{\isacharcomma}ack{\isacharparenleft}Suc\ m{\isacharcomma}n{\isacharparenright}{\isacharparenright}{\isachardoublequote}%
\begin{isamarkuptext}%
\noindent
The lexicographic product decreases if either its first component
decreases (as in the second equation and in the outer call in the
third equation) or its first component stays the same and the second
component decreases (as in the inner call in the third equation).

In general, \isacommand{recdef} supports termination proofs based on
arbitrary \emph{wellfounded relations}, i.e.\ \emph{wellfounded
recursion}\indexbold{recursion!wellfounded}\index{wellfounded
recursion|see{recursion, wellfounded}}.  A relation $<$ is
\bfindex{wellfounded} if it has no infinite descending chain $\cdots <
a@2 < a@1 < a@0$. Clearly, a function definition is total iff the set
of all pairs $(r,l)$, where $l$ is the argument on the left-hand side
of an equation and $r$ the argument of some recursive call on the
corresponding right-hand side, induces a wellfounded relation.  For a
systematic account of termination proofs via wellfounded relations
see, for example, \cite{Baader-Nipkow}. The HOL library formalizes
some of the theory of wellfounded relations. For example
\isa{wf\ r}\index{*wf|bold} means that relation \isa{r{\isasymColon}{\isacharparenleft}{\isacharprime}a\ {\isasymtimes}\ {\isacharprime}a{\isacharparenright}\ set} is
wellfounded.

Each \isacommand{recdef} definition should be accompanied (after the
name of the function) by a wellfounded relation on the argument type
of the function. For example, \isa{measure} is defined by
\begin{isabelle}%
\ \ \ \ \ measure\ f\ {\isasymequiv}\ {\isacharbraceleft}{\isacharparenleft}y{\isacharcomma}\ x{\isacharparenright}{\isachardot}\ f\ y\ {\isacharless}\ f\ x{\isacharbraceright}%
\end{isabelle}
and it has been proved that \isa{measure\ f} is always wellfounded.

In addition to \isa{measure}, the library provides
a number of further constructions for obtaining wellfounded relations.
Above we have already met \isa{{\isacharless}{\isacharasterisk}lex{\isacharasterisk}{\isachargreater}} of type
\begin{isabelle}%
\ \ \ \ \ {\isachardoublequote}{\isacharparenleft}{\isacharprime}a\ {\isasymtimes}\ {\isacharprime}a{\isacharparenright}set\ {\isasymRightarrow}\ {\isacharparenleft}{\isacharprime}b\ {\isasymtimes}\ {\isacharprime}b{\isacharparenright}set\ {\isasymRightarrow}\ {\isacharparenleft}{\isacharparenleft}{\isacharprime}a\ {\isasymtimes}\ {\isacharprime}b{\isacharparenright}\ {\isasymtimes}\ {\isacharparenleft}{\isacharprime}a\ {\isasymtimes}\ {\isacharprime}b{\isacharparenright}{\isacharparenright}set{\isachardoublequote}%
\end{isabelle}
Of course the lexicographic product can also be interated, as in the following
function definition:%
\end{isamarkuptext}%
\isacommand{consts}\ contrived\ {\isacharcolon}{\isacharcolon}\ {\isachardoublequote}nat\ {\isasymtimes}\ nat\ {\isasymtimes}\ nat\ {\isasymRightarrow}\ nat{\isachardoublequote}\isanewline
\isacommand{recdef}\ contrived\isanewline
\ \ {\isachardoublequote}measure{\isacharparenleft}{\isasymlambda}i{\isachardot}\ i{\isacharparenright}\ {\isacharless}{\isacharasterisk}lex{\isacharasterisk}{\isachargreater}\ measure{\isacharparenleft}{\isasymlambda}j{\isachardot}\ j{\isacharparenright}\ {\isacharless}{\isacharasterisk}lex{\isacharasterisk}{\isachargreater}\ measure{\isacharparenleft}{\isasymlambda}k{\isachardot}\ k{\isacharparenright}{\isachardoublequote}\isanewline
{\isachardoublequote}contrived{\isacharparenleft}i{\isacharcomma}j{\isacharcomma}Suc\ k{\isacharparenright}\ {\isacharequal}\ contrived{\isacharparenleft}i{\isacharcomma}j{\isacharcomma}k{\isacharparenright}{\isachardoublequote}\isanewline
{\isachardoublequote}contrived{\isacharparenleft}i{\isacharcomma}Suc\ j{\isacharcomma}{\isadigit{0}}{\isacharparenright}\ {\isacharequal}\ contrived{\isacharparenleft}i{\isacharcomma}j{\isacharcomma}j{\isacharparenright}{\isachardoublequote}\isanewline
{\isachardoublequote}contrived{\isacharparenleft}Suc\ i{\isacharcomma}{\isadigit{0}}{\isacharcomma}{\isadigit{0}}{\isacharparenright}\ {\isacharequal}\ contrived{\isacharparenleft}i{\isacharcomma}i{\isacharcomma}i{\isacharparenright}{\isachardoublequote}\isanewline
{\isachardoublequote}contrived{\isacharparenleft}{\isadigit{0}}{\isacharcomma}{\isadigit{0}}{\isacharcomma}{\isadigit{0}}{\isacharparenright}\ \ \ \ \ {\isacharequal}\ {\isadigit{0}}{\isachardoublequote}%
\begin{isamarkuptext}%
Lexicographic products of measure functions already go a long way. A
further useful construction is the embedding of some type in an
existing wellfounded relation via the inverse image of a function:
\begin{isabelle}%
\ \ \ \ \ inv{\isacharunderscore}image\ {\isacharparenleft}r{\isasymColon}{\isacharparenleft}{\isacharprime}b\ {\isasymtimes}\ {\isacharprime}b{\isacharparenright}\ set{\isacharparenright}\ {\isacharparenleft}f{\isasymColon}{\isacharprime}a\ {\isasymRightarrow}\ {\isacharprime}b{\isacharparenright}\ {\isasymequiv}\isanewline
\ \ \ \ \ {\isacharbraceleft}{\isacharparenleft}x{\isasymColon}{\isacharprime}a{\isacharcomma}\ y{\isasymColon}{\isacharprime}a{\isacharparenright}{\isachardot}\ {\isacharparenleft}f\ x{\isacharcomma}\ f\ y{\isacharparenright}\ {\isasymin}\ r{\isacharbraceright}%
\end{isabelle}
\begin{sloppypar}
\noindent
For example, \isa{measure} is actually defined as \isa{inv{\isacharunderscore}mage\ less{\isacharunderscore}than}, where
\isa{less{\isacharunderscore}than} of type \isa{{\isacharparenleft}nat\ {\isasymtimes}\ nat{\isacharparenright}\ set} is the less-than relation on type \isa{nat}
(as opposed to \isa{op\ {\isacharless}}, which is of type \isa{{\isacharbrackleft}nat{\isacharcomma}\ nat{\isacharbrackright}\ {\isasymRightarrow}\ bool}).
\end{sloppypar}

%Finally there is also {finite_psubset} the proper subset relation on finite sets

All the above constructions are known to \isacommand{recdef}. Thus you
will never have to prove wellfoundedness of any relation composed
solely of these building blocks. But of course the proof of
termination of your function definition, i.e.\ that the arguments
decrease with every recursive call, may still require you to provide
additional lemmas.

It is also possible to use your own wellfounded relations with \isacommand{recdef}.
Here is a simplistic example:%
\end{isamarkuptext}%
\isacommand{consts}\ f\ {\isacharcolon}{\isacharcolon}\ {\isachardoublequote}nat\ {\isasymRightarrow}\ nat{\isachardoublequote}\isanewline
\isacommand{recdef}\ f\ {\isachardoublequote}id{\isacharparenleft}less{\isacharunderscore}than{\isacharparenright}{\isachardoublequote}\isanewline
{\isachardoublequote}f\ {\isadigit{0}}\ {\isacharequal}\ {\isadigit{0}}{\isachardoublequote}\isanewline
{\isachardoublequote}f\ {\isacharparenleft}Suc\ n{\isacharparenright}\ {\isacharequal}\ f\ n{\isachardoublequote}%
\begin{isamarkuptext}%
Since \isacommand{recdef} is not prepared for \isa{id}, the identity
function, this leads to the complaint that it could not prove
\isa{wf\ {\isacharparenleft}id\ less{\isacharunderscore}than{\isacharparenright}}, the wellfoundedness of \isa{id\ less{\isacharunderscore}than}. We should first have proved that \isa{id} preserves wellfoundedness%
\end{isamarkuptext}%
\isacommand{lemma}\ wf{\isacharunderscore}id{\isacharcolon}\ {\isachardoublequote}wf\ r\ {\isasymLongrightarrow}\ wf{\isacharparenleft}id\ r{\isacharparenright}{\isachardoublequote}\isanewline
\isacommand{by}\ simp%
\begin{isamarkuptext}%
\noindent
and should have added the following hint to our above definition:%
\end{isamarkuptext}%
{\isacharparenleft}\isakeyword{hints}\ recdef{\isacharunderscore}wf\ add{\isacharcolon}\ wf{\isacharunderscore}id{\isacharparenright}\end{isabellebody}%
%%% Local Variables:
%%% mode: latex
%%% TeX-master: "root"
%%% End:


\subsection{Recursion over nested datatypes}
\label{sec:nested-recdef}
%
\begin{isabellebody}%
\def\isabellecontext{Nested{\isadigit{0}}}%
%
\begin{isamarkuptext}%
\index{datatypes!nested}%
In \S\ref{sec:nested-datatype} we defined the datatype of terms%
\end{isamarkuptext}%
\isacommand{datatype}\ {\isacharparenleft}{\isacharprime}a{\isacharcomma}{\isacharprime}b{\isacharparenright}{\isachardoublequote}term{\isachardoublequote}\ {\isacharequal}\ Var\ {\isacharprime}a\ {\isacharbar}\ App\ {\isacharprime}b\ {\isachardoublequote}{\isacharparenleft}{\isacharprime}a{\isacharcomma}{\isacharprime}b{\isacharparenright}term\ list{\isachardoublequote}%
\begin{isamarkuptext}%
\noindent
and closed with the observation that the associated schema for the definition
of primitive recursive functions leads to overly verbose definitions. Moreover,
if you have worked exercise~\ref{ex:trev-trev} you will have noticed that
you needed to declare essentially the same function as \isa{rev}
and prove many standard properties of list reversal all over again. 
We will now show you how \isacommand{recdef} can simplify
definitions and proofs about nested recursive datatypes. As an example we
choose exercise~\ref{ex:trev-trev}:%
\end{isamarkuptext}%
\isacommand{consts}\ trev\ \ {\isacharcolon}{\isacharcolon}\ {\isachardoublequote}{\isacharparenleft}{\isacharprime}a{\isacharcomma}{\isacharprime}b{\isacharparenright}term\ {\isasymRightarrow}\ {\isacharparenleft}{\isacharprime}a{\isacharcomma}{\isacharprime}b{\isacharparenright}term{\isachardoublequote}\end{isabellebody}%
%%% Local Variables:
%%% mode: latex
%%% TeX-master: "root"
%%% End:

\begin{isabelle}%
\isacommand{consts}\ trev\ \ {\isacharcolon}{\isacharcolon}\ {\isachardoublequote}{\isacharparenleft}{\isacharprime}a{\isacharcomma}{\isacharprime}b{\isacharparenright}term\ {\isacharequal}{\isachargreater}\ {\isacharparenleft}{\isacharprime}a{\isacharcomma}{\isacharprime}b{\isacharparenright}term{\isachardoublequote}%
\begin{isamarkuptext}%
\noindent
Although the definition of \isa{trev} is quite natural, we will have
overcome a minor difficulty in convincing Isabelle of is termination.
It is precisely this difficulty that is the \textit{rasion d'\^etre} of
this subsection.

Defining \isa{trev} by \isacommand{recdef} rather than \isacommand{primrec}
simplifies matters because we are now free to use the recursion equation
suggested at the end of \S\ref{sec:nested-datatype}:%
\end{isamarkuptext}%
\isacommand{recdef}\ trev\ {\isachardoublequote}measure\ size{\isachardoublequote}\isanewline
\ {\isachardoublequote}trev\ {\isacharparenleft}Var\ x{\isacharparenright}\ {\isacharequal}\ Var\ x{\isachardoublequote}\isanewline
\ {\isachardoublequote}trev\ {\isacharparenleft}App\ f\ ts{\isacharparenright}\ {\isacharequal}\ App\ f\ {\isacharparenleft}rev{\isacharparenleft}map\ trev\ ts{\isacharparenright}{\isacharparenright}{\isachardoublequote}%
\begin{isamarkuptext}%
FIXME: recdef should complain and generate unprovable termination condition!
moveto todo

Remember that function \isa{size} is defined for each \isacommand{datatype}.
However, the definition does not succeed. Isabelle complains about an unproved termination
condition
\begin{quote}

\begin{isabelle}%
\mbox{t}\ {\isasymin}\ set\ \mbox{ts}\ {\isasymlongrightarrow}\ size\ \mbox{t}\ {\isacharless}\ Suc\ {\isacharparenleft}term{\isacharunderscore}size\ \mbox{ts}{\isacharparenright}
\end{isabelle}%

\end{quote}
where \isa{set} returns the set of elements of a list---no special knowledge of sets is
required in the following.
First we have to understand why the recursive call of \isa{trev} underneath \isa{map} leads
to the above condition. The reason is that \isacommand{recdef} ``knows'' that \isa{map} will
apply \isa{trev} only to elements of \isa{\mbox{ts}}. Thus the above condition expresses that
the size of the argument \isa{\mbox{t}\ {\isasymin}\ set\ \mbox{ts}} of any recursive call of \isa{trev} is strictly
less than \isa{size\ {\isacharparenleft}App\ \mbox{f}\ \mbox{ts}{\isacharparenright}\ {\isacharequal}\ Suc\ {\isacharparenleft}term{\isacharunderscore}size\ \mbox{ts}{\isacharparenright}}.
We will now prove the termination condition and continue with our definition.
Below we return to the question of how \isacommand{recdef} ``knows'' about \isa{map}.%
\end{isamarkuptext}%
\end{isabelle}%
%%% Local Variables:
%%% mode: latex
%%% TeX-master: "root"
%%% End:

%
\begin{isabellebody}%
\def\isabellecontext{Nested{\isadigit{2}}}%
%
\begin{isamarkuptext}%
\noindent
The termintion condition is easily proved by induction:%
\end{isamarkuptext}%
\isacommand{lemma}\ {\isacharbrackleft}simp{\isacharbrackright}{\isacharcolon}\ {\isachardoublequote}t\ {\isasymin}\ set\ ts\ {\isasymlongrightarrow}\ size\ t\ {\isacharless}\ Suc{\isacharparenleft}term{\isacharunderscore}list{\isacharunderscore}size\ ts{\isacharparenright}{\isachardoublequote}\isanewline
\isacommand{by}{\isacharparenleft}induct{\isacharunderscore}tac\ ts{\isacharcomma}\ auto{\isacharparenright}%
\begin{isamarkuptext}%
\noindent
By making this theorem a simplification rule, \isacommand{recdef}
applies it automatically and the definition of \isa{trev}
succeeds now. As a reward for our effort, we can now prove the desired
lemma directly.  We no longer need the verbose
induction schema for type \isa{term} and can use the simpler one arising from
\isa{trev}:%
\end{isamarkuptext}%
\isacommand{lemma}\ {\isachardoublequote}trev{\isacharparenleft}trev\ t{\isacharparenright}\ {\isacharequal}\ t{\isachardoublequote}\isanewline
\isacommand{apply}{\isacharparenleft}induct{\isacharunderscore}tac\ t\ rule{\isacharcolon}trev{\isachardot}induct{\isacharparenright}%
\begin{isamarkuptxt}%
\noindent
This leaves us with a trivial base case \isa{trev\ {\isacharparenleft}trev\ {\isacharparenleft}Var\ x{\isacharparenright}{\isacharparenright}\ {\isacharequal}\ Var\ x} and the step case
\begin{isabelle}%
\ \ \ \ \ {\isasymforall}t{\isachardot}\ t\ {\isasymin}\ set\ ts\ {\isasymlongrightarrow}\ trev\ {\isacharparenleft}trev\ t{\isacharparenright}\ {\isacharequal}\ t\ {\isasymLongrightarrow}\isanewline
\isaindent{\ \ \ \ \ }trev\ {\isacharparenleft}trev\ {\isacharparenleft}App\ f\ ts{\isacharparenright}{\isacharparenright}\ {\isacharequal}\ App\ f\ ts%
\end{isabelle}
both of which are solved by simplification:%
\end{isamarkuptxt}%
\isacommand{by}{\isacharparenleft}simp{\isacharunderscore}all\ add{\isacharcolon}rev{\isacharunderscore}map\ sym{\isacharbrackleft}OF\ map{\isacharunderscore}compose{\isacharbrackright}\ cong{\isacharcolon}map{\isacharunderscore}cong{\isacharparenright}%
\begin{isamarkuptext}%
\noindent
If the proof of the induction step mystifies you, we recommend that you go through
the chain of simplification steps in detail; you will probably need the help of
\isa{trace{\isacharunderscore}simp}. Theorem \isa{map{\isacharunderscore}cong} is discussed below.
%\begin{quote}
%{term[display]"trev(trev(App f ts))"}\\
%{term[display]"App f (rev(map trev (rev(map trev ts))))"}\\
%{term[display]"App f (map trev (rev(rev(map trev ts))))"}\\
%{term[display]"App f (map trev (map trev ts))"}\\
%{term[display]"App f (map (trev o trev) ts)"}\\
%{term[display]"App f (map (%x. x) ts)"}\\
%{term[display]"App f ts"}
%\end{quote}

The definition of \isa{trev} above is superior to the one in
\S\ref{sec:nested-datatype} because it uses \isa{rev}
and lets us use existing facts such as \hbox{\isa{rev\ {\isacharparenleft}rev\ xs{\isacharparenright}\ {\isacharequal}\ xs}}.
Thus this proof is a good example of an important principle:
\begin{quote}
\emph{Chose your definitions carefully\\
because they determine the complexity of your proofs.}
\end{quote}

Let us now return to the question of how \isacommand{recdef} can come up with
sensible termination conditions in the presence of higher-order functions
like \isa{map}. For a start, if nothing were known about \isa{map},
\isa{map\ trev\ ts} might apply \isa{trev} to arbitrary terms, and thus
\isacommand{recdef} would try to prove the unprovable \isa{size\ t\ {\isacharless}\ Suc\ {\isacharparenleft}term{\isacharunderscore}list{\isacharunderscore}size\ ts{\isacharparenright}}, without any assumption about \isa{t}.  Therefore
\isacommand{recdef} has been supplied with the congruence theorem
\isa{map{\isacharunderscore}cong}:
\begin{isabelle}%
\ \ \ \ \ {\isasymlbrakk}xs\ {\isacharequal}\ ys{\isacharsemicolon}\ {\isasymAnd}x{\isachardot}\ x\ {\isasymin}\ set\ ys\ {\isasymLongrightarrow}\ f\ x\ {\isacharequal}\ g\ x{\isasymrbrakk}\isanewline
\isaindent{\ \ \ \ \ }{\isasymLongrightarrow}\ map\ f\ xs\ {\isacharequal}\ map\ g\ ys%
\end{isabelle}
Its second premise expresses (indirectly) that the second argument of
\isa{map} is only applied to elements of its third argument. Congruence
rules for other higher-order functions on lists look very similar. If you get
into a situation where you need to supply \isacommand{recdef} with new
congruence rules, you can either append a hint locally
to the specific occurrence of \isacommand{recdef}%
\end{isamarkuptext}%
{\isacharparenleft}\isakeyword{hints}\ recdef{\isacharunderscore}cong{\isacharcolon}\ map{\isacharunderscore}cong{\isacharparenright}%
\begin{isamarkuptext}%
\noindent
or declare them globally
by giving them the \isaindexbold{recdef_cong} attribute as in%
\end{isamarkuptext}%
\isacommand{declare}\ map{\isacharunderscore}cong{\isacharbrackleft}recdef{\isacharunderscore}cong{\isacharbrackright}%
\begin{isamarkuptext}%
Note that the \isa{cong} and \isa{recdef{\isacharunderscore}cong} attributes are
intentionally kept apart because they control different activities, namely
simplification and making recursive definitions.
% The local \isa{cong} in
% the hints section of \isacommand{recdef} is merely short for \isa{recdef{\isacharunderscore}cong}.
%The simplifier's congruence rules cannot be used by recdef.
%For example the weak congruence rules for if and case would prevent
%recdef from generating sensible termination conditions.%
\end{isamarkuptext}%
\end{isabellebody}%
%%% Local Variables:
%%% mode: latex
%%% TeX-master: "root"
%%% End:


\subsection{Partial functions}
\index{partial function}
%
\begin{isabellebody}%
\def\isabellecontext{Partial}%
\isamarkupfalse%
%
\begin{isamarkuptext}%
\noindent Throughout this tutorial, we have emphasized
that all functions in HOL are total.  We cannot hope to define
truly partial functions, but must make them total.  A straightforward
method is to lift the result type of the function from $\tau$ to
$\tau$~\isa{option} (see \ref{sec:option}), where \isa{None} is
returned if the function is applied to an argument not in its
domain. Function \isa{assoc} in \S\ref{sec:Trie} is a simple example.
We do not pursue this schema further because it should be clear
how it works. Its main drawback is that the result of such a lifted
function has to be unpacked first before it can be processed
further. Its main advantage is that you can distinguish if the
function was applied to an argument in its domain or not. If you do
not need to make this distinction, for example because the function is
never used outside its domain, it is easier to work with
\emph{underdefined}\index{functions!underdefined} functions: for
certain arguments we only know that a result exists, but we do not
know what it is. When defining functions that are normally considered
partial, underdefinedness turns out to be a very reasonable
alternative.

We have already seen an instance of underdefinedness by means of
non-exhaustive pattern matching: the definition of \isa{last} in
\S\ref{sec:recdef-examples}. The same is allowed for \isacommand{primrec}%
\end{isamarkuptext}%
\isamarkuptrue%
\isacommand{consts}\ hd\ {\isacharcolon}{\isacharcolon}\ {\isachardoublequote}{\isacharprime}a\ list\ {\isasymRightarrow}\ {\isacharprime}a{\isachardoublequote}\isanewline
\isamarkupfalse%
\isacommand{primrec}\ {\isachardoublequote}hd\ {\isacharparenleft}x{\isacharhash}xs{\isacharparenright}\ {\isacharequal}\ x{\isachardoublequote}\isamarkupfalse%
%
\begin{isamarkuptext}%
\noindent
although it generates a warning.
Even ordinary definitions allow underdefinedness, this time by means of
preconditions:%
\end{isamarkuptext}%
\isamarkuptrue%
\isacommand{constdefs}\ minus\ {\isacharcolon}{\isacharcolon}\ {\isachardoublequote}nat\ {\isasymRightarrow}\ nat\ {\isasymRightarrow}\ nat{\isachardoublequote}\isanewline
{\isachardoublequote}n\ {\isasymle}\ m\ {\isasymLongrightarrow}\ minus\ m\ n\ {\isasymequiv}\ m\ {\isacharminus}\ n{\isachardoublequote}\isamarkupfalse%
%
\begin{isamarkuptext}%
The rest of this section is devoted to the question of how to define
partial recursive functions by other means than non-exhaustive pattern
matching.%
\end{isamarkuptext}%
\isamarkuptrue%
%
\isamarkupsubsubsection{Guarded Recursion%
}
\isamarkuptrue%
%
\begin{isamarkuptext}%
\index{recursion!guarded}%
Neither \isacommand{primrec} nor \isacommand{recdef} allow to
prefix an equation with a condition in the way ordinary definitions do
(see \isa{minus} above). Instead we have to move the condition over
to the right-hand side of the equation. Given a partial function $f$
that should satisfy the recursion equation $f(x) = t$ over its domain
$dom(f)$, we turn this into the \isacommand{recdef}
\begin{isabelle}%
\ \ \ \ \ f\ x\ {\isacharequal}\ {\isacharparenleft}if\ x\ {\isasymin}\ dom\ f\ then\ t\ else\ arbitrary{\isacharparenright}%
\end{isabelle}
where \isa{arbitrary} is a predeclared constant of type \isa{{\isacharprime}a}
which has no definition. Thus we know nothing about its value,
which is ideal for specifying underdefined functions on top of it.

As a simple example we define division on \isa{nat}:%
\end{isamarkuptext}%
\isamarkuptrue%
\isacommand{consts}\ divi\ {\isacharcolon}{\isacharcolon}\ {\isachardoublequote}nat\ {\isasymtimes}\ nat\ {\isasymRightarrow}\ nat{\isachardoublequote}\isanewline
\isamarkupfalse%
\isacommand{recdef}\ divi\ {\isachardoublequote}measure{\isacharparenleft}{\isasymlambda}{\isacharparenleft}m{\isacharcomma}n{\isacharparenright}{\isachardot}\ m{\isacharparenright}{\isachardoublequote}\isanewline
\ \ {\isachardoublequote}divi{\isacharparenleft}m{\isacharcomma}{\isadigit{0}}{\isacharparenright}\ {\isacharequal}\ arbitrary{\isachardoublequote}\isanewline
\ \ {\isachardoublequote}divi{\isacharparenleft}m{\isacharcomma}n{\isacharparenright}\ {\isacharequal}\ {\isacharparenleft}if\ m\ {\isacharless}\ n\ then\ {\isadigit{0}}\ else\ divi{\isacharparenleft}m{\isacharminus}n{\isacharcomma}n{\isacharparenright}{\isacharplus}{\isadigit{1}}{\isacharparenright}{\isachardoublequote}\isamarkupfalse%
%
\begin{isamarkuptext}%
\noindent Of course we could also have defined
\isa{divi\ {\isacharparenleft}m{\isacharcomma}\ {\isadigit{0}}{\isacharparenright}} to be some specific number, for example 0. The
latter option is chosen for the predefined \isa{div} function, which
simplifies proofs at the expense of deviating from the
standard mathematical division function.

As a more substantial example we consider the problem of searching a graph.
For simplicity our graph is given by a function \isa{f} of
type \isa{{\isacharprime}a\ {\isasymRightarrow}\ {\isacharprime}a} which
maps each node to its successor; the graph has out-degree 1.
The task is to find the end of a chain, modelled by a node pointing to
itself. Here is a first attempt:
\begin{isabelle}%
\ \ \ \ \ find\ {\isacharparenleft}f{\isacharcomma}\ x{\isacharparenright}\ {\isacharequal}\ {\isacharparenleft}if\ f\ x\ {\isacharequal}\ x\ then\ x\ else\ find\ {\isacharparenleft}f{\isacharcomma}\ f\ x{\isacharparenright}{\isacharparenright}%
\end{isabelle}
This may be viewed as a fixed point finder or as the second half of the well
known \emph{Union-Find} algorithm.
The snag is that it may not terminate if \isa{f} has non-trivial cycles.
Phrased differently, the relation%
\end{isamarkuptext}%
\isamarkuptrue%
\isacommand{constdefs}\ step{\isadigit{1}}\ {\isacharcolon}{\isacharcolon}\ {\isachardoublequote}{\isacharparenleft}{\isacharprime}a\ {\isasymRightarrow}\ {\isacharprime}a{\isacharparenright}\ {\isasymRightarrow}\ {\isacharparenleft}{\isacharprime}a\ {\isasymtimes}\ {\isacharprime}a{\isacharparenright}set{\isachardoublequote}\isanewline
\ \ {\isachardoublequote}step{\isadigit{1}}\ f\ {\isasymequiv}\ {\isacharbraceleft}{\isacharparenleft}y{\isacharcomma}x{\isacharparenright}{\isachardot}\ y\ {\isacharequal}\ f\ x\ {\isasymand}\ y\ {\isasymnoteq}\ x{\isacharbraceright}{\isachardoublequote}\isamarkupfalse%
%
\begin{isamarkuptext}%
\noindent
must be well-founded. Thus we make the following definition:%
\end{isamarkuptext}%
\isamarkuptrue%
\isacommand{consts}\ find\ {\isacharcolon}{\isacharcolon}\ {\isachardoublequote}{\isacharparenleft}{\isacharprime}a\ {\isasymRightarrow}\ {\isacharprime}a{\isacharparenright}\ {\isasymtimes}\ {\isacharprime}a\ {\isasymRightarrow}\ {\isacharprime}a{\isachardoublequote}\isanewline
\isamarkupfalse%
\isacommand{recdef}\ find\ {\isachardoublequote}same{\isacharunderscore}fst\ {\isacharparenleft}{\isasymlambda}f{\isachardot}\ wf{\isacharparenleft}step{\isadigit{1}}\ f{\isacharparenright}{\isacharparenright}\ step{\isadigit{1}}{\isachardoublequote}\isanewline
\ \ {\isachardoublequote}find{\isacharparenleft}f{\isacharcomma}x{\isacharparenright}\ {\isacharequal}\ {\isacharparenleft}if\ wf{\isacharparenleft}step{\isadigit{1}}\ f{\isacharparenright}\isanewline
\ \ \ \ \ \ \ \ \ \ \ \ \ \ \ \ then\ if\ f\ x\ {\isacharequal}\ x\ then\ x\ else\ find{\isacharparenleft}f{\isacharcomma}\ f\ x{\isacharparenright}\isanewline
\ \ \ \ \ \ \ \ \ \ \ \ \ \ \ \ else\ arbitrary{\isacharparenright}{\isachardoublequote}\isanewline
{\isacharparenleft}\isakeyword{hints}\ recdef{\isacharunderscore}simp{\isacharcolon}\ step{\isadigit{1}}{\isacharunderscore}def{\isacharparenright}\isamarkupfalse%
%
\begin{isamarkuptext}%
\noindent
The recursion equation itself should be clear enough: it is our aborted
first attempt augmented with a check that there are no non-trivial loops.
To express the required well-founded relation we employ the
predefined combinator \isa{same{\isacharunderscore}fst} of type
\begin{isabelle}%
\ \ \ \ \ {\isacharparenleft}{\isacharprime}a\ {\isasymRightarrow}\ bool{\isacharparenright}\ {\isasymRightarrow}\ {\isacharparenleft}{\isacharprime}a\ {\isasymRightarrow}\ {\isacharparenleft}{\isacharprime}b{\isasymtimes}{\isacharprime}b{\isacharparenright}set{\isacharparenright}\ {\isasymRightarrow}\ {\isacharparenleft}{\isacharparenleft}{\isacharprime}a{\isasymtimes}{\isacharprime}b{\isacharparenright}\ {\isasymtimes}\ {\isacharparenleft}{\isacharprime}a{\isasymtimes}{\isacharprime}b{\isacharparenright}{\isacharparenright}set%
\end{isabelle}
defined as
\begin{isabelle}%
\ \ \ \ \ same{\isacharunderscore}fst\ P\ R\ {\isasymequiv}\ {\isacharbraceleft}{\isacharparenleft}{\isacharparenleft}x{\isacharprime}{\isacharcomma}\ y{\isacharprime}{\isacharparenright}{\isacharcomma}\ x{\isacharcomma}\ y{\isacharparenright}{\isachardot}\ x{\isacharprime}\ {\isacharequal}\ x\ {\isasymand}\ P\ x\ {\isasymand}\ {\isacharparenleft}y{\isacharprime}{\isacharcomma}\ y{\isacharparenright}\ {\isasymin}\ R\ x{\isacharbraceright}%
\end{isabelle}
This combinator is designed for
recursive functions on pairs where the first component of the argument is
passed unchanged to all recursive calls. Given a constraint on the first
component and a relation on the second component, \isa{same{\isacharunderscore}fst} builds the
required relation on pairs.  The theorem
\begin{isabelle}%
\ \ \ \ \ {\isacharparenleft}{\isasymAnd}x{\isachardot}\ P\ x\ {\isasymLongrightarrow}\ wf\ {\isacharparenleft}R\ x{\isacharparenright}{\isacharparenright}\ {\isasymLongrightarrow}\ wf\ {\isacharparenleft}same{\isacharunderscore}fst\ P\ R{\isacharparenright}%
\end{isabelle}
is known to the well-foundedness prover of \isacommand{recdef}.  Thus
well-foundedness of the relation given to \isacommand{recdef} is immediate.
Furthermore, each recursive call descends along that relation: the first
argument stays unchanged and the second one descends along \isa{step{\isadigit{1}}\ f}. The proof requires unfolding the definition of \isa{step{\isadigit{1}}},
as specified in the \isacommand{hints} above.

Normally you will then derive the following conditional variant from
the recursion equation:%
\end{isamarkuptext}%
\isamarkuptrue%
\isacommand{lemma}\ {\isacharbrackleft}simp{\isacharbrackright}{\isacharcolon}\isanewline
\ \ {\isachardoublequote}wf{\isacharparenleft}step{\isadigit{1}}\ f{\isacharparenright}\ {\isasymLongrightarrow}\ find{\isacharparenleft}f{\isacharcomma}x{\isacharparenright}\ {\isacharequal}\ {\isacharparenleft}if\ f\ x\ {\isacharequal}\ x\ then\ x\ else\ find{\isacharparenleft}f{\isacharcomma}\ f\ x{\isacharparenright}{\isacharparenright}{\isachardoublequote}\isanewline
\isamarkupfalse%
\isacommand{by}\ simp\isamarkupfalse%
%
\begin{isamarkuptext}%
\noindent Then you should disable the original recursion equation:%
\end{isamarkuptext}%
\isamarkuptrue%
\isacommand{declare}\ find{\isachardot}simps{\isacharbrackleft}simp\ del{\isacharbrackright}\isamarkupfalse%
%
\begin{isamarkuptext}%
Reasoning about such underdefined functions is like that for other
recursive functions.  Here is a simple example of recursion induction:%
\end{isamarkuptext}%
\isamarkuptrue%
\isacommand{lemma}\ {\isachardoublequote}wf{\isacharparenleft}step{\isadigit{1}}\ f{\isacharparenright}\ {\isasymlongrightarrow}\ f{\isacharparenleft}find{\isacharparenleft}f{\isacharcomma}x{\isacharparenright}{\isacharparenright}\ {\isacharequal}\ find{\isacharparenleft}f{\isacharcomma}x{\isacharparenright}{\isachardoublequote}\isanewline
\isamarkupfalse%
\isacommand{apply}{\isacharparenleft}induct{\isacharunderscore}tac\ f\ x\ rule{\isacharcolon}find{\isachardot}induct{\isacharparenright}\isanewline
\isamarkupfalse%
\isacommand{apply}\ simp\isanewline
\isamarkupfalse%
\isacommand{done}\isamarkupfalse%
%
\isamarkupsubsubsection{The {\tt\slshape while} Combinator%
}
\isamarkuptrue%
%
\begin{isamarkuptext}%
If the recursive function happens to be tail recursive, its
definition becomes a triviality if based on the predefined \cdx{while}
combinator.  The latter lives in the Library theory \thydx{While_Combinator}.
% which is not part of {text Main} but needs to
% be included explicitly among the ancestor theories.

Constant \isa{while} is of type \isa{{\isacharparenleft}{\isacharprime}a\ {\isasymRightarrow}\ bool{\isacharparenright}\ {\isasymRightarrow}\ {\isacharparenleft}{\isacharprime}a\ {\isasymRightarrow}\ {\isacharprime}a{\isacharparenright}\ {\isasymRightarrow}\ {\isacharprime}a}
and satisfies the recursion equation \begin{isabelle}%
\ \ \ \ \ while\ b\ c\ s\ {\isacharequal}\ {\isacharparenleft}if\ b\ s\ then\ while\ b\ c\ {\isacharparenleft}c\ s{\isacharparenright}\ else\ s{\isacharparenright}%
\end{isabelle}
That is, \isa{while\ b\ c\ s} is equivalent to the imperative program
\begin{verbatim}
     x := s; while b(x) do x := c(x); return x
\end{verbatim}
In general, \isa{s} will be a tuple or record.  As an example
consider the following definition of function \isa{find}:%
\end{isamarkuptext}%
\isamarkuptrue%
\isacommand{constdefs}\ find{\isadigit{2}}\ {\isacharcolon}{\isacharcolon}\ {\isachardoublequote}{\isacharparenleft}{\isacharprime}a\ {\isasymRightarrow}\ {\isacharprime}a{\isacharparenright}\ {\isasymRightarrow}\ {\isacharprime}a\ {\isasymRightarrow}\ {\isacharprime}a{\isachardoublequote}\isanewline
\ \ {\isachardoublequote}find{\isadigit{2}}\ f\ x\ {\isasymequiv}\isanewline
\ \ \ fst{\isacharparenleft}while\ {\isacharparenleft}{\isasymlambda}{\isacharparenleft}x{\isacharcomma}x{\isacharprime}{\isacharparenright}{\isachardot}\ x{\isacharprime}\ {\isasymnoteq}\ x{\isacharparenright}\ {\isacharparenleft}{\isasymlambda}{\isacharparenleft}x{\isacharcomma}x{\isacharprime}{\isacharparenright}{\isachardot}\ {\isacharparenleft}x{\isacharprime}{\isacharcomma}f\ x{\isacharprime}{\isacharparenright}{\isacharparenright}\ {\isacharparenleft}x{\isacharcomma}f\ x{\isacharparenright}{\isacharparenright}{\isachardoublequote}\isamarkupfalse%
%
\begin{isamarkuptext}%
\noindent
The loop operates on two ``local variables'' \isa{x} and \isa{x{\isacharprime}}
containing the ``current'' and the ``next'' value of function \isa{f}.
They are initialized with the global \isa{x} and \isa{f\ x}. At the
end \isa{fst} selects the local \isa{x}.

Although the definition of tail recursive functions via \isa{while} avoids
termination proofs, there is no free lunch. When proving properties of
functions defined by \isa{while}, termination rears its ugly head
again. Here is \tdx{while_rule}, the well known proof rule for total
correctness of loops expressed with \isa{while}:
\begin{isabelle}%
\ \ \ \ \ {\isasymlbrakk}P\ s{\isacharsemicolon}\ {\isasymAnd}s{\isachardot}\ {\isasymlbrakk}P\ s{\isacharsemicolon}\ b\ s{\isasymrbrakk}\ {\isasymLongrightarrow}\ P\ {\isacharparenleft}c\ s{\isacharparenright}{\isacharsemicolon}\isanewline
\isaindent{\ \ \ \ \ \ \ \ }{\isasymAnd}s{\isachardot}\ {\isasymlbrakk}P\ s{\isacharsemicolon}\ {\isasymnot}\ b\ s{\isasymrbrakk}\ {\isasymLongrightarrow}\ Q\ s{\isacharsemicolon}\ wf\ r{\isacharsemicolon}\isanewline
\isaindent{\ \ \ \ \ \ \ \ }{\isasymAnd}s{\isachardot}\ {\isasymlbrakk}P\ s{\isacharsemicolon}\ b\ s{\isasymrbrakk}\ {\isasymLongrightarrow}\ {\isacharparenleft}c\ s{\isacharcomma}\ s{\isacharparenright}\ {\isasymin}\ r{\isasymrbrakk}\isanewline
\isaindent{\ \ \ \ \ }{\isasymLongrightarrow}\ Q\ {\isacharparenleft}while\ b\ c\ s{\isacharparenright}%
\end{isabelle} \isa{P} needs to be true of
the initial state \isa{s} and invariant under \isa{c} (premises 1
and~2). The post-condition \isa{Q} must become true when leaving the loop
(premise~3). And each loop iteration must descend along a well-founded
relation \isa{r} (premises 4 and~5).

Let us now prove that \isa{find{\isadigit{2}}} does indeed find a fixed point. Instead
of induction we apply the above while rule, suitably instantiated.
Only the final premise of \isa{while{\isacharunderscore}rule} is left unproved
by \isa{auto} but falls to \isa{simp}:%
\end{isamarkuptext}%
\isamarkuptrue%
\isacommand{lemma}\ lem{\isacharcolon}\ {\isachardoublequote}wf{\isacharparenleft}step{\isadigit{1}}\ f{\isacharparenright}\ {\isasymLongrightarrow}\isanewline
\ \ {\isasymexists}y{\isachardot}\ while\ {\isacharparenleft}{\isasymlambda}{\isacharparenleft}x{\isacharcomma}x{\isacharprime}{\isacharparenright}{\isachardot}\ x{\isacharprime}\ {\isasymnoteq}\ x{\isacharparenright}\ {\isacharparenleft}{\isasymlambda}{\isacharparenleft}x{\isacharcomma}x{\isacharprime}{\isacharparenright}{\isachardot}\ {\isacharparenleft}x{\isacharprime}{\isacharcomma}f\ x{\isacharprime}{\isacharparenright}{\isacharparenright}\ {\isacharparenleft}x{\isacharcomma}f\ x{\isacharparenright}\ {\isacharequal}\ {\isacharparenleft}y{\isacharcomma}y{\isacharparenright}\ {\isasymand}\isanewline
\ \ \ \ \ \ \ f\ y\ {\isacharequal}\ y{\isachardoublequote}\isanewline
\isamarkupfalse%
\isacommand{apply}{\isacharparenleft}rule{\isacharunderscore}tac\ P\ {\isacharequal}\ {\isachardoublequote}{\isasymlambda}{\isacharparenleft}x{\isacharcomma}x{\isacharprime}{\isacharparenright}{\isachardot}\ x{\isacharprime}\ {\isacharequal}\ f\ x{\isachardoublequote}\ \isakeyword{and}\isanewline
\ \ \ \ \ \ \ \ \ \ \ \ \ \ \ r\ {\isacharequal}\ {\isachardoublequote}inv{\isacharunderscore}image\ {\isacharparenleft}step{\isadigit{1}}\ f{\isacharparenright}\ fst{\isachardoublequote}\ \isakeyword{in}\ while{\isacharunderscore}rule{\isacharparenright}\isanewline
\isamarkupfalse%
\isacommand{apply}\ auto\isanewline
\isamarkupfalse%
\isacommand{apply}{\isacharparenleft}simp\ add{\isacharcolon}inv{\isacharunderscore}image{\isacharunderscore}def\ step{\isadigit{1}}{\isacharunderscore}def{\isacharparenright}\isanewline
\isamarkupfalse%
\isacommand{done}\isamarkupfalse%
%
\begin{isamarkuptext}%
The theorem itself is a simple consequence of this lemma:%
\end{isamarkuptext}%
\isamarkuptrue%
\isacommand{theorem}\ {\isachardoublequote}wf{\isacharparenleft}step{\isadigit{1}}\ f{\isacharparenright}\ {\isasymLongrightarrow}\ f{\isacharparenleft}find{\isadigit{2}}\ f\ x{\isacharparenright}\ {\isacharequal}\ find{\isadigit{2}}\ f\ x{\isachardoublequote}\isanewline
\isamarkupfalse%
\isacommand{apply}{\isacharparenleft}drule{\isacharunderscore}tac\ x\ {\isacharequal}\ x\ \isakeyword{in}\ lem{\isacharparenright}\isanewline
\isamarkupfalse%
\isacommand{apply}{\isacharparenleft}auto\ simp\ add{\isacharcolon}find{\isadigit{2}}{\isacharunderscore}def{\isacharparenright}\isanewline
\isamarkupfalse%
\isacommand{done}\isamarkupfalse%
%
\begin{isamarkuptext}%
Let us conclude this section on partial functions by a
discussion of the merits of the \isa{while} combinator. We have
already seen that the advantage of not having to
provide a termination argument when defining a function via \isa{while} merely puts off the evil hour. On top of that, tail recursive
functions tend to be more complicated to reason about. So why use
\isa{while} at all? The only reason is executability: the recursion
equation for \isa{while} is a directly executable functional
program. This is in stark contrast to guarded recursion as introduced
above which requires an explicit test \isa{x\ {\isasymin}\ dom\ f} in the
function body.  Unless \isa{dom} is trivial, this leads to a
definition that is impossible to execute or prohibitively slow.
Thus, if you are aiming for an efficiently executable definition
of a partial function, you are likely to need \isa{while}.%
\end{isamarkuptext}%
\isamarkuptrue%
\isamarkupfalse%
\end{isabellebody}%
%%% Local Variables:
%%% mode: latex
%%% TeX-master: "root"
%%% End:


\index{*recdef|)}

\section{Advanced induction techniques}
\label{sec:advanced-ind}
\index{induction|(}
%
\begin{isabellebody}%
\def\isabellecontext{AdvancedInd}%
%
\begin{isamarkuptext}%
\noindent
Now that we have learned about rules and logic, we take another look at the
finer points of induction. The two questions we answer are: what to do if the
proposition to be proved is not directly amenable to induction
(\S\ref{sec:ind-var-in-prems}), and how to utilize (\S\ref{sec:complete-ind})
and even derive (\S\ref{sec:derive-ind}) new induction schemas. We conclude
with an extended example of induction (\S\ref{sec:CTL-revisited}).%
\end{isamarkuptext}%
%
\isamarkupsubsection{Massaging the Proposition%
}
%
\begin{isamarkuptext}%
\label{sec:ind-var-in-prems}
Often we have assumed that the theorem we want to prove is already in a form
that is amenable to induction, but sometimes it isn't.
Here is an example.
Since \isa{hd} and \isa{last} return the first and last element of a
non-empty list, this lemma looks easy to prove:%
\end{isamarkuptext}%
\isacommand{lemma}\ {\isachardoublequote}xs\ {\isasymnoteq}\ {\isacharbrackleft}{\isacharbrackright}\ {\isasymLongrightarrow}\ hd{\isacharparenleft}rev\ xs{\isacharparenright}\ {\isacharequal}\ last\ xs{\isachardoublequote}\isanewline
\isacommand{apply}{\isacharparenleft}induct{\isacharunderscore}tac\ xs{\isacharparenright}%
\begin{isamarkuptxt}%
\noindent
But induction produces the warning
\begin{quote}\tt
Induction variable occurs also among premises!
\end{quote}
and leads to the base case
\begin{isabelle}%
\ {\isadigit{1}}{\isachardot}\ xs\ {\isasymnoteq}\ {\isacharbrackleft}{\isacharbrackright}\ {\isasymLongrightarrow}\ hd\ {\isacharparenleft}rev\ {\isacharbrackleft}{\isacharbrackright}{\isacharparenright}\ {\isacharequal}\ last\ {\isacharbrackleft}{\isacharbrackright}%
\end{isabelle}
After simplification, it becomes
\begin{isabelle}
\ 1.\ xs\ {\isasymnoteq}\ []\ {\isasymLongrightarrow}\ hd\ []\ =\ last\ []
\end{isabelle}
We cannot prove this equality because we do not know what \isa{hd} and
\isa{last} return when applied to \isa{{\isacharbrackleft}{\isacharbrackright}}.

We should not have ignored the warning. Because the induction
formula is only the conclusion, induction does not affect the occurrence of \isa{xs} in the premises.  
Thus the case that should have been trivial
becomes unprovable. Fortunately, the solution is easy:\footnote{A very similar
heuristic applies to rule inductions; see \S\ref{sec:rtc}.}
\begin{quote}
\emph{Pull all occurrences of the induction variable into the conclusion
using \isa{{\isasymlongrightarrow}}.}
\end{quote}
Thus we should state the lemma as an ordinary 
implication~(\isa{{\isasymlongrightarrow}}), letting
\isa{rule{\isacharunderscore}format} (\S\ref{sec:forward}) convert the
result to the usual \isa{{\isasymLongrightarrow}} form:%
\end{isamarkuptxt}%
\isacommand{lemma}\ hd{\isacharunderscore}rev\ {\isacharbrackleft}rule{\isacharunderscore}format{\isacharbrackright}{\isacharcolon}\ {\isachardoublequote}xs\ {\isasymnoteq}\ {\isacharbrackleft}{\isacharbrackright}\ {\isasymlongrightarrow}\ hd{\isacharparenleft}rev\ xs{\isacharparenright}\ {\isacharequal}\ last\ xs{\isachardoublequote}%
\begin{isamarkuptxt}%
\noindent
This time, induction leaves us with a trivial base case:
\begin{isabelle}%
\ {\isadigit{1}}{\isachardot}\ {\isacharbrackleft}{\isacharbrackright}\ {\isasymnoteq}\ {\isacharbrackleft}{\isacharbrackright}\ {\isasymlongrightarrow}\ hd\ {\isacharparenleft}rev\ {\isacharbrackleft}{\isacharbrackright}{\isacharparenright}\ {\isacharequal}\ last\ {\isacharbrackleft}{\isacharbrackright}%
\end{isabelle}
And \isa{auto} completes the proof.

If there are multiple premises $A@1$, \dots, $A@n$ containing the
induction variable, you should turn the conclusion $C$ into
\[ A@1 \longrightarrow \cdots A@n \longrightarrow C \]
Additionally, you may also have to universally quantify some other variables,
which can yield a fairly complex conclusion.  However, \isa{rule{\isacharunderscore}format} 
can remove any number of occurrences of \isa{{\isasymforall}} and
\isa{{\isasymlongrightarrow}}.%
\end{isamarkuptxt}%
%
\begin{isamarkuptext}%
A second reason why your proposition may not be amenable to induction is that
you want to induct on a whole term, rather than an individual variable. In
general, when inducting on some term $t$ you must rephrase the conclusion $C$
as
\[ \forall y@1 \dots y@n.~ x = t \longrightarrow C \]
where $y@1 \dots y@n$ are the free variables in $t$ and $x$ is new, and
perform induction on $x$ afterwards. An example appears in
\S\ref{sec:complete-ind} below.

The very same problem may occur in connection with rule induction. Remember
that it requires a premise of the form $(x@1,\dots,x@k) \in R$, where $R$ is
some inductively defined set and the $x@i$ are variables.  If instead we have
a premise $t \in R$, where $t$ is not just an $n$-tuple of variables, we
replace it with $(x@1,\dots,x@k) \in R$, and rephrase the conclusion $C$ as
\[ \forall y@1 \dots y@n.~ (x@1,\dots,x@k) = t \longrightarrow C \]
For an example see \S\ref{sec:CTL-revisited} below.

Of course, all premises that share free variables with $t$ need to be pulled into
the conclusion as well, under the \isa{{\isasymforall}}, again using \isa{{\isasymlongrightarrow}} as shown above.%
\end{isamarkuptext}%
%
\isamarkupsubsection{Beyond Structural and Recursion Induction%
}
%
\begin{isamarkuptext}%
\label{sec:complete-ind}
So far, inductive proofs were by structural induction for
primitive recursive functions and recursion induction for total recursive
functions. But sometimes structural induction is awkward and there is no
recursive function that could furnish a more appropriate
induction schema. In such cases a general-purpose induction schema can
be helpful. We show how to apply such induction schemas by an example.

Structural induction on \isa{nat} is
usually known as mathematical induction. There is also \emph{complete}
induction, where you must prove $P(n)$ under the assumption that $P(m)$
holds for all $m<n$. In Isabelle, this is the theorem \isa{nat{\isacharunderscore}less{\isacharunderscore}induct}:
\begin{isabelle}%
\ \ \ \ \ {\isacharparenleft}{\isasymAnd}n{\isachardot}\ {\isasymforall}m{\isachardot}\ m\ {\isacharless}\ n\ {\isasymlongrightarrow}\ P\ m\ {\isasymLongrightarrow}\ P\ n{\isacharparenright}\ {\isasymLongrightarrow}\ P\ n%
\end{isabelle}
Here is an example of its application.%
\end{isamarkuptext}%
\isacommand{consts}\ f\ {\isacharcolon}{\isacharcolon}\ {\isachardoublequote}nat\ {\isasymRightarrow}\ nat{\isachardoublequote}\isanewline
\isacommand{axioms}\ f{\isacharunderscore}ax{\isacharcolon}\ {\isachardoublequote}f{\isacharparenleft}f{\isacharparenleft}n{\isacharparenright}{\isacharparenright}\ {\isacharless}\ f{\isacharparenleft}Suc{\isacharparenleft}n{\isacharparenright}{\isacharparenright}{\isachardoublequote}%
\begin{isamarkuptext}%
\noindent
The axiom for \isa{f} implies \isa{n\ {\isasymle}\ f\ n}, which can
be proved by induction on \mbox{\isa{f\ n}}. Following the recipe outlined
above, we have to phrase the proposition as follows to allow induction:%
\end{isamarkuptext}%
\isacommand{lemma}\ f{\isacharunderscore}incr{\isacharunderscore}lem{\isacharcolon}\ {\isachardoublequote}{\isasymforall}i{\isachardot}\ k\ {\isacharequal}\ f\ i\ {\isasymlongrightarrow}\ i\ {\isasymle}\ f\ i{\isachardoublequote}%
\begin{isamarkuptxt}%
\noindent
To perform induction on \isa{k} using \isa{nat{\isacharunderscore}less{\isacharunderscore}induct}, we use
the same general induction method as for recursion induction (see
\S\ref{sec:recdef-induction}):%
\end{isamarkuptxt}%
\isacommand{apply}{\isacharparenleft}induct{\isacharunderscore}tac\ k\ rule{\isacharcolon}\ nat{\isacharunderscore}less{\isacharunderscore}induct{\isacharparenright}%
\begin{isamarkuptxt}%
\noindent
which leaves us with the following proof state:
\begin{isabelle}%
\ {\isadigit{1}}{\isachardot}\ {\isasymAnd}n{\isachardot}\ {\isasymforall}m{\isachardot}\ m\ {\isacharless}\ n\ {\isasymlongrightarrow}\ {\isacharparenleft}{\isasymforall}i{\isachardot}\ m\ {\isacharequal}\ f\ i\ {\isasymlongrightarrow}\ i\ {\isasymle}\ f\ i{\isacharparenright}\ {\isasymLongrightarrow}\isanewline
\isaindent{\ {\isadigit{1}}{\isachardot}\ {\isasymAnd}n{\isachardot}\ }{\isasymforall}i{\isachardot}\ n\ {\isacharequal}\ f\ i\ {\isasymlongrightarrow}\ i\ {\isasymle}\ f\ i%
\end{isabelle}
After stripping the \isa{{\isasymforall}i}, the proof continues with a case
distinction on \isa{i}. The case \isa{i\ {\isacharequal}\ {\isadigit{0}}} is trivial and we focus on
the other case:%
\end{isamarkuptxt}%
\isacommand{apply}{\isacharparenleft}rule\ allI{\isacharparenright}\isanewline
\isacommand{apply}{\isacharparenleft}case{\isacharunderscore}tac\ i{\isacharparenright}\isanewline
\ \isacommand{apply}{\isacharparenleft}simp{\isacharparenright}%
\begin{isamarkuptxt}%
\begin{isabelle}%
\ {\isadigit{1}}{\isachardot}\ {\isasymAnd}n\ i\ nat{\isachardot}\isanewline
\isaindent{\ {\isadigit{1}}{\isachardot}\ \ \ \ }{\isasymlbrakk}{\isasymforall}m{\isachardot}\ m\ {\isacharless}\ n\ {\isasymlongrightarrow}\ {\isacharparenleft}{\isasymforall}i{\isachardot}\ m\ {\isacharequal}\ f\ i\ {\isasymlongrightarrow}\ i\ {\isasymle}\ f\ i{\isacharparenright}{\isacharsemicolon}\ i\ {\isacharequal}\ Suc\ nat{\isasymrbrakk}\isanewline
\isaindent{\ {\isadigit{1}}{\isachardot}\ \ \ \ }{\isasymLongrightarrow}\ n\ {\isacharequal}\ f\ i\ {\isasymlongrightarrow}\ i\ {\isasymle}\ f\ i%
\end{isabelle}%
\end{isamarkuptxt}%
\isacommand{by}{\isacharparenleft}blast\ intro{\isacharbang}{\isacharcolon}\ f{\isacharunderscore}ax\ Suc{\isacharunderscore}leI\ intro{\isacharcolon}\ le{\isacharunderscore}less{\isacharunderscore}trans{\isacharparenright}%
\begin{isamarkuptext}%
\noindent
If you find the last step puzzling, here are the two lemmas it employs:
\begin{isabelle}
\isa{m\ {\isacharless}\ n\ {\isasymLongrightarrow}\ Suc\ m\ {\isasymle}\ n}
\rulename{Suc_leI}\isanewline
\isa{{\isasymlbrakk}i\ {\isasymle}\ j{\isacharsemicolon}\ j\ {\isacharless}\ k{\isasymrbrakk}\ {\isasymLongrightarrow}\ i\ {\isacharless}\ k}
\rulename{le_less_trans}
\end{isabelle}
%
The proof goes like this (writing \isa{j} instead of \isa{nat}).
Since \isa{i\ {\isacharequal}\ Suc\ j} it suffices to show
\hbox{\isa{j\ {\isacharless}\ f\ {\isacharparenleft}Suc\ j{\isacharparenright}}},
by \isa{Suc{\isacharunderscore}leI}\@.  This is
proved as follows. From \isa{f{\isacharunderscore}ax} we have \isa{f\ {\isacharparenleft}f\ j{\isacharparenright}\ {\isacharless}\ f\ {\isacharparenleft}Suc\ j{\isacharparenright}}
(1) which implies \isa{f\ j\ {\isasymle}\ f\ {\isacharparenleft}f\ j{\isacharparenright}} by the induction hypothesis.
Using (1) once more we obtain \isa{f\ j\ {\isacharless}\ f\ {\isacharparenleft}Suc\ j{\isacharparenright}} (2) by the transitivity
rule \isa{le{\isacharunderscore}less{\isacharunderscore}trans}.
Using the induction hypothesis once more we obtain \isa{j\ {\isasymle}\ f\ j}
which, together with (2) yields \isa{j\ {\isacharless}\ f\ {\isacharparenleft}Suc\ j{\isacharparenright}} (again by
\isa{le{\isacharunderscore}less{\isacharunderscore}trans}).

This last step shows both the power and the danger of automatic proofs: they
will usually not tell you how the proof goes, because it can be very hard to
translate the internal proof into a human-readable format. Therefore
Chapter~\ref{sec:part2?} introduces a language for writing readable
proofs.

We can now derive the desired \isa{i\ {\isasymle}\ f\ i} from \isa{f{\isacharunderscore}incr{\isacharunderscore}lem}:%
\end{isamarkuptext}%
\isacommand{lemmas}\ f{\isacharunderscore}incr\ {\isacharequal}\ f{\isacharunderscore}incr{\isacharunderscore}lem{\isacharbrackleft}rule{\isacharunderscore}format{\isacharcomma}\ OF\ refl{\isacharbrackright}%
\begin{isamarkuptext}%
\noindent
The final \isa{refl} gets rid of the premise \isa{{\isacharquery}k\ {\isacharequal}\ f\ {\isacharquery}i}. 
We could have included this derivation in the original statement of the lemma:%
\end{isamarkuptext}%
\isacommand{lemma}\ f{\isacharunderscore}incr{\isacharbrackleft}rule{\isacharunderscore}format{\isacharcomma}\ OF\ refl{\isacharbrackright}{\isacharcolon}\ {\isachardoublequote}{\isasymforall}i{\isachardot}\ k\ {\isacharequal}\ f\ i\ {\isasymlongrightarrow}\ i\ {\isasymle}\ f\ i{\isachardoublequote}%
\begin{isamarkuptext}%
\begin{warn}
We discourage the use of axioms because of the danger of
inconsistencies.  Axiom \isa{f{\isacharunderscore}ax} does
not introduce an inconsistency because, for example, the identity function
satisfies it.  Axioms can be useful in exploratory developments, say when 
you assume some well-known theorems so that you can quickly demonstrate some
point about methodology.  If your example turns into a substantial proof
development, you should replace the axioms by proofs.
\end{warn}

\bigskip
In general, \isa{induct{\isacharunderscore}tac} can be applied with any rule $r$
whose conclusion is of the form ${?}P~?x@1 \dots ?x@n$, in which case the
format is
\begin{quote}
\isacommand{apply}\isa{{\isacharparenleft}induct{\isacharunderscore}tac} $y@1 \dots y@n$ \isa{rule{\isacharcolon}} $r$\isa{{\isacharparenright}}
\end{quote}\index{*induct_tac}%
where $y@1, \dots, y@n$ are variables in the first subgoal.
A further example of a useful induction rule is \isa{length{\isacharunderscore}induct},
induction on the length of a list:\indexbold{*length_induct}
\begin{isabelle}%
\ \ \ \ \ {\isacharparenleft}{\isasymAnd}xs{\isachardot}\ {\isasymforall}ys{\isachardot}\ length\ ys\ {\isacharless}\ length\ xs\ {\isasymlongrightarrow}\ P\ ys\ {\isasymLongrightarrow}\ P\ xs{\isacharparenright}\ {\isasymLongrightarrow}\ P\ xs%
\end{isabelle}

In fact, \isa{induct{\isacharunderscore}tac} even allows the conclusion of
$r$ to be an (iterated) conjunction of formulae of the above form, in
which case the application is
\begin{quote}
\isacommand{apply}\isa{{\isacharparenleft}induct{\isacharunderscore}tac} $y@1 \dots y@n$ \isa{and} \dots\ \isa{and} $z@1 \dots z@m$ \isa{rule{\isacharcolon}} $r$\isa{{\isacharparenright}}
\end{quote}

\begin{exercise}
From the axiom and lemma for \isa{f}, show that \isa{f} is the
identity function.
\end{exercise}%
\end{isamarkuptext}%
%
\isamarkupsubsection{Derivation of New Induction Schemas%
}
%
\begin{isamarkuptext}%
\label{sec:derive-ind}
Induction schemas are ordinary theorems and you can derive new ones
whenever you wish.  This section shows you how to, using the example
of \isa{nat{\isacharunderscore}less{\isacharunderscore}induct}. Assume we only have structural induction
available for \isa{nat} and want to derive complete induction. This
requires us to generalize the statement first:%
\end{isamarkuptext}%
\isacommand{lemma}\ induct{\isacharunderscore}lem{\isacharcolon}\ {\isachardoublequote}{\isacharparenleft}{\isasymAnd}n{\isacharcolon}{\isacharcolon}nat{\isachardot}\ {\isasymforall}m{\isacharless}n{\isachardot}\ P\ m\ {\isasymLongrightarrow}\ P\ n{\isacharparenright}\ {\isasymLongrightarrow}\ {\isasymforall}m{\isacharless}n{\isachardot}\ P\ m{\isachardoublequote}\isanewline
\isacommand{apply}{\isacharparenleft}induct{\isacharunderscore}tac\ n{\isacharparenright}%
\begin{isamarkuptxt}%
\noindent
The base case is vacuously true. For the induction step (\isa{m\ {\isacharless}\ Suc\ n}) we distinguish two cases: case \isa{m\ {\isacharless}\ n} is true by induction
hypothesis and case \isa{m\ {\isacharequal}\ n} follows from the assumption, again using
the induction hypothesis:%
\end{isamarkuptxt}%
\ \isacommand{apply}{\isacharparenleft}blast{\isacharparenright}\isanewline
\isacommand{by}{\isacharparenleft}blast\ elim{\isacharcolon}less{\isacharunderscore}SucE{\isacharparenright}%
\begin{isamarkuptext}%
\noindent
The elimination rule \isa{less{\isacharunderscore}SucE} expresses the case distinction:
\begin{isabelle}%
\ \ \ \ \ {\isasymlbrakk}m\ {\isacharless}\ Suc\ n{\isacharsemicolon}\ m\ {\isacharless}\ n\ {\isasymLongrightarrow}\ P{\isacharsemicolon}\ m\ {\isacharequal}\ n\ {\isasymLongrightarrow}\ P{\isasymrbrakk}\ {\isasymLongrightarrow}\ P%
\end{isabelle}

Now it is straightforward to derive the original version of
\isa{nat{\isacharunderscore}less{\isacharunderscore}induct} by manipulting the conclusion of the above lemma:
instantiate \isa{n} by \isa{Suc\ n} and \isa{m} by \isa{n} and
remove the trivial condition \isa{n\ {\isacharless}\ Suc\ n}. Fortunately, this
happens automatically when we add the lemma as a new premise to the
desired goal:%
\end{isamarkuptext}%
\isacommand{theorem}\ nat{\isacharunderscore}less{\isacharunderscore}induct{\isacharcolon}\ {\isachardoublequote}{\isacharparenleft}{\isasymAnd}n{\isacharcolon}{\isacharcolon}nat{\isachardot}\ {\isasymforall}m{\isacharless}n{\isachardot}\ P\ m\ {\isasymLongrightarrow}\ P\ n{\isacharparenright}\ {\isasymLongrightarrow}\ P\ n{\isachardoublequote}\isanewline
\isacommand{by}{\isacharparenleft}insert\ induct{\isacharunderscore}lem{\isacharcomma}\ blast{\isacharparenright}%
\begin{isamarkuptext}%
Finally we should remind the reader that HOL already provides the mother of
all inductions, well-founded induction (see \S\ref{sec:Well-founded}).  For
example theorem \isa{nat{\isacharunderscore}less{\isacharunderscore}induct} is
a special case of \isa{wf{\isacharunderscore}induct} where \isa{r} is \isa{{\isacharless}} on
\isa{nat}. The details can be found in theory \isa{Wellfounded_Recursion}.%
\end{isamarkuptext}%
\end{isabellebody}%
%%% Local Variables:
%%% mode: latex
%%% TeX-master: "root"
%%% End:

%
\begin{isabellebody}%
\def\isabellecontext{CTLind}%
%
\isamarkupsubsection{CTL revisited%
}
%
\begin{isamarkuptext}%
\label{sec:CTL-revisited}
The purpose of this section is twofold: we want to demonstrate
some of the induction principles and heuristics discussed above and we want to
show how inductive definitions can simplify proofs.
In \S\ref{sec:CTL} we gave a fairly involved proof of the correctness of a
model checker for CTL\@. In particular the proof of the
\isa{infinity{\isacharunderscore}lemma} on the way to \isa{AF{\isacharunderscore}lemma{\isadigit{2}}} is not as
simple as one might intuitively expect, due to the \isa{SOME} operator
involved. Below we give a simpler proof of \isa{AF{\isacharunderscore}lemma{\isadigit{2}}}
based on an auxiliary inductive definition.

Let us call a (finite or infinite) path \emph{\isa{A}-avoiding} if it does
not touch any node in the set \isa{A}. Then \isa{AF{\isacharunderscore}lemma{\isadigit{2}}} says
that if no infinite path from some state \isa{s} is \isa{A}-avoiding,
then \isa{s\ {\isasymin}\ lfp\ {\isacharparenleft}af\ A{\isacharparenright}}. We prove this by inductively defining the set
\isa{Avoid\ s\ A} of states reachable from \isa{s} by a finite \isa{A}-avoiding path:
% Second proof of opposite direction, directly by well-founded induction
% on the initial segment of M that avoids A.%
\end{isamarkuptext}%
\isacommand{consts}\ Avoid\ {\isacharcolon}{\isacharcolon}\ {\isachardoublequote}state\ {\isasymRightarrow}\ state\ set\ {\isasymRightarrow}\ state\ set{\isachardoublequote}\isanewline
\isacommand{inductive}\ {\isachardoublequote}Avoid\ s\ A{\isachardoublequote}\isanewline
\isakeyword{intros}\ {\isachardoublequote}s\ {\isasymin}\ Avoid\ s\ A{\isachardoublequote}\isanewline
\ \ \ \ \ \ \ {\isachardoublequote}{\isasymlbrakk}\ t\ {\isasymin}\ Avoid\ s\ A{\isacharsemicolon}\ t\ {\isasymnotin}\ A{\isacharsemicolon}\ {\isacharparenleft}t{\isacharcomma}u{\isacharparenright}\ {\isasymin}\ M\ {\isasymrbrakk}\ {\isasymLongrightarrow}\ u\ {\isasymin}\ Avoid\ s\ A{\isachardoublequote}%
\begin{isamarkuptext}%
It is easy to see that for any infinite \isa{A}-avoiding path \isa{f}
with \isa{f\ {\isadigit{0}}\ {\isasymin}\ Avoid\ s\ A} there is an infinite \isa{A}-avoiding path
starting with \isa{s} because (by definition of \isa{Avoid}) there is a
finite \isa{A}-avoiding path from \isa{s} to \isa{f\ {\isadigit{0}}}.
The proof is by induction on \isa{f\ {\isadigit{0}}\ {\isasymin}\ Avoid\ s\ A}. However,
this requires the following
reformulation, as explained in \S\ref{sec:ind-var-in-prems} above;
the \isa{rule{\isacharunderscore}format} directive undoes the reformulation after the proof.%
\end{isamarkuptext}%
\isacommand{lemma}\ ex{\isacharunderscore}infinite{\isacharunderscore}path{\isacharbrackleft}rule{\isacharunderscore}format{\isacharbrackright}{\isacharcolon}\isanewline
\ \ {\isachardoublequote}t\ {\isasymin}\ Avoid\ s\ A\ \ {\isasymLongrightarrow}\isanewline
\ \ \ {\isasymforall}f{\isasymin}Paths\ t{\isachardot}\ {\isacharparenleft}{\isasymforall}i{\isachardot}\ f\ i\ {\isasymnotin}\ A{\isacharparenright}\ {\isasymlongrightarrow}\ {\isacharparenleft}{\isasymexists}p{\isasymin}Paths\ s{\isachardot}\ {\isasymforall}i{\isachardot}\ p\ i\ {\isasymnotin}\ A{\isacharparenright}{\isachardoublequote}\isanewline
\isacommand{apply}{\isacharparenleft}erule\ Avoid{\isachardot}induct{\isacharparenright}\isanewline
\ \isacommand{apply}{\isacharparenleft}blast{\isacharparenright}\isanewline
\isacommand{apply}{\isacharparenleft}clarify{\isacharparenright}\isanewline
\isacommand{apply}{\isacharparenleft}drule{\isacharunderscore}tac\ x\ {\isacharequal}\ {\isachardoublequote}{\isasymlambda}i{\isachardot}\ case\ i\ of\ {\isadigit{0}}\ {\isasymRightarrow}\ t\ {\isacharbar}\ Suc\ i\ {\isasymRightarrow}\ f\ i{\isachardoublequote}\ \isakeyword{in}\ bspec{\isacharparenright}\isanewline
\isacommand{apply}{\isacharparenleft}simp{\isacharunderscore}all\ add{\isacharcolon}Paths{\isacharunderscore}def\ split{\isacharcolon}nat{\isachardot}split{\isacharparenright}\isanewline
\isacommand{done}%
\begin{isamarkuptext}%
\noindent
The base case (\isa{t\ {\isacharequal}\ s}) is trivial (\isa{blast}).
In the induction step, we have an infinite \isa{A}-avoiding path \isa{f}
starting from \isa{u}, a successor of \isa{t}. Now we simply instantiate
the \isa{{\isasymforall}f{\isasymin}Paths\ t} in the induction hypothesis by the path starting with
\isa{t} and continuing with \isa{f}. That is what the above $\lambda$-term
expresses. That fact that this is a path starting with \isa{t} and that
the instantiated induction hypothesis implies the conclusion is shown by
simplification.

Now we come to the key lemma. It says that if \isa{t} can be reached by a
finite \isa{A}-avoiding path from \isa{s}, then \isa{t\ {\isasymin}\ lfp\ {\isacharparenleft}af\ A{\isacharparenright}},
provided there is no infinite \isa{A}-avoiding path starting from \isa{s}.%
\end{isamarkuptext}%
\isacommand{lemma}\ Avoid{\isacharunderscore}in{\isacharunderscore}lfp{\isacharbrackleft}rule{\isacharunderscore}format{\isacharparenleft}no{\isacharunderscore}asm{\isacharparenright}{\isacharbrackright}{\isacharcolon}\isanewline
\ \ {\isachardoublequote}{\isasymforall}p{\isasymin}Paths\ s{\isachardot}\ {\isasymexists}i{\isachardot}\ p\ i\ {\isasymin}\ A\ {\isasymLongrightarrow}\ t\ {\isasymin}\ Avoid\ s\ A\ {\isasymlongrightarrow}\ t\ {\isasymin}\ lfp{\isacharparenleft}af\ A{\isacharparenright}{\isachardoublequote}%
\begin{isamarkuptxt}%
\noindent
The trick is not to induct on \isa{t\ {\isasymin}\ Avoid\ s\ A}, as already the base
case would be a problem, but to proceed by well-founded induction \isa{t}. Hence \isa{t\ {\isasymin}\ Avoid\ s\ A} needs to be brought into the conclusion as
well, which the directive \isa{rule{\isacharunderscore}format} undoes at the end (see below).
But induction with respect to which well-founded relation? The restriction
of \isa{M} to \isa{Avoid\ s\ A}:
\begin{isabelle}%
\ \ \ \ \ {\isacharbraceleft}{\isacharparenleft}y{\isacharcomma}\ x{\isacharparenright}{\isachardot}\ {\isacharparenleft}x{\isacharcomma}\ y{\isacharparenright}\ {\isasymin}\ M\ {\isasymand}\ x\ {\isasymin}\ Avoid\ s\ A\ {\isasymand}\ y\ {\isasymin}\ Avoid\ s\ A\ {\isasymand}\ x\ {\isasymnotin}\ A{\isacharbraceright}%
\end{isabelle}
As we shall see in a moment, the absence of infinite \isa{A}-avoiding paths
starting from \isa{s} implies well-foundedness of this relation. For the
moment we assume this and proceed with the induction:%
\end{isamarkuptxt}%
\isacommand{apply}{\isacharparenleft}subgoal{\isacharunderscore}tac\isanewline
\ \ {\isachardoublequote}wf{\isacharbraceleft}{\isacharparenleft}y{\isacharcomma}x{\isacharparenright}{\isachardot}\ {\isacharparenleft}x{\isacharcomma}y{\isacharparenright}{\isasymin}M\ {\isasymand}\ x\ {\isasymin}\ Avoid\ s\ A\ {\isasymand}\ y\ {\isasymin}\ Avoid\ s\ A\ {\isasymand}\ x\ {\isasymnotin}\ A{\isacharbraceright}{\isachardoublequote}{\isacharparenright}\isanewline
\ \isacommand{apply}{\isacharparenleft}erule{\isacharunderscore}tac\ a\ {\isacharequal}\ t\ \isakeyword{in}\ wf{\isacharunderscore}induct{\isacharparenright}\isanewline
\ \isacommand{apply}{\isacharparenleft}clarsimp{\isacharparenright}%
\begin{isamarkuptxt}%
\noindent
Now can assume additionally (induction hypothesis) that if \isa{t\ {\isasymnotin}\ A}
then all successors of \isa{t} that are in \isa{Avoid\ s\ A} are in
\isa{lfp\ {\isacharparenleft}af\ A{\isacharparenright}}. To prove the actual goal we unfold \isa{lfp} once. Now
we have to prove that \isa{t} is in \isa{A} or all successors of \isa{t} are in \isa{lfp\ {\isacharparenleft}af\ A{\isacharparenright}}. If \isa{t} is not in \isa{A}, the second
\isa{Avoid}-rule implies that all successors of \isa{t} are in
\isa{Avoid\ s\ A} (because we also assume \isa{t\ {\isasymin}\ Avoid\ s\ A}), and
hence, by the induction hypothesis, all successors of \isa{t} are indeed in
\isa{lfp\ {\isacharparenleft}af\ A{\isacharparenright}}. Mechanically:%
\end{isamarkuptxt}%
\ \isacommand{apply}{\isacharparenleft}rule\ ssubst\ {\isacharbrackleft}OF\ lfp{\isacharunderscore}unfold{\isacharbrackleft}OF\ mono{\isacharunderscore}af{\isacharbrackright}{\isacharbrackright}{\isacharparenright}\isanewline
\ \isacommand{apply}{\isacharparenleft}simp\ only{\isacharcolon}\ af{\isacharunderscore}def{\isacharparenright}\isanewline
\ \isacommand{apply}{\isacharparenleft}blast\ intro{\isacharcolon}Avoid{\isachardot}intros{\isacharparenright}%
\begin{isamarkuptxt}%
Having proved the main goal we return to the proof obligation that the above
relation is indeed well-founded. This is proved by contraposition: we assume
the relation is not well-founded. Thus there exists an infinite \isa{A}-avoiding path all in \isa{Avoid\ s\ A}, by theorem
\isa{wf{\isacharunderscore}iff{\isacharunderscore}no{\isacharunderscore}infinite{\isacharunderscore}down{\isacharunderscore}chain}:
\begin{isabelle}%
\ \ \ \ \ wf\ r\ {\isacharequal}\ {\isacharparenleft}{\isasymnot}\ {\isacharparenleft}{\isasymexists}f{\isachardot}\ {\isasymforall}i{\isachardot}\ {\isacharparenleft}f\ {\isacharparenleft}Suc\ i{\isacharparenright}{\isacharcomma}\ f\ i{\isacharparenright}\ {\isasymin}\ r{\isacharparenright}{\isacharparenright}%
\end{isabelle}
From lemma \isa{ex{\isacharunderscore}infinite{\isacharunderscore}path} the existence of an infinite
\isa{A}-avoiding path starting in \isa{s} follows, just as required for
the contraposition.%
\end{isamarkuptxt}%
\isacommand{apply}{\isacharparenleft}erule\ contrapos{\isacharunderscore}pp{\isacharparenright}\isanewline
\isacommand{apply}{\isacharparenleft}simp\ add{\isacharcolon}wf{\isacharunderscore}iff{\isacharunderscore}no{\isacharunderscore}infinite{\isacharunderscore}down{\isacharunderscore}chain{\isacharparenright}\isanewline
\isacommand{apply}{\isacharparenleft}erule\ exE{\isacharparenright}\isanewline
\isacommand{apply}{\isacharparenleft}rule\ ex{\isacharunderscore}infinite{\isacharunderscore}path{\isacharparenright}\isanewline
\isacommand{apply}{\isacharparenleft}auto\ simp\ add{\isacharcolon}Paths{\isacharunderscore}def{\isacharparenright}\isanewline
\isacommand{done}%
\begin{isamarkuptext}%
The \isa{{\isacharparenleft}no{\isacharunderscore}asm{\isacharparenright}} modifier of the \isa{rule{\isacharunderscore}format} directive means
that the assumption is left unchanged---otherwise the \isa{{\isasymforall}p} is turned
into a \isa{{\isasymAnd}p}, which would complicate matters below. As it is,
\isa{Avoid{\isacharunderscore}in{\isacharunderscore}lfp} is now
\begin{isabelle}%
\ \ \ \ \ {\isasymlbrakk}{\isasymforall}p{\isasymin}Paths\ s{\isachardot}\ {\isasymexists}i{\isachardot}\ p\ i\ {\isasymin}\ A{\isacharsemicolon}\ t\ {\isasymin}\ Avoid\ s\ A{\isasymrbrakk}\ {\isasymLongrightarrow}\ t\ {\isasymin}\ lfp\ {\isacharparenleft}af\ A{\isacharparenright}%
\end{isabelle}
The main theorem is simply the corollary where \isa{t\ {\isacharequal}\ s},
in which case the assumption \isa{t\ {\isasymin}\ Avoid\ s\ A} is trivially true
by the first \isa{Avoid}-rule). Isabelle confirms this:%
\end{isamarkuptext}%
\isacommand{theorem}\ AF{\isacharunderscore}lemma{\isadigit{2}}{\isacharcolon}\isanewline
\ \ {\isachardoublequote}{\isacharbraceleft}s{\isachardot}\ {\isasymforall}p\ {\isasymin}\ Paths\ s{\isachardot}\ {\isasymexists}\ i{\isachardot}\ p\ i\ {\isasymin}\ A{\isacharbraceright}\ {\isasymsubseteq}\ lfp{\isacharparenleft}af\ A{\isacharparenright}{\isachardoublequote}\isanewline
\isacommand{by}{\isacharparenleft}auto\ elim{\isacharcolon}Avoid{\isacharunderscore}in{\isacharunderscore}lfp\ intro{\isacharcolon}Avoid{\isachardot}intros{\isacharparenright}\isanewline
\isanewline
\end{isabellebody}%
%%% Local Variables:
%%% mode: latex
%%% TeX-master: "root"
%%% End:

\index{induction|)}

%\chapter{Theory Presentation} Document preparation / Syntax Matters!
\chapter{Case Study: Verifying a Security Protocol}
\label{chap:crypto}

\index{protocols!security|(}

%crypto primitives 
\def\lbb{\mathopen{\{\kern-.30em|}}
\def\rbb{\mathclose{|\kern-.32em\}}}
\def\comp#1{\lbb#1\rbb}

Communications security is an ancient art.  Julius Caesar is said to have
encrypted his messages, shifting each letter three places along the
alphabet.  Mary Queen of Scots was convicted of treason after a cipher used
in her letters was broken.  Today's postal system
incorporates security features.  The envelope provides a degree of
\emph{secrecy}.  The signature provides \emph{authenticity} (proof of
origin), as do departmental stamps and letterheads.

Networks are vulnerable: messages pass through many computers, any of which
might be controlled by an adversary, who thus can capture or redirect
messages.  People who wish to communicate securely over such a network can
use cryptography, but if they are to understand each other, they need to
follow a
\emph{protocol}: a pre-arranged sequence of message formats. 

Protocols can be attacked in many ways, even if encryption is unbreakable. 
A \emph{splicing attack} involves an adversary's sending a message composed
of parts of several old messages.  This fake message may have the correct
format, fooling an honest party.  The adversary might be able to masquerade
as somebody else, or he might obtain a secret key.

\emph{Nonces} help prevent splicing attacks. A typical nonce is a 20-byte
random number. Each message that requires a reply incorporates a nonce. The
reply must include a copy of that nonce, to prove that it is not a replay of
a past message.  The nonce in the reply must be cryptographically
protected, since otherwise an adversary could easily replace it by a
different one. You should be starting to see that protocol design is
tricky!

Researchers are developing methods for proving the correctness of security
protocols.  The Needham-Schroeder public-key
protocol~\cite{needham-schroeder} has become a standard test case. 
Proposed in 1978, it was found to be defective nearly two decades
later~\cite{lowe-fdr}.  This toy protocol will be useful in demonstrating
how to verify protocols using Isabelle.


\section{The Needham-Schroeder Public-Key Protocol}\label{sec:ns-protocol}

\index{Needham-Schroeder protocol|(}%
This protocol uses public-key cryptography. Each person has a private key, known only to
himself, and a public key, known to everybody. If Alice wants to send Bob a secret message, she
encrypts it using Bob's public key (which everybody knows), and sends it to Bob. Only Bob has the
matching private key, which is needed in order to decrypt Alice's message.

The core of the Needham-Schroeder protocol consists of three messages:
\begin{alignat*}{2}
  &1.&\quad  A\to B  &: \comp{Na,A}\sb{Kb} \\
  &2.&\quad  B\to A  &: \comp{Na,Nb}\sb{Ka} \\
  &3.&\quad  A\to B  &: \comp{Nb}\sb{Kb}
\end{alignat*}
First, let's understand the notation. In the first message, Alice
sends Bob a message consisting of a nonce generated by Alice~($Na$)
paired  with Alice's name~($A$) and encrypted using Bob's public
key~($Kb$). In the second message, Bob sends Alice a message
consisting of $Na$ paired with a nonce generated by Bob~($Nb$), 
encrypted using Alice's public key~($Ka$). In the last message, Alice
returns $Nb$ to Bob, encrypted using his public key.

When Alice receives Message~2, she knows that Bob has acted on her
message, since only he could have decrypted
$\comp{Na,A}\sb{Kb}$ and extracted~$Na$.  That is precisely what
nonces are for.  Similarly, message~3 assures Bob that Alice is
active.  But the protocol was widely believed~\cite{ban89} to satisfy a
further property: that
$Na$ and~$Nb$ were secrets shared by Alice and Bob.  (Many
protocols generate such shared secrets, which can be used
to lessen the reliance on slow public-key operations.)  
Lowe\index{Lowe, Gavin|(} found this
claim to be false: if Alice runs the protocol with someone untrustworthy
(Charlie say), then he can start a new run with another agent (Bob say). 
Charlie uses Alice as an oracle, masquerading as
Alice to Bob~\cite{lowe-fdr}.
\begin{alignat*}{4}
  &1.&\quad  A\to C  &: \comp{Na,A}\sb{Kc}   &&
      \qquad 1'.&\quad  C\to B  &: \comp{Na,A}\sb{Kb} \\
  &2.&\quad  B\to A  &: \comp{Na,Nb}\sb{Ka} \\
  &3.&\quad  A\to C  &: \comp{Nb}\sb{Kc}  &&
      \qquad 3'.&\quad  C\to B  &: \comp{Nb}\sb{Kb}
\end{alignat*}
In messages~1 and~3, Charlie removes the encryption using his private
key and re-encrypts Alice's messages using Bob's public key. Bob is
left thinking he has run the protocol with Alice, which was not
Alice's intention, and Bob is unaware that the ``secret'' nonces are
known to Charlie.  This is a typical man-in-the-middle attack launched
by an insider.

Whether this counts as an attack has been disputed.  In protocols of this
type, we normally assume that the other party is honest.  To be honest
means to obey the protocol rules, so Alice's running the protocol with
Charlie does not make her dishonest, just careless.  After Lowe's
attack, Alice has no grounds for complaint: this protocol does not have to
guarantee anything if you run it with a bad person.  Bob does have
grounds for complaint, however: the protocol tells him that he is
communicating with Alice (who is honest) but it does not guarantee
secrecy of the nonces.

Lowe also suggested a correction, namely to include Bob's name in
message~2:
\begin{alignat*}{2}
  &1.&\quad  A\to B  &: \comp{Na,A}\sb{Kb} \\
  &2.&\quad  B\to A  &: \comp{Na,Nb,B}\sb{Ka} \\
  &3.&\quad  A\to B  &: \comp{Nb}\sb{Kb}
\end{alignat*}
If Charlie tries the same attack, Alice will receive the message
$\comp{Na,Nb,B}\sb{Ka}$ when she was expecting to receive
$\comp{Na,Nb,C}\sb{Ka}$.  She will abandon the run, and eventually so
will Bob.  Below, we shall look at parts of this protocol's correctness
proof. 

In ground-breaking work, Lowe~\cite{lowe-fdr}\index{Lowe, Gavin|)}
showed how such attacks
could be found automatically using a model checker.  An alternative,
which we shall examine below, is to prove protocols correct.  Proofs
can be done under more realistic assumptions because our model does
not have to be finite.  The strategy is to formalize the operational
semantics of the system and to prove security properties using rule
induction.%
\index{Needham-Schroeder protocol|)}


%
\begin{isabellebody}%
\def\isabellecontext{Message}%
%
\isadelimtheory
%
\endisadelimtheory
%
\isatagtheory
%
\endisatagtheory
{\isafoldtheory}%
%
\isadelimtheory
%
\endisadelimtheory
%
\isadelimML
%
\endisadelimML
%
\isatagML
%
\endisatagML
{\isafoldML}%
%
\isadelimML
%
\endisadelimML
%
\isadelimproof
%
\endisadelimproof
%
\isatagproof
%
\endisatagproof
{\isafoldproof}%
%
\isadelimproof
%
\endisadelimproof
%
\isamarkupsection{Agents and Messages%
}
\isamarkuptrue%
%
\begin{isamarkuptext}%
All protocol specifications refer to a syntactic theory of messages. 
Datatype
\isa{agent} introduces the constant \isa{Server} (a trusted central
machine, needed for some protocols), an infinite population of
friendly agents, and the~\isa{Spy}:%
\end{isamarkuptext}%
\isamarkuptrue%
\isacommand{datatype}\isamarkupfalse%
\ agent\ {\isaliteral{3D}{\isacharequal}}\ Server\ {\isaliteral{7C}{\isacharbar}}\ Friend\ nat\ {\isaliteral{7C}{\isacharbar}}\ Spy%
\begin{isamarkuptext}%
Keys are just natural numbers.  Function \isa{invKey} maps a public key to
the matching private key, and vice versa:%
\end{isamarkuptext}%
\isamarkuptrue%
\isacommand{type{\isaliteral{5F}{\isacharunderscore}}synonym}\isamarkupfalse%
\ key\ {\isaliteral{3D}{\isacharequal}}\ nat\isanewline
\isacommand{consts}\isamarkupfalse%
\ invKey\ {\isaliteral{3A}{\isacharcolon}}{\isaliteral{3A}{\isacharcolon}}\ {\isaliteral{22}{\isachardoublequoteopen}}key\ {\isaliteral{5C3C52696768746172726F773E}{\isasymRightarrow}}\ key{\isaliteral{22}{\isachardoublequoteclose}}%
\isadelimproof
%
\endisadelimproof
%
\isatagproof
%
\endisatagproof
{\isafoldproof}%
%
\isadelimproof
%
\endisadelimproof
%
\begin{isamarkuptext}%
Datatype
\isa{msg} introduces the message forms, which include agent names, nonces,
keys, compound messages, and encryptions.%
\end{isamarkuptext}%
\isamarkuptrue%
\isacommand{datatype}\isamarkupfalse%
\isanewline
\ \ \ \ \ msg\ {\isaliteral{3D}{\isacharequal}}\ Agent\ \ agent\isanewline
\ \ \ \ \ \ \ \ \ {\isaliteral{7C}{\isacharbar}}\ Nonce\ \ nat\isanewline
\ \ \ \ \ \ \ \ \ {\isaliteral{7C}{\isacharbar}}\ Key\ \ \ \ key\isanewline
\ \ \ \ \ \ \ \ \ {\isaliteral{7C}{\isacharbar}}\ MPair\ \ msg\ msg\isanewline
\ \ \ \ \ \ \ \ \ {\isaliteral{7C}{\isacharbar}}\ Crypt\ \ key\ msg%
\begin{isamarkuptext}%
\noindent
The notation $\comp{X\sb 1,\ldots X\sb{n-1},X\sb n}$
abbreviates
$\isa{MPair}\,X\sb 1\,\ldots\allowbreak(\isa{MPair}\,X\sb{n-1}\,X\sb n)$.

Since datatype constructors are injective, we have the theorem
\begin{isabelle}%
Crypt\ K\ X\ {\isaliteral{3D}{\isacharequal}}\ Crypt\ K{\isaliteral{27}{\isacharprime}}\ X{\isaliteral{27}{\isacharprime}}\ {\isaliteral{5C3C4C6F6E6772696768746172726F773E}{\isasymLongrightarrow}}\ K\ {\isaliteral{3D}{\isacharequal}}\ K{\isaliteral{27}{\isacharprime}}\ {\isaliteral{5C3C616E643E}{\isasymand}}\ X\ {\isaliteral{3D}{\isacharequal}}\ X{\isaliteral{27}{\isacharprime}}%
\end{isabelle}
A ciphertext can be decrypted using only one key and
can yield only one plaintext.  In the real world, decryption with the
wrong key succeeds but yields garbage.  Our model of encryption is
realistic if encryption adds some redundancy to the plaintext, such as a
checksum, so that garbage can be detected.%
\end{isamarkuptext}%
\isamarkuptrue%
%
\isadelimproof
%
\endisadelimproof
%
\isatagproof
%
\endisatagproof
{\isafoldproof}%
%
\isadelimproof
%
\endisadelimproof
%
\isadelimproof
%
\endisadelimproof
%
\isatagproof
%
\endisatagproof
{\isafoldproof}%
%
\isadelimproof
%
\endisadelimproof
%
\isadelimproof
%
\endisadelimproof
%
\isatagproof
%
\endisatagproof
{\isafoldproof}%
%
\isadelimproof
%
\endisadelimproof
%
\isadelimproof
%
\endisadelimproof
%
\isatagproof
%
\endisatagproof
{\isafoldproof}%
%
\isadelimproof
%
\endisadelimproof
%
\isadelimproof
%
\endisadelimproof
%
\isatagproof
%
\endisatagproof
{\isafoldproof}%
%
\isadelimproof
%
\endisadelimproof
%
\isadelimproof
%
\endisadelimproof
%
\isatagproof
%
\endisatagproof
{\isafoldproof}%
%
\isadelimproof
%
\endisadelimproof
%
\isadelimproof
%
\endisadelimproof
%
\isatagproof
%
\endisatagproof
{\isafoldproof}%
%
\isadelimproof
%
\endisadelimproof
%
\isadelimproof
%
\endisadelimproof
%
\isatagproof
%
\endisatagproof
{\isafoldproof}%
%
\isadelimproof
%
\endisadelimproof
%
\isadelimproof
%
\endisadelimproof
%
\isatagproof
%
\endisatagproof
{\isafoldproof}%
%
\isadelimproof
%
\endisadelimproof
%
\isadelimproof
%
\endisadelimproof
%
\isatagproof
%
\endisatagproof
{\isafoldproof}%
%
\isadelimproof
%
\endisadelimproof
%
\isadelimproof
%
\endisadelimproof
%
\isatagproof
%
\endisatagproof
{\isafoldproof}%
%
\isadelimproof
%
\endisadelimproof
%
\isadelimproof
%
\endisadelimproof
%
\isatagproof
%
\endisatagproof
{\isafoldproof}%
%
\isadelimproof
%
\endisadelimproof
%
\isadelimproof
%
\endisadelimproof
%
\isatagproof
%
\endisatagproof
{\isafoldproof}%
%
\isadelimproof
%
\endisadelimproof
%
\isadelimproof
%
\endisadelimproof
%
\isatagproof
%
\endisatagproof
{\isafoldproof}%
%
\isadelimproof
%
\endisadelimproof
%
\isadelimproof
%
\endisadelimproof
%
\isatagproof
%
\endisatagproof
{\isafoldproof}%
%
\isadelimproof
%
\endisadelimproof
%
\isadelimproof
%
\endisadelimproof
%
\isatagproof
%
\endisatagproof
{\isafoldproof}%
%
\isadelimproof
%
\endisadelimproof
%
\isadelimproof
%
\endisadelimproof
%
\isatagproof
%
\endisatagproof
{\isafoldproof}%
%
\isadelimproof
%
\endisadelimproof
%
\isadelimproof
%
\endisadelimproof
%
\isatagproof
%
\endisatagproof
{\isafoldproof}%
%
\isadelimproof
%
\endisadelimproof
%
\isadelimproof
%
\endisadelimproof
%
\isatagproof
%
\endisatagproof
{\isafoldproof}%
%
\isadelimproof
%
\endisadelimproof
%
\isadelimproof
%
\endisadelimproof
%
\isatagproof
%
\endisatagproof
{\isafoldproof}%
%
\isadelimproof
%
\endisadelimproof
%
\isadelimproof
%
\endisadelimproof
%
\isatagproof
%
\endisatagproof
{\isafoldproof}%
%
\isadelimproof
%
\endisadelimproof
%
\isadelimproof
%
\endisadelimproof
%
\isatagproof
%
\endisatagproof
{\isafoldproof}%
%
\isadelimproof
%
\endisadelimproof
%
\isadelimproof
%
\endisadelimproof
%
\isatagproof
%
\endisatagproof
{\isafoldproof}%
%
\isadelimproof
%
\endisadelimproof
%
\isadelimproof
%
\endisadelimproof
%
\isatagproof
%
\endisatagproof
{\isafoldproof}%
%
\isadelimproof
%
\endisadelimproof
%
\isadelimproof
%
\endisadelimproof
%
\isatagproof
%
\endisatagproof
{\isafoldproof}%
%
\isadelimproof
%
\endisadelimproof
%
\isadelimproof
%
\endisadelimproof
%
\isatagproof
%
\endisatagproof
{\isafoldproof}%
%
\isadelimproof
%
\endisadelimproof
%
\isadelimproof
%
\endisadelimproof
%
\isatagproof
%
\endisatagproof
{\isafoldproof}%
%
\isadelimproof
%
\endisadelimproof
%
\isadelimproof
%
\endisadelimproof
%
\isatagproof
%
\endisatagproof
{\isafoldproof}%
%
\isadelimproof
%
\endisadelimproof
%
\isadelimproof
%
\endisadelimproof
%
\isatagproof
%
\endisatagproof
{\isafoldproof}%
%
\isadelimproof
%
\endisadelimproof
%
\isadelimproof
%
\endisadelimproof
%
\isatagproof
%
\endisatagproof
{\isafoldproof}%
%
\isadelimproof
%
\endisadelimproof
%
\isadelimproof
%
\endisadelimproof
%
\isatagproof
%
\endisatagproof
{\isafoldproof}%
%
\isadelimproof
%
\endisadelimproof
%
\isadelimproof
%
\endisadelimproof
%
\isatagproof
%
\endisatagproof
{\isafoldproof}%
%
\isadelimproof
%
\endisadelimproof
%
\isadelimproof
%
\endisadelimproof
%
\isatagproof
%
\endisatagproof
{\isafoldproof}%
%
\isadelimproof
%
\endisadelimproof
%
\isadelimproof
%
\endisadelimproof
%
\isatagproof
%
\endisatagproof
{\isafoldproof}%
%
\isadelimproof
%
\endisadelimproof
%
\isadelimproof
%
\endisadelimproof
%
\isatagproof
%
\endisatagproof
{\isafoldproof}%
%
\isadelimproof
%
\endisadelimproof
%
\isadelimproof
%
\endisadelimproof
%
\isatagproof
%
\endisatagproof
{\isafoldproof}%
%
\isadelimproof
%
\endisadelimproof
%
\isadelimproof
%
\endisadelimproof
%
\isatagproof
%
\endisatagproof
{\isafoldproof}%
%
\isadelimproof
%
\endisadelimproof
%
\isadelimproof
%
\endisadelimproof
%
\isatagproof
%
\endisatagproof
{\isafoldproof}%
%
\isadelimproof
%
\endisadelimproof
%
\isadelimproof
%
\endisadelimproof
%
\isatagproof
%
\endisatagproof
{\isafoldproof}%
%
\isadelimproof
%
\endisadelimproof
%
\isadelimproof
%
\endisadelimproof
%
\isatagproof
%
\endisatagproof
{\isafoldproof}%
%
\isadelimproof
%
\endisadelimproof
%
\isadelimproof
%
\endisadelimproof
%
\isatagproof
%
\endisatagproof
{\isafoldproof}%
%
\isadelimproof
%
\endisadelimproof
%
\isadelimproof
%
\endisadelimproof
%
\isatagproof
%
\endisatagproof
{\isafoldproof}%
%
\isadelimproof
%
\endisadelimproof
%
\isadelimproof
%
\endisadelimproof
%
\isatagproof
%
\endisatagproof
{\isafoldproof}%
%
\isadelimproof
%
\endisadelimproof
%
\isamarkupsection{Modelling the Adversary%
}
\isamarkuptrue%
%
\begin{isamarkuptext}%
The spy is part of the system and must be built into the model.  He is
a malicious user who does not have to follow the protocol.  He
watches the network and uses any keys he knows to decrypt messages.
Thus he accumulates additional keys and nonces.  These he can use to
compose new messages, which he may send to anybody.  

Two functions enable us to formalize this behaviour: \isa{analz} and
\isa{synth}.  Each function maps a sets of messages to another set of
messages. The set \isa{analz\ H} formalizes what the adversary can learn
from the set of messages~$H$.  The closure properties of this set are
defined inductively.%
\end{isamarkuptext}%
\isamarkuptrue%
\isacommand{inductive{\isaliteral{5F}{\isacharunderscore}}set}\isamarkupfalse%
\isanewline
\ \ analz\ {\isaliteral{3A}{\isacharcolon}}{\isaliteral{3A}{\isacharcolon}}\ {\isaliteral{22}{\isachardoublequoteopen}}msg\ set\ {\isaliteral{5C3C52696768746172726F773E}{\isasymRightarrow}}\ msg\ set{\isaliteral{22}{\isachardoublequoteclose}}\isanewline
\ \ \isakeyword{for}\ H\ {\isaliteral{3A}{\isacharcolon}}{\isaliteral{3A}{\isacharcolon}}\ {\isaliteral{22}{\isachardoublequoteopen}}msg\ set{\isaliteral{22}{\isachardoublequoteclose}}\isanewline
\ \ \isakeyword{where}\isanewline
\ \ \ \ Inj\ {\isaliteral{5B}{\isacharbrackleft}}intro{\isaliteral{2C}{\isacharcomma}}simp{\isaliteral{5D}{\isacharbrackright}}\ {\isaliteral{3A}{\isacharcolon}}\ {\isaliteral{22}{\isachardoublequoteopen}}X\ {\isaliteral{5C3C696E3E}{\isasymin}}\ H\ {\isaliteral{5C3C4C6F6E6772696768746172726F773E}{\isasymLongrightarrow}}\ X\ {\isaliteral{5C3C696E3E}{\isasymin}}\ analz\ H{\isaliteral{22}{\isachardoublequoteclose}}\isanewline
\ \ {\isaliteral{7C}{\isacharbar}}\ Fst{\isaliteral{3A}{\isacharcolon}}\ \ \ \ \ {\isaliteral{22}{\isachardoublequoteopen}}{\isaliteral{5C3C6C62726163653E}{\isasymlbrace}}X{\isaliteral{2C}{\isacharcomma}}Y{\isaliteral{5C3C7262726163653E}{\isasymrbrace}}\ {\isaliteral{5C3C696E3E}{\isasymin}}\ analz\ H\ {\isaliteral{5C3C4C6F6E6772696768746172726F773E}{\isasymLongrightarrow}}\ X\ {\isaliteral{5C3C696E3E}{\isasymin}}\ analz\ H{\isaliteral{22}{\isachardoublequoteclose}}\isanewline
\ \ {\isaliteral{7C}{\isacharbar}}\ Snd{\isaliteral{3A}{\isacharcolon}}\ \ \ \ \ {\isaliteral{22}{\isachardoublequoteopen}}{\isaliteral{5C3C6C62726163653E}{\isasymlbrace}}X{\isaliteral{2C}{\isacharcomma}}Y{\isaliteral{5C3C7262726163653E}{\isasymrbrace}}\ {\isaliteral{5C3C696E3E}{\isasymin}}\ analz\ H\ {\isaliteral{5C3C4C6F6E6772696768746172726F773E}{\isasymLongrightarrow}}\ Y\ {\isaliteral{5C3C696E3E}{\isasymin}}\ analz\ H{\isaliteral{22}{\isachardoublequoteclose}}\isanewline
\ \ {\isaliteral{7C}{\isacharbar}}\ Decrypt\ {\isaliteral{5B}{\isacharbrackleft}}dest{\isaliteral{5D}{\isacharbrackright}}{\isaliteral{3A}{\isacharcolon}}\ \isanewline
\ \ \ \ \ \ \ \ \ \ \ \ \ {\isaliteral{22}{\isachardoublequoteopen}}{\isaliteral{5C3C6C6272616B6B3E}{\isasymlbrakk}}Crypt\ K\ X\ {\isaliteral{5C3C696E3E}{\isasymin}}\ analz\ H{\isaliteral{3B}{\isacharsemicolon}}\ Key{\isaliteral{28}{\isacharparenleft}}invKey\ K{\isaliteral{29}{\isacharparenright}}\ {\isaliteral{5C3C696E3E}{\isasymin}}\ analz\ H{\isaliteral{5C3C726272616B6B3E}{\isasymrbrakk}}\isanewline
\ \ \ \ \ \ \ \ \ \ \ \ \ \ {\isaliteral{5C3C4C6F6E6772696768746172726F773E}{\isasymLongrightarrow}}\ X\ {\isaliteral{5C3C696E3E}{\isasymin}}\ analz\ H{\isaliteral{22}{\isachardoublequoteclose}}%
\isadelimproof
%
\endisadelimproof
%
\isatagproof
%
\endisatagproof
{\isafoldproof}%
%
\isadelimproof
%
\endisadelimproof
%
\isadelimproof
%
\endisadelimproof
%
\isatagproof
%
\endisatagproof
{\isafoldproof}%
%
\isadelimproof
%
\endisadelimproof
%
\isadelimproof
%
\endisadelimproof
%
\isatagproof
%
\endisatagproof
{\isafoldproof}%
%
\isadelimproof
%
\endisadelimproof
%
\isadelimproof
%
\endisadelimproof
%
\isatagproof
%
\endisatagproof
{\isafoldproof}%
%
\isadelimproof
%
\endisadelimproof
%
\isadelimproof
%
\endisadelimproof
%
\isatagproof
%
\endisatagproof
{\isafoldproof}%
%
\isadelimproof
%
\endisadelimproof
%
\isadelimproof
%
\endisadelimproof
%
\isatagproof
%
\endisatagproof
{\isafoldproof}%
%
\isadelimproof
%
\endisadelimproof
%
\isadelimproof
%
\endisadelimproof
%
\isatagproof
%
\endisatagproof
{\isafoldproof}%
%
\isadelimproof
%
\endisadelimproof
%
\isadelimproof
%
\endisadelimproof
%
\isatagproof
%
\endisatagproof
{\isafoldproof}%
%
\isadelimproof
%
\endisadelimproof
%
\isadelimproof
%
\endisadelimproof
%
\isatagproof
%
\endisatagproof
{\isafoldproof}%
%
\isadelimproof
%
\endisadelimproof
%
\isadelimproof
%
\endisadelimproof
%
\isatagproof
%
\endisatagproof
{\isafoldproof}%
%
\isadelimproof
%
\endisadelimproof
%
\isadelimproof
%
\endisadelimproof
%
\isatagproof
%
\endisatagproof
{\isafoldproof}%
%
\isadelimproof
%
\endisadelimproof
%
\isadelimproof
%
\endisadelimproof
%
\isatagproof
%
\endisatagproof
{\isafoldproof}%
%
\isadelimproof
%
\endisadelimproof
%
\isadelimproof
%
\endisadelimproof
%
\isatagproof
%
\endisatagproof
{\isafoldproof}%
%
\isadelimproof
%
\endisadelimproof
%
\isadelimproof
%
\endisadelimproof
%
\isatagproof
%
\endisatagproof
{\isafoldproof}%
%
\isadelimproof
%
\endisadelimproof
%
\isadelimproof
%
\endisadelimproof
%
\isatagproof
%
\endisatagproof
{\isafoldproof}%
%
\isadelimproof
%
\endisadelimproof
%
\isadelimproof
%
\endisadelimproof
%
\isatagproof
%
\endisatagproof
{\isafoldproof}%
%
\isadelimproof
%
\endisadelimproof
%
\isadelimproof
%
\endisadelimproof
%
\isatagproof
%
\endisatagproof
{\isafoldproof}%
%
\isadelimproof
%
\endisadelimproof
%
\isadelimproof
%
\endisadelimproof
%
\isatagproof
%
\endisatagproof
{\isafoldproof}%
%
\isadelimproof
%
\endisadelimproof
%
\isadelimproof
%
\endisadelimproof
%
\isatagproof
%
\endisatagproof
{\isafoldproof}%
%
\isadelimproof
%
\endisadelimproof
%
\isadelimproof
%
\endisadelimproof
%
\isatagproof
%
\endisatagproof
{\isafoldproof}%
%
\isadelimproof
%
\endisadelimproof
%
\isadelimproof
%
\endisadelimproof
%
\isatagproof
%
\endisatagproof
{\isafoldproof}%
%
\isadelimproof
%
\endisadelimproof
%
\isadelimproof
%
\endisadelimproof
%
\isatagproof
%
\endisatagproof
{\isafoldproof}%
%
\isadelimproof
%
\endisadelimproof
%
\isadelimproof
%
\endisadelimproof
%
\isatagproof
%
\endisatagproof
{\isafoldproof}%
%
\isadelimproof
%
\endisadelimproof
%
\isadelimproof
%
\endisadelimproof
%
\isatagproof
%
\endisatagproof
{\isafoldproof}%
%
\isadelimproof
%
\endisadelimproof
%
\isadelimproof
%
\endisadelimproof
%
\isatagproof
%
\endisatagproof
{\isafoldproof}%
%
\isadelimproof
%
\endisadelimproof
%
\isadelimproof
%
\endisadelimproof
%
\isatagproof
%
\endisatagproof
{\isafoldproof}%
%
\isadelimproof
%
\endisadelimproof
%
\isadelimproof
%
\endisadelimproof
%
\isatagproof
%
\endisatagproof
{\isafoldproof}%
%
\isadelimproof
%
\endisadelimproof
%
\isadelimproof
%
\endisadelimproof
%
\isatagproof
%
\endisatagproof
{\isafoldproof}%
%
\isadelimproof
%
\endisadelimproof
%
\isadelimproof
%
\endisadelimproof
%
\isatagproof
%
\endisatagproof
{\isafoldproof}%
%
\isadelimproof
%
\endisadelimproof
%
\isadelimproof
%
\endisadelimproof
%
\isatagproof
%
\endisatagproof
{\isafoldproof}%
%
\isadelimproof
%
\endisadelimproof
%
\isadelimproof
%
\endisadelimproof
%
\isatagproof
%
\endisatagproof
{\isafoldproof}%
%
\isadelimproof
%
\endisadelimproof
%
\isadelimproof
%
\endisadelimproof
%
\isatagproof
%
\endisatagproof
{\isafoldproof}%
%
\isadelimproof
%
\endisadelimproof
%
\begin{isamarkuptext}%
Note the \isa{Decrypt} rule: the spy can decrypt a
message encrypted with key~$K$ if he has the matching key,~$K^{-1}$. 
Properties proved by rule induction include the following:
\begin{isabelle}%
G\ {\isaliteral{5C3C73756273657465713E}{\isasymsubseteq}}\ H\ {\isaliteral{5C3C4C6F6E6772696768746172726F773E}{\isasymLongrightarrow}}\ analz\ G\ {\isaliteral{5C3C73756273657465713E}{\isasymsubseteq}}\ analz\ H\rulename{analz{\isaliteral{5F}{\isacharunderscore}}mono}\par\smallskip%
analz\ {\isaliteral{28}{\isacharparenleft}}analz\ H{\isaliteral{29}{\isacharparenright}}\ {\isaliteral{3D}{\isacharequal}}\ analz\ H\rulename{analz{\isaliteral{5F}{\isacharunderscore}}idem}%
\end{isabelle}

The set of fake messages that an intruder could invent
starting from~\isa{H} is \isa{synth{\isaliteral{28}{\isacharparenleft}}analz\ H{\isaliteral{29}{\isacharparenright}}}, where \isa{synth\ H}
formalizes what the adversary can build from the set of messages~$H$.%
\end{isamarkuptext}%
\isamarkuptrue%
\isacommand{inductive{\isaliteral{5F}{\isacharunderscore}}set}\isamarkupfalse%
\isanewline
\ \ synth\ {\isaliteral{3A}{\isacharcolon}}{\isaliteral{3A}{\isacharcolon}}\ {\isaliteral{22}{\isachardoublequoteopen}}msg\ set\ {\isaliteral{5C3C52696768746172726F773E}{\isasymRightarrow}}\ msg\ set{\isaliteral{22}{\isachardoublequoteclose}}\isanewline
\ \ \isakeyword{for}\ H\ {\isaliteral{3A}{\isacharcolon}}{\isaliteral{3A}{\isacharcolon}}\ {\isaliteral{22}{\isachardoublequoteopen}}msg\ set{\isaliteral{22}{\isachardoublequoteclose}}\isanewline
\ \ \isakeyword{where}\isanewline
\ \ \ \ Inj\ \ \ \ {\isaliteral{5B}{\isacharbrackleft}}intro{\isaliteral{5D}{\isacharbrackright}}{\isaliteral{3A}{\isacharcolon}}\ {\isaliteral{22}{\isachardoublequoteopen}}X\ {\isaliteral{5C3C696E3E}{\isasymin}}\ H\ {\isaliteral{5C3C4C6F6E6772696768746172726F773E}{\isasymLongrightarrow}}\ X\ {\isaliteral{5C3C696E3E}{\isasymin}}\ synth\ H{\isaliteral{22}{\isachardoublequoteclose}}\isanewline
\ \ {\isaliteral{7C}{\isacharbar}}\ Agent\ \ {\isaliteral{5B}{\isacharbrackleft}}intro{\isaliteral{5D}{\isacharbrackright}}{\isaliteral{3A}{\isacharcolon}}\ {\isaliteral{22}{\isachardoublequoteopen}}Agent\ agt\ {\isaliteral{5C3C696E3E}{\isasymin}}\ synth\ H{\isaliteral{22}{\isachardoublequoteclose}}\isanewline
\ \ {\isaliteral{7C}{\isacharbar}}\ MPair\ \ {\isaliteral{5B}{\isacharbrackleft}}intro{\isaliteral{5D}{\isacharbrackright}}{\isaliteral{3A}{\isacharcolon}}\isanewline
\ \ \ \ \ \ \ \ \ \ \ \ \ \ {\isaliteral{22}{\isachardoublequoteopen}}{\isaliteral{5C3C6C6272616B6B3E}{\isasymlbrakk}}X\ {\isaliteral{5C3C696E3E}{\isasymin}}\ synth\ H{\isaliteral{3B}{\isacharsemicolon}}\ \ Y\ {\isaliteral{5C3C696E3E}{\isasymin}}\ synth\ H{\isaliteral{5C3C726272616B6B3E}{\isasymrbrakk}}\ {\isaliteral{5C3C4C6F6E6772696768746172726F773E}{\isasymLongrightarrow}}\ {\isaliteral{5C3C6C62726163653E}{\isasymlbrace}}X{\isaliteral{2C}{\isacharcomma}}Y{\isaliteral{5C3C7262726163653E}{\isasymrbrace}}\ {\isaliteral{5C3C696E3E}{\isasymin}}\ synth\ H{\isaliteral{22}{\isachardoublequoteclose}}\isanewline
\ \ {\isaliteral{7C}{\isacharbar}}\ Crypt\ \ {\isaliteral{5B}{\isacharbrackleft}}intro{\isaliteral{5D}{\isacharbrackright}}{\isaliteral{3A}{\isacharcolon}}\isanewline
\ \ \ \ \ \ \ \ \ \ \ \ \ \ {\isaliteral{22}{\isachardoublequoteopen}}{\isaliteral{5C3C6C6272616B6B3E}{\isasymlbrakk}}X\ {\isaliteral{5C3C696E3E}{\isasymin}}\ synth\ H{\isaliteral{3B}{\isacharsemicolon}}\ \ Key\ K\ {\isaliteral{5C3C696E3E}{\isasymin}}\ H{\isaliteral{5C3C726272616B6B3E}{\isasymrbrakk}}\ {\isaliteral{5C3C4C6F6E6772696768746172726F773E}{\isasymLongrightarrow}}\ Crypt\ K\ X\ {\isaliteral{5C3C696E3E}{\isasymin}}\ synth\ H{\isaliteral{22}{\isachardoublequoteclose}}%
\isadelimproof
%
\endisadelimproof
%
\isatagproof
%
\endisatagproof
{\isafoldproof}%
%
\isadelimproof
%
\endisadelimproof
%
\isadelimproof
%
\endisadelimproof
%
\isatagproof
%
\endisatagproof
{\isafoldproof}%
%
\isadelimproof
%
\endisadelimproof
%
\isadelimproof
%
\endisadelimproof
%
\isatagproof
%
\endisatagproof
{\isafoldproof}%
%
\isadelimproof
%
\endisadelimproof
%
\begin{isamarkuptext}%
The set includes all agent names.  Nonces and keys are assumed to be
unguessable, so none are included beyond those already in~$H$.   Two
elements of \isa{synth\ H} can be combined, and an element can be encrypted
using a key present in~$H$.

Like \isa{analz}, this set operator is monotone and idempotent.  It also
satisfies an interesting equation involving \isa{analz}:
\begin{isabelle}%
analz\ {\isaliteral{28}{\isacharparenleft}}synth\ H{\isaliteral{29}{\isacharparenright}}\ {\isaliteral{3D}{\isacharequal}}\ analz\ H\ {\isaliteral{5C3C756E696F6E3E}{\isasymunion}}\ synth\ H\rulename{analz{\isaliteral{5F}{\isacharunderscore}}synth}%
\end{isabelle}
Rule inversion plays a major role in reasoning about \isa{synth}, through
declarations such as this one:%
\end{isamarkuptext}%
\isamarkuptrue%
\isacommand{inductive{\isaliteral{5F}{\isacharunderscore}}cases}\isamarkupfalse%
\ Nonce{\isaliteral{5F}{\isacharunderscore}}synth\ {\isaliteral{5B}{\isacharbrackleft}}elim{\isaliteral{21}{\isacharbang}}{\isaliteral{5D}{\isacharbrackright}}{\isaliteral{3A}{\isacharcolon}}\ {\isaliteral{22}{\isachardoublequoteopen}}Nonce\ n\ {\isaliteral{5C3C696E3E}{\isasymin}}\ synth\ H{\isaliteral{22}{\isachardoublequoteclose}}%
\begin{isamarkuptext}%
\noindent
The resulting elimination rule replaces every assumption of the form
\isa{Nonce\ n\ {\isaliteral{5C3C696E3E}{\isasymin}}\ synth\ H} by \isa{Nonce\ n\ {\isaliteral{5C3C696E3E}{\isasymin}}\ H},
expressing that a nonce cannot be guessed.  

A third operator, \isa{parts}, is useful for stating correctness
properties.  The set
\isa{parts\ H} consists of the components of elements of~$H$.  This set
includes~\isa{H} and is closed under the projections from a compound
message to its immediate parts. 
Its definition resembles that of \isa{analz} except in the rule
corresponding to the constructor \isa{Crypt}: 
\begin{isabelle}%
\ \ \ \ \ Crypt\ K\ X\ {\isaliteral{5C3C696E3E}{\isasymin}}\ parts\ H\ {\isaliteral{5C3C4C6F6E6772696768746172726F773E}{\isasymLongrightarrow}}\ X\ {\isaliteral{5C3C696E3E}{\isasymin}}\ parts\ H%
\end{isabelle}
The body of an encrypted message is always regarded as part of it.  We can
use \isa{parts} to express general well-formedness properties of a protocol,
for example, that an uncompromised agent's private key will never be
included as a component of any message.%
\end{isamarkuptext}%
\isamarkuptrue%
%
\isadelimproof
%
\endisadelimproof
%
\isatagproof
%
\endisatagproof
{\isafoldproof}%
%
\isadelimproof
%
\endisadelimproof
%
\isadelimproof
%
\endisadelimproof
%
\isatagproof
%
\endisatagproof
{\isafoldproof}%
%
\isadelimproof
%
\endisadelimproof
%
\isadelimproof
%
\endisadelimproof
%
\isatagproof
%
\endisatagproof
{\isafoldproof}%
%
\isadelimproof
%
\endisadelimproof
%
\isadelimproof
%
\endisadelimproof
%
\isatagproof
%
\endisatagproof
{\isafoldproof}%
%
\isadelimproof
%
\endisadelimproof
%
\isadelimproof
%
\endisadelimproof
%
\isatagproof
%
\endisatagproof
{\isafoldproof}%
%
\isadelimproof
%
\endisadelimproof
%
\isadelimproof
%
\endisadelimproof
%
\isatagproof
%
\endisatagproof
{\isafoldproof}%
%
\isadelimproof
%
\endisadelimproof
%
\isadelimproof
%
\endisadelimproof
%
\isatagproof
%
\endisatagproof
{\isafoldproof}%
%
\isadelimproof
%
\endisadelimproof
%
\isadelimproof
%
\endisadelimproof
%
\isatagproof
%
\endisatagproof
{\isafoldproof}%
%
\isadelimproof
%
\endisadelimproof
%
\isadelimproof
%
\endisadelimproof
%
\isatagproof
%
\endisatagproof
{\isafoldproof}%
%
\isadelimproof
%
\endisadelimproof
%
\isadelimproof
%
\endisadelimproof
%
\isatagproof
%
\endisatagproof
{\isafoldproof}%
%
\isadelimproof
%
\endisadelimproof
%
\isadelimproof
%
\endisadelimproof
%
\isatagproof
%
\endisatagproof
{\isafoldproof}%
%
\isadelimproof
%
\endisadelimproof
%
\isadelimproof
%
\endisadelimproof
%
\isatagproof
%
\endisatagproof
{\isafoldproof}%
%
\isadelimproof
%
\endisadelimproof
%
\isadelimproof
%
\endisadelimproof
%
\isatagproof
%
\endisatagproof
{\isafoldproof}%
%
\isadelimproof
%
\endisadelimproof
%
\isadelimproof
%
\endisadelimproof
%
\isatagproof
%
\endisatagproof
{\isafoldproof}%
%
\isadelimproof
%
\endisadelimproof
%
\isadelimproof
%
\endisadelimproof
%
\isatagproof
%
\endisatagproof
{\isafoldproof}%
%
\isadelimproof
%
\endisadelimproof
%
\isadelimproof
%
\endisadelimproof
%
\isatagproof
%
\endisatagproof
{\isafoldproof}%
%
\isadelimproof
%
\endisadelimproof
%
\isadelimproof
%
\endisadelimproof
%
\isatagproof
%
\endisatagproof
{\isafoldproof}%
%
\isadelimproof
%
\endisadelimproof
%
\isadelimproof
%
\endisadelimproof
%
\isatagproof
%
\endisatagproof
{\isafoldproof}%
%
\isadelimproof
%
\endisadelimproof
%
\isadelimproof
%
\endisadelimproof
%
\isatagproof
%
\endisatagproof
{\isafoldproof}%
%
\isadelimproof
%
\endisadelimproof
%
\isadelimproof
%
\endisadelimproof
%
\isatagproof
%
\endisatagproof
{\isafoldproof}%
%
\isadelimproof
%
\endisadelimproof
%
\isadelimproof
%
\endisadelimproof
%
\isatagproof
%
\endisatagproof
{\isafoldproof}%
%
\isadelimproof
%
\endisadelimproof
%
\isadelimproof
%
\endisadelimproof
%
\isatagproof
%
\endisatagproof
{\isafoldproof}%
%
\isadelimproof
%
\endisadelimproof
%
\isadelimproof
%
\endisadelimproof
%
\isatagproof
%
\endisatagproof
{\isafoldproof}%
%
\isadelimproof
%
\endisadelimproof
%
\isadelimML
%
\endisadelimML
%
\isatagML
%
\endisatagML
{\isafoldML}%
%
\isadelimML
%
\endisadelimML
%
\isadelimproof
%
\endisadelimproof
%
\isatagproof
%
\endisatagproof
{\isafoldproof}%
%
\isadelimproof
%
\endisadelimproof
%
\isadelimproof
%
\endisadelimproof
%
\isatagproof
%
\endisatagproof
{\isafoldproof}%
%
\isadelimproof
%
\endisadelimproof
%
\isadelimproof
%
\endisadelimproof
%
\isatagproof
%
\endisatagproof
{\isafoldproof}%
%
\isadelimproof
%
\endisadelimproof
%
\isadelimproof
%
\endisadelimproof
%
\isatagproof
%
\endisatagproof
{\isafoldproof}%
%
\isadelimproof
%
\endisadelimproof
%
\isadelimproof
%
\endisadelimproof
%
\isatagproof
%
\endisatagproof
{\isafoldproof}%
%
\isadelimproof
%
\endisadelimproof
%
\isadelimproof
%
\endisadelimproof
%
\isatagproof
%
\endisatagproof
{\isafoldproof}%
%
\isadelimproof
%
\endisadelimproof
%
\isadelimML
%
\endisadelimML
%
\isatagML
%
\endisatagML
{\isafoldML}%
%
\isadelimML
%
\endisadelimML
%
\isadelimtheory
%
\endisadelimtheory
%
\isatagtheory
%
\endisatagtheory
{\isafoldtheory}%
%
\isadelimtheory
%
\endisadelimtheory
\end{isabellebody}%
%%% Local Variables:
%%% mode: latex
%%% TeX-master: "root"
%%% End:

%
\begin{isabellebody}%
\def\isabellecontext{Event}%
%
\isadelimtheory
%
\endisadelimtheory
%
\isatagtheory
%
\endisatagtheory
{\isafoldtheory}%
%
\isadelimtheory
%
\endisadelimtheory
%
\isadelimproof
%
\endisadelimproof
%
\isatagproof
%
\endisatagproof
{\isafoldproof}%
%
\isadelimproof
%
\endisadelimproof
%
\isadelimproof
%
\endisadelimproof
%
\isatagproof
%
\endisatagproof
{\isafoldproof}%
%
\isadelimproof
%
\endisadelimproof
%
\isadelimproof
%
\endisadelimproof
%
\isatagproof
%
\endisatagproof
{\isafoldproof}%
%
\isadelimproof
%
\endisadelimproof
%
\isadelimproof
%
\endisadelimproof
%
\isatagproof
%
\endisatagproof
{\isafoldproof}%
%
\isadelimproof
%
\endisadelimproof
%
\isadelimproof
%
\endisadelimproof
%
\isatagproof
%
\endisatagproof
{\isafoldproof}%
%
\isadelimproof
%
\endisadelimproof
%
\isadelimproof
%
\endisadelimproof
%
\isatagproof
%
\endisatagproof
{\isafoldproof}%
%
\isadelimproof
%
\endisadelimproof
%
\isadelimproof
%
\endisadelimproof
%
\isatagproof
%
\endisatagproof
{\isafoldproof}%
%
\isadelimproof
%
\endisadelimproof
%
\isadelimproof
%
\endisadelimproof
%
\isatagproof
%
\endisatagproof
{\isafoldproof}%
%
\isadelimproof
%
\endisadelimproof
%
\isadelimproof
%
\endisadelimproof
%
\isatagproof
%
\endisatagproof
{\isafoldproof}%
%
\isadelimproof
%
\endisadelimproof
%
\isadelimproof
%
\endisadelimproof
%
\isatagproof
%
\endisatagproof
{\isafoldproof}%
%
\isadelimproof
%
\endisadelimproof
%
\isadelimproof
%
\endisadelimproof
%
\isatagproof
%
\endisatagproof
{\isafoldproof}%
%
\isadelimproof
%
\endisadelimproof
%
\isadelimproof
%
\endisadelimproof
%
\isatagproof
%
\endisatagproof
{\isafoldproof}%
%
\isadelimproof
%
\endisadelimproof
%
\isadelimproof
%
\endisadelimproof
%
\isatagproof
%
\endisatagproof
{\isafoldproof}%
%
\isadelimproof
%
\endisadelimproof
%
\isadelimproof
%
\endisadelimproof
%
\isatagproof
%
\endisatagproof
{\isafoldproof}%
%
\isadelimproof
%
\endisadelimproof
%
\isadelimproof
%
\endisadelimproof
%
\isatagproof
%
\endisatagproof
{\isafoldproof}%
%
\isadelimproof
%
\endisadelimproof
%
\isadelimproof
%
\endisadelimproof
%
\isatagproof
%
\endisatagproof
{\isafoldproof}%
%
\isadelimproof
%
\endisadelimproof
%
\isadelimproof
%
\endisadelimproof
%
\isatagproof
%
\endisatagproof
{\isafoldproof}%
%
\isadelimproof
%
\endisadelimproof
%
\isadelimproof
%
\endisadelimproof
%
\isatagproof
%
\endisatagproof
{\isafoldproof}%
%
\isadelimproof
%
\endisadelimproof
%
\isadelimproof
%
\endisadelimproof
%
\isatagproof
%
\endisatagproof
{\isafoldproof}%
%
\isadelimproof
%
\endisadelimproof
%
\isadelimproof
%
\endisadelimproof
%
\isatagproof
%
\endisatagproof
{\isafoldproof}%
%
\isadelimproof
%
\endisadelimproof
%
\isadelimproof
%
\endisadelimproof
%
\isatagproof
%
\endisatagproof
{\isafoldproof}%
%
\isadelimproof
%
\endisadelimproof
%
\isadelimproof
%
\endisadelimproof
%
\isatagproof
%
\endisatagproof
{\isafoldproof}%
%
\isadelimproof
%
\endisadelimproof
%
\isadelimproof
%
\endisadelimproof
%
\isatagproof
%
\endisatagproof
{\isafoldproof}%
%
\isadelimproof
%
\endisadelimproof
%
\isadelimproof
%
\endisadelimproof
%
\isatagproof
%
\endisatagproof
{\isafoldproof}%
%
\isadelimproof
%
\endisadelimproof
%
\isadelimproof
%
\endisadelimproof
%
\isatagproof
%
\endisatagproof
{\isafoldproof}%
%
\isadelimproof
%
\endisadelimproof
%
\isadelimproof
%
\endisadelimproof
%
\isatagproof
%
\endisatagproof
{\isafoldproof}%
%
\isadelimproof
%
\endisadelimproof
%
\isadelimproof
%
\endisadelimproof
%
\isatagproof
%
\endisatagproof
{\isafoldproof}%
%
\isadelimproof
%
\endisadelimproof
%
\isadelimproof
%
\endisadelimproof
%
\isatagproof
%
\endisatagproof
{\isafoldproof}%
%
\isadelimproof
%
\endisadelimproof
%
\isadelimML
%
\endisadelimML
%
\isatagML
%
\endisatagML
{\isafoldML}%
%
\isadelimML
%
\endisadelimML
%
\isadelimproof
%
\endisadelimproof
%
\isatagproof
%
\endisatagproof
{\isafoldproof}%
%
\isadelimproof
%
\endisadelimproof
%
\isadelimproof
%
\endisadelimproof
%
\isatagproof
%
\endisatagproof
{\isafoldproof}%
%
\isadelimproof
%
\endisadelimproof
%
\isadelimproof
%
\endisadelimproof
%
\isatagproof
%
\endisatagproof
{\isafoldproof}%
%
\isadelimproof
%
\endisadelimproof
%
\isadelimML
%
\endisadelimML
%
\isatagML
%
\endisatagML
{\isafoldML}%
%
\isadelimML
%
\endisadelimML
%
\isadelimML
%
\endisadelimML
%
\isatagML
%
\endisatagML
{\isafoldML}%
%
\isadelimML
%
\endisadelimML
%
\isamarkupsection{Event Traces \label{sec:events}%
}
\isamarkuptrue%
%
\begin{isamarkuptext}%
The system's behaviour is formalized as a set of traces of
\emph{events}.  The most important event, \isa{Says\ A\ B\ X}, expresses
$A\to B : X$, which is the attempt by~$A$ to send~$B$ the message~$X$.
A trace is simply a list, constructed in reverse
using~\isa{{\isaliteral{23}{\isacharhash}}}.  Other event types include reception of messages (when
we want to make it explicit) and an agent's storing a fact.

Sometimes the protocol requires an agent to generate a new nonce. The
probability that a 20-byte random number has appeared before is effectively
zero.  To formalize this important property, the set \isa{used\ evs}
denotes the set of all items mentioned in the trace~\isa{evs}.
The function \isa{used} has a straightforward
recursive definition.  Here is the case for \isa{Says} event:
\begin{isabelle}%
\ \ \ \ \ used\ {\isaliteral{28}{\isacharparenleft}}Says\ A\ B\ X\ {\isaliteral{23}{\isacharhash}}\ evs{\isaliteral{29}{\isacharparenright}}\ {\isaliteral{3D}{\isacharequal}}\ parts\ {\isaliteral{7B}{\isacharbraceleft}}X{\isaliteral{7D}{\isacharbraceright}}\ {\isaliteral{5C3C756E696F6E3E}{\isasymunion}}\ used\ evs%
\end{isabelle}

The function \isa{knows} formalizes an agent's knowledge.  Mostly we only
care about the spy's knowledge, and \isa{knows\ Spy\ evs} is the set of items
available to the spy in the trace~\isa{evs}.  Already in the empty trace,
the spy starts with some secrets at his disposal, such as the private keys
of compromised users.  After each \isa{Says} event, the spy learns the
message that was sent:
\begin{isabelle}%
\ \ \ \ \ knows\ Spy\ {\isaliteral{28}{\isacharparenleft}}Says\ A\ B\ X\ {\isaliteral{23}{\isacharhash}}\ evs{\isaliteral{29}{\isacharparenright}}\ {\isaliteral{3D}{\isacharequal}}\ insert\ X\ {\isaliteral{28}{\isacharparenleft}}knows\ Spy\ evs{\isaliteral{29}{\isacharparenright}}%
\end{isabelle}
Combinations of functions express other important
sets of messages derived from~\isa{evs}:
\begin{itemize}
\item \isa{analz\ {\isaliteral{28}{\isacharparenleft}}knows\ Spy\ evs{\isaliteral{29}{\isacharparenright}}} is everything that the spy could
learn by decryption
\item \isa{synth\ {\isaliteral{28}{\isacharparenleft}}analz\ {\isaliteral{28}{\isacharparenleft}}knows\ Spy\ evs{\isaliteral{29}{\isacharparenright}}{\isaliteral{29}{\isacharparenright}}} is everything that the spy
could generate
\end{itemize}%
\end{isamarkuptext}%
\isamarkuptrue%
%
\isadelimtheory
%
\endisadelimtheory
%
\isatagtheory
%
\endisatagtheory
{\isafoldtheory}%
%
\isadelimtheory
%
\endisadelimtheory
\end{isabellebody}%
%%% Local Variables:
%%% mode: latex
%%% TeX-master: "root"
%%% End:

%
\begin{isabellebody}%
\def\isabellecontext{Public}%
%
\isadelimtheory
%
\endisadelimtheory
%
\isatagtheory
%
\endisatagtheory
{\isafoldtheory}%
%
\isadelimtheory
%
\endisadelimtheory
%
\begin{isamarkuptext}%
The function
\isa{pubK} maps agents to their public keys.  The function
\isa{priK} maps agents to their private keys.  It is defined in terms of
\isa{invKey} and \isa{pubK} by a translation; therefore \isa{priK} is
not a proper constant, so we declare it using \isacommand{syntax}
(cf.\ \S\ref{sec:syntax-translations}).%
\end{isamarkuptext}%
\isamarkuptrue%
\isacommand{consts}\isamarkupfalse%
\ pubK\ {\isacharcolon}{\isacharcolon}\ {\isachardoublequoteopen}agent\ {\isacharequal}{\isachargreater}\ key{\isachardoublequoteclose}\isanewline
\isacommand{syntax}\isamarkupfalse%
\ priK\ {\isacharcolon}{\isacharcolon}\ {\isachardoublequoteopen}agent\ {\isacharequal}{\isachargreater}\ key{\isachardoublequoteclose}\isanewline
\isacommand{translations}\isamarkupfalse%
\ {\isachardoublequoteopen}priK\ x{\isachardoublequoteclose}\ {\isasymrightleftharpoons}\ {\isachardoublequoteopen}invKey{\isacharparenleft}pubK\ x{\isacharparenright}{\isachardoublequoteclose}%
\begin{isamarkuptext}%
\noindent
The set \isa{bad} consists of those agents whose private keys are known to
the spy.

Two axioms are asserted about the public-key cryptosystem. 
No two agents have the same public key, and no private key equals
any public key.%
\end{isamarkuptext}%
\isamarkuptrue%
\isacommand{axioms}\isamarkupfalse%
\isanewline
\ \ inj{\isacharunderscore}pubK{\isacharcolon}\ \ \ \ \ \ \ \ {\isachardoublequoteopen}inj\ pubK{\isachardoublequoteclose}\isanewline
\ \ priK{\isacharunderscore}neq{\isacharunderscore}pubK{\isacharcolon}\ \ \ {\isachardoublequoteopen}priK\ A\ {\isachartilde}{\isacharequal}\ pubK\ B{\isachardoublequoteclose}%
\isadelimproof
%
\endisadelimproof
%
\isatagproof
%
\endisatagproof
{\isafoldproof}%
%
\isadelimproof
%
\endisadelimproof
%
\isadelimproof
%
\endisadelimproof
%
\isatagproof
%
\endisatagproof
{\isafoldproof}%
%
\isadelimproof
%
\endisadelimproof
%
\isadelimproof
%
\endisadelimproof
%
\isatagproof
%
\endisatagproof
{\isafoldproof}%
%
\isadelimproof
%
\endisadelimproof
%
\isadelimproof
%
\endisadelimproof
%
\isatagproof
%
\endisatagproof
{\isafoldproof}%
%
\isadelimproof
%
\endisadelimproof
%
\isadelimproof
%
\endisadelimproof
%
\isatagproof
%
\endisatagproof
{\isafoldproof}%
%
\isadelimproof
%
\endisadelimproof
%
\isadelimproof
%
\endisadelimproof
%
\isatagproof
%
\endisatagproof
{\isafoldproof}%
%
\isadelimproof
%
\endisadelimproof
%
\isadelimproof
%
\endisadelimproof
%
\isatagproof
%
\endisatagproof
{\isafoldproof}%
%
\isadelimproof
%
\endisadelimproof
%
\isadelimproof
%
\endisadelimproof
%
\isatagproof
%
\endisatagproof
{\isafoldproof}%
%
\isadelimproof
%
\endisadelimproof
%
\isadelimproof
%
\endisadelimproof
%
\isatagproof
%
\endisatagproof
{\isafoldproof}%
%
\isadelimproof
%
\endisadelimproof
%
\isadelimproof
%
\endisadelimproof
%
\isatagproof
%
\endisatagproof
{\isafoldproof}%
%
\isadelimproof
%
\endisadelimproof
%
\isadelimproof
%
\endisadelimproof
%
\isatagproof
%
\endisatagproof
{\isafoldproof}%
%
\isadelimproof
%
\endisadelimproof
%
\isadelimproof
%
\endisadelimproof
%
\isatagproof
%
\endisatagproof
{\isafoldproof}%
%
\isadelimproof
%
\endisadelimproof
%
\isadelimproof
%
\endisadelimproof
%
\isatagproof
%
\endisatagproof
{\isafoldproof}%
%
\isadelimproof
%
\endisadelimproof
%
\isadelimproof
%
\endisadelimproof
%
\isatagproof
%
\endisatagproof
{\isafoldproof}%
%
\isadelimproof
%
\endisadelimproof
%
\isadelimproof
%
\endisadelimproof
%
\isatagproof
%
\endisatagproof
{\isafoldproof}%
%
\isadelimproof
%
\endisadelimproof
%
\isadelimproof
%
\endisadelimproof
%
\isatagproof
%
\endisatagproof
{\isafoldproof}%
%
\isadelimproof
%
\endisadelimproof
%
\isadelimproof
%
\endisadelimproof
%
\isatagproof
%
\endisatagproof
{\isafoldproof}%
%
\isadelimproof
%
\endisadelimproof
%
\isadelimproof
%
\endisadelimproof
%
\isatagproof
%
\endisatagproof
{\isafoldproof}%
%
\isadelimproof
%
\endisadelimproof
%
\isadelimproof
%
\endisadelimproof
%
\isatagproof
%
\endisatagproof
{\isafoldproof}%
%
\isadelimproof
%
\endisadelimproof
%
\isadelimML
%
\endisadelimML
%
\isatagML
%
\endisatagML
{\isafoldML}%
%
\isadelimML
%
\endisadelimML
%
\isadelimtheory
%
\endisadelimtheory
%
\isatagtheory
%
\endisatagtheory
{\isafoldtheory}%
%
\isadelimtheory
%
\endisadelimtheory
\end{isabellebody}%
%%% Local Variables:
%%% mode: latex
%%% TeX-master: "root"
%%% End:

%
\begin{isabellebody}%
\def\isabellecontext{NS{\isacharunderscore}Public}%
%
\isadelimtheory
%
\endisadelimtheory
%
\isatagtheory
%
\endisatagtheory
{\isafoldtheory}%
%
\isadelimtheory
%
\endisadelimtheory
%
\isamarkupsection{Modelling the Protocol \label{sec:modelling}%
}
\isamarkuptrue%
%
\begin{figure}
\begin{isabelle}
\isacommand{inductive{\isacharunderscore}set}\isamarkupfalse%
\ ns{\isacharunderscore}public\ {\isacharcolon}{\isacharcolon}\ {\isachardoublequoteopen}event\ list\ set{\isachardoublequoteclose}\isanewline
\ \ \isakeyword{where}\isanewline
\isanewline
\ \ \ Nil{\isacharcolon}\ \ {\isachardoublequoteopen}{\isacharbrackleft}{\isacharbrackright}\ {\isasymin}\ ns{\isacharunderscore}public{\isachardoublequoteclose}\isanewline
\isanewline
\isanewline
\ {\isacharbar}\ Fake{\isacharcolon}\ {\isachardoublequoteopen}{\isasymlbrakk}evsf\ {\isasymin}\ ns{\isacharunderscore}public{\isacharsemicolon}\ \ X\ {\isasymin}\ synth\ {\isacharparenleft}analz\ {\isacharparenleft}knows\ Spy\ evsf{\isacharparenright}{\isacharparenright}{\isasymrbrakk}\isanewline
\ \ \ \ \ \ \ \ \ \ {\isasymLongrightarrow}\ Says\ Spy\ B\ X\ \ {\isacharhash}\ evsf\ {\isasymin}\ ns{\isacharunderscore}public{\isachardoublequoteclose}\isanewline
\isanewline
\isanewline
\ {\isacharbar}\ NS{\isadigit{1}}{\isacharcolon}\ \ {\isachardoublequoteopen}{\isasymlbrakk}evs{\isadigit{1}}\ {\isasymin}\ ns{\isacharunderscore}public{\isacharsemicolon}\ \ Nonce\ NA\ {\isasymnotin}\ used\ evs{\isadigit{1}}{\isasymrbrakk}\isanewline
\ \ \ \ \ \ \ \ \ \ {\isasymLongrightarrow}\ Says\ A\ B\ {\isacharparenleft}Crypt\ {\isacharparenleft}pubK\ B{\isacharparenright}\ {\isasymlbrace}Nonce\ NA{\isacharcomma}\ Agent\ A{\isasymrbrace}{\isacharparenright}\isanewline
\ \ \ \ \ \ \ \ \ \ \ \ \ \ \ \ \ {\isacharhash}\ evs{\isadigit{1}}\ \ {\isasymin}\ \ ns{\isacharunderscore}public{\isachardoublequoteclose}\isanewline
\isanewline
\isanewline
\ {\isacharbar}\ NS{\isadigit{2}}{\isacharcolon}\ \ {\isachardoublequoteopen}{\isasymlbrakk}evs{\isadigit{2}}\ {\isasymin}\ ns{\isacharunderscore}public{\isacharsemicolon}\ \ Nonce\ NB\ {\isasymnotin}\ used\ evs{\isadigit{2}}{\isacharsemicolon}\isanewline
\ \ \ \ \ \ \ \ \ \ \ Says\ A{\isacharprime}\ B\ {\isacharparenleft}Crypt\ {\isacharparenleft}pubK\ B{\isacharparenright}\ {\isasymlbrace}Nonce\ NA{\isacharcomma}\ Agent\ A{\isasymrbrace}{\isacharparenright}\ {\isasymin}\ set\ evs{\isadigit{2}}{\isasymrbrakk}\isanewline
\ \ \ \ \ \ \ \ \ \ {\isasymLongrightarrow}\ Says\ B\ A\ {\isacharparenleft}Crypt\ {\isacharparenleft}pubK\ A{\isacharparenright}\ {\isasymlbrace}Nonce\ NA{\isacharcomma}\ Nonce\ NB{\isacharcomma}\ Agent\ B{\isasymrbrace}{\isacharparenright}\isanewline
\ \ \ \ \ \ \ \ \ \ \ \ \ \ \ \ {\isacharhash}\ evs{\isadigit{2}}\ \ {\isasymin}\ \ ns{\isacharunderscore}public{\isachardoublequoteclose}\isanewline
\isanewline
\isanewline
\ {\isacharbar}\ NS{\isadigit{3}}{\isacharcolon}\ \ {\isachardoublequoteopen}{\isasymlbrakk}evs{\isadigit{3}}\ {\isasymin}\ ns{\isacharunderscore}public{\isacharsemicolon}\isanewline
\ \ \ \ \ \ \ \ \ \ \ Says\ A\ \ B\ {\isacharparenleft}Crypt\ {\isacharparenleft}pubK\ B{\isacharparenright}\ {\isasymlbrace}Nonce\ NA{\isacharcomma}\ Agent\ A{\isasymrbrace}{\isacharparenright}\ {\isasymin}\ set\ evs{\isadigit{3}}{\isacharsemicolon}\isanewline
\ \ \ \ \ \ \ \ \ \ \ Says\ B{\isacharprime}\ A\ {\isacharparenleft}Crypt\ {\isacharparenleft}pubK\ A{\isacharparenright}\ {\isasymlbrace}Nonce\ NA{\isacharcomma}\ Nonce\ NB{\isacharcomma}\ Agent\ B{\isasymrbrace}{\isacharparenright}\isanewline
\ \ \ \ \ \ \ \ \ \ \ \ \ \ {\isasymin}\ set\ evs{\isadigit{3}}{\isasymrbrakk}\isanewline
\ \ \ \ \ \ \ \ \ \ {\isasymLongrightarrow}\ Says\ A\ B\ {\isacharparenleft}Crypt\ {\isacharparenleft}pubK\ B{\isacharparenright}\ {\isacharparenleft}Nonce\ NB{\isacharparenright}{\isacharparenright}\ {\isacharhash}\ evs{\isadigit{3}}\ {\isasymin}\ ns{\isacharunderscore}public{\isachardoublequoteclose}%
\end{isabelle}
\caption{An Inductive Protocol Definition}\label{fig:ns_public}
\end{figure}
%
\begin{isamarkuptext}%
Let us formalize the Needham-Schroeder public-key protocol, as corrected by
Lowe:
\begin{alignat*%
}{2}
  &1.&\quad  A\to B  &: \comp{Na,A}\sb{Kb} \\
  &2.&\quad  B\to A  &: \comp{Na,Nb,B}\sb{Ka} \\
  &3.&\quad  A\to B  &: \comp{Nb}\sb{Kb}
\end{alignat*%
}

Each protocol step is specified by a rule of an inductive definition.  An
event trace has type \isa{event\ list}, so we declare the constant
\isa{ns{\isacharunderscore}public} to be a set of such traces.

Figure~\ref{fig:ns_public} presents the inductive definition.  The
\isa{Nil} rule introduces the empty trace.  The \isa{Fake} rule models the
adversary's sending a message built from components taken from past
traffic, expressed using the functions \isa{synth} and
\isa{analz}. 
The next three rules model how honest agents would perform the three
protocol steps.  

Here is a detailed explanation of rule \isa{NS{\isadigit{2}}}.
A trace containing an event of the form
\begin{isabelle}%
\ \ \ \ \ Says\ A{\isacharprime}\ B\ {\isacharparenleft}Crypt\ {\isacharparenleft}pubK\ B{\isacharparenright}\ {\isasymlbrace}Nonce\ NA{\isacharcomma}\ Agent\ A{\isasymrbrace}{\isacharparenright}%
\end{isabelle}
may be extended by an event of the form
\begin{isabelle}%
\ \ \ \ \ Says\ B\ A\ {\isacharparenleft}Crypt\ {\isacharparenleft}pubK\ A{\isacharparenright}\ {\isasymlbrace}Nonce\ NA{\isacharcomma}\ Nonce\ NB{\isacharcomma}\ Agent\ B{\isasymrbrace}{\isacharparenright}%
\end{isabelle}
where \isa{NB} is a fresh nonce: \isa{Nonce\ NB\ {\isasymnotin}\ used\ evs{\isadigit{2}}}.
Writing the sender as \isa{A{\isacharprime}} indicates that \isa{B} does not 
know who sent the message.  Calling the trace variable \isa{evs{\isadigit{2}}} rather
than simply \isa{evs} helps us know where we are in a proof after many
case-splits: every subgoal mentioning \isa{evs{\isadigit{2}}} involves message~2 of the
protocol.

Benefits of this approach are simplicity and clarity.  The semantic model
is set theory, proofs are by induction and the translation from the informal
notation to the inductive rules is straightforward.%
\end{isamarkuptext}%
\isamarkuptrue%
%
\isamarkupsection{Proving Elementary Properties \label{sec:regularity}%
}
\isamarkuptrue%
%
\isadelimproof
%
\endisadelimproof
%
\isatagproof
%
\endisatagproof
{\isafoldproof}%
%
\isadelimproof
%
\endisadelimproof
%
\begin{isamarkuptext}%
Secrecy properties can be hard to prove.  The conclusion of a typical
secrecy theorem is 
\isa{X\ {\isasymnotin}\ analz\ {\isacharparenleft}knows\ Spy\ evs{\isacharparenright}}.  The difficulty arises from
having to reason about \isa{analz}, or less formally, showing that the spy
can never learn~\isa{X}.  Much easier is to prove that \isa{X} can never
occur at all.  Such \emph{regularity} properties are typically expressed
using \isa{parts} rather than \isa{analz}.

The following lemma states that \isa{A}'s private key is potentially
known to the spy if and only if \isa{A} belongs to the set \isa{bad} of
compromised agents.  The statement uses \isa{parts}: the very presence of
\isa{A}'s private key in a message, whether protected by encryption or
not, is enough to confirm that \isa{A} is compromised.  The proof, like
nearly all protocol proofs, is by induction over traces.%
\end{isamarkuptext}%
\isamarkuptrue%
\isacommand{lemma}\isamarkupfalse%
\ Spy{\isacharunderscore}see{\isacharunderscore}priK\ {\isacharbrackleft}simp{\isacharbrackright}{\isacharcolon}\isanewline
\ \ \ \ \ \ {\isachardoublequoteopen}evs\ {\isasymin}\ ns{\isacharunderscore}public\isanewline
\ \ \ \ \ \ \ {\isasymLongrightarrow}\ {\isacharparenleft}Key\ {\isacharparenleft}priK\ A{\isacharparenright}\ {\isasymin}\ parts\ {\isacharparenleft}knows\ Spy\ evs{\isacharparenright}{\isacharparenright}\ {\isacharequal}\ {\isacharparenleft}A\ {\isasymin}\ bad{\isacharparenright}{\isachardoublequoteclose}\isanewline
%
\isadelimproof
%
\endisadelimproof
%
\isatagproof
\isacommand{apply}\isamarkupfalse%
\ {\isacharparenleft}erule\ ns{\isacharunderscore}public{\isachardot}induct{\isacharcomma}\ simp{\isacharunderscore}all{\isacharparenright}%
\begin{isamarkuptxt}%
The induction yields five subgoals, one for each rule in the definition of
\isa{ns{\isacharunderscore}public}.  The idea is to prove that the protocol property holds initially
(rule \isa{Nil}), is preserved by each of the legitimate protocol steps (rules
\isa{NS{\isadigit{1}}}--\isa{{\isadigit{3}}}), and even is preserved in the face of anything the
spy can do (rule \isa{Fake}).  

The proof is trivial.  No legitimate protocol rule sends any keys
at all, so only \isa{Fake} is relevant. Indeed, simplification leaves
only the \isa{Fake} case, as indicated by the variable name \isa{evsf}:
\begin{isabelle}%
\ {\isadigit{1}}{\isachardot}\ {\isasymAnd}evsf\ X{\isachardot}\isanewline
\isaindent{\ {\isadigit{1}}{\isachardot}\ \ \ \ }{\isasymlbrakk}evsf\ {\isasymin}\ ns{\isacharunderscore}public{\isacharsemicolon}\isanewline
\isaindent{\ {\isadigit{1}}{\isachardot}\ \ \ \ \ }{\isacharparenleft}Key\ {\isacharparenleft}priK\ A{\isacharparenright}\ {\isasymin}\ parts\ {\isacharparenleft}knows\ Spy\ evsf{\isacharparenright}{\isacharparenright}\ {\isacharequal}\ {\isacharparenleft}A\ {\isasymin}\ bad{\isacharparenright}{\isacharsemicolon}\isanewline
\isaindent{\ {\isadigit{1}}{\isachardot}\ \ \ \ \ }X\ {\isasymin}\ synth\ {\isacharparenleft}analz\ {\isacharparenleft}knows\ Spy\ evsf{\isacharparenright}{\isacharparenright}{\isasymrbrakk}\isanewline
\isaindent{\ {\isadigit{1}}{\isachardot}\ \ \ \ }{\isasymLongrightarrow}\ {\isacharparenleft}Key\ {\isacharparenleft}priK\ A{\isacharparenright}\ {\isasymin}\ parts\ {\isacharparenleft}insert\ X\ {\isacharparenleft}knows\ Spy\ evsf{\isacharparenright}{\isacharparenright}{\isacharparenright}\ {\isacharequal}\isanewline
\isaindent{\ {\isadigit{1}}{\isachardot}\ \ \ \ {\isasymLongrightarrow}\ }{\isacharparenleft}A\ {\isasymin}\ bad{\isacharparenright}%
\end{isabelle}%
\end{isamarkuptxt}%
\isamarkuptrue%
\isacommand{by}\isamarkupfalse%
\ blast%
\endisatagproof
{\isafoldproof}%
%
\isadelimproof
%
\endisadelimproof
%
\isadelimproof
%
\endisadelimproof
%
\isatagproof
%
\endisatagproof
{\isafoldproof}%
%
\isadelimproof
%
\endisadelimproof
%
\begin{isamarkuptext}%
The \isa{Fake} case is proved automatically.  If
\isa{priK\ A} is in the extended trace then either (1) it was already in the
original trace or (2) it was
generated by the spy, who must have known this key already. 
Either way, the induction hypothesis applies.

\emph{Unicity} lemmas are regularity lemmas stating that specified items
can occur only once in a trace.  The following lemma states that a nonce
cannot be used both as $Na$ and as $Nb$ unless
it is known to the spy.  Intuitively, it holds because honest agents
always choose fresh values as nonces; only the spy might reuse a value,
and he doesn't know this particular value.  The proof script is short:
induction, simplification, \isa{blast}.  The first line uses the rule
\isa{rev{\isacharunderscore}mp} to prepare the induction by moving two assumptions into the 
induction formula.%
\end{isamarkuptext}%
\isamarkuptrue%
\isacommand{lemma}\isamarkupfalse%
\ no{\isacharunderscore}nonce{\isacharunderscore}NS{\isadigit{1}}{\isacharunderscore}NS{\isadigit{2}}{\isacharcolon}\isanewline
\ \ \ \ {\isachardoublequoteopen}{\isasymlbrakk}Crypt\ {\isacharparenleft}pubK\ C{\isacharparenright}\ {\isasymlbrace}NA{\isacharprime}{\isacharcomma}\ Nonce\ NA{\isacharcomma}\ Agent\ D{\isasymrbrace}\ {\isasymin}\ parts\ {\isacharparenleft}knows\ Spy\ evs{\isacharparenright}{\isacharsemicolon}\isanewline
\ \ \ \ \ \ Crypt\ {\isacharparenleft}pubK\ B{\isacharparenright}\ {\isasymlbrace}Nonce\ NA{\isacharcomma}\ Agent\ A{\isasymrbrace}\ {\isasymin}\ parts\ {\isacharparenleft}knows\ Spy\ evs{\isacharparenright}{\isacharsemicolon}\isanewline
\ \ \ \ \ \ evs\ {\isasymin}\ ns{\isacharunderscore}public{\isasymrbrakk}\isanewline
\ \ \ \ \ {\isasymLongrightarrow}\ Nonce\ NA\ {\isasymin}\ analz\ {\isacharparenleft}knows\ Spy\ evs{\isacharparenright}{\isachardoublequoteclose}\isanewline
%
\isadelimproof
%
\endisadelimproof
%
\isatagproof
\isacommand{apply}\isamarkupfalse%
\ {\isacharparenleft}erule\ rev{\isacharunderscore}mp{\isacharcomma}\ erule\ rev{\isacharunderscore}mp{\isacharparenright}\isanewline
\isacommand{apply}\isamarkupfalse%
\ {\isacharparenleft}erule\ ns{\isacharunderscore}public{\isachardot}induct{\isacharcomma}\ simp{\isacharunderscore}all{\isacharparenright}\isanewline
\isacommand{apply}\isamarkupfalse%
\ {\isacharparenleft}blast\ intro{\isacharcolon}\ analz{\isacharunderscore}insertI{\isacharparenright}{\isacharplus}\isanewline
\isacommand{done}\isamarkupfalse%
%
\endisatagproof
{\isafoldproof}%
%
\isadelimproof
%
\endisadelimproof
%
\begin{isamarkuptext}%
The following unicity lemma states that, if \isa{NA} is secret, then its
appearance in any instance of message~1 determines the other components. 
The proof is similar to the previous one.%
\end{isamarkuptext}%
\isamarkuptrue%
\isacommand{lemma}\isamarkupfalse%
\ unique{\isacharunderscore}NA{\isacharcolon}\isanewline
\ \ \ \ \ {\isachardoublequoteopen}{\isasymlbrakk}Crypt{\isacharparenleft}pubK\ B{\isacharparenright}\ \ {\isasymlbrace}Nonce\ NA{\isacharcomma}\ Agent\ A\ {\isasymrbrace}\ {\isasymin}\ parts{\isacharparenleft}knows\ Spy\ evs{\isacharparenright}{\isacharsemicolon}\isanewline
\ \ \ \ \ \ \ Crypt{\isacharparenleft}pubK\ B{\isacharprime}{\isacharparenright}\ {\isasymlbrace}Nonce\ NA{\isacharcomma}\ Agent\ A{\isacharprime}{\isasymrbrace}\ {\isasymin}\ parts{\isacharparenleft}knows\ Spy\ evs{\isacharparenright}{\isacharsemicolon}\isanewline
\ \ \ \ \ \ \ Nonce\ NA\ {\isasymnotin}\ analz\ {\isacharparenleft}knows\ Spy\ evs{\isacharparenright}{\isacharsemicolon}\ evs\ {\isasymin}\ ns{\isacharunderscore}public{\isasymrbrakk}\isanewline
\ \ \ \ \ \ {\isasymLongrightarrow}\ A{\isacharequal}A{\isacharprime}\ {\isasymand}\ B{\isacharequal}B{\isacharprime}{\isachardoublequoteclose}%
\isadelimproof
%
\endisadelimproof
%
\isatagproof
%
\endisatagproof
{\isafoldproof}%
%
\isadelimproof
%
\endisadelimproof
%
\isamarkupsection{Proving Secrecy Theorems \label{sec:secrecy}%
}
\isamarkuptrue%
%
\isadelimproof
%
\endisadelimproof
%
\isatagproof
%
\endisatagproof
{\isafoldproof}%
%
\isadelimproof
%
\endisadelimproof
%
\isadelimproof
%
\endisadelimproof
%
\isatagproof
%
\endisatagproof
{\isafoldproof}%
%
\isadelimproof
%
\endisadelimproof
%
\isadelimproof
%
\endisadelimproof
%
\isatagproof
%
\endisatagproof
{\isafoldproof}%
%
\isadelimproof
%
\endisadelimproof
%
\isadelimproof
%
\endisadelimproof
%
\isatagproof
%
\endisatagproof
{\isafoldproof}%
%
\isadelimproof
%
\endisadelimproof
%
\isadelimproof
%
\endisadelimproof
%
\isatagproof
%
\endisatagproof
{\isafoldproof}%
%
\isadelimproof
%
\endisadelimproof
%
\begin{isamarkuptext}%
The secrecy theorems for Bob (the second participant) are especially
important because they fail for the original protocol.  The following
theorem states that if Bob sends message~2 to Alice, and both agents are
uncompromised, then Bob's nonce will never reach the spy.%
\end{isamarkuptext}%
\isamarkuptrue%
\isacommand{theorem}\isamarkupfalse%
\ Spy{\isacharunderscore}not{\isacharunderscore}see{\isacharunderscore}NB\ {\isacharbrackleft}dest{\isacharbrackright}{\isacharcolon}\isanewline
\ {\isachardoublequoteopen}{\isasymlbrakk}Says\ B\ A\ {\isacharparenleft}Crypt\ {\isacharparenleft}pubK\ A{\isacharparenright}\ {\isasymlbrace}Nonce\ NA{\isacharcomma}\ Nonce\ NB{\isacharcomma}\ Agent\ B{\isasymrbrace}{\isacharparenright}\ {\isasymin}\ set\ evs{\isacharsemicolon}\isanewline
\ \ \ A\ {\isasymnotin}\ bad{\isacharsemicolon}\ \ B\ {\isasymnotin}\ bad{\isacharsemicolon}\ \ evs\ {\isasymin}\ ns{\isacharunderscore}public{\isasymrbrakk}\isanewline
\ \ {\isasymLongrightarrow}\ Nonce\ NB\ {\isasymnotin}\ analz\ {\isacharparenleft}knows\ Spy\ evs{\isacharparenright}{\isachardoublequoteclose}%
\isadelimproof
%
\endisadelimproof
%
\isatagproof
%
\begin{isamarkuptxt}%
To prove it, we must formulate the induction properly (one of the
assumptions mentions~\isa{evs}), apply induction, and simplify:%
\end{isamarkuptxt}%
\isamarkuptrue%
\isacommand{apply}\isamarkupfalse%
\ {\isacharparenleft}erule\ rev{\isacharunderscore}mp{\isacharcomma}\ erule\ ns{\isacharunderscore}public{\isachardot}induct{\isacharcomma}\ simp{\isacharunderscore}all{\isacharparenright}%
\begin{isamarkuptxt}%
The proof states are too complicated to present in full.  
Let's examine the simplest subgoal, that for message~1.  The following
event has just occurred:
\[ 1.\quad  A'\to B'  : \comp{Na',A'}\sb{Kb'} \]
The variables above have been primed because this step
belongs to a different run from that referred to in the theorem
statement --- the theorem
refers to a past instance of message~2, while this subgoal
concerns message~1 being sent just now.
In the Isabelle subgoal, instead of primed variables like $B'$ and $Na'$
we have \isa{Ba} and~\isa{NAa}:
\begin{isabelle}%
\ {\isadigit{1}}{\isachardot}\ {\isasymAnd}evs{\isadigit{1}}\ NAa\ Ba{\isachardot}\isanewline
\isaindent{\ {\isadigit{1}}{\isachardot}\ \ \ \ }{\isasymlbrakk}A\ {\isasymnotin}\ bad{\isacharsemicolon}\ B\ {\isasymnotin}\ bad{\isacharsemicolon}\ evs{\isadigit{1}}\ {\isasymin}\ ns{\isacharunderscore}public{\isacharsemicolon}\isanewline
\isaindent{\ {\isadigit{1}}{\isachardot}\ \ \ \ \ }Says\ B\ A\ {\isacharparenleft}Crypt\ {\isacharparenleft}pubK\ A{\isacharparenright}\ {\isasymlbrace}Nonce\ NA{\isacharcomma}\ Nonce\ NB{\isacharcomma}\ Agent\ B{\isasymrbrace}{\isacharparenright}\isanewline
\isaindent{\ {\isadigit{1}}{\isachardot}\ \ \ \ \ }{\isasymin}\ set\ evs{\isadigit{1}}\ {\isasymlongrightarrow}\isanewline
\isaindent{\ {\isadigit{1}}{\isachardot}\ \ \ \ \ }Nonce\ NB\ {\isasymnotin}\ analz\ {\isacharparenleft}knows\ Spy\ evs{\isadigit{1}}{\isacharparenright}{\isacharsemicolon}\isanewline
\isaindent{\ {\isadigit{1}}{\isachardot}\ \ \ \ \ }Nonce\ NAa\ {\isasymnotin}\ used\ evs{\isadigit{1}}{\isasymrbrakk}\isanewline
\isaindent{\ {\isadigit{1}}{\isachardot}\ \ \ \ }{\isasymLongrightarrow}\ Ba\ {\isasymin}\ bad\ {\isasymlongrightarrow}\isanewline
\isaindent{\ {\isadigit{1}}{\isachardot}\ \ \ \ {\isasymLongrightarrow}\ }Says\ B\ A\ {\isacharparenleft}Crypt\ {\isacharparenleft}pubK\ A{\isacharparenright}\ {\isasymlbrace}Nonce\ NA{\isacharcomma}\ Nonce\ NB{\isacharcomma}\ Agent\ B{\isasymrbrace}{\isacharparenright}\isanewline
\isaindent{\ {\isadigit{1}}{\isachardot}\ \ \ \ {\isasymLongrightarrow}\ }{\isasymin}\ set\ evs{\isadigit{1}}\ {\isasymlongrightarrow}\isanewline
\isaindent{\ {\isadigit{1}}{\isachardot}\ \ \ \ {\isasymLongrightarrow}\ }NB\ {\isasymnoteq}\ NAa%
\end{isabelle}
The simplifier has used a 
default simplification rule that does a case
analysis for each encrypted message on whether or not the decryption key
is compromised.
\begin{isabelle}%
analz\ {\isacharparenleft}insert\ {\isacharparenleft}Crypt\ K\ X{\isacharparenright}\ H{\isacharparenright}\ {\isacharequal}\isanewline
{\isacharparenleft}if\ Key\ {\isacharparenleft}invKey\ K{\isacharparenright}\ {\isasymin}\ analz\ H\isanewline
\isaindent{{\isacharparenleft}}then\ insert\ {\isacharparenleft}Crypt\ K\ X{\isacharparenright}\ {\isacharparenleft}analz\ {\isacharparenleft}insert\ X\ H{\isacharparenright}{\isacharparenright}\isanewline
\isaindent{{\isacharparenleft}}else\ insert\ {\isacharparenleft}Crypt\ K\ X{\isacharparenright}\ {\isacharparenleft}analz\ H{\isacharparenright}{\isacharparenright}\rulename{analz{\isacharunderscore}Crypt{\isacharunderscore}if}%
\end{isabelle}
The simplifier has also used \isa{Spy{\isacharunderscore}see{\isacharunderscore}priK}, proved in
{\S}\ref{sec:regularity} above, to yield \isa{Ba\ {\isasymin}\ bad}.

Recall that this subgoal concerns the case
where the last message to be sent was
\[ 1.\quad  A'\to B'  : \comp{Na',A'}\sb{Kb'}. \]
This message can compromise $Nb$ only if $Nb=Na'$ and $B'$ is compromised,
allowing the spy to decrypt the message.  The Isabelle subgoal says
precisely this, if we allow for its choice of variable names.
Proving \isa{NB\ {\isasymnoteq}\ NAa} is easy: \isa{NB} was
sent earlier, while \isa{NAa} is fresh; formally, we have
the assumption \isa{Nonce\ NAa\ {\isasymnotin}\ used\ evs{\isadigit{1}}}. 

Note that our reasoning concerned \isa{B}'s participation in another
run.  Agents may engage in several runs concurrently, and some attacks work
by interleaving the messages of two runs.  With model checking, this
possibility can cause a state-space explosion, and for us it
certainly complicates proofs.  The biggest subgoal concerns message~2.  It
splits into several cases, such as whether or not the message just sent is
the very message mentioned in the theorem statement.
Some of the cases are proved by unicity, others by
the induction hypothesis.  For all those complications, the proofs are
automatic by \isa{blast} with the theorem \isa{no{\isacharunderscore}nonce{\isacharunderscore}NS{\isadigit{1}}{\isacharunderscore}NS{\isadigit{2}}}.

The remaining theorems about the protocol are not hard to prove.  The
following one asserts a form of \emph{authenticity}: if
\isa{B} has sent an instance of message~2 to~\isa{A} and has received the
expected reply, then that reply really originated with~\isa{A}.  The
proof is a simple induction.%
\end{isamarkuptxt}%
\isamarkuptrue%
%
\endisatagproof
{\isafoldproof}%
%
\isadelimproof
%
\endisadelimproof
%
\isadelimproof
%
\endisadelimproof
%
\isatagproof
%
\endisatagproof
{\isafoldproof}%
%
\isadelimproof
%
\endisadelimproof
\isacommand{theorem}\isamarkupfalse%
\ B{\isacharunderscore}trusts{\isacharunderscore}NS{\isadigit{3}}{\isacharcolon}\isanewline
\ {\isachardoublequoteopen}{\isasymlbrakk}Says\ B\ A\ \ {\isacharparenleft}Crypt\ {\isacharparenleft}pubK\ A{\isacharparenright}\ {\isasymlbrace}Nonce\ NA{\isacharcomma}\ Nonce\ NB{\isacharcomma}\ Agent\ B{\isasymrbrace}{\isacharparenright}\ {\isasymin}\ set\ evs{\isacharsemicolon}\isanewline
\ \ \ Says\ A{\isacharprime}\ B\ {\isacharparenleft}Crypt\ {\isacharparenleft}pubK\ B{\isacharparenright}\ {\isacharparenleft}Nonce\ NB{\isacharparenright}{\isacharparenright}\ {\isasymin}\ set\ evs{\isacharsemicolon}\isanewline
\ \ \ A\ {\isasymnotin}\ bad{\isacharsemicolon}\ \ B\ {\isasymnotin}\ bad{\isacharsemicolon}\ \ evs\ {\isasymin}\ ns{\isacharunderscore}public{\isasymrbrakk}\isanewline
\ \ {\isasymLongrightarrow}\ Says\ A\ B\ {\isacharparenleft}Crypt\ {\isacharparenleft}pubK\ B{\isacharparenright}\ {\isacharparenleft}Nonce\ NB{\isacharparenright}{\isacharparenright}\ {\isasymin}\ set\ evs{\isachardoublequoteclose}%
\isadelimproof
%
\endisadelimproof
%
\isatagproof
%
\endisatagproof
{\isafoldproof}%
%
\isadelimproof
%
\endisadelimproof
%
\isadelimproof
%
\endisadelimproof
%
\isatagproof
%
\endisatagproof
{\isafoldproof}%
%
\isadelimproof
%
\endisadelimproof
%
\begin{isamarkuptext}%
From similar assumptions, we can prove that \isa{A} started the protocol
run by sending an instance of message~1 involving the nonce~\isa{NA}\@. 
For this theorem, the conclusion is 
\begin{isabelle}%
Says\ A\ B\ {\isacharparenleft}Crypt\ {\isacharparenleft}pubK\ B{\isacharparenright}\ {\isasymlbrace}Nonce\ NA{\isacharcomma}\ Agent\ A{\isasymrbrace}{\isacharparenright}\ {\isasymin}\ set\ evs%
\end{isabelle}
Analogous theorems can be proved for~\isa{A}, stating that nonce~\isa{NA}
remains secret and that message~2 really originates with~\isa{B}.  Even the
flawed protocol establishes these properties for~\isa{A};
the flaw only harms the second participant.

\medskip

Detailed information on this protocol verification technique can be found
elsewhere~\cite{paulson-jcs}, including proofs of an Internet
protocol~\cite{paulson-tls}.  We must stress that the protocol discussed
in this chapter is trivial.  There are only three messages; no keys are
exchanged; we merely have to prove that encrypted data remains secret. 
Real world protocols are much longer and distribute many secrets to their
participants.  To be realistic, the model has to include the possibility
of keys being lost dynamically due to carelessness.  If those keys have
been used to encrypt other sensitive information, there may be cascading
losses.  We may still be able to establish a bound on the losses and to
prove that other protocol runs function
correctly~\cite{paulson-yahalom}.  Proofs of real-world protocols follow
the strategy illustrated above, but the subgoals can
be much bigger and there are more of them.
\index{protocols!security|)}%
\end{isamarkuptext}%
\isamarkuptrue%
%
\isadelimtheory
%
\endisadelimtheory
%
\isatagtheory
%
\endisatagtheory
{\isafoldtheory}%
%
\isadelimtheory
%
\endisadelimtheory
\end{isabellebody}%
%%% Local Variables:
%%% mode: latex
%%% TeX-master: "root"
%%% End:


%\chapter{Structured Proofs}
%\label{ch:Isar}
%\chapter{Case Study: UNIX File-System Security}
%\chapter{The Tricks of the Trade}
\appendix

\chapter{Appendix}
\label{sec:Appendix}

\begin{figure}[htbp]
\begin{center}
\begin{tabular}{|llll|}
\hline
\texttt{arities} &
\texttt{binder} &
\texttt{classes} &
\texttt{consts} \\
\texttt{default} &
\texttt{defs} &
\texttt{end} &
\texttt{global} \\
\texttt{infixl} &
\texttt{infixr} &
\texttt{instance} &
\texttt{local} \\
\texttt{mixfix} &
\texttt{ML} &
\texttt{MLtext} &
\texttt{nonterminals} \\
\texttt{oracle} &
\texttt{output} &
\texttt{path} &
\texttt{rules} \\
\texttt{setup} &
\texttt{syntax} &
\texttt{translations} &
\texttt{types} \\
\texttt{constdefs} &
\texttt{axclass} &&\\
\hline
\end{tabular}
\end{center}
\caption{Keywords in theory files}
\label{fig:keywords}
\end{figure}

\begin{figure}[htbp]
\begin{center}
\begin{tabular}{|lllll|}
\hline
\texttt{ALL} &
\texttt{case} &
\texttt{div} &
\texttt{dvd} &
\texttt{else} \\
\texttt{EX} &
\texttt{if} &
\texttt{in} &
\texttt{INT} &
\texttt{Int} \\
\texttt{LEAST} &
\texttt{let} &
\texttt{mod} &
\texttt{O} &
\texttt{o} \\
\texttt{of} &
\texttt{op} &
\texttt{PROP} &
\texttt{SIGMA} &
\texttt{then} \\
\texttt{Times} &
\texttt{UN} &
\texttt{Un} &&\\
\hline
\end{tabular}
\end{center}
\caption{Reserved words in HOL}
\label{fig:ReservedWords}
\end{figure}


\bibliographystyle{plain}
\bibliography{../manual}
\printindex
\end{document}
