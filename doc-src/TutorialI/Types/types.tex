\chapter{More about Types}
\label{ch:more-types}

So far we have learned about a few basic types (for example \isa{bool} and
\isa{nat}), type abbreviations (\isacommand{types}) and recursive datatypes
(\isacommand{datatype}). This chapter will introduce more
advanced material:
\begin{itemize}
\item Pairs ({\S}\ref{sec:products}) and records ({\S}\ref{sec:records}),
and how to reason about them.
\item Type classes: how to specify and reason about axiomatic collections of
  types ({\S}\ref{sec:axclass}). This section leads on to a discussion of
  Isabelle's numeric types ({\S}\ref{sec:numbers}).  
\item Introducing your own types: how to define types that
  cannot be constructed with any of the basic methods
  ({\S}\ref{sec:adv-typedef}).
\end{itemize}

The material in this section goes beyond the needs of most novices.
Serious users should at least skim the sections as far as type classes.
That material is fairly advanced; read the beginning to understand what it
is about, but consult the rest only when necessary.

\index{pairs and tuples|(}
%
\begin{isabellebody}%
\def\isabellecontext{Pairs}%
%
\isamarkupsection{Pairs%
}
%
\begin{isamarkuptext}%
\label{sec:products}
Pairs were already introduced in \S\ref{sec:pairs}, but only with a minimal
repertoire of operations: pairing and the two projections \isa{fst} and
\isa{snd}. In any nontrivial application of pairs you will find that this
quickly leads to unreadable formulae involvings nests of projections. This
section is concerned with introducing some syntactic sugar to overcome this
problem: pattern matching with tuples.%
\end{isamarkuptext}%
%
\isamarkupsubsection{Pattern Matching with Tuples%
}
%
\begin{isamarkuptext}%
Tuples may be used as patterns in $\lambda$-abstractions,
for example \isa{{\isasymlambda}{\isacharparenleft}x{\isacharcomma}y{\isacharcomma}z{\isacharparenright}{\isachardot}x{\isacharplus}y{\isacharplus}z} and \isa{{\isasymlambda}{\isacharparenleft}{\isacharparenleft}x{\isacharcomma}y{\isacharparenright}{\isacharcomma}z{\isacharparenright}{\isachardot}x{\isacharplus}y{\isacharplus}z}. In fact,
tuple patterns can be used in most variable binding constructs,
and they can be nested. Here are
some typical examples:
\begin{quote}
\isa{let\ {\isacharparenleft}x{\isacharcomma}\ y{\isacharparenright}\ {\isacharequal}\ f\ z\ in\ {\isacharparenleft}y{\isacharcomma}\ x{\isacharparenright}}\\
\isa{case\ xs\ of\ {\isacharbrackleft}{\isacharbrackright}\ {\isasymRightarrow}\ {\isadigit{0}}\ {\isacharbar}\ {\isacharparenleft}x{\isacharcomma}\ y{\isacharparenright}\ {\isacharhash}\ zs\ {\isasymRightarrow}\ x\ {\isacharplus}\ y}\\
\isa{{\isasymforall}{\isacharparenleft}x{\isacharcomma}y{\isacharparenright}{\isasymin}A{\isachardot}\ x{\isacharequal}y}\\
\isa{{\isacharbraceleft}{\isacharparenleft}x{\isacharcomma}y{\isacharcomma}z{\isacharparenright}{\isachardot}\ x{\isacharequal}z{\isacharbraceright}}\\
\isa{{\isasymUnion}{\isacharparenleft}x{\isacharcomma}\ y{\isacharparenright}{\isasymin}A{\isachardot}\ {\isacharbraceleft}x\ {\isacharplus}\ y{\isacharbraceright}}
\end{quote}%
\end{isamarkuptext}%
%
\begin{isamarkuptext}%
The intuitive meanings of these expressions should be obvious.
Unfortunately, we need to know in more detail what the notation really stands
for once we have to reason about it.  Abstraction
over pairs and tuples is merely a convenient shorthand for a more complex
internal representation.  Thus the internal and external form of a term may
differ, which can affect proofs. If you want to avoid this complication,
stick to \isa{fst} and \isa{snd} and write \isa{{\isasymlambda}p{\isachardot}\ fst\ p\ {\isacharplus}\ snd\ p}
instead of \isa{{\isasymlambda}{\isacharparenleft}x{\isacharcomma}y{\isacharparenright}{\isachardot}\ x{\isacharplus}y} (which denote the same function but are quite
different terms).

Internally, \isa{{\isasymlambda}{\isacharparenleft}x{\isacharcomma}\ y{\isacharparenright}{\isachardot}\ t} becomes \isa{split\ {\isacharparenleft}{\isasymlambda}x\ y{\isachardot}\ t{\isacharparenright}}, where
\isa{split}\indexbold{*split (constant)}
is the uncurrying function of type \isa{{\isacharparenleft}{\isacharprime}a\ {\isasymRightarrow}\ {\isacharprime}b\ {\isasymRightarrow}\ {\isacharprime}c{\isacharparenright}\ {\isasymRightarrow}\ {\isacharprime}a\ {\isasymtimes}\ {\isacharprime}b\ {\isasymRightarrow}\ {\isacharprime}c} defined as
\begin{center}
\isa{split\ {\isasymequiv}\ {\isasymlambda}c\ p{\isachardot}\ c\ {\isacharparenleft}fst\ p{\isacharparenright}\ {\isacharparenleft}snd\ p{\isacharparenright}}
\hfill(\isa{split{\isacharunderscore}def})
\end{center}
Pattern matching in
other variable binding constructs is translated similarly. Thus we need to
understand how to reason about such constructs.%
\end{isamarkuptext}%
%
\isamarkupsubsection{Theorem Proving%
}
%
\begin{isamarkuptext}%
The most obvious approach is the brute force expansion of \isa{split}:%
\end{isamarkuptext}%
\isacommand{lemma}\ {\isachardoublequote}{\isacharparenleft}{\isasymlambda}{\isacharparenleft}x{\isacharcomma}y{\isacharparenright}{\isachardot}x{\isacharparenright}\ p\ {\isacharequal}\ fst\ p{\isachardoublequote}\isanewline
\isacommand{by}{\isacharparenleft}simp\ add{\isacharcolon}split{\isacharunderscore}def{\isacharparenright}%
\begin{isamarkuptext}%
This works well if rewriting with \isa{split{\isacharunderscore}def} finishes the
proof, as it does above.  But if it doesn't, you end up with exactly what
we are trying to avoid: nests of \isa{fst} and \isa{snd}. Thus this
approach is neither elegant nor very practical in large examples, although it
can be effective in small ones.

If we step back and ponder why the above lemma presented a problem in the
first place, we quickly realize that what we would like is to replace \isa{p} with some concrete pair \isa{{\isacharparenleft}a{\isacharcomma}\ b{\isacharparenright}}, in which case both sides of the
equation would simplify to \isa{a} because of the simplification rules
\isa{split\ c\ {\isacharparenleft}a{\isacharcomma}\ b{\isacharparenright}\ {\isacharequal}\ c\ a\ b} and \isa{fst\ {\isacharparenleft}a{\isacharcomma}\ b{\isacharparenright}\ {\isacharequal}\ a}.  This is the
key problem one faces when reasoning about pattern matching with pairs: how to
convert some atomic term into a pair.

In case of a subterm of the form \isa{split\ f\ p} this is easy: the split
rule \isa{split{\isacharunderscore}split} replaces \isa{p} by a pair:%
\end{isamarkuptext}%
\isacommand{lemma}\ {\isachardoublequote}{\isacharparenleft}{\isasymlambda}{\isacharparenleft}x{\isacharcomma}y{\isacharparenright}{\isachardot}y{\isacharparenright}\ p\ {\isacharequal}\ snd\ p{\isachardoublequote}\isanewline
\isacommand{apply}{\isacharparenleft}split\ split{\isacharunderscore}split{\isacharparenright}%
\begin{isamarkuptxt}%
\begin{isabelle}%
\ {\isadigit{1}}{\isachardot}\ {\isasymforall}x\ y{\isachardot}\ p\ {\isacharequal}\ {\isacharparenleft}x{\isacharcomma}\ y{\isacharparenright}\ {\isasymlongrightarrow}\ y\ {\isacharequal}\ snd\ p%
\end{isabelle}
This subgoal is easily proved by simplification. Thus we could have combined
simplification and splitting in one command that proves the goal outright:%
\end{isamarkuptxt}%
\isacommand{by}{\isacharparenleft}simp\ split{\isacharcolon}\ split{\isacharunderscore}split{\isacharparenright}%
\begin{isamarkuptext}%
Let us look at a second example:%
\end{isamarkuptext}%
\isacommand{lemma}\ {\isachardoublequote}let\ {\isacharparenleft}x{\isacharcomma}y{\isacharparenright}\ {\isacharequal}\ p\ in\ fst\ p\ {\isacharequal}\ x{\isachardoublequote}\isanewline
\isacommand{apply}{\isacharparenleft}simp\ only{\isacharcolon}\ Let{\isacharunderscore}def{\isacharparenright}%
\begin{isamarkuptxt}%
\begin{isabelle}%
\ {\isadigit{1}}{\isachardot}\ {\isacharparenleft}{\isasymlambda}{\isacharparenleft}x{\isacharcomma}\ y{\isacharparenright}{\isachardot}\ fst\ p\ {\isacharequal}\ x{\isacharparenright}\ p%
\end{isabelle}
A paired \isa{let} reduces to a paired $\lambda$-abstraction, which
can be split as above. The same is true for paired set comprehension:%
\end{isamarkuptxt}%
\isacommand{lemma}\ {\isachardoublequote}p\ {\isasymin}\ {\isacharbraceleft}{\isacharparenleft}x{\isacharcomma}y{\isacharparenright}{\isachardot}\ x{\isacharequal}y{\isacharbraceright}\ {\isasymlongrightarrow}\ fst\ p\ {\isacharequal}\ snd\ p{\isachardoublequote}\isanewline
\isacommand{apply}\ simp%
\begin{isamarkuptxt}%
\begin{isabelle}%
\ {\isadigit{1}}{\isachardot}\ split\ op\ {\isacharequal}\ p\ {\isasymlongrightarrow}\ fst\ p\ {\isacharequal}\ snd\ p%
\end{isabelle}
Again, simplification produces a term suitable for \isa{split{\isacharunderscore}split}
as above. If you are worried about the funny form of the premise:
\isa{split\ op\ {\isacharequal}} is the same as \isa{{\isasymlambda}{\isacharparenleft}x{\isacharcomma}y{\isacharparenright}{\isachardot}\ x{\isacharequal}y}.
The same procedure works for%
\end{isamarkuptxt}%
\isacommand{lemma}\ {\isachardoublequote}p\ {\isasymin}\ {\isacharbraceleft}{\isacharparenleft}x{\isacharcomma}y{\isacharparenright}{\isachardot}\ x{\isacharequal}y{\isacharbraceright}\ {\isasymLongrightarrow}\ fst\ p\ {\isacharequal}\ snd\ p{\isachardoublequote}%
\begin{isamarkuptxt}%
\noindent
except that we now have to use \isa{split{\isacharunderscore}split{\isacharunderscore}asm}, because
\isa{split} occurs in the assumptions.

However, splitting \isa{split} is not always a solution, as no \isa{split}
may be present in the goal. Consider the following function:%
\end{isamarkuptxt}%
\isacommand{consts}\ swap\ {\isacharcolon}{\isacharcolon}\ {\isachardoublequote}{\isacharprime}a\ {\isasymtimes}\ {\isacharprime}b\ {\isasymRightarrow}\ {\isacharprime}b\ {\isasymtimes}\ {\isacharprime}a{\isachardoublequote}\isanewline
\isacommand{primrec}\isanewline
\ \ {\isachardoublequote}swap\ {\isacharparenleft}x{\isacharcomma}y{\isacharparenright}\ {\isacharequal}\ {\isacharparenleft}y{\isacharcomma}x{\isacharparenright}{\isachardoublequote}%
\begin{isamarkuptext}%
\noindent
Note that the above \isacommand{primrec} definition is admissible
because \isa{{\isasymtimes}} is a datatype. When we now try to prove%
\end{isamarkuptext}%
\isacommand{lemma}\ {\isachardoublequote}swap{\isacharparenleft}swap\ p{\isacharparenright}\ {\isacharequal}\ p{\isachardoublequote}%
\begin{isamarkuptxt}%
\noindent
simplification will do nothing, because the defining equation for \isa{swap}
expects a pair. Again, we need to turn \isa{p} into a pair first, but this
time there is no \isa{split} in sight. In this case the only thing we can do
is to split the term by hand:%
\end{isamarkuptxt}%
\isacommand{apply}{\isacharparenleft}case{\isacharunderscore}tac\ p{\isacharparenright}%
\begin{isamarkuptxt}%
\noindent
\begin{isabelle}%
\ {\isadigit{1}}{\isachardot}\ {\isasymAnd}a\ b{\isachardot}\ p\ {\isacharequal}\ {\isacharparenleft}a{\isacharcomma}\ b{\isacharparenright}\ {\isasymLongrightarrow}\ swap\ {\isacharparenleft}swap\ p{\isacharparenright}\ {\isacharequal}\ p%
\end{isabelle}
Again, \isa{case{\isacharunderscore}tac} is applicable because \isa{{\isasymtimes}} is a datatype.
The subgoal is easily proved by \isa{simp}.

Splitting by \isa{case{\isacharunderscore}tac} also solves the previous examples and may thus
appear preferable to the more arcane methods introduced first. However, see
the warning about \isa{case{\isacharunderscore}tac} in \S\ref{sec:struct-ind-case}.

In case the term to be split is a quantified variable, there are more options.
You can split \emph{all} \isa{{\isasymAnd}}-quantified variables in a goal
with the rewrite rule \isa{split{\isacharunderscore}paired{\isacharunderscore}all}:%
\end{isamarkuptxt}%
\isacommand{lemma}\ {\isachardoublequote}{\isasymAnd}p\ q{\isachardot}\ swap{\isacharparenleft}swap\ p{\isacharparenright}\ {\isacharequal}\ q\ {\isasymlongrightarrow}\ p\ {\isacharequal}\ q{\isachardoublequote}\isanewline
\isacommand{apply}{\isacharparenleft}simp\ only{\isacharcolon}split{\isacharunderscore}paired{\isacharunderscore}all{\isacharparenright}%
\begin{isamarkuptxt}%
\noindent
\begin{isabelle}%
\ {\isadigit{1}}{\isachardot}\ {\isasymAnd}a\ b\ aa\ ba{\isachardot}\isanewline
\isaindent{\ {\isadigit{1}}{\isachardot}\ \ \ \ }swap\ {\isacharparenleft}swap\ {\isacharparenleft}a{\isacharcomma}\ b{\isacharparenright}{\isacharparenright}\ {\isacharequal}\ {\isacharparenleft}aa{\isacharcomma}\ ba{\isacharparenright}\ {\isasymlongrightarrow}\ {\isacharparenleft}a{\isacharcomma}\ b{\isacharparenright}\ {\isacharequal}\ {\isacharparenleft}aa{\isacharcomma}\ ba{\isacharparenright}%
\end{isabelle}%
\end{isamarkuptxt}%
\isacommand{apply}\ simp\isanewline
\isacommand{done}%
\begin{isamarkuptext}%
\noindent
Note that we have intentionally included only \isa{split{\isacharunderscore}paired{\isacharunderscore}all}
in the first simplification step. This time the reason was not merely
pedagogical:
\isa{split{\isacharunderscore}paired{\isacharunderscore}all} may interfere with certain congruence
rules of the simplifier, i.e.%
\end{isamarkuptext}%
\isacommand{apply}{\isacharparenleft}simp\ add{\isacharcolon}split{\isacharunderscore}paired{\isacharunderscore}all{\isacharparenright}%
\begin{isamarkuptext}%
\noindent
may fail (here it does not) where the above two stages succeed.

Finally, all \isa{{\isasymforall}} and \isa{{\isasymexists}}-quantified variables are split
automatically by the simplifier:%
\end{isamarkuptext}%
\isacommand{lemma}\ {\isachardoublequote}{\isasymforall}p{\isachardot}\ {\isasymexists}q{\isachardot}\ swap\ p\ {\isacharequal}\ swap\ q{\isachardoublequote}\isanewline
\isacommand{apply}\ simp\isanewline
\isacommand{done}%
\begin{isamarkuptext}%
\noindent
In case you would like to turn off this automatic splitting, just disable the
responsible simplification rules:
\begin{center}
\isa{{\isacharparenleft}{\isasymforall}x{\isachardot}\ P\ x{\isacharparenright}\ {\isacharequal}\ {\isacharparenleft}{\isasymforall}a\ b{\isachardot}\ P\ {\isacharparenleft}a{\isacharcomma}\ b{\isacharparenright}{\isacharparenright}}
\hfill
(\isa{split{\isacharunderscore}paired{\isacharunderscore}All})\\
\isa{{\isacharparenleft}{\isasymexists}x{\isachardot}\ P\ x{\isacharparenright}\ {\isacharequal}\ {\isacharparenleft}{\isasymexists}a\ b{\isachardot}\ P\ {\isacharparenleft}a{\isacharcomma}\ b{\isacharparenright}{\isacharparenright}}
\hfill
(\isa{split{\isacharunderscore}paired{\isacharunderscore}Ex})
\end{center}%
\end{isamarkuptext}%
\end{isabellebody}%
%%% Local Variables:
%%% mode: latex
%%% TeX-master: "root"
%%% End:
    %%%Section "Pairs and Tuples"
\index{pairs and tuples|)}

%
\begin{isabellebody}%
\def\isabellecontext{Records}%
%
\isamarkupheader{Records \label{sec:records}%
}
\isamarkuptrue%
%
\isadelimtheory
%
\endisadelimtheory
%
\isatagtheory
%
\endisatagtheory
{\isafoldtheory}%
%
\isadelimtheory
%
\endisadelimtheory
%
\begin{isamarkuptext}%
\index{records|(}%
  Records are familiar from programming languages.  A record of $n$
  fields is essentially an $n$-tuple, but the record's components have
  names, which can make expressions easier to read and reduces the
  risk of confusing one field for another.

  A record of Isabelle/HOL covers a collection of fields, with select
  and update operations.  Each field has a specified type, which may
  be polymorphic.  The field names are part of the record type, and
  the order of the fields is significant --- as it is in Pascal but
  not in Standard ML.  If two different record types have field names
  in common, then the ambiguity is resolved in the usual way, by
  qualified names.

  Record types can also be defined by extending other record types.
  Extensible records make use of the reserved pseudo-field \cdx{more},
  which is present in every record type.  Generic record operations
  work on all possible extensions of a given type scheme; polymorphism
  takes care of structural sub-typing behind the scenes.  There are
  also explicit coercion functions between fixed record types.%
\end{isamarkuptext}%
\isamarkuptrue%
%
\isamarkupsubsection{Record Basics%
}
\isamarkuptrue%
%
\begin{isamarkuptext}%
Record types are not primitive in Isabelle and have a delicate
  internal representation \cite{NaraschewskiW-TPHOLs98}, based on
  nested copies of the primitive product type.  A \commdx{record}
  declaration introduces a new record type scheme by specifying its
  fields, which are packaged internally to hold up the perception of
  the record as a distinguished entity.  Here is a simple example:%
\end{isamarkuptext}%
\isamarkuptrue%
\isacommand{record}\isamarkupfalse%
\ point\ {\isaliteral{3D}{\isacharequal}}\isanewline
\ \ Xcoord\ {\isaliteral{3A}{\isacharcolon}}{\isaliteral{3A}{\isacharcolon}}\ int\isanewline
\ \ Ycoord\ {\isaliteral{3A}{\isacharcolon}}{\isaliteral{3A}{\isacharcolon}}\ int%
\begin{isamarkuptext}%
\noindent
  Records of type \isa{point} have two fields named \isa{Xcoord}
  and \isa{Ycoord}, both of type~\isa{int}.  We now define a
  constant of type \isa{point}:%
\end{isamarkuptext}%
\isamarkuptrue%
\isacommand{definition}\isamarkupfalse%
\ pt{\isadigit{1}}\ {\isaliteral{3A}{\isacharcolon}}{\isaliteral{3A}{\isacharcolon}}\ point\ \isakeyword{where}\isanewline
{\isaliteral{22}{\isachardoublequoteopen}}pt{\isadigit{1}}\ {\isaliteral{5C3C65717569763E}{\isasymequiv}}\ {\isaliteral{28}{\isacharparenleft}}{\isaliteral{7C}{\isacharbar}}\ Xcoord\ {\isaliteral{3D}{\isacharequal}}\ {\isadigit{9}}{\isadigit{9}}{\isadigit{9}}{\isaliteral{2C}{\isacharcomma}}\ Ycoord\ {\isaliteral{3D}{\isacharequal}}\ {\isadigit{2}}{\isadigit{3}}\ {\isaliteral{7C}{\isacharbar}}{\isaliteral{29}{\isacharparenright}}{\isaliteral{22}{\isachardoublequoteclose}}%
\begin{isamarkuptext}%
\noindent
  We see above the ASCII notation for record brackets.  You can also
  use the symbolic brackets \isa{{\isaliteral{5C3C6C706172723E}{\isasymlparr}}} and \isa{{\isaliteral{5C3C72706172723E}{\isasymrparr}}}.  Record type
  expressions can be also written directly with individual fields.
  The type name above is merely an abbreviation.%
\end{isamarkuptext}%
\isamarkuptrue%
\isacommand{definition}\isamarkupfalse%
\ pt{\isadigit{2}}\ {\isaliteral{3A}{\isacharcolon}}{\isaliteral{3A}{\isacharcolon}}\ {\isaliteral{22}{\isachardoublequoteopen}}{\isaliteral{5C3C6C706172723E}{\isasymlparr}}Xcoord\ {\isaliteral{3A}{\isacharcolon}}{\isaliteral{3A}{\isacharcolon}}\ int{\isaliteral{2C}{\isacharcomma}}\ Ycoord\ {\isaliteral{3A}{\isacharcolon}}{\isaliteral{3A}{\isacharcolon}}\ int{\isaliteral{5C3C72706172723E}{\isasymrparr}}{\isaliteral{22}{\isachardoublequoteclose}}\ \isakeyword{where}\isanewline
{\isaliteral{22}{\isachardoublequoteopen}}pt{\isadigit{2}}\ {\isaliteral{5C3C65717569763E}{\isasymequiv}}\ {\isaliteral{5C3C6C706172723E}{\isasymlparr}}Xcoord\ {\isaliteral{3D}{\isacharequal}}\ {\isaliteral{2D}{\isacharminus}}{\isadigit{4}}{\isadigit{5}}{\isaliteral{2C}{\isacharcomma}}\ Ycoord\ {\isaliteral{3D}{\isacharequal}}\ {\isadigit{9}}{\isadigit{7}}{\isaliteral{5C3C72706172723E}{\isasymrparr}}{\isaliteral{22}{\isachardoublequoteclose}}%
\begin{isamarkuptext}%
For each field, there is a \emph{selector}\index{selector!record}
  function of the same name.  For example, if \isa{p} has type \isa{point} then \isa{Xcoord\ p} denotes the value of the \isa{Xcoord} field of~\isa{p}.  Expressions involving field selection
  of explicit records are simplified automatically:%
\end{isamarkuptext}%
\isamarkuptrue%
\isacommand{lemma}\isamarkupfalse%
\ {\isaliteral{22}{\isachardoublequoteopen}}Xcoord\ {\isaliteral{5C3C6C706172723E}{\isasymlparr}}Xcoord\ {\isaliteral{3D}{\isacharequal}}\ a{\isaliteral{2C}{\isacharcomma}}\ Ycoord\ {\isaliteral{3D}{\isacharequal}}\ b{\isaliteral{5C3C72706172723E}{\isasymrparr}}\ {\isaliteral{3D}{\isacharequal}}\ a{\isaliteral{22}{\isachardoublequoteclose}}\isanewline
%
\isadelimproof
\ \ %
\endisadelimproof
%
\isatagproof
\isacommand{by}\isamarkupfalse%
\ simp%
\endisatagproof
{\isafoldproof}%
%
\isadelimproof
%
\endisadelimproof
%
\begin{isamarkuptext}%
The \emph{update}\index{update!record} operation is functional.  For
  example, \isa{p{\isaliteral{5C3C6C706172723E}{\isasymlparr}}Xcoord\ {\isaliteral{3A}{\isacharcolon}}{\isaliteral{3D}{\isacharequal}}\ {\isadigit{0}}{\isaliteral{5C3C72706172723E}{\isasymrparr}}} is a record whose \isa{Xcoord}
  value is zero and whose \isa{Ycoord} value is copied from~\isa{p}.  Updates of explicit records are also simplified automatically:%
\end{isamarkuptext}%
\isamarkuptrue%
\isacommand{lemma}\isamarkupfalse%
\ {\isaliteral{22}{\isachardoublequoteopen}}{\isaliteral{5C3C6C706172723E}{\isasymlparr}}Xcoord\ {\isaliteral{3D}{\isacharequal}}\ a{\isaliteral{2C}{\isacharcomma}}\ Ycoord\ {\isaliteral{3D}{\isacharequal}}\ b{\isaliteral{5C3C72706172723E}{\isasymrparr}}{\isaliteral{5C3C6C706172723E}{\isasymlparr}}Xcoord\ {\isaliteral{3A}{\isacharcolon}}{\isaliteral{3D}{\isacharequal}}\ {\isadigit{0}}{\isaliteral{5C3C72706172723E}{\isasymrparr}}\ {\isaliteral{3D}{\isacharequal}}\isanewline
\ \ \ \ \ \ \ \ \ {\isaliteral{5C3C6C706172723E}{\isasymlparr}}Xcoord\ {\isaliteral{3D}{\isacharequal}}\ {\isadigit{0}}{\isaliteral{2C}{\isacharcomma}}\ Ycoord\ {\isaliteral{3D}{\isacharequal}}\ b{\isaliteral{5C3C72706172723E}{\isasymrparr}}{\isaliteral{22}{\isachardoublequoteclose}}\isanewline
%
\isadelimproof
\ \ %
\endisadelimproof
%
\isatagproof
\isacommand{by}\isamarkupfalse%
\ simp%
\endisatagproof
{\isafoldproof}%
%
\isadelimproof
%
\endisadelimproof
%
\begin{isamarkuptext}%
\begin{warn}
  Field names are declared as constants and can no longer be used as
  variables.  It would be unwise, for example, to call the fields of
  type \isa{point} simply \isa{x} and~\isa{y}.
  \end{warn}%
\end{isamarkuptext}%
\isamarkuptrue%
%
\isamarkupsubsection{Extensible Records and Generic Operations%
}
\isamarkuptrue%
%
\begin{isamarkuptext}%
\index{records!extensible|(}%

  Now, let us define coloured points (type \isa{cpoint}) to be
  points extended with a field \isa{col} of type \isa{colour}:%
\end{isamarkuptext}%
\isamarkuptrue%
\isacommand{datatype}\isamarkupfalse%
\ colour\ {\isaliteral{3D}{\isacharequal}}\ Red\ {\isaliteral{7C}{\isacharbar}}\ Green\ {\isaliteral{7C}{\isacharbar}}\ Blue\isanewline
\isanewline
\isacommand{record}\isamarkupfalse%
\ cpoint\ {\isaliteral{3D}{\isacharequal}}\ point\ {\isaliteral{2B}{\isacharplus}}\isanewline
\ \ col\ {\isaliteral{3A}{\isacharcolon}}{\isaliteral{3A}{\isacharcolon}}\ colour%
\begin{isamarkuptext}%
\noindent
  The fields of this new type are \isa{Xcoord}, \isa{Ycoord} and
  \isa{col}, in that order.%
\end{isamarkuptext}%
\isamarkuptrue%
\isacommand{definition}\isamarkupfalse%
\ cpt{\isadigit{1}}\ {\isaliteral{3A}{\isacharcolon}}{\isaliteral{3A}{\isacharcolon}}\ cpoint\ \isakeyword{where}\isanewline
{\isaliteral{22}{\isachardoublequoteopen}}cpt{\isadigit{1}}\ {\isaliteral{5C3C65717569763E}{\isasymequiv}}\ {\isaliteral{5C3C6C706172723E}{\isasymlparr}}Xcoord\ {\isaliteral{3D}{\isacharequal}}\ {\isadigit{9}}{\isadigit{9}}{\isadigit{9}}{\isaliteral{2C}{\isacharcomma}}\ Ycoord\ {\isaliteral{3D}{\isacharequal}}\ {\isadigit{2}}{\isadigit{3}}{\isaliteral{2C}{\isacharcomma}}\ col\ {\isaliteral{3D}{\isacharequal}}\ Green{\isaliteral{5C3C72706172723E}{\isasymrparr}}{\isaliteral{22}{\isachardoublequoteclose}}%
\begin{isamarkuptext}%
We can define generic operations that work on arbitrary
  instances of a record scheme, e.g.\ covering \isa{point}, \isa{cpoint}, and any further extensions.  Every record structure has an
  implicit pseudo-field, \cdx{more}, that keeps the extension as an
  explicit value.  Its type is declared as completely
  polymorphic:~\isa{{\isaliteral{27}{\isacharprime}}a}.  When a fixed record value is expressed
  using just its standard fields, the value of \isa{more} is
  implicitly set to \isa{{\isaliteral{28}{\isacharparenleft}}{\isaliteral{29}{\isacharparenright}}}, the empty tuple, which has type
  \isa{unit}.  Within the record brackets, you can refer to the
  \isa{more} field by writing ``\isa{{\isaliteral{5C3C646F74733E}{\isasymdots}}}'' (three dots):%
\end{isamarkuptext}%
\isamarkuptrue%
\isacommand{lemma}\isamarkupfalse%
\ {\isaliteral{22}{\isachardoublequoteopen}}Xcoord\ {\isaliteral{5C3C6C706172723E}{\isasymlparr}}Xcoord\ {\isaliteral{3D}{\isacharequal}}\ a{\isaliteral{2C}{\isacharcomma}}\ Ycoord\ {\isaliteral{3D}{\isacharequal}}\ b{\isaliteral{2C}{\isacharcomma}}\ {\isaliteral{5C3C646F74733E}{\isasymdots}}\ {\isaliteral{3D}{\isacharequal}}\ p{\isaliteral{5C3C72706172723E}{\isasymrparr}}\ {\isaliteral{3D}{\isacharequal}}\ a{\isaliteral{22}{\isachardoublequoteclose}}\isanewline
%
\isadelimproof
\ \ %
\endisadelimproof
%
\isatagproof
\isacommand{by}\isamarkupfalse%
\ simp%
\endisatagproof
{\isafoldproof}%
%
\isadelimproof
%
\endisadelimproof
%
\begin{isamarkuptext}%
This lemma applies to any record whose first two fields are \isa{Xcoord} and~\isa{Ycoord}.  Note that \isa{{\isaliteral{5C3C6C706172723E}{\isasymlparr}}Xcoord\ {\isaliteral{3D}{\isacharequal}}\ a{\isaliteral{2C}{\isacharcomma}}\ Ycoord\ {\isaliteral{3D}{\isacharequal}}\ b{\isaliteral{2C}{\isacharcomma}}\ {\isaliteral{5C3C646F74733E}{\isasymdots}}\ {\isaliteral{3D}{\isacharequal}}\ {\isaliteral{28}{\isacharparenleft}}{\isaliteral{29}{\isacharparenright}}{\isaliteral{5C3C72706172723E}{\isasymrparr}}} is exactly the same as \isa{{\isaliteral{5C3C6C706172723E}{\isasymlparr}}Xcoord\ {\isaliteral{3D}{\isacharequal}}\ a{\isaliteral{2C}{\isacharcomma}}\ Ycoord\ {\isaliteral{3D}{\isacharequal}}\ b{\isaliteral{5C3C72706172723E}{\isasymrparr}}}.  Selectors and updates are always polymorphic wrt.\ the
  \isa{more} part of a record scheme, its value is just ignored (for
  select) or copied (for update).

  The \isa{more} pseudo-field may be manipulated directly as well,
  but the identifier needs to be qualified:%
\end{isamarkuptext}%
\isamarkuptrue%
\isacommand{lemma}\isamarkupfalse%
\ {\isaliteral{22}{\isachardoublequoteopen}}point{\isaliteral{2E}{\isachardot}}more\ cpt{\isadigit{1}}\ {\isaliteral{3D}{\isacharequal}}\ {\isaliteral{5C3C6C706172723E}{\isasymlparr}}col\ {\isaliteral{3D}{\isacharequal}}\ Green{\isaliteral{5C3C72706172723E}{\isasymrparr}}{\isaliteral{22}{\isachardoublequoteclose}}\isanewline
%
\isadelimproof
\ \ %
\endisadelimproof
%
\isatagproof
\isacommand{by}\isamarkupfalse%
\ {\isaliteral{28}{\isacharparenleft}}simp\ add{\isaliteral{3A}{\isacharcolon}}\ cpt{\isadigit{1}}{\isaliteral{5F}{\isacharunderscore}}def{\isaliteral{29}{\isacharparenright}}%
\endisatagproof
{\isafoldproof}%
%
\isadelimproof
%
\endisadelimproof
%
\begin{isamarkuptext}%
\noindent
  We see that the colour part attached to this \isa{point} is a
  rudimentary record in its own right, namely \isa{{\isaliteral{5C3C6C706172723E}{\isasymlparr}}col\ {\isaliteral{3D}{\isacharequal}}\ Green{\isaliteral{5C3C72706172723E}{\isasymrparr}}}.  In order to select or update \isa{col}, this fragment
  needs to be put back into the context of the parent type scheme, say
  as \isa{more} part of another \isa{point}.

  To define generic operations, we need to know a bit more about
  records.  Our definition of \isa{point} above has generated two
  type abbreviations:

  \medskip
  \begin{tabular}{l}
  \isa{point}~\isa{{\isaliteral{3D}{\isacharequal}}}~\isa{{\isaliteral{5C3C6C706172723E}{\isasymlparr}}Xcoord\ {\isaliteral{3A}{\isacharcolon}}{\isaliteral{3A}{\isacharcolon}}\ int{\isaliteral{2C}{\isacharcomma}}\ Ycoord\ {\isaliteral{3A}{\isacharcolon}}{\isaliteral{3A}{\isacharcolon}}\ int{\isaliteral{5C3C72706172723E}{\isasymrparr}}} \\
  \isa{{\isaliteral{27}{\isacharprime}}a\ point{\isaliteral{5F}{\isacharunderscore}}scheme}~\isa{{\isaliteral{3D}{\isacharequal}}}~\isa{{\isaliteral{5C3C6C706172723E}{\isasymlparr}}Xcoord\ {\isaliteral{3A}{\isacharcolon}}{\isaliteral{3A}{\isacharcolon}}\ int{\isaliteral{2C}{\isacharcomma}}\ Ycoord\ {\isaliteral{3A}{\isacharcolon}}{\isaliteral{3A}{\isacharcolon}}\ int{\isaliteral{2C}{\isacharcomma}}\ {\isaliteral{5C3C646F74733E}{\isasymdots}}\ {\isaliteral{3A}{\isacharcolon}}{\isaliteral{3A}{\isacharcolon}}\ {\isaliteral{27}{\isacharprime}}a{\isaliteral{5C3C72706172723E}{\isasymrparr}}} \\
  \end{tabular}
  \medskip
  
\noindent
  Type \isa{point} is for fixed records having exactly the two fields
  \isa{Xcoord} and~\isa{Ycoord}, while the polymorphic type \isa{{\isaliteral{27}{\isacharprime}}a\ point{\isaliteral{5F}{\isacharunderscore}}scheme} comprises all possible extensions to those two
  fields.  Note that \isa{unit\ point{\isaliteral{5F}{\isacharunderscore}}scheme} coincides with \isa{point}, and \isa{{\isaliteral{5C3C6C706172723E}{\isasymlparr}}col\ {\isaliteral{3A}{\isacharcolon}}{\isaliteral{3A}{\isacharcolon}}\ colour{\isaliteral{5C3C72706172723E}{\isasymrparr}}\ point{\isaliteral{5F}{\isacharunderscore}}scheme} with \isa{cpoint}.

  In the following example we define two operations --- methods, if we
  regard records as objects --- to get and set any point's \isa{Xcoord} field.%
\end{isamarkuptext}%
\isamarkuptrue%
\isacommand{definition}\isamarkupfalse%
\ getX\ {\isaliteral{3A}{\isacharcolon}}{\isaliteral{3A}{\isacharcolon}}\ {\isaliteral{22}{\isachardoublequoteopen}}{\isaliteral{27}{\isacharprime}}a\ point{\isaliteral{5F}{\isacharunderscore}}scheme\ {\isaliteral{5C3C52696768746172726F773E}{\isasymRightarrow}}\ int{\isaliteral{22}{\isachardoublequoteclose}}\ \isakeyword{where}\isanewline
{\isaliteral{22}{\isachardoublequoteopen}}getX\ r\ {\isaliteral{5C3C65717569763E}{\isasymequiv}}\ Xcoord\ r{\isaliteral{22}{\isachardoublequoteclose}}\isanewline
\isacommand{definition}\isamarkupfalse%
\ setX\ {\isaliteral{3A}{\isacharcolon}}{\isaliteral{3A}{\isacharcolon}}\ {\isaliteral{22}{\isachardoublequoteopen}}{\isaliteral{27}{\isacharprime}}a\ point{\isaliteral{5F}{\isacharunderscore}}scheme\ {\isaliteral{5C3C52696768746172726F773E}{\isasymRightarrow}}\ int\ {\isaliteral{5C3C52696768746172726F773E}{\isasymRightarrow}}\ {\isaliteral{27}{\isacharprime}}a\ point{\isaliteral{5F}{\isacharunderscore}}scheme{\isaliteral{22}{\isachardoublequoteclose}}\ \isakeyword{where}\isanewline
{\isaliteral{22}{\isachardoublequoteopen}}setX\ r\ a\ {\isaliteral{5C3C65717569763E}{\isasymequiv}}\ r{\isaliteral{5C3C6C706172723E}{\isasymlparr}}Xcoord\ {\isaliteral{3A}{\isacharcolon}}{\isaliteral{3D}{\isacharequal}}\ a{\isaliteral{5C3C72706172723E}{\isasymrparr}}{\isaliteral{22}{\isachardoublequoteclose}}%
\begin{isamarkuptext}%
Here is a generic method that modifies a point, incrementing its
  \isa{Xcoord} field.  The \isa{Ycoord} and \isa{more} fields
  are copied across.  It works for any record type scheme derived from
  \isa{point} (including \isa{cpoint} etc.):%
\end{isamarkuptext}%
\isamarkuptrue%
\isacommand{definition}\isamarkupfalse%
\ incX\ {\isaliteral{3A}{\isacharcolon}}{\isaliteral{3A}{\isacharcolon}}\ {\isaliteral{22}{\isachardoublequoteopen}}{\isaliteral{27}{\isacharprime}}a\ point{\isaliteral{5F}{\isacharunderscore}}scheme\ {\isaliteral{5C3C52696768746172726F773E}{\isasymRightarrow}}\ {\isaliteral{27}{\isacharprime}}a\ point{\isaliteral{5F}{\isacharunderscore}}scheme{\isaliteral{22}{\isachardoublequoteclose}}\ \isakeyword{where}\isanewline
{\isaliteral{22}{\isachardoublequoteopen}}incX\ r\ {\isaliteral{5C3C65717569763E}{\isasymequiv}}\isanewline
\ \ {\isaliteral{5C3C6C706172723E}{\isasymlparr}}Xcoord\ {\isaliteral{3D}{\isacharequal}}\ Xcoord\ r\ {\isaliteral{2B}{\isacharplus}}\ {\isadigit{1}}{\isaliteral{2C}{\isacharcomma}}\ Ycoord\ {\isaliteral{3D}{\isacharequal}}\ Ycoord\ r{\isaliteral{2C}{\isacharcomma}}\ {\isaliteral{5C3C646F74733E}{\isasymdots}}\ {\isaliteral{3D}{\isacharequal}}\ point{\isaliteral{2E}{\isachardot}}more\ r{\isaliteral{5C3C72706172723E}{\isasymrparr}}{\isaliteral{22}{\isachardoublequoteclose}}%
\begin{isamarkuptext}%
Generic theorems can be proved about generic methods.  This trivial
  lemma relates \isa{incX} to \isa{getX} and \isa{setX}:%
\end{isamarkuptext}%
\isamarkuptrue%
\isacommand{lemma}\isamarkupfalse%
\ {\isaliteral{22}{\isachardoublequoteopen}}incX\ r\ {\isaliteral{3D}{\isacharequal}}\ setX\ r\ {\isaliteral{28}{\isacharparenleft}}getX\ r\ {\isaliteral{2B}{\isacharplus}}\ {\isadigit{1}}{\isaliteral{29}{\isacharparenright}}{\isaliteral{22}{\isachardoublequoteclose}}\isanewline
%
\isadelimproof
\ \ %
\endisadelimproof
%
\isatagproof
\isacommand{by}\isamarkupfalse%
\ {\isaliteral{28}{\isacharparenleft}}simp\ add{\isaliteral{3A}{\isacharcolon}}\ getX{\isaliteral{5F}{\isacharunderscore}}def\ setX{\isaliteral{5F}{\isacharunderscore}}def\ incX{\isaliteral{5F}{\isacharunderscore}}def{\isaliteral{29}{\isacharparenright}}%
\endisatagproof
{\isafoldproof}%
%
\isadelimproof
%
\endisadelimproof
%
\begin{isamarkuptext}%
\begin{warn}
  If you use the symbolic record brackets \isa{{\isaliteral{5C3C6C706172723E}{\isasymlparr}}} and \isa{{\isaliteral{5C3C72706172723E}{\isasymrparr}}},
  then you must also use the symbolic ellipsis, ``\isa{{\isaliteral{5C3C646F74733E}{\isasymdots}}}'', rather
  than three consecutive periods, ``\isa{{\isaliteral{2E}{\isachardot}}{\isaliteral{2E}{\isachardot}}{\isaliteral{2E}{\isachardot}}}''.  Mixing the ASCII
  and symbolic versions causes a syntax error.  (The two versions are
  more distinct on screen than they are on paper.)
  \end{warn}%
  \index{records!extensible|)}%
\end{isamarkuptext}%
\isamarkuptrue%
%
\isamarkupsubsection{Record Equality%
}
\isamarkuptrue%
%
\begin{isamarkuptext}%
Two records are equal\index{equality!of records} if all pairs of
  corresponding fields are equal.  Concrete record equalities are
  simplified automatically:%
\end{isamarkuptext}%
\isamarkuptrue%
\isacommand{lemma}\isamarkupfalse%
\ {\isaliteral{22}{\isachardoublequoteopen}}{\isaliteral{28}{\isacharparenleft}}{\isaliteral{5C3C6C706172723E}{\isasymlparr}}Xcoord\ {\isaliteral{3D}{\isacharequal}}\ a{\isaliteral{2C}{\isacharcomma}}\ Ycoord\ {\isaliteral{3D}{\isacharequal}}\ b{\isaliteral{5C3C72706172723E}{\isasymrparr}}\ {\isaliteral{3D}{\isacharequal}}\ {\isaliteral{5C3C6C706172723E}{\isasymlparr}}Xcoord\ {\isaliteral{3D}{\isacharequal}}\ a{\isaliteral{27}{\isacharprime}}{\isaliteral{2C}{\isacharcomma}}\ Ycoord\ {\isaliteral{3D}{\isacharequal}}\ b{\isaliteral{27}{\isacharprime}}{\isaliteral{5C3C72706172723E}{\isasymrparr}}{\isaliteral{29}{\isacharparenright}}\ {\isaliteral{3D}{\isacharequal}}\isanewline
\ \ \ \ {\isaliteral{28}{\isacharparenleft}}a\ {\isaliteral{3D}{\isacharequal}}\ a{\isaliteral{27}{\isacharprime}}\ {\isaliteral{5C3C616E643E}{\isasymand}}\ b\ {\isaliteral{3D}{\isacharequal}}\ b{\isaliteral{27}{\isacharprime}}{\isaliteral{29}{\isacharparenright}}{\isaliteral{22}{\isachardoublequoteclose}}\isanewline
%
\isadelimproof
\ \ %
\endisadelimproof
%
\isatagproof
\isacommand{by}\isamarkupfalse%
\ simp%
\endisatagproof
{\isafoldproof}%
%
\isadelimproof
%
\endisadelimproof
%
\begin{isamarkuptext}%
The following equality is similar, but generic, in that \isa{r}
  can be any instance of \isa{{\isaliteral{27}{\isacharprime}}a\ point{\isaliteral{5F}{\isacharunderscore}}scheme}:%
\end{isamarkuptext}%
\isamarkuptrue%
\isacommand{lemma}\isamarkupfalse%
\ {\isaliteral{22}{\isachardoublequoteopen}}r{\isaliteral{5C3C6C706172723E}{\isasymlparr}}Xcoord\ {\isaliteral{3A}{\isacharcolon}}{\isaliteral{3D}{\isacharequal}}\ a{\isaliteral{2C}{\isacharcomma}}\ Ycoord\ {\isaliteral{3A}{\isacharcolon}}{\isaliteral{3D}{\isacharequal}}\ b{\isaliteral{5C3C72706172723E}{\isasymrparr}}\ {\isaliteral{3D}{\isacharequal}}\ r{\isaliteral{5C3C6C706172723E}{\isasymlparr}}Ycoord\ {\isaliteral{3A}{\isacharcolon}}{\isaliteral{3D}{\isacharequal}}\ b{\isaliteral{2C}{\isacharcomma}}\ Xcoord\ {\isaliteral{3A}{\isacharcolon}}{\isaliteral{3D}{\isacharequal}}\ a{\isaliteral{5C3C72706172723E}{\isasymrparr}}{\isaliteral{22}{\isachardoublequoteclose}}\isanewline
%
\isadelimproof
\ \ %
\endisadelimproof
%
\isatagproof
\isacommand{by}\isamarkupfalse%
\ simp%
\endisatagproof
{\isafoldproof}%
%
\isadelimproof
%
\endisadelimproof
%
\begin{isamarkuptext}%
\noindent
  We see above the syntax for iterated updates.  We could equivalently
  have written the left-hand side as \isa{r{\isaliteral{5C3C6C706172723E}{\isasymlparr}}Xcoord\ {\isaliteral{3A}{\isacharcolon}}{\isaliteral{3D}{\isacharequal}}\ a{\isaliteral{5C3C72706172723E}{\isasymrparr}}{\isaliteral{5C3C6C706172723E}{\isasymlparr}}Ycoord\ {\isaliteral{3A}{\isacharcolon}}{\isaliteral{3D}{\isacharequal}}\ b{\isaliteral{5C3C72706172723E}{\isasymrparr}}}.

  Record equality is \emph{extensional}:
  \index{extensionality!for records} a record is determined entirely
  by the values of its fields.%
\end{isamarkuptext}%
\isamarkuptrue%
\isacommand{lemma}\isamarkupfalse%
\ {\isaliteral{22}{\isachardoublequoteopen}}r\ {\isaliteral{3D}{\isacharequal}}\ {\isaliteral{5C3C6C706172723E}{\isasymlparr}}Xcoord\ {\isaliteral{3D}{\isacharequal}}\ Xcoord\ r{\isaliteral{2C}{\isacharcomma}}\ Ycoord\ {\isaliteral{3D}{\isacharequal}}\ Ycoord\ r{\isaliteral{5C3C72706172723E}{\isasymrparr}}{\isaliteral{22}{\isachardoublequoteclose}}\isanewline
%
\isadelimproof
\ \ %
\endisadelimproof
%
\isatagproof
\isacommand{by}\isamarkupfalse%
\ simp%
\endisatagproof
{\isafoldproof}%
%
\isadelimproof
%
\endisadelimproof
%
\begin{isamarkuptext}%
\noindent
  The generic version of this equality includes the pseudo-field
  \isa{more}:%
\end{isamarkuptext}%
\isamarkuptrue%
\isacommand{lemma}\isamarkupfalse%
\ {\isaliteral{22}{\isachardoublequoteopen}}r\ {\isaliteral{3D}{\isacharequal}}\ {\isaliteral{5C3C6C706172723E}{\isasymlparr}}Xcoord\ {\isaliteral{3D}{\isacharequal}}\ Xcoord\ r{\isaliteral{2C}{\isacharcomma}}\ Ycoord\ {\isaliteral{3D}{\isacharequal}}\ Ycoord\ r{\isaliteral{2C}{\isacharcomma}}\ {\isaliteral{5C3C646F74733E}{\isasymdots}}\ {\isaliteral{3D}{\isacharequal}}\ point{\isaliteral{2E}{\isachardot}}more\ r{\isaliteral{5C3C72706172723E}{\isasymrparr}}{\isaliteral{22}{\isachardoublequoteclose}}\isanewline
%
\isadelimproof
\ \ %
\endisadelimproof
%
\isatagproof
\isacommand{by}\isamarkupfalse%
\ simp%
\endisatagproof
{\isafoldproof}%
%
\isadelimproof
%
\endisadelimproof
%
\begin{isamarkuptext}%
The simplifier can prove many record equalities
  automatically, but general equality reasoning can be tricky.
  Consider proving this obvious fact:%
\end{isamarkuptext}%
\isamarkuptrue%
\isacommand{lemma}\isamarkupfalse%
\ {\isaliteral{22}{\isachardoublequoteopen}}r{\isaliteral{5C3C6C706172723E}{\isasymlparr}}Xcoord\ {\isaliteral{3A}{\isacharcolon}}{\isaliteral{3D}{\isacharequal}}\ a{\isaliteral{5C3C72706172723E}{\isasymrparr}}\ {\isaliteral{3D}{\isacharequal}}\ r{\isaliteral{5C3C6C706172723E}{\isasymlparr}}Xcoord\ {\isaliteral{3A}{\isacharcolon}}{\isaliteral{3D}{\isacharequal}}\ a{\isaliteral{27}{\isacharprime}}{\isaliteral{5C3C72706172723E}{\isasymrparr}}\ {\isaliteral{5C3C4C6F6E6772696768746172726F773E}{\isasymLongrightarrow}}\ a\ {\isaliteral{3D}{\isacharequal}}\ a{\isaliteral{27}{\isacharprime}}{\isaliteral{22}{\isachardoublequoteclose}}\isanewline
%
\isadelimproof
\ \ %
\endisadelimproof
%
\isatagproof
\isacommand{apply}\isamarkupfalse%
\ simp{\isaliteral{3F}{\isacharquery}}\isanewline
\ \ \isacommand{oops}\isamarkupfalse%
%
\endisatagproof
{\isafoldproof}%
%
\isadelimproof
%
\endisadelimproof
%
\begin{isamarkuptext}%
\noindent
  Here the simplifier can do nothing, since general record equality is
  not eliminated automatically.  One way to proceed is by an explicit
  forward step that applies the selector \isa{Xcoord} to both sides
  of the assumed record equality:%
\end{isamarkuptext}%
\isamarkuptrue%
\isacommand{lemma}\isamarkupfalse%
\ {\isaliteral{22}{\isachardoublequoteopen}}r{\isaliteral{5C3C6C706172723E}{\isasymlparr}}Xcoord\ {\isaliteral{3A}{\isacharcolon}}{\isaliteral{3D}{\isacharequal}}\ a{\isaliteral{5C3C72706172723E}{\isasymrparr}}\ {\isaliteral{3D}{\isacharequal}}\ r{\isaliteral{5C3C6C706172723E}{\isasymlparr}}Xcoord\ {\isaliteral{3A}{\isacharcolon}}{\isaliteral{3D}{\isacharequal}}\ a{\isaliteral{27}{\isacharprime}}{\isaliteral{5C3C72706172723E}{\isasymrparr}}\ {\isaliteral{5C3C4C6F6E6772696768746172726F773E}{\isasymLongrightarrow}}\ a\ {\isaliteral{3D}{\isacharequal}}\ a{\isaliteral{27}{\isacharprime}}{\isaliteral{22}{\isachardoublequoteclose}}\isanewline
%
\isadelimproof
\ \ %
\endisadelimproof
%
\isatagproof
\isacommand{apply}\isamarkupfalse%
\ {\isaliteral{28}{\isacharparenleft}}drule{\isaliteral{5F}{\isacharunderscore}}tac\ f\ {\isaliteral{3D}{\isacharequal}}\ Xcoord\ \isakeyword{in}\ arg{\isaliteral{5F}{\isacharunderscore}}cong{\isaliteral{29}{\isacharparenright}}%
\begin{isamarkuptxt}%
\begin{isabelle}%
\ {\isadigit{1}}{\isaliteral{2E}{\isachardot}}\ Xcoord\ {\isaliteral{28}{\isacharparenleft}}r{\isaliteral{5C3C6C706172723E}{\isasymlparr}}Xcoord\ {\isaliteral{3A}{\isacharcolon}}{\isaliteral{3D}{\isacharequal}}\ a{\isaliteral{5C3C72706172723E}{\isasymrparr}}{\isaliteral{29}{\isacharparenright}}\ {\isaliteral{3D}{\isacharequal}}\ Xcoord\ {\isaliteral{28}{\isacharparenleft}}r{\isaliteral{5C3C6C706172723E}{\isasymlparr}}Xcoord\ {\isaliteral{3A}{\isacharcolon}}{\isaliteral{3D}{\isacharequal}}\ a{\isaliteral{27}{\isacharprime}}{\isaliteral{5C3C72706172723E}{\isasymrparr}}{\isaliteral{29}{\isacharparenright}}\ {\isaliteral{5C3C4C6F6E6772696768746172726F773E}{\isasymLongrightarrow}}\ a\ {\isaliteral{3D}{\isacharequal}}\ a{\isaliteral{27}{\isacharprime}}%
\end{isabelle}
    Now, \isa{simp} will reduce the assumption to the desired
    conclusion.%
\end{isamarkuptxt}%
\isamarkuptrue%
\ \ \isacommand{apply}\isamarkupfalse%
\ simp\isanewline
\ \ \isacommand{done}\isamarkupfalse%
%
\endisatagproof
{\isafoldproof}%
%
\isadelimproof
%
\endisadelimproof
%
\begin{isamarkuptext}%
The \isa{cases} method is preferable to such a forward proof.  We
  state the desired lemma again:%
\end{isamarkuptext}%
\isamarkuptrue%
\isacommand{lemma}\isamarkupfalse%
\ {\isaliteral{22}{\isachardoublequoteopen}}r{\isaliteral{5C3C6C706172723E}{\isasymlparr}}Xcoord\ {\isaliteral{3A}{\isacharcolon}}{\isaliteral{3D}{\isacharequal}}\ a{\isaliteral{5C3C72706172723E}{\isasymrparr}}\ {\isaliteral{3D}{\isacharequal}}\ r{\isaliteral{5C3C6C706172723E}{\isasymlparr}}Xcoord\ {\isaliteral{3A}{\isacharcolon}}{\isaliteral{3D}{\isacharequal}}\ a{\isaliteral{27}{\isacharprime}}{\isaliteral{5C3C72706172723E}{\isasymrparr}}\ {\isaliteral{5C3C4C6F6E6772696768746172726F773E}{\isasymLongrightarrow}}\ a\ {\isaliteral{3D}{\isacharequal}}\ a{\isaliteral{27}{\isacharprime}}{\isaliteral{22}{\isachardoublequoteclose}}%
\isadelimproof
%
\endisadelimproof
%
\isatagproof
%
\begin{isamarkuptxt}%
The \methdx{cases} method adds an equality to replace the
  named record term by an explicit record expression, listing all
  fields.  It even includes the pseudo-field \isa{more}, since the
  record equality stated here is generic for all extensions.%
\end{isamarkuptxt}%
\isamarkuptrue%
\ \ \isacommand{apply}\isamarkupfalse%
\ {\isaliteral{28}{\isacharparenleft}}cases\ r{\isaliteral{29}{\isacharparenright}}%
\begin{isamarkuptxt}%
\begin{isabelle}%
\ {\isadigit{1}}{\isaliteral{2E}{\isachardot}}\ {\isaliteral{5C3C416E643E}{\isasymAnd}}Xcoord\ Ycoord\ more{\isaliteral{2E}{\isachardot}}\isanewline
\isaindent{\ {\isadigit{1}}{\isaliteral{2E}{\isachardot}}\ \ \ \ }{\isaliteral{5C3C6C6272616B6B3E}{\isasymlbrakk}}r{\isaliteral{5C3C6C706172723E}{\isasymlparr}}Xcoord\ {\isaliteral{3A}{\isacharcolon}}{\isaliteral{3D}{\isacharequal}}\ a{\isaliteral{5C3C72706172723E}{\isasymrparr}}\ {\isaliteral{3D}{\isacharequal}}\ r{\isaliteral{5C3C6C706172723E}{\isasymlparr}}Xcoord\ {\isaliteral{3A}{\isacharcolon}}{\isaliteral{3D}{\isacharequal}}\ a{\isaliteral{27}{\isacharprime}}{\isaliteral{5C3C72706172723E}{\isasymrparr}}{\isaliteral{3B}{\isacharsemicolon}}\isanewline
\isaindent{\ {\isadigit{1}}{\isaliteral{2E}{\isachardot}}\ \ \ \ \ }r\ {\isaliteral{3D}{\isacharequal}}\ {\isaliteral{5C3C6C706172723E}{\isasymlparr}}Xcoord\ {\isaliteral{3D}{\isacharequal}}\ Xcoord{\isaliteral{2C}{\isacharcomma}}\ Ycoord\ {\isaliteral{3D}{\isacharequal}}\ Ycoord{\isaliteral{2C}{\isacharcomma}}\ {\isaliteral{5C3C646F74733E}{\isasymdots}}\ {\isaliteral{3D}{\isacharequal}}\ more{\isaliteral{5C3C72706172723E}{\isasymrparr}}{\isaliteral{5C3C726272616B6B3E}{\isasymrbrakk}}\isanewline
\isaindent{\ {\isadigit{1}}{\isaliteral{2E}{\isachardot}}\ \ \ \ }{\isaliteral{5C3C4C6F6E6772696768746172726F773E}{\isasymLongrightarrow}}\ a\ {\isaliteral{3D}{\isacharequal}}\ a{\isaliteral{27}{\isacharprime}}%
\end{isabelle} Again, \isa{simp} finishes the proof.  Because \isa{r} is now represented as
  an explicit record construction, the updates can be applied and the
  record equality can be replaced by equality of the corresponding
  fields (due to injectivity).%
\end{isamarkuptxt}%
\isamarkuptrue%
\ \ \isacommand{apply}\isamarkupfalse%
\ simp\isanewline
\ \ \isacommand{done}\isamarkupfalse%
%
\endisatagproof
{\isafoldproof}%
%
\isadelimproof
%
\endisadelimproof
%
\begin{isamarkuptext}%
The generic cases method does not admit references to locally bound
  parameters of a goal.  In longer proof scripts one might have to
  fall back on the primitive \isa{rule{\isaliteral{5F}{\isacharunderscore}}tac} used together with the
  internal field representation rules of records.  The above use of
  \isa{{\isaliteral{28}{\isacharparenleft}}cases\ r{\isaliteral{29}{\isacharparenright}}} would become \isa{{\isaliteral{28}{\isacharparenleft}}rule{\isaliteral{5F}{\isacharunderscore}}tac\ r\ {\isaliteral{3D}{\isacharequal}}\ r\ in\ point{\isaliteral{2E}{\isachardot}}cases{\isaliteral{5F}{\isacharunderscore}}scheme{\isaliteral{29}{\isacharparenright}}}.%
\end{isamarkuptext}%
\isamarkuptrue%
%
\isamarkupsubsection{Extending and Truncating Records%
}
\isamarkuptrue%
%
\begin{isamarkuptext}%
Each record declaration introduces a number of derived operations to
  refer collectively to a record's fields and to convert between fixed
  record types.  They can, for instance, convert between types \isa{point} and \isa{cpoint}.  We can add a colour to a point or convert
  a \isa{cpoint} to a \isa{point} by forgetting its colour.

  \begin{itemize}

  \item Function \cdx{make} takes as arguments all of the record's
  fields (including those inherited from ancestors).  It returns the
  corresponding record.

  \item Function \cdx{fields} takes the record's very own fields and
  returns a record fragment consisting of just those fields.  This may
  be filled into the \isa{more} part of the parent record scheme.

  \item Function \cdx{extend} takes two arguments: a record to be
  extended and a record containing the new fields.

  \item Function \cdx{truncate} takes a record (possibly an extension
  of the original record type) and returns a fixed record, removing
  any additional fields.

  \end{itemize}
  These functions provide useful abbreviations for standard
  record expressions involving constructors and selectors.  The
  definitions, which are \emph{not} unfolded by default, are made
  available by the collective name of \isa{defs} (\isa{point{\isaliteral{2E}{\isachardot}}defs}, \isa{cpoint{\isaliteral{2E}{\isachardot}}defs}, etc.).
  For example, here are the versions of those functions generated for
  record \isa{point}.  We omit \isa{point{\isaliteral{2E}{\isachardot}}fields}, which happens to
  be the same as \isa{point{\isaliteral{2E}{\isachardot}}make}.

  \begin{isabelle}%
point{\isaliteral{2E}{\isachardot}}make\ Xcoord\ Ycoord\ {\isaliteral{5C3C65717569763E}{\isasymequiv}}\ {\isaliteral{5C3C6C706172723E}{\isasymlparr}}Xcoord\ {\isaliteral{3D}{\isacharequal}}\ Xcoord{\isaliteral{2C}{\isacharcomma}}\ Ycoord\ {\isaliteral{3D}{\isacharequal}}\ Ycoord{\isaliteral{5C3C72706172723E}{\isasymrparr}}\isasep\isanewline%
point{\isaliteral{2E}{\isachardot}}extend\ r\ more\ {\isaliteral{5C3C65717569763E}{\isasymequiv}}\isanewline
{\isaliteral{5C3C6C706172723E}{\isasymlparr}}Xcoord\ {\isaliteral{3D}{\isacharequal}}\ Xcoord\ r{\isaliteral{2C}{\isacharcomma}}\ Ycoord\ {\isaliteral{3D}{\isacharequal}}\ Ycoord\ r{\isaliteral{2C}{\isacharcomma}}\ {\isaliteral{5C3C646F74733E}{\isasymdots}}\ {\isaliteral{3D}{\isacharequal}}\ more{\isaliteral{5C3C72706172723E}{\isasymrparr}}\isasep\isanewline%
point{\isaliteral{2E}{\isachardot}}truncate\ r\ {\isaliteral{5C3C65717569763E}{\isasymequiv}}\ {\isaliteral{5C3C6C706172723E}{\isasymlparr}}Xcoord\ {\isaliteral{3D}{\isacharequal}}\ Xcoord\ r{\isaliteral{2C}{\isacharcomma}}\ Ycoord\ {\isaliteral{3D}{\isacharequal}}\ Ycoord\ r{\isaliteral{5C3C72706172723E}{\isasymrparr}}%
\end{isabelle}
  Contrast those with the corresponding functions for record \isa{cpoint}.  Observe \isa{cpoint{\isaliteral{2E}{\isachardot}}fields} in particular.
  \begin{isabelle}%
cpoint{\isaliteral{2E}{\isachardot}}make\ Xcoord\ Ycoord\ col\ {\isaliteral{5C3C65717569763E}{\isasymequiv}}\isanewline
{\isaliteral{5C3C6C706172723E}{\isasymlparr}}Xcoord\ {\isaliteral{3D}{\isacharequal}}\ Xcoord{\isaliteral{2C}{\isacharcomma}}\ Ycoord\ {\isaliteral{3D}{\isacharequal}}\ Ycoord{\isaliteral{2C}{\isacharcomma}}\ col\ {\isaliteral{3D}{\isacharequal}}\ col{\isaliteral{5C3C72706172723E}{\isasymrparr}}\isasep\isanewline%
cpoint{\isaliteral{2E}{\isachardot}}fields\ col\ {\isaliteral{5C3C65717569763E}{\isasymequiv}}\ {\isaliteral{5C3C6C706172723E}{\isasymlparr}}col\ {\isaliteral{3D}{\isacharequal}}\ col{\isaliteral{5C3C72706172723E}{\isasymrparr}}\isasep\isanewline%
cpoint{\isaliteral{2E}{\isachardot}}extend\ r\ more\ {\isaliteral{5C3C65717569763E}{\isasymequiv}}\isanewline
{\isaliteral{5C3C6C706172723E}{\isasymlparr}}Xcoord\ {\isaliteral{3D}{\isacharequal}}\ Xcoord\ r{\isaliteral{2C}{\isacharcomma}}\ Ycoord\ {\isaliteral{3D}{\isacharequal}}\ Ycoord\ r{\isaliteral{2C}{\isacharcomma}}\ col\ {\isaliteral{3D}{\isacharequal}}\ col\ r{\isaliteral{2C}{\isacharcomma}}\ {\isaliteral{5C3C646F74733E}{\isasymdots}}\ {\isaliteral{3D}{\isacharequal}}\ more{\isaliteral{5C3C72706172723E}{\isasymrparr}}\isasep\isanewline%
cpoint{\isaliteral{2E}{\isachardot}}truncate\ r\ {\isaliteral{5C3C65717569763E}{\isasymequiv}}\isanewline
{\isaliteral{5C3C6C706172723E}{\isasymlparr}}Xcoord\ {\isaliteral{3D}{\isacharequal}}\ Xcoord\ r{\isaliteral{2C}{\isacharcomma}}\ Ycoord\ {\isaliteral{3D}{\isacharequal}}\ Ycoord\ r{\isaliteral{2C}{\isacharcomma}}\ col\ {\isaliteral{3D}{\isacharequal}}\ col\ r{\isaliteral{5C3C72706172723E}{\isasymrparr}}%
\end{isabelle}

  To demonstrate these functions, we declare a new coloured point by
  extending an ordinary point.  Function \isa{point{\isaliteral{2E}{\isachardot}}extend} augments
  \isa{pt{\isadigit{1}}} with a colour value, which is converted into an
  appropriate record fragment by \isa{cpoint{\isaliteral{2E}{\isachardot}}fields}.%
\end{isamarkuptext}%
\isamarkuptrue%
\isacommand{definition}\isamarkupfalse%
\ cpt{\isadigit{2}}\ {\isaliteral{3A}{\isacharcolon}}{\isaliteral{3A}{\isacharcolon}}\ cpoint\ \isakeyword{where}\isanewline
{\isaliteral{22}{\isachardoublequoteopen}}cpt{\isadigit{2}}\ {\isaliteral{5C3C65717569763E}{\isasymequiv}}\ point{\isaliteral{2E}{\isachardot}}extend\ pt{\isadigit{1}}\ {\isaliteral{28}{\isacharparenleft}}cpoint{\isaliteral{2E}{\isachardot}}fields\ Green{\isaliteral{29}{\isacharparenright}}{\isaliteral{22}{\isachardoublequoteclose}}%
\begin{isamarkuptext}%
The coloured points \isa{cpt{\isadigit{1}}} and \isa{cpt{\isadigit{2}}} are equal.  The
  proof is trivial, by unfolding all the definitions.  We deliberately
  omit the definition of~\isa{pt{\isadigit{1}}} in order to reveal the underlying
  comparison on type \isa{point}.%
\end{isamarkuptext}%
\isamarkuptrue%
\isacommand{lemma}\isamarkupfalse%
\ {\isaliteral{22}{\isachardoublequoteopen}}cpt{\isadigit{1}}\ {\isaliteral{3D}{\isacharequal}}\ cpt{\isadigit{2}}{\isaliteral{22}{\isachardoublequoteclose}}\isanewline
%
\isadelimproof
\ \ %
\endisadelimproof
%
\isatagproof
\isacommand{apply}\isamarkupfalse%
\ {\isaliteral{28}{\isacharparenleft}}simp\ add{\isaliteral{3A}{\isacharcolon}}\ cpt{\isadigit{1}}{\isaliteral{5F}{\isacharunderscore}}def\ cpt{\isadigit{2}}{\isaliteral{5F}{\isacharunderscore}}def\ point{\isaliteral{2E}{\isachardot}}defs\ cpoint{\isaliteral{2E}{\isachardot}}defs{\isaliteral{29}{\isacharparenright}}%
\begin{isamarkuptxt}%
\begin{isabelle}%
\ {\isadigit{1}}{\isaliteral{2E}{\isachardot}}\ Xcoord\ pt{\isadigit{1}}\ {\isaliteral{3D}{\isacharequal}}\ {\isadigit{9}}{\isadigit{9}}{\isadigit{9}}\ {\isaliteral{5C3C616E643E}{\isasymand}}\ Ycoord\ pt{\isadigit{1}}\ {\isaliteral{3D}{\isacharequal}}\ {\isadigit{2}}{\isadigit{3}}%
\end{isabelle}%
\end{isamarkuptxt}%
\isamarkuptrue%
\ \ \isacommand{apply}\isamarkupfalse%
\ {\isaliteral{28}{\isacharparenleft}}simp\ add{\isaliteral{3A}{\isacharcolon}}\ pt{\isadigit{1}}{\isaliteral{5F}{\isacharunderscore}}def{\isaliteral{29}{\isacharparenright}}\isanewline
\ \ \isacommand{done}\isamarkupfalse%
%
\endisatagproof
{\isafoldproof}%
%
\isadelimproof
%
\endisadelimproof
%
\begin{isamarkuptext}%
In the example below, a coloured point is truncated to leave a
  point.  We use the \isa{truncate} function of the target record.%
\end{isamarkuptext}%
\isamarkuptrue%
\isacommand{lemma}\isamarkupfalse%
\ {\isaliteral{22}{\isachardoublequoteopen}}point{\isaliteral{2E}{\isachardot}}truncate\ cpt{\isadigit{2}}\ {\isaliteral{3D}{\isacharequal}}\ pt{\isadigit{1}}{\isaliteral{22}{\isachardoublequoteclose}}\isanewline
%
\isadelimproof
\ \ %
\endisadelimproof
%
\isatagproof
\isacommand{by}\isamarkupfalse%
\ {\isaliteral{28}{\isacharparenleft}}simp\ add{\isaliteral{3A}{\isacharcolon}}\ pt{\isadigit{1}}{\isaliteral{5F}{\isacharunderscore}}def\ cpt{\isadigit{2}}{\isaliteral{5F}{\isacharunderscore}}def\ point{\isaliteral{2E}{\isachardot}}defs{\isaliteral{29}{\isacharparenright}}%
\endisatagproof
{\isafoldproof}%
%
\isadelimproof
%
\endisadelimproof
%
\begin{isamarkuptext}%
\begin{exercise}
  Extend record \isa{cpoint} to have a further field, \isa{intensity}, of type~\isa{nat}.  Experiment with generic operations
  (using polymorphic selectors and updates) and explicit coercions
  (using \isa{extend}, \isa{truncate} etc.) among the three record
  types.
  \end{exercise}

  \begin{exercise}
  (For Java programmers.)
  Model a small class hierarchy using records.
  \end{exercise}
  \index{records|)}%
\end{isamarkuptext}%
\isamarkuptrue%
%
\isadelimtheory
%
\endisadelimtheory
%
\isatagtheory
%
\endisatagtheory
{\isafoldtheory}%
%
\isadelimtheory
%
\endisadelimtheory
\end{isabellebody}%
%%% Local Variables:
%%% mode: latex
%%% TeX-master: "root"
%%% End:
  %%%Section "Records"


\section{Type Classes} %%%Section
\label{sec:axclass}
\index{axiomatic type classes|(}
\index{*axclass|(}

The programming language Haskell has popularized the notion of type
classes: a type class is a set of types with a
common interface: all types in that class must provide the functions
in the interface.  Isabelle offers a similar type class concept: in
addition, properties (\emph{class axioms}) can be specified which any
instance of this type class must obey.  Thus we can talk about a type
$\tau$ being in a class $C$, which is written $\tau :: C$.  This is the case
if $\tau$ satisfies the axioms of $C$. Furthermore, type classes can be
organized in a hierarchy.  Thus there is the notion of a class $D$
being a subclass\index{subclasses} of a class $C$, written $D
< C$. This is the case if all axioms of $C$ are also provable in $D$.

In this section we introduce the most important concepts behind type
classes by means of a running example from algebra.  This should give
you an intuition how to use type classes and to understand
specifications involving type classes.  Type classes are covered more
deeply in a separate tutorial \cite{isabelle-classes}.

\subsection{Overloading}
\label{sec:overloading}
\index{overloading|(}

%
\begin{isabellebody}%
\def\isabellecontext{Overloading}%
%
\isadelimtheory
%
\endisadelimtheory
%
\isatagtheory
%
\endisatagtheory
{\isafoldtheory}%
%
\isadelimtheory
%
\endisadelimtheory
%
\begin{isamarkuptext}%
Type classes allow \emph{overloading}; thus a constant may
have multiple definitions at non-overlapping types.%
\end{isamarkuptext}%
\isamarkuptrue%
%
\isamarkupsubsubsection{Overloading%
}
\isamarkuptrue%
%
\begin{isamarkuptext}%
We can introduce a binary infix addition operator \isa{{\isaliteral{5C3C6F74696D65733E}{\isasymotimes}}}
for arbitrary types by means of a type class:%
\end{isamarkuptext}%
\isamarkuptrue%
\isacommand{class}\isamarkupfalse%
\ plus\ {\isaliteral{3D}{\isacharequal}}\isanewline
\ \ \isakeyword{fixes}\ plus\ {\isaliteral{3A}{\isacharcolon}}{\isaliteral{3A}{\isacharcolon}}\ {\isaliteral{22}{\isachardoublequoteopen}}{\isaliteral{27}{\isacharprime}}a\ {\isaliteral{5C3C52696768746172726F773E}{\isasymRightarrow}}\ {\isaliteral{27}{\isacharprime}}a\ {\isaliteral{5C3C52696768746172726F773E}{\isasymRightarrow}}\ {\isaliteral{27}{\isacharprime}}a{\isaliteral{22}{\isachardoublequoteclose}}\ {\isaliteral{28}{\isacharparenleft}}\isakeyword{infixl}\ {\isaliteral{22}{\isachardoublequoteopen}}{\isaliteral{5C3C6F706C75733E}{\isasymoplus}}{\isaliteral{22}{\isachardoublequoteclose}}\ {\isadigit{7}}{\isadigit{0}}{\isaliteral{29}{\isacharparenright}}%
\begin{isamarkuptext}%
\noindent This introduces a new class \isa{plus},
along with a constant \isa{plus} with nice infix syntax.
\isa{plus} is also named \emph{class operation}.  The type
of \isa{plus} carries a class constraint \isa{{\isaliteral{22}{\isachardoublequote}}{\isaliteral{27}{\isacharprime}}a\ {\isaliteral{3A}{\isacharcolon}}{\isaliteral{3A}{\isacharcolon}}\ plus{\isaliteral{22}{\isachardoublequote}}} on its type variable, meaning that only types of class
\isa{plus} can be instantiated for \isa{{\isaliteral{22}{\isachardoublequote}}{\isaliteral{27}{\isacharprime}}a{\isaliteral{22}{\isachardoublequote}}}.
To breathe life into \isa{plus} we need to declare a type
to be an \bfindex{instance} of \isa{plus}:%
\end{isamarkuptext}%
\isamarkuptrue%
\isacommand{instantiation}\isamarkupfalse%
\ nat\ {\isaliteral{3A}{\isacharcolon}}{\isaliteral{3A}{\isacharcolon}}\ plus\isanewline
\isakeyword{begin}%
\begin{isamarkuptext}%
\noindent Command \isacommand{instantiation} opens a local
theory context.  Here we can now instantiate \isa{plus} on
\isa{nat}:%
\end{isamarkuptext}%
\isamarkuptrue%
\isacommand{primrec}\isamarkupfalse%
\ plus{\isaliteral{5F}{\isacharunderscore}}nat\ {\isaliteral{3A}{\isacharcolon}}{\isaliteral{3A}{\isacharcolon}}\ {\isaliteral{22}{\isachardoublequoteopen}}nat\ {\isaliteral{5C3C52696768746172726F773E}{\isasymRightarrow}}\ nat\ {\isaliteral{5C3C52696768746172726F773E}{\isasymRightarrow}}\ nat{\isaliteral{22}{\isachardoublequoteclose}}\ \isakeyword{where}\isanewline
\ \ \ \ {\isaliteral{22}{\isachardoublequoteopen}}{\isaliteral{28}{\isacharparenleft}}{\isadigit{0}}{\isaliteral{3A}{\isacharcolon}}{\isaliteral{3A}{\isacharcolon}}nat{\isaliteral{29}{\isacharparenright}}\ {\isaliteral{5C3C6F706C75733E}{\isasymoplus}}\ n\ {\isaliteral{3D}{\isacharequal}}\ n{\isaliteral{22}{\isachardoublequoteclose}}\isanewline
\ \ {\isaliteral{7C}{\isacharbar}}\ {\isaliteral{22}{\isachardoublequoteopen}}Suc\ m\ {\isaliteral{5C3C6F706C75733E}{\isasymoplus}}\ n\ {\isaliteral{3D}{\isacharequal}}\ Suc\ {\isaliteral{28}{\isacharparenleft}}m\ {\isaliteral{5C3C6F706C75733E}{\isasymoplus}}\ n{\isaliteral{29}{\isacharparenright}}{\isaliteral{22}{\isachardoublequoteclose}}%
\begin{isamarkuptext}%
\noindent Note that the name \isa{plus} carries a
suffix \isa{{\isaliteral{5F}{\isacharunderscore}}nat}; by default, the local name of a class operation
\isa{f} to be instantiated on type constructor \isa{{\isaliteral{5C3C6B617070613E}{\isasymkappa}}} is mangled
as \isa{f{\isaliteral{5F}{\isacharunderscore}}{\isaliteral{5C3C6B617070613E}{\isasymkappa}}}.  In case of uncertainty, these names may be inspected
using the \hyperlink{command.print-context}{\mbox{\isa{\isacommand{print{\isaliteral{5F}{\isacharunderscore}}context}}}} command or the corresponding
ProofGeneral button.

Although class \isa{plus} has no axioms, the instantiation must be
formally concluded by a (trivial) instantiation proof ``..'':%
\end{isamarkuptext}%
\isamarkuptrue%
\isacommand{instance}\isamarkupfalse%
%
\isadelimproof
\ %
\endisadelimproof
%
\isatagproof
\isacommand{{\isaliteral{2E}{\isachardot}}{\isaliteral{2E}{\isachardot}}}\isamarkupfalse%
%
\endisatagproof
{\isafoldproof}%
%
\isadelimproof
%
\endisadelimproof
%
\begin{isamarkuptext}%
\noindent More interesting \isacommand{instance} proofs will
arise below.

The instantiation is finished by an explicit%
\end{isamarkuptext}%
\isamarkuptrue%
\isacommand{end}\isamarkupfalse%
%
\begin{isamarkuptext}%
\noindent From now on, terms like \isa{Suc\ {\isaliteral{28}{\isacharparenleft}}m\ {\isaliteral{5C3C6F706C75733E}{\isasymoplus}}\ {\isadigit{2}}{\isaliteral{29}{\isacharparenright}}} are
legal.%
\end{isamarkuptext}%
\isamarkuptrue%
\isacommand{instantiation}\isamarkupfalse%
\ prod\ {\isaliteral{3A}{\isacharcolon}}{\isaliteral{3A}{\isacharcolon}}\ {\isaliteral{28}{\isacharparenleft}}plus{\isaliteral{2C}{\isacharcomma}}\ plus{\isaliteral{29}{\isacharparenright}}\ plus\isanewline
\isakeyword{begin}%
\begin{isamarkuptext}%
\noindent Here we instantiate the product type \isa{prod} to
class \isa{plus}, given that its type arguments are of
class \isa{plus}:%
\end{isamarkuptext}%
\isamarkuptrue%
\isacommand{fun}\isamarkupfalse%
\ plus{\isaliteral{5F}{\isacharunderscore}}prod\ {\isaliteral{3A}{\isacharcolon}}{\isaliteral{3A}{\isacharcolon}}\ {\isaliteral{22}{\isachardoublequoteopen}}{\isaliteral{27}{\isacharprime}}a\ {\isaliteral{5C3C74696D65733E}{\isasymtimes}}\ {\isaliteral{27}{\isacharprime}}b\ {\isaliteral{5C3C52696768746172726F773E}{\isasymRightarrow}}\ {\isaliteral{27}{\isacharprime}}a\ {\isaliteral{5C3C74696D65733E}{\isasymtimes}}\ {\isaliteral{27}{\isacharprime}}b\ {\isaliteral{5C3C52696768746172726F773E}{\isasymRightarrow}}\ {\isaliteral{27}{\isacharprime}}a\ {\isaliteral{5C3C74696D65733E}{\isasymtimes}}\ {\isaliteral{27}{\isacharprime}}b{\isaliteral{22}{\isachardoublequoteclose}}\ \isakeyword{where}\isanewline
\ \ {\isaliteral{22}{\isachardoublequoteopen}}{\isaliteral{28}{\isacharparenleft}}x{\isaliteral{2C}{\isacharcomma}}\ y{\isaliteral{29}{\isacharparenright}}\ {\isaliteral{5C3C6F706C75733E}{\isasymoplus}}\ {\isaliteral{28}{\isacharparenleft}}w{\isaliteral{2C}{\isacharcomma}}\ z{\isaliteral{29}{\isacharparenright}}\ {\isaliteral{3D}{\isacharequal}}\ {\isaliteral{28}{\isacharparenleft}}x\ {\isaliteral{5C3C6F706C75733E}{\isasymoplus}}\ w{\isaliteral{2C}{\isacharcomma}}\ y\ {\isaliteral{5C3C6F706C75733E}{\isasymoplus}}\ z{\isaliteral{29}{\isacharparenright}}{\isaliteral{22}{\isachardoublequoteclose}}%
\begin{isamarkuptext}%
\noindent Obviously, overloaded specifications may include
recursion over the syntactic structure of types.%
\end{isamarkuptext}%
\isamarkuptrue%
\isacommand{instance}\isamarkupfalse%
%
\isadelimproof
\ %
\endisadelimproof
%
\isatagproof
\isacommand{{\isaliteral{2E}{\isachardot}}{\isaliteral{2E}{\isachardot}}}\isamarkupfalse%
%
\endisatagproof
{\isafoldproof}%
%
\isadelimproof
%
\endisadelimproof
\isanewline
\isanewline
\isacommand{end}\isamarkupfalse%
%
\begin{isamarkuptext}%
\noindent This way we have encoded the canonical lifting of
binary operations to products by means of type classes.%
\end{isamarkuptext}%
\isamarkuptrue%
%
\isadelimtheory
%
\endisadelimtheory
%
\isatagtheory
%
\endisatagtheory
{\isafoldtheory}%
%
\isadelimtheory
%
\endisadelimtheory
\end{isabellebody}%
%%% Local Variables:
%%% mode: latex
%%% TeX-master: "root"
%%% End:


\index{overloading|)}

%
\begin{isabellebody}%
\def\isabellecontext{Axioms}%
%
\isadelimtheory
%
\endisadelimtheory
%
\isatagtheory
%
\endisatagtheory
{\isafoldtheory}%
%
\isadelimtheory
%
\endisadelimtheory
%
\isamarkupsubsection{Axioms%
}
\isamarkuptrue%
%
\begin{isamarkuptext}%
Attaching axioms to our classes lets us reason on the level of
classes.  The results will be applicable to all types in a class, just
as in axiomatic mathematics.

\begin{warn}
Proofs in this section use structured \emph{Isar} proofs, which are not
covered in this tutorial; but see \cite{Nipkow-TYPES02}.%
\end{warn}%
\end{isamarkuptext}%
\isamarkuptrue%
%
\isamarkupsubsubsection{Semigroups%
}
\isamarkuptrue%
%
\begin{isamarkuptext}%
We specify \emph{semigroups} as subclass of \isa{plus}:%
\end{isamarkuptext}%
\isamarkuptrue%
\isacommand{class}\isamarkupfalse%
\ semigroup\ {\isacharequal}\ plus\ {\isacharplus}\isanewline
\ \ \isakeyword{assumes}\ assoc{\isacharcolon}\ {\isachardoublequoteopen}{\isacharparenleft}x\ {\isasymoplus}\ y{\isacharparenright}\ {\isasymoplus}\ z\ {\isacharequal}\ x\ {\isasymoplus}\ {\isacharparenleft}y\ {\isasymoplus}\ z{\isacharparenright}{\isachardoublequoteclose}%
\begin{isamarkuptext}%
\noindent This \hyperlink{command.class}{\mbox{\isa{\isacommand{class}}}} specification requires that
all instances of \isa{semigroup} obey \hyperlink{fact.assoc:}{\mbox{\isa{assoc{\isacharcolon}}}}~\isa{{\isachardoublequote}{\isasymAnd}x\ y\ z\ {\isasymColon}\ {\isacharprime}a{\isasymColon}semigroup{\isachardot}\ {\isacharparenleft}x\ {\isasymoplus}\ y{\isacharparenright}\ {\isasymoplus}\ z\ {\isacharequal}\ x\ {\isasymoplus}\ {\isacharparenleft}y\ {\isasymoplus}\ z{\isacharparenright}{\isachardoublequote}}.

We can use this class axiom to derive further abstract theorems
relative to class \isa{semigroup}:%
\end{isamarkuptext}%
\isamarkuptrue%
\isacommand{lemma}\isamarkupfalse%
\ assoc{\isacharunderscore}left{\isacharcolon}\isanewline
\ \ \isakeyword{fixes}\ x\ y\ z\ {\isacharcolon}{\isacharcolon}\ {\isachardoublequoteopen}{\isacharprime}a{\isasymColon}semigroup{\isachardoublequoteclose}\isanewline
\ \ \isakeyword{shows}\ {\isachardoublequoteopen}x\ {\isasymoplus}\ {\isacharparenleft}y\ {\isasymoplus}\ z{\isacharparenright}\ {\isacharequal}\ {\isacharparenleft}x\ {\isasymoplus}\ y{\isacharparenright}\ {\isasymoplus}\ z{\isachardoublequoteclose}\isanewline
%
\isadelimproof
\ \ %
\endisadelimproof
%
\isatagproof
\isacommand{using}\isamarkupfalse%
\ assoc\ \isacommand{by}\isamarkupfalse%
\ {\isacharparenleft}rule\ sym{\isacharparenright}%
\endisatagproof
{\isafoldproof}%
%
\isadelimproof
%
\endisadelimproof
%
\begin{isamarkuptext}%
\noindent The \isa{semigroup} constraint on type \isa{{\isacharprime}a} restricts instantiations of \isa{{\isacharprime}a} to types of class
\isa{semigroup} and during the proof enables us to use the fact
\hyperlink{fact.assoc}{\mbox{\isa{assoc}}} whose type parameter is itself constrained to class
\isa{semigroup}.  The main advantage of classes is that theorems
can be proved in the abstract and freely reused for each instance.

On instantiation, we have to give a proof that the given operations
obey the class axioms:%
\end{isamarkuptext}%
\isamarkuptrue%
\isacommand{instantiation}\isamarkupfalse%
\ nat\ {\isacharcolon}{\isacharcolon}\ semigroup\isanewline
\isakeyword{begin}\isanewline
\isanewline
\isacommand{instance}\isamarkupfalse%
%
\isadelimproof
\ %
\endisadelimproof
%
\isatagproof
\isacommand{proof}\isamarkupfalse%
%
\begin{isamarkuptxt}%
\noindent The proof opens with a default proof step, which for
instance judgements invokes method \hyperlink{method.intro-classes}{\mbox{\isa{intro{\isacharunderscore}classes}}}.%
\end{isamarkuptxt}%
\isamarkuptrue%
\ \ \isacommand{fix}\isamarkupfalse%
\ m\ n\ q\ {\isacharcolon}{\isacharcolon}\ nat\isanewline
\ \ \isacommand{show}\isamarkupfalse%
\ {\isachardoublequoteopen}{\isacharparenleft}m\ {\isasymoplus}\ n{\isacharparenright}\ {\isasymoplus}\ q\ {\isacharequal}\ m\ {\isasymoplus}\ {\isacharparenleft}n\ {\isasymoplus}\ q{\isacharparenright}{\isachardoublequoteclose}\isanewline
\ \ \ \ \isacommand{by}\isamarkupfalse%
\ {\isacharparenleft}induct\ m{\isacharparenright}\ simp{\isacharunderscore}all\isanewline
\isacommand{qed}\isamarkupfalse%
%
\endisatagproof
{\isafoldproof}%
%
\isadelimproof
%
\endisadelimproof
\isanewline
\isanewline
\isacommand{end}\isamarkupfalse%
%
\begin{isamarkuptext}%
\noindent Again, the interesting things enter the stage with
parametric types:%
\end{isamarkuptext}%
\isamarkuptrue%
\isacommand{instantiation}\isamarkupfalse%
\ {\isacharasterisk}\ {\isacharcolon}{\isacharcolon}\ {\isacharparenleft}semigroup{\isacharcomma}\ semigroup{\isacharparenright}\ semigroup\isanewline
\isakeyword{begin}\isanewline
\isanewline
\isacommand{instance}\isamarkupfalse%
%
\isadelimproof
\ %
\endisadelimproof
%
\isatagproof
\isacommand{proof}\isamarkupfalse%
\isanewline
\ \ \isacommand{fix}\isamarkupfalse%
\ p\isactrlisub {\isadigit{1}}\ p\isactrlisub {\isadigit{2}}\ p\isactrlisub {\isadigit{3}}\ {\isacharcolon}{\isacharcolon}\ {\isachardoublequoteopen}{\isacharprime}a{\isasymColon}semigroup\ {\isasymtimes}\ {\isacharprime}b{\isasymColon}semigroup{\isachardoublequoteclose}\isanewline
\ \ \isacommand{show}\isamarkupfalse%
\ {\isachardoublequoteopen}p\isactrlisub {\isadigit{1}}\ {\isasymoplus}\ p\isactrlisub {\isadigit{2}}\ {\isasymoplus}\ p\isactrlisub {\isadigit{3}}\ {\isacharequal}\ p\isactrlisub {\isadigit{1}}\ {\isasymoplus}\ {\isacharparenleft}p\isactrlisub {\isadigit{2}}\ {\isasymoplus}\ p\isactrlisub {\isadigit{3}}{\isacharparenright}{\isachardoublequoteclose}\isanewline
\ \ \ \ \isacommand{by}\isamarkupfalse%
\ {\isacharparenleft}cases\ p\isactrlisub {\isadigit{1}}{\isacharcomma}\ cases\ p\isactrlisub {\isadigit{2}}{\isacharcomma}\ cases\ p\isactrlisub {\isadigit{3}}{\isacharparenright}\ {\isacharparenleft}simp\ add{\isacharcolon}\ assoc{\isacharparenright}%
\begin{isamarkuptxt}%
\noindent Associativity of product semigroups is established
using the hypothetical associativity \hyperlink{fact.assoc}{\mbox{\isa{assoc}}} of the type
components, which holds due to the \isa{semigroup} constraints
imposed on the type components by the \hyperlink{command.instance}{\mbox{\isa{\isacommand{instance}}}} proposition.
Indeed, this pattern often occurs with parametric types and type
classes.%
\end{isamarkuptxt}%
\isamarkuptrue%
\isacommand{qed}\isamarkupfalse%
%
\endisatagproof
{\isafoldproof}%
%
\isadelimproof
%
\endisadelimproof
\isanewline
\isanewline
\isacommand{end}\isamarkupfalse%
%
\isamarkupsubsubsection{Monoids%
}
\isamarkuptrue%
%
\begin{isamarkuptext}%
We define a subclass \isa{monoidl} (a semigroup with a
left-hand neutral) by extending \isa{semigroup} with one additional
parameter \isa{neutral} together with its property:%
\end{isamarkuptext}%
\isamarkuptrue%
\isacommand{class}\isamarkupfalse%
\ monoidl\ {\isacharequal}\ semigroup\ {\isacharplus}\isanewline
\ \ \isakeyword{fixes}\ neutral\ {\isacharcolon}{\isacharcolon}\ {\isachardoublequoteopen}{\isacharprime}a{\isachardoublequoteclose}\ {\isacharparenleft}{\isachardoublequoteopen}{\isasymzero}{\isachardoublequoteclose}{\isacharparenright}\isanewline
\ \ \isakeyword{assumes}\ neutl{\isacharcolon}\ {\isachardoublequoteopen}{\isasymzero}\ {\isasymoplus}\ x\ {\isacharequal}\ x{\isachardoublequoteclose}%
\begin{isamarkuptext}%
\noindent Again, we prove some instances, by providing
suitable parameter definitions and proofs for the additional
specifications.%
\end{isamarkuptext}%
\isamarkuptrue%
\isacommand{instantiation}\isamarkupfalse%
\ nat\ {\isacharcolon}{\isacharcolon}\ monoidl\isanewline
\isakeyword{begin}\isanewline
\isanewline
\isacommand{definition}\isamarkupfalse%
\isanewline
\ \ neutral{\isacharunderscore}nat{\isacharunderscore}def{\isacharcolon}\ {\isachardoublequoteopen}{\isasymzero}\ {\isacharequal}\ {\isacharparenleft}{\isadigit{0}}{\isasymColon}nat{\isacharparenright}{\isachardoublequoteclose}\isanewline
\isanewline
\isacommand{instance}\isamarkupfalse%
%
\isadelimproof
\ %
\endisadelimproof
%
\isatagproof
\isacommand{proof}\isamarkupfalse%
\isanewline
\ \ \isacommand{fix}\isamarkupfalse%
\ n\ {\isacharcolon}{\isacharcolon}\ nat\isanewline
\ \ \isacommand{show}\isamarkupfalse%
\ {\isachardoublequoteopen}{\isasymzero}\ {\isasymoplus}\ n\ {\isacharequal}\ n{\isachardoublequoteclose}\isanewline
\ \ \ \ \isacommand{unfolding}\isamarkupfalse%
\ neutral{\isacharunderscore}nat{\isacharunderscore}def\ \isacommand{by}\isamarkupfalse%
\ simp\isanewline
\isacommand{qed}\isamarkupfalse%
%
\endisatagproof
{\isafoldproof}%
%
\isadelimproof
%
\endisadelimproof
\isanewline
\isanewline
\isacommand{end}\isamarkupfalse%
%
\begin{isamarkuptext}%
\noindent In contrast to the examples above, we here have both
specification of class operations and a non-trivial instance proof.

This covers products as well:%
\end{isamarkuptext}%
\isamarkuptrue%
\isacommand{instantiation}\isamarkupfalse%
\ {\isacharasterisk}\ {\isacharcolon}{\isacharcolon}\ {\isacharparenleft}monoidl{\isacharcomma}\ monoidl{\isacharparenright}\ monoidl\isanewline
\isakeyword{begin}\isanewline
\isanewline
\isacommand{definition}\isamarkupfalse%
\isanewline
\ \ neutral{\isacharunderscore}prod{\isacharunderscore}def{\isacharcolon}\ {\isachardoublequoteopen}{\isasymzero}\ {\isacharequal}\ {\isacharparenleft}{\isasymzero}{\isacharcomma}\ {\isasymzero}{\isacharparenright}{\isachardoublequoteclose}\isanewline
\isanewline
\isacommand{instance}\isamarkupfalse%
%
\isadelimproof
\ %
\endisadelimproof
%
\isatagproof
\isacommand{proof}\isamarkupfalse%
\isanewline
\ \ \isacommand{fix}\isamarkupfalse%
\ p\ {\isacharcolon}{\isacharcolon}\ {\isachardoublequoteopen}{\isacharprime}a{\isasymColon}monoidl\ {\isasymtimes}\ {\isacharprime}b{\isasymColon}monoidl{\isachardoublequoteclose}\isanewline
\ \ \isacommand{show}\isamarkupfalse%
\ {\isachardoublequoteopen}{\isasymzero}\ {\isasymoplus}\ p\ {\isacharequal}\ p{\isachardoublequoteclose}\isanewline
\ \ \ \ \isacommand{by}\isamarkupfalse%
\ {\isacharparenleft}cases\ p{\isacharparenright}\ {\isacharparenleft}simp\ add{\isacharcolon}\ neutral{\isacharunderscore}prod{\isacharunderscore}def\ neutl{\isacharparenright}\isanewline
\isacommand{qed}\isamarkupfalse%
%
\endisatagproof
{\isafoldproof}%
%
\isadelimproof
%
\endisadelimproof
\isanewline
\isanewline
\isacommand{end}\isamarkupfalse%
%
\begin{isamarkuptext}%
\noindent Fully-fledged monoids are modelled by another
subclass which does not add new parameters but tightens the
specification:%
\end{isamarkuptext}%
\isamarkuptrue%
\isacommand{class}\isamarkupfalse%
\ monoid\ {\isacharequal}\ monoidl\ {\isacharplus}\isanewline
\ \ \isakeyword{assumes}\ neutr{\isacharcolon}\ {\isachardoublequoteopen}x\ {\isasymoplus}\ {\isasymzero}\ {\isacharequal}\ x{\isachardoublequoteclose}%
\begin{isamarkuptext}%
\noindent Corresponding instances for \isa{nat} and products
are left as an exercise to the reader.%
\end{isamarkuptext}%
\isamarkuptrue%
%
\isamarkupsubsubsection{Groups%
}
\isamarkuptrue%
%
\begin{isamarkuptext}%
\noindent To finish our small algebra example, we add a \isa{group} class:%
\end{isamarkuptext}%
\isamarkuptrue%
\isacommand{class}\isamarkupfalse%
\ group\ {\isacharequal}\ monoidl\ {\isacharplus}\isanewline
\ \ \isakeyword{fixes}\ inv\ {\isacharcolon}{\isacharcolon}\ {\isachardoublequoteopen}{\isacharprime}a\ {\isasymRightarrow}\ {\isacharprime}a{\isachardoublequoteclose}\ {\isacharparenleft}{\isachardoublequoteopen}{\isasymdiv}\ {\isacharunderscore}{\isachardoublequoteclose}\ {\isacharbrackleft}{\isadigit{8}}{\isadigit{1}}{\isacharbrackright}\ {\isadigit{8}}{\isadigit{0}}{\isacharparenright}\isanewline
\ \ \isakeyword{assumes}\ invl{\isacharcolon}\ {\isachardoublequoteopen}{\isasymdiv}\ x\ {\isasymoplus}\ x\ {\isacharequal}\ {\isasymzero}{\isachardoublequoteclose}%
\begin{isamarkuptext}%
\noindent We continue with a further example for abstract
proofs relative to type classes:%
\end{isamarkuptext}%
\isamarkuptrue%
\isacommand{lemma}\isamarkupfalse%
\ left{\isacharunderscore}cancel{\isacharcolon}\isanewline
\ \ \isakeyword{fixes}\ x\ y\ z\ {\isacharcolon}{\isacharcolon}\ {\isachardoublequoteopen}{\isacharprime}a{\isasymColon}group{\isachardoublequoteclose}\isanewline
\ \ \isakeyword{shows}\ {\isachardoublequoteopen}x\ {\isasymoplus}\ y\ {\isacharequal}\ x\ {\isasymoplus}\ z\ {\isasymlongleftrightarrow}\ y\ {\isacharequal}\ z{\isachardoublequoteclose}\isanewline
%
\isadelimproof
%
\endisadelimproof
%
\isatagproof
\isacommand{proof}\isamarkupfalse%
\isanewline
\ \ \isacommand{assume}\isamarkupfalse%
\ {\isachardoublequoteopen}x\ {\isasymoplus}\ y\ {\isacharequal}\ x\ {\isasymoplus}\ z{\isachardoublequoteclose}\isanewline
\ \ \isacommand{then}\isamarkupfalse%
\ \isacommand{have}\isamarkupfalse%
\ {\isachardoublequoteopen}{\isasymdiv}\ x\ {\isasymoplus}\ {\isacharparenleft}x\ {\isasymoplus}\ y{\isacharparenright}\ {\isacharequal}\ {\isasymdiv}\ x\ {\isasymoplus}\ {\isacharparenleft}x\ {\isasymoplus}\ z{\isacharparenright}{\isachardoublequoteclose}\ \isacommand{by}\isamarkupfalse%
\ simp\isanewline
\ \ \isacommand{then}\isamarkupfalse%
\ \isacommand{have}\isamarkupfalse%
\ {\isachardoublequoteopen}{\isacharparenleft}{\isasymdiv}\ x\ {\isasymoplus}\ x{\isacharparenright}\ {\isasymoplus}\ y\ {\isacharequal}\ {\isacharparenleft}{\isasymdiv}\ x\ {\isasymoplus}\ x{\isacharparenright}\ {\isasymoplus}\ z{\isachardoublequoteclose}\ \isacommand{by}\isamarkupfalse%
\ {\isacharparenleft}simp\ add{\isacharcolon}\ assoc{\isacharparenright}\isanewline
\ \ \isacommand{then}\isamarkupfalse%
\ \isacommand{show}\isamarkupfalse%
\ {\isachardoublequoteopen}y\ {\isacharequal}\ z{\isachardoublequoteclose}\ \isacommand{by}\isamarkupfalse%
\ {\isacharparenleft}simp\ add{\isacharcolon}\ invl\ neutl{\isacharparenright}\isanewline
\isacommand{next}\isamarkupfalse%
\isanewline
\ \ \isacommand{assume}\isamarkupfalse%
\ {\isachardoublequoteopen}y\ {\isacharequal}\ z{\isachardoublequoteclose}\isanewline
\ \ \isacommand{then}\isamarkupfalse%
\ \isacommand{show}\isamarkupfalse%
\ {\isachardoublequoteopen}x\ {\isasymoplus}\ y\ {\isacharequal}\ x\ {\isasymoplus}\ z{\isachardoublequoteclose}\ \isacommand{by}\isamarkupfalse%
\ simp\isanewline
\isacommand{qed}\isamarkupfalse%
%
\endisatagproof
{\isafoldproof}%
%
\isadelimproof
%
\endisadelimproof
%
\begin{isamarkuptext}%
\noindent Any \isa{group} is also a \isa{monoid}; this
can be made explicit by claiming an additional subclass relation,
together with a proof of the logical difference:%
\end{isamarkuptext}%
\isamarkuptrue%
\isacommand{instance}\isamarkupfalse%
\ group\ {\isasymsubseteq}\ monoid\isanewline
%
\isadelimproof
%
\endisadelimproof
%
\isatagproof
\isacommand{proof}\isamarkupfalse%
\isanewline
\ \ \isacommand{fix}\isamarkupfalse%
\ x\isanewline
\ \ \isacommand{from}\isamarkupfalse%
\ invl\ \isacommand{have}\isamarkupfalse%
\ {\isachardoublequoteopen}{\isasymdiv}\ x\ {\isasymoplus}\ x\ {\isacharequal}\ {\isasymzero}{\isachardoublequoteclose}\ \isacommand{{\isachardot}}\isamarkupfalse%
\isanewline
\ \ \isacommand{then}\isamarkupfalse%
\ \isacommand{have}\isamarkupfalse%
\ {\isachardoublequoteopen}{\isasymdiv}\ x\ {\isasymoplus}\ {\isacharparenleft}x\ {\isasymoplus}\ {\isasymzero}{\isacharparenright}\ {\isacharequal}\ {\isasymdiv}\ x\ {\isasymoplus}\ x{\isachardoublequoteclose}\isanewline
\ \ \ \ \isacommand{by}\isamarkupfalse%
\ {\isacharparenleft}simp\ add{\isacharcolon}\ neutl\ invl\ assoc\ {\isacharbrackleft}symmetric{\isacharbrackright}{\isacharparenright}\isanewline
\ \ \isacommand{then}\isamarkupfalse%
\ \isacommand{show}\isamarkupfalse%
\ {\isachardoublequoteopen}x\ {\isasymoplus}\ {\isasymzero}\ {\isacharequal}\ x{\isachardoublequoteclose}\ \isacommand{by}\isamarkupfalse%
\ {\isacharparenleft}simp\ add{\isacharcolon}\ left{\isacharunderscore}cancel{\isacharparenright}\isanewline
\isacommand{qed}\isamarkupfalse%
%
\endisatagproof
{\isafoldproof}%
%
\isadelimproof
%
\endisadelimproof
%
\begin{isamarkuptext}%
\noindent The proof result is propagated to the type system,
making \isa{group} an instance of \isa{monoid} by adding an
additional edge to the graph of subclass relation; see also
Figure~\ref{fig:subclass}.

\begin{figure}[htbp]
 \begin{center}
   \small
   \unitlength 0.6mm
   \begin{picture}(40,60)(0,0)
     \put(20,60){\makebox(0,0){\isa{semigroup}}}
     \put(20,40){\makebox(0,0){\isa{monoidl}}}
     \put(00,20){\makebox(0,0){\isa{monoid}}}
     \put(40,00){\makebox(0,0){\isa{group}}}
     \put(20,55){\vector(0,-1){10}}
     \put(15,35){\vector(-1,-1){10}}
     \put(25,35){\vector(1,-3){10}}
   \end{picture}
   \hspace{8em}
   \begin{picture}(40,60)(0,0)
     \put(20,60){\makebox(0,0){\isa{semigroup}}}
     \put(20,40){\makebox(0,0){\isa{monoidl}}}
     \put(00,20){\makebox(0,0){\isa{monoid}}}
     \put(40,00){\makebox(0,0){\isa{group}}}
     \put(20,55){\vector(0,-1){10}}
     \put(15,35){\vector(-1,-1){10}}
     \put(05,15){\vector(3,-1){30}}
   \end{picture}
   \caption{Subclass relationship of monoids and groups:
      before and after establishing the relationship
      \isa{group\ {\isasymsubseteq}\ monoid};  transitive edges are left out.}
   \label{fig:subclass}
 \end{center}
\end{figure}%
\end{isamarkuptext}%
\isamarkuptrue%
%
\isamarkupsubsubsection{Inconsistencies%
}
\isamarkuptrue%
%
\begin{isamarkuptext}%
The reader may be wondering what happens if we attach an
inconsistent set of axioms to a class. So far we have always avoided
to add new axioms to HOL for fear of inconsistencies and suddenly it
seems that we are throwing all caution to the wind. So why is there no
problem?

The point is that by construction, all type variables in the axioms of
a \isacommand{class} are automatically constrained with the class
being defined (as shown for axiom \isa{refl} above). These
constraints are always carried around and Isabelle takes care that
they are never lost, unless the type variable is instantiated with a
type that has been shown to belong to that class. Thus you may be able
to prove \isa{False} from your axioms, but Isabelle will remind you
that this theorem has the hidden hypothesis that the class is
non-empty.

Even if each individual class is consistent, intersections of
(unrelated) classes readily become inconsistent in practice. Now we
know this need not worry us.%
\end{isamarkuptext}%
\isamarkuptrue%
%
\isamarkupsubsubsection{Syntactic Classes and Predefined Overloading%
}
\isamarkuptrue%
%
\begin{isamarkuptext}%
In our algebra example, we have started with a \emph{syntactic
class} \isa{plus} which only specifies operations but no axioms; it
would have been also possible to start immediately with class \isa{semigroup}, specifying the \isa{{\isasymoplus}} operation and associativity at
the same time.

Which approach is more appropriate depends.  Usually it is more
convenient to introduce operations and axioms in the same class: then
the type checker will automatically insert the corresponding class
constraints whenever the operations occur, reducing the need of manual
annotations.  However, when operations are decorated with popular
syntax, syntactic classes can be an option to re-use the syntax in
different contexts; this is indeed the way most overloaded constants
in HOL are introduced, of which the most important are listed in
Table~\ref{tab:overloading} in the appendix.  Section
\ref{sec:numeric-classes} covers a range of corresponding classes
\emph{with} axioms.

Further note that classes may contain axioms but \emph{no} operations.
An example is class \isa{finite} from theory \hyperlink{theory.Finite-Set}{\mbox{\isa{Finite{\isacharunderscore}Set}}}
which specifies a type to be finite: \isa{{\isachardoublequote}finite\ {\isacharparenleft}UNIV\ {\isasymColon}\ {\isacharprime}a{\isasymColon}finite\ set{\isacharparenright}{\isachardoublequote}}.%
\end{isamarkuptext}%
\isamarkuptrue%
%
\isadelimtheory
%
\endisadelimtheory
%
\isatagtheory
%
\endisatagtheory
{\isafoldtheory}%
%
\isadelimtheory
%
\endisadelimtheory
\end{isabellebody}%
%%% Local Variables:
%%% mode: latex
%%% TeX-master: "root"
%%% End:


\index{type classes|)}
\index{*class|)}

Our examples until now have used the type of \textbf{natural numbers},
\isa{nat}.  This is a recursive datatype generated by the constructors
zero  and successor, so it works well with inductive proofs and primitive
recursive function definitions. Isabelle/HOL also has the type \isa{int}
of \textbf{integers}, which gives up induction in exchange  for proper subtraction.

The integers are preferable to the natural  numbers for reasoning about
complicated arithmetic expressions. For  example, a termination proof
typically involves an integer metric  that is shown to decrease at each
loop iteration. Even if the  metric cannot become negative, proofs about it
are usually easier  if the integers are used rather than the natural
numbers. 

The logic Isabelle/HOL-Real also has the type \isa{real} of real numbers
and even the type \isa{hypreal} of non-standard reals. These
\textbf{hyperreals} include  infinitesimals, which represent infinitely
small and infinitely  large quantities; they greatly facilitate proofs
about limits,  differentiation and integration.  Isabelle has no subtyping, 
so the numeric types are distinct and there are 
functions to convert between them. 

Many theorems involving numeric types can be proved automatically by
Isabelle's arithmetic decision procedure, the method
\isa{arith}.  Linear arithmetic comprises addition, subtraction
and multiplication by constant factors; subterms involving other operators
are regarded as variables.  The procedure can be slow, especially if the
subgoal to be proved involves subtraction over type \isa{nat}, which 
causes case splits.  

The simplifier reduces arithmetic expressions in other
ways, such as dividing through by common factors.  For problems that lie
outside the scope of automation, the library has hundreds of
theorems about multiplication, division, etc., that can be brought to
bear.  You can find find them by browsing the library.  Some
useful lemmas are shown below.

\subsection{Numeric Literals}
\label{sec:numerals}

Literals are available for the types of natural numbers, integers 
and reals and denote integer values of arbitrary size. 
\REMARK{hypreal?}
They begin 
with a number sign (\isa{\#}), have an optional minus sign (\isa{-}) and 
then one or more decimal digits. Examples are \isa{\#0}, \isa{\#-3} 
and \isa{\#441223334678}.

Literals look like constants, but they abbreviate 
terms, representing the number in a two's complement binary notation. 
Isabelle performs arithmetic on literals by rewriting, rather 
than using the hardware arithmetic. In most cases arithmetic 
is fast enough, even for large numbers. The arithmetic operations 
provided for literals are addition, subtraction, multiplication, 
integer division and remainder. 

\emph{Beware}: the arithmetic operators are 
overloaded, so you must be careful to ensure that each numeric 
expression refers to a specific type, if necessary by inserting 
type constraints.  Here is an example of what can go wrong:
\begin{isabelle}
\isacommand{lemma}\ "\#2\ *\ m\ =\ m\ +\ m"
\end{isabelle}
%
Carefully observe how Isabelle displays the subgoal:
\begin{isabelle}
\ 1.\ (\#2::'a)\ *\ m\ =\ m\ +\ m
\end{isabelle}
The type \isa{'a} given for the literal \isa{\#2} warns us that no numeric
type has been specified.  The problem is underspecified.  Given a type
constraint such as \isa{nat}, \isa{int} or \isa{real}, it becomes trivial.


\subsection{The type of natural numbers, {\tt\slshape nat}}

This type requires no introduction: we have been using it from the
start.  Hundreds of useful lemmas about arithmetic on type \isa{nat} are
proved in the theories \isa{Nat}, \isa{NatArith} and \isa{Divides}.  Only
in exceptional circumstances should you resort to induction.

\subsubsection{Literals}
The notational options for the natural numbers can be confusing. The 
constant \isa{0} is overloaded to serve as the neutral value 
in a variety of additive types. The symbols \isa{1} and \isa{2} are 
not constants but abbreviations for \isa{Suc 0} and \isa{Suc(Suc 0)},
respectively. The literals \isa{\#0}, \isa{\#1} and \isa{\#2}  are
entirely different from \isa{0}, \isa{1} and \isa{2}. You  will
sometimes prefer one notation to the other. Literals are  obviously
necessary to express large values, while \isa{0} and \isa{Suc}  are
needed in order to match many theorems, including the rewrite  rules for
primitive recursive functions. The following default  simplification rules
replace small literals by zero and successor: 
\begin{isabelle}
\#0\ =\ 0
\rulename{numeral_0_eq_0}\isanewline
\#1\ =\ 1
\rulename{numeral_1_eq_1}\isanewline
\#2\ +\ n\ =\ Suc\ (Suc\ n)
\rulename{add_2_eq_Suc}\isanewline
n\ +\ \#2\ =\ Suc\ (Suc\ n)
\rulename{add_2_eq_Suc'}
\end{isabelle}
In special circumstances, you may wish to remove or reorient 
these rules. 

\subsubsection{Typical lemmas}
Inequalities involving addition and subtraction alone can be proved
automatically.  Lemmas such as these can be used to prove inequalities
involving multiplication and division:
\begin{isabelle}
\isasymlbrakk i\ \isasymle \ j;\ k\ \isasymle \ l\isasymrbrakk \ \isasymLongrightarrow \ i\ *\ k\ \isasymle \ j\ *\ l%
\rulename{mult_le_mono}\isanewline
\isasymlbrakk i\ <\ j;\ 0\ <\ k\isasymrbrakk \ \isasymLongrightarrow \ i\
*\ k\ <\ j\ *\ k%
\rulename{mult_less_mono1}\isanewline
m\ \isasymle \ n\ \isasymLongrightarrow \ m\ div\ k\ \isasymle \ n\ div\ k%
\rulename{div_le_mono}
\end{isabelle}
%
Various distributive laws concerning multiplication are available:
\begin{isabelle}
(m\ +\ n)\ *\ k\ =\ m\ *\ k\ +\ n\ *\ k%
\rulename{add_mult_distrib}\isanewline
(m\ -\ n)\ *\ k\ =\ m\ *\ k\ -\ n\ *\ k%
\rulename{diff_mult_distrib}\isanewline
(m\ mod\ n)\ *\ k\ =\ (m\ *\ k)\ mod\ (n\ *\ k)
\rulename{mod_mult_distrib}
\end{isabelle}

\subsubsection{Division}
The library contains the basic facts about quotient and remainder
(including the analogous equation, \isa{div_if}):
\begin{isabelle}
m\ mod\ n\ =\ (if\ m\ <\ n\ then\ m\ else\ (m\ -\ n)\ mod\ n)
\rulename{mod_if}\isanewline
m\ div\ n\ *\ n\ +\ m\ mod\ n\ =\ m%
\rulename{mod_div_equality}
\end{isabelle}

Many less obvious facts about quotient and remainder are also provided. 
Here is a selection:
\begin{isabelle}
a\ *\ b\ div\ c\ =\ a\ *\ (b\ div\ c)\ +\ a\ *\ (b\ mod\ c)\ div\ c%
\rulename{div_mult1_eq}\isanewline
a\ *\ b\ mod\ c\ =\ a\ *\ (b\ mod\ c)\ mod\ c%
\rulename{mod_mult1_eq}\isanewline
a\ div\ (b*c)\ =\ a\ div\ b\ div\ c%
\rulename{div_mult2_eq}\isanewline
a\ mod\ (b*c)\ =\ b * (a\ div\ b\ mod\ c)\ +\ a\ mod\ b%
\rulename{mod_mult2_eq}\isanewline
0\ <\ c\ \isasymLongrightarrow \ (c\ *\ a)\ div\ (c\ *\ b)\ =\ a\ div\ b%
\rulename{div_mult_mult1}
\end{isabelle}

Surprisingly few of these results depend upon the
divisors' being nonzero.  Isabelle/HOL defines division by zero:
\begin{isabelle}
a\ div\ 0\ =\ 0
\rulename{DIVISION_BY_ZERO_DIV}\isanewline
a\ mod\ 0\ =\ a%
\rulename{DIVISION_BY_ZERO_MOD}
\end{isabelle}
As a concession to convention, these equations are not installed as default
simplification rules but are merely used to remove nonzero-divisor
hypotheses by case analysis.  In \isa{div_mult_mult1} above, one of
the two divisors (namely~\isa{c}) must be still be nonzero.

The \textbf{divides} relation has the standard definition, which
is overloaded over all numeric types: 
\begin{isabelle}
m\ dvd\ n\ \isasymequiv\ {\isasymexists}k.\ n\ =\ m\ *\ k
\rulename{dvd_def}
\end{isabelle}
%
Section~\ref{sec:proving-euclid} discusses proofs involving this
relation.  Here are some of the facts proved about it:
\begin{isabelle}
\isasymlbrakk m\ dvd\ n;\ n\ dvd\ m\isasymrbrakk \ \isasymLongrightarrow \ m\ =\ n%
\rulename{dvd_anti_sym}\isanewline
\isasymlbrakk k\ dvd\ m;\ k\ dvd\ n\isasymrbrakk \ \isasymLongrightarrow \ k\ dvd\ (m\ +\ n)
\rulename{dvd_add}
\end{isabelle}

\subsubsection{Simplifier tricks}
The rule \isa{diff_mult_distrib} shown above is one of the few facts
about \isa{m\ -\ n} that is not subject to
the condition \isa{n\ \isasymle \  m}.  Natural number subtraction has few
nice properties; often it is best to remove it from a subgoal
using this split rule:
\begin{isabelle}
P(a-b)\ =\ ((a<b\ \isasymlongrightarrow \ P\
0)\ \isasymand \ (\isasymforall d.\ a\ =\ b+d\ \isasymlongrightarrow \ P\
d))
\rulename{nat_diff_split}
\end{isabelle}
For example, it proves the following fact, which lies outside the scope of
linear arithmetic:
\begin{isabelle}
\isacommand{lemma}\ "(n-1)*(n+1)\ =\ n*n\ -\ 1"\isanewline
\isacommand{apply}\ (simp\ split:\ nat_diff_split)\isanewline
\isacommand{done}
\end{isabelle}

Suppose that two expressions are equal, differing only in 
associativity and commutativity of addition.  Simplifying with the
following equations sorts the terms and groups them to the right, making
the two expressions identical:
\begin{isabelle}
m\ +\ n\ +\ k\ =\ m\ +\ (n\ +\ k)
\rulename{add_assoc}\isanewline
m\ +\ n\ =\ n\ +\ m%
\rulename{add_commute}\isanewline
x\ +\ (y\ +\ z)\ =\ y\ +\ (x\
+\ z)
\rulename{add_left_commute}
\end{isabelle}
The name \isa{add_ac} refers to the list of all three theorems, similarly
there is \isa{mult_ac}.  Here is an example of the sorting effect.  Start
with this goal:
\begin{isabelle}
\ 1.\ Suc\ (i\ +\ j\ *\ l\ *\ k\ +\ m\ *\ n)\ =\
f\ (n\ *\ m\ +\ i\ +\ k\ *\ j\ *\ l)
\end{isabelle}
%
Simplify using  \isa{add_ac} and \isa{mult_ac}:
\begin{isabelle}
\isacommand{apply}\ (simp\ add:\ add_ac\ mult_ac)
\end{isabelle}
%
Here is the resulting subgoal:
\begin{isabelle}
\ 1.\ Suc\ (i\ +\ (m\ *\ n\ +\ j\ *\ (k\ *\ l)))\
=\ f\ (i\ +\ (m\ *\ n\ +\ j\ *\ (k\ *\ l)))%
\end{isabelle}


\subsection{The type of integers, {\tt\slshape int}}

Reasoning methods resemble those for the natural numbers, but
induction and the constant \isa{Suc} are not available.

Concerning simplifier tricks, we have no need to eliminate subtraction (it
is well-behaved), but the simplifier can sort the operands of integer
operators.  The name \isa{zadd_ac} refers to the associativity and
commutativity theorems for integer addition, while \isa{zmult_ac} has the
analogous theorems for multiplication.  The prefix~\isa{z} in many theorem
names recalls the use of $\Bbb{Z}$ to denote the set of integers.

For division and remainder, the treatment of negative divisors follows
traditional mathematical practice: the sign of the remainder follows that
of the divisor:
\begin{isabelle}
\#0\ <\ b\ \isasymLongrightarrow \ \#0\ \isasymle \ a\ mod\ b%
\rulename{pos_mod_sign}\isanewline
\#0\ <\ b\ \isasymLongrightarrow \ a\ mod\ b\ <\ b%
\rulename{pos_mod_bound}\isanewline
b\ <\ \#0\ \isasymLongrightarrow \ a\ mod\ b\ \isasymle \ \#0
\rulename{neg_mod_sign}\isanewline
b\ <\ \#0\ \isasymLongrightarrow \ b\ <\ a\ mod\ b%
\rulename{neg_mod_bound}
\end{isabelle}
ML treats negative divisors in the same way, but most computer hardware
treats signed operands using the same rules as for multiplication.

The library provides many lemmas for proving inequalities involving integer
multiplication and division, similar to those shown above for
type~\isa{nat}.  The absolute value function \isa{abs} is
defined for the integers; we have for example the obvious law
\begin{isabelle}
\isasymbar x\ *\ y\isasymbar \ =\ \isasymbar x\isasymbar \ *\ \isasymbar y\isasymbar 
\rulename{abs_mult}
\end{isabelle}

Again, many facts about quotients and remainders are provided:
\begin{isabelle}
(a\ +\ b)\ div\ c\ =\isanewline
a\ div\ c\ +\ b\ div\ c\ +\ (a\ mod\ c\ +\ b\ mod\ c)\ div\ c%
\rulename{zdiv_zadd1_eq}
\par\smallskip
(a\ +\ b)\ mod\ c\ =\ (a\ mod\ c\ +\ b\ mod\ c)\ mod\ c%
\rulename{zmod_zadd1_eq}
\end{isabelle}

\begin{isabelle}
(a\ *\ b)\ div\ c\ =\ a\ *\ (b\ div\ c)\ +\ a\ *\ (b\ mod\ c)\ div\ c%
\rulename{zdiv_zmult1_eq}\isanewline
(a\ *\ b)\ mod\ c\ =\ a\ *\ (b\ mod\ c)\ mod\ c%
\rulename{zmod_zmult1_eq}
\end{isabelle}

\begin{isabelle}
\#0\ <\ c\ \isasymLongrightarrow \ a\ div\ (b*c)\ =\ a\ div\ b\ div\ c%
\rulename{zdiv_zmult2_eq}\isanewline
\#0\ <\ c\ \isasymLongrightarrow \ a\ mod\ (b*c)\ =\ b*(a\ div\ b\ mod\
c)\ +\ a\ mod\ b%
\rulename{zmod_zmult2_eq}
\end{isabelle}
The last two differ from their natural number analogues by requiring
\isa{c} to be positive.  Since division by zero yields zero, we could allow
\isa{c} to be zero.  However, \isa{c} cannot be negative: a counterexample
is
$\isa{a} = 7$, $\isa{b} = 2$ and $\isa{c} = -3$, when the left-hand side of
\isa{zdiv_zmult2_eq} is $-2$ while the right-hand side is $-1$.


\subsection{The type of real numbers, {\tt\slshape real}}

As with the other numeric types, the simplifier can sort the operands of
addition and multiplication.  The name \isa{real_add_ac} refers to the
associativity and commutativity theorems for addition; similarly
\isa{real_mult_ac} contains those properties for multiplication. 

\textbf{To be written.  Inverse, abs, theorems about density, etc.?}
             %%%Section "Numbers"

%
\begin{isabellebody}%
\def\isabellecontext{Typedefs}%
\isamarkupfalse%
%
\isamarkupsection{Introducing New Types%
}
\isamarkuptrue%
%
\begin{isamarkuptext}%
\label{sec:adv-typedef}
For most applications, a combination of predefined types like \isa{bool} and
\isa{{\isasymRightarrow}} with recursive datatypes and records is quite sufficient. Very
occasionally you may feel the need for a more advanced type.  If you
are certain that your type is not definable by any of the
standard means, then read on.
\begin{warn}
  Types in HOL must be non-empty; otherwise the quantifier rules would be
  unsound, because $\exists x.\ x=x$ is a theorem.
\end{warn}%
\end{isamarkuptext}%
\isamarkuptrue%
%
\isamarkupsubsection{Declaring New Types%
}
\isamarkuptrue%
%
\begin{isamarkuptext}%
\label{sec:typedecl}
\index{types!declaring|(}%
\index{typedecl@\isacommand {typedecl} (command)}%
The most trivial way of introducing a new type is by a \textbf{type
declaration}:%
\end{isamarkuptext}%
\isamarkuptrue%
\isacommand{typedecl}\ my{\isacharunderscore}new{\isacharunderscore}type\isamarkupfalse%
%
\begin{isamarkuptext}%
\noindent
This does not define \isa{my{\isacharunderscore}new{\isacharunderscore}type} at all but merely introduces its
name. Thus we know nothing about this type, except that it is
non-empty. Such declarations without definitions are
useful if that type can be viewed as a parameter of the theory.
A typical example is given in \S\ref{sec:VMC}, where we define a transition
relation over an arbitrary type of states.

In principle we can always get rid of such type declarations by making those
types parameters of every other type, thus keeping the theory generic. In
practice, however, the resulting clutter can make types hard to read.

If you are looking for a quick and dirty way of introducing a new type
together with its properties: declare the type and state its properties as
axioms. Example:%
\end{isamarkuptext}%
\isamarkuptrue%
\isacommand{axioms}\isanewline
just{\isacharunderscore}one{\isacharcolon}\ {\isachardoublequote}{\isasymexists}x{\isacharcolon}{\isacharcolon}my{\isacharunderscore}new{\isacharunderscore}type{\isachardot}\ {\isasymforall}y{\isachardot}\ x\ {\isacharequal}\ y{\isachardoublequote}\isamarkupfalse%
%
\begin{isamarkuptext}%
\noindent
However, we strongly discourage this approach, except at explorative stages
of your development. It is extremely easy to write down contradictory sets of
axioms, in which case you will be able to prove everything but it will mean
nothing.  In the example above, the axiomatic approach is
unnecessary: a one-element type called \isa{unit} is already defined in HOL.
\index{types!declaring|)}%
\end{isamarkuptext}%
\isamarkuptrue%
%
\isamarkupsubsection{Defining New Types%
}
\isamarkuptrue%
%
\begin{isamarkuptext}%
\label{sec:typedef}
\index{types!defining|(}%
\index{typedecl@\isacommand {typedef} (command)|(}%
Now we come to the most general means of safely introducing a new type, the
\textbf{type definition}. All other means, for example
\isacommand{datatype}, are based on it. The principle is extremely simple:
any non-empty subset of an existing type can be turned into a new type.  Thus
a type definition is merely a notational device: you introduce a new name for
a subset of an existing type. This does not add any logical power to HOL,
because you could base all your work directly on the subset of the existing
type. However, the resulting theories could easily become indigestible
because instead of implicit types you would have explicit sets in your
formulae.

Let us work a simple example, the definition of a three-element type.
It is easily represented by the first three natural numbers:%
\end{isamarkuptext}%
\isamarkuptrue%
\isacommand{typedef}\ three\ {\isacharequal}\ {\isachardoublequote}{\isacharbraceleft}n{\isacharcolon}{\isacharcolon}nat{\isachardot}\ n\ {\isasymle}\ {\isadigit{2}}{\isacharbraceright}{\isachardoublequote}\isamarkupfalse%
%
\begin{isamarkuptxt}%
\noindent
In order to enforce that the representing set on the right-hand side is
non-empty, this definition actually starts a proof to that effect:
\begin{isabelle}%
\ {\isadigit{1}}{\isachardot}\ {\isasymexists}x{\isachardot}\ x\ {\isasymin}\ {\isacharbraceleft}n{\isachardot}\ n\ {\isasymle}\ {\isadigit{2}}{\isacharbraceright}%
\end{isabelle}
Fortunately, this is easy enough to show: take 0 as a witness.%
\end{isamarkuptxt}%
\isamarkuptrue%
\isacommand{apply}{\isacharparenleft}rule{\isacharunderscore}tac\ x\ {\isacharequal}\ {\isadigit{0}}\ \isakeyword{in}\ exI{\isacharparenright}\isanewline
\isamarkupfalse%
\isacommand{by}\ simp\isamarkupfalse%
%
\begin{isamarkuptext}%
This type definition introduces the new type \isa{three} and asserts
that it is a copy of the set \isa{{\isacharbraceleft}{\isadigit{0}}{\isasymColon}{\isacharprime}a{\isacharcomma}\ {\isadigit{1}}{\isasymColon}{\isacharprime}a{\isacharcomma}\ {\isadigit{2}}{\isasymColon}{\isacharprime}a{\isacharbraceright}}. This assertion
is expressed via a bijection between the \emph{type} \isa{three} and the
\emph{set} \isa{{\isacharbraceleft}{\isadigit{0}}{\isasymColon}{\isacharprime}a{\isacharcomma}\ {\isadigit{1}}{\isasymColon}{\isacharprime}a{\isacharcomma}\ {\isadigit{2}}{\isasymColon}{\isacharprime}a{\isacharbraceright}}. To this end, the command declares the following
constants behind the scenes:
\begin{center}
\begin{tabular}{rcl}
\isa{three} &::& \isa{nat\ set} \\
\isa{Rep{\isacharunderscore}three} &::& \isa{three\ {\isasymRightarrow}\ nat}\\
\isa{Abs{\isacharunderscore}three} &::& \isa{nat\ {\isasymRightarrow}\ three}
\end{tabular}
\end{center}
where constant \isa{three} is explicitly defined as the representing set:
\begin{center}
\isa{three\ {\isasymequiv}\ {\isacharbraceleft}n{\isachardot}\ n\ {\isasymle}\ {\isadigit{2}}{\isacharbraceright}}\hfill(\isa{three{\isacharunderscore}def})
\end{center}
The situation is best summarized with the help of the following diagram,
where squares are types and circles are sets:
\begin{center}
\unitlength1mm
\thicklines
\begin{picture}(100,40)
\put(3,13){\framebox(15,15){\isa{three}}}
\put(55,5){\framebox(30,30){\isa{three}}}
\put(70,32){\makebox(0,0){\isa{nat}}}
\put(70,20){\circle{40}}
\put(10,15){\vector(1,0){60}}
\put(25,14){\makebox(0,0)[tl]{\isa{Rep{\isacharunderscore}three}}}
\put(70,25){\vector(-1,0){60}}
\put(25,26){\makebox(0,0)[bl]{\isa{Abs{\isacharunderscore}three}}}
\end{picture}
\end{center}
Finally, \isacommand{typedef} asserts that \isa{Rep{\isacharunderscore}three} is
surjective on the subset \isa{three} and \isa{Abs{\isacharunderscore}three} and \isa{Rep{\isacharunderscore}three} are inverses of each other:
\begin{center}
\begin{tabular}{@ {}r@ {\qquad\qquad}l@ {}}
\isa{Rep{\isacharunderscore}three\ x\ {\isasymin}\ three} & (\isa{Rep{\isacharunderscore}three}) \\
\isa{Abs{\isacharunderscore}three\ {\isacharparenleft}Rep{\isacharunderscore}three\ x{\isacharparenright}\ {\isacharequal}\ x} & (\isa{Rep{\isacharunderscore}three{\isacharunderscore}inverse}) \\
\isa{y\ {\isasymin}\ three\ {\isasymLongrightarrow}\ Rep{\isacharunderscore}three\ {\isacharparenleft}Abs{\isacharunderscore}three\ y{\isacharparenright}\ {\isacharequal}\ y} & (\isa{Abs{\isacharunderscore}three{\isacharunderscore}inverse})
\end{tabular}
\end{center}
%
From this example it should be clear what \isacommand{typedef} does
in general given a name (here \isa{three}) and a set
(here \isa{{\isacharbraceleft}n{\isachardot}\ n\ {\isasymle}\ {\isacharparenleft}{\isadigit{2}}{\isasymColon}{\isacharprime}a{\isacharparenright}{\isacharbraceright}}).

Our next step is to define the basic functions expected on the new type.
Although this depends on the type at hand, the following strategy works well:
\begin{itemize}
\item define a small kernel of basic functions that can express all other
functions you anticipate.
\item define the kernel in terms of corresponding functions on the
representing type using \isa{Abs} and \isa{Rep} to convert between the
two levels.
\end{itemize}
In our example it suffices to give the three elements of type \isa{three}
names:%
\end{isamarkuptext}%
\isamarkuptrue%
\isacommand{constdefs}\isanewline
\ \ A{\isacharcolon}{\isacharcolon}\ three\isanewline
\ {\isachardoublequote}A\ {\isasymequiv}\ Abs{\isacharunderscore}three\ {\isadigit{0}}{\isachardoublequote}\isanewline
\ \ B{\isacharcolon}{\isacharcolon}\ three\isanewline
\ {\isachardoublequote}B\ {\isasymequiv}\ Abs{\isacharunderscore}three\ {\isadigit{1}}{\isachardoublequote}\isanewline
\ \ C\ {\isacharcolon}{\isacharcolon}\ three\isanewline
\ {\isachardoublequote}C\ {\isasymequiv}\ Abs{\isacharunderscore}three\ {\isadigit{2}}{\isachardoublequote}\isamarkupfalse%
%
\begin{isamarkuptext}%
So far, everything was easy. But it is clear that reasoning about \isa{three} will be hell if we have to go back to \isa{nat} every time. Thus our
aim must be to raise our level of abstraction by deriving enough theorems
about type \isa{three} to characterize it completely. And those theorems
should be phrased in terms of \isa{A}, \isa{B} and \isa{C}, not \isa{Abs{\isacharunderscore}three} and \isa{Rep{\isacharunderscore}three}. Because of the simplicity of the example,
we merely need to prove that \isa{A}, \isa{B} and \isa{C} are distinct
and that they exhaust the type.

In processing our \isacommand{typedef} declaration, 
Isabelle helpfully proves several lemmas.
One, \isa{Abs{\isacharunderscore}three{\isacharunderscore}inject},
expresses that \isa{Abs{\isacharunderscore}three} is injective on the representing subset:
\begin{center}
\isa{{\isasymlbrakk}x\ {\isasymin}\ three{\isacharsemicolon}\ y\ {\isasymin}\ three{\isasymrbrakk}\ {\isasymLongrightarrow}\ {\isacharparenleft}Abs{\isacharunderscore}three\ x\ {\isacharequal}\ Abs{\isacharunderscore}three\ y{\isacharparenright}\ {\isacharequal}\ {\isacharparenleft}x\ {\isacharequal}\ y{\isacharparenright}}
\end{center}
Another, \isa{Rep{\isacharunderscore}three{\isacharunderscore}inject}, expresses that the representation
function is also injective:
\begin{center}
\isa{{\isacharparenleft}Rep{\isacharunderscore}three\ x\ {\isacharequal}\ Rep{\isacharunderscore}three\ y{\isacharparenright}\ {\isacharequal}\ {\isacharparenleft}x\ {\isacharequal}\ y{\isacharparenright}}
\end{center}
Distinctness of \isa{A}, \isa{B} and \isa{C} follows immediately
if we expand their definitions and rewrite with the injectivity
of \isa{Abs{\isacharunderscore}three}:%
\end{isamarkuptext}%
\isamarkuptrue%
\isacommand{lemma}\ {\isachardoublequote}A\ {\isasymnoteq}\ B\ {\isasymand}\ B\ {\isasymnoteq}\ A\ {\isasymand}\ A\ {\isasymnoteq}\ C\ {\isasymand}\ C\ {\isasymnoteq}\ A\ {\isasymand}\ B\ {\isasymnoteq}\ C\ {\isasymand}\ C\ {\isasymnoteq}\ B{\isachardoublequote}\isanewline
\isamarkupfalse%
\isacommand{by}{\isacharparenleft}simp\ add{\isacharcolon}\ Abs{\isacharunderscore}three{\isacharunderscore}inject\ A{\isacharunderscore}def\ B{\isacharunderscore}def\ C{\isacharunderscore}def\ three{\isacharunderscore}def{\isacharparenright}\isamarkupfalse%
%
\begin{isamarkuptext}%
\noindent
Of course we rely on the simplifier to solve goals like \isa{{\isacharparenleft}{\isadigit{0}}{\isasymColon}{\isacharprime}a{\isacharparenright}\ {\isasymnoteq}\ {\isacharparenleft}{\isadigit{1}}{\isasymColon}{\isacharprime}a{\isacharparenright}}.

The fact that \isa{A}, \isa{B} and \isa{C} exhaust type \isa{three} is
best phrased as a case distinction theorem: if you want to prove \isa{P\ x}
(where \isa{x} is of type \isa{three}) it suffices to prove \isa{P\ A},
\isa{P\ B} and \isa{P\ C}. First we prove the analogous proposition for the
representation:%
\end{isamarkuptext}%
\isamarkuptrue%
\isacommand{lemma}\ cases{\isacharunderscore}lemma{\isacharcolon}\ {\isachardoublequote}{\isasymlbrakk}\ Q\ {\isadigit{0}}{\isacharsemicolon}\ Q\ {\isadigit{1}}{\isacharsemicolon}\ Q\ {\isadigit{2}}{\isacharsemicolon}\ n\ {\isasymin}\ three\ {\isasymrbrakk}\ {\isasymLongrightarrow}\ \ Q\ n{\isachardoublequote}\isamarkupfalse%
%
\begin{isamarkuptxt}%
\noindent
Expanding \isa{three{\isacharunderscore}def} yields the premise \isa{n\ {\isasymle}\ {\isadigit{2}}}. Repeated
elimination with \isa{le{\isacharunderscore}SucE}
\begin{isabelle}%
{\isasymlbrakk}{\isacharquery}m\ {\isasymle}\ Suc\ {\isacharquery}n{\isacharsemicolon}\ {\isacharquery}m\ {\isasymle}\ {\isacharquery}n\ {\isasymLongrightarrow}\ {\isacharquery}R{\isacharsemicolon}\ {\isacharquery}m\ {\isacharequal}\ Suc\ {\isacharquery}n\ {\isasymLongrightarrow}\ {\isacharquery}R{\isasymrbrakk}\ {\isasymLongrightarrow}\ {\isacharquery}R%
\end{isabelle}
reduces \isa{n\ {\isasymle}\ {\isadigit{2}}} to the three cases \isa{n\ {\isasymle}\ {\isadigit{0}}}, \isa{n\ {\isacharequal}\ {\isadigit{1}}} and
\isa{n\ {\isacharequal}\ {\isadigit{2}}} which are trivial for simplification:%
\end{isamarkuptxt}%
\isamarkuptrue%
\isacommand{apply}{\isacharparenleft}simp\ add{\isacharcolon}\ three{\isacharunderscore}def\ numerals{\isacharparenright}\ \ \ \isanewline
\isamarkupfalse%
\isacommand{apply}{\isacharparenleft}{\isacharparenleft}erule\ le{\isacharunderscore}SucE{\isacharparenright}{\isacharplus}{\isacharparenright}\isanewline
\isamarkupfalse%
\isacommand{apply}\ simp{\isacharunderscore}all\isanewline
\isamarkupfalse%
\isacommand{done}\isamarkupfalse%
%
\begin{isamarkuptext}%
Now the case distinction lemma on type \isa{three} is easy to derive if you 
know how:%
\end{isamarkuptext}%
\isamarkuptrue%
\isacommand{lemma}\ three{\isacharunderscore}cases{\isacharcolon}\ {\isachardoublequote}{\isasymlbrakk}\ P\ A{\isacharsemicolon}\ P\ B{\isacharsemicolon}\ P\ C\ {\isasymrbrakk}\ {\isasymLongrightarrow}\ P\ x{\isachardoublequote}\isamarkupfalse%
%
\begin{isamarkuptxt}%
\noindent
We start by replacing the \isa{x} by \isa{Abs{\isacharunderscore}three\ {\isacharparenleft}Rep{\isacharunderscore}three\ x{\isacharparenright}}:%
\end{isamarkuptxt}%
\isamarkuptrue%
\isacommand{apply}{\isacharparenleft}rule\ subst{\isacharbrackleft}OF\ Rep{\isacharunderscore}three{\isacharunderscore}inverse{\isacharbrackright}{\isacharparenright}\isamarkupfalse%
%
\begin{isamarkuptxt}%
\noindent
This substitution step worked nicely because there was just a single
occurrence of a term of type \isa{three}, namely \isa{x}.
When we now apply \isa{cases{\isacharunderscore}lemma}, \isa{Q} becomes \isa{{\isasymlambda}n{\isachardot}\ P\ {\isacharparenleft}Abs{\isacharunderscore}three\ n{\isacharparenright}} because \isa{Rep{\isacharunderscore}three\ x} is the only term of type \isa{nat}:%
\end{isamarkuptxt}%
\isamarkuptrue%
\isacommand{apply}{\isacharparenleft}rule\ cases{\isacharunderscore}lemma{\isacharparenright}\isamarkupfalse%
%
\begin{isamarkuptxt}%
\begin{isabelle}%
\ {\isadigit{1}}{\isachardot}\ {\isasymlbrakk}P\ A{\isacharsemicolon}\ P\ B{\isacharsemicolon}\ P\ C{\isasymrbrakk}\ {\isasymLongrightarrow}\ P\ {\isacharparenleft}Abs{\isacharunderscore}three\ {\isadigit{0}}{\isacharparenright}\isanewline
\ {\isadigit{2}}{\isachardot}\ {\isasymlbrakk}P\ A{\isacharsemicolon}\ P\ B{\isacharsemicolon}\ P\ C{\isasymrbrakk}\ {\isasymLongrightarrow}\ P\ {\isacharparenleft}Abs{\isacharunderscore}three\ {\isadigit{1}}{\isacharparenright}\isanewline
\ {\isadigit{3}}{\isachardot}\ {\isasymlbrakk}P\ A{\isacharsemicolon}\ P\ B{\isacharsemicolon}\ P\ C{\isasymrbrakk}\ {\isasymLongrightarrow}\ P\ {\isacharparenleft}Abs{\isacharunderscore}three\ {\isadigit{2}}{\isacharparenright}\isanewline
\ {\isadigit{4}}{\isachardot}\ {\isasymlbrakk}P\ A{\isacharsemicolon}\ P\ B{\isacharsemicolon}\ P\ C{\isasymrbrakk}\ {\isasymLongrightarrow}\ Rep{\isacharunderscore}three\ x\ {\isasymin}\ three%
\end{isabelle}
The resulting subgoals are easily solved by simplification:%
\end{isamarkuptxt}%
\isamarkuptrue%
\isacommand{apply}{\isacharparenleft}simp{\isacharunderscore}all\ add{\isacharcolon}A{\isacharunderscore}def\ B{\isacharunderscore}def\ C{\isacharunderscore}def\ Rep{\isacharunderscore}three{\isacharparenright}\isanewline
\isamarkupfalse%
\isacommand{done}\isamarkupfalse%
%
\begin{isamarkuptext}%
\noindent
This concludes the derivation of the characteristic theorems for
type \isa{three}.

The attentive reader has realized long ago that the
above lengthy definition can be collapsed into one line:%
\end{isamarkuptext}%
\isamarkuptrue%
\isacommand{datatype}\ three{\isacharprime}\ {\isacharequal}\ A\ {\isacharbar}\ B\ {\isacharbar}\ C\isamarkupfalse%
%
\begin{isamarkuptext}%
\noindent
In fact, the \isacommand{datatype} command performs internally more or less
the same derivations as we did, which gives you some idea what life would be
like without \isacommand{datatype}.

Although \isa{three} could be defined in one line, we have chosen this
example to demonstrate \isacommand{typedef} because its simplicity makes the
key concepts particularly easy to grasp. If you would like to see a
non-trivial example that cannot be defined more directly, we recommend the
definition of \emph{finite multisets} in the Library~\cite{isabelle-HOL-lib}.

Let us conclude by summarizing the above procedure for defining a new type.
Given some abstract axiomatic description $P$ of a type $ty$ in terms of a
set of functions $F$, this involves three steps:
\begin{enumerate}
\item Find an appropriate type $\tau$ and subset $A$ which has the desired
  properties $P$, and make a type definition based on this representation.
\item Define the required functions $F$ on $ty$ by lifting
analogous functions on the representation via $Abs_ty$ and $Rep_ty$.
\item Prove that $P$ holds for $ty$ by lifting $P$ from the representation.
\end{enumerate}
You can now forget about the representation and work solely in terms of the
abstract functions $F$ and properties $P$.%
\index{typedecl@\isacommand {typedef} (command)|)}%
\index{types!defining|)}%
\end{isamarkuptext}%
\isamarkuptrue%
\isamarkupfalse%
\end{isabellebody}%
%%% Local Variables:
%%% mode: latex
%%% TeX-master: "root"
%%% End:
    %%%Section "Introducing New Types"
