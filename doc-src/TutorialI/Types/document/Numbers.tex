%
\begin{isabelle}
\def\isabellecontext{Numbers}
\isanewline
\isacommand{theory}\ Numbers\ =\ Real:\isanewline
\isanewline
\isacommand{ML}\ "Pretty.setmargin\ 64"\isanewline
\isacommand{ML}\ "IsarOutput.indent\ :=\ 0"
\begin{isamarkuptext}
numeric literals; default simprules; can re-orient%
\end{isamarkuptext}
\isacommand{lemma}\ "\#2\ *\ m\ =\ m\ +\ m"
\begin{isamarkuptxt}
\begin{isabelle}
\ 1.\ (\#2::'a)\ *\ m\ =\ m\ +\ m%
\end{isabelle}
\end{isamarkuptxt}
\isacommand{oops}\isanewline
\isanewline
\isacommand{consts}\ h\ ::\ "nat\ \isasymRightarrow \ nat"\isanewline
\isacommand{recdef}\ h\ "\isacharbraceleft \isacharbraceright "\isanewline
"h\ i\ =\ (if\ i\ =\ \#3\ then\ \#2\ else\ i)"
\begin{isamarkuptext}
\isa{h\ \#3\ =\ \#2}
\isa{h\ i\ =\ i}
\end{isamarkuptext}
%
\begin{isamarkuptext}
\begin{isabelle}
\#0\ =\ 0
\rulename{numeral_0_eq_0}
\end{isabelle}

\begin{isabelle}
\#1\ =\ 1
\rulename{numeral_1_eq_1}
\end{isabelle}

\begin{isabelle}
\#2\ +\ n\ =\ Suc\ (Suc\ n)
\rulename{add_2_eq_Suc}
\end{isabelle}

\begin{isabelle}
n\ +\ \#2\ =\ Suc\ (Suc\ n)
\rulename{add_2_eq_Suc'}
\end{isabelle}

\begin{isabelle}
m\ +\ n\ +\ k\ =\ m\ +\ (n\ +\ k)
\rulename{add_assoc}
\end{isabelle}

\begin{isabelle}
m\ +\ n\ =\ n\ +\ m%
\rulename{add_commute}
\end{isabelle}

\begin{isabelle}
x\ +\ (y\ +\ z)\ =\ y\ +\ (x\ +\ z)
\rulename{add_left_commute}
\end{isabelle}

these form add_ac; similarly there is mult_ac%
\end{isamarkuptext}
\isacommand{lemma}\ "Suc(i\ +\ j*l*k\ +\ m*n)\ =\ f\ (n*m\ +\ i\ +\ k*j*l)"
\begin{isamarkuptxt}
\begin{isabelle}
\ 1.\ Suc\ (i\ +\ j\ *\ l\ *\ k\ +\ m\ *\ n)\ =\ f\ (n\ *\ m\ +\ i\ +\ k\ *\ j\ *\ l)
\end{isabelle}
\end{isamarkuptxt}
\isacommand{apply}\ (simp\ add:\ add_ac\ mult_ac)
\begin{isamarkuptxt}
\begin{isabelle}
\ 1.\ Suc\ (i\ +\ (m\ *\ n\ +\ j\ *\ (k\ *\ l)))\ =\isanewline
\isaindent{\ 1.\ }f\ (i\ +\ (m\ *\ n\ +\ j\ *\ (k\ *\ l)))
\end{isabelle}
\end{isamarkuptxt}
\isacommand{oops}
\begin{isamarkuptext}
\begin{isabelle}
\isasymlbrakk i\ \isasymle \ j;\ k\ \isasymle \ l\isasymrbrakk \ \isasymLongrightarrow \ i\ *\ k\ \isasymle \ j\ *\ l%
\rulename{mult_le_mono}
\end{isabelle}

\begin{isabelle}
\isasymlbrakk i\ <\ j;\ 0\ <\ k\isasymrbrakk \ \isasymLongrightarrow \ i\ *\ k\ <\ j\ *\ k%
\rulename{mult_less_mono1}
\end{isabelle}

\begin{isabelle}
m\ \isasymle \ n\ \isasymLongrightarrow \ m\ div\ k\ \isasymle \ n\ div\ k%
\rulename{div_le_mono}
\end{isabelle}

\begin{isabelle}
(m\ +\ n)\ *\ k\ =\ m\ *\ k\ +\ n\ *\ k%
\rulename{add_mult_distrib}
\end{isabelle}

\begin{isabelle}
(m\ -\ n)\ *\ k\ =\ m\ *\ k\ -\ n\ *\ k%
\rulename{diff_mult_distrib}
\end{isabelle}

\begin{isabelle}
m\ mod\ n\ *\ k\ =\ m\ *\ k\ mod\ (n\ *\ k)
\rulename{mod_mult_distrib}
\end{isabelle}

\begin{isabelle}
P\ (a\ -\ b)\ =\ ((a\ <\ b\ \isasymlongrightarrow \ P\ 0)\ \isasymand \ (\isasymforall d.\ a\ =\ b\ +\ d\ \isasymlongrightarrow \ P\ d))
\rulename{nat_diff_split}
\end{isabelle}
\end{isamarkuptext}
\isacommand{lemma}\ "(n-1)*(n+1)\ =\ n*n\ -\ 1"\isanewline
\isacommand{apply}\ (simp\ split:\ nat_diff_split)\isanewline
\isacommand{done}
\begin{isamarkuptext}
\begin{isabelle}
m\ mod\ n\ =\ (if\ m\ <\ n\ then\ m\ else\ (m\ -\ n)\ mod\ n)
\rulename{mod_if}
\end{isabelle}

\begin{isabelle}
m\ div\ n\ *\ n\ +\ m\ mod\ n\ =\ m%
\rulename{mod_div_equality}
\end{isabelle}


\begin{isabelle}
a\ *\ b\ div\ c\ =\ a\ *\ (b\ div\ c)\ +\ a\ *\ (b\ mod\ c)\ div\ c%
\rulename{div_mult1_eq}
\end{isabelle}

\begin{isabelle}
a\ *\ b\ mod\ c\ =\ a\ *\ (b\ mod\ c)\ mod\ c%
\rulename{mod_mult1_eq}
\end{isabelle}

\begin{isabelle}
a\ div\ (b\ *\ c)\ =\ a\ div\ b\ div\ c%
\rulename{div_mult2_eq}
\end{isabelle}

\begin{isabelle}
a\ mod\ (b\ *\ c)\ =\ b\ *\ (a\ div\ b\ mod\ c)\ +\ a\ mod\ b%
\rulename{mod_mult2_eq}
\end{isabelle}

\begin{isabelle}
0\ <\ c\ \isasymLongrightarrow \ c\ *\ a\ div\ (c\ *\ b)\ =\ a\ div\ b%
\rulename{div_mult_mult1}
\end{isabelle}

\begin{isabelle}
a\ div\ 0\ =\ 0
\rulename{DIVISION_BY_ZERO_DIV}
\end{isabelle}

\begin{isabelle}
a\ mod\ 0\ =\ a%
\rulename{DIVISION_BY_ZERO_MOD}
\end{isabelle}

\begin{isabelle}
\isasymlbrakk m\ dvd\ n;\ n\ dvd\ m\isasymrbrakk \ \isasymLongrightarrow \ m\ =\ n%
\rulename{dvd_anti_sym}
\end{isabelle}

\begin{isabelle}
\isasymlbrakk k\ dvd\ m;\ k\ dvd\ n\isasymrbrakk \ \isasymLongrightarrow \ k\ dvd\ m\ +\ n%
\rulename{dvd_add}
\end{isabelle}

For the integers, I'd list a few theorems that somehow involve negative 
numbers.  

Division, remainder of negatives


\begin{isabelle}
\#0\ <\ b\ \isasymLongrightarrow \ \#0\ \isasymle \ a\ mod\ b%
\rulename{pos_mod_sign}
\end{isabelle}

\begin{isabelle}
\#0\ <\ b\ \isasymLongrightarrow \ a\ mod\ b\ <\ b%
\rulename{pos_mod_bound}
\end{isabelle}

\begin{isabelle}
b\ <\ \#0\ \isasymLongrightarrow \ a\ mod\ b\ \isasymle \ \#0
\rulename{neg_mod_sign}
\end{isabelle}

\begin{isabelle}
b\ <\ \#0\ \isasymLongrightarrow \ b\ <\ a\ mod\ b%
\rulename{neg_mod_bound}
\end{isabelle}

\begin{isabelle}
(a\ +\ b)\ div\ c\ =\ a\ div\ c\ +\ b\ div\ c\ +\ (a\ mod\ c\ +\ b\ mod\ c)\ div\ c%
\rulename{zdiv_zadd1_eq}
\end{isabelle}

\begin{isabelle}
(a\ +\ b)\ mod\ c\ =\ (a\ mod\ c\ +\ b\ mod\ c)\ mod\ c%
\rulename{zmod_zadd1_eq}
\end{isabelle}

\begin{isabelle}
a\ *\ b\ div\ c\ =\ a\ *\ (b\ div\ c)\ +\ a\ *\ (b\ mod\ c)\ div\ c%
\rulename{zdiv_zmult1_eq}
\end{isabelle}

\begin{isabelle}
a\ *\ b\ mod\ c\ =\ a\ *\ (b\ mod\ c)\ mod\ c%
\rulename{zmod_zmult1_eq}
\end{isabelle}

\begin{isabelle}
\#0\ <\ c\ \isasymLongrightarrow \ a\ div\ (b\ *\ c)\ =\ a\ div\ b\ div\ c%
\rulename{zdiv_zmult2_eq}
\end{isabelle}

\begin{isabelle}
\#0\ <\ c\ \isasymLongrightarrow \ a\ mod\ (b\ *\ c)\ =\ b\ *\ (a\ div\ b\ mod\ c)\ +\ a\ mod\ b%
\rulename{zmod_zmult2_eq}
\end{isabelle}

\begin{isabelle}
\isasymbar x\ *\ y\isasymbar \ =\ \isasymbar x\isasymbar \ *\ \isasymbar y\isasymbar 
\rulename{abs_mult}
\end{isabelle}
\end{isamarkuptext}
\isacommand{lemma}\ "abs\ (x+y)\ \isasymle \ abs\ x\ +\ abs\ (y\ ::\ int)"\isanewline
\isacommand{by}\ arith\isanewline
\isanewline
\isacommand{lemma}\ "abs\ (\#2*x)\ =\ \#2\ *\ abs\ (x\ ::\ int)"\isanewline
\isacommand{by}\ (simp\ add:\ zabs_def)
\begin{isamarkuptext}
REALS

\begin{isabelle}
\isasymbar r\isasymbar \ \isacharcircum \ n\ =\ \isasymbar r\ \isacharcircum \ n\isasymbar 
\rulename{realpow_abs}
\end{isabelle}

\begin{isabelle}
x\ <\ y\ \isasymLongrightarrow \ \isasymexists r.\ x\ <\ r\ \isasymand \ r\ <\ y%
\rulename{real_dense}
\end{isabelle}

\begin{isabelle}
\isasymbar r\isasymbar \ \isacharcircum \ n\ =\ \isasymbar r\ \isacharcircum \ n\isasymbar 
\rulename{realpow_abs}
\end{isabelle}

\begin{isabelle}
x\ *\ (y\ /\ z)\ =\ x\ *\ y\ /\ z%
\rulename{real_times_divide1_eq}
\end{isabelle}

\begin{isabelle}
y\ /\ z\ *\ x\ =\ y\ *\ x\ /\ z%
\rulename{real_times_divide2_eq}
\end{isabelle}

\begin{isabelle}
x\ /\ (y\ /\ z)\ =\ x\ *\ z\ /\ y%
\rulename{real_divide_divide1_eq}
\end{isabelle}

\begin{isabelle}
x\ /\ y\ /\ z\ =\ x\ /\ (y\ *\ z)
\rulename{real_divide_divide2_eq}
\end{isabelle}

\begin{isabelle}
-\ x\ /\ y\ =\ -\ (x\ /\ y)
\rulename{real_minus_divide_eq}
\end{isabelle}

\begin{isabelle}
x\ /\ -\ y\ =\ -\ (x\ /\ y)
\rulename{real_divide_minus_eq}
\end{isabelle}

This last NOT a simprule

\begin{isabelle}
(x\ +\ y)\ /\ z\ =\ x\ /\ z\ +\ y\ /\ z%
\rulename{real_add_divide_distrib}
\end{isabelle}
\end{isamarkuptext}
\isacommand{lemma}\ "\#3/\#4\ <\ (\#7/\#8\ ::\ real)"\isanewline
\isacommand{by}\ (simp)\ \isanewline
\isanewline
\isacommand{lemma}\ "P\ ((\#3/\#4)\ *\ (\#8/\#15\ ::\ real))"
\begin{isamarkuptxt}
\begin{isabelle}
\ 1.\ P\ (\#3\ /\ \#4\ *\ (\#8\ /\ \#15))
\end{isabelle}
\end{isamarkuptxt}
\isacommand{apply}(simp)
\begin{isamarkuptxt}
\begin{isabelle}
\ 1.\ P\ (\#2\ /\ \#5)
\end{isabelle}
\end{isamarkuptxt}
\isacommand{oops}\isanewline
\isanewline
\isacommand{lemma}\ "(\#3/\#4)\ *\ (\#8/\#15)\ <\ (x\ ::\ real)"
\begin{isamarkuptxt}
\begin{isabelle}
\ 1.\ \#3\ /\ \#4\ *\ (\#8\ /\ \#15)\ <\ x%
\end{isabelle}
\end{isamarkuptxt}
\isacommand{apply}(simp)
\begin{isamarkuptxt}
\begin{isabelle}
\ 1.\ \#2\ <\ x\ *\ \#5
\end{isabelle}
\end{isamarkuptxt}
\isacommand{oops}\isanewline
\isanewline
\isanewline
\isanewline
\isacommand{end}\isanewline
\end{isabelle}
%%% Local Variables:
%%% mode: latex
%%% TeX-master: "root"
%%% End:
