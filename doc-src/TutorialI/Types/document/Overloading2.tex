%
\begin{isabellebody}%
\def\isabellecontext{Overloading{\isadigit{2}}}%
\isamarkupfalse%
%
\begin{isamarkuptext}%
Of course this is not the only possible definition of the two relations.
Componentwise comparison of lists of equal length also makes sense. This time
the elements of the list must also be of class \isa{ordrel} to permit their
comparison:%
\end{isamarkuptext}%
\isamarkuptrue%
\isacommand{instance}\ list\ {\isacharcolon}{\isacharcolon}\ {\isacharparenleft}ordrel{\isacharparenright}ordrel\isanewline
\isamarkupfalse%
\isacommand{by}\ intro{\isacharunderscore}classes\isanewline
\isanewline
\isamarkupfalse%
\isacommand{defs}\ {\isacharparenleft}\isakeyword{overloaded}{\isacharparenright}\isanewline
le{\isacharunderscore}list{\isacharunderscore}def{\isacharcolon}\ {\isachardoublequote}xs\ {\isacharless}{\isacharless}{\isacharequal}\ {\isacharparenleft}ys{\isacharcolon}{\isacharcolon}{\isacharprime}a{\isacharcolon}{\isacharcolon}ordrel\ list{\isacharparenright}\ {\isasymequiv}\isanewline
\ \ \ \ \ \ \ \ \ \ \ \ \ \ size\ xs\ {\isacharequal}\ size\ ys\ {\isasymand}\ {\isacharparenleft}{\isasymforall}i{\isacharless}size\ xs{\isachardot}\ xs{\isacharbang}i\ {\isacharless}{\isacharless}{\isacharequal}\ ys{\isacharbang}i{\isacharparenright}{\isachardoublequote}\isamarkupfalse%
%
\begin{isamarkuptext}%
\noindent
The infix function \isa{{\isacharbang}} yields the nth element of a list.

\begin{warn}
A type constructor can be instantiated in only one way to
a given type class.  For example, our two instantiations of \isa{list} must
reside in separate theories with disjoint scopes.
\end{warn}%
\end{isamarkuptext}%
\isamarkuptrue%
%
\isamarkupsubsubsection{Predefined Overloading%
}
\isamarkuptrue%
%
\begin{isamarkuptext}%
HOL comes with a number of overloaded constants and corresponding classes.
The most important ones are listed in Table~\ref{tab:overloading} in the appendix. They are
defined on all numeric types and sometimes on other types as well, for example
$-$ and \isa{{\isasymle}} on sets.

In addition there is a special syntax for bounded quantifiers:
\begin{center}
\begin{tabular}{lcl}
\isa{{\isasymforall}x{\isasymle}y{\isachardot}\ P\ x} & \isa{{\isasymrightleftharpoons}} & \isa{{\isachardoublequote}{\isasymforall}x{\isachardot}\ x\ {\isasymle}\ y\ {\isasymlongrightarrow}\ P\ x{\isachardoublequote}} \\
\isa{{\isasymexists}x{\isasymle}y{\isachardot}\ P\ x} & \isa{{\isasymrightleftharpoons}} & \isa{{\isachardoublequote}{\isasymexists}x{\isachardot}\ x\ {\isasymle}\ y\ {\isasymand}\ P\ x{\isachardoublequote}}
\end{tabular}
\end{center}
And analogously for \isa{{\isacharless}} instead of \isa{{\isasymle}}.%
\end{isamarkuptext}%
\isamarkuptrue%
\isamarkupfalse%
\end{isabellebody}%
%%% Local Variables:
%%% mode: latex
%%% TeX-master: "root"
%%% End:
