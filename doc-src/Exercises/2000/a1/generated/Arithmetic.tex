%
\begin{isabellebody}%
\def\isabellecontext{Arithmetic}%
\isamarkupfalse%
%
\isamarkupsubsection{Arithmetic%
}
\isamarkuptrue%
%
\isamarkupsubsubsection{Power%
}
\isamarkuptrue%
%
\begin{isamarkuptext}%
Define a primitive recursive function $pow~x~n$ that
computes $x^n$ on natural numbers.%
\end{isamarkuptext}%
\isamarkuptrue%
\isacommand{consts}\isanewline
\ \ pow\ {\isacharcolon}{\isacharcolon}\ {\isachardoublequote}nat\ {\isacharequal}{\isachargreater}\ nat\ {\isacharequal}{\isachargreater}\ nat{\isachardoublequote}\isamarkupfalse%
%
\begin{isamarkuptext}%
Prove the well known equation $x^{m \cdot n} = (x^m)^n$:%
\end{isamarkuptext}%
\isamarkuptrue%
\isacommand{theorem}\ pow{\isacharunderscore}mult{\isacharcolon}\ {\isachardoublequote}pow\ x\ {\isacharparenleft}m\ {\isacharasterisk}\ n{\isacharparenright}\ {\isacharequal}\ pow\ {\isacharparenleft}pow\ x\ m{\isacharparenright}\ n{\isachardoublequote}\isamarkupfalse%
\isamarkupfalse%
%
\begin{isamarkuptext}%
Hint: prove a suitable lemma first.  If you need to appeal to
associativity and commutativity of multiplication: the corresponding
simplification rules are named \isa{mult{\isacharunderscore}ac}.%
\end{isamarkuptext}%
\isamarkuptrue%
%
\isamarkupsubsubsection{Summation%
}
\isamarkuptrue%
%
\begin{isamarkuptext}%
Define a (primitive recursive) function $sum~ns$ that sums a list
of natural numbers: $sum [n_1, \dots, n_k] = n_1 + \cdots + n_k$.%
\end{isamarkuptext}%
\isamarkuptrue%
\isacommand{consts}\isanewline
\ \ sum\ {\isacharcolon}{\isacharcolon}\ {\isachardoublequote}nat\ list\ {\isacharequal}{\isachargreater}\ nat{\isachardoublequote}\isamarkupfalse%
%
\begin{isamarkuptext}%
Show that $sum$ is compatible with $rev$. You may need a lemma.%
\end{isamarkuptext}%
\isamarkuptrue%
\isacommand{theorem}\ sum{\isacharunderscore}rev{\isacharcolon}\ {\isachardoublequote}sum\ {\isacharparenleft}rev\ ns{\isacharparenright}\ {\isacharequal}\ sum\ ns{\isachardoublequote}\isamarkupfalse%
\isamarkupfalse%
%
\begin{isamarkuptext}%
Define a function $Sum~f~k$ that sums $f$ from $0$
up to $k-1$: $Sum~f~k = f~0 + \cdots + f(k - 1)$.%
\end{isamarkuptext}%
\isamarkuptrue%
\isacommand{consts}\isanewline
\ \ Sum\ {\isacharcolon}{\isacharcolon}\ {\isachardoublequote}{\isacharparenleft}nat\ {\isacharequal}{\isachargreater}\ nat{\isacharparenright}\ {\isacharequal}{\isachargreater}\ nat\ {\isacharequal}{\isachargreater}\ nat{\isachardoublequote}\isamarkupfalse%
%
\begin{isamarkuptext}%
Show the following equations for the pointwise summation of functions.
Determine first what the expression \isa{whatever} should be.%
\end{isamarkuptext}%
\isamarkuptrue%
\isacommand{theorem}\ {\isachardoublequote}Sum\ {\isacharparenleft}{\isacharpercent}i{\isachardot}\ f\ i\ {\isacharplus}\ g\ i{\isacharparenright}\ k\ {\isacharequal}\ Sum\ f\ k\ {\isacharplus}\ Sum\ g\ k{\isachardoublequote}\isamarkupfalse%
\isanewline
\isamarkupfalse%
\isacommand{theorem}\ {\isachardoublequote}Sum\ f\ {\isacharparenleft}k\ {\isacharplus}\ l{\isacharparenright}\ {\isacharequal}\ Sum\ f\ k\ {\isacharplus}\ Sum\ whatever\ l{\isachardoublequote}\isamarkupfalse%
\isamarkupfalse%
%
\begin{isamarkuptext}%
What is the relationship between \isa{sum} and \isa{Sum}?
Prove the following equation, suitably instantiated.%
\end{isamarkuptext}%
\isamarkuptrue%
\isacommand{theorem}\ {\isachardoublequote}Sum\ f\ k\ {\isacharequal}\ sum\ whatever{\isachardoublequote}\isamarkupfalse%
\isamarkupfalse%
%
\begin{isamarkuptext}%
Hint: familiarize yourself with the predefined functions \isa{map} and
\isa{{\isacharbrackleft}i{\isachardot}{\isachardot}j{\isacharparenleft}{\isacharbrackright}} on lists in theory List.%
\end{isamarkuptext}%
\isamarkuptrue%
\isamarkupfalse%
\end{isabellebody}%
%%% Local Variables:
%%% mode: latex
%%% TeX-master: "root"
%%% End:
