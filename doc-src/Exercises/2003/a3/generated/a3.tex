%
\begin{isabellebody}%
\def\isabellecontext{a{\isadigit{3}}}%
\isamarkupfalse%
%
\isamarkupsubsection{Computing with natural numbers - Magical Methods%
}
\isamarkuptrue%
%
\begin{isamarkuptext}%
A book about Vedic Mathematics describes three methods to make the calculation of squares of natural numbers easier:

\begin{itemize}
\item {\em MM1}: Numbers whose predecessors have squares that are known or can easily be calculated. For example:
\\ Needed: $61^2$  
\\ Given: $60^2 = 3600$
\\ Observe: $61^2 = 3600 + 60 + 61 = 3721$

\item {\em MM2}: Numbers greater than, but near 100. For example:
\\ Needed: $102^2$
\\ Let $h = 102 - 100 = 2$ , $h^2 = 4$
\\ Observe: $102^2 = (102+h)$ shifted two places to the left $ + h^2 = 10404$
 
\item {\em MM3}: Numbers ending in $5$. For example:
\\ Needed: $85^2$
\\ Observe: $85^2 = (8 * 9)$ appended to $ 25 = 7225$
\\ Needed: $995^2$
\\ Observe: $995^2 = (99 * 100)$ appended to $ 25 = 990025 $
\end{itemize}


In this exercise we will show that these methods are not so magical after all!

\begin{itemize}
\item Based on {\em MM1} define a function \isa{sq} that calculates the square of a natural number.
\item Prove the correctness of \isa{sq} (i.e.\ \isa{sq\ n\ {\isacharequal}\ n\ {\isacharasterisk}\ n}).
\item Formulate and prove the correctness of {\em MM2}.\\ Hints:
  \begin{itemize}
  \item Generalise {\em MM2} for an arbitrary constant (instead of $100$).
  \item Universally quantify all variables other than the induction variable.
\end{itemize}
\item Formulate and prove the correctness of {\em MM3}.\\ Hints:
  \begin{itemize}
  \item Try to formulate the property `numbers ending in $5$' such that it is easy to get to the rest of the number.
  \item Proving the binomial formula for $(a+b)^2$ can be of some help.
  \end{itemize}
\end{itemize}%
\end{isamarkuptext}%
\isamarkuptrue%
\isamarkupfalse%
\end{isabellebody}%
%%% Local Variables:
%%% mode: latex
%%% TeX-master: "root"
%%% End:
