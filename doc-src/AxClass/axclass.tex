

%% $Id$

\documentclass[11pt,a4paper]{article}
\usepackage{isabelle,pdfsetup}

\renewcommand{\isamarkupheader}[1]{\section{#1}}
\parindent 0pt \parskip 0.5ex

\newcommand{\name}[1]{\textsf{#1}}

\newcommand{\idt}[1]{{\mathord{\mathit{#1}}}}
\newcommand{\var}[1]{{?\!#1}}
\DeclareMathSymbol{\dshsym}{\mathalpha}{letters}{"2D}
\newcommand{\dsh}{\dshsym}

\newcommand{\To}{\to}
\newcommand{\dt}{{\mathpunct.}}
\newcommand{\ap}{\mathbin{\!}}
\newcommand{\lam}[1]{\mathop{\lambda} #1\dt\;}
\newcommand{\all}[1]{\forall #1\dt\;}
\newcommand{\ex}[1]{\exists #1\dt\;}

%%% Local Variables: 
%%% mode: latex
%%% TeX-master: "root"
%%% End: 


\begin{document}

\title{Using Axiomatic Type Classes in Isabelle \\ a tutorial}
\author{Markus Wenzel}
%\date{29 August 1995}
\maketitle

\Isa\ features a \Haskell-like type system with ordered type classes
already since 1991 (see \cite{Nipkow93}). Initially, classes mainly
served as a \E{syntactic tool} to formulate polymorphic object logics
in a clean way (like many-sorted \FOL, see
\cite[\S1.1.2--1.1.3]{Paulson94}).

Applying classes at the \E{logical level} to provide a simple notion
of abstract theories and instantiations to concrete ones, has also
been long proposed (see \cite{Nipkow-slides} and
\cite[\S4]{Nipkow93}). At that time, \Isa\ still lacked built-in
support for these \E{axiomatic type classes}. More importantly, their
semantics was not yet fully fleshed out and unnecessarily complicated.

How simple things really are has been shown in \cite[esp.\ 
\S4]{Wenzel94} which also provided an implementation of the axclass
package that was eventually released with \Isa94. Unfortunately there
was a snag: That version of \Isa\ still contained an old conceptual
bug in the core machinery which caused theories to become inconsistent
after introducing empty sorts. Note that empty intersections of
axiomatic type classes easily occur, unless all basic classes are very
trivial.

These problems prevented the axclass package to be used seriously ---
they have been fixed in \Isa94-2.


\section{Some simple examples} \label{sec:ex}

Axiomatic type classes are a concept of \Isa's meta-logic.  They may
be applied to any object-logic that directly uses the meta type
system. Subsequently, we present various examples that are formulated
within \Isa/\HOL\footnote{See also directory
  \TT{HOL/AxClasses/Tutorial}.}, except the one of
\secref{sec:ex-natclass} which is in \Isa/\FOL\footnote{See also files
  \TT{FOL/ex/NatClass.thy} and \TT{FOL/ex/NatClass.ML}.}.


\subsection{Semigroups}

An axiomatic type class is simply a class of types that all meet
certain \E{axioms}. Thus, type classes may be also understood as type
predicates --- i.e.\ abstractions over a single type argument
$\alpha$.  Class axioms typically contain polymorphic constants that
depend on this type $\alpha$. These \E{characteristic constants}
behave like operations associated with the ``carrier'' $\alpha$.

We illustrate these basic concepts with the following theory
\TT{Semigroup}:

\begin{ascbox}
Semigroup = HOL +\medskip
consts
  "<*>"         :: "['a, 'a] => 'a"             (infixl 70)\medskip
axclass
  semigroup < term
  assoc         "(x <*> y) <*> z = x <*> (y <*> z)"\medskip
end
\end{ascbox}

\TT{Semigroup} first declares a polymorphic constant $\TIMES ::
[\alpha, \alpha] \To \alpha$ and then defines the class \TT{semigroup}
of all those types $\tau$ such that $\TIMES :: [\tau, \tau] \To \tau$
is an associative operator\footnote{Note that $\TIMES$ is used here
  instead of $*$, because the latter is already declared in theory
  \TT{HOL} in a slightly different way.}. The axiom \TT{assoc}
contains exactly one type variable, which is invisible in the above
presentation, though.  Also note that free term variables (like $x$,
$y$, $z$) are allowed for user convenience --- conceptually all of
these are bound by outermost universal quantifiers.

\medskip

In general, type classes may be used to describe \E{structures} with
exactly one carrier $\alpha$ and a fixed \E{signature}. Different
signatures require different classes. In the following theory
\TT{Semigroups}, class \TT{plus\_sg} represents semigroups of the form
$(\tau, \PLUS^\tau)$ while \TT{times\_sg} represents semigroups
$(\tau, \TIMES^\tau)$:

\begin{ascbox}
Semigroups = HOL +\medskip
consts
  "<+>"         :: "['a, 'a] => 'a"             (infixl 65)
  "<*>"         :: "['a, 'a] => 'a"             (infixl 70)
  assoc         :: "(['a, 'a] => 'a) => bool"\medskip
defs
  assoc_def     "assoc f == ALL x y z. f (f x y) z = f x (f y z)"\medskip
axclass
  plus_sg < term
  plus_assoc    "assoc (op <+>)"\medskip
axclass
  times_sg < term
  times_assoc   "assoc (op <*>)"\medskip
end
\end{ascbox}

Even if both classes \TT{plus\_sg} and \TT{times\_sg} represent
semigroups in a sense, they are not the same!


\subsection{Basic group theory}

The meta type system of \Isa\ supports \E{intersections} and
\E{inclusions} of type classes. These directly correspond to
intersections and inclusions of type predicates in a purely set
theoretic sense. This is sufficient as a means to describe simple
hierarchies of structures.  As an illustration, we use the well-known
example of semigroups, monoids, general groups and abelian groups.


\subsubsection{Monoids and Groups}

First a small theory that provides some polymorphic constants to be
used later for the signature parts:

\begin{ascbox}
Sigs = HOL +\medskip
consts
  "<*>"     :: "['a, 'a] => 'a"       (infixl 70)
  inverse   :: "'a => 'a"
  "1"       :: "'a"                   ("1")\medskip
end
\end{ascbox}

Next comes the theory \TT{Monoid} which defines class
\TT{monoid}\footnote{Note that multiple class axioms are allowed for
  user convenience --- they simply represent the conjunction of their
  respective universal closures.}:

\begin{ascbox}
Monoid = Sigs +\medskip
axclass
  monoid < term
  assoc         "(x <*> y) <*> z = x <*> (y <*> z)"
  left_unit     "1 <*> x = x"
  right_unit    "x <*> 1 = x"\medskip
end
\end{ascbox}

So class \TT{monoid} contains exactly those types $\tau$ where $\TIMES
:: [\tau, \tau] \To \tau$ and $\TT{1} :: \tau$ are specified
appropriately, such that $\TIMES$ is associative and $\TT{1}$ is a
left and right unit for $\TIMES$.

\medskip

Independently of \TT{Monoid}, we now define theory \TT{Group} which
introduces a linear hierarchy of semigroups, general groups and
abelian groups:

\begin{ascbox}
Group = Sigs +\medskip
axclass
  semigroup < term
  assoc         "(x <*> y) <*> z = x <*> (y <*> z)"\brk
axclass
  group < semigroup
  left_unit     "1 <*> x = x"
  left_inverse  "inverse x <*> x = 1"\medskip
axclass
  agroup < group
  commut        "x <*> y = y <*> x"\medskip
end
\end{ascbox}

Class \TT{group} inherits associativity of $\TIMES$ from
\TT{semigroup} and adds the other two group axioms. Similarly,
\TT{agroup} is defined as the subset of \TT{group} such that for all
of its elements $\tau$, the operation $\TIMES :: [\tau, \tau] \To
\tau$ is even commutative.


\subsubsection{Abstract reasoning}

In a sense, axiomatic type classes may be viewed as \E{abstract
  theories}.  Above class definitions internally generate the
following abstract axioms:

\begin{ascbox}
assoc:        (?x::?'a::semigroup) <*> (?y::?'a::semigroup)
                <*> (?z::?'a::semigroup) = ?x <*> (?y <*> ?z)\medskip
left_unit:    1 <*> (?x::?'a::group) = ?x
left_inverse: inverse (?x::?'a::group) <*> ?x = 1\medskip
commut:       (?x::?'a::agroup) <*> (?y::?'a::agroup) = ?y <*> ?x
\end{ascbox}

All of these contain type variables $\alpha :: c$ that are restricted
to types of some class $c$. These \E{sort constraints} express a
logical precondition for the whole formula. For example, \TT{assoc}
states that for all $\tau$, provided that $\tau :: \TT{semigroup}$,
the operation $\TIMES :: [\tau, \tau] \To \tau$ is associative.

\medskip

From a purely technical point of view, abstract axioms are just
ordinary \Isa-theorems (of \ML-type \TT{thm}). They may be used for
\Isa-proofs without special treatment. Such ``abstract proofs''
usually yield new abstract theorems. For example, in theory \TT{Group}
we may derive:

\begin{ascbox}
right_unit:      (?x::?'a::group) <*> 1 = ?x
right_inverse:   (?x::?'a::group) <*> inverse ?x = 1
inverse_product: inverse ((?x::?'a::group) <*> (?y::?'a::group)) =
                   inverse ?y <*> inverse ?x
inverse_inv:     inverse (inverse (?x::?'a::group)) = ?x
ex1_inverse:     ALL x::?'a::group. EX! y::?'a::group. y <*> x = 1
\end{ascbox}

Abstract theorems (which include abstract axioms, of course) may be
instantiated to only those types $\tau$ where the appropriate class
membership $\tau :: c$ is known at \Isa's type signature level.  Since
we have $\TT{agroup} \subseteq \TT{group} \subseteq \TT{semigroup}$ by
definition, all theorems of \TT{semigroup} and \TT{group} are
automatically inherited by \TT{group} and \TT{agroup}.


\subsubsection{Abstract instantiation}

Until now, theories \TT{Monoid} and \TT{Group} were independent.
Forming their union $\TT{Monoid} + \TT{Group}$ we get the following
class hierarchy at the type signature level (left hand side):

\begin{center}
  \small
  \unitlength 0.75mm
  \begin{picture}(65,90)(0,-10)
    \put(15,10){\line(0,1){10}} \put(15,30){\line(0,1){10}}
    \put(15,50){\line(1,1){10}} \put(35,60){\line(1,-1){10}}
    \put(15,5){\makebox(0,0){\tt agroup}}
    \put(15,25){\makebox(0,0){\tt group}}
    \put(15,45){\makebox(0,0){\tt semigroup}}
    \put(30,65){\makebox(0,0){\tt term}} \put(50,45){\makebox(0,0){\tt
    monoid}}
  \end{picture}
  \hspace{4em}
  \begin{picture}(30,90)(0,0)
    \put(15,10){\line(0,1){10}} \put(15,30){\line(0,1){10}}
    \put(15,50){\line(0,1){10}} \put(15,70){\line(0,1){10}}
    \put(15,5){\makebox(0,0){\tt agroup}}
    \put(15,25){\makebox(0,0){\tt group}}
    \put(15,45){\makebox(0,0){\tt monoid}}
    \put(15,65){\makebox(0,0){\tt semigroup}}
    \put(15,85){\makebox(0,0){\tt term}}
  \end{picture}
\end{center}

We know by definition that $\TIMES$ is associative for monoids, i.e.\ 
$\TT{monoid} \subseteq \TT{semigroup}$. Furthermore we have
\TT{assoc}, \TT{left\_unit} and \TT{right\_unit} for groups (where
\TT{right\_unit} is derivable from the group axioms), that is
$\TT{group} \subseteq \TT{monoid}$. This way we get the refined class
hierarchy shown above at the right hand side.

The additional inclusions $\TT{monoid} \subseteq \TT{semigroup}$ and
$\TT{group} \subseteq \TT{monoid}$ are established by logical means
and have to be explicitly made known at the type signature level. This
is what the theory section \TT{instance} does. So we define
\TT{MonoidGroupInsts} as follows:

\begin{ascbox}
MonoidGroupInsts = Monoid + Group +\medskip
instance
  monoid < semigroup            (Monoid.assoc)\medskip
instance
  group < monoid                (assoc, left_unit, right_unit)\medskip
end
\end{ascbox}

The \TT{instance} section does really check that the class inclusions
(or type arities) to be added are derivable. To this end, it sets up a
suitable goal and attempts a proof with the help of some internal
axioms and user supplied theorems. These latter \E{witnesses} usually
should be appropriate type instances of the class axioms (as are
\TT{Monoid.assoc} and \TT{assoc}, \TT{left\_unit}, \TT{right\_unit}
above).

The most important internal tool for instantiation proofs is the class
introduction rule that is automatically generated by \TT{axclass}. For
class \TT{group} this is axiom \TT{groupI}:

\begin{ascbox}
groupI:   [| OFCLASS(?'a::term, semigroup_class);
             !!x::?'a::term. 1 <*> x = x;
             !!x::?'a::term. inverse x <*> x = 1 |]
          ==> OFCLASS(?'a::term, group_class)
\end{ascbox}

There are also axioms \TT{monoidI}, \TT{semigroupI} and \TT{agroupI}
of a similar form.  Note that $\TT{OFCLASS}(\tau, c \TT{\_class})$
expresses class membership $\tau :: c$ as a proposition of the
meta-logic.


\subsubsection{Concrete instantiation}

So far we have covered the case of \TT{instance $c_1$ < $c_2$} that
could be described as \E{abstract instantiation} --- $c_1$ is more
special than $c_2$ and thus an instance of $c_2$. Even more
interesting for practical applications are \E{concrete instantiations}
of axiomatic type classes. That is, certain simple schemes $(\alpha_1,
\ldots, \alpha_n)t :: c$ of class membership may be transferred to
\Isa's type signature level. This form of \TT{instance} is a ``safe''
variant of the old-style \TT{arities} theory section.

As an example, we show that type \TT{bool} with exclusive-or as
operation $\TIMES$, identity as \TT{inverse}, and \TT{False} as \TT{1}
forms an abelian group. We first define theory \TT{Xor}:

\begin{ascbox}
Xor = Group +\medskip
defs
  prod_bool_def     "x <*> y   == x ~= (y::bool)"
  inverse_bool_def  "inverse x == x::bool"
  unit_bool_def     "1         == False"\medskip
end
\end{ascbox}

It is important to note that above \TT{defs} are just overloaded
meta-level constant definitions. In particular, type classes are not
yet involved at all! This form of constant definition with overloading
(and optional primitive recursion over types) would be also possible
in traditional versions of \HOL\ that lack type classes.
% (see FIXME <foundation> for more details)
Nonetheless, such definitions are best applied in the context of
classes.

\medskip

Since we chose the \TT{defs} of theory \TT{Xor} suitably, the class
membership $\TT{bool} :: \TT{agroup}$ is now derivable as follows:

\begin{ascbox}
open AxClass;
goal_arity Xor.thy ("bool", [], "agroup");
\out{Level 0}
\out{OFCLASS(bool, agroup_class)}
\out{ 1. OFCLASS(bool, agroup_class)}\brk
by (axclass_tac []);
\out{Level 1}
\out{OFCLASS(bool, agroup_class)}
\out{ 1. !!(x::bool) (y::bool) z::bool. x <*> y <*> z = x <*> (y <*> z)}
\out{ 2. !!x::bool. 1 <*> x = x}
\out{ 3. !!x::bool. inverse x <*> x = 1}
\out{ 4. !!(x::bool) y::bool. x <*> y = y <*> x}
\end{ascbox}

The tactic \TT{axclass\_tac} has applied \TT{agroupI} internally to
expand the class definition. It now remains to be shown that the
\TT{agroup} axioms hold for bool. To this end, we expand the
definitions of \TT{Xor} and solve the new subgoals by \TT{fast\_tac}
of \HOL:

\begin{ascbox}
by (rewrite_goals_tac [Xor.prod_bool_def, Xor.inverse_bool_def,
        Xor.unit_bool_def]);
\out{Level 2}
\out{OFCLASS(bool, agroup_class)}
\out{ 1. !!(x::bool) (y::bool) z::bool. x ~= y ~= z = (x ~= (y ~= z))}
\out{ 2. !!x::bool. False ~= x = x}
\out{ 3. !!x::bool. x ~= x = False}
\out{ 4. !!(x::bool) y::bool. x ~= y = (y ~= x)}
by (ALLGOALS (fast_tac HOL_cs));
\out{Level 3}
\out{OFCLASS(bool, agroup_class)}
\out{No subgoals!}
qed "bool_in_agroup";
\out{val bool_in_agroup = "OFCLASS(bool, agroup_class)"}
\end{ascbox}

The result is theorem $\TT{OFCLASS}(\TT{bool}, \TT{agroup\_class})$
which expresses $\TT{bool} :: \TT{agroup}$ as a meta-proposition. This
is not yet known at the type signature level, though.

\medskip

What we have done here by hand, can be also performed via
\TT{instance} in a similar way behind the scenes. This may look as
follows\footnote{Subsequently, theory \TT{Xor} is no longer
  required.}:

\begin{ascbox}
BoolGroupInsts = Group +\medskip
defs
  prod_bool_def     "x <*> y   == x ~= (y::bool)"
  inverse_bool_def  "inverse x == x::bool"
  unit_bool_def     "1         == False"\medskip
instance
  bool :: agroup                \{| ALLGOALS (fast_tac HOL_cs) |\}\medskip
end
\end{ascbox}

This way, we have $\TT{bool} :: \TT{agroup}$ in the type signature of
\TT{BoolGroupInsts}, and all abstract group theorems are transferred
to \TT{bool} for free.

Similarly, we could now also instantiate our group theory classes to
many other concrete types. For example, $\TT{int} :: \TT{agroup}$ (by
defining $\TIMES$ as addition, \TT{inverse} as negation and \TT{1} as
zero, say) or $\TT{list} :: (\TT{term})\TT{semigroup}$ (e.g.\ if
$\TIMES$ is list append). Thus, the characteristic constants $\TIMES$,
\TT{inverse}, \TT{1} really become overloaded, i.e.\ have different
meanings on different types.


\subsubsection{Lifting and Functors}

As already mentioned above, \HOL-like systems not only support
overloaded definitions of polymorphic constants (without requiring
type classes), but even primitive recursion over types. That is,
definitional equations $c^\tau \Eq t$ may also contain constants of
name $c$ on the RHS --- provided that these have types that are
strictly simpler (structurally) than $\tau$.

This feature enables us to \E{lift operations}, e.g.\ to Cartesian
products, direct sums or function spaces. Below, theory
\TT{ProdGroupInsts} lifts $\TIMES$ componentwise to two-place
Cartesian products $\alpha \times \beta$:

\begin{ascbox}
ProdGroupInsts = Prod + Group +\medskip
defs
  prod_prod_def "p <*> q == (fst p <*> fst q, snd p <*> snd q)"\medskip
instance
  "*" :: (semigroup, semigroup) semigroup
    \{| simp_tac (prod_ss addsimps [assoc]) 1 |\}
end
\end{ascbox}

Note that \TT{prod\_prod\_def} basically has the form $\edrv
{\TIMES}^{\alpha \times \beta} \Eq \ldots {\TIMES}^\alpha \ldots
{\TIMES}^\beta \ldots$, i.e.\ the recursive occurrences
$\TIMES^\alpha$ and $\TIMES^\beta$ are really at ``simpler'' types.

It is very easy to see that associativity of $\TIMES^\alpha$,
$\TIMES^\beta$ transfers to ${\TIMES}^{\alpha \times \beta}$. Hence
the two-place type constructor $\times$ maps semigroups into
semigroups. This fact is proven and put into \Isa's type signature by
above \TT{instance} section.

\medskip

Thus, if we view class instances as ``structures'', overloaded
constant definitions with primitive recursion over types indirectly
provide some kind of ``functors'' (mappings between abstract theories,
that is).


\subsection{Syntactic classes}

There is still a feature of \Isa's type system left that we have not
yet used: When declaring polymorphic constants $c :: \tau$, the type
variables occurring in $\tau$ may be constrained by type classes (or
even general sorts). Note that by default, in \Isa/\HOL\ the
declaration:

\begin{ascbox}
  <*> :: ['a, 'a] => 'a
\end{ascbox}

is actually an abbreviation for:

\begin{ascbox}
  <*> :: ['a::term, 'a::term] => 'a::term
\end{ascbox}

Since class \TT{term} is the universal class of \HOL, this is not
really a restriction at all. A less trivial example is the following
theory \TT{Product}:

\begin{ascbox}
Product = HOL +\medskip
axclass
  product < term\medskip
consts
  "<*>" :: "['a::product, 'a] => 'a"      (infixl 70)\medskip
end
\end{ascbox}

Here class \TT{product} is defined as subclass of \TT{term}, but
without any additional axioms. This effects in logical equivalence of
\TT{product} and \TT{term}, since \TT{productI} is as follows:

\begin{ascbox}
productI: OFCLASS(?'a::logic, term_class) ==>
            OFCLASS(?'a::logic, product_class)
\end{ascbox}

I.e.\ $\TT{term} \subseteq \TT{product}$. The converse inclusion is
already contained in the type signature of theory \TT{Product}.

Now, what is the difference of declaring $\TIMES :: [\alpha ::
\TT{product}, \alpha] \To \alpha$ vs.\ declaring $\TIMES :: [\alpha ::
\TT{term}, \alpha] \To \alpha$? In this special case (where
$\TT{product} \Eq \TT{term}$), it should be obvious that both
declarations are the same from the logic's point of view.  It is even
most sensible in the general case not to attach any \E{logical}
meaning to sort constraints occurring in constant \E{declarations}
(see \cite[page 79]{Wenzel94} for more details).

On the other hand there are syntactic differences, of course.
Constants $\TIMES^\tau$ are rejected by the type-checker, unless $\tau
:: \TT{product}$ is part of the type signature.  In our example, this
arity may be always added when required by means of a trivial
\TT{instance}.

Thus, we can follow this discipline: Overloaded polymorphic constants
have their type arguments restricted to an associated (logically
trivial) class $c$. Only immediately before \E{specifying} these
constants on a certain type $\tau$ do we instantiate $\tau :: c$.

This is done for class \TT{product} and type \TT{bool} in theory
\TT{ProductInsts} below:

\begin{ascbox}
ProductInsts = Product +\medskip
instance
  bool :: product\medskip
defs
  prod_bool_def "x <*> y  == x & y::bool"\medskip
end
\end{ascbox}

Note that \TT{instance bool ::\ product} does not require any
witnesses, since this statement is logically trivial. Nonetheless,
\TT{instance} really performs a proof.

Only after $\TT{bool} :: \TT{product}$ is made known to the type
checker, does \TT{prod\_bool\_def} become syntactically well-formed.

\medskip

It is very important to see that above \TT{defs} are not directly
connected with \TT{instance} at all! We were just following our
convention to specify $\TIMES$ on \TT{bool} after having instantiated
$\TT{bool} :: \TT{product}$. \Isa\ does not require these definitions,
which is in contrast to programming languages like \Haskell.

\medskip

While \Isa\ type classes and those of \Haskell\ are almost the same as
far as type-checking and type inference are concerned, there are major
semantic differences. \Haskell\ classes require their instances to
\E{provide operations} of certain \E{names}. Therefore, its
\TT{instance} has a \TT{where} part that tells the system what these
``member functions'' should be.

This style of \TT{instance} won't make much sense in \Isa, because its
meta-logic has no corresponding notion of ``providing operations'' or
``names''.


\subsection{Defining natural numbers in FOL}
\label{sec:ex-natclass}

Axiomatic type classes abstract over exactly one type argument. Thus,
any \E{axiomatic} theory extension where each axiom refers to at most
one type variable, may be trivially turned into a \E{definitional}
one.

We illustrate this with the natural numbers in \Isa/\FOL:

\begin{ascbox}
NatClass = FOL +\medskip
consts
  "0"           :: "'a"                                 ("0")
  Suc           :: "'a => 'a"
  rec           :: "['a, 'a, ['a, 'a] => 'a] => 'a"\medskip
axclass
  nat < term
  induct        "[| P(0); !!x. P(x) ==> P(Suc(x)) |] ==> P(n)"
  Suc_inject    "Suc(m) = Suc(n) ==> m = n"
  Suc_neq_0     "Suc(m) = 0 ==> R"
  rec_0         "rec(0, a, f) = a"
  rec_Suc       "rec(Suc(m), a, f) = f(m, rec(m, a, f))"\medskip
consts
  "+"           :: "['a::nat, 'a] => 'a"                (infixl 60)\medskip
defs
  add_def       "m + n == rec(m, n, %x y. Suc(y))"\medskip
end
\end{ascbox}

\TT{NatClass} is an abstract version of \TT{Nat}\footnote{See
  directory \TT{FOL/ex}.}. Basically, we have just replaced all
occurrences of \E{type} \TT{nat} by $\alpha$ and used the natural
number axioms to define \E{class} \TT{nat}\footnote{There's a snag:
  The original recursion operator \TT{rec} had to be made monomorphic,
  in a sense.}. Thus class \TT{nat} contains exactly those types
$\tau$ that are isomorphic to ``the'' natural numbers (with signature
\TT{0}, \TT{Suc}, \TT{rec}).

Furthermore, theory \TT{NatClass} defines an ``abstract constant'' $+$
based on class \TT{nat}.

\medskip

What we have done here can be also viewed as \E{type specification}.
Of course, it still remains open if there is some type at all that
meets the class axioms. Now a very nice property of axiomatic type
classes is, that abstract reasoning is always possible --- independent
of satisfiability. The meta-logic won't break, even if some class (or
general sort) turns out to be empty (``inconsistent'')
later\footnote{As a consequence of an old bug, this is \E{not} true
  for pre-\Isa94-2 versions.}.

For example, we may derive the following abstract natural numbers
theorems:

\begin{ascbox}
add_0:    0 + (?n::?'a::nat) = ?n
add_Suc:  Suc(?m::?'a::nat) + (?n::?'a::nat) = Suc(?m + ?n)
\end{ascbox}

See also file \TT{FOL/ex/NatClass.ML}. Note that this required only
some trivial modifications of the original \TT{Nat.ML}.


\section{The user interface of Isabelle's axclass package}

The actual axiomatic type class package of \Isa/\Pure\ mainly consists
of two new theory sections: \TT{axclass} and \TT{instance}.  Some
typical applications of these have already been demonstrated in
\secref{sec:ex}, below their syntax and semantics are presented more
completely.


\subsection{The axclass section}

Within theory files, \TT{axclass} introduces an axiomatic type class
definition. Its concrete syntax is:

\begin{matharray}{l}
  \TT{axclass} \\
  \ \ c \TT{ < } c_1\TT, \ldots\TT, c_n \\
  \ \ \idt{id}_1\ \idt{axm}_1 \\
  \ \ \vdots \\
  \ \ \idt{id}_m\ \idt{axm}_m
\end{matharray}

Where $c, c_1, \ldots, c_n$ are classes (category $\idt{id}$ or
$\idt{string}$) and $\idt{axm}_1, \ldots, \idt{axm}_m$ (with $m \ge
0$) are formulas (category $\idt{string}$).

Class $c$ has to be new, and sort $\{c_1, \ldots, c_n\}$ a subsort of
\TT{logic}. Each class axiom $\idt{axm}_j$ may contain any term
variables, but at most one type variable (which need not be the same
for all axioms). The sort of this type variable has to be a supersort
of $\{c_1, \ldots, c_n\}$.

\medskip

The \TT{axclass} section declares $c$ as subclass of $c_1, \ldots,
c_n$ to the type signature.

Furthermore, $\idt{axm}_1, \ldots, \idt{axm}_m$ are turned into the
``abstract axioms'' of $c$ with names $\idt{id}_1, \ldots,
\idt{id}_m$.  This is done by replacing all occurring type variables
by $\alpha :: c$. Original axioms that do not contain any type
variable will be prefixed by the logical precondition
$\TT{OFCLASS}(\alpha :: \TT{logic}, c\TT{\_class})$.

Another axiom of name $c\TT{I}$ --- the ``class $c$ introduction
rule'' --- is built from the respective universal closures of
$\idt{axm}_1, \ldots, \idt{axm}_m$ appropriately.


\subsection{The instance section}

Section \TT{instance} proves class inclusions or type arities at the
logical level and then transfers these into the type signature.

Its concrete syntax is:

\begin{matharray}{l}
  \TT{instance} \\
  \ \ [\ c_1 \TT{ < } c_2 \ |\
      t \TT{ ::\ (}\idt{sort}_1\TT, \ldots \TT, \idt{sort}_n\TT) \idt{sort}\ ] \\
  \ \ [\ \TT(\idt{name}_1 \TT, \ldots\TT, \idt{name}_m\TT)\ ] \\
  \ \ [\ \TT{\{|} \idt{text} \TT{|\}}\ ]
\end{matharray}

Where $c_1, c_2$ are classes and $t$ is an $n$-place type constructor
(all of category $\idt{id}$ or $\idt{string})$. Furthermore,
$\idt{sort}_i$ are sorts in the usual \Isa-syntax.

\medskip

Internally, \TT{instance} first sets up an appropriate goal that
expresses the class inclusion or type arity as a meta-proposition.
Then tactic \TT{AxClass.axclass\_tac} is applied with all preceding
meta-definitions of the current theory file and the user-supplied
witnesses. The latter are $\idt{name}_1, \ldots, \idt{name}_m$, where
$\idt{id}$ refers to an \ML-name of a theorem, and $\idt{string}$ to an
axiom of the current theory node\footnote{Thus, the user may reference
  axioms from above this \TT{instance} in the theory file. Note
  that new axioms appear at the \ML-toplevel only after the file is
  processed completely.}.

Tactic \TT{AxClass.axclass\_tac} first unfolds the class definition by
resolving with rule $c\TT{I}$, and then applies the witnesses
according to their form: Meta-definitions are unfolded, all other
formulas are repeatedly resolved\footnote{This is done in a way that
  enables proper object-\E{rules} to be used as witnesses for
  corresponding class axioms.} with.

The final optional argument $\idt{text}$ is \ML-code of an arbitrary
user tactic which is applied last to any remaining goals.

\medskip

Because of the complexity of \TT{instance}'s witnessing mechanisms,
new users of the axclass package are advised to only use the simple
form $\TT{instance}\ \ldots\ (\idt{id}_1, \ldots, \idt{id}_m)$, where
the identifiers refer to theorems that are appropriate type instances
of the class axioms. This typically requires an auxiliary theory,
though, which defines some constants and then proves these witnesses.


\begin{thebibliography}{}

\bibitem[Nipkow, 1993a]{Nipkow-slides} T. Nipkow. Axiomatic Type
  Classes (in Isabelle). Presentation at the workshop \E{Types for
    Proof and Programs}, Nijmegen, 1993.

\bibitem[Nipkow, 1993b]{Nipkow93} T. Nipkow. Order-Sorted Polymorphism
  in Isabelle. In G. Huet, G. Plotkin, editors, \E{Logical
    Environments}, pp.\ 164--188, Cambridge University Press, 1993.

\bibitem[Paulson, 1994]{Paulson94} L.C. Paulson. \E{Isabelle --- A
    Generic Theorem Prover}. LNCS 828, 1994.

\bibitem[Wenzel, 1994]{Wenzel94} M. Wenzel. \E{Axiomatische
    Typ-Klassen in Isabelle}. Diplom\-arbeit, TU M�nchen, 1994.

\end{thebibliography}

\end{document}
