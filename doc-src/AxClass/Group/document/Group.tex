%
\begin{isabellebody}%
\def\isabellecontext{Group}%
%
\isamarkupheader{Basic group theory%
}
\isamarkuptrue%
%
\isadelimtheory
%
\endisadelimtheory
%
\isatagtheory
\isacommand{theory}\isamarkupfalse%
\ Group\ \isakeyword{imports}\ Main\ \isakeyword{begin}%
\endisatagtheory
{\isafoldtheory}%
%
\isadelimtheory
%
\endisadelimtheory
%
\begin{isamarkuptext}%
\medskip\noindent The meta-level type system of Isabelle supports
  \emph{intersections} and \emph{inclusions} of type classes. These
  directly correspond to intersections and inclusions of type
  predicates in a purely set theoretic sense. This is sufficient as a
  means to describe simple hierarchies of structures.  As an
  illustration, we use the well-known example of semigroups, monoids,
  general groups and Abelian groups.%
\end{isamarkuptext}%
\isamarkuptrue%
%
\isamarkupsubsection{Monoids and Groups%
}
\isamarkuptrue%
%
\begin{isamarkuptext}%
First we declare some polymorphic constants required later for the
  signature parts of our structures.%
\end{isamarkuptext}%
\isamarkuptrue%
\isacommand{consts}\isamarkupfalse%
\isanewline
\ \ times\ {\isacharcolon}{\isacharcolon}\ {\isachardoublequoteopen}{\isacharprime}a\ {\isasymRightarrow}\ {\isacharprime}a\ {\isasymRightarrow}\ {\isacharprime}a{\isachardoublequoteclose}\ \ \ \ {\isacharparenleft}\isakeyword{infixl}\ {\isachardoublequoteopen}{\isasymodot}{\isachardoublequoteclose}\ {\isadigit{7}}{\isadigit{0}}{\isacharparenright}\isanewline
\ \ invers\ {\isacharcolon}{\isacharcolon}\ {\isachardoublequoteopen}{\isacharprime}a\ {\isasymRightarrow}\ {\isacharprime}a{\isachardoublequoteclose}\ \ \ \ {\isacharparenleft}{\isachardoublequoteopen}{\isacharparenleft}{\isacharunderscore}{\isasyminv}{\isacharparenright}{\isachardoublequoteclose}\ {\isacharbrackleft}{\isadigit{1}}{\isadigit{0}}{\isadigit{0}}{\isadigit{0}}{\isacharbrackright}\ {\isadigit{9}}{\isadigit{9}}{\isadigit{9}}{\isacharparenright}\isanewline
\ \ one\ {\isacharcolon}{\isacharcolon}\ {\isacharprime}a\ \ \ \ {\isacharparenleft}{\isachardoublequoteopen}{\isasymone}{\isachardoublequoteclose}{\isacharparenright}%
\begin{isamarkuptext}%
\noindent Next we define class \isa{monoid} of monoids with
  operations \isa{{\isasymodot}} and \isa{{\isasymone}}.  Note that multiple class
  axioms are allowed for user convenience --- they simply represent
  the conjunction of their respective universal closures.%
\end{isamarkuptext}%
\isamarkuptrue%
\isacommand{axclass}\isamarkupfalse%
\ monoid\ {\isasymsubseteq}\ type\isanewline
\ \ assoc{\isacharcolon}\ {\isachardoublequoteopen}{\isacharparenleft}x\ {\isasymodot}\ y{\isacharparenright}\ {\isasymodot}\ z\ {\isacharequal}\ x\ {\isasymodot}\ {\isacharparenleft}y\ {\isasymodot}\ z{\isacharparenright}{\isachardoublequoteclose}\isanewline
\ \ left{\isacharunderscore}unit{\isacharcolon}\ {\isachardoublequoteopen}{\isasymone}\ {\isasymodot}\ x\ {\isacharequal}\ x{\isachardoublequoteclose}\isanewline
\ \ right{\isacharunderscore}unit{\isacharcolon}\ {\isachardoublequoteopen}x\ {\isasymodot}\ {\isasymone}\ {\isacharequal}\ x{\isachardoublequoteclose}%
\begin{isamarkuptext}%
\noindent So class \isa{monoid} contains exactly those types
  \isa{{\isasymtau}} where \isa{{\isasymodot}\ {\isasymColon}\ {\isasymtau}\ {\isasymRightarrow}\ {\isasymtau}\ {\isasymRightarrow}\ {\isasymtau}} and \isa{{\isasymone}\ {\isasymColon}\ {\isasymtau}}
  are specified appropriately, such that \isa{{\isasymodot}} is associative and
  \isa{{\isasymone}} is a left and right unit element for the \isa{{\isasymodot}}
  operation.%
\end{isamarkuptext}%
\isamarkuptrue%
%
\begin{isamarkuptext}%
\medskip Independently of \isa{monoid}, we now define a linear
  hierarchy of semigroups, general groups and Abelian groups.  Note
  that the names of class axioms are automatically qualified with each
  class name, so we may re-use common names such as \isa{assoc}.%
\end{isamarkuptext}%
\isamarkuptrue%
\isacommand{axclass}\isamarkupfalse%
\ semigroup\ {\isasymsubseteq}\ type\isanewline
\ \ assoc{\isacharcolon}\ {\isachardoublequoteopen}{\isacharparenleft}x\ {\isasymodot}\ y{\isacharparenright}\ {\isasymodot}\ z\ {\isacharequal}\ x\ {\isasymodot}\ {\isacharparenleft}y\ {\isasymodot}\ z{\isacharparenright}{\isachardoublequoteclose}\isanewline
\isanewline
\isacommand{axclass}\isamarkupfalse%
\ group\ {\isasymsubseteq}\ semigroup\isanewline
\ \ left{\isacharunderscore}unit{\isacharcolon}\ {\isachardoublequoteopen}{\isasymone}\ {\isasymodot}\ x\ {\isacharequal}\ x{\isachardoublequoteclose}\isanewline
\ \ left{\isacharunderscore}inverse{\isacharcolon}\ {\isachardoublequoteopen}x{\isasyminv}\ {\isasymodot}\ x\ {\isacharequal}\ {\isasymone}{\isachardoublequoteclose}\isanewline
\isanewline
\isacommand{axclass}\isamarkupfalse%
\ agroup\ {\isasymsubseteq}\ group\isanewline
\ \ commute{\isacharcolon}\ {\isachardoublequoteopen}x\ {\isasymodot}\ y\ {\isacharequal}\ y\ {\isasymodot}\ x{\isachardoublequoteclose}%
\begin{isamarkuptext}%
\noindent Class \isa{group} inherits associativity of \isa{{\isasymodot}}
  from \isa{semigroup} and adds two further group axioms. Similarly,
  \isa{agroup} is defined as the subset of \isa{group} such that
  for all of its elements \isa{{\isasymtau}}, the operation \isa{{\isasymodot}\ {\isasymColon}\ {\isasymtau}\ {\isasymRightarrow}\ {\isasymtau}\ {\isasymRightarrow}\ {\isasymtau}} is even commutative.%
\end{isamarkuptext}%
\isamarkuptrue%
%
\isamarkupsubsection{Abstract reasoning%
}
\isamarkuptrue%
%
\begin{isamarkuptext}%
In a sense, axiomatic type classes may be viewed as \emph{abstract
  theories}.  Above class definitions gives rise to abstract axioms
  \isa{assoc}, \isa{left{\isacharunderscore}unit}, \isa{left{\isacharunderscore}inverse}, \isa{commute}, where any of these contain a type variable \isa{{\isacharprime}a\ {\isasymColon}\ c} that is restricted to types of the corresponding class \isa{c}.  \emph{Sort constraints} like this express a logical
  precondition for the whole formula.  For example, \isa{assoc}
  states that for all \isa{{\isasymtau}}, provided that \isa{{\isasymtau}\ {\isasymColon}\ semigroup}, the operation \isa{{\isasymodot}\ {\isasymColon}\ {\isasymtau}\ {\isasymRightarrow}\ {\isasymtau}\ {\isasymRightarrow}\ {\isasymtau}} is associative.

  \medskip From a technical point of view, abstract axioms are just
  ordinary Isabelle theorems, which may be used in proofs without
  special treatment.  Such ``abstract proofs'' usually yield new
  ``abstract theorems''.  For example, we may now derive the following
  well-known laws of general groups.%
\end{isamarkuptext}%
\isamarkuptrue%
\isacommand{theorem}\isamarkupfalse%
\ group{\isacharunderscore}right{\isacharunderscore}inverse{\isacharcolon}\ {\isachardoublequoteopen}x\ {\isasymodot}\ x{\isasyminv}\ {\isacharequal}\ {\isacharparenleft}{\isasymone}{\isasymColon}{\isacharprime}a{\isasymColon}group{\isacharparenright}{\isachardoublequoteclose}\isanewline
%
\isadelimproof
%
\endisadelimproof
%
\isatagproof
\isacommand{proof}\isamarkupfalse%
\ {\isacharminus}\isanewline
\ \ \isacommand{have}\isamarkupfalse%
\ {\isachardoublequoteopen}x\ {\isasymodot}\ x{\isasyminv}\ {\isacharequal}\ {\isasymone}\ {\isasymodot}\ {\isacharparenleft}x\ {\isasymodot}\ x{\isasyminv}{\isacharparenright}{\isachardoublequoteclose}\isanewline
\ \ \ \ \isacommand{by}\isamarkupfalse%
\ {\isacharparenleft}simp\ only{\isacharcolon}\ group{\isacharunderscore}class{\isachardot}left{\isacharunderscore}unit{\isacharparenright}\isanewline
\ \ \isacommand{also}\isamarkupfalse%
\ \isacommand{have}\isamarkupfalse%
\ {\isachardoublequoteopen}{\isachardot}{\isachardot}{\isachardot}\ {\isacharequal}\ {\isasymone}\ {\isasymodot}\ x\ {\isasymodot}\ x{\isasyminv}{\isachardoublequoteclose}\isanewline
\ \ \ \ \isacommand{by}\isamarkupfalse%
\ {\isacharparenleft}simp\ only{\isacharcolon}\ semigroup{\isacharunderscore}class{\isachardot}assoc{\isacharparenright}\isanewline
\ \ \isacommand{also}\isamarkupfalse%
\ \isacommand{have}\isamarkupfalse%
\ {\isachardoublequoteopen}{\isachardot}{\isachardot}{\isachardot}\ {\isacharequal}\ {\isacharparenleft}x{\isasyminv}{\isacharparenright}{\isasyminv}\ {\isasymodot}\ x{\isasyminv}\ {\isasymodot}\ x\ {\isasymodot}\ x{\isasyminv}{\isachardoublequoteclose}\isanewline
\ \ \ \ \isacommand{by}\isamarkupfalse%
\ {\isacharparenleft}simp\ only{\isacharcolon}\ group{\isacharunderscore}class{\isachardot}left{\isacharunderscore}inverse{\isacharparenright}\isanewline
\ \ \isacommand{also}\isamarkupfalse%
\ \isacommand{have}\isamarkupfalse%
\ {\isachardoublequoteopen}{\isachardot}{\isachardot}{\isachardot}\ {\isacharequal}\ {\isacharparenleft}x{\isasyminv}{\isacharparenright}{\isasyminv}\ {\isasymodot}\ {\isacharparenleft}x{\isasyminv}\ {\isasymodot}\ x{\isacharparenright}\ {\isasymodot}\ x{\isasyminv}{\isachardoublequoteclose}\isanewline
\ \ \ \ \isacommand{by}\isamarkupfalse%
\ {\isacharparenleft}simp\ only{\isacharcolon}\ semigroup{\isacharunderscore}class{\isachardot}assoc{\isacharparenright}\isanewline
\ \ \isacommand{also}\isamarkupfalse%
\ \isacommand{have}\isamarkupfalse%
\ {\isachardoublequoteopen}{\isachardot}{\isachardot}{\isachardot}\ {\isacharequal}\ {\isacharparenleft}x{\isasyminv}{\isacharparenright}{\isasyminv}\ {\isasymodot}\ {\isasymone}\ {\isasymodot}\ x{\isasyminv}{\isachardoublequoteclose}\isanewline
\ \ \ \ \isacommand{by}\isamarkupfalse%
\ {\isacharparenleft}simp\ only{\isacharcolon}\ group{\isacharunderscore}class{\isachardot}left{\isacharunderscore}inverse{\isacharparenright}\isanewline
\ \ \isacommand{also}\isamarkupfalse%
\ \isacommand{have}\isamarkupfalse%
\ {\isachardoublequoteopen}{\isachardot}{\isachardot}{\isachardot}\ {\isacharequal}\ {\isacharparenleft}x{\isasyminv}{\isacharparenright}{\isasyminv}\ {\isasymodot}\ {\isacharparenleft}{\isasymone}\ {\isasymodot}\ x{\isasyminv}{\isacharparenright}{\isachardoublequoteclose}\isanewline
\ \ \ \ \isacommand{by}\isamarkupfalse%
\ {\isacharparenleft}simp\ only{\isacharcolon}\ semigroup{\isacharunderscore}class{\isachardot}assoc{\isacharparenright}\isanewline
\ \ \isacommand{also}\isamarkupfalse%
\ \isacommand{have}\isamarkupfalse%
\ {\isachardoublequoteopen}{\isachardot}{\isachardot}{\isachardot}\ {\isacharequal}\ {\isacharparenleft}x{\isasyminv}{\isacharparenright}{\isasyminv}\ {\isasymodot}\ x{\isasyminv}{\isachardoublequoteclose}\isanewline
\ \ \ \ \isacommand{by}\isamarkupfalse%
\ {\isacharparenleft}simp\ only{\isacharcolon}\ group{\isacharunderscore}class{\isachardot}left{\isacharunderscore}unit{\isacharparenright}\isanewline
\ \ \isacommand{also}\isamarkupfalse%
\ \isacommand{have}\isamarkupfalse%
\ {\isachardoublequoteopen}{\isachardot}{\isachardot}{\isachardot}\ {\isacharequal}\ {\isasymone}{\isachardoublequoteclose}\isanewline
\ \ \ \ \isacommand{by}\isamarkupfalse%
\ {\isacharparenleft}simp\ only{\isacharcolon}\ group{\isacharunderscore}class{\isachardot}left{\isacharunderscore}inverse{\isacharparenright}\isanewline
\ \ \isacommand{finally}\isamarkupfalse%
\ \isacommand{show}\isamarkupfalse%
\ {\isacharquery}thesis\ \isacommand{{\isachardot}}\isamarkupfalse%
\isanewline
\isacommand{qed}\isamarkupfalse%
%
\endisatagproof
{\isafoldproof}%
%
\isadelimproof
%
\endisadelimproof
%
\begin{isamarkuptext}%
\noindent With \isa{group{\isacharunderscore}right{\isacharunderscore}inverse} already available, \isa{group{\isacharunderscore}right{\isacharunderscore}unit}\label{thm:group-right-unit} is now established
  much easier.%
\end{isamarkuptext}%
\isamarkuptrue%
\isacommand{theorem}\isamarkupfalse%
\ group{\isacharunderscore}right{\isacharunderscore}unit{\isacharcolon}\ {\isachardoublequoteopen}x\ {\isasymodot}\ {\isasymone}\ {\isacharequal}\ {\isacharparenleft}x{\isasymColon}{\isacharprime}a{\isasymColon}group{\isacharparenright}{\isachardoublequoteclose}\isanewline
%
\isadelimproof
%
\endisadelimproof
%
\isatagproof
\isacommand{proof}\isamarkupfalse%
\ {\isacharminus}\isanewline
\ \ \isacommand{have}\isamarkupfalse%
\ {\isachardoublequoteopen}x\ {\isasymodot}\ {\isasymone}\ {\isacharequal}\ x\ {\isasymodot}\ {\isacharparenleft}x{\isasyminv}\ {\isasymodot}\ x{\isacharparenright}{\isachardoublequoteclose}\isanewline
\ \ \ \ \isacommand{by}\isamarkupfalse%
\ {\isacharparenleft}simp\ only{\isacharcolon}\ group{\isacharunderscore}class{\isachardot}left{\isacharunderscore}inverse{\isacharparenright}\isanewline
\ \ \isacommand{also}\isamarkupfalse%
\ \isacommand{have}\isamarkupfalse%
\ {\isachardoublequoteopen}{\isachardot}{\isachardot}{\isachardot}\ {\isacharequal}\ x\ {\isasymodot}\ x{\isasyminv}\ {\isasymodot}\ x{\isachardoublequoteclose}\isanewline
\ \ \ \ \isacommand{by}\isamarkupfalse%
\ {\isacharparenleft}simp\ only{\isacharcolon}\ semigroup{\isacharunderscore}class{\isachardot}assoc{\isacharparenright}\isanewline
\ \ \isacommand{also}\isamarkupfalse%
\ \isacommand{have}\isamarkupfalse%
\ {\isachardoublequoteopen}{\isachardot}{\isachardot}{\isachardot}\ {\isacharequal}\ {\isasymone}\ {\isasymodot}\ x{\isachardoublequoteclose}\isanewline
\ \ \ \ \isacommand{by}\isamarkupfalse%
\ {\isacharparenleft}simp\ only{\isacharcolon}\ group{\isacharunderscore}right{\isacharunderscore}inverse{\isacharparenright}\isanewline
\ \ \isacommand{also}\isamarkupfalse%
\ \isacommand{have}\isamarkupfalse%
\ {\isachardoublequoteopen}{\isachardot}{\isachardot}{\isachardot}\ {\isacharequal}\ x{\isachardoublequoteclose}\isanewline
\ \ \ \ \isacommand{by}\isamarkupfalse%
\ {\isacharparenleft}simp\ only{\isacharcolon}\ group{\isacharunderscore}class{\isachardot}left{\isacharunderscore}unit{\isacharparenright}\isanewline
\ \ \isacommand{finally}\isamarkupfalse%
\ \isacommand{show}\isamarkupfalse%
\ {\isacharquery}thesis\ \isacommand{{\isachardot}}\isamarkupfalse%
\isanewline
\isacommand{qed}\isamarkupfalse%
%
\endisatagproof
{\isafoldproof}%
%
\isadelimproof
%
\endisadelimproof
%
\begin{isamarkuptext}%
\medskip Abstract theorems may be instantiated to only those types
  \isa{{\isasymtau}} where the appropriate class membership \isa{{\isasymtau}\ {\isasymColon}\ c} is
  known at Isabelle's type signature level.  Since we have \isa{agroup\ {\isasymsubseteq}\ group\ {\isasymsubseteq}\ semigroup} by definition, all theorems of \isa{semigroup} and \isa{group} are automatically inherited by \isa{group} and \isa{agroup}.%
\end{isamarkuptext}%
\isamarkuptrue%
%
\isamarkupsubsection{Abstract instantiation%
}
\isamarkuptrue%
%
\begin{isamarkuptext}%
From the definition, the \isa{monoid} and \isa{group} classes
  have been independent.  Note that for monoids, \isa{right{\isacharunderscore}unit}
  had to be included as an axiom, but for groups both \isa{right{\isacharunderscore}unit} and \isa{right{\isacharunderscore}inverse} are derivable from the other
  axioms.  With \isa{group{\isacharunderscore}right{\isacharunderscore}unit} derived as a theorem of group
  theory (see page~\pageref{thm:group-right-unit}), we may now
  instantiate \isa{monoid\ {\isasymsubseteq}\ semigroup} and \isa{group\ {\isasymsubseteq}\ monoid} properly as follows (cf.\ \figref{fig:monoid-group}).

 \begin{figure}[htbp]
   \begin{center}
     \small
     \unitlength 0.6mm
     \begin{picture}(65,90)(0,-10)
       \put(15,10){\line(0,1){10}} \put(15,30){\line(0,1){10}}
       \put(15,50){\line(1,1){10}} \put(35,60){\line(1,-1){10}}
       \put(15,5){\makebox(0,0){\isa{agroup}}}
       \put(15,25){\makebox(0,0){\isa{group}}}
       \put(15,45){\makebox(0,0){\isa{semigroup}}}
       \put(30,65){\makebox(0,0){\isa{type}}} \put(50,45){\makebox(0,0){\isa{monoid}}}
     \end{picture}
     \hspace{4em}
     \begin{picture}(30,90)(0,0)
       \put(15,10){\line(0,1){10}} \put(15,30){\line(0,1){10}}
       \put(15,50){\line(0,1){10}} \put(15,70){\line(0,1){10}}
       \put(15,5){\makebox(0,0){\isa{agroup}}}
       \put(15,25){\makebox(0,0){\isa{group}}}
       \put(15,45){\makebox(0,0){\isa{monoid}}}
       \put(15,65){\makebox(0,0){\isa{semigroup}}}
       \put(15,85){\makebox(0,0){\isa{type}}}
     \end{picture}
     \caption{Monoids and groups: according to definition, and by proof}
     \label{fig:monoid-group}
   \end{center}
 \end{figure}%
\end{isamarkuptext}%
\isamarkuptrue%
\isacommand{instance}\isamarkupfalse%
\ monoid\ {\isasymsubseteq}\ semigroup\isanewline
%
\isadelimproof
%
\endisadelimproof
%
\isatagproof
\isacommand{proof}\isamarkupfalse%
\isanewline
\ \ \isacommand{fix}\isamarkupfalse%
\ x\ y\ z\ {\isacharcolon}{\isacharcolon}\ {\isachardoublequoteopen}{\isacharprime}a{\isasymColon}monoid{\isachardoublequoteclose}\isanewline
\ \ \isacommand{show}\isamarkupfalse%
\ {\isachardoublequoteopen}x\ {\isasymodot}\ y\ {\isasymodot}\ z\ {\isacharequal}\ x\ {\isasymodot}\ {\isacharparenleft}y\ {\isasymodot}\ z{\isacharparenright}{\isachardoublequoteclose}\isanewline
\ \ \ \ \isacommand{by}\isamarkupfalse%
\ {\isacharparenleft}rule\ monoid{\isacharunderscore}class{\isachardot}assoc{\isacharparenright}\isanewline
\isacommand{qed}\isamarkupfalse%
%
\endisatagproof
{\isafoldproof}%
%
\isadelimproof
\isanewline
%
\endisadelimproof
\isanewline
\isacommand{instance}\isamarkupfalse%
\ group\ {\isasymsubseteq}\ monoid\isanewline
%
\isadelimproof
%
\endisadelimproof
%
\isatagproof
\isacommand{proof}\isamarkupfalse%
\isanewline
\ \ \isacommand{fix}\isamarkupfalse%
\ x\ y\ z\ {\isacharcolon}{\isacharcolon}\ {\isachardoublequoteopen}{\isacharprime}a{\isasymColon}group{\isachardoublequoteclose}\isanewline
\ \ \isacommand{show}\isamarkupfalse%
\ {\isachardoublequoteopen}x\ {\isasymodot}\ y\ {\isasymodot}\ z\ {\isacharequal}\ x\ {\isasymodot}\ {\isacharparenleft}y\ {\isasymodot}\ z{\isacharparenright}{\isachardoublequoteclose}\isanewline
\ \ \ \ \isacommand{by}\isamarkupfalse%
\ {\isacharparenleft}rule\ semigroup{\isacharunderscore}class{\isachardot}assoc{\isacharparenright}\isanewline
\ \ \isacommand{show}\isamarkupfalse%
\ {\isachardoublequoteopen}{\isasymone}\ {\isasymodot}\ x\ {\isacharequal}\ x{\isachardoublequoteclose}\isanewline
\ \ \ \ \isacommand{by}\isamarkupfalse%
\ {\isacharparenleft}rule\ group{\isacharunderscore}class{\isachardot}left{\isacharunderscore}unit{\isacharparenright}\isanewline
\ \ \isacommand{show}\isamarkupfalse%
\ {\isachardoublequoteopen}x\ {\isasymodot}\ {\isasymone}\ {\isacharequal}\ x{\isachardoublequoteclose}\isanewline
\ \ \ \ \isacommand{by}\isamarkupfalse%
\ {\isacharparenleft}rule\ group{\isacharunderscore}right{\isacharunderscore}unit{\isacharparenright}\isanewline
\isacommand{qed}\isamarkupfalse%
%
\endisatagproof
{\isafoldproof}%
%
\isadelimproof
%
\endisadelimproof
%
\begin{isamarkuptext}%
\medskip The $\INSTANCE$ command sets up an appropriate goal that
  represents the class inclusion (or type arity, see
  \secref{sec:inst-arity}) to be proven (see also
  \cite{isabelle-isar-ref}).  The initial proof step causes
  back-chaining of class membership statements wrt.\ the hierarchy of
  any classes defined in the current theory; the effect is to reduce
  to the initial statement to a number of goals that directly
  correspond to any class axioms encountered on the path upwards
  through the class hierarchy.%
\end{isamarkuptext}%
\isamarkuptrue%
%
\isamarkupsubsection{Concrete instantiation \label{sec:inst-arity}%
}
\isamarkuptrue%
%
\begin{isamarkuptext}%
So far we have covered the case of the form $\INSTANCE$~\isa{c\isactrlsub {\isadigit{1}}\ {\isasymsubseteq}\ c\isactrlsub {\isadigit{2}}}, namely \emph{abstract instantiation} ---
  $c@1$ is more special than \isa{c\isactrlsub {\isadigit{1}}} and thus an instance
  of \isa{c\isactrlsub {\isadigit{2}}}.  Even more interesting for practical
  applications are \emph{concrete instantiations} of axiomatic type
  classes.  That is, certain simple schemes \isa{{\isacharparenleft}{\isasymalpha}\isactrlsub {\isadigit{1}}{\isacharcomma}\ {\isasymdots}{\isacharcomma}\ {\isasymalpha}\isactrlsub n{\isacharparenright}\ t\ {\isasymColon}\ c} of class membership may be established at the
  logical level and then transferred to Isabelle's type signature
  level.

  \medskip As a typical example, we show that type \isa{bool} with
  exclusive-or as \isa{{\isasymodot}} operation, identity as \isa{{\isasyminv}}, and
  \isa{False} as \isa{{\isasymone}} forms an Abelian group.%
\end{isamarkuptext}%
\isamarkuptrue%
\isacommand{defs}\isamarkupfalse%
\ {\isacharparenleft}\isakeyword{overloaded}{\isacharparenright}\isanewline
\ \ times{\isacharunderscore}bool{\isacharunderscore}def{\isacharcolon}\ {\isachardoublequoteopen}x\ {\isasymodot}\ y\ {\isasymequiv}\ x\ {\isasymnoteq}\ {\isacharparenleft}y{\isasymColon}bool{\isacharparenright}{\isachardoublequoteclose}\isanewline
\ \ inverse{\isacharunderscore}bool{\isacharunderscore}def{\isacharcolon}\ {\isachardoublequoteopen}x{\isasyminv}\ {\isasymequiv}\ x{\isasymColon}bool{\isachardoublequoteclose}\isanewline
\ \ unit{\isacharunderscore}bool{\isacharunderscore}def{\isacharcolon}\ {\isachardoublequoteopen}{\isasymone}\ {\isasymequiv}\ False{\isachardoublequoteclose}%
\begin{isamarkuptext}%
\medskip It is important to note that above $\DEFS$ are just
  overloaded meta-level constant definitions, where type classes are
  not yet involved at all.  This form of constant definition with
  overloading (and optional recursion over the syntactic structure of
  simple types) are admissible as definitional extensions of plain HOL
  \cite{Wenzel:1997:TPHOL}.  The Haskell-style type system is not
  required for overloading.  Nevertheless, overloaded definitions are
  best applied in the context of type classes.

  \medskip Since we have chosen above $\DEFS$ of the generic group
  operations on type \isa{bool} appropriately, the class membership
  \isa{bool\ {\isasymColon}\ agroup} may be now derived as follows.%
\end{isamarkuptext}%
\isamarkuptrue%
\isacommand{instance}\isamarkupfalse%
\ bool\ {\isacharcolon}{\isacharcolon}\ agroup\isanewline
%
\isadelimproof
%
\endisadelimproof
%
\isatagproof
\isacommand{proof}\isamarkupfalse%
\ {\isacharparenleft}intro{\isacharunderscore}classes{\isacharcomma}\isanewline
\ \ \ \ unfold\ times{\isacharunderscore}bool{\isacharunderscore}def\ inverse{\isacharunderscore}bool{\isacharunderscore}def\ unit{\isacharunderscore}bool{\isacharunderscore}def{\isacharparenright}\isanewline
\ \ \isacommand{fix}\isamarkupfalse%
\ x\ y\ z\isanewline
\ \ \isacommand{show}\isamarkupfalse%
\ {\isachardoublequoteopen}{\isacharparenleft}{\isacharparenleft}x\ {\isasymnoteq}\ y{\isacharparenright}\ {\isasymnoteq}\ z{\isacharparenright}\ {\isacharequal}\ {\isacharparenleft}x\ {\isasymnoteq}\ {\isacharparenleft}y\ {\isasymnoteq}\ z{\isacharparenright}{\isacharparenright}{\isachardoublequoteclose}\ \isacommand{by}\isamarkupfalse%
\ blast\isanewline
\ \ \isacommand{show}\isamarkupfalse%
\ {\isachardoublequoteopen}{\isacharparenleft}False\ {\isasymnoteq}\ x{\isacharparenright}\ {\isacharequal}\ x{\isachardoublequoteclose}\ \isacommand{by}\isamarkupfalse%
\ blast\isanewline
\ \ \isacommand{show}\isamarkupfalse%
\ {\isachardoublequoteopen}{\isacharparenleft}x\ {\isasymnoteq}\ x{\isacharparenright}\ {\isacharequal}\ False{\isachardoublequoteclose}\ \isacommand{by}\isamarkupfalse%
\ blast\isanewline
\ \ \isacommand{show}\isamarkupfalse%
\ {\isachardoublequoteopen}{\isacharparenleft}x\ {\isasymnoteq}\ y{\isacharparenright}\ {\isacharequal}\ {\isacharparenleft}y\ {\isasymnoteq}\ x{\isacharparenright}{\isachardoublequoteclose}\ \isacommand{by}\isamarkupfalse%
\ blast\isanewline
\isacommand{qed}\isamarkupfalse%
%
\endisatagproof
{\isafoldproof}%
%
\isadelimproof
%
\endisadelimproof
%
\begin{isamarkuptext}%
The result of an $\INSTANCE$ statement is both expressed as a
  theorem of Isabelle's meta-logic, and as a type arity of the type
  signature.  The latter enables type-inference system to take care of
  this new instance automatically.

  \medskip We could now also instantiate our group theory classes to
  many other concrete types.  For example, \isa{int\ {\isasymColon}\ agroup}
  (e.g.\ by defining \isa{{\isasymodot}} as addition, \isa{{\isasyminv}} as negation
  and \isa{{\isasymone}} as zero) or \isa{list\ {\isasymColon}\ {\isacharparenleft}type{\isacharparenright}\ semigroup}
  (e.g.\ if \isa{{\isasymodot}} is defined as list append).  Thus, the
  characteristic constants \isa{{\isasymodot}}, \isa{{\isasyminv}}, \isa{{\isasymone}}
  really become overloaded, i.e.\ have different meanings on different
  types.%
\end{isamarkuptext}%
\isamarkuptrue%
%
\isamarkupsubsection{Lifting and Functors%
}
\isamarkuptrue%
%
\begin{isamarkuptext}%
As already mentioned above, overloading in the simply-typed HOL
  systems may include recursion over the syntactic structure of types.
  That is, definitional equations \isa{c\isactrlsup {\isasymtau}\ {\isasymequiv}\ t} may also
  contain constants of name \isa{c} on the right-hand side --- if
  these have types that are structurally simpler than \isa{{\isasymtau}}.

  This feature enables us to \emph{lift operations}, say to Cartesian
  products, direct sums or function spaces.  Subsequently we lift
  \isa{{\isasymodot}} component-wise to binary products \isa{{\isacharprime}a\ {\isasymtimes}\ {\isacharprime}b}.%
\end{isamarkuptext}%
\isamarkuptrue%
\isacommand{defs}\isamarkupfalse%
\ {\isacharparenleft}\isakeyword{overloaded}{\isacharparenright}\isanewline
\ \ times{\isacharunderscore}prod{\isacharunderscore}def{\isacharcolon}\ {\isachardoublequoteopen}p\ {\isasymodot}\ q\ {\isasymequiv}\ {\isacharparenleft}fst\ p\ {\isasymodot}\ fst\ q{\isacharcomma}\ snd\ p\ {\isasymodot}\ snd\ q{\isacharparenright}{\isachardoublequoteclose}%
\begin{isamarkuptext}%
It is very easy to see that associativity of \isa{{\isasymodot}} on \isa{{\isacharprime}a}
  and \isa{{\isasymodot}} on \isa{{\isacharprime}b} transfers to \isa{{\isasymodot}} on \isa{{\isacharprime}a\ {\isasymtimes}\ {\isacharprime}b}.  Hence the binary type constructor \isa{{\isasymodot}} maps semigroups
  to semigroups.  This may be established formally as follows.%
\end{isamarkuptext}%
\isamarkuptrue%
\isacommand{instance}\isamarkupfalse%
\ {\isacharasterisk}\ {\isacharcolon}{\isacharcolon}\ {\isacharparenleft}semigroup{\isacharcomma}\ semigroup{\isacharparenright}\ semigroup\isanewline
%
\isadelimproof
%
\endisadelimproof
%
\isatagproof
\isacommand{proof}\isamarkupfalse%
\ {\isacharparenleft}intro{\isacharunderscore}classes{\isacharcomma}\ unfold\ times{\isacharunderscore}prod{\isacharunderscore}def{\isacharparenright}\isanewline
\ \ \isacommand{fix}\isamarkupfalse%
\ p\ q\ r\ {\isacharcolon}{\isacharcolon}\ {\isachardoublequoteopen}{\isacharprime}a{\isasymColon}semigroup\ {\isasymtimes}\ {\isacharprime}b{\isasymColon}semigroup{\isachardoublequoteclose}\isanewline
\ \ \isacommand{show}\isamarkupfalse%
\isanewline
\ \ \ \ {\isachardoublequoteopen}{\isacharparenleft}fst\ {\isacharparenleft}fst\ p\ {\isasymodot}\ fst\ q{\isacharcomma}\ snd\ p\ {\isasymodot}\ snd\ q{\isacharparenright}\ {\isasymodot}\ fst\ r{\isacharcomma}\isanewline
\ \ \ \ \ \ snd\ {\isacharparenleft}fst\ p\ {\isasymodot}\ fst\ q{\isacharcomma}\ snd\ p\ {\isasymodot}\ snd\ q{\isacharparenright}\ {\isasymodot}\ snd\ r{\isacharparenright}\ {\isacharequal}\isanewline
\ \ \ \ \ \ \ {\isacharparenleft}fst\ p\ {\isasymodot}\ fst\ {\isacharparenleft}fst\ q\ {\isasymodot}\ fst\ r{\isacharcomma}\ snd\ q\ {\isasymodot}\ snd\ r{\isacharparenright}{\isacharcomma}\isanewline
\ \ \ \ \ \ \ \ snd\ p\ {\isasymodot}\ snd\ {\isacharparenleft}fst\ q\ {\isasymodot}\ fst\ r{\isacharcomma}\ snd\ q\ {\isasymodot}\ snd\ r{\isacharparenright}{\isacharparenright}{\isachardoublequoteclose}\isanewline
\ \ \ \ \isacommand{by}\isamarkupfalse%
\ {\isacharparenleft}simp\ add{\isacharcolon}\ semigroup{\isacharunderscore}class{\isachardot}assoc{\isacharparenright}\isanewline
\isacommand{qed}\isamarkupfalse%
%
\endisatagproof
{\isafoldproof}%
%
\isadelimproof
%
\endisadelimproof
%
\begin{isamarkuptext}%
Thus, if we view class instances as ``structures'', then overloaded
  constant definitions with recursion over types indirectly provide
  some kind of ``functors'' --- i.e.\ mappings between abstract
  theories.%
\end{isamarkuptext}%
\isamarkuptrue%
%
\isadelimtheory
%
\endisadelimtheory
%
\isatagtheory
\isacommand{end}\isamarkupfalse%
%
\endisatagtheory
{\isafoldtheory}%
%
\isadelimtheory
%
\endisadelimtheory
\isanewline
\end{isabellebody}%
%%% Local Variables:
%%% mode: latex
%%% TeX-master: "root"
%%% End:
