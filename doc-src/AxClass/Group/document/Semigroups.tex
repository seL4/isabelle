%
\begin{isabellebody}%
\def\isabellecontext{Semigroups}%
\isamarkupfalse%
%
\isamarkupheader{Semigroups%
}
\isamarkuptrue%
%
\isadelimtheory
%
\endisadelimtheory
%
\isatagtheory
\isacommand{theory}\isamarkupfalse%
\ Semigroups\ \isakeyword{imports}\ Main\ \isakeyword{begin}%
\endisatagtheory
{\isafoldtheory}%
%
\isadelimtheory
%
\endisadelimtheory
%
\begin{isamarkuptext}%
\medskip\noindent An axiomatic type class is simply a class of types
  that all meet certain properties, which are also called \emph{class
  axioms}. Thus, type classes may be also understood as type
  predicates --- i.e.\ abstractions over a single type argument \isa{{\isacharprime}a}.  Class axioms typically contain polymorphic constants that
  depend on this type \isa{{\isacharprime}a}.  These \emph{characteristic
  constants} behave like operations associated with the ``carrier''
  type \isa{{\isacharprime}a}.

  We illustrate these basic concepts by the following formulation of
  semigroups.%
\end{isamarkuptext}%
\isamarkuptrue%
\isacommand{consts}\isamarkupfalse%
\isanewline
\ \ times\ {\isacharcolon}{\isacharcolon}\ {\isachardoublequoteopen}{\isacharprime}a\ {\isasymRightarrow}\ {\isacharprime}a\ {\isasymRightarrow}\ {\isacharprime}a{\isachardoublequoteclose}\ \ \ \ {\isacharparenleft}\isakeyword{infixl}\ {\isachardoublequoteopen}{\isasymodot}{\isachardoublequoteclose}\ {\isadigit{7}}{\isadigit{0}}{\isacharparenright}\isanewline
\isacommand{axclass}\isamarkupfalse%
\ semigroup\ {\isasymsubseteq}\ type\isanewline
\ \ assoc{\isacharcolon}\ {\isachardoublequoteopen}{\isacharparenleft}x\ {\isasymodot}\ y{\isacharparenright}\ {\isasymodot}\ z\ {\isacharequal}\ x\ {\isasymodot}\ {\isacharparenleft}y\ {\isasymodot}\ z{\isacharparenright}{\isachardoublequoteclose}%
\begin{isamarkuptext}%
\noindent Above we have first declared a polymorphic constant \isa{{\isasymodot}\ {\isasymColon}\ {\isacharprime}a\ {\isasymRightarrow}\ {\isacharprime}a\ {\isasymRightarrow}\ {\isacharprime}a} and then defined the class \isa{semigroup} of
  all types \isa{{\isasymtau}} such that \isa{{\isasymodot}\ {\isasymColon}\ {\isasymtau}\ {\isasymRightarrow}\ {\isasymtau}\ {\isasymRightarrow}\ {\isasymtau}} is indeed an
  associative operator.  The \isa{assoc} axiom contains exactly one
  type variable, which is invisible in the above presentation, though.
  Also note that free term variables (like \isa{x}, \isa{y},
  \isa{z}) are allowed for user convenience --- conceptually all of
  these are bound by outermost universal quantifiers.

  \medskip In general, type classes may be used to describe
  \emph{structures} with exactly one carrier \isa{{\isacharprime}a} and a fixed
  \emph{signature}.  Different signatures require different classes.
  Below, class \isa{plus{\isacharunderscore}semigroup} represents semigroups \isa{{\isacharparenleft}{\isasymtau}{\isacharcomma}\ {\isasymoplus}\isactrlsup {\isasymtau}{\isacharparenright}}, while the original \isa{semigroup} would
  correspond to semigroups of the form \isa{{\isacharparenleft}{\isasymtau}{\isacharcomma}\ {\isasymodot}\isactrlsup {\isasymtau}{\isacharparenright}}.%
\end{isamarkuptext}%
\isamarkuptrue%
\isacommand{consts}\isamarkupfalse%
\isanewline
\ \ plus\ {\isacharcolon}{\isacharcolon}\ {\isachardoublequoteopen}{\isacharprime}a\ {\isasymRightarrow}\ {\isacharprime}a\ {\isasymRightarrow}\ {\isacharprime}a{\isachardoublequoteclose}\ \ \ \ {\isacharparenleft}\isakeyword{infixl}\ {\isachardoublequoteopen}{\isasymoplus}{\isachardoublequoteclose}\ {\isadigit{7}}{\isadigit{0}}{\isacharparenright}\isanewline
\isacommand{axclass}\isamarkupfalse%
\ plus{\isacharunderscore}semigroup\ {\isasymsubseteq}\ type\isanewline
\ \ assoc{\isacharcolon}\ {\isachardoublequoteopen}{\isacharparenleft}x\ {\isasymoplus}\ y{\isacharparenright}\ {\isasymoplus}\ z\ {\isacharequal}\ x\ {\isasymoplus}\ {\isacharparenleft}y\ {\isasymoplus}\ z{\isacharparenright}{\isachardoublequoteclose}%
\begin{isamarkuptext}%
\noindent Even if classes \isa{plus{\isacharunderscore}semigroup} and \isa{semigroup} both represent semigroups in a sense, they are certainly
  not quite the same.%
\end{isamarkuptext}%
\isamarkuptrue%
%
\isadelimtheory
%
\endisadelimtheory
%
\isatagtheory
\isacommand{end}\isamarkupfalse%
%
\endisatagtheory
{\isafoldtheory}%
%
\isadelimtheory
%
\endisadelimtheory
\isanewline
\end{isabellebody}%
%%% Local Variables:
%%% mode: latex
%%% TeX-master: "root"
%%% End:
