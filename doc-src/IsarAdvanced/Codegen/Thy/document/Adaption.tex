%
\begin{isabellebody}%
\def\isabellecontext{Adaption}%
%
\isadelimtheory
%
\endisadelimtheory
%
\isatagtheory
\isacommand{theory}\isamarkupfalse%
\ Adaption\isanewline
\isakeyword{imports}\ Setup\isanewline
\isakeyword{begin}%
\endisatagtheory
{\isafoldtheory}%
%
\isadelimtheory
%
\endisadelimtheory
%
\isamarkupsection{Adaption to target languages \label{sec:adaption}%
}
\isamarkuptrue%
%
\isamarkupsubsection{Common adaption cases%
}
\isamarkuptrue%
%
\begin{isamarkuptext}%
The \hyperlink{theory.HOL}{\mbox{\isa{HOL}}} \hyperlink{theory.Main}{\mbox{\isa{Main}}} theory already provides a code
  generator setup
  which should be suitable for most applications. Common extensions
  and modifications are available by certain theories of the \isa{HOL}
  library; beside being useful in applications, they may serve
  as a tutorial for customising the code generator setup (see below
  \secref{sec:adaption_mechanisms}).

  \begin{description}

    \item[\hyperlink{theory.Code-Integer}{\mbox{\isa{Code{\isacharunderscore}Integer}}}] represents \isa{HOL} integers by big
       integer literals in target languages.
    \item[\hyperlink{theory.Code-Char}{\mbox{\isa{Code{\isacharunderscore}Char}}}] represents \isa{HOL} characters by 
       character literals in target languages.
    \item[\hyperlink{theory.Code-Char-chr}{\mbox{\isa{Code{\isacharunderscore}Char{\isacharunderscore}chr}}}] like \isa{Code{\isacharunderscore}Char},
       but also offers treatment of character codes; includes
       \hyperlink{theory.Code-Char-chr}{\mbox{\isa{Code{\isacharunderscore}Char{\isacharunderscore}chr}}}.
    \item[\hyperlink{theory.Efficient-Nat}{\mbox{\isa{Efficient{\isacharunderscore}Nat}}}] \label{eff_nat} implements natural numbers by integers,
       which in general will result in higher efficiency; pattern
       matching with \isa{{\isadigit{0}}} / \isa{Suc}
       is eliminated;  includes \hyperlink{theory.Code-Integer}{\mbox{\isa{Code{\isacharunderscore}Integer}}}.
    \item[\hyperlink{theory.Code-Index}{\mbox{\isa{Code{\isacharunderscore}Index}}}] provides an additional datatype
       \isa{index} which is mapped to target-language built-in integers.
       Useful for code setups which involve e.g. indexing of
       target-language arrays.
    \item[\hyperlink{theory.Code-Message}{\mbox{\isa{Code{\isacharunderscore}Message}}}] provides an additional datatype
       \isa{message{\isacharunderscore}string} which is isomorphic to strings;
       \isa{message{\isacharunderscore}string}s are mapped to target-language strings.
       Useful for code setups which involve e.g. printing (error) messages.

  \end{description}

  \begin{warn}
    When importing any of these theories, they should form the last
    items in an import list.  Since these theories adapt the
    code generator setup in a non-conservative fashion,
    strange effects may occur otherwise.
  \end{warn}%
\end{isamarkuptext}%
\isamarkuptrue%
%
\isamarkupsubsection{Adaption mechanisms \label{sec:adaption_mechanisms}%
}
\isamarkuptrue%
%
\begin{isamarkuptext}%
\begin{warn}
    The mechanisms shown here are especially for the curious;  the user
    rarely needs to do anything on his own beyond the defaults in \isa{HOL}.
    Adaption is a delicated task which requires a lot of dilligence since
    it happend \emph{completely} outside the logic.
  \end{warn}%
\end{isamarkuptext}%
\isamarkuptrue%
%
\begin{isamarkuptext}%
\noindent Consider the following function and its corresponding
  SML code:%
\end{isamarkuptext}%
\isamarkuptrue%
%
\isadelimquoteme
%
\endisadelimquoteme
%
\isatagquoteme
\isacommand{primrec}\isamarkupfalse%
\ in{\isacharunderscore}interval\ {\isacharcolon}{\isacharcolon}\ {\isachardoublequoteopen}nat\ {\isasymtimes}\ nat\ {\isasymRightarrow}\ nat\ {\isasymRightarrow}\ bool{\isachardoublequoteclose}\ \isakeyword{where}\isanewline
\ \ {\isachardoublequoteopen}in{\isacharunderscore}interval\ {\isacharparenleft}k{\isacharcomma}\ l{\isacharparenright}\ n\ {\isasymlongleftrightarrow}\ k\ {\isasymle}\ n\ {\isasymand}\ n\ {\isasymle}\ l{\isachardoublequoteclose}%
\endisatagquoteme
{\isafoldquoteme}%
%
\isadelimquoteme
%
\endisadelimquoteme
%
\isadeliminvisible
%
\endisadeliminvisible
%
\isataginvisible
%
\endisataginvisible
{\isafoldinvisible}%
%
\isadeliminvisible
%
\endisadeliminvisible
%
\isadelimquoteme
%
\endisadelimquoteme
%
\isatagquoteme
%
\begin{isamarkuptext}%
\isaverbatim%
\noindent%
\verb|structure Example = |\newline%
\verb|struct|\newline%
\newline%
\verb|datatype nat = Suc of nat |\verb,|,\verb| Zero_nat;|\newline%
\newline%
\verb|datatype boola = False |\verb,|,\verb| True;|\newline%
\newline%
\verb|fun anda x True = x|\newline%
\verb|  |\verb,|,\verb| anda x False = False|\newline%
\verb|  |\verb,|,\verb| anda True x = x|\newline%
\verb|  |\verb,|,\verb| anda False x = False;|\newline%
\newline%
\verb|fun less_nat m (Suc n) = less_eq_nat m n|\newline%
\verb|  |\verb,|,\verb| less_nat n Zero_nat = False|\newline%
\verb|and less_eq_nat (Suc m) n = less_nat m n|\newline%
\verb|  |\verb,|,\verb| less_eq_nat Zero_nat n = True;|\newline%
\newline%
\verb|fun in_interval (k, l) n = anda (less_eq_nat k n) (less_eq_nat n l);|\newline%
\newline%
\verb|end; (*struct Example*)|%
\end{isamarkuptext}%
\isamarkuptrue%
%
\endisatagquoteme
{\isafoldquoteme}%
%
\isadelimquoteme
%
\endisadelimquoteme
%
\begin{isamarkuptext}%
\noindent Though this is correct code, it is a little bit unsatisfactory:
  boolean values and operators are materialised as distinguished
  entities with have nothing to do with the SML-built-in notion
  of \qt{bool}.  This results in less readable code;
  additionally, eager evaluation may cause programs to
  loop or break which would perfectly terminate when
  the existing SML \verb|bool| would be used.  To map
  the HOL \isa{bool} on SML \verb|bool|, we may use
  \qn{custom serialisations}:%
\end{isamarkuptext}%
\isamarkuptrue%
%
\isadelimtt
%
\endisadelimtt
%
\isatagtt
\isacommand{code{\isacharunderscore}type}\isamarkupfalse%
\ bool\isanewline
\ \ {\isacharparenleft}SML\ {\isachardoublequoteopen}bool{\isachardoublequoteclose}{\isacharparenright}\isanewline
\isacommand{code{\isacharunderscore}const}\isamarkupfalse%
\ True\ \isakeyword{and}\ False\ \isakeyword{and}\ {\isachardoublequoteopen}op\ {\isasymand}{\isachardoublequoteclose}\isanewline
\ \ {\isacharparenleft}SML\ {\isachardoublequoteopen}true{\isachardoublequoteclose}\ \isakeyword{and}\ {\isachardoublequoteopen}false{\isachardoublequoteclose}\ \isakeyword{and}\ {\isachardoublequoteopen}{\isacharunderscore}\ andalso\ {\isacharunderscore}{\isachardoublequoteclose}{\isacharparenright}%
\endisatagtt
{\isafoldtt}%
%
\isadelimtt
%
\endisadelimtt
%
\begin{isamarkuptext}%
\noindent The \hyperlink{command.code-type}{\mbox{\isa{\isacommand{code{\isacharunderscore}type}}}} command takes a type constructor
  as arguments together with a list of custom serialisations.
  Each custom serialisation starts with a target language
  identifier followed by an expression, which during
  code serialisation is inserted whenever the type constructor
  would occur.  For constants, \hyperlink{command.code-const}{\mbox{\isa{\isacommand{code{\isacharunderscore}const}}}} implements
  the corresponding mechanism.  Each ``\verb|_|'' in
  a serialisation expression is treated as a placeholder
  for the type constructor's (the constant's) arguments.%
\end{isamarkuptext}%
\isamarkuptrue%
%
\isadelimquoteme
%
\endisadelimquoteme
%
\isatagquoteme
%
\begin{isamarkuptext}%
\isaverbatim%
\noindent%
\verb|structure Example = |\newline%
\verb|struct|\newline%
\newline%
\verb|datatype nat = Suc of nat |\verb,|,\verb| Zero_nat;|\newline%
\newline%
\verb|fun less_nat m (Suc n) = less_eq_nat m n|\newline%
\verb|  |\verb,|,\verb| less_nat n Zero_nat = false|\newline%
\verb|and less_eq_nat (Suc m) n = less_nat m n|\newline%
\verb|  |\verb,|,\verb| less_eq_nat Zero_nat n = true;|\newline%
\newline%
\verb|fun in_interval (k, l) n = (less_eq_nat k n) andalso (less_eq_nat n l);|\newline%
\newline%
\verb|end; (*struct Example*)|%
\end{isamarkuptext}%
\isamarkuptrue%
%
\endisatagquoteme
{\isafoldquoteme}%
%
\isadelimquoteme
%
\endisadelimquoteme
%
\begin{isamarkuptext}%
\noindent This still is not perfect: the parentheses
  around the \qt{andalso} expression are superfluous.
  Though the serializer
  by no means attempts to imitate the rich Isabelle syntax
  framework, it provides some common idioms, notably
  associative infixes with precedences which may be used here:%
\end{isamarkuptext}%
\isamarkuptrue%
%
\isadelimtt
%
\endisadelimtt
%
\isatagtt
\isacommand{code{\isacharunderscore}const}\isamarkupfalse%
\ {\isachardoublequoteopen}op\ {\isasymand}{\isachardoublequoteclose}\isanewline
\ \ {\isacharparenleft}SML\ \isakeyword{infixl}\ {\isadigit{1}}\ {\isachardoublequoteopen}andalso{\isachardoublequoteclose}{\isacharparenright}%
\endisatagtt
{\isafoldtt}%
%
\isadelimtt
%
\endisadelimtt
%
\isadelimquoteme
%
\endisadelimquoteme
%
\isatagquoteme
%
\begin{isamarkuptext}%
\isaverbatim%
\noindent%
\verb|structure Example = |\newline%
\verb|struct|\newline%
\newline%
\verb|datatype nat = Suc of nat |\verb,|,\verb| Zero_nat;|\newline%
\newline%
\verb|fun less_nat m (Suc n) = less_eq_nat m n|\newline%
\verb|  |\verb,|,\verb| less_nat n Zero_nat = false|\newline%
\verb|and less_eq_nat (Suc m) n = less_nat m n|\newline%
\verb|  |\verb,|,\verb| less_eq_nat Zero_nat n = true;|\newline%
\newline%
\verb|fun in_interval (k, l) n = less_eq_nat k n andalso less_eq_nat n l;|\newline%
\newline%
\verb|end; (*struct Example*)|%
\end{isamarkuptext}%
\isamarkuptrue%
%
\endisatagquoteme
{\isafoldquoteme}%
%
\isadelimquoteme
%
\endisadelimquoteme
%
\begin{isamarkuptext}%
\noindent Next, we try to map HOL pairs to SML pairs, using the
  infix ``\verb|*|'' type constructor and parentheses:%
\end{isamarkuptext}%
\isamarkuptrue%
%
\isadeliminvisible
%
\endisadeliminvisible
%
\isataginvisible
%
\endisataginvisible
{\isafoldinvisible}%
%
\isadeliminvisible
%
\endisadeliminvisible
%
\isadelimtt
%
\endisadelimtt
%
\isatagtt
\isacommand{code{\isacharunderscore}type}\isamarkupfalse%
\ {\isacharasterisk}\isanewline
\ \ {\isacharparenleft}SML\ \isakeyword{infix}\ {\isadigit{2}}\ {\isachardoublequoteopen}{\isacharasterisk}{\isachardoublequoteclose}{\isacharparenright}\isanewline
\isacommand{code{\isacharunderscore}const}\isamarkupfalse%
\ Pair\isanewline
\ \ {\isacharparenleft}SML\ {\isachardoublequoteopen}{\isacharbang}{\isacharparenleft}{\isacharparenleft}{\isacharunderscore}{\isacharparenright}{\isacharcomma}{\isacharslash}\ {\isacharparenleft}{\isacharunderscore}{\isacharparenright}{\isacharparenright}{\isachardoublequoteclose}{\isacharparenright}%
\endisatagtt
{\isafoldtt}%
%
\isadelimtt
%
\endisadelimtt
%
\begin{isamarkuptext}%
\noindent The initial bang ``\verb|!|'' tells the serializer to never put
  parentheses around the whole expression (they are already present),
  while the parentheses around argument place holders
  tell not to put parentheses around the arguments.
  The slash ``\verb|/|'' (followed by arbitrary white space)
  inserts a space which may be used as a break if necessary
  during pretty printing.

  These examples give a glimpse what mechanisms
  custom serialisations provide; however their usage
  requires careful thinking in order not to introduce
  inconsistencies -- or, in other words:
  custom serialisations are completely axiomatic.

  A further noteworthy details is that any special
  character in a custom serialisation may be quoted
  using ``\verb|'|''; thus, in
  ``\verb|fn '_ => _|'' the first
  ``\verb|_|'' is a proper underscore while the
  second ``\verb|_|'' is a placeholder.

  The HOL theories provide further
  examples for custom serialisations.%
\end{isamarkuptext}%
\isamarkuptrue%
%
\isamarkupsubsection{\isa{Haskell} serialisation%
}
\isamarkuptrue%
%
\begin{isamarkuptext}%
For convenience, the default
  \isa{HOL} setup for \isa{Haskell} maps the \isa{eq} class to
  its counterpart in \isa{Haskell}, giving custom serialisations
  for the class \isa{eq} (by command \hyperlink{command.code-class}{\mbox{\isa{\isacommand{code{\isacharunderscore}class}}}}) and its operation
  \isa{eq{\isacharunderscore}class{\isachardot}eq}%
\end{isamarkuptext}%
\isamarkuptrue%
%
\isadelimtt
%
\endisadelimtt
%
\isatagtt
\isacommand{code{\isacharunderscore}class}\isamarkupfalse%
\ eq\isanewline
\ \ {\isacharparenleft}Haskell\ {\isachardoublequoteopen}Eq{\isachardoublequoteclose}\ \isakeyword{where}\ {\isachardoublequoteopen}HOL{\isachardot}eq{\isachardoublequoteclose}\ {\isasymequiv}\ {\isachardoublequoteopen}{\isacharparenleft}{\isacharequal}{\isacharequal}{\isacharparenright}{\isachardoublequoteclose}{\isacharparenright}\isanewline
\isanewline
\isacommand{code{\isacharunderscore}const}\isamarkupfalse%
\ {\isachardoublequoteopen}op\ {\isacharequal}{\isachardoublequoteclose}\isanewline
\ \ {\isacharparenleft}Haskell\ \isakeyword{infixl}\ {\isadigit{4}}\ {\isachardoublequoteopen}{\isacharequal}{\isacharequal}{\isachardoublequoteclose}{\isacharparenright}%
\endisatagtt
{\isafoldtt}%
%
\isadelimtt
%
\endisadelimtt
%
\begin{isamarkuptext}%
\noindent A problem now occurs whenever a type which
  is an instance of \isa{eq} in \isa{HOL} is mapped
  on a \isa{Haskell}-built-in type which is also an instance
  of \isa{Haskell} \isa{Eq}:%
\end{isamarkuptext}%
\isamarkuptrue%
%
\isadelimquoteme
%
\endisadelimquoteme
%
\isatagquoteme
\isacommand{typedecl}\isamarkupfalse%
\ bar\isanewline
\isanewline
\isacommand{instantiation}\isamarkupfalse%
\ bar\ {\isacharcolon}{\isacharcolon}\ eq\isanewline
\isakeyword{begin}\isanewline
\isanewline
\isacommand{definition}\isamarkupfalse%
\ {\isachardoublequoteopen}eq{\isacharunderscore}class{\isachardot}eq\ {\isacharparenleft}x{\isasymColon}bar{\isacharparenright}\ y\ {\isasymlongleftrightarrow}\ x\ {\isacharequal}\ y{\isachardoublequoteclose}\isanewline
\isanewline
\isacommand{instance}\isamarkupfalse%
\ \isacommand{by}\isamarkupfalse%
\ default\ {\isacharparenleft}simp\ add{\isacharcolon}\ eq{\isacharunderscore}bar{\isacharunderscore}def{\isacharparenright}\isanewline
\isanewline
\isacommand{end}\isamarkupfalse%
%
\endisatagquoteme
{\isafoldquoteme}%
%
\isadelimquoteme
%
\endisadelimquoteme
\isanewline
%
\isadelimtt
\isanewline
%
\endisadelimtt
%
\isatagtt
\isacommand{code{\isacharunderscore}type}\isamarkupfalse%
\ bar\isanewline
\ \ {\isacharparenleft}Haskell\ {\isachardoublequoteopen}Integer{\isachardoublequoteclose}{\isacharparenright}%
\endisatagtt
{\isafoldtt}%
%
\isadelimtt
%
\endisadelimtt
%
\begin{isamarkuptext}%
\noindent The code generator would produce
  an additional instance, which of course is rejectedby the \isa{Haskell}
  compiler.
  To suppress this additional instance, use
  \isa{code{\isacharunderscore}instance}:%
\end{isamarkuptext}%
\isamarkuptrue%
%
\isadelimtt
%
\endisadelimtt
%
\isatagtt
\isacommand{code{\isacharunderscore}instance}\isamarkupfalse%
\ bar\ {\isacharcolon}{\isacharcolon}\ eq\isanewline
\ \ {\isacharparenleft}Haskell\ {\isacharminus}{\isacharparenright}%
\endisatagtt
{\isafoldtt}%
%
\isadelimtt
%
\endisadelimtt
\isanewline
%
\isadelimtheory
\isanewline
%
\endisadelimtheory
%
\isatagtheory
\isacommand{end}\isamarkupfalse%
%
\endisatagtheory
{\isafoldtheory}%
%
\isadelimtheory
%
\endisadelimtheory
\isanewline
\end{isabellebody}%
%%% Local Variables:
%%% mode: latex
%%% TeX-master: "root"
%%% End:
