\documentclass[a4paper,12pt]{article}
\usepackage[T1]{fontenc}
\usepackage{amsmath}
\usepackage{amssymb}
\usepackage[english,french]{babel}
\usepackage{color}
\usepackage{footmisc}
\usepackage{graphicx}
%\usepackage{mathpazo}
\usepackage{multicol}
\usepackage{stmaryrd}
%\usepackage[scaled=.85]{beramono}
\usepackage{../iman,../pdfsetup}

%\oddsidemargin=4.6mm
%\evensidemargin=4.6mm
%\textwidth=150mm
%\topmargin=4.6mm
%\headheight=0mm
%\headsep=0mm
%\textheight=234mm

\def\Colon{\mathord{:\mkern-1.5mu:}}
%\def\lbrakk{\mathopen{\lbrack\mkern-3.25mu\lbrack}}
%\def\rbrakk{\mathclose{\rbrack\mkern-3.255mu\rbrack}}
\def\lparr{\mathopen{(\mkern-4mu\mid}}
\def\rparr{\mathclose{\mid\mkern-4mu)}}

\def\unk{{?}}
\def\undef{(\lambda x.\; \unk)}
%\def\unr{\textit{others}}
\def\unr{\ldots}
\def\Abs#1{\hbox{\rm{\flqq}}{\,#1\,}\hbox{\rm{\frqq}}}
\def\Q{{\smash{\lower.2ex\hbox{$\scriptstyle?$}}}}

\urlstyle{tt}

\begin{document}

\selectlanguage{english}

\title{\includegraphics[scale=0.5]{isabelle_sledgehammer} \\[4ex]
Hammering Away \\[\smallskipamount]
\Large A User's Guide to Sledgehammer for Isabelle/HOL}
\author{\hbox{} \\
Jasmin Christian Blanchette \\
{\normalsize Institut f\"ur Informatik, Technische Universit\"at M\"unchen} \\
\hbox{}}

\maketitle

\tableofcontents

\setlength{\parskip}{.7em plus .2em minus .1em}
\setlength{\parindent}{0pt}
\setlength{\abovedisplayskip}{\parskip}
\setlength{\abovedisplayshortskip}{.9\parskip}
\setlength{\belowdisplayskip}{\parskip}
\setlength{\belowdisplayshortskip}{.9\parskip}

% General-purpose enum environment with correct spacing
\newenvironment{enum}%
    {\begin{list}{}{%
        \setlength{\topsep}{.1\parskip}%
        \setlength{\partopsep}{.1\parskip}%
        \setlength{\itemsep}{\parskip}%
        \advance\itemsep by-\parsep}}
    {\end{list}}

\def\pre{\begingroup\vskip0pt plus1ex\advance\leftskip by\leftmargin
\advance\rightskip by\leftmargin}
\def\post{\vskip0pt plus1ex\endgroup}

\def\prew{\pre\advance\rightskip by-\leftmargin}
\def\postw{\post}

\section{Introduction}
\label{introduction}

Sledgehammer is a tool that applies first-order automatic theorem provers (ATPs)
on the current goal. The supported ATPs are E \cite{schulz-2002}, SPASS
\cite{weidenbach-et-al-2009}, and Vampire \cite{riazanov-voronkov-2002}, which
can be run locally or remotely via the SystemOnTPTP web service
\cite{sutcliffe-2000}.

The problem passed to ATPs consists of the current goal together with a
heuristic selection of facts (theorems) from the current theory context,
filtered by relevance. The result of a successful ATP proof search is some
source text that usually (but not always) reconstructs the proof within
Isabelle, without requiring the ATPs again. The reconstructed proof relies on
the general-purpose Metis prover \cite{metis}, which is fully integrated into
Isabelle/HOL, with explicit inferences going through the kernel. Thus its
results are correct by construction.

\newbox\boxA
\setbox\boxA=\hbox{\texttt{nospam}}

Examples of Sledgehammer use can be found in Isabelle's
\texttt{src/HOL/Metis\_Examples} directory.
Comments and bug reports concerning Sledgehammer or this manual should be
directed to
\texttt{blan{\color{white}nospam}\kern-\wd\boxA{}chette@\allowbreak
in.\allowbreak tum.\allowbreak de}.

\vskip2.5\smallskipamount

%\textbf{Acknowledgment.} The author would like to thank Mark Summerfield for
%suggesting several textual improvements.

\section{Installation}
\label{installation}

Sledgehammer is part of Isabelle, so you don't need to install it. However, it
relies on third-party automatic theorem provers (ATPs). Currently, E, SPASS, and
Vampire are supported. All of these are available remotely via SystemOnTPTP
\cite{sutcliffe-2000}, but if you want better performance you will need to
install at least E and SPASS locally.

There are three main ways to install E and SPASS:

\begin{enum}
\item[$\bullet$] If you installed an official Isabelle package with everything
inside, it should already include properly setup executables for E and SPASS,
ready to use.

\item[$\bullet$] Otherwise, you can download the Isabelle-aware E and SPASS
binary packages from Isabelle's download page. Extract the archives, then add a
line to your \texttt{\char`\~/.isabelle/etc/components} file with the absolute path to
E or SPASS. For example, if \texttt{\char`\~/.isabelle/etc/components} does not exist
yet and you extracted SPASS to \texttt{/usr/local/spass-3.7}, create
the file \texttt{\char`\~/.isabelle/etc/components} with the single line

\prew
\texttt{/usr/local/spass-3.7}
\postw

\item[$\bullet$] If you prefer to build E or SPASS yourself, feel free to do so
and set the environment variable \texttt{E\_HOME} or \texttt{SPASS\_HOME} to the
directory that contains the \texttt{eproof} or \texttt{SPASS} executable,
respectively.
\end{enum}

To check whether E and SPASS are installed, follow the example in
\S\ref{first-steps}.

\section{First Steps}
\label{first-steps}

To illustrate Sledgehammer in context, let us start a theory file and
attempt to prove a simple lemma:

\prew
\textbf{theory}~\textit{Scratch} \\
\textbf{imports}~\textit{Main} \\
\textbf{begin} \\[2\smallskipamount]
%
\textbf{lemma} ``$[a] = [b] \,\longleftrightarrow\, a = b$'' \\
\textbf{sledgehammer}
\postw

After a few seconds, Sledgehammer produces the following output:

\prew
\slshape
Sledgehammer: ATP ``\textit{e}'' for subgoal 1: \\
$([a] = [b]) = (a = b)$ \\
Try this command: \textbf{by} (\textit{metis hd.simps}). \\
To minimize the number of lemmas, try this command: \\
\textbf{sledgehammer} \textit{minimize} [\textit{atp} = \textit{e}] (\textit{hd.simps}). \\[3\smallskipamount]
%
Sledgehammer: ATP ``\textit{spass}'' for subgoal 1: \\
$([a] = [b]) = (a = b)$ \\
Try this command: \textbf{by} (\textit{metis insert\_Nil last\_ConsL}). \\
To minimize the number of lemmas, try this command: \\
\textbf{sledgehammer} \textit{minimize} [\textit{atp} = \textit{spass}] (\textit{insert\_Nil last\_ConsL}). \\[3\smallskipamount]
%
Sledgehammer: ATP ``\textit{remote\_vampire}'' for subgoal 1: \\
$([a] = [b]) = (a = b)$ \\
Try this command: \textbf{by} (\textit{metis One\_nat\_def\_raw empty\_replicate} \\
\phantom{Try this command: \textbf{by} (\textit{metis~}}\textit{insert\_Nil last\_ConsL replicate\_Suc}). \\
To minimize the number of lemmas, try this command: \\
\textbf{sledgehammer} \textit{minimize} [\textit{atp} = \textit{remote\_vampire}] \\
\phantom{\textbf{sledgehammer}~}(\textit{One\_nat\_def\_raw empty\_replicate insert\_Nil} \\
\phantom{\textbf{sledgehammer}~(}\textit{last\_ConsL replicate\_Suc}).
\postw

Sledgehammer ran E, SPASS, and the remote version of Vampire in parallel. If E
and SPASS are not installed (\S\ref{installation}), you will see references to
their remote American cousins \textit{remote\_e} and \textit{remote\_spass}
instead of \textit{e} and \textit{spass}.

Based on each ATP proof, Sledgehammer gives a one-liner proof that uses the
\textit{metis} method. You can click them and insert them into the theory text.
You can click the ``\textbf{sledgehammer} \textit{minimize}'' command if you
want to look for a shorter (and faster) proof. But here the proof found by E
looks perfect, so click it to finish the proof.

You can ask Sledgehammer for an Isar text proof by passing the
\textit{isar\_proof} option:

\prew
\textbf{sledgehammer} [\textit{isar\_proof}]
\postw

When Isar proof construction is successful, it can yield proofs that are more
readable and also faster than the \textit{metis} one-liners. This feature is
experimental.

\section{Command Syntax}
\label{command-syntax}

Sledgehammer can be invoked at any point when there is an open goal by entering
the \textbf{sledgehammer} command in the theory file. Its general syntax is as
follows:

\prew
\textbf{sledgehammer} \textit{subcommand\/$^?$ options\/$^?$ facts\_override\/$^?$ num\/$^?$}
\postw

For convenience, Sledgehammer is also available in the ``Commands'' submenu of
the ``Isabelle'' menu in Proof General or by pressing the Emacs key sequence C-c
C-a C-s. This is equivalent to entering the \textbf{sledgehammer} command with
no arguments in the theory text.

In the general syntax, the \textit{subcommand} may be any of the following:

\begin{enum}
\item[$\bullet$] \textbf{\textit{run} (the default):} Runs Sledgehammer on subgoal number
\textit{num} (1 by default), with the given options and facts.

\item[$\bullet$] \textbf{\textit{minimize}:} Attempts to minimize the provided facts
(specified in the \textit{facts\_override} argument) to obtain a simpler proof
involving fewer facts. The options and goal number are as for \textit{run}.

\item[$\bullet$] \textbf{\textit{messages}:} Redisplays recent messages issued by
Sledgehammer. This allows you to examine results that might have been lost due
to Sledgehammer's asynchronous nature. The \textit{num} argument specifies a
limit on the number of messages to display (5 by default).

\item[$\bullet$] \textbf{\textit{available\_atps}:} Prints the list of installed ATPs.
See \S\ref{installation} and \S\ref{mode-of-operation} for more information on
how to install ATPs.

\item[$\bullet$] \textbf{\textit{running\_atps}:} Prints information about currently
running ATPs, including elapsed runtime and remaining time until timeout.

\item[$\bullet$] \textbf{\textit{kill\_atps}:} Terminates all running ATPs.

\item[$\bullet$] \textbf{\textit{refresh\_tptp}:} Refreshes the list of remote
ATPs available at System\-On\-TPTP \cite{sutcliffe-2000}.
\end{enum}

Sledgehammer's behavior can be influenced by various \textit{options}, which can
be specified in brackets after the \textbf{sledgehammer} command. The
\textit{options} are a list of key--value pairs of the form ``[$k_1 = v_1,
\ldots, k_n = v_n$]''. For Boolean options, ``= \textit{true}'' is optional. For
example:

\prew
\textbf{sledgehammer} [\textit{isar\_proof}, \,\textit{timeout} = 120$\,s$]
\postw

Default values can be set using \textbf{sledgehammer\_\allowbreak params}:

\prew
\textbf{sledgehammer\_params} \textit{options}
\postw

The supported options are described in \S\ref{option-reference}.

The \textit{facts\_override} argument lets you alter the set of facts that go
through the relevance filter. It may be of the form ``(\textit{facts})'', where
\textit{facts} is a space-separated list of Isabelle facts (theorems, local
assumptions, etc.), in which case the relevance filter is bypassed and the given
facts are used. It may also be of the form (\textit{add}:\ \textit{facts}$_1$),
(\textit{del}:\ \textit{facts}$_2$), or (\textit{add}:\ \textit{facts}$_1$\
\textit{del}:\ \textit{facts}$_2$), where the relevance filter is instructed to
proceed as usual except that it should consider \textit{facts}$_1$
highly-relevant and \textit{facts}$_2$ fully irrelevant.

\section{Option Reference}
\label{option-reference}

\def\flushitem#1{\item[]\noindent\kern-\leftmargin \textbf{#1}}
\def\qty#1{$\left<\textit{#1}\right>$}
\def\qtybf#1{$\mathbf{\left<\textbf{\textit{#1}}\right>}$}
\def\optrue#1#2{\flushitem{\textit{#1} $\bigl[$= \qtybf{bool}$\bigr]$\quad [\textit{true}]\hfill (neg.: \textit{#2})}\nopagebreak\\[\parskip]}
\def\opfalse#1#2{\flushitem{\textit{#1} $\bigl[$= \qtybf{bool}$\bigr]$\quad [\textit{false}]\hfill (neg.: \textit{#2})}\nopagebreak\\[\parskip]}
\def\opsmart#1#2{\flushitem{\textit{#1} $\bigl[$= \qtybf{bool\_or\_smart}$\bigr]$\quad [\textit{smart}]\hfill (neg.: \textit{#2})}\nopagebreak\\[\parskip]}
\def\opsmartx#1#2{\flushitem{\textit{#1} $\bigl[$= \qtybf{bool\_or\_smart}$\bigr]$\quad [\textit{smart}]\hfill\\\hbox{}\hfill (neg.: \textit{#2})}\nopagebreak\\[\parskip]}
\def\opnodefault#1#2{\flushitem{\textit{#1} = \qtybf{#2}} \nopagebreak\\[\parskip]}
\def\opdefault#1#2#3{\flushitem{\textit{#1} = \qtybf{#2}\quad [\textit{#3}]} \nopagebreak\\[\parskip]}
\def\oparg#1#2#3{\flushitem{\textit{#1} \qtybf{#2} = \qtybf{#3}} \nopagebreak\\[\parskip]}
\def\opargbool#1#2#3{\flushitem{\textit{#1} \qtybf{#2} $\bigl[$= \qtybf{bool}$\bigr]$\hfill (neg.: \textit{#3})}\nopagebreak\\[\parskip]}
\def\opargboolorsmart#1#2#3{\flushitem{\textit{#1} \qtybf{#2} $\bigl[$= \qtybf{bool\_or\_smart}$\bigr]$\hfill (neg.: \textit{#3})}\nopagebreak\\[\parskip]}

Sledgehammer's options are categorized as follows:\ mode of operation
(\S\ref{mode-of-operation}), problem encoding (\S\ref{problem-encoding}), output
format (\S\ref{output-format}), and timeouts (\S\ref{timeouts}).

The descriptions below refer to the following syntactic quantities:

\begin{enum}
\item[$\bullet$] \qtybf{string}: A string.
\item[$\bullet$] \qtybf{bool\/}: \textit{true} or \textit{false}.
\item[$\bullet$] \qtybf{bool\_or\_smart\/}: \textit{true}, \textit{false}, or \textit{smart}.
\item[$\bullet$] \qtybf{int\/}: An integer.
\item[$\bullet$] \qtybf{time}: An integer followed by $\textit{min}$ (minutes), $s$ (seconds), or \textit{ms}
(milliseconds), or the keyword \textit{none} ($\infty$ years).
\end{enum}

Default values are indicated in square brackets. Boolean options have a negated
counterpart (e.g., \textit{debug} vs.\ \textit{no\_debug}). When setting
Boolean options, ``= \textit{true}'' may be omitted.

\subsection{Mode of Operation}
\label{mode-of-operation}

\begin{enum}
%\optrue{blocking}{non\_blocking}
%Specifies whether the \textbf{sledgehammer} command should operate synchronously.
%The asynchronous (non-blocking) mode lets the user start proving the putative
%theorem while Sledgehammer looks for a counterexample, but it can also be more
%confusing. For technical reasons, automatic runs currently always block.

\opnodefault{atps}{string}
Specifies the ATPs (automated theorem provers) to use as a space-separated list
(e.g., ``\textit{e}~\textit{spass}''). The following ATPs are supported:

\begin{enum}
\item[$\bullet$] \textbf{\textit{e}:} E is an ATP developed by Stephan Schulz
\cite{schulz-2002}. To use E, set the environment variable
\texttt{E\_HOME} to the directory that contains the \texttt{eproof} executable,
or install the prebuilt E package from Isabelle's download page. See
\S\ref{installation} for details.

\item[$\bullet$] \textbf{\textit{spass}:} SPASS is an ATP developed by Christoph
Weidenbach et al.\ \cite{weidenbach-et-al-2009}. To use SPASS, set the
environment variable \texttt{SPASS\_HOME} to the directory that contains the
\texttt{SPASS} executable, or install the prebuilt SPASS package from Isabelle's
download page. Sledgehammer requires version 3.5 or above. See
\S\ref{installation} for details.

\item[$\bullet$] \textbf{\textit{vampire}:} Vampire is an ATP developed by
Andrei Voronkov and his colleagues \cite{riazanov-voronkov-2002}. To use
Vampire, set the environment variable \texttt{VAMPIRE\_HOME} to the directory
that contains the \texttt{vampire} executable.

\item[$\bullet$] \textbf{\textit{remote\_e}:} The remote version of E executes
on Geoff Sutcliffe's Miami servers \cite{sutcliffe-2000}.

\item[$\bullet$] \textbf{\textit{remote\_spass}:} The remote version of SPASS
executes on Geoff Sutcliffe's Miami servers.

\item[$\bullet$] \textbf{\textit{remote\_vampire}:} The remote version of
Vampire executes on Geoff Sutcliffe's Miami servers. Version 9 is used.

\end{enum}

By default, Sledgehammer will run E, SPASS, and Vampire in parallel. For E and
SPASS, it will use any locally installed version if available, falling back
on the remote versions if necessary. For historical reasons, the default value
of this option can be overridden using the option ``Sledgehammer: ATPs'' from
the ``Isabelle'' menu in Proof General.

It is a good idea to run several ATPs in parallel, although it could slow down
your machine. Tobias Nipkow observed that running E, SPASS, and Vampire together
for 5 seconds yields the same success rate as running the most effective of
these (Vampire) for 120 seconds \cite{boehme-nipkow-2010}.

\opnodefault{atp}{string}
Alias for \textit{atps}.

\opfalse{overlord}{no\_overlord}
Specifies whether Sledgehammer should put its temporary files in
\texttt{\$ISA\-BELLE\_\allowbreak HOME\_\allowbreak USER}, which is useful for
debugging Sledgehammer but also unsafe if several instances of the tool are run
simultaneously. The files are identified by the prefix \texttt{prob\_}; you may
safely remove them after Sledgehammer has run.

\nopagebreak
{\small See also \textit{debug} (\S\ref{output-format}).}
\end{enum}

\subsection{Problem Encoding}
\label{problem-encoding}

\begin{enum}
\opfalse{explicit\_apply}{implicit\_apply}
Specifies whether function application should be encoded as an explicit
``apply'' operator. If the option is set to \textit{false}, each function will
be directly applied to as many arguments as possible. Enabling this option can
sometimes help discover higher-order proofs that otherwise would not be found.

\opfalse{full\_types}{partial\_types}
Specifies whether full-type information is exported. Enabling this option can
prevent the discovery of type-incorrect proofs, but it also tends to slow down
the ATPs significantly. For historical reasons, the default value of this option
can be overridden using the option ``Sledgehammer: ATPs'' from the ``Isabelle''
menu in Proof General.

\opdefault{relevance\_threshold}{int}{50}
Specifies the threshold above which facts are considered relevant by the
relevance filter. The option ranges from 0 to 100, where 0 means that all
theorems are relevant.

\opdefault{relevance\_convergence}{int}{320}
Specifies the convergence quotient, multiplied by 100, used by the relevance
filter. This quotient is used by the relevance filter to scale down the
relevance of facts at each iteration of the filter.

\opsmartx{theory\_relevant}{theory\_irrelevant}
Specifies whether the theory from which a fact comes should be taken into
consideration by the relevance filter. If the option is set to \textit{smart},
it is taken to be \textit{true} for SPASS and \textit{false} for E and Vampire,
because empirical results suggest that these are the best settings.

\opfalse{defs\_relevant}{defs\_irrelevant}
Specifies whether the definition of constants occurring in the formula to prove
should be considered particularly relevant. Enabling this option tends to lead
to larger problems and typically slows down the ATPs.

\optrue{respect\_no\_atp}{ignore\_no\_atp}
Specifies whether Sledgehammer should honor the \textit{no\_atp} attributes. The
\textit{no\_atp} attributes marks theorems that tend to confuse ATPs, typically
because they can lead to unsound ATP proofs \cite{boehme-nipkow-2010}. It is
normally a good idea to leave this option enabled, unless you are debugging
Sledgehammer.

\end{enum}

\subsection{Output Format}
\label{output-format}

\begin{enum}

\opfalse{verbose}{quiet}
Specifies whether the \textbf{sledgehammer} command should explain what it does.

\opfalse{debug}{no\_debug}
Specifies whether Nitpick should display additional debugging information beyond
what \textit{verbose} already displays. Enabling \textit{debug} also enables
\textit{verbose} behind the scenes.

\nopagebreak
{\small See also \textit{overlord} (\S\ref{mode-of-operation}).}

\opfalse{isar\_proof}{no\_isar\_proof}
Specifies whether Isar proofs should be output in addition to one-liner
\textit{metis} proofs. Isar proof construction is still experimental and often
fails; however, they are usually faster and sometimes more robust than
\textit{metis} proofs.

\opdefault{isar\_shrink\_factor}{int}{1}
Specifies the granularity of the Isar proof. A value of $n$ indicates that each
Isar proof step should correspond to a group of up to $n$ consecutive proof
steps in the ATP proof.

\end{enum}

\subsection{Timeouts}
\label{timeouts}

\begin{enum}
\opdefault{timeout}{time}{$\mathbf{60}$ s}
Specifies the maximum amount of time that the ATPs should spend looking for a
proof. For historical reasons, the default value of this option can be
overridden using the option ``Sledgehammer: Time Limit'' from the ``Isabelle''
menu in Proof General.

\opdefault{minimize\_timeout}{time}{$\mathbf{5}$\,s}
Specifies the maximum amount of time that the ATPs should spend looking for a
proof for \textbf{sledgehammer}~\textit{minimize}.
\end{enum}

\let\em=\sl
\bibliography{../manual}{}
\bibliographystyle{abbrv}

\end{document}
