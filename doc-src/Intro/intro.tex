\documentstyle[a4,12pt,proof209,iman,extra]{article}
%% $Id$
%% run    bibtex intro         to prepare bibliography
%% run    ../sedindex intro    to prepare index file
%prth *(\(.*\));          \1;      
%{\\out \(.*\)}          {\\out val it = "\1" : thm}

\title{Introduction to Isabelle}   
\author{{\em Lawrence C. Paulson}\\
        Computer Laboratory \\ University of Cambridge \\[2ex]
        {\small{\em Electronic mail\/}: {\tt lcp@cl.cam.ac.uk}}
}
\date{} 
\makeindex

\underscoreoff

\setcounter{secnumdepth}{2} \setcounter{tocdepth}{2}

\sloppy
\binperiod     %%%treat . like a binary operator

\newcommand\qeq{\stackrel{?}{\equiv}}  %for disagreement pairs in unification
\newcommand{\nand}{\mathbin{\lnot\&}} 
\newcommand{\xor}{\mathbin{\#}}

\pagenumbering{roman} 
\begin{document}
\pagestyle{empty}
\begin{titlepage}
\maketitle 
\thispagestyle{empty}
\vfill
{\small Copyright \copyright{} \number\year{} by Lawrence C. Paulson}
\end{titlepage}

\pagestyle{headings}
\part*{Preface}
\index{Isabelle!overview} \index{Isabelle!object-logics supported}
Isabelle~\cite{paulson-natural,paulson-found,paulson700} is a generic theorem
prover.  It has been instantiated to support reasoning in several
object-logics:
\begin{itemize}
\item first-order logic, constructive and classical versions
\item higher-order logic, similar to that of Gordon's {\sc
hol}~\cite{mgordon-hol}
\item Zermelo-Fraenkel set theory~\cite{suppes72}
\item an extensional version of Martin-L\"of's Type Theory~\cite{nordstrom90}
\item the classical first-order sequent calculus, {\sc lk}
\item the modal logics $T$, $S4$, and $S43$
\item the Logic for Computable Functions~\cite{paulson87}
\end{itemize}
A logic's syntax and inference rules are specified declaratively; this
allows single-step proof construction.  Isabelle provides control
structures for expressing search procedures.  Isabelle also provides
several generic tools, such as simplifiers and classical theorem provers,
which can be applied to object-logics.

\index{ML}
Isabelle is a large system, but beginners can get by with a small
repertoire of commands and a basic knowledge of how Isabelle works.  Some
knowledge of Standard~\ML{} is essential, because \ML{} is Isabelle's user
interface.  Advanced Isabelle theorem proving can involve writing \ML{}
code, possibly with Isabelle's sources at hand.  My book
on~\ML{}~\cite{paulson91} covers much material connected with Isabelle,
including a simple theorem prover.  Users must be familiar with logic as
used in computer science; there are many good
texts~\cite{galton90,reeves90}.

\index{LCF}
{\sc lcf}, developed by Robin Milner and colleagues~\cite{mgordon79}, is an
ancestor of {\sc hol}, Nuprl, and several other systems.  Isabelle borrows
ideas from {\sc lcf}: formulae are~\ML{} values; theorems belong to an
abstract type; tactics and tacticals support backward proof.  But {\sc lcf}
represents object-level rules by functions, while Isabelle represents them
by terms.  You may find my other writings~\cite{paulson87,paulson-handbook}
helpful in understanding the relationship between {\sc lcf} and Isabelle.

\index{Isabelle!release history} Isabelle was first distributed in 1986.
The 1987 version introduced a higher-order meta-logic with an improved
treatment of quantifiers.  The 1988 version added limited polymorphism and
support for natural deduction.  The 1989 version included a parser and
pretty printer generator.  The 1992 version introduced type classes, to
support many-sorted and higher-order logics.  The current version provides
greater support for theories and is much faster.  Isabelle is still under
development and will continue to change.

\subsubsection*{Overview} 
This manual consists of three parts.  Part~I discusses the Isabelle's
foundations.  Part~II, presents simple on-line sessions, starting with
forward proof.  It also covers basic tactics and tacticals, and some
commands for invoking them.  Part~III contains further examples for users
with a bit of experience.  It explains how to derive rules define theories,
and concludes with an extended example: a Prolog interpreter.

Isabelle's Reference Manual and Object-Logics manual contain more details.
They assume familiarity with the concepts presented here.


\subsubsection*{Acknowledgements} 
Tobias Nipkow contributed most of the section on defining theories.
Stefan Berghofer, Sara Kalvala and Markus Wenzel suggested improvements.

Tobias Nipkow has made immense contributions to Isabelle, including the
parser generator, type classes, and the simplifier.  Carsten Clasohm and
Markus Wenzel made major contributions; Sonia Mahjoub and Karin Nimmermann
also helped.  Isabelle was developed using Dave Matthews's Standard~{\sc
  ml} compiler, Poly/{\sc ml}.  Many people have contributed to Isabelle's
standard object-logics, including Martin Coen, Philippe de Groote, Philippe
No\"el.  The research has been funded by the SERC (grants GR/G53279,
GR/H40570) and by ESPRIT (projects 3245: Logical Frameworks, and 6453:
Types).

\newpage
\pagestyle{plain} \tableofcontents 
\newpage

\newfont{\sanssi}{cmssi12}
\vspace*{2.5cm}
\begin{quote}
\raggedleft
{\sanssi
You can only find truth with logic\\
if you have already found truth without it.}\\
\bigskip

G.K. Chesterton, {\em The Man who was Orthodox}
\end{quote}

\clearfirst  \pagestyle{headings}
%% $Id$
\part{Foundations} 
The following sections discuss Isabelle's logical foundations in detail:
representing logical syntax in the typed $\lambda$-calculus; expressing
inference rules in Isabelle's meta-logic; combining rules by resolution.

If you wish to use Isabelle immediately, please turn to
page~\pageref{chap:getting}.  You can always read about foundations later,
either by returning to this point or by looking up particular items in the
index.

\begin{figure} 
\begin{eqnarray*}
  \neg P   & \hbox{abbreviates} & P\imp\bot \\
  P\bimp Q & \hbox{abbreviates} & (P\imp Q) \conj (Q\imp P)
\end{eqnarray*}
\vskip 4ex

\(\begin{array}{c@{\qquad\qquad}c}
  \infer[({\conj}I)]{P\conj Q}{P & Q}  &
  \infer[({\conj}E1)]{P}{P\conj Q} \qquad 
  \infer[({\conj}E2)]{Q}{P\conj Q} \\[4ex]

  \infer[({\disj}I1)]{P\disj Q}{P} \qquad 
  \infer[({\disj}I2)]{P\disj Q}{Q} &
  \infer[({\disj}E)]{R}{P\disj Q & \infer*{R}{[P]} & \infer*{R}{[Q]}}\\[4ex]

  \infer[({\imp}I)]{P\imp Q}{\infer*{Q}{[P]}} &
  \infer[({\imp}E)]{Q}{P\imp Q & P}  \\[4ex]

  &
  \infer[({\bot}E)]{P}{\bot}\\[4ex]

  \infer[({\forall}I)*]{\forall x.P}{P} &
  \infer[({\forall}E)]{P[t/x]}{\forall x.P} \\[3ex]

  \infer[({\exists}I)]{\exists x.P}{P[t/x]} &
  \infer[({\exists}E)*]{Q}{{\exists x.P} & \infer*{Q}{[P]} } \\[3ex]

  {t=t} \,(refl)   &  \vcenter{\infer[(subst)]{P[u/x]}{t=u & P[t/x]}} 
\end{array} \)

\bigskip\bigskip
*{\em Eigenvariable conditions\/}:

$\forall I$: provided $x$ is not free in the assumptions

$\exists E$: provided $x$ is not free in $Q$ or any assumption except $P$
\caption{Intuitionistic first-order logic} \label{fol-fig}
\end{figure}

\section{Formalizing logical syntax in Isabelle}\label{sec:logical-syntax}
\index{Isabelle!formalizing syntax|bold}
Figure~\ref{fol-fig} presents intuitionistic first-order logic,
including equality.  Let us see how to formalize
this logic in Isabelle, illustrating the main features of Isabelle's
polymorphic meta-logic.

Isabelle represents syntax using the typed $\lambda$-calculus.  We declare
a type for each syntactic category of the logic.  We declare a constant for
each symbol of the logic, giving each $n$-place operation an $n$-argument
curried function type.  Most importantly, $\lambda$-abstraction represents
variable binding in quantifiers.

\index{$\To$|bold}\index{types}
Isabelle has \ML-style type constructors such as~$(\alpha)list$, where
$(bool)list$ is the type of lists of booleans.  Function types have the
form $\sigma\To\tau$, where $\sigma$ and $\tau$ are types.  Curried
function types may be abbreviated:
\[  \sigma@1\To (\cdots \sigma@n\To \tau\cdots)  \quad \hbox{as} \quad
   [\sigma@1, \ldots, \sigma@n] \To \tau $$ 
 
The syntax for terms is summarised below.  Note that function application is
written $t(u)$ rather than the usual $t\,u$.
\[ 
\begin{array}{ll}
  t :: \tau   & \hbox{type constraint, on a term or bound variable} \\
  \lambda x.t   & \hbox{abstraction} \\
  \lambda x@1\ldots x@n.t
        & \hbox{curried abstraction, $\lambda x@1. \ldots \lambda x@n.t$} \\
  t(u)          & \hbox{application} \\
  t (u@1, \ldots, u@n) & \hbox{curried application, $t(u@1)\ldots(u@n)$} 
\end{array}
\]


\subsection{Simple types and constants}\index{types!simple|bold} 

The syntactic categories of our logic (Fig.\ts\ref{fol-fig}) are {\bf
  formulae} and {\bf terms}.  Formulae denote truth values, so (following
tradition) let us call their type~$o$.  To allow~0 and~$Suc(t)$ as terms,
let us declare a type~$nat$ of natural numbers.  Later, we shall see
how to admit terms of other types.

After declaring the types~$o$ and~$nat$, we may declare constants for the
symbols of our logic.  Since $\bot$ denotes a truth value (falsity) and 0
denotes a number, we put \begin{eqnarray*}
  \bot  & :: & o \\
  0     & :: & nat.
\end{eqnarray*}
If a symbol requires operands, the corresponding constant must have a
function type.  In our logic, the successor function
($Suc$) is from natural numbers to natural numbers, negation ($\neg$) is a
function from truth values to truth values, and the binary connectives are
curried functions taking two truth values as arguments: 
\begin{eqnarray*}
  Suc    & :: & nat\To nat  \\
  {\neg} & :: & o\To o      \\
  \conj,\disj,\imp,\bimp  & :: & [o,o]\To o 
\end{eqnarray*}
The binary connectives can be declared as infixes, with appropriate
precedences, so that we write $P\conj Q\disj R$ instead of
$\disj(\conj(P,Q), R)$.

\S\ref{sec:defining-theories} below describes the syntax of Isabelle theory
files and illustrates it by extending our logic with mathematical induction.


\subsection{Polymorphic types and constants} \label{polymorphic}
\index{types!polymorphic|bold}
\index{equality!polymorphic}

Which type should we assign to the equality symbol?  If we tried
$[nat,nat]\To o$, then equality would be restricted to the natural
numbers; we would have to declare different equality symbols for each
type.  Isabelle's type system is polymorphic, so we could declare
\begin{eqnarray*}
  {=}  & :: & [\alpha,\alpha]\To o,
\end{eqnarray*}
where the type variable~$\alpha$ ranges over all types.
But this is also wrong.  The declaration is too polymorphic; $\alpha$
includes types like~$o$ and $nat\To nat$.  Thus, it admits
$\bot=\neg(\bot)$ and $Suc=Suc$ as formulae, which is acceptable in
higher-order logic but not in first-order logic.

Isabelle's {\bf type classes}\index{classes} control
polymorphism~\cite{nipkow-prehofer}.  Each type variable belongs to a
class, which denotes a set of types.  Classes are partially ordered by the
subclass relation, which is essentially the subset relation on the sets of
types.  They closely resemble the classes of the functional language
Haskell~\cite{haskell-tutorial,haskell-report}.

Isabelle provides the built-in class $logic$, which consists of the logical
types: the ones we want to reason about.  Let us declare a class $term$, to
consist of all legal types of terms in our logic.  The subclass structure
is now $term\le logic$.

We put $nat$ in class $term$ by declaring $nat{::}term$.  We declare the
equality constant by
\begin{eqnarray*}
  {=}  & :: & [\alpha{::}term,\alpha]\To o 
\end{eqnarray*}
where $\alpha{::}term$ constrains the type variable~$\alpha$ to class
$term$.  Such type variables resemble Standard~\ML's equality type
variables.

We give function types and~$o$ the class $logic$ rather than~$term$, since
they are not legal types for terms.  We may introduce new types of class
$term$ --- for instance, type $string$ or $real$ --- at any time.  We can
even declare type constructors such as $(\alpha)list$, and state that type
$(\tau)list$ belongs to class~$term$ provided $\tau$ does; equality
applies to lists of natural numbers but not to lists of formulae.  We may
summarize this paragraph by a set of {\bf arity declarations} for type
constructors: \index{$\To$|bold}\index{arities!declaring}
\begin{eqnarray*}
  o     & :: & logic \\
  {\To} & :: & (logic,logic)logic \\
  nat, string, real     & :: & term \\
  list  & :: & (term)term
\end{eqnarray*}
In higher-order logic, equality does apply to truth values and
functions;  this requires the arity declarations ${o::term}$
and ${{\To}::(term,term)term}$.  The class system can also handle
overloading.\index{overloading|bold} We could declare $arith$ to be the
subclass of $term$ consisting of the `arithmetic' types, such as~$nat$.
Then we could declare the operators
\begin{eqnarray*}
  {+},{-},{\times},{/}  & :: & [\alpha{::}arith,\alpha]\To \alpha
\end{eqnarray*}
If we declare new types $real$ and $complex$ of class $arith$, then we
effectively have three sets of operators:
\begin{eqnarray*}
  {+},{-},{\times},{/}  & :: & [nat,nat]\To nat \\
  {+},{-},{\times},{/}  & :: & [real,real]\To real \\
  {+},{-},{\times},{/}  & :: & [complex,complex]\To complex 
\end{eqnarray*}
Isabelle will regard these as distinct constants, each of which can be defined
separately.  We could even introduce the type $(\alpha)vector$ and declare
its arity as $(arith)arith$.  Then we could declare the constant
\begin{eqnarray*}
  {+}  & :: & [(\alpha)vector,(\alpha)vector]\To (\alpha)vector 
\end{eqnarray*}
and specify it in terms of ${+} :: [\alpha,\alpha]\To \alpha$.

A type variable may belong to any finite number of classes.  Suppose that
we had declared yet another class $ord \le term$, the class of all
`ordered' types, and a constant
\begin{eqnarray*}
  {\le}  & :: & [\alpha{::}ord,\alpha]\To o.
\end{eqnarray*}
In this context the variable $x$ in $x \le (x+x)$ will be assigned type
$\alpha{::}\{arith,ord\}$, which means $\alpha$ belongs to both $arith$ and
$ord$.  Semantically the set $\{arith,ord\}$ should be understood as the
intersection of the sets of types represented by $arith$ and $ord$.  Such
intersections of classes are called \bfindex{sorts}.  The empty
intersection of classes, $\{\}$, contains all types and is thus the {\bf
  universal sort}.

Even with overloading, each term has a unique, most general type.  For this
to be possible, the class and type declarations must satisfy certain
technical constraints~\cite{nipkow-prehofer}.


\subsection{Higher types and quantifiers}
\index{types!higher|bold}
Quantifiers are regarded as operations upon functions.  Ignoring polymorphism
for the moment, consider the formula $\forall x. P(x)$, where $x$ ranges
over type~$nat$.  This is true if $P(x)$ is true for all~$x$.  Abstracting
$P(x)$ into a function, this is the same as saying that $\lambda x.P(x)$
returns true for all arguments.  Thus, the universal quantifier can be
represented by a constant
\begin{eqnarray*}
  \forall  & :: & (nat\To o) \To o,
\end{eqnarray*}
which is essentially an infinitary truth table.  The representation of $\forall
x. P(x)$ is $\forall(\lambda x. P(x))$.  

The existential quantifier is treated
in the same way.  Other binding operators are also easily handled; for
instance, the summation operator $\Sigma@{k=i}^j f(k)$ can be represented as
$\Sigma(i,j,\lambda k.f(k))$, where
\begin{eqnarray*}
  \Sigma  & :: & [nat,nat, nat\To nat] \To nat.
\end{eqnarray*}
Quantifiers may be polymorphic.  We may define $\forall$ and~$\exists$ over
all legal types of terms, not just the natural numbers, and
allow summations over all arithmetic types:
\begin{eqnarray*}
   \forall,\exists      & :: & (\alpha{::}term\To o) \To o \\
   \Sigma               & :: & [nat,nat, nat\To \alpha{::}arith] \To \alpha
\end{eqnarray*}
Observe that the index variables still have type $nat$, while the values
being summed may belong to any arithmetic type.


\section{Formalizing logical rules in Isabelle}
\index{meta-logic|bold}
\index{Isabelle!formalizing rules|bold}
\index{$\Imp$|bold}\index{$\Forall$|bold}\index{$\equiv$|bold}
\index{implication!meta-level|bold}
\index{quantifiers!meta-level|bold}
\index{equality!meta-level|bold}

Object-logics are formalized by extending Isabelle's
meta-logic~\cite{paulson89}, which is intuitionistic higher-order logic.
The meta-level connectives are {\bf implication}, the {\bf universal
  quantifier}, and {\bf equality}.
\begin{itemize}
  \item The implication \(\phi\Imp \psi\) means `\(\phi\) implies
\(\psi\)', and expresses logical {\bf entailment}.  

  \item The quantification \(\Forall x.\phi\) means `\(\phi\) is true for
all $x$', and expresses {\bf generality} in rules and axiom schemes. 

\item The equality \(a\equiv b\) means `$a$ equals $b$', for expressing
  \bfindex{definitions} (see~\S\ref{definitions}).  Equalities left over
  from the unification process, so called \bfindex{flex-flex equations},
  are written $a\qeq b$.  The two equality symbols have the same logical
  meaning. 

\end{itemize}
The syntax of the meta-logic is formalized in precisely in the same manner
as object-logics, using the typed $\lambda$-calculus.  Analogous to
type~$o$ above, there is a built-in type $prop$ of meta-level truth values.
Meta-level formulae will have this type.  Type $prop$ belongs to
class~$logic$; also, $\sigma\To\tau$ belongs to $logic$ provided $\sigma$
and $\tau$ do.  Here are the types of the built-in connectives:
\begin{eqnarray*}
  \Imp     & :: & [prop,prop]\To prop \\
  \Forall  & :: & (\alpha{::}logic\To prop) \To prop \\
  {\equiv} & :: & [\alpha{::}\{\},\alpha]\To prop \\
  \qeq & :: & [\alpha{::}\{\},\alpha]\To prop c
\end{eqnarray*}
The restricted polymorphism in $\Forall$ excludes certain types, those used
just for parsing. 

In our formalization of first-order logic, we declared a type~$o$ of
object-level truth values, rather than using~$prop$ for this purpose.  If
we declared the object-level connectives to have types such as
${\neg}::prop\To prop$, then these connectives would be applicable to
meta-level formulae.  Keeping $prop$ and $o$ as separate types maintains
the distinction between the meta-level and the object-level.  To formalize
the inference rules, we shall need to relate the two levels; accordingly,
we declare the constant
\index{Trueprop@{$Trueprop$}}
\begin{eqnarray*}
  Trueprop & :: & o\To prop.
\end{eqnarray*}
We may regard $Trueprop$ as a meta-level predicate, reading $Trueprop(P)$ as
`$P$ is true at the object-level.'  Put another way, $Trueprop$ is a coercion
from $o$ to $prop$.


\subsection{Expressing propositional rules}
\index{rules!propositional|bold}
We shall illustrate the use of the meta-logic by formalizing the rules of
Fig.\ts\ref{fol-fig}.  Each object-level rule is expressed as a meta-level
axiom. 

One of the simplest rules is $(\conj E1)$.  Making
everything explicit, its formalization in the meta-logic is
$$ \Forall P\;Q. Trueprop(P\conj Q) \Imp Trueprop(P).   \eqno(\conj E1) $$
This may look formidable, but it has an obvious reading: for all object-level
truth values $P$ and~$Q$, if $P\conj Q$ is true then so is~$P$.  The
reading is correct because the meta-logic has simple models, where
types denote sets and $\Forall$ really means `for all.'

\index{Trueprop@{$Trueprop$}}
Isabelle adopts notational conventions to ease the writing of rules.  We may
hide the occurrences of $Trueprop$ by making it an implicit coercion.
Outer universal quantifiers may be dropped.  Finally, the nested implication
\[  \phi@1\Imp(\cdots \phi@n\Imp\psi\cdots) \]
may be abbreviated as $\List{\phi@1; \ldots; \phi@n} \Imp \psi$, which
formalizes a rule of $n$~premises.

Using these conventions, the conjunction rules become the following axioms.
These fully specify the properties of~$\conj$:
$$ \List{P; Q} \Imp P\conj Q                 \eqno(\conj I) $$
$$ P\conj Q \Imp P  \qquad  P\conj Q \Imp Q  \eqno(\conj E1,2) $$

\noindent
Next, consider the disjunction rules.  The discharge of assumption in
$(\disj E)$ is expressed  using $\Imp$:
$$ P \Imp P\disj Q  \qquad  Q \Imp P\disj Q  \eqno(\disj I1,2) $$
$$ \List{P\disj Q; P\Imp R; Q\Imp R} \Imp R  \eqno(\disj E) $$

\noindent
To understand this treatment of assumptions\index{assumptions} in natural
deduction, look at implication.  The rule $({\imp}I)$ is the classic
example of natural deduction: to prove that $P\imp Q$ is true, assume $P$
is true and show that $Q$ must then be true.  More concisely, if $P$
implies $Q$ (at the meta-level), then $P\imp Q$ is true (at the
object-level).  Showing the coercion explicitly, this is formalized as
\[ (Trueprop(P)\Imp Trueprop(Q)) \Imp Trueprop(P\imp Q). \]
The rule $({\imp}E)$ is straightforward; hiding $Trueprop$, the axioms to
specify $\imp$ are 
$$ (P \Imp Q)  \Imp  P\imp Q   \eqno({\imp}I) $$
$$ \List{P\imp Q; P}  \Imp Q.  \eqno({\imp}E) $$

\noindent
Finally, the intuitionistic contradiction rule is formalized as the axiom
$$ \bot \Imp P.   \eqno(\bot E) $$

\begin{warn}
Earlier versions of Isabelle, and certain
papers~\cite{paulson89,paulson700}, use $\List{P}$ to mean $Trueprop(P)$.
\index{Trueprop@{$Trueprop$}}
\end{warn}

\subsection{Quantifier rules and substitution}
\index{rules!quantifier|bold}\index{substitution|bold}
\index{variables!bound}
Isabelle expresses variable binding using $\lambda$-abstraction; for instance,
$\forall x.P$ is formalized as $\forall(\lambda x.P)$.  Recall that $F(t)$
is Isabelle's syntax for application of the function~$F$ to the argument~$t$;
it is not a meta-notation for substitution.  On the other hand, a substitution
will take place if $F$ has the form $\lambda x.P$;  Isabelle transforms
$(\lambda x.P)(t)$ to~$P[t/x]$ by $\beta$-conversion.  Thus, we can express
inference rules that involve substitution for bound variables.

\index{parameters|bold}\index{eigenvariables|see{parameters}}
A logic may attach provisos to certain of its rules, especially quantifier
rules.  We cannot hope to formalize arbitrary provisos.  Fortunately, those
typical of quantifier rules always have the same form, namely `$x$ not free in
\ldots {\it (some set of formulae)},' where $x$ is a variable (called a {\bf
parameter} or {\bf eigenvariable}) in some premise.  Isabelle treats
provisos using~$\Forall$, its inbuilt notion of `for all'.

\index{$\Forall$}
The purpose of the proviso `$x$ not free in \ldots' is
to ensure that the premise may not make assumptions about the value of~$x$,
and therefore holds for all~$x$.  We formalize $(\forall I)$ by
\[ \left(\Forall x. Trueprop(P(x))\right) \Imp Trueprop(\forall x.P(x)). \]
This means, `if $P(x)$ is true for all~$x$, then $\forall x.P(x)$ is true.'
The $\forall E$ rule exploits $\beta$-conversion.  Hiding $Trueprop$, the
$\forall$ axioms are
$$ \left(\Forall x. P(x)\right)  \Imp  \forall x.P(x)   \eqno(\forall I) $$
$$ (\forall x.P(x))  \Imp P(t).  \eqno(\forall E)$$

\noindent
We have defined the object-level universal quantifier~($\forall$)
using~$\Forall$.  But we do not require meta-level counterparts of all the
connectives of the object-logic!  Consider the existential quantifier: 
$$ P(t)  \Imp  \exists x.P(x)  \eqno(\exists I)$$
$$ \List{\exists x.P(x);\; \Forall x. P(x)\Imp Q} \Imp Q  \eqno(\exists E) $$
Let us verify $(\exists E)$ semantically.  Suppose that the premises
hold; since $\exists x.P(x)$ is true, we may choose $a$ such that $P(a)$ is
true.  Instantiating $\Forall x. P(x)\Imp Q$ with $a$ yields $P(a)\Imp Q$, and
we obtain the desired conclusion, $Q$.

The treatment of substitution deserves mention.  The rule
\[ \infer{P[u/t]}{t=u & P} \]
would be hard to formalize in Isabelle.  It calls for replacing~$t$ by $u$
throughout~$P$, which cannot be expressed using $\beta$-conversion.  Our
rule~$(subst)$ uses the occurrences of~$x$ in~$P$ as a template for
substitution, inferring $P[u/x]$ from~$P[t/x]$.  When we formalize this as
an axiom, the template becomes a function variable:
$$ \List{t=u; P(t)} \Imp P(u).  \eqno(subst)$$


\subsection{Signatures and theories}
\index{signatures|bold}\index{theories|bold}
A {\bf signature} contains the information necessary for type checking,
parsing and pretty printing.  It specifies classes and their
relationships; types, with their arities, and constants, with
their types.  It also contains syntax rules, specified using mixfix
declarations.

Two signatures can be merged provided their specifications are compatible ---
they must not, for example, assign different types to the same constant.
Under similar conditions, a signature can be extended.  Signatures are
managed internally by Isabelle; users seldom encounter them.

A {\bf theory} consists of a signature plus a collection of axioms.  The
{\bf pure} theory contains only the meta-logic.  Theories can be combined
provided their signatures are compatible.  A theory definition extends an
existing theory with further signature specifications --- classes, types,
constants and mixfix declarations --- plus a list of axioms, expressed as
strings to be parsed.  A theory can formalize a small piece of mathematics,
such as lists and their operations, or an entire logic.  A mathematical
development typically involves many theories in a hierarchy.  For example,
the pure theory could be extended to form a theory for
Fig.\ts\ref{fol-fig}; this could be extended in two separate ways to form a
theory for natural numbers and a theory for lists; the union of these two
could be extended into a theory defining the length of a list:
\begin{tt}
\[
\begin{array}{c@{}c@{}c@{}c@{}c}
     {}   &     {} & \hbox{Length} &  {}   &     {}   \\
     {}   &     {}   &  \uparrow &     {}   &     {}   \\
     {}   &     {} &\hbox{Nat}+\hbox{List}&  {}   &     {}   \\
     {}   & \nearrow &     {}    & \nwarrow &     {}   \\
 \hbox{Nat} &   {}   &     {}    &     {}   & \hbox{List} \\
     {}   & \nwarrow &     {}    & \nearrow &     {}   \\
     {}   &     {}   &\hbox{FOL} &     {}   &     {}   \\
     {}   &     {}   &  \uparrow &     {}   &     {}   \\
     {}   &     {}   &\hbox{Pure}&     {}  &     {}
\end{array}
\]
\end{tt}
Each Isabelle proof typically works within a single theory, which is
associated with the proof state.  However, many different theories may
coexist at the same time, and you may work in each of these during a single
session.  

\begin{warn}
  Confusing problems arise if you work in the wrong theory.  Each theory
  defines its own syntax.  An identifier may be regarded in one theory as a
  constant and in another as a variable.
\end{warn}

\section{Proof construction in Isabelle}
\index{Isabelle!proof construction in|bold} 

I have elsewhere described the meta-logic and demonstrated it by
formalizing first-order logic~\cite{paulson89}.  There is a one-to-one
correspondence between meta-level proofs and object-level proofs.  To each
use of a meta-level axiom, such as $(\forall I)$, there is a use of the
corresponding object-level rule.  Object-level assumptions and parameters
have meta-level counterparts.  The meta-level formalization is {\bf
  faithful}, admitting no incorrect object-level inferences, and {\bf
  adequate}, admitting all correct object-level inferences.  These
properties must be demonstrated separately for each object-logic.

The meta-logic is defined by a collection of inference rules, including
equational rules for the $\lambda$-calculus, and logical rules.  The rules
for~$\Imp$ and~$\Forall$ resemble those for~$\imp$ and~$\forall$ in
Fig.\ts\ref{fol-fig}.  Proofs performed using the primitive meta-rules
would be lengthy; Isabelle proofs normally use certain derived rules.
{\bf Resolution}, in particular, is convenient for backward proof.

Unification is central to theorem proving.  It supports quantifier
reasoning by allowing certain `unknown' terms to be instantiated later,
possibly in stages.  When proving that the time required to sort $n$
integers is proportional to~$n^2$, we need not state the constant of
proportionality; when proving that a hardware adder will deliver the sum of
its inputs, we need not state how many clock ticks will be required.  Such
quantities often emerge from the proof.

Isabelle provides {\bf schematic variables}, or \bfindex{unknowns}, for
unification.  Logically, unknowns are free variables.  But while ordinary
variables remain fixed, unification may instantiate unknowns.  Unknowns are
written with a ?\ prefix and are frequently subscripted: $\Var{a}$,
$\Var{a@1}$, $\Var{a@2}$, \ldots, $\Var{P}$, $\Var{P@1}$, \ldots.

Recall that an inference rule of the form
\[ \infer{\phi}{\phi@1 & \ldots & \phi@n} \]
is formalized in Isabelle's meta-logic as the axiom
$\List{\phi@1; \ldots; \phi@n} \Imp \phi$.
Such axioms resemble Prolog's Horn clauses, and can be combined by
resolution --- Isabelle's principal proof method.  Resolution yields both
forward and backward proof.  Backward proof works by unifying a goal with
the conclusion of a rule, whose premises become new subgoals.  Forward proof
works by unifying theorems with the premises of a rule, deriving a new theorem.

Isabelle axioms require an extended notion of resolution.
They differ from Horn clauses in two major respects:
\begin{itemize}
  \item They are written in the typed $\lambda$-calculus, and therefore must be
resolved using higher-order unification.

\item The constituents of a clause need not be atomic formulae.  Any
  formula of the form $Trueprop(\cdots)$ is atomic, but axioms such as
  ${\imp}I$ and $\forall I$ contain non-atomic formulae.
\end{itemize}
Isabelle has little in common with classical resolution theorem provers
such as Otter~\cite{wos-bledsoe}.  At the meta-level, Isabelle proves
theorems in their positive form, not by refutation.  However, an
object-logic that includes a contradiction rule may employ a refutation
proof procedure.


\subsection{Higher-order unification}
\index{unification!higher-order|bold}
Unification is equation solving.  The solution of $f(\Var{x},c) \qeq
f(d,\Var{y})$ is $\Var{x}\equiv d$ and $\Var{y}\equiv c$.  {\bf
Higher-order unification} is equation solving for typed $\lambda$-terms.
To handle $\beta$-conversion, it must reduce $(\lambda x.t)u$ to $t[u/x]$.
That is easy --- in the typed $\lambda$-calculus, all reduction sequences
terminate at a normal form.  But it must guess the unknown
function~$\Var{f}$ in order to solve the equation
\begin{equation} \label{hou-eqn}
 \Var{f}(t) \qeq g(u@1,\ldots,u@k).
\end{equation}
Huet's~\cite{huet75} search procedure solves equations by imitation and
projection.  {\bf Imitation}\index{imitation|bold} makes~$\Var{f}$ apply
leading symbol (if a constant) of the right-hand side.  To solve
equation~(\ref{hou-eqn}), it guesses
\[ \Var{f} \equiv \lambda x. g(\Var{h@1}(x),\ldots,\Var{h@k}(x)), \]
where $\Var{h@1}$, \ldots, $\Var{h@k}$ are new unknowns.  Assuming there are no
other occurrences of~$\Var{f}$, equation~(\ref{hou-eqn}) simplifies to the
set of equations
\[ \Var{h@1}(t)\qeq u@1 \quad\ldots\quad \Var{h@k}(t)\qeq u@k. \]
If the procedure solves these equations, instantiating $\Var{h@1}$, \ldots,
$\Var{h@k}$, then it yields an instantiation for~$\Var{f}$.

\index{projection|bold}
{\bf Projection} makes $\Var{f}$ apply one of its arguments.  To solve
equation~(\ref{hou-eqn}), if $t$ expects~$m$ arguments and delivers a
result of suitable type, it guesses
\[ \Var{f} \equiv \lambda x. x(\Var{h@1}(x),\ldots,\Var{h@m}(x)), \]
where $\Var{h@1}$, \ldots, $\Var{h@m}$ are new unknowns.  Assuming there are no
other occurrences of~$\Var{f}$, equation~(\ref{hou-eqn}) simplifies to the 
equation 
\[ t(\Var{h@1}(t),\ldots,\Var{h@m}(t)) \qeq g(u@1,\ldots,u@k). $$ 

\begin{warn}
Huet's unification procedure is complete.  Isabelle's polymorphic version,
which solves for type unknowns as well as for term unknowns, is incomplete.
The problem is that projection requires type information.  In
equation~(\ref{hou-eqn}), if the type of~$t$ is unknown, then projections
are possible for all~$m\geq0$, and the types of the $\Var{h@i}$ will be
similarly unconstrained.  Therefore, Isabelle never attempts such
projections, and may fail to find unifiers where a type unknown turns out
to be a function type.
\end{warn}

\index{unknowns!of function type|bold}
Given $\Var{f}(t@1,\ldots,t@n)\qeq u$, Huet's procedure could make up to
$n+1$ guesses.  The search tree and set of unifiers may be infinite.  But
higher-order unification can work effectively, provided you are careful
with {\bf function unknowns}:
\begin{itemize}
  \item Equations with no function unknowns are solved using first-order
unification, extended to treat bound variables.  For example, $\lambda x.x
\qeq \lambda x.\Var{y}$ has no solution because $\Var{y}\equiv x$ would
capture the free variable~$x$.

  \item An occurrence of the term $\Var{f}(x,y,z)$, where the arguments are
distinct bound variables, causes no difficulties.  Its projections can only
match the corresponding variables.

  \item Even an equation such as $\Var{f}(a)\qeq a+a$ is all right.  It has
four solutions, but Isabelle evaluates them lazily, trying projection before
imitation. The first solution is usually the one desired:
\[ \Var{f}\equiv \lambda x. x+x \quad
   \Var{f}\equiv \lambda x. a+x \quad
   \Var{f}\equiv \lambda x. x+a \quad
   \Var{f}\equiv \lambda x. a+a \]
  \item  Equations such as $\Var{f}(\Var{x},\Var{y})\qeq t$ and
$\Var{f}(\Var{g}(x))\qeq t$ admit vast numbers of unifiers, and must be
avoided. 
\end{itemize}
In problematic cases, you may have to instantiate some unknowns before
invoking unification. 


\subsection{Joining rules by resolution} \label{joining}
\index{resolution|bold}
Let $\List{\psi@1; \ldots; \psi@m} \Imp \psi$ and $\List{\phi@1; \ldots;
\phi@n} \Imp \phi$ be two Isabelle theorems, representing object-level rules. 
Choosing some~$i$ from~1 to~$n$, suppose that $\psi$ and $\phi@i$ have a
higher-order unifier.  Writing $Xs$ for the application of substitution~$s$ to
expression~$X$, this means there is some~$s$ such that $\psi s\equiv \phi@i s$.
By resolution, we may conclude
\[ (\List{\phi@1; \ldots; \phi@{i-1}; \psi@1; \ldots; \psi@m;
          \phi@{i+1}; \ldots; \phi@n} \Imp \phi)s.
\]
The substitution~$s$ may instantiate unknowns in both rules.  In short,
resolution is the following rule:
\[ \infer[(\psi s\equiv \phi@i s)]
         {(\List{\phi@1; \ldots; \phi@{i-1}; \psi@1; \ldots; \psi@m;
          \phi@{i+1}; \ldots; \phi@n} \Imp \phi)s}
         {\List{\psi@1; \ldots; \psi@m} \Imp \psi & &
          \List{\phi@1; \ldots; \phi@n} \Imp \phi}
\]
It operates at the meta-level, on Isabelle theorems, and is justified by
the properties of $\Imp$ and~$\Forall$.  It takes the number~$i$ (for
$1\leq i\leq n$) as a parameter and may yield infinitely many conclusions,
one for each unifier of $\psi$ with $\phi@i$.  Isabelle returns these
conclusions as a sequence (lazy list).

Resolution expects the rules to have no outer quantifiers~($\Forall$).  It
may rename or instantiate any schematic variables, but leaves free
variables unchanged.  When constructing a theory, Isabelle puts the rules
into a standard form containing no free variables; for instance, $({\imp}E)$
becomes
\[ \List{\Var{P}\imp \Var{Q}; \Var{P}}  \Imp \Var{Q}. 
\]
When resolving two rules, the unknowns in the first rule are renamed, by
subscripting, to make them distinct from the unknowns in the second rule.  To
resolve $({\imp}E)$ with itself, the first copy of the rule would become
\[ \List{\Var{P@1}\imp \Var{Q@1}; \Var{P@1}}  \Imp \Var{Q@1}. \]
Resolving this with $({\imp}E)$ in the first premise, unifying $\Var{Q@1}$ with
$\Var{P}\imp \Var{Q}$, is the meta-level inference
\[ \infer{\List{\Var{P@1}\imp (\Var{P}\imp \Var{Q}); \Var{P@1}; \Var{P}} 
           \Imp\Var{Q}.}
         {\List{\Var{P@1}\imp \Var{Q@1}; \Var{P@1}}  \Imp \Var{Q@1} & &
          \List{\Var{P}\imp \Var{Q}; \Var{P}}  \Imp \Var{Q}}
\]
Renaming the unknowns in the resolvent, we have derived the
object-level rule
\[ \infer{Q.}{R\imp (P\imp Q)  &  R  &  P}  \]
\index{rules!derived}
Joining rules in this fashion is a simple way of proving theorems.  The
derived rules are conservative extensions of the object-logic, and may permit
simpler proofs.  Let us consider another example.  Suppose we have the axiom
$$ \forall x\,y. Suc(x)=Suc(y)\imp x=y. \eqno (inject) $$

\noindent 
The standard form of $(\forall E)$ is
$\forall x.\Var{P}(x)  \Imp \Var{P}(\Var{t})$.
Resolving $(inject)$ with $(\forall E)$ replaces $\Var{P}$ by
$\lambda x. \forall y. Suc(x)=Suc(y)\imp x=y$ and leaves $\Var{t}$
unchanged, yielding  
\[ \forall y. Suc(\Var{t})=Suc(y)\imp \Var{t}=y. \]
Resolving this with $(\forall E)$ puts a subscript on~$\Var{t}$
and yields
\[ Suc(\Var{t@1})=Suc(\Var{t})\imp \Var{t@1}=\Var{t}. \]
Resolving this with $({\imp}E)$ increases the subscripts and yields
\[ Suc(\Var{t@2})=Suc(\Var{t@1})\Imp \Var{t@2}=\Var{t@1}. 
\]
We have derived the rule
\[ \infer{m=n,}{Suc(m)=Suc(n)} \]
which goes directly from $Suc(m)=Suc(n)$ to $m=n$.  It is handy for simplifying
an equation like $Suc(Suc(Suc(m)))=Suc(Suc(Suc(0)))$.  


\section{Lifting a rule into a context}
The rules $({\imp}I)$ and $(\forall I)$ may seem unsuitable for
resolution.  They have non-atomic premises, namely $P\Imp Q$ and $\Forall
x.P(x)$, while the conclusions of all the rules are atomic (they have the form
$Trueprop(\cdots)$).  Isabelle gets round the problem through a meta-inference
called \bfindex{lifting}.  Let us consider how to construct proofs such as
\[ \infer[({\imp}I)]{P\imp(Q\imp R)}
         {\infer[({\imp}I)]{Q\imp R}
                        {\infer*{R}{[P,Q]}}}
   \qquad
   \infer[(\forall I)]{\forall x\,y.P(x,y)}
         {\infer[(\forall I)]{\forall y.P(x,y)}{P(x,y)}}
\]

\subsection{Lifting over assumptions}
\index{lifting!over assumptions|bold}
Lifting over $\theta\Imp{}$ is the following meta-inference rule:
\[ \infer{\List{\theta\Imp\phi@1; \ldots; \theta\Imp\phi@n} \Imp
          (\theta \Imp \phi)}
         {\List{\phi@1; \ldots; \phi@n} \Imp \phi} \]
This is clearly sound: if $\List{\phi@1; \ldots; \phi@n} \Imp \phi$ is true and
$\theta\Imp\phi@1$, \ldots, $\theta\Imp\phi@n$ and $\theta$ are all true then
$\phi$ must be true.  Iterated lifting over a series of meta-formulae
$\theta@k$, \ldots, $\theta@1$ yields an object-rule whose conclusion is
$\List{\theta@1; \ldots; \theta@k} \Imp \phi$.  Typically the $\theta@i$ are
the assumptions in a natural deduction proof; lifting copies them into a rule's
premises and conclusion.

When resolving two rules, Isabelle lifts the first one if necessary.  The
standard form of $({\imp}I)$ is
\[ (\Var{P} \Imp \Var{Q})  \Imp  \Var{P}\imp \Var{Q}.   \]
To resolve this rule with itself, Isabelle modifies one copy as follows: it
renames the unknowns to $\Var{P@1}$ and $\Var{Q@1}$, then lifts the rule over
$\Var{P}\Imp{}$ to obtain
\[ (\Var{P}\Imp (\Var{P@1} \Imp \Var{Q@1})) \Imp (\Var{P} \Imp 
   (\Var{P@1}\imp \Var{Q@1})).   \]
Using the $\List{\cdots}$ abbreviation, this can be written as
\[ \List{\List{\Var{P}; \Var{P@1}} \Imp \Var{Q@1}; \Var{P}} 
   \Imp  \Var{P@1}\imp \Var{Q@1}.   \]
Unifying $\Var{P}\Imp \Var{P@1}\imp\Var{Q@1}$ with $\Var{P} \Imp
\Var{Q}$ instantiates $\Var{Q}$ to ${\Var{P@1}\imp\Var{Q@1}}$.
Resolution yields
\[ (\List{\Var{P}; \Var{P@1}} \Imp \Var{Q@1}) \Imp
\Var{P}\imp(\Var{P@1}\imp\Var{Q@1}).   \]
This represents the derived rule
\[ \infer{P\imp(Q\imp R).}{\infer*{R}{[P,Q]}} \]

\subsection{Lifting over parameters}
\index{lifting!over parameters|bold}
An analogous form of lifting handles premises of the form $\Forall x\ldots\,$. 
Here, lifting prefixes an object-rule's premises and conclusion with $\Forall
x$.  At the same time, lifting introduces a dependence upon~$x$.  It replaces
each unknown $\Var{a}$ in the rule by $\Var{a'}(x)$, where $\Var{a'}$ is a new
unknown (by subscripting) of suitable type --- necessarily a function type.  In
short, lifting is the meta-inference
\[ \infer{\List{\Forall x.\phi@1^x; \ldots; \Forall x.\phi@n^x} 
           \Imp \Forall x.\phi^x,}
         {\List{\phi@1; \ldots; \phi@n} \Imp \phi} \]
%
where $\phi^x$ stands for the result of lifting unknowns over~$x$ in
$\phi$.  It is not hard to verify that this meta-inference is sound.  If
$\phi\Imp\psi$ then $\phi^x\Imp\psi^x$ for all~$x$; so if $\phi^x$ is true
for all~$x$ then so is $\psi^x$.  Thus, from $\phi\Imp\psi$ we conclude
$(\Forall x.\phi^x) \Imp (\Forall x.\psi^x)$.

For example, $(\disj I)$ might be lifted to
\[ (\Forall x.\Var{P@1}(x)) \Imp (\Forall x. \Var{P@1}(x)\disj \Var{Q@1}(x))\]
and $(\forall I)$ to
\[ (\Forall x\,y.\Var{P@1}(x,y)) \Imp (\Forall x. \forall y.\Var{P@1}(x,y)). \]
Isabelle has renamed a bound variable in $(\forall I)$ from $x$ to~$y$,
avoiding a clash.  Resolving the above with $(\forall I)$ is the meta-inference
\[ \infer{\Forall x\,y.\Var{P@1}(x,y)) \Imp \forall x\,y.\Var{P@1}(x,y)) }
         {(\Forall x\,y.\Var{P@1}(x,y)) \Imp 
               (\Forall x. \forall y.\Var{P@1}(x,y))  &
          (\Forall x.\Var{P}(x)) \Imp (\forall x.\Var{P}(x))} \]
Here, $\Var{P}$ is replaced by $\lambda x.\forall y.\Var{P@1}(x,y)$; the
resolvent expresses the derived rule
\[ \vcenter{ \infer{\forall x\,y.Q(x,y)}{Q(x,y)} }
   \quad\hbox{provided $x$, $y$ not free in the assumptions} 
\] 
I discuss lifting and parameters at length elsewhere~\cite{paulson89}.
Miller goes into even greater detail~\cite{miller-mixed}.


\section{Backward proof by resolution}
\index{resolution!in backward proof}\index{proof!backward|bold} 
\index{tactics}\index{tacticals}

Resolution is convenient for deriving simple rules and for reasoning
forward from facts.  It can also support backward proof, where we start
with a goal and refine it to progressively simpler subgoals until all have
been solved.  {\sc lcf} and its descendants {\sc hol} and Nuprl provide
tactics and tacticals, which constitute a high-level language for
expressing proof searches.  {\bf Tactics} refine subgoals while {\bf
  tacticals} combine tactics.

\index{LCF}
Isabelle's tactics and tacticals work differently from {\sc lcf}'s.  An
Isabelle rule is bidirectional: there is no distinction between
inputs and outputs.  {\sc lcf} has a separate tactic for each rule;
Isabelle performs refinement by any rule in a uniform fashion, using
resolution.

\index{subgoals|bold}\index{main goal|bold}
Isabelle works with meta-level theorems of the form
\( \List{\phi@1; \ldots; \phi@n} \Imp \phi \).
We have viewed this as the {\bf rule} with premises
$\phi@1$,~\ldots,~$\phi@n$ and conclusion~$\phi$.  It can also be viewed as
the \bfindex{proof state} with subgoals $\phi@1$,~\ldots,~$\phi@n$ and main
goal~$\phi$.

To prove the formula~$\phi$, take $\phi\Imp \phi$ as the initial proof
state.  This assertion is, trivially, a theorem.  At a later stage in the
backward proof, a typical proof state is $\List{\phi@1; \ldots; \phi@n}
\Imp \phi$.  This proof state is a theorem, ensuring that the subgoals
$\phi@1$,~\ldots,~$\phi@n$ imply~$\phi$.  If $n=0$ then we have
proved~$\phi$ outright.  If $\phi$ contains unknowns, they may become
instantiated during the proof; a proof state may be $\List{\phi@1; \ldots;
\phi@n} \Imp \phi'$, where $\phi'$ is an instance of~$\phi$.

\subsection{Refinement by resolution}
\index{refinement|bold}
To refine subgoal~$i$ of a proof state by a rule, perform the following
resolution: 
\[ \infer{\hbox{new proof state}}{\hbox{rule} & & \hbox{proof state}} \]
If the rule is $\List{\psi'@1; \ldots; \psi'@m} \Imp \psi'$ after lifting
over subgoal~$i$, and the proof state is $\List{\phi@1; \ldots; \phi@n}
\Imp \phi$, then the new proof state is (for~$1\leq i\leq n$)
\[ (\List{\phi@1; \ldots; \phi@{i-1}; \psi'@1;
          \ldots; \psi'@m; \phi@{i+1}; \ldots; \phi@n} \Imp \phi)s.  \]
Substitution~$s$ unifies $\psi'$ with~$\phi@i$.  In the proof state,
subgoal~$i$ is replaced by $m$ new subgoals, the rule's instantiated premises.
If some of the rule's unknowns are left un-instantiated, they become new
unknowns in the proof state.  Refinement by~$(\exists I)$, namely
\[ \Var{P}(\Var{t}) \Imp \exists x. \Var{P}(x), \]
inserts a new unknown derived from~$\Var{t}$ by subscripting and lifting.
We do not have to specify an `existential witness' when
applying~$(\exists I)$.  Further resolutions may instantiate unknowns in
the proof state.

\subsection{Proof by assumption}
\index{proof!by assumption|bold}
In the course of a natural deduction proof, parameters $x@1$, \ldots,~$x@l$ and
assumptions $\theta@1$, \ldots, $\theta@k$ accumulate, forming a context for
each subgoal.  Repeated lifting steps can lift a rule into any context.  To
aid readability, Isabelle puts contexts into a normal form, gathering the
parameters at the front:
\begin{equation} \label{context-eqn}
\Forall x@1 \ldots x@l. \List{\theta@1; \ldots; \theta@k}\Imp\theta. 
\end{equation}
Under the usual reading of the connectives, this expresses that $\theta$
follows from $\theta@1$,~\ldots~$\theta@k$ for arbitrary
$x@1$,~\ldots,~$x@l$.  It is trivially true if $\theta$ equals any of
$\theta@1$,~\ldots~$\theta@k$, or is unifiable with any of them.  This
models proof by assumption in natural deduction.

Isabelle automates the meta-inference for proof by assumption.  Its arguments
are the meta-theorem $\List{\phi@1; \ldots; \phi@n} \Imp \phi$, and some~$i$
from~1 to~$n$, where $\phi@i$ has the form~(\ref{context-eqn}).  Its results
are meta-theorems of the form
\[ (\List{\phi@1; \ldots; \phi@{i-1}; \phi@{i+1}; \phi@n} \Imp \phi)s \]
for each $s$ and~$j$ such that $s$ unifies $\lambda x@1 \ldots x@l. \theta@j$
with $\lambda x@1 \ldots x@l. \theta$.  Isabelle supplies the parameters
$x@1$,~\ldots,~$x@l$ to higher-order unification as bound variables, which
regards them as unique constants with a limited scope --- this enforces
parameter provisos~\cite{paulson89}.

The premise represents a proof state with~$n$ subgoals, of which the~$i$th
is to be solved by assumption.  Isabelle searches the subgoal's context for
an assumption~$\theta@j$ that can solve it.  For each unifier, the
meta-inference returns an instantiated proof state from which the $i$th
subgoal has been removed.  Isabelle searches for a unifying assumption; for
readability and robustness, proofs do not refer to assumptions by number.

Consider the proof state 
\[ (\List{P(a); P(b)} \Imp P(\Var{x})) \Imp Q(\Var{x}). \]
Proof by assumption (with $i=1$, the only possibility) yields two results:
\begin{itemize}
  \item $Q(a)$, instantiating $\Var{x}\equiv a$
  \item $Q(b)$, instantiating $\Var{x}\equiv b$
\end{itemize}
Here, proof by assumption affects the main goal.  It could also affect
other subgoals; if we also had the subgoal ${\List{P(b); P(c)} \Imp
  P(\Var{x})}$, then $\Var{x}\equiv a$ would transform it to ${\List{P(b);
    P(c)} \Imp P(a)}$, which might be unprovable.


\subsection{A propositional proof} \label{prop-proof}
\index{examples!propositional}
Our first example avoids quantifiers.  Given the main goal $P\disj P\imp
P$, Isabelle creates the initial state
\[ (P\disj P\imp P)\Imp (P\disj P\imp P). \] 
%
Bear in mind that every proof state we derive will be a meta-theorem,
expressing that the subgoals imply the main goal.  Our aim is to reach the
state $P\disj P\imp P$; this meta-theorem is the desired result.

The first step is to refine subgoal~1 by (${\imp}I)$, creating a new state
where $P\disj P$ is an assumption:
\[ (P\disj P\Imp P)\Imp (P\disj P\imp P) \]
The next step is $(\disj E)$, which replaces subgoal~1 by three new subgoals. 
Because of lifting, each subgoal contains a copy of the context --- the
assumption $P\disj P$.  (In fact, this assumption is now redundant; we shall
shortly see how to get rid of it!)  The new proof state is the following
meta-theorem, laid out for clarity:
\[ \begin{array}{l@{}l@{\qquad\qquad}l} 
  \lbrakk\;& P\disj P\Imp \Var{P@1}\disj\Var{Q@1}; & \hbox{(subgoal 1)} \\
           & \List{P\disj P; \Var{P@1}} \Imp P;    & \hbox{(subgoal 2)} \\
           & \List{P\disj P; \Var{Q@1}} \Imp P     & \hbox{(subgoal 3)} \\
  \rbrakk\;& \Imp (P\disj P\imp P)                 & \hbox{(main goal)}
   \end{array} 
\]
Notice the unknowns in the proof state.  Because we have applied $(\disj E)$,
we must prove some disjunction, $\Var{P@1}\disj\Var{Q@1}$.  Of course,
subgoal~1 is provable by assumption.  This instantiates both $\Var{P@1}$ and
$\Var{Q@1}$ to~$P$ throughout the proof state:
\[ \begin{array}{l@{}l@{\qquad\qquad}l} 
    \lbrakk\;& \List{P\disj P; P} \Imp P; & \hbox{(subgoal 1)} \\
             & \List{P\disj P; P} \Imp P  & \hbox{(subgoal 2)} \\
    \rbrakk\;& \Imp (P\disj P\imp P)      & \hbox{(main goal)}
   \end{array} \]
Both of the remaining subgoals can be proved by assumption.  After two such
steps, the proof state is $P\disj P\imp P$.


\subsection{A quantifier proof}
\index{examples!with quantifiers}
To illustrate quantifiers and $\Forall$-lifting, let us prove
$(\exists x.P(f(x)))\imp(\exists x.P(x))$.  The initial proof
state is the trivial meta-theorem 
\[ (\exists x.P(f(x)))\imp(\exists x.P(x)) \Imp 
   (\exists x.P(f(x)))\imp(\exists x.P(x)). \]
As above, the first step is refinement by (${\imp}I)$: 
\[ (\exists x.P(f(x))\Imp \exists x.P(x)) \Imp 
   (\exists x.P(f(x)))\imp(\exists x.P(x)) 
\]
The next step is $(\exists E)$, which replaces subgoal~1 by two new subgoals.
Both have the assumption $\exists x.P(f(x))$.  The new proof
state is the meta-theorem  
\[ \begin{array}{l@{}l@{\qquad\qquad}l} 
   \lbrakk\;& \exists x.P(f(x)) \Imp \exists x.\Var{P@1}(x); & \hbox{(subgoal 1)} \\
            & \Forall x.\List{\exists x.P(f(x)); \Var{P@1}(x)} \Imp 
                       \exists x.P(x)  & \hbox{(subgoal 2)} \\
    \rbrakk\;& \Imp (\exists x.P(f(x)))\imp(\exists x.P(x))  & \hbox{(main goal)}
   \end{array} 
\]
The unknown $\Var{P@1}$ appears in both subgoals.  Because we have applied
$(\exists E)$, we must prove $\exists x.\Var{P@1}(x)$, where $\Var{P@1}(x)$ may
become any formula possibly containing~$x$.  Proving subgoal~1 by assumption
instantiates $\Var{P@1}$ to~$\lambda x.P(f(x))$:  
\[ \left(\Forall x.\List{\exists x.P(f(x)); P(f(x))} \Imp 
         \exists x.P(x)\right) 
   \Imp (\exists x.P(f(x)))\imp(\exists x.P(x)) 
\]
The next step is refinement by $(\exists I)$.  The rule is lifted into the
context of the parameter~$x$ and the assumption $P(f(x))$.  This copies
the context to the subgoal and allows the existential witness to
depend upon~$x$: 
\[ \left(\Forall x.\List{\exists x.P(f(x)); P(f(x))} \Imp 
         P(\Var{x@2}(x))\right) 
   \Imp (\exists x.P(f(x)))\imp(\exists x.P(x)) 
\]
The existential witness, $\Var{x@2}(x)$, consists of an unknown
applied to a parameter.  Proof by assumption unifies $\lambda x.P(f(x))$ 
with $\lambda x.P(\Var{x@2}(x))$, instantiating $\Var{x@2}$ to $f$.  The final
proof state contains no subgoals: $(\exists x.P(f(x)))\imp(\exists x.P(x))$.


\subsection{Tactics and tacticals}
\index{tactics|bold}\index{tacticals|bold}
{\bf Tactics} perform backward proof.  Isabelle tactics differ from those
of {\sc lcf}, {\sc hol} and Nuprl by operating on entire proof states,
rather than on individual subgoals.  An Isabelle tactic is a function that
takes a proof state and returns a sequence (lazy list) of possible
successor states.  Lazy lists are coded in ML as functions, a standard
technique~\cite{paulson91}.  Isabelle represents proof states by theorems.

Basic tactics execute the meta-rules described above, operating on a
given subgoal.  The {\bf resolution tactics} take a list of rules and
return next states for each combination of rule and unifier.  The {\bf
assumption tactic} examines the subgoal's assumptions and returns next
states for each combination of assumption and unifier.  Lazy lists are
essential because higher-order resolution may return infinitely many
unifiers.  If there are no matching rules or assumptions then no next
states are generated; a tactic application that returns an empty list is
said to {\bf fail}.

Sequences realize their full potential with {\bf tacticals} --- operators
for combining tactics.  Depth-first search, breadth-first search and
best-first search (where a heuristic function selects the best state to
explore) return their outcomes as a sequence.  Isabelle provides such
procedures in the form of tacticals.  Simpler procedures can be expressed
directly using the basic tacticals {\it THEN}, {\it ORELSE} and {\it REPEAT}:
\begin{description}
\item[$tac1\;THEN\;tac2$] is a tactic for sequential composition.  Applied
to a proof state, it returns all states reachable in two steps by applying
$tac1$ followed by~$tac2$.

\item[$tac1\;ORELSE\;tac2$] is a choice tactic.  Applied to a state, it
tries~$tac1$ and returns the result if non-empty; otherwise, it uses~$tac2$.

\item[$REPEAT\;tac$] is a repetition tactic.  Applied to a state, it
returns all states reachable by applying~$tac$ as long as possible (until
it would fail).  $REPEAT$ can be expressed in a few lines of \ML{} using
{\it THEN} and~{\it ORELSE}.
\end{description}
For instance, this tactic repeatedly applies $tac1$ and~$tac2$, giving
$tac1$ priority:
\[ REPEAT(tac1\;ORELSE\;tac2) \]


\section{Variations on resolution}
In principle, resolution and proof by assumption suffice to prove all
theorems.  However, specialized forms of resolution are helpful for working
with elimination rules.  Elim-resolution applies an elimination rule to an
assumption; destruct-resolution is similar, but applies a rule in a forward
style.

The last part of the section shows how the techniques for proving theorems
can also serve to derive rules.

\subsection{Elim-resolution}
\index{elim-resolution|bold}
Consider proving the theorem $((R\disj R)\disj R)\disj R\imp R$.  By
$({\imp}I)$, we prove $R$ from the assumption $((R\disj R)\disj R)\disj R$.
Applying $(\disj E)$ to this assumption yields two subgoals, one that
assumes~$R$ (and is therefore trivial) and one that assumes $(R\disj
R)\disj R$.  This subgoal admits another application of $(\disj E)$.  Since
natural deduction never discards assumptions, we eventually generate a
subgoal containing much that is redundant:
\[ \List{((R\disj R)\disj R)\disj R; (R\disj R)\disj R; R\disj R; R} \Imp R. \]
In general, using $(\disj E)$ on the assumption $P\disj Q$ creates two new
subgoals with the additional assumption $P$ or~$Q$.  In these subgoals,
$P\disj Q$ is redundant.  It wastes space; worse, it could make a theorem
prover repeatedly apply~$(\disj E)$, looping.  Other elimination rules pose
similar problems.  In first-order logic, only universally quantified
assumptions are sometimes needed more than once --- say, to prove
$P(f(f(a)))$ from the assumptions $\forall x.P(x)\imp P(f(x))$ and~$P(a)$.

Many logics can be formulated as sequent calculi that delete redundant
assumptions after use.  The rule $(\disj E)$ might become
\[ \infer[\disj\hbox{-left}]
         {\Gamma,P\disj Q,\Delta \turn \Theta}
         {\Gamma,P,\Delta \turn \Theta && \Gamma,Q,\Delta \turn \Theta}  \] 
In backward proof, a goal containing $P\disj Q$ on the left of the~$\turn$
(that is, as an assumption) splits into two subgoals, replacing $P\disj Q$
by $P$ or~$Q$.  But the sequent calculus, with its explicit handling of
assumptions, can be tiresome to use.

Elim-resolution is Isabelle's way of getting sequent calculus behaviour
from natural deduction rules.  It lets an elimination rule consume an
assumption.  Elim-resolution combines two meta-theorems:
\begin{itemize}
  \item a rule $\List{\psi@1; \ldots; \psi@m} \Imp \psi$
  \item a proof state $\List{\phi@1; \ldots; \phi@n} \Imp \phi$
\end{itemize}
The rule must have at least one premise, thus $m>0$.  Write the rule's
lifted form as $\List{\psi'@1; \ldots; \psi'@m} \Imp \psi'$.  Suppose we
wish to change subgoal number~$i$.

Ordinary resolution would attempt to reduce~$\phi@i$,
replacing subgoal~$i$ by $m$ new ones.  Elim-resolution tries
simultaneously to reduce~$\phi@i$ and to solve~$\psi'@1$ by assumption; it
returns a sequence of next states.  Each of these replaces subgoal~$i$ by
instances of $\psi'@2$, \ldots, $\psi'@m$ from which the selected
assumption has been deleted.  Suppose $\phi@i$ has the parameter~$x$ and
assumptions $\theta@1$,~\ldots,~$\theta@k$.  Then $\psi'@1$, the rule's first
premise after lifting, will be
\( \Forall x. \List{\theta@1; \ldots; \theta@k}\Imp \psi^{x}@1 \).
Elim-resolution tries to unify $\psi'\qeq\phi@i$ and
$\lambda x. \theta@j \qeq \lambda x. \psi^{x}@1$ simultaneously, for
$j=1$,~\ldots,~$k$. 

Let us redo the example from~\S\ref{prop-proof}.  The elimination rule
is~$(\disj E)$,
\[ \List{\Var{P}\disj \Var{Q};\; \Var{P}\Imp \Var{R};\; \Var{Q}\Imp \Var{R}}
      \Imp \Var{R}  \]
and the proof state is $(P\disj P\Imp P)\Imp (P\disj P\imp P)$.  The
lifted rule would be
\[ \begin{array}{l@{}l}
  \lbrakk\;& P\disj P \Imp \Var{P@1}\disj\Var{Q@1}; \\
           & \List{P\disj P ;\; \Var{P@1}} \Imp \Var{R@1};    \\
           & \List{P\disj P ;\; \Var{Q@1}} \Imp \Var{R@1}     \\
  \rbrakk\;& \Imp \Var{R@1}
  \end{array} 
\]
Unification would take the simultaneous equations
$P\disj P \qeq \Var{P@1}\disj\Var{Q@1}$ and $\Var{R@1} \qeq P$, yielding
$\Var{P@1}\equiv\Var{Q@1}\equiv\Var{R@1} \equiv P$.  The new proof state
would be simply
\[ \List{P \Imp P;\; P \Imp P} \Imp (P\disj P\imp P). 
\]
Elim-resolution's simultaneous unification gives better control
than ordinary resolution.  Recall the substitution rule:
$$ \List{\Var{t}=\Var{u}; \Var{P}(\Var{t})} \Imp \Var{P}(\Var{u}) 
   \eqno(subst) $$
Unsuitable for ordinary resolution because $\Var{P}(\Var{u})$ admits many
unifiers, $(subst)$ works well with elim-resolution.  It deletes some
assumption of the form $x=y$ and replaces every~$y$ by~$x$ in the subgoal
formula.  The simultaneous unification instantiates $\Var{u}$ to~$y$; if
$y$ is not an unknown, then $\Var{P}(y)$ can easily be unified with another
formula.  

In logical parlance, the premise containing the connective to be eliminated
is called the \bfindex{major premise}.  Elim-resolution expects the major
premise to come first.  The order of the premises is significant in
Isabelle.

\subsection{Destruction rules} \label{destruct}
\index{destruction rules|bold}\index{proof!forward}
Looking back to Fig.\ts\ref{fol-fig}, notice that there are two kinds of
elimination rule.  The rules $({\conj}E1)$, $({\conj}E2)$, $({\imp}E)$ and
$({\forall}E)$ extract the conclusion from the major premise.  In Isabelle
parlance, such rules are called \bfindex{destruction rules}; they are readable
and easy to use in forward proof.  The rules $({\disj}E)$, $({\bot}E)$ and
$({\exists}E)$ work by discharging assumptions; they support backward proof
in a style reminiscent of the sequent calculus.

The latter style is the most general form of elimination rule.  In natural
deduction, there is no way to recast $({\disj}E)$, $({\bot}E)$ or
$({\exists}E)$ as destruction rules.  But we can write general elimination
rules for $\conj$, $\imp$ and~$\forall$:
\[
\infer{R}{P\conj Q & \infer*{R}{[P,Q]}} \qquad
\infer{R}{P\imp Q & P & \infer*{R}{[Q]}}  \qquad
\infer{Q}{\forall x.P & \infer*{Q}{[P[t/x]]}} 
\]
Because they are concise, destruction rules are simpler to derive than the
corresponding elimination rules.  To facilitate their use in backward
proof, Isabelle provides a means of transforming a destruction rule such as
\[ \infer[\quad\hbox{to the elimination rule}\quad]{Q}{P@1 & \ldots & P@m} 
   \infer{R.}{P@1 & \ldots & P@m & \infer*{R}{[Q]}} 
\]
\index{destruct-resolution|bold}
{\bf Destruct-resolution} combines this transformation with
elim-resolution.  It applies a destruction rule to some assumption of a
subgoal.  Given the rule above, it replaces the assumption~$P@1$ by~$Q$,
with new subgoals of showing instances of $P@2$, \ldots,~$P@m$.
Destruct-resolution works forward from a subgoal's assumptions.  Ordinary
resolution performs forward reasoning from theorems, as illustrated
in~\S\ref{joining}. 


\subsection{Deriving rules by resolution}  \label{deriving}
\index{rules!derived|bold}
The meta-logic, itself a form of the predicate calculus, is defined by a
system of natural deduction rules.  Each theorem may depend upon
meta-assumptions.  The theorem that~$\phi$ follows from the assumptions
$\phi@1$, \ldots, $\phi@n$ is written
\[ \phi \quad [\phi@1,\ldots,\phi@n]. \]
A more conventional notation might be $\phi@1,\ldots,\phi@n \turn \phi$,
but Isabelle's notation is more readable with large formulae.

Meta-level natural deduction provides a convenient mechanism for deriving
new object-level rules.  To derive the rule
\[ \infer{\phi,}{\theta@1 & \ldots & \theta@k} \]
assume the premises $\theta@1$,~\ldots,~$\theta@k$ at the
meta-level.  Then prove $\phi$, possibly using these assumptions.
Starting with a proof state $\phi\Imp\phi$, assumptions may accumulate,
reaching a final state such as
\[ \phi \quad [\theta@1,\ldots,\theta@k]. \]
The meta-rule for $\Imp$ introduction discharges an assumption.
Discharging them in the order $\theta@k,\ldots,\theta@1$ yields the
meta-theorem $\List{\theta@1; \ldots; \theta@k} \Imp \phi$, with no
assumptions.  This represents the desired rule.
Let us derive the general $\conj$ elimination rule:
$$ \infer{R}{P\conj Q & \infer*{R}{[P,Q]}}  \eqno(\conj E)$$
We assume $P\conj Q$ and $\List{P;Q}\Imp R$, and commence backward proof in
the state $R\Imp R$.  Resolving this with the second assumption yields the
state 
\[ \phantom{\List{P\conj Q;\; P\conj Q}}
   \llap{$\List{P;Q}$}\Imp R \quad [\,\List{P;Q}\Imp R\,]. \]
Resolving subgoals~1 and~2 with $({\conj}E1)$ and $({\conj}E2)$,
respectively, yields the state
\[ \List{P\conj Q;\; P\conj Q}\Imp R \quad [\,\List{P;Q}\Imp R\,]. \]
Resolving both subgoals with the assumption $P\conj Q$ yields
\[ R \quad [\, \List{P;Q}\Imp R, P\conj Q \,]. \]
The proof may use the meta-assumptions in any order, and as often as
necessary; when finished, we discharge them in the correct order to
obtain the desired form:
\[ \List{P\conj Q;\; \List{P;Q}\Imp R} \Imp R \]
We have derived the rule using free variables, which prevents their
premature instantiation during the proof; we may now replace them by
schematic variables.

\begin{warn}
Schematic variables are not allowed in meta-assumptions because they would
complicate lifting.  Meta-assumptions remain fixed throughout a proof.
\end{warn}


%% $Id$
\part{Getting Started with Isabelle}\label{chap:getting}
Let us consider how to perform simple proofs using Isabelle.  At
present, Isabelle's user interface is \ML.  Proofs are conducted by
applying certain \ML{} functions, which update a stored proof state.
Basically, all syntax must be expressed using plain {\sc ascii}
characters. There are also mechanisms built into Isabelle that support
alternative syntaxes, for example using mathematical symbols from a
special screen font. The meta-logic and major object-logics already
provide such fancy output as an option.

Object-logics are built upon Pure Isabelle, which implements the
meta-logic and provides certain fundamental data structures: types,
terms, signatures, theorems and theories, tactics and tacticals.
These data structures have the corresponding \ML{} types {\tt typ},
{\tt term}, {\tt Sign.sg}, {\tt thm}, {\tt theory} and {\tt tactic};
tacticals have function types such as {\tt tactic->tactic}.  Isabelle
users can operate on these data structures by writing \ML{} programs.

\section{Forward proof}\label{sec:forward} \index{forward proof|(}
This section describes the concrete syntax for types, terms and theorems,
and demonstrates forward proof.

\subsection{Lexical matters}
\index{identifiers}\index{reserved words} 
An {\bf identifier} is a string of letters, digits, underscores~(\verb|_|)
and single quotes~({\tt'}), beginning with a letter.  Single quotes are
regarded as primes; for instance {\tt x'} is read as~$x'$.  Identifiers are
separated by white space and special characters.  {\bf Reserved words} are
identifiers that appear in Isabelle syntax definitions.

An Isabelle theory can declare symbols composed of special characters, such
as {\tt=}, {\tt==}, {\tt=>} and {\tt==>}.  (The latter three are part of
the syntax of the meta-logic.)  Such symbols may be run together; thus if
\verb|}| and \verb|{| are used for set brackets then \verb|{{a},{a,b}}| is
valid notation for a set of sets --- but only if \verb|}}| and \verb|{{|
have not been declared as symbols!  The parser resolves any ambiguity by
taking the longest possible symbol that has been declared.  Thus the string
{\tt==>} is read as a single symbol.  But \hbox{\tt= =>} is read as two
symbols.

Identifiers that are not reserved words may serve as free variables or
constants.  A {\bf type identifier} consists of an identifier prefixed by a
prime, for example {\tt'a} and \hbox{\tt'hello}.  Type identifiers stand
for (free) type variables, which remain fixed during a proof.
\index{type identifiers}

An {\bf unknown}\index{unknowns} (or type unknown) consists of a question
mark, an identifier (or type identifier), and a subscript.  The subscript,
a non-negative integer,
allows the renaming of unknowns prior to unification.%
\footnote{The subscript may appear after the identifier, separated by a
  dot; this prevents ambiguity when the identifier ends with a digit.  Thus
  {\tt?z6.0} has identifier {\tt"z6"} and subscript~0, while {\tt?a0.5}
  has identifier {\tt"a0"} and subscript~5.  If the identifier does not
  end with a digit, then no dot appears and a subscript of~0 is omitted;
  for example, {\tt?hello} has identifier {\tt"hello"} and subscript
  zero, while {\tt?z6} has identifier {\tt"z"} and subscript~6.  The same
  conventions apply to type unknowns.  The question mark is {\it not\/}
  part of the identifier!}


\subsection{Syntax of types and terms}
\index{classes!built-in|bold}\index{syntax!of types and terms}

Classes are denoted by identifiers; the built-in class \cldx{logic}
contains the `logical' types.  Sorts are lists of classes enclosed in
braces~\} and \{; singleton sorts may be abbreviated by dropping the braces.

\index{types!syntax of|bold}\index{sort constraints} Types are written
with a syntax like \ML's.  The built-in type \tydx{prop} is the type
of propositions.  Type variables can be constrained to particular
classes or sorts, for example {\tt 'a::term} and {\tt ?'b::\ttlbrace
  ord, arith\ttrbrace}.
\[\dquotes
\index{*:: symbol}\index{*=> symbol}
\index{{}@{\tt\ttlbrace} symbol}\index{{}@{\tt\ttrbrace} symbol}
\index{*[ symbol}\index{*] symbol}
\begin{array}{lll}
    \multicolumn{3}{c}{\hbox{ASCII Notation for Types}} \\ \hline
  \alpha "::" C              & \alpha :: C        & \hbox{class constraint} \\
  \alpha "::" "\ttlbrace" C@1 "," \ldots "," C@n "\ttrbrace" &
     \alpha :: \{C@1,\dots,C@n\}             & \hbox{sort constraint} \\
  \sigma " => " \tau        & \sigma\To\tau & \hbox{function type} \\
  "[" \sigma@1 "," \ldots "," \sigma@n "] => " \tau &
     [\sigma@1,\ldots,\sigma@n] \To\tau & \hbox{curried function type} \\
  "(" \tau@1"," \ldots "," \tau@n ")" tycon & 
     (\tau@1, \ldots, \tau@n)tycon      & \hbox{type construction}
\end{array} 
\]
Terms are those of the typed $\lambda$-calculus.
\index{terms!syntax of|bold}\index{type constraints}
\[\dquotes
\index symbol}\index{lambda abs@$\lambda$-abstractions}
\index{*:: symbol}
\begin{array}{lll}
    \multicolumn{3}{c}{\hbox{ASCII Notation for Terms}} \\ \hline
  t "::" \sigma         & t :: \sigma   & \hbox{type constraint} \\
  "\%" x "." t          & \lambda x.t   & \hbox{abstraction} \\
  "\%" x@1\ldots x@n "." t  & \lambda x@1\ldots x@n.t & 
     \hbox{curried abstraction} \\
  t "(" u@1"," \ldots "," u@n ")" & 
  t (u@1, \ldots, u@n) & \hbox{curried application}
\end{array}  
\]
The theorems and rules of an object-logic are represented by theorems in
the meta-logic, which are expressed using meta-formulae.  Since the
meta-logic is higher-order, meta-formulae~$\phi$, $\psi$, $\theta$,~\ldots{}
are just terms of type~{\tt prop}.  
\index{meta-implication}
\index{meta-quantifiers}\index{meta-equality}
\index{*"!"! symbol}\index{*"["| symbol}\index{*"|"] symbol}
\index{*== symbol}\index{*=?= symbol}\index{*==> symbol}
\[\dquotes
  \begin{array}{l@{\quad}l@{\quad}l}
    \multicolumn{3}{c}{\hbox{ASCII Notation for Meta-Formulae}} \\ \hline
  a " == " b    & a\equiv b &   \hbox{meta-equality} \\
  a " =?= " b   & a\qeq b &     \hbox{flex-flex constraint} \\
  \phi " ==> " \psi & \phi\Imp \psi & \hbox{meta-implication} \\
  "[|" \phi@1 ";" \ldots ";" \phi@n "|] ==> " \psi & 
  \List{\phi@1;\ldots;\phi@n} \Imp \psi & \hbox{nested implication} \\
  "!!" x "." \phi & \Forall x.\phi & \hbox{meta-quantification} \\
  "!!" x@1\ldots x@n "." \phi & 
  \Forall x@1. \ldots x@n.\phi & \hbox{nested quantification}
  \end{array}
\]
Flex-flex constraints are meta-equalities arising from unification; they
require special treatment.  See~\S\ref{flexflex}.
\index{flex-flex constraints}

\index{*Trueprop constant}
Most logics define the implicit coercion $Trueprop$ from object-formulae to
propositions.  This could cause an ambiguity: in $P\Imp Q$, do the
variables $P$ and $Q$ stand for meta-formulae or object-formulae?  If the
latter, $P\Imp Q$ really abbreviates $Trueprop(P)\Imp Trueprop(Q)$.  To
prevent such ambiguities, Isabelle's syntax does not allow a meta-formula
to consist of a variable.  Variables of type~\tydx{prop} are seldom
useful, but you can make a variable stand for a meta-formula by prefixing
it with the symbol {\tt PROP}:\index{*PROP symbol}
\begin{ttbox} 
PROP ?psi ==> PROP ?theta 
\end{ttbox}

Symbols of object-logics are typically rendered into {\sc ascii} as
follows:
\[ \begin{tabular}{l@{\quad}l@{\quad}l}
      \tt True          & $\top$        & true \\
      \tt False         & $\bot$        & false \\
      \tt $P$ \& $Q$    & $P\conj Q$    & conjunction \\
      \tt $P$ | $Q$     & $P\disj Q$    & disjunction \\
      \verb'~' $P$      & $\neg P$      & negation \\
      \tt $P$ --> $Q$   & $P\imp Q$     & implication \\
      \tt $P$ <-> $Q$   & $P\bimp Q$    & bi-implication \\
      \tt ALL $x\,y\,z$ .\ $P$  & $\forall x\,y\,z.P$   & for all \\
      \tt EX  $x\,y\,z$ .\ $P$  & $\exists x\,y\,z.P$   & there exists
   \end{tabular}
\]
To illustrate the notation, consider two axioms for first-order logic:
$$ \List{P; Q} \Imp P\conj Q                 \eqno(\conj I) $$
$$ \List{\exists x.P(x); \Forall x. P(x)\imp Q} \Imp Q \eqno(\exists E) $$
$({\conj}I)$ translates into {\sc ascii} characters as
\begin{ttbox}
[| ?P; ?Q |] ==> ?P & ?Q
\end{ttbox}
The schematic variables let unification instantiate the rule.  To avoid
cluttering logic definitions with question marks, Isabelle converts any
free variables in a rule to schematic variables; we normally declare
$({\conj}I)$ as
\begin{ttbox}
[| P; Q |] ==> P & Q
\end{ttbox}
This variables convention agrees with the treatment of variables in goals.
Free variables in a goal remain fixed throughout the proof.  After the
proof is finished, Isabelle converts them to scheme variables in the
resulting theorem.  Scheme variables in a goal may be replaced by terms
during the proof, supporting answer extraction, program synthesis, and so
forth.

For a final example, the rule $(\exists E)$ is rendered in {\sc ascii} as
\begin{ttbox}
[| EX x.P(x);  !!x. P(x) ==> Q |] ==> Q
\end{ttbox}


\subsection{Basic operations on theorems}
\index{theorems!basic operations on|bold}
\index{LCF system}
Meta-level theorems have the \ML{} type \mltydx{thm}.  They represent the
theorems and inference rules of object-logics.  Isabelle's meta-logic is
implemented using the {\sc lcf} approach: each meta-level inference rule is
represented by a function from theorems to theorems.  Object-level rules
are taken as axioms.

The main theorem printing commands are {\tt prth}, {\tt prths} and~{\tt
  prthq}.  Of the other operations on theorems, most useful are {\tt RS}
and {\tt RSN}, which perform resolution.

\index{theorems!printing of}
\begin{ttdescription}
\item[\ttindex{prth} {\it thm};]
  pretty-prints {\it thm\/} at the terminal.

\item[\ttindex{prths} {\it thms};]
  pretty-prints {\it thms}, a list of theorems.

\item[\ttindex{prthq} {\it thmq};]
  pretty-prints {\it thmq}, a sequence of theorems; this is useful for
  inspecting the output of a tactic.

\item[$thm1$ RS $thm2$] \index{*RS} 
  resolves the conclusion of $thm1$ with the first premise of~$thm2$.

\item[$thm1$ RSN $(i,thm2)$] \index{*RSN} 
  resolves the conclusion of $thm1$ with the $i$th premise of~$thm2$.

\item[\ttindex{standard} $thm$]  
  puts $thm$ into a standard format.  It also renames schematic variables
  to have subscript zero, improving readability and reducing subscript
  growth.
\end{ttdescription}
The rules of a theory are normally bound to \ML\ identifiers.  Suppose we
are running an Isabelle session containing theory~\FOL, natural deduction
first-order logic.\footnote{For a listing of the \FOL{} rules and their
  \ML{} names, turn to
\iflabelundefined{fol-rules}{{\em Isabelle's Object-Logics}}%
           {page~\pageref{fol-rules}}.}
Let us try an example given in~\S\ref{joining}.  We
first print \tdx{mp}, which is the rule~$({\imp}E)$, then resolve it with
itself.
\begin{ttbox} 
prth mp; 
{\out [| ?P --> ?Q; ?P |] ==> ?Q}
{\out val it = "[| ?P --> ?Q; ?P |] ==> ?Q" : thm}
prth (mp RS mp);
{\out [| ?P1 --> ?P --> ?Q; ?P1; ?P |] ==> ?Q}
{\out val it = "[| ?P1 --> ?P --> ?Q; ?P1; ?P |] ==> ?Q" : thm}
\end{ttbox}
User input appears in {\footnotesize\tt typewriter characters}, and output
appears in {\sltt slanted typewriter characters}.  \ML's response {\out val
  }~\ldots{} is compiler-dependent and will sometimes be suppressed.  This
session illustrates two formats for the display of theorems.  Isabelle's
top-level displays theorems as \ML{} values, enclosed in quotes.  Printing
commands like {\tt prth} omit the quotes and the surrounding {\tt val
  \ldots :\ thm}.  Ignoring their side-effects, the commands are identity
functions.

To contrast {\tt RS} with {\tt RSN}, we resolve
\tdx{conjunct1}, which stands for~$(\conj E1)$, with~\tdx{mp}.
\begin{ttbox} 
conjunct1 RS mp;
{\out val it = "[| (?P --> ?Q) & ?Q1; ?P |] ==> ?Q" : thm}
conjunct1 RSN (2,mp);
{\out val it = "[| ?P --> ?Q; ?P & ?Q1 |] ==> ?Q" : thm}
\end{ttbox}
These correspond to the following proofs:
\[ \infer[({\imp}E)]{Q}{\infer[({\conj}E1)]{P\imp Q}{(P\imp Q)\conj Q@1} & P}
   \qquad
   \infer[({\imp}E)]{Q}{P\imp Q & \infer[({\conj}E1)]{P}{P\conj Q@1}} 
\]
%
Rules can be derived by pasting other rules together.  Let us join
\tdx{spec}, which stands for~$(\forall E)$, with {\tt mp} and {\tt
  conjunct1}.  In \ML{}, the identifier~{\tt it} denotes the value just
printed.
\begin{ttbox} 
spec;
{\out val it = "ALL x. ?P(x) ==> ?P(?x)" : thm}
it RS mp;
{\out val it = "[| ALL x. ?P3(x) --> ?Q2(x); ?P3(?x1) |] ==>}
{\out           ?Q2(?x1)" : thm}
it RS conjunct1;
{\out val it = "[| ALL x. ?P4(x) --> ?P6(x) & ?Q5(x); ?P4(?x2) |] ==>}
{\out           ?P6(?x2)" : thm}
standard it;
{\out val it = "[| ALL x. ?P(x) --> ?Pa(x) & ?Q(x); ?P(?x) |] ==>}
{\out           ?Pa(?x)" : thm}
\end{ttbox}
By resolving $(\forall E)$ with (${\imp}E)$ and (${\conj}E1)$, we have
derived a destruction rule for formulae of the form $\forall x.
P(x)\imp(Q(x)\conj R(x))$.  Used with destruct-resolution, such specialized
rules provide a way of referring to particular assumptions.
\index{assumptions!use of}

\subsection{*Flex-flex constraints} \label{flexflex}
\index{flex-flex constraints|bold}\index{unknowns!function}
In higher-order unification, {\bf flex-flex} equations are those where both
sides begin with a function unknown, such as $\Var{f}(0)\qeq\Var{g}(0)$.
They admit a trivial unifier, here $\Var{f}\equiv \lambda x.\Var{a}$ and
$\Var{g}\equiv \lambda y.\Var{a}$, where $\Var{a}$ is a new unknown.  They
admit many other unifiers, such as $\Var{f} \equiv \lambda x.\Var{g}(0)$
and $\{\Var{f} \equiv \lambda x.x,\, \Var{g} \equiv \lambda x.0\}$.  Huet's
procedure does not enumerate the unifiers; instead, it retains flex-flex
equations as constraints on future unifications.  Flex-flex constraints
occasionally become attached to a proof state; more frequently, they appear
during use of {\tt RS} and~{\tt RSN}:
\begin{ttbox} 
refl;
{\out val it = "?a = ?a" : thm}
exI;
{\out val it = "?P(?x) ==> EX x. ?P(x)" : thm}
refl RS exI;
{\out val it = "?a3(?x) =?= ?a2(?x) ==> EX x. ?a3(x) = ?a2(x)" : thm}
\end{ttbox}

\noindent
Renaming variables, this is $\exists x.\Var{f}(x)=\Var{g}(x)$ with
the constraint ${\Var{f}(\Var{u})\qeq\Var{g}(\Var{u})}$.  Instances
satisfying the constraint include $\exists x.\Var{f}(x)=\Var{f}(x)$ and
$\exists x.x=\Var{u}$.  Calling \ttindex{flexflex_rule} removes all
constraints by applying the trivial unifier:\index{*prthq}
\begin{ttbox} 
prthq (flexflex_rule it);
{\out EX x. ?a4 = ?a4}
\end{ttbox} 
Isabelle simplifies flex-flex equations to eliminate redundant bound
variables.  In $\lambda x\,y.\Var{f}(k(y),x) \qeq \lambda x\,y.\Var{g}(y)$,
there is no bound occurrence of~$x$ on the right side; thus, there will be
none on the left in a common instance of these terms.  Choosing a new
variable~$\Var{h}$, Isabelle assigns $\Var{f}\equiv \lambda u\,v.?h(u)$,
simplifying the left side to $\lambda x\,y.\Var{h}(k(y))$.  Dropping $x$
from the equation leaves $\lambda y.\Var{h}(k(y)) \qeq \lambda
y.\Var{g}(y)$.  By $\eta$-conversion, this simplifies to the assignment
$\Var{g}\equiv\lambda y.?h(k(y))$.

\begin{warn}
\ttindex{RS} and \ttindex{RSN} fail (by raising exception {\tt THM}) unless
the resolution delivers {\bf exactly one} resolvent.  For multiple results,
use \ttindex{RL} and \ttindex{RLN}, which operate on theorem lists.  The
following example uses \ttindex{read_instantiate} to create an instance
of \tdx{refl} containing no schematic variables.
\begin{ttbox} 
val reflk = read_instantiate [("a","k")] refl;
{\out val reflk = "k = k" : thm}
\end{ttbox}

\noindent
A flex-flex constraint is no longer possible; resolution does not find a
unique unifier:
\begin{ttbox} 
reflk RS exI;
{\out uncaught exception THM}
\end{ttbox}
Using \ttindex{RL} this time, we discover that there are four unifiers, and
four resolvents:
\begin{ttbox} 
[reflk] RL [exI];
{\out val it = ["EX x. x = x", "EX x. k = x",}
{\out           "EX x. x = k", "EX x. k = k"] : thm list}
\end{ttbox} 
\end{warn}

\index{forward proof|)}

\section{Backward proof}
Although {\tt RS} and {\tt RSN} are fine for simple forward reasoning,
large proofs require tactics.  Isabelle provides a suite of commands for
conducting a backward proof using tactics.

\subsection{The basic tactics}
The tactics {\tt assume_tac}, {\tt
resolve_tac}, {\tt eresolve_tac}, and {\tt dresolve_tac} suffice for most
single-step proofs.  Although {\tt eresolve_tac} and {\tt dresolve_tac} are
not strictly necessary, they simplify proofs involving elimination and
destruction rules.  All the tactics act on a subgoal designated by a
positive integer~$i$, failing if~$i$ is out of range.  The resolution
tactics try their list of theorems in left-to-right order.

\begin{ttdescription}
\item[\ttindex{assume_tac} {\it i}] \index{tactics!assumption}
  is the tactic that attempts to solve subgoal~$i$ by assumption.  Proof by
  assumption is not a trivial step; it can falsify other subgoals by
  instantiating shared variables.  There may be several ways of solving the
  subgoal by assumption.

\item[\ttindex{resolve_tac} {\it thms} {\it i}]\index{tactics!resolution}
  is the basic resolution tactic, used for most proof steps.  The $thms$
  represent object-rules, which are resolved against subgoal~$i$ of the
  proof state.  For each rule, resolution forms next states by unifying the
  conclusion with the subgoal and inserting instantiated premises in its
  place.  A rule can admit many higher-order unifiers.  The tactic fails if
  none of the rules generates next states.

\item[\ttindex{eresolve_tac} {\it thms} {\it i}] \index{elim-resolution}
  performs elim-resolution.  Like {\tt resolve_tac~{\it thms}~{\it i\/}}
  followed by {\tt assume_tac~{\it i}}, it applies a rule then solves its
  first premise by assumption.  But {\tt eresolve_tac} additionally deletes
  that assumption from any subgoals arising from the resolution.

\item[\ttindex{dresolve_tac} {\it thms} {\it i}]
  \index{forward proof}\index{destruct-resolution}
  performs destruct-resolution with the~$thms$, as described
  in~\S\ref{destruct}.  It is useful for forward reasoning from the
  assumptions.
\end{ttdescription}

\subsection{Commands for backward proof}
\index{proofs!commands for}
Tactics are normally applied using the subgoal module, which maintains a
proof state and manages the proof construction.  It allows interactive
backtracking through the proof space, going away to prove lemmas, etc.; of
its many commands, most important are the following:
\begin{ttdescription}
\item[\ttindex{goal} {\it theory} {\it formula}; ] 
begins a new proof, where $theory$ is usually an \ML\ identifier
and the {\it formula\/} is written as an \ML\ string.

\item[\ttindex{by} {\it tactic}; ] 
applies the {\it tactic\/} to the current proof
state, raising an exception if the tactic fails.

\item[\ttindex{undo}(); ]
  reverts to the previous proof state.  Undo can be repeated but cannot be
  undone.  Do not omit the parentheses; typing {\tt\ \ undo;\ \ } merely
  causes \ML\ to echo the value of that function.

\item[\ttindex{result}();]
returns the theorem just proved, in a standard format.  It fails if
unproved subgoals are left, etc.

\item[\ttindex{qed} {\it name};] is the usual way of ending a proof.
  It gets the theorem using {\tt result}, stores it in Isabelle's
  theorem database and binds it to an \ML{} identifier.

\end{ttdescription}
The commands and tactics given above are cumbersome for interactive use.
Although our examples will use the full commands, you may prefer Isabelle's
shortcuts:
\begin{center} \tt
\index{*br} \index{*be} \index{*bd} \index{*ba}
\begin{tabular}{l@{\qquad\rm abbreviates\qquad}l}
    ba {\it i};           & by (assume_tac {\it i}); \\

    br {\it thm} {\it i}; & by (resolve_tac [{\it thm}] {\it i}); \\

    be {\it thm} {\it i}; & by (eresolve_tac [{\it thm}] {\it i}); \\

    bd {\it thm} {\it i}; & by (dresolve_tac [{\it thm}] {\it i}); 
\end{tabular}
\end{center}

\subsection{A trivial example in propositional logic}
\index{examples!propositional}

Directory {\tt FOL} of the Isabelle distribution defines the theory of
first-order logic.  Let us try the example from \S\ref{prop-proof},
entering the goal $P\disj P\imp P$ in that theory.\footnote{To run these
  examples, see the file {\tt FOL/ex/intro.ML}.  The files {\tt README} and
  {\tt Makefile} on the directories {\tt Pure} and {\tt FOL} explain how to
  build first-order logic.}
\begin{ttbox}
goal FOL.thy "P|P --> P"; 
{\out Level 0} 
{\out P | P --> P} 
{\out 1. P | P --> P} 
\end{ttbox}\index{level of a proof}
Isabelle responds by printing the initial proof state, which has $P\disj
P\imp P$ as the main goal and the only subgoal.  The {\bf level} of the
state is the number of {\tt by} commands that have been applied to reach
it.  We now use \ttindex{resolve_tac} to apply the rule \tdx{impI},
or~$({\imp}I)$, to subgoal~1:
\begin{ttbox}
by (resolve_tac [impI] 1); 
{\out Level 1} 
{\out P | P --> P} 
{\out 1. P | P ==> P}
\end{ttbox}
In the new proof state, subgoal~1 is $P$ under the assumption $P\disj P$.
(The meta-implication {\tt==>} indicates assumptions.)  We apply
\tdx{disjE}, or~(${\disj}E)$, to that subgoal:
\begin{ttbox}
by (resolve_tac [disjE] 1); 
{\out Level 2} 
{\out P | P --> P} 
{\out 1. P | P ==> ?P1 | ?Q1} 
{\out 2. [| P | P; ?P1 |] ==> P} 
{\out 3. [| P | P; ?Q1 |] ==> P}
\end{ttbox}
At Level~2 there are three subgoals, each provable by assumption.  We
deviate from~\S\ref{prop-proof} by tackling subgoal~3 first, using
\ttindex{assume_tac}.  This affects subgoal~1, updating {\tt?Q1} to~{\tt
  P}.
\begin{ttbox}
by (assume_tac 3); 
{\out Level 3} 
{\out P | P --> P} 
{\out 1. P | P ==> ?P1 | P} 
{\out 2. [| P | P; ?P1 |] ==> P}
\end{ttbox}
Next we tackle subgoal~2, instantiating {\tt?P1} to~{\tt P} in subgoal~1.
\begin{ttbox}
by (assume_tac 2); 
{\out Level 4} 
{\out P | P --> P} 
{\out 1. P | P ==> P | P}
\end{ttbox}
Lastly we prove the remaining subgoal by assumption:
\begin{ttbox}
by (assume_tac 1); 
{\out Level 5} 
{\out P | P --> P} 
{\out No subgoals!}
\end{ttbox}
Isabelle tells us that there are no longer any subgoals: the proof is
complete.  Calling {\tt qed} stores the theorem.
\begin{ttbox}
qed "mythm";
{\out val mythm = "?P | ?P --> ?P" : thm} 
\end{ttbox}
Isabelle has replaced the free variable~{\tt P} by the scheme
variable~{\tt?P}\@.  Free variables in the proof state remain fixed
throughout the proof.  Isabelle finally converts them to scheme variables
so that the resulting theorem can be instantiated with any formula.

As an exercise, try doing the proof as in \S\ref{prop-proof}, observing how
instantiations affect the proof state.


\subsection{Part of a distributive law}
\index{examples!propositional}
To demonstrate the tactics \ttindex{eresolve_tac}, \ttindex{dresolve_tac}
and the tactical {\tt REPEAT}, let us prove part of the distributive
law 
\[ (P\conj Q)\disj R \,\bimp\, (P\disj R)\conj (Q\disj R). \]
We begin by stating the goal to Isabelle and applying~$({\imp}I)$ to it:
\begin{ttbox}
goal FOL.thy "(P & Q) | R  --> (P | R)";
{\out Level 0}
{\out P & Q | R --> P | R}
{\out  1. P & Q | R --> P | R}
\ttbreak
by (resolve_tac [impI] 1);
{\out Level 1}
{\out P & Q | R --> P | R}
{\out  1. P & Q | R ==> P | R}
\end{ttbox}
Previously we applied~(${\disj}E)$ using {\tt resolve_tac}, but 
\ttindex{eresolve_tac} deletes the assumption after use.  The resulting proof
state is simpler.
\begin{ttbox}
by (eresolve_tac [disjE] 1);
{\out Level 2}
{\out P & Q | R --> P | R}
{\out  1. P & Q ==> P | R}
{\out  2. R ==> P | R}
\end{ttbox}
Using \ttindex{dresolve_tac}, we can apply~(${\conj}E1)$ to subgoal~1,
replacing the assumption $P\conj Q$ by~$P$.  Normally we should apply the
rule~(${\conj}E)$, given in~\S\ref{destruct}.  That is an elimination rule
and requires {\tt eresolve_tac}; it would replace $P\conj Q$ by the two
assumptions~$P$ and~$Q$.  Because the present example does not need~$Q$, we
may try out {\tt dresolve_tac}.
\begin{ttbox}
by (dresolve_tac [conjunct1] 1);
{\out Level 3}
{\out P & Q | R --> P | R}
{\out  1. P ==> P | R}
{\out  2. R ==> P | R}
\end{ttbox}
The next two steps apply~(${\disj}I1$) and~(${\disj}I2$) in an obvious manner.
\begin{ttbox}
by (resolve_tac [disjI1] 1);
{\out Level 4}
{\out P & Q | R --> P | R}
{\out  1. P ==> P}
{\out  2. R ==> P | R}
\ttbreak
by (resolve_tac [disjI2] 2);
{\out Level 5}
{\out P & Q | R --> P | R}
{\out  1. P ==> P}
{\out  2. R ==> R}
\end{ttbox}
Two calls of {\tt assume_tac} can finish the proof.  The
tactical~\ttindex{REPEAT} here expresses a tactic that calls {\tt assume_tac~1}
as many times as possible.  We can restrict attention to subgoal~1 because
the other subgoals move up after subgoal~1 disappears.
\begin{ttbox}
by (REPEAT (assume_tac 1));
{\out Level 6}
{\out P & Q | R --> P | R}
{\out No subgoals!}
\end{ttbox}


\section{Quantifier reasoning}
\index{quantifiers}\index{parameters}\index{unknowns}\index{unknowns!function}
This section illustrates how Isabelle enforces quantifier provisos.
Suppose that $x$, $y$ and~$z$ are parameters of a subgoal.  Quantifier
rules create terms such as~$\Var{f}(x,z)$, where~$\Var{f}$ is a function
unknown.  Instantiating $\Var{f}$ to $\lambda x\,z.t$ has the effect of
replacing~$\Var{f}(x,z)$ by~$t$, where the term~$t$ may contain free
occurrences of~$x$ and~$z$.  On the other hand, no instantiation
of~$\Var{f}$ can replace~$\Var{f}(x,z)$ by a term containing free
occurrences of~$y$, since parameters are bound variables.

\subsection{Two quantifier proofs: a success and a failure}
\index{examples!with quantifiers}
Let us contrast a proof of the theorem $\forall x.\exists y.x=y$ with an
attempted proof of the non-theorem $\exists y.\forall x.x=y$.  The former
proof succeeds, and the latter fails, because of the scope of quantified
variables~\cite{paulson-found}.  Unification helps even in these trivial
proofs. In $\forall x.\exists y.x=y$ the $y$ that `exists' is simply $x$,
but we need never say so. This choice is forced by the reflexive law for
equality, and happens automatically.

\paragraph{The successful proof.}
The proof of $\forall x.\exists y.x=y$ demonstrates the introduction rules
$(\forall I)$ and~$(\exists I)$.  We state the goal and apply $(\forall I)$:
\begin{ttbox}
goal FOL.thy "ALL x. EX y. x=y";
{\out Level 0}
{\out ALL x. EX y. x = y}
{\out  1. ALL x. EX y. x = y}
\ttbreak
by (resolve_tac [allI] 1);
{\out Level 1}
{\out ALL x. EX y. x = y}
{\out  1. !!x. EX y. x = y}
\end{ttbox}
The variable~{\tt x} is no longer universally quantified, but is a
parameter in the subgoal; thus, it is universally quantified at the
meta-level.  The subgoal must be proved for all possible values of~{\tt x}.

To remove the existential quantifier, we apply the rule $(\exists I)$:
\begin{ttbox}
by (resolve_tac [exI] 1);
{\out Level 2}
{\out ALL x. EX y. x = y}
{\out  1. !!x. x = ?y1(x)}
\end{ttbox}
The bound variable {\tt y} has become {\tt?y1(x)}.  This term consists of
the function unknown~{\tt?y1} applied to the parameter~{\tt x}.
Instances of {\tt?y1(x)} may or may not contain~{\tt x}.  We resolve the
subgoal with the reflexivity axiom.
\begin{ttbox}
by (resolve_tac [refl] 1);
{\out Level 3}
{\out ALL x. EX y. x = y}
{\out No subgoals!}
\end{ttbox}
Let us consider what has happened in detail.  The reflexivity axiom is
lifted over~$x$ to become $\Forall x.\Var{f}(x)=\Var{f}(x)$, which is
unified with $\Forall x.x=\Var{y@1}(x)$.  The function unknowns $\Var{f}$
and~$\Var{y@1}$ are both instantiated to the identity function, and
$x=\Var{y@1}(x)$ collapses to~$x=x$ by $\beta$-reduction.

\paragraph{The unsuccessful proof.}
We state the goal $\exists y.\forall x.x=y$, which is not a theorem, and
try~$(\exists I)$:
\begin{ttbox}
goal FOL.thy "EX y. ALL x. x=y";
{\out Level 0}
{\out EX y. ALL x. x = y}
{\out  1. EX y. ALL x. x = y}
\ttbreak
by (resolve_tac [exI] 1);
{\out Level 1}
{\out EX y. ALL x. x = y}
{\out  1. ALL x. x = ?y}
\end{ttbox}
The unknown {\tt ?y} may be replaced by any term, but this can never
introduce another bound occurrence of~{\tt x}.  We now apply~$(\forall I)$:
\begin{ttbox}
by (resolve_tac [allI] 1);
{\out Level 2}
{\out EX y. ALL x. x = y}
{\out  1. !!x. x = ?y}
\end{ttbox}
Compare our position with the previous Level~2.  Instead of {\tt?y1(x)} we
have~{\tt?y}, whose instances may not contain the bound variable~{\tt x}.
The reflexivity axiom does not unify with subgoal~1.
\begin{ttbox}
by (resolve_tac [refl] 1);
{\out by: tactic failed}
\end{ttbox}
There can be no proof of $\exists y.\forall x.x=y$ by the soundness of
first-order logic.  I have elsewhere proved the faithfulness of Isabelle's
encoding of first-order logic~\cite{paulson-found}; there could, of course, be
faults in the implementation.


\subsection{Nested quantifiers}
\index{examples!with quantifiers}
Multiple quantifiers create complex terms.  Proving 
\[ (\forall x\,y.P(x,y)) \imp (\forall z\,w.P(w,z)) \] 
will demonstrate how parameters and unknowns develop.  If they appear in
the wrong order, the proof will fail.

This section concludes with a demonstration of {\tt REPEAT}
and~{\tt ORELSE}.  
\begin{ttbox}
goal FOL.thy "(ALL x y.P(x,y))  -->  (ALL z w.P(w,z))";
{\out Level 0}
{\out (ALL x y. P(x,y)) --> (ALL z w. P(w,z))}
{\out  1. (ALL x y. P(x,y)) --> (ALL z w. P(w,z))}
\ttbreak
by (resolve_tac [impI] 1);
{\out Level 1}
{\out (ALL x y. P(x,y)) --> (ALL z w. P(w,z))}
{\out  1. ALL x y. P(x,y) ==> ALL z w. P(w,z)}
\end{ttbox}

\paragraph{The wrong approach.}
Using {\tt dresolve_tac}, we apply the rule $(\forall E)$, bound to the
\ML\ identifier \tdx{spec}.  Then we apply $(\forall I)$.
\begin{ttbox}
by (dresolve_tac [spec] 1);
{\out Level 2}
{\out (ALL x y. P(x,y)) --> (ALL z w. P(w,z))}
{\out  1. ALL y. P(?x1,y) ==> ALL z w. P(w,z)}
\ttbreak
by (resolve_tac [allI] 1);
{\out Level 3}
{\out (ALL x y. P(x,y)) --> (ALL z w. P(w,z))}
{\out  1. !!z. ALL y. P(?x1,y) ==> ALL w. P(w,z)}
\end{ttbox}
The unknown {\tt ?x1} and the parameter {\tt z} have appeared.  We again
apply $(\forall E)$ and~$(\forall I)$.
\begin{ttbox}
by (dresolve_tac [spec] 1);
{\out Level 4}
{\out (ALL x y. P(x,y)) --> (ALL z w. P(w,z))}
{\out  1. !!z. P(?x1,?y3(z)) ==> ALL w. P(w,z)}
\ttbreak
by (resolve_tac [allI] 1);
{\out Level 5}
{\out (ALL x y. P(x,y)) --> (ALL z w. P(w,z))}
{\out  1. !!z w. P(?x1,?y3(z)) ==> P(w,z)}
\end{ttbox}
The unknown {\tt ?y3} and the parameter {\tt w} have appeared.  Each
unknown is applied to the parameters existing at the time of its creation;
instances of~{\tt ?x1} cannot contain~{\tt z} or~{\tt w}, while instances
of {\tt?y3(z)} can only contain~{\tt z}.  Due to the restriction on~{\tt ?x1},
proof by assumption will fail.
\begin{ttbox}
by (assume_tac 1);
{\out by: tactic failed}
{\out uncaught exception ERROR}
\end{ttbox}

\paragraph{The right approach.}
To do this proof, the rules must be applied in the correct order.
Parameters should be created before unknowns.  The
\ttindex{choplev} command returns to an earlier stage of the proof;
let us return to the result of applying~$({\imp}I)$:
\begin{ttbox}
choplev 1;
{\out Level 1}
{\out (ALL x y. P(x,y)) --> (ALL z w. P(w,z))}
{\out  1. ALL x y. P(x,y) ==> ALL z w. P(w,z)}
\end{ttbox}
Previously we made the mistake of applying $(\forall E)$ before $(\forall I)$.
\begin{ttbox}
by (resolve_tac [allI] 1);
{\out Level 2}
{\out (ALL x y. P(x,y)) --> (ALL z w. P(w,z))}
{\out  1. !!z. ALL x y. P(x,y) ==> ALL w. P(w,z)}
\ttbreak
by (resolve_tac [allI] 1);
{\out Level 3}
{\out (ALL x y. P(x,y)) --> (ALL z w. P(w,z))}
{\out  1. !!z w. ALL x y. P(x,y) ==> P(w,z)}
\end{ttbox}
The parameters {\tt z} and~{\tt w} have appeared.  We now create the
unknowns:
\begin{ttbox}
by (dresolve_tac [spec] 1);
{\out Level 4}
{\out (ALL x y. P(x,y)) --> (ALL z w. P(w,z))}
{\out  1. !!z w. ALL y. P(?x3(z,w),y) ==> P(w,z)}
\ttbreak
by (dresolve_tac [spec] 1);
{\out Level 5}
{\out (ALL x y. P(x,y)) --> (ALL z w. P(w,z))}
{\out  1. !!z w. P(?x3(z,w),?y4(z,w)) ==> P(w,z)}
\end{ttbox}
Both {\tt?x3(z,w)} and~{\tt?y4(z,w)} could become any terms containing {\tt
z} and~{\tt w}:
\begin{ttbox}
by (assume_tac 1);
{\out Level 6}
{\out (ALL x y. P(x,y)) --> (ALL z w. P(w,z))}
{\out No subgoals!}
\end{ttbox}

\paragraph{A one-step proof using tacticals.}
\index{tacticals} \index{examples!of tacticals} 

Repeated application of rules can be effective, but the rules should be
attempted in the correct order.  Let us return to the original goal using
\ttindex{choplev}:
\begin{ttbox}
choplev 0;
{\out Level 0}
{\out (ALL x y. P(x,y)) --> (ALL z w. P(w,z))}
{\out  1. (ALL x y. P(x,y)) --> (ALL z w. P(w,z))}
\end{ttbox}
As we have just seen, \tdx{allI} should be attempted
before~\tdx{spec}, while \ttindex{assume_tac} generally can be
attempted first.  Such priorities can easily be expressed
using~\ttindex{ORELSE}, and repeated using~\ttindex{REPEAT}.
\begin{ttbox}
by (REPEAT (assume_tac 1 ORELSE resolve_tac [impI,allI] 1
     ORELSE dresolve_tac [spec] 1));
{\out Level 1}
{\out (ALL x y. P(x,y)) --> (ALL z w. P(w,z))}
{\out No subgoals!}
\end{ttbox}


\subsection{A realistic quantifier proof}
\index{examples!with quantifiers}
To see the practical use of parameters and unknowns, let us prove half of
the equivalence 
\[ (\forall x. P(x) \imp Q) \,\bimp\, ((\exists x. P(x)) \imp Q). \]
We state the left-to-right half to Isabelle in the normal way.
Since $\imp$ is nested to the right, $({\imp}I)$ can be applied twice; we
use {\tt REPEAT}:
\begin{ttbox}
goal FOL.thy "(ALL x.P(x) --> Q) --> (EX x.P(x)) --> Q";
{\out Level 0}
{\out (ALL x. P(x) --> Q) --> (EX x. P(x)) --> Q}
{\out  1. (ALL x. P(x) --> Q) --> (EX x. P(x)) --> Q}
\ttbreak
by (REPEAT (resolve_tac [impI] 1));
{\out Level 1}
{\out (ALL x. P(x) --> Q) --> (EX x. P(x)) --> Q}
{\out  1. [| ALL x. P(x) --> Q; EX x. P(x) |] ==> Q}
\end{ttbox}
We can eliminate the universal or the existential quantifier.  The
existential quantifier should be eliminated first, since this creates a
parameter.  The rule~$(\exists E)$ is bound to the
identifier~\tdx{exE}.
\begin{ttbox}
by (eresolve_tac [exE] 1);
{\out Level 2}
{\out (ALL x. P(x) --> Q) --> (EX x. P(x)) --> Q}
{\out  1. !!x. [| ALL x. P(x) --> Q; P(x) |] ==> Q}
\end{ttbox}
The only possibility now is $(\forall E)$, a destruction rule.  We use 
\ttindex{dresolve_tac}, which discards the quantified assumption; it is
only needed once.
\begin{ttbox}
by (dresolve_tac [spec] 1);
{\out Level 3}
{\out (ALL x. P(x) --> Q) --> (EX x. P(x)) --> Q}
{\out  1. !!x. [| P(x); P(?x3(x)) --> Q |] ==> Q}
\end{ttbox}
Because we applied $(\exists E)$ before $(\forall E)$, the unknown
term~{\tt?x3(x)} may depend upon the parameter~{\tt x}.

Although $({\imp}E)$ is a destruction rule, it works with 
\ttindex{eresolve_tac} to perform backward chaining.  This technique is
frequently useful.  
\begin{ttbox}
by (eresolve_tac [mp] 1);
{\out Level 4}
{\out (ALL x. P(x) --> Q) --> (EX x. P(x)) --> Q}
{\out  1. !!x. P(x) ==> P(?x3(x))}
\end{ttbox}
The tactic has reduced~{\tt Q} to~{\tt P(?x3(x))}, deleting the
implication.  The final step is trivial, thanks to the occurrence of~{\tt x}.
\begin{ttbox}
by (assume_tac 1);
{\out Level 5}
{\out (ALL x. P(x) --> Q) --> (EX x. P(x)) --> Q}
{\out No subgoals!}
\end{ttbox}


\subsection{The classical reasoner}
\index{classical reasoner}
Although Isabelle cannot compete with fully automatic theorem provers, it
provides enough automation to tackle substantial examples.  The classical
reasoner can be set up for any classical natural deduction logic;
see \iflabelundefined{chap:classical}{the {\em Reference Manual\/}}%
        {Chap.\ts\ref{chap:classical}}. 

Rules are packaged into {\bf classical sets}.  The classical reasoner
provides several tactics, which apply rules using naive algorithms.
Unification handles quantifiers as shown above.  The most useful tactic
is~\ttindex{Blast_tac}.  

Let us solve problems~40 and~60 of Pelletier~\cite{pelletier86}.  (The
backslashes~\hbox{\verb|\|\ldots\verb|\|} are an \ML{} string escape
sequence, to break the long string over two lines.)
\begin{ttbox}
goal FOL.thy "(EX y. ALL x. J(y,x) <-> ~J(x,x))  \ttback
\ttback       -->  ~ (ALL x. EX y. ALL z. J(z,y) <-> ~ J(z,x))";
{\out Level 0}
{\out (EX y. ALL x. J(y,x) <-> ~J(x,x)) -->}
{\out ~(ALL x. EX y. ALL z. J(z,y) <-> ~J(z,x))}
{\out  1. (EX y. ALL x. J(y,x) <-> ~J(x,x)) -->}
{\out     ~(ALL x. EX y. ALL z. J(z,y) <-> ~J(z,x))}
\end{ttbox}
\ttindex{Blast_tac} proves subgoal~1 at a stroke.
\begin{ttbox}
by (Blast_tac 1);
{\out Depth = 0}
{\out Depth = 1}
{\out Level 1}
{\out (EX y. ALL x. J(y,x) <-> ~J(x,x)) -->}
{\out ~(ALL x. EX y. ALL z. J(z,y) <-> ~J(z,x))}
{\out No subgoals!}
\end{ttbox}
Sceptics may examine the proof by calling the package's single-step
tactics, such as~{\tt step_tac}.  This would take up much space, however,
so let us proceed to the next example:
\begin{ttbox}
goal FOL.thy "ALL x. P(x,f(x)) <-> \ttback
\ttback       (EX y. (ALL z. P(z,y) --> P(z,f(x))) & P(x,y))";
{\out Level 0}
{\out ALL x. P(x,f(x)) <-> (EX y. (ALL z. P(z,y) --> P(z,f(x))) & P(x,y))}
{\out  1. ALL x. P(x,f(x)) <->}
{\out     (EX y. (ALL z. P(z,y) --> P(z,f(x))) & P(x,y))}
\end{ttbox}
Again, subgoal~1 succumbs immediately.
\begin{ttbox}
by (Blast_tac 1);
{\out Depth = 0}
{\out Depth = 1}
{\out Level 1}
{\out ALL x. P(x,f(x)) <-> (EX y. (ALL z. P(z,y) --> P(z,f(x))) & P(x,y))}
{\out No subgoals!}
\end{ttbox}
The classical reasoner is not restricted to the usual logical connectives.
The natural deduction rules for unions and intersections resemble those for
disjunction and conjunction.  The rules for infinite unions and
intersections resemble those for quantifiers.  Given such rules, the classical
reasoner is effective for reasoning in set theory.
  

\chapter{Advanced Simplification, Recursion and Induction}

Although we have already learned a lot about simplification, recursion and
induction, there are some advanced proof techniques that we have not covered
yet and which are worth learning. The three sections of this chapter are almost
independent of each other and can be read in any order. Only the notion of
\emph{congruence rules}, introduced in the section on simplification, is
required for parts of the section on recursion.

%
\begin{isabellebody}%
\def\isabellecontext{simp}%
%
\isamarkupsubsubsection{Simplification rules%
}
%
\begin{isamarkuptext}%
\indexbold{simplification rule}
To facilitate simplification, theorems can be declared to be simplification
rules (with the help of the attribute \isa{{\isacharbrackleft}simp{\isacharbrackright}}\index{*simp
  (attribute)}), in which case proofs by simplification make use of these
rules automatically. In addition the constructs \isacommand{datatype} and
\isacommand{primrec} (and a few others) invisibly declare useful
simplification rules. Explicit definitions are \emph{not} declared
simplification rules automatically!

Not merely equations but pretty much any theorem can become a simplification
rule. The simplifier will try to make sense of it.  For example, a theorem
\isa{{\isasymnot}\ P} is automatically turned into \isa{P\ {\isacharequal}\ False}. The details
are explained in \S\ref{sec:SimpHow}.

The simplification attribute of theorems can be turned on and off as follows:
\begin{quote}
\isacommand{declare} \textit{theorem-name}\isa{{\isacharbrackleft}simp{\isacharbrackright}}\\
\isacommand{declare} \textit{theorem-name}\isa{{\isacharbrackleft}simp\ del{\isacharbrackright}}
\end{quote}
As a rule of thumb, equations that really simplify (like \isa{rev\ {\isacharparenleft}rev\ xs{\isacharparenright}\ {\isacharequal}\ xs} and \isa{xs\ {\isacharat}\ {\isacharbrackleft}{\isacharbrackright}\ {\isacharequal}\ xs}) should be made simplification
rules.  Those of a more specific nature (e.g.\ distributivity laws, which
alter the structure of terms considerably) should only be used selectively,
i.e.\ they should not be default simplification rules.  Conversely, it may
also happen that a simplification rule needs to be disabled in certain
proofs.  Frequent changes in the simplification status of a theorem may
indicate a badly designed theory.
\begin{warn}
  Simplification may not terminate, for example if both $f(x) = g(x)$ and
  $g(x) = f(x)$ are simplification rules. It is the user's responsibility not
  to include simplification rules that can lead to nontermination, either on
  their own or in combination with other simplification rules.
\end{warn}%
\end{isamarkuptext}%
%
\isamarkupsubsubsection{The simplification method%
}
%
\begin{isamarkuptext}%
\index{*simp (method)|bold}
The general format of the simplification method is
\begin{quote}
\isa{simp} \textit{list of modifiers}
\end{quote}
where the list of \emph{modifiers} helps to fine tune the behaviour and may
be empty. Most if not all of the proofs seen so far could have been performed
with \isa{simp} instead of \isa{auto}, except that \isa{simp} attacks
only the first subgoal and may thus need to be repeated---use
\isaindex{simp_all} to simplify all subgoals.
Note that \isa{simp} fails if nothing changes.%
\end{isamarkuptext}%
%
\isamarkupsubsubsection{Adding and deleting simplification rules%
}
%
\begin{isamarkuptext}%
If a certain theorem is merely needed in a few proofs by simplification,
we do not need to make it a global simplification rule. Instead we can modify
the set of simplification rules used in a simplification step by adding rules
to it and/or deleting rules from it. The two modifiers for this are
\begin{quote}
\isa{add{\isacharcolon}} \textit{list of theorem names}\\
\isa{del{\isacharcolon}} \textit{list of theorem names}
\end{quote}
In case you want to use only a specific list of theorems and ignore all
others:
\begin{quote}
\isa{only{\isacharcolon}} \textit{list of theorem names}
\end{quote}%
\end{isamarkuptext}%
%
\isamarkupsubsubsection{Assumptions%
}
%
\begin{isamarkuptext}%
\index{simplification!with/of assumptions}
By default, assumptions are part of the simplification process: they are used
as simplification rules and are simplified themselves. For example:%
\end{isamarkuptext}%
\isacommand{lemma}\ {\isachardoublequote}{\isasymlbrakk}\ xs\ {\isacharat}\ zs\ {\isacharequal}\ ys\ {\isacharat}\ xs{\isacharsemicolon}\ {\isacharbrackleft}{\isacharbrackright}\ {\isacharat}\ xs\ {\isacharequal}\ {\isacharbrackleft}{\isacharbrackright}\ {\isacharat}\ {\isacharbrackleft}{\isacharbrackright}\ {\isasymrbrakk}\ {\isasymLongrightarrow}\ ys\ {\isacharequal}\ zs{\isachardoublequote}\isanewline
\isacommand{apply}\ simp\isanewline
\isacommand{done}%
\begin{isamarkuptext}%
\noindent
The second assumption simplifies to \isa{xs\ {\isacharequal}\ {\isacharbrackleft}{\isacharbrackright}}, which in turn
simplifies the first assumption to \isa{zs\ {\isacharequal}\ ys}, thus reducing the
conclusion to \isa{ys\ {\isacharequal}\ ys} and hence to \isa{True}.

In some cases this may be too much of a good thing and may lead to
nontermination:%
\end{isamarkuptext}%
\isacommand{lemma}\ {\isachardoublequote}{\isasymforall}x{\isachardot}\ f\ x\ {\isacharequal}\ g\ {\isacharparenleft}f\ {\isacharparenleft}g\ x{\isacharparenright}{\isacharparenright}\ {\isasymLongrightarrow}\ f\ {\isacharbrackleft}{\isacharbrackright}\ {\isacharequal}\ f\ {\isacharbrackleft}{\isacharbrackright}\ {\isacharat}\ {\isacharbrackleft}{\isacharbrackright}{\isachardoublequote}%
\begin{isamarkuptxt}%
\noindent
cannot be solved by an unmodified application of \isa{simp} because the
simplification rule \isa{f\ x\ {\isacharequal}\ g\ {\isacharparenleft}f\ {\isacharparenleft}g\ x{\isacharparenright}{\isacharparenright}} extracted from the assumption
does not terminate. Isabelle notices certain simple forms of
nontermination but not this one. The problem can be circumvented by
explicitly telling the simplifier to ignore the assumptions:%
\end{isamarkuptxt}%
\isacommand{apply}{\isacharparenleft}simp\ {\isacharparenleft}no{\isacharunderscore}asm{\isacharparenright}{\isacharparenright}\isanewline
\isacommand{done}%
\begin{isamarkuptext}%
\noindent
There are three options that influence the treatment of assumptions:
\begin{description}
\item[\isa{{\isacharparenleft}no{\isacharunderscore}asm{\isacharparenright}}]\indexbold{*no_asm}
 means that assumptions are completely ignored.
\item[\isa{{\isacharparenleft}no{\isacharunderscore}asm{\isacharunderscore}simp{\isacharparenright}}]\indexbold{*no_asm_simp}
 means that the assumptions are not simplified but
  are used in the simplification of the conclusion.
\item[\isa{{\isacharparenleft}no{\isacharunderscore}asm{\isacharunderscore}use{\isacharparenright}}]\indexbold{*no_asm_use}
 means that the assumptions are simplified but are not
  used in the simplification of each other or the conclusion.
\end{description}
Neither \isa{{\isacharparenleft}no{\isacharunderscore}asm{\isacharunderscore}simp{\isacharparenright}} nor \isa{{\isacharparenleft}no{\isacharunderscore}asm{\isacharunderscore}use{\isacharparenright}} allow to simplify
the above problematic subgoal.

Note that only one of the above options is allowed, and it must precede all
other arguments.%
\end{isamarkuptext}%
%
\isamarkupsubsubsection{Rewriting with definitions%
}
%
\begin{isamarkuptext}%
\index{simplification!with definitions}
Constant definitions (\S\ref{sec:ConstDefinitions}) can
be used as simplification rules, but by default they are not.  Hence the
simplifier does not expand them automatically, just as it should be:
definitions are introduced for the purpose of abbreviating complex
concepts. Of course we need to expand the definitions initially to derive
enough lemmas that characterize the concept sufficiently for us to forget the
original definition. For example, given%
\end{isamarkuptext}%
\isacommand{constdefs}\ exor\ {\isacharcolon}{\isacharcolon}\ {\isachardoublequote}bool\ {\isasymRightarrow}\ bool\ {\isasymRightarrow}\ bool{\isachardoublequote}\isanewline
\ \ \ \ \ \ \ \ \ {\isachardoublequote}exor\ A\ B\ {\isasymequiv}\ {\isacharparenleft}A\ {\isasymand}\ {\isasymnot}B{\isacharparenright}\ {\isasymor}\ {\isacharparenleft}{\isasymnot}A\ {\isasymand}\ B{\isacharparenright}{\isachardoublequote}%
\begin{isamarkuptext}%
\noindent
we may want to prove%
\end{isamarkuptext}%
\isacommand{lemma}\ {\isachardoublequote}exor\ A\ {\isacharparenleft}{\isasymnot}A{\isacharparenright}{\isachardoublequote}%
\begin{isamarkuptxt}%
\noindent
Typically, the opening move consists in \emph{unfolding} the definition(s), which we need to
get started, but nothing else:\indexbold{*unfold}\indexbold{definition!unfolding}%
\end{isamarkuptxt}%
\isacommand{apply}{\isacharparenleft}simp\ only{\isacharcolon}exor{\isacharunderscore}def{\isacharparenright}%
\begin{isamarkuptxt}%
\noindent
In this particular case, the resulting goal
\begin{isabelle}%
\ {\isadigit{1}}{\isachardot}\ A\ {\isasymand}\ {\isasymnot}\ {\isasymnot}\ A\ {\isasymor}\ {\isasymnot}\ A\ {\isasymand}\ {\isasymnot}\ A%
\end{isabelle}
can be proved by simplification. Thus we could have proved the lemma outright by%
\end{isamarkuptxt}%
\isacommand{apply}{\isacharparenleft}simp\ add{\isacharcolon}\ exor{\isacharunderscore}def{\isacharparenright}%
\begin{isamarkuptext}%
\noindent
Of course we can also unfold definitions in the middle of a proof.

You should normally not turn a definition permanently into a simplification
rule because this defeats the whole purpose of an abbreviation.

\begin{warn}
  If you have defined $f\,x\,y~\isasymequiv~t$ then you can only expand
  occurrences of $f$ with at least two arguments. Thus it is safer to define
  $f$~\isasymequiv~\isasymlambda$x\,y.\;t$.
\end{warn}%
\end{isamarkuptext}%
%
\isamarkupsubsubsection{Simplifying let-expressions%
}
%
\begin{isamarkuptext}%
\index{simplification!of let-expressions}
Proving a goal containing \isaindex{let}-expressions almost invariably
requires the \isa{let}-con\-structs to be expanded at some point. Since
\isa{let}-\isa{in} is just syntactic sugar for a predefined constant
(called \isa{Let}), expanding \isa{let}-constructs means rewriting with
\isa{Let{\isacharunderscore}def}:%
\end{isamarkuptext}%
\isacommand{lemma}\ {\isachardoublequote}{\isacharparenleft}let\ xs\ {\isacharequal}\ {\isacharbrackleft}{\isacharbrackright}\ in\ xs{\isacharat}ys{\isacharat}xs{\isacharparenright}\ {\isacharequal}\ ys{\isachardoublequote}\isanewline
\isacommand{apply}{\isacharparenleft}simp\ add{\isacharcolon}\ Let{\isacharunderscore}def{\isacharparenright}\isanewline
\isacommand{done}%
\begin{isamarkuptext}%
If, in a particular context, there is no danger of a combinatorial explosion
of nested \isa{let}s one could even simlify with \isa{Let{\isacharunderscore}def} by
default:%
\end{isamarkuptext}%
\isacommand{declare}\ Let{\isacharunderscore}def\ {\isacharbrackleft}simp{\isacharbrackright}%
\isamarkupsubsubsection{Conditional equations%
}
%
\begin{isamarkuptext}%
So far all examples of rewrite rules were equations. The simplifier also
accepts \emph{conditional} equations, for example%
\end{isamarkuptext}%
\isacommand{lemma}\ hd{\isacharunderscore}Cons{\isacharunderscore}tl{\isacharbrackleft}simp{\isacharbrackright}{\isacharcolon}\ {\isachardoublequote}xs\ {\isasymnoteq}\ {\isacharbrackleft}{\isacharbrackright}\ \ {\isasymLongrightarrow}\ \ hd\ xs\ {\isacharhash}\ tl\ xs\ {\isacharequal}\ xs{\isachardoublequote}\isanewline
\isacommand{apply}{\isacharparenleft}case{\isacharunderscore}tac\ xs{\isacharcomma}\ simp{\isacharcomma}\ simp{\isacharparenright}\isanewline
\isacommand{done}%
\begin{isamarkuptext}%
\noindent
Note the use of ``\ttindexboldpos{,}{$Isar}'' to string together a
sequence of methods. Assuming that the simplification rule
\isa{{\isacharparenleft}rev\ xs\ {\isacharequal}\ {\isacharbrackleft}{\isacharbrackright}{\isacharparenright}\ {\isacharequal}\ {\isacharparenleft}xs\ {\isacharequal}\ {\isacharbrackleft}{\isacharbrackright}{\isacharparenright}}
is present as well,%
\end{isamarkuptext}%
\isacommand{lemma}\ {\isachardoublequote}xs\ {\isasymnoteq}\ {\isacharbrackleft}{\isacharbrackright}\ {\isasymLongrightarrow}\ hd{\isacharparenleft}rev\ xs{\isacharparenright}\ {\isacharhash}\ tl{\isacharparenleft}rev\ xs{\isacharparenright}\ {\isacharequal}\ rev\ xs{\isachardoublequote}%
\begin{isamarkuptext}%
\noindent
is proved by plain simplification:
the conditional equation \isa{hd{\isacharunderscore}Cons{\isacharunderscore}tl} above
can simplify \isa{hd\ {\isacharparenleft}rev\ xs{\isacharparenright}\ {\isacharhash}\ tl\ {\isacharparenleft}rev\ xs{\isacharparenright}} to \isa{rev\ xs}
because the corresponding precondition \isa{rev\ xs\ {\isasymnoteq}\ {\isacharbrackleft}{\isacharbrackright}}
simplifies to \isa{xs\ {\isasymnoteq}\ {\isacharbrackleft}{\isacharbrackright}}, which is exactly the local
assumption of the subgoal.%
\end{isamarkuptext}%
%
\isamarkupsubsubsection{Automatic case splits%
}
%
\begin{isamarkuptext}%
\indexbold{case splits}\index{*split|(}
Goals containing \isa{if}-expressions are usually proved by case
distinction on the condition of the \isa{if}. For example the goal%
\end{isamarkuptext}%
\isacommand{lemma}\ {\isachardoublequote}{\isasymforall}xs{\isachardot}\ if\ xs\ {\isacharequal}\ {\isacharbrackleft}{\isacharbrackright}\ then\ rev\ xs\ {\isacharequal}\ {\isacharbrackleft}{\isacharbrackright}\ else\ rev\ xs\ {\isasymnoteq}\ {\isacharbrackleft}{\isacharbrackright}{\isachardoublequote}%
\begin{isamarkuptxt}%
\noindent
can be split by a degenerate form of simplification%
\end{isamarkuptxt}%
\isacommand{apply}{\isacharparenleft}simp\ only{\isacharcolon}\ split{\isacharcolon}\ split{\isacharunderscore}if{\isacharparenright}%
\begin{isamarkuptxt}%
\noindent
\begin{isabelle}%
\ {\isadigit{1}}{\isachardot}\ {\isasymforall}xs{\isachardot}\ {\isacharparenleft}xs\ {\isacharequal}\ {\isacharbrackleft}{\isacharbrackright}\ {\isasymlongrightarrow}\ rev\ xs\ {\isacharequal}\ {\isacharbrackleft}{\isacharbrackright}{\isacharparenright}\ {\isasymand}\ {\isacharparenleft}xs\ {\isasymnoteq}\ {\isacharbrackleft}{\isacharbrackright}\ {\isasymlongrightarrow}\ rev\ xs\ {\isasymnoteq}\ {\isacharbrackleft}{\isacharbrackright}{\isacharparenright}%
\end{isabelle}
where no simplification rules are included (\isa{only{\isacharcolon}} is followed by the
empty list of theorems) but the rule \isaindexbold{split_if} for
splitting \isa{if}s is added (via the modifier \isa{split{\isacharcolon}}). Because
case-splitting on \isa{if}s is almost always the right proof strategy, the
simplifier performs it automatically. Try \isacommand{apply}\isa{{\isacharparenleft}simp{\isacharparenright}}
on the initial goal above.

This splitting idea generalizes from \isa{if} to \isaindex{case}:%
\end{isamarkuptxt}%
\isanewline
\isacommand{lemma}\ {\isachardoublequote}{\isacharparenleft}case\ xs\ of\ {\isacharbrackleft}{\isacharbrackright}\ {\isasymRightarrow}\ zs\ {\isacharbar}\ y{\isacharhash}ys\ {\isasymRightarrow}\ y{\isacharhash}{\isacharparenleft}ys{\isacharat}zs{\isacharparenright}{\isacharparenright}\ {\isacharequal}\ xs{\isacharat}zs{\isachardoublequote}\isanewline
\isacommand{apply}{\isacharparenleft}simp\ only{\isacharcolon}\ split{\isacharcolon}\ list{\isachardot}split{\isacharparenright}%
\begin{isamarkuptxt}%
\begin{isabelle}%
\ {\isadigit{1}}{\isachardot}\ {\isacharparenleft}xs\ {\isacharequal}\ {\isacharbrackleft}{\isacharbrackright}\ {\isasymlongrightarrow}\ zs\ {\isacharequal}\ xs\ {\isacharat}\ zs{\isacharparenright}\ {\isasymand}\isanewline
\ \ \ \ {\isacharparenleft}{\isasymforall}a\ list{\isachardot}\ xs\ {\isacharequal}\ a\ {\isacharhash}\ list\ {\isasymlongrightarrow}\ a\ {\isacharhash}\ list\ {\isacharat}\ zs\ {\isacharequal}\ xs\ {\isacharat}\ zs{\isacharparenright}%
\end{isabelle}
In contrast to \isa{if}-expressions, the simplifier does not split
\isa{case}-expressions by default because this can lead to nontermination
in case of recursive datatypes. Again, if the \isa{only{\isacharcolon}} modifier is
dropped, the above goal is solved,%
\end{isamarkuptxt}%
\isacommand{apply}{\isacharparenleft}simp\ split{\isacharcolon}\ list{\isachardot}split{\isacharparenright}%
\begin{isamarkuptext}%
\noindent%
which \isacommand{apply}\isa{{\isacharparenleft}simp{\isacharparenright}} alone will not do.

In general, every datatype $t$ comes with a theorem
$t$\isa{{\isachardot}split} which can be declared to be a \bfindex{split rule} either
locally as above, or by giving it the \isa{split} attribute globally:%
\end{isamarkuptext}%
\isacommand{declare}\ list{\isachardot}split\ {\isacharbrackleft}split{\isacharbrackright}%
\begin{isamarkuptext}%
\noindent
The \isa{split} attribute can be removed with the \isa{del} modifier,
either locally%
\end{isamarkuptext}%
\isacommand{apply}{\isacharparenleft}simp\ split\ del{\isacharcolon}\ split{\isacharunderscore}if{\isacharparenright}%
\begin{isamarkuptext}%
\noindent
or globally:%
\end{isamarkuptext}%
\isacommand{declare}\ list{\isachardot}split\ {\isacharbrackleft}split\ del{\isacharbrackright}%
\begin{isamarkuptext}%
The above split rules intentionally only affect the conclusion of a
subgoal.  If you want to split an \isa{if} or \isa{case}-expression in
the assumptions, you have to apply \isa{split{\isacharunderscore}if{\isacharunderscore}asm} or
$t$\isa{{\isachardot}split{\isacharunderscore}asm}:%
\end{isamarkuptext}%
\isacommand{lemma}\ {\isachardoublequote}if\ xs\ {\isacharequal}\ {\isacharbrackleft}{\isacharbrackright}\ then\ ys\ {\isachartilde}{\isacharequal}\ {\isacharbrackleft}{\isacharbrackright}\ else\ ys\ {\isacharequal}\ {\isacharbrackleft}{\isacharbrackright}\ {\isacharequal}{\isacharequal}{\isachargreater}\ xs\ {\isacharat}\ ys\ {\isachartilde}{\isacharequal}\ {\isacharbrackleft}{\isacharbrackright}{\isachardoublequote}\isanewline
\isacommand{apply}{\isacharparenleft}simp\ only{\isacharcolon}\ split{\isacharcolon}\ split{\isacharunderscore}if{\isacharunderscore}asm{\isacharparenright}%
\begin{isamarkuptxt}%
\noindent
In contrast to splitting the conclusion, this actually creates two
separate subgoals (which are solved by \isa{simp{\isacharunderscore}all}):
\begin{isabelle}%
\ {\isadigit{1}}{\isachardot}\ {\isasymlbrakk}xs\ {\isacharequal}\ {\isacharbrackleft}{\isacharbrackright}{\isacharsemicolon}\ ys\ {\isasymnoteq}\ {\isacharbrackleft}{\isacharbrackright}{\isasymrbrakk}\ {\isasymLongrightarrow}\ {\isacharbrackleft}{\isacharbrackright}\ {\isacharat}\ ys\ {\isasymnoteq}\ {\isacharbrackleft}{\isacharbrackright}\isanewline
\ {\isadigit{2}}{\isachardot}\ {\isasymlbrakk}xs\ {\isasymnoteq}\ {\isacharbrackleft}{\isacharbrackright}{\isacharsemicolon}\ ys\ {\isacharequal}\ {\isacharbrackleft}{\isacharbrackright}{\isasymrbrakk}\ {\isasymLongrightarrow}\ xs\ {\isacharat}\ {\isacharbrackleft}{\isacharbrackright}\ {\isasymnoteq}\ {\isacharbrackleft}{\isacharbrackright}%
\end{isabelle}
If you need to split both in the assumptions and the conclusion,
use $t$\isa{{\isachardot}splits} which subsumes $t$\isa{{\isachardot}split} and
$t$\isa{{\isachardot}split{\isacharunderscore}asm}. Analogously, there is \isa{if{\isacharunderscore}splits}.

\begin{warn}
  The simplifier merely simplifies the condition of an \isa{if} but not the
  \isa{then} or \isa{else} parts. The latter are simplified only after the
  condition reduces to \isa{True} or \isa{False}, or after splitting. The
  same is true for \isaindex{case}-expressions: only the selector is
  simplified at first, until either the expression reduces to one of the
  cases or it is split.
\end{warn}

\index{*split|)}%
\end{isamarkuptxt}%
%
\isamarkupsubsubsection{Arithmetic%
}
%
\begin{isamarkuptext}%
\index{arithmetic}
The simplifier routinely solves a small class of linear arithmetic formulae
(over type \isa{nat} and other numeric types): it only takes into account
assumptions and conclusions that are (possibly negated) (in)equalities
(\isa{{\isacharequal}}, \isasymle, \isa{{\isacharless}}) and it only knows about addition. Thus%
\end{isamarkuptext}%
\isacommand{lemma}\ {\isachardoublequote}{\isasymlbrakk}\ {\isasymnot}\ m\ {\isacharless}\ n{\isacharsemicolon}\ m\ {\isacharless}\ n{\isacharplus}{\isadigit{1}}\ {\isasymrbrakk}\ {\isasymLongrightarrow}\ m\ {\isacharequal}\ n{\isachardoublequote}%
\begin{isamarkuptext}%
\noindent
is proved by simplification, whereas the only slightly more complex%
\end{isamarkuptext}%
\isacommand{lemma}\ {\isachardoublequote}{\isasymnot}\ m\ {\isacharless}\ n\ {\isasymand}\ m\ {\isacharless}\ n{\isacharplus}{\isadigit{1}}\ {\isasymLongrightarrow}\ m\ {\isacharequal}\ n{\isachardoublequote}%
\begin{isamarkuptext}%
\noindent
is not proved by simplification and requires \isa{arith}.%
\end{isamarkuptext}%
%
\isamarkupsubsubsection{Tracing%
}
%
\begin{isamarkuptext}%
\indexbold{tracing the simplifier}
Using the simplifier effectively may take a bit of experimentation.  Set the
\isaindexbold{trace_simp} \rmindex{flag} to get a better idea of what is going
on:%
\end{isamarkuptext}%
\isacommand{ML}\ {\isachardoublequote}set\ trace{\isacharunderscore}simp{\isachardoublequote}\isanewline
\isacommand{lemma}\ {\isachardoublequote}rev\ {\isacharbrackleft}a{\isacharbrackright}\ {\isacharequal}\ {\isacharbrackleft}{\isacharbrackright}{\isachardoublequote}\isanewline
\isacommand{apply}{\isacharparenleft}simp{\isacharparenright}%
\begin{isamarkuptext}%
\noindent
produces the trace

\begin{ttbox}\makeatother
Applying instance of rewrite rule:
rev (?x1 \# ?xs1) == rev ?xs1 @ [?x1]
Rewriting:
rev [x] == rev [] @ [x]
Applying instance of rewrite rule:
rev [] == []
Rewriting:
rev [] == []
Applying instance of rewrite rule:
[] @ ?y == ?y
Rewriting:
[] @ [x] == [x]
Applying instance of rewrite rule:
?x3 \# ?t3 = ?t3 == False
Rewriting:
[x] = [] == False
\end{ttbox}

In more complicated cases, the trace can be quite lenghty, especially since
invocations of the simplifier are often nested (e.g.\ when solving conditions
of rewrite rules). Thus it is advisable to reset it:%
\end{isamarkuptext}%
\isacommand{ML}\ {\isachardoublequote}reset\ trace{\isacharunderscore}simp{\isachardoublequote}\isanewline
\end{isabellebody}%
%%% Local Variables:
%%% mode: latex
%%% TeX-master: "root"
%%% End:


\section{Advanced forms of recursion}
\index{*recdef|(}

The purpose of this section is to introduce advanced forms of
\isacommand{recdef}: how to establish termination by means other than measure
functions, how to define recursive function over nested recursive datatypes,
and how to deal with partial functions.

If, after reading this section, you feel that the definition of recursive
functions is overly complicated by the requirement of
totality, you should ponder the alternative, a logic of partial functions,
where recursive definitions are always wellformed. For a start, there are many
such logics, and no clear winner has emerged. And in all of these logics you
are (more or less frequently) required to reason about the definedness of
terms explicitly. Thus one shifts definedness arguments from definition time to
proof time. In HOL you may have to work hard to define a function, but proofs
can then proceed unencumbered by worries about undefinedness.

\subsection{Beyond measure}
\label{sec:beyond-measure}
%
\begin{isabellebody}%
\def\isabellecontext{WFrec}%
%
\begin{isamarkuptext}%
\noindent
So far, all recursive definitions where shown to terminate via measure
functions. Sometimes this can be quite inconvenient or even
impossible. Fortunately, \isacommand{recdef} supports much more
general definitions. For example, termination of Ackermann's function
can be shown by means of the lexicographic product \isa{{\isacharless}{\isacharasterisk}lex{\isacharasterisk}{\isachargreater}}:%
\end{isamarkuptext}%
\isacommand{consts}\ ack\ {\isacharcolon}{\isacharcolon}\ {\isachardoublequote}nat{\isasymtimes}nat\ {\isasymRightarrow}\ nat{\isachardoublequote}\isanewline
\isacommand{recdef}\ ack\ {\isachardoublequote}measure{\isacharparenleft}{\isasymlambda}m{\isachardot}\ m{\isacharparenright}\ {\isacharless}{\isacharasterisk}lex{\isacharasterisk}{\isachargreater}\ measure{\isacharparenleft}{\isasymlambda}n{\isachardot}\ n{\isacharparenright}{\isachardoublequote}\isanewline
\ \ {\isachardoublequote}ack{\isacharparenleft}{\isadigit{0}}{\isacharcomma}n{\isacharparenright}\ \ \ \ \ \ \ \ \ {\isacharequal}\ Suc\ n{\isachardoublequote}\isanewline
\ \ {\isachardoublequote}ack{\isacharparenleft}Suc\ m{\isacharcomma}{\isadigit{0}}{\isacharparenright}\ \ \ \ \ {\isacharequal}\ ack{\isacharparenleft}m{\isacharcomma}\ {\isadigit{1}}{\isacharparenright}{\isachardoublequote}\isanewline
\ \ {\isachardoublequote}ack{\isacharparenleft}Suc\ m{\isacharcomma}Suc\ n{\isacharparenright}\ {\isacharequal}\ ack{\isacharparenleft}m{\isacharcomma}ack{\isacharparenleft}Suc\ m{\isacharcomma}n{\isacharparenright}{\isacharparenright}{\isachardoublequote}%
\begin{isamarkuptext}%
\noindent
The lexicographic product decreases if either its first component
decreases (as in the second equation and in the outer call in the
third equation) or its first component stays the same and the second
component decreases (as in the inner call in the third equation).

In general, \isacommand{recdef} supports termination proofs based on
arbitrary \emph{wellfounded relations}, i.e.\ \emph{wellfounded
recursion}\indexbold{recursion!wellfounded}\index{wellfounded
recursion|see{recursion, wellfounded}}.  A relation $<$ is
\bfindex{wellfounded} if it has no infinite descending chain $\cdots <
a@2 < a@1 < a@0$. Clearly, a function definition is total iff the set
of all pairs $(r,l)$, where $l$ is the argument on the left-hand side
of an equation and $r$ the argument of some recursive call on the
corresponding right-hand side, induces a wellfounded relation.  For a
systematic account of termination proofs via wellfounded relations
see, for example, \cite{Baader-Nipkow}. The HOL library formalizes
some of the theory of wellfounded relations. For example
\isa{wf\ r}\index{*wf|bold} means that relation \isa{r{\isasymColon}{\isacharparenleft}{\isacharprime}a\ {\isasymtimes}\ {\isacharprime}a{\isacharparenright}\ set} is
wellfounded.

Each \isacommand{recdef} definition should be accompanied (after the
name of the function) by a wellfounded relation on the argument type
of the function. For example, \isa{measure} is defined by
\begin{isabelle}%
\ \ \ \ \ measure\ f\ {\isasymequiv}\ {\isacharbraceleft}{\isacharparenleft}y{\isacharcomma}\ x{\isacharparenright}{\isachardot}\ f\ y\ {\isacharless}\ f\ x{\isacharbraceright}%
\end{isabelle}
and it has been proved that \isa{measure\ f} is always wellfounded.

In addition to \isa{measure}, the library provides
a number of further constructions for obtaining wellfounded relations.
Above we have already met \isa{{\isacharless}{\isacharasterisk}lex{\isacharasterisk}{\isachargreater}} of type
\begin{isabelle}%
\ \ \ \ \ {\isachardoublequote}{\isacharparenleft}{\isacharprime}a\ {\isasymtimes}\ {\isacharprime}a{\isacharparenright}set\ {\isasymRightarrow}\ {\isacharparenleft}{\isacharprime}b\ {\isasymtimes}\ {\isacharprime}b{\isacharparenright}set\ {\isasymRightarrow}\ {\isacharparenleft}{\isacharparenleft}{\isacharprime}a\ {\isasymtimes}\ {\isacharprime}b{\isacharparenright}\ {\isasymtimes}\ {\isacharparenleft}{\isacharprime}a\ {\isasymtimes}\ {\isacharprime}b{\isacharparenright}{\isacharparenright}set{\isachardoublequote}%
\end{isabelle}
Of course the lexicographic product can also be interated, as in the following
function definition:%
\end{isamarkuptext}%
\isacommand{consts}\ contrived\ {\isacharcolon}{\isacharcolon}\ {\isachardoublequote}nat\ {\isasymtimes}\ nat\ {\isasymtimes}\ nat\ {\isasymRightarrow}\ nat{\isachardoublequote}\isanewline
\isacommand{recdef}\ contrived\isanewline
\ \ {\isachardoublequote}measure{\isacharparenleft}{\isasymlambda}i{\isachardot}\ i{\isacharparenright}\ {\isacharless}{\isacharasterisk}lex{\isacharasterisk}{\isachargreater}\ measure{\isacharparenleft}{\isasymlambda}j{\isachardot}\ j{\isacharparenright}\ {\isacharless}{\isacharasterisk}lex{\isacharasterisk}{\isachargreater}\ measure{\isacharparenleft}{\isasymlambda}k{\isachardot}\ k{\isacharparenright}{\isachardoublequote}\isanewline
{\isachardoublequote}contrived{\isacharparenleft}i{\isacharcomma}j{\isacharcomma}Suc\ k{\isacharparenright}\ {\isacharequal}\ contrived{\isacharparenleft}i{\isacharcomma}j{\isacharcomma}k{\isacharparenright}{\isachardoublequote}\isanewline
{\isachardoublequote}contrived{\isacharparenleft}i{\isacharcomma}Suc\ j{\isacharcomma}{\isadigit{0}}{\isacharparenright}\ {\isacharequal}\ contrived{\isacharparenleft}i{\isacharcomma}j{\isacharcomma}j{\isacharparenright}{\isachardoublequote}\isanewline
{\isachardoublequote}contrived{\isacharparenleft}Suc\ i{\isacharcomma}{\isadigit{0}}{\isacharcomma}{\isadigit{0}}{\isacharparenright}\ {\isacharequal}\ contrived{\isacharparenleft}i{\isacharcomma}i{\isacharcomma}i{\isacharparenright}{\isachardoublequote}\isanewline
{\isachardoublequote}contrived{\isacharparenleft}{\isadigit{0}}{\isacharcomma}{\isadigit{0}}{\isacharcomma}{\isadigit{0}}{\isacharparenright}\ \ \ \ \ {\isacharequal}\ {\isadigit{0}}{\isachardoublequote}%
\begin{isamarkuptext}%
Lexicographic products of measure functions already go a long way. A
further useful construction is the embedding of some type in an
existing wellfounded relation via the inverse image of a function:
\begin{isabelle}%
\ \ \ \ \ inv{\isacharunderscore}image\ {\isacharparenleft}r{\isasymColon}{\isacharparenleft}{\isacharprime}b\ {\isasymtimes}\ {\isacharprime}b{\isacharparenright}\ set{\isacharparenright}\ {\isacharparenleft}f{\isasymColon}{\isacharprime}a\ {\isasymRightarrow}\ {\isacharprime}b{\isacharparenright}\ {\isasymequiv}\isanewline
\ \ \ \ \ {\isacharbraceleft}{\isacharparenleft}x{\isasymColon}{\isacharprime}a{\isacharcomma}\ y{\isasymColon}{\isacharprime}a{\isacharparenright}{\isachardot}\ {\isacharparenleft}f\ x{\isacharcomma}\ f\ y{\isacharparenright}\ {\isasymin}\ r{\isacharbraceright}%
\end{isabelle}
\begin{sloppypar}
\noindent
For example, \isa{measure} is actually defined as \isa{inv{\isacharunderscore}mage\ less{\isacharunderscore}than}, where
\isa{less{\isacharunderscore}than} of type \isa{{\isacharparenleft}nat\ {\isasymtimes}\ nat{\isacharparenright}\ set} is the less-than relation on type \isa{nat}
(as opposed to \isa{op\ {\isacharless}}, which is of type \isa{{\isacharbrackleft}nat{\isacharcomma}\ nat{\isacharbrackright}\ {\isasymRightarrow}\ bool}).
\end{sloppypar}

%Finally there is also {finite_psubset} the proper subset relation on finite sets

All the above constructions are known to \isacommand{recdef}. Thus you
will never have to prove wellfoundedness of any relation composed
solely of these building blocks. But of course the proof of
termination of your function definition, i.e.\ that the arguments
decrease with every recursive call, may still require you to provide
additional lemmas.

It is also possible to use your own wellfounded relations with \isacommand{recdef}.
Here is a simplistic example:%
\end{isamarkuptext}%
\isacommand{consts}\ f\ {\isacharcolon}{\isacharcolon}\ {\isachardoublequote}nat\ {\isasymRightarrow}\ nat{\isachardoublequote}\isanewline
\isacommand{recdef}\ f\ {\isachardoublequote}id{\isacharparenleft}less{\isacharunderscore}than{\isacharparenright}{\isachardoublequote}\isanewline
{\isachardoublequote}f\ {\isadigit{0}}\ {\isacharequal}\ {\isadigit{0}}{\isachardoublequote}\isanewline
{\isachardoublequote}f\ {\isacharparenleft}Suc\ n{\isacharparenright}\ {\isacharequal}\ f\ n{\isachardoublequote}%
\begin{isamarkuptext}%
Since \isacommand{recdef} is not prepared for \isa{id}, the identity
function, this leads to the complaint that it could not prove
\isa{wf\ {\isacharparenleft}id\ less{\isacharunderscore}than{\isacharparenright}}, the wellfoundedness of \isa{id\ less{\isacharunderscore}than}. We should first have proved that \isa{id} preserves wellfoundedness%
\end{isamarkuptext}%
\isacommand{lemma}\ wf{\isacharunderscore}id{\isacharcolon}\ {\isachardoublequote}wf\ r\ {\isasymLongrightarrow}\ wf{\isacharparenleft}id\ r{\isacharparenright}{\isachardoublequote}\isanewline
\isacommand{by}\ simp%
\begin{isamarkuptext}%
\noindent
and should have added the following hint to our above definition:%
\end{isamarkuptext}%
{\isacharparenleft}\isakeyword{hints}\ recdef{\isacharunderscore}wf\ add{\isacharcolon}\ wf{\isacharunderscore}id{\isacharparenright}\end{isabellebody}%
%%% Local Variables:
%%% mode: latex
%%% TeX-master: "root"
%%% End:


\subsection{Recursion over nested datatypes}
\label{sec:nested-recdef}
%
\begin{isabellebody}%
\def\isabellecontext{Nested{\isadigit{0}}}%
%
\begin{isamarkuptext}%
\index{datatypes!nested}%
In \S\ref{sec:nested-datatype} we defined the datatype of terms%
\end{isamarkuptext}%
\isacommand{datatype}\ {\isacharparenleft}{\isacharprime}a{\isacharcomma}{\isacharprime}b{\isacharparenright}{\isachardoublequote}term{\isachardoublequote}\ {\isacharequal}\ Var\ {\isacharprime}a\ {\isacharbar}\ App\ {\isacharprime}b\ {\isachardoublequote}{\isacharparenleft}{\isacharprime}a{\isacharcomma}{\isacharprime}b{\isacharparenright}term\ list{\isachardoublequote}%
\begin{isamarkuptext}%
\noindent
and closed with the observation that the associated schema for the definition
of primitive recursive functions leads to overly verbose definitions. Moreover,
if you have worked exercise~\ref{ex:trev-trev} you will have noticed that
you needed to declare essentially the same function as \isa{rev}
and prove many standard properties of list reversal all over again. 
We will now show you how \isacommand{recdef} can simplify
definitions and proofs about nested recursive datatypes. As an example we
choose exercise~\ref{ex:trev-trev}:%
\end{isamarkuptext}%
\isacommand{consts}\ trev\ \ {\isacharcolon}{\isacharcolon}\ {\isachardoublequote}{\isacharparenleft}{\isacharprime}a{\isacharcomma}{\isacharprime}b{\isacharparenright}term\ {\isasymRightarrow}\ {\isacharparenleft}{\isacharprime}a{\isacharcomma}{\isacharprime}b{\isacharparenright}term{\isachardoublequote}\end{isabellebody}%
%%% Local Variables:
%%% mode: latex
%%% TeX-master: "root"
%%% End:

\begin{isabelle}%
\isacommand{consts}\ trev\ \ {\isacharcolon}{\isacharcolon}\ {\isachardoublequote}{\isacharparenleft}{\isacharprime}a{\isacharcomma}{\isacharprime}b{\isacharparenright}term\ {\isacharequal}{\isachargreater}\ {\isacharparenleft}{\isacharprime}a{\isacharcomma}{\isacharprime}b{\isacharparenright}term{\isachardoublequote}%
\begin{isamarkuptext}%
\noindent
Although the definition of \isa{trev} is quite natural, we will have
overcome a minor difficulty in convincing Isabelle of is termination.
It is precisely this difficulty that is the \textit{rasion d'\^etre} of
this subsection.

Defining \isa{trev} by \isacommand{recdef} rather than \isacommand{primrec}
simplifies matters because we are now free to use the recursion equation
suggested at the end of \S\ref{sec:nested-datatype}:%
\end{isamarkuptext}%
\isacommand{recdef}\ trev\ {\isachardoublequote}measure\ size{\isachardoublequote}\isanewline
\ {\isachardoublequote}trev\ {\isacharparenleft}Var\ x{\isacharparenright}\ {\isacharequal}\ Var\ x{\isachardoublequote}\isanewline
\ {\isachardoublequote}trev\ {\isacharparenleft}App\ f\ ts{\isacharparenright}\ {\isacharequal}\ App\ f\ {\isacharparenleft}rev{\isacharparenleft}map\ trev\ ts{\isacharparenright}{\isacharparenright}{\isachardoublequote}%
\begin{isamarkuptext}%
FIXME: recdef should complain and generate unprovable termination condition!
moveto todo

Remember that function \isa{size} is defined for each \isacommand{datatype}.
However, the definition does not succeed. Isabelle complains about an unproved termination
condition
\begin{quote}

\begin{isabelle}%
\mbox{t}\ {\isasymin}\ set\ \mbox{ts}\ {\isasymlongrightarrow}\ size\ \mbox{t}\ {\isacharless}\ Suc\ {\isacharparenleft}term{\isacharunderscore}size\ \mbox{ts}{\isacharparenright}
\end{isabelle}%

\end{quote}
where \isa{set} returns the set of elements of a list---no special knowledge of sets is
required in the following.
First we have to understand why the recursive call of \isa{trev} underneath \isa{map} leads
to the above condition. The reason is that \isacommand{recdef} ``knows'' that \isa{map} will
apply \isa{trev} only to elements of \isa{\mbox{ts}}. Thus the above condition expresses that
the size of the argument \isa{\mbox{t}\ {\isasymin}\ set\ \mbox{ts}} of any recursive call of \isa{trev} is strictly
less than \isa{size\ {\isacharparenleft}App\ \mbox{f}\ \mbox{ts}{\isacharparenright}\ {\isacharequal}\ Suc\ {\isacharparenleft}term{\isacharunderscore}size\ \mbox{ts}{\isacharparenright}}.
We will now prove the termination condition and continue with our definition.
Below we return to the question of how \isacommand{recdef} ``knows'' about \isa{map}.%
\end{isamarkuptext}%
\end{isabelle}%
%%% Local Variables:
%%% mode: latex
%%% TeX-master: "root"
%%% End:

%
\begin{isabellebody}%
\def\isabellecontext{Nested{\isadigit{2}}}%
%
\begin{isamarkuptext}%
\noindent
The termintion condition is easily proved by induction:%
\end{isamarkuptext}%
\isacommand{lemma}\ {\isacharbrackleft}simp{\isacharbrackright}{\isacharcolon}\ {\isachardoublequote}t\ {\isasymin}\ set\ ts\ {\isasymlongrightarrow}\ size\ t\ {\isacharless}\ Suc{\isacharparenleft}term{\isacharunderscore}list{\isacharunderscore}size\ ts{\isacharparenright}{\isachardoublequote}\isanewline
\isacommand{by}{\isacharparenleft}induct{\isacharunderscore}tac\ ts{\isacharcomma}\ auto{\isacharparenright}%
\begin{isamarkuptext}%
\noindent
By making this theorem a simplification rule, \isacommand{recdef}
applies it automatically and the definition of \isa{trev}
succeeds now. As a reward for our effort, we can now prove the desired
lemma directly.  We no longer need the verbose
induction schema for type \isa{term} and can use the simpler one arising from
\isa{trev}:%
\end{isamarkuptext}%
\isacommand{lemma}\ {\isachardoublequote}trev{\isacharparenleft}trev\ t{\isacharparenright}\ {\isacharequal}\ t{\isachardoublequote}\isanewline
\isacommand{apply}{\isacharparenleft}induct{\isacharunderscore}tac\ t\ rule{\isacharcolon}trev{\isachardot}induct{\isacharparenright}%
\begin{isamarkuptxt}%
\noindent
This leaves us with a trivial base case \isa{trev\ {\isacharparenleft}trev\ {\isacharparenleft}Var\ x{\isacharparenright}{\isacharparenright}\ {\isacharequal}\ Var\ x} and the step case
\begin{isabelle}%
\ \ \ \ \ {\isasymforall}t{\isachardot}\ t\ {\isasymin}\ set\ ts\ {\isasymlongrightarrow}\ trev\ {\isacharparenleft}trev\ t{\isacharparenright}\ {\isacharequal}\ t\ {\isasymLongrightarrow}\isanewline
\isaindent{\ \ \ \ \ }trev\ {\isacharparenleft}trev\ {\isacharparenleft}App\ f\ ts{\isacharparenright}{\isacharparenright}\ {\isacharequal}\ App\ f\ ts%
\end{isabelle}
both of which are solved by simplification:%
\end{isamarkuptxt}%
\isacommand{by}{\isacharparenleft}simp{\isacharunderscore}all\ add{\isacharcolon}rev{\isacharunderscore}map\ sym{\isacharbrackleft}OF\ map{\isacharunderscore}compose{\isacharbrackright}\ cong{\isacharcolon}map{\isacharunderscore}cong{\isacharparenright}%
\begin{isamarkuptext}%
\noindent
If the proof of the induction step mystifies you, we recommend that you go through
the chain of simplification steps in detail; you will probably need the help of
\isa{trace{\isacharunderscore}simp}. Theorem \isa{map{\isacharunderscore}cong} is discussed below.
%\begin{quote}
%{term[display]"trev(trev(App f ts))"}\\
%{term[display]"App f (rev(map trev (rev(map trev ts))))"}\\
%{term[display]"App f (map trev (rev(rev(map trev ts))))"}\\
%{term[display]"App f (map trev (map trev ts))"}\\
%{term[display]"App f (map (trev o trev) ts)"}\\
%{term[display]"App f (map (%x. x) ts)"}\\
%{term[display]"App f ts"}
%\end{quote}

The definition of \isa{trev} above is superior to the one in
\S\ref{sec:nested-datatype} because it uses \isa{rev}
and lets us use existing facts such as \hbox{\isa{rev\ {\isacharparenleft}rev\ xs{\isacharparenright}\ {\isacharequal}\ xs}}.
Thus this proof is a good example of an important principle:
\begin{quote}
\emph{Chose your definitions carefully\\
because they determine the complexity of your proofs.}
\end{quote}

Let us now return to the question of how \isacommand{recdef} can come up with
sensible termination conditions in the presence of higher-order functions
like \isa{map}. For a start, if nothing were known about \isa{map},
\isa{map\ trev\ ts} might apply \isa{trev} to arbitrary terms, and thus
\isacommand{recdef} would try to prove the unprovable \isa{size\ t\ {\isacharless}\ Suc\ {\isacharparenleft}term{\isacharunderscore}list{\isacharunderscore}size\ ts{\isacharparenright}}, without any assumption about \isa{t}.  Therefore
\isacommand{recdef} has been supplied with the congruence theorem
\isa{map{\isacharunderscore}cong}:
\begin{isabelle}%
\ \ \ \ \ {\isasymlbrakk}xs\ {\isacharequal}\ ys{\isacharsemicolon}\ {\isasymAnd}x{\isachardot}\ x\ {\isasymin}\ set\ ys\ {\isasymLongrightarrow}\ f\ x\ {\isacharequal}\ g\ x{\isasymrbrakk}\isanewline
\isaindent{\ \ \ \ \ }{\isasymLongrightarrow}\ map\ f\ xs\ {\isacharequal}\ map\ g\ ys%
\end{isabelle}
Its second premise expresses (indirectly) that the second argument of
\isa{map} is only applied to elements of its third argument. Congruence
rules for other higher-order functions on lists look very similar. If you get
into a situation where you need to supply \isacommand{recdef} with new
congruence rules, you can either append a hint locally
to the specific occurrence of \isacommand{recdef}%
\end{isamarkuptext}%
{\isacharparenleft}\isakeyword{hints}\ recdef{\isacharunderscore}cong{\isacharcolon}\ map{\isacharunderscore}cong{\isacharparenright}%
\begin{isamarkuptext}%
\noindent
or declare them globally
by giving them the \isaindexbold{recdef_cong} attribute as in%
\end{isamarkuptext}%
\isacommand{declare}\ map{\isacharunderscore}cong{\isacharbrackleft}recdef{\isacharunderscore}cong{\isacharbrackright}%
\begin{isamarkuptext}%
Note that the \isa{cong} and \isa{recdef{\isacharunderscore}cong} attributes are
intentionally kept apart because they control different activities, namely
simplification and making recursive definitions.
% The local \isa{cong} in
% the hints section of \isacommand{recdef} is merely short for \isa{recdef{\isacharunderscore}cong}.
%The simplifier's congruence rules cannot be used by recdef.
%For example the weak congruence rules for if and case would prevent
%recdef from generating sensible termination conditions.%
\end{isamarkuptext}%
\end{isabellebody}%
%%% Local Variables:
%%% mode: latex
%%% TeX-master: "root"
%%% End:


\subsection{Partial functions}
\index{partial function}
%
\begin{isabellebody}%
\def\isabellecontext{Partial}%
\isamarkupfalse%
%
\begin{isamarkuptext}%
\noindent Throughout this tutorial, we have emphasized
that all functions in HOL are total.  We cannot hope to define
truly partial functions, but must make them total.  A straightforward
method is to lift the result type of the function from $\tau$ to
$\tau$~\isa{option} (see \ref{sec:option}), where \isa{None} is
returned if the function is applied to an argument not in its
domain. Function \isa{assoc} in \S\ref{sec:Trie} is a simple example.
We do not pursue this schema further because it should be clear
how it works. Its main drawback is that the result of such a lifted
function has to be unpacked first before it can be processed
further. Its main advantage is that you can distinguish if the
function was applied to an argument in its domain or not. If you do
not need to make this distinction, for example because the function is
never used outside its domain, it is easier to work with
\emph{underdefined}\index{functions!underdefined} functions: for
certain arguments we only know that a result exists, but we do not
know what it is. When defining functions that are normally considered
partial, underdefinedness turns out to be a very reasonable
alternative.

We have already seen an instance of underdefinedness by means of
non-exhaustive pattern matching: the definition of \isa{last} in
\S\ref{sec:recdef-examples}. The same is allowed for \isacommand{primrec}%
\end{isamarkuptext}%
\isamarkuptrue%
\isacommand{consts}\ hd\ {\isacharcolon}{\isacharcolon}\ {\isachardoublequote}{\isacharprime}a\ list\ {\isasymRightarrow}\ {\isacharprime}a{\isachardoublequote}\isanewline
\isamarkupfalse%
\isacommand{primrec}\ {\isachardoublequote}hd\ {\isacharparenleft}x{\isacharhash}xs{\isacharparenright}\ {\isacharequal}\ x{\isachardoublequote}\isamarkupfalse%
%
\begin{isamarkuptext}%
\noindent
although it generates a warning.
Even ordinary definitions allow underdefinedness, this time by means of
preconditions:%
\end{isamarkuptext}%
\isamarkuptrue%
\isacommand{constdefs}\ minus\ {\isacharcolon}{\isacharcolon}\ {\isachardoublequote}nat\ {\isasymRightarrow}\ nat\ {\isasymRightarrow}\ nat{\isachardoublequote}\isanewline
{\isachardoublequote}n\ {\isasymle}\ m\ {\isasymLongrightarrow}\ minus\ m\ n\ {\isasymequiv}\ m\ {\isacharminus}\ n{\isachardoublequote}\isamarkupfalse%
%
\begin{isamarkuptext}%
The rest of this section is devoted to the question of how to define
partial recursive functions by other means than non-exhaustive pattern
matching.%
\end{isamarkuptext}%
\isamarkuptrue%
%
\isamarkupsubsubsection{Guarded Recursion%
}
\isamarkuptrue%
%
\begin{isamarkuptext}%
\index{recursion!guarded}%
Neither \isacommand{primrec} nor \isacommand{recdef} allow to
prefix an equation with a condition in the way ordinary definitions do
(see \isa{minus} above). Instead we have to move the condition over
to the right-hand side of the equation. Given a partial function $f$
that should satisfy the recursion equation $f(x) = t$ over its domain
$dom(f)$, we turn this into the \isacommand{recdef}
\begin{isabelle}%
\ \ \ \ \ f\ x\ {\isacharequal}\ {\isacharparenleft}if\ x\ {\isasymin}\ dom\ f\ then\ t\ else\ arbitrary{\isacharparenright}%
\end{isabelle}
where \isa{arbitrary} is a predeclared constant of type \isa{{\isacharprime}a}
which has no definition. Thus we know nothing about its value,
which is ideal for specifying underdefined functions on top of it.

As a simple example we define division on \isa{nat}:%
\end{isamarkuptext}%
\isamarkuptrue%
\isacommand{consts}\ divi\ {\isacharcolon}{\isacharcolon}\ {\isachardoublequote}nat\ {\isasymtimes}\ nat\ {\isasymRightarrow}\ nat{\isachardoublequote}\isanewline
\isamarkupfalse%
\isacommand{recdef}\ divi\ {\isachardoublequote}measure{\isacharparenleft}{\isasymlambda}{\isacharparenleft}m{\isacharcomma}n{\isacharparenright}{\isachardot}\ m{\isacharparenright}{\isachardoublequote}\isanewline
\ \ {\isachardoublequote}divi{\isacharparenleft}m{\isacharcomma}{\isadigit{0}}{\isacharparenright}\ {\isacharequal}\ arbitrary{\isachardoublequote}\isanewline
\ \ {\isachardoublequote}divi{\isacharparenleft}m{\isacharcomma}n{\isacharparenright}\ {\isacharequal}\ {\isacharparenleft}if\ m\ {\isacharless}\ n\ then\ {\isadigit{0}}\ else\ divi{\isacharparenleft}m{\isacharminus}n{\isacharcomma}n{\isacharparenright}{\isacharplus}{\isadigit{1}}{\isacharparenright}{\isachardoublequote}\isamarkupfalse%
%
\begin{isamarkuptext}%
\noindent Of course we could also have defined
\isa{divi\ {\isacharparenleft}m{\isacharcomma}\ {\isadigit{0}}{\isacharparenright}} to be some specific number, for example 0. The
latter option is chosen for the predefined \isa{div} function, which
simplifies proofs at the expense of deviating from the
standard mathematical division function.

As a more substantial example we consider the problem of searching a graph.
For simplicity our graph is given by a function \isa{f} of
type \isa{{\isacharprime}a\ {\isasymRightarrow}\ {\isacharprime}a} which
maps each node to its successor; the graph has out-degree 1.
The task is to find the end of a chain, modelled by a node pointing to
itself. Here is a first attempt:
\begin{isabelle}%
\ \ \ \ \ find\ {\isacharparenleft}f{\isacharcomma}\ x{\isacharparenright}\ {\isacharequal}\ {\isacharparenleft}if\ f\ x\ {\isacharequal}\ x\ then\ x\ else\ find\ {\isacharparenleft}f{\isacharcomma}\ f\ x{\isacharparenright}{\isacharparenright}%
\end{isabelle}
This may be viewed as a fixed point finder or as the second half of the well
known \emph{Union-Find} algorithm.
The snag is that it may not terminate if \isa{f} has non-trivial cycles.
Phrased differently, the relation%
\end{isamarkuptext}%
\isamarkuptrue%
\isacommand{constdefs}\ step{\isadigit{1}}\ {\isacharcolon}{\isacharcolon}\ {\isachardoublequote}{\isacharparenleft}{\isacharprime}a\ {\isasymRightarrow}\ {\isacharprime}a{\isacharparenright}\ {\isasymRightarrow}\ {\isacharparenleft}{\isacharprime}a\ {\isasymtimes}\ {\isacharprime}a{\isacharparenright}set{\isachardoublequote}\isanewline
\ \ {\isachardoublequote}step{\isadigit{1}}\ f\ {\isasymequiv}\ {\isacharbraceleft}{\isacharparenleft}y{\isacharcomma}x{\isacharparenright}{\isachardot}\ y\ {\isacharequal}\ f\ x\ {\isasymand}\ y\ {\isasymnoteq}\ x{\isacharbraceright}{\isachardoublequote}\isamarkupfalse%
%
\begin{isamarkuptext}%
\noindent
must be well-founded. Thus we make the following definition:%
\end{isamarkuptext}%
\isamarkuptrue%
\isacommand{consts}\ find\ {\isacharcolon}{\isacharcolon}\ {\isachardoublequote}{\isacharparenleft}{\isacharprime}a\ {\isasymRightarrow}\ {\isacharprime}a{\isacharparenright}\ {\isasymtimes}\ {\isacharprime}a\ {\isasymRightarrow}\ {\isacharprime}a{\isachardoublequote}\isanewline
\isamarkupfalse%
\isacommand{recdef}\ find\ {\isachardoublequote}same{\isacharunderscore}fst\ {\isacharparenleft}{\isasymlambda}f{\isachardot}\ wf{\isacharparenleft}step{\isadigit{1}}\ f{\isacharparenright}{\isacharparenright}\ step{\isadigit{1}}{\isachardoublequote}\isanewline
\ \ {\isachardoublequote}find{\isacharparenleft}f{\isacharcomma}x{\isacharparenright}\ {\isacharequal}\ {\isacharparenleft}if\ wf{\isacharparenleft}step{\isadigit{1}}\ f{\isacharparenright}\isanewline
\ \ \ \ \ \ \ \ \ \ \ \ \ \ \ \ then\ if\ f\ x\ {\isacharequal}\ x\ then\ x\ else\ find{\isacharparenleft}f{\isacharcomma}\ f\ x{\isacharparenright}\isanewline
\ \ \ \ \ \ \ \ \ \ \ \ \ \ \ \ else\ arbitrary{\isacharparenright}{\isachardoublequote}\isanewline
{\isacharparenleft}\isakeyword{hints}\ recdef{\isacharunderscore}simp{\isacharcolon}\ step{\isadigit{1}}{\isacharunderscore}def{\isacharparenright}\isamarkupfalse%
%
\begin{isamarkuptext}%
\noindent
The recursion equation itself should be clear enough: it is our aborted
first attempt augmented with a check that there are no non-trivial loops.
To express the required well-founded relation we employ the
predefined combinator \isa{same{\isacharunderscore}fst} of type
\begin{isabelle}%
\ \ \ \ \ {\isacharparenleft}{\isacharprime}a\ {\isasymRightarrow}\ bool{\isacharparenright}\ {\isasymRightarrow}\ {\isacharparenleft}{\isacharprime}a\ {\isasymRightarrow}\ {\isacharparenleft}{\isacharprime}b{\isasymtimes}{\isacharprime}b{\isacharparenright}set{\isacharparenright}\ {\isasymRightarrow}\ {\isacharparenleft}{\isacharparenleft}{\isacharprime}a{\isasymtimes}{\isacharprime}b{\isacharparenright}\ {\isasymtimes}\ {\isacharparenleft}{\isacharprime}a{\isasymtimes}{\isacharprime}b{\isacharparenright}{\isacharparenright}set%
\end{isabelle}
defined as
\begin{isabelle}%
\ \ \ \ \ same{\isacharunderscore}fst\ P\ R\ {\isasymequiv}\ {\isacharbraceleft}{\isacharparenleft}{\isacharparenleft}x{\isacharprime}{\isacharcomma}\ y{\isacharprime}{\isacharparenright}{\isacharcomma}\ x{\isacharcomma}\ y{\isacharparenright}{\isachardot}\ x{\isacharprime}\ {\isacharequal}\ x\ {\isasymand}\ P\ x\ {\isasymand}\ {\isacharparenleft}y{\isacharprime}{\isacharcomma}\ y{\isacharparenright}\ {\isasymin}\ R\ x{\isacharbraceright}%
\end{isabelle}
This combinator is designed for
recursive functions on pairs where the first component of the argument is
passed unchanged to all recursive calls. Given a constraint on the first
component and a relation on the second component, \isa{same{\isacharunderscore}fst} builds the
required relation on pairs.  The theorem
\begin{isabelle}%
\ \ \ \ \ {\isacharparenleft}{\isasymAnd}x{\isachardot}\ P\ x\ {\isasymLongrightarrow}\ wf\ {\isacharparenleft}R\ x{\isacharparenright}{\isacharparenright}\ {\isasymLongrightarrow}\ wf\ {\isacharparenleft}same{\isacharunderscore}fst\ P\ R{\isacharparenright}%
\end{isabelle}
is known to the well-foundedness prover of \isacommand{recdef}.  Thus
well-foundedness of the relation given to \isacommand{recdef} is immediate.
Furthermore, each recursive call descends along that relation: the first
argument stays unchanged and the second one descends along \isa{step{\isadigit{1}}\ f}. The proof requires unfolding the definition of \isa{step{\isadigit{1}}},
as specified in the \isacommand{hints} above.

Normally you will then derive the following conditional variant from
the recursion equation:%
\end{isamarkuptext}%
\isamarkuptrue%
\isacommand{lemma}\ {\isacharbrackleft}simp{\isacharbrackright}{\isacharcolon}\isanewline
\ \ {\isachardoublequote}wf{\isacharparenleft}step{\isadigit{1}}\ f{\isacharparenright}\ {\isasymLongrightarrow}\ find{\isacharparenleft}f{\isacharcomma}x{\isacharparenright}\ {\isacharequal}\ {\isacharparenleft}if\ f\ x\ {\isacharequal}\ x\ then\ x\ else\ find{\isacharparenleft}f{\isacharcomma}\ f\ x{\isacharparenright}{\isacharparenright}{\isachardoublequote}\isanewline
\isamarkupfalse%
\isacommand{by}\ simp\isamarkupfalse%
%
\begin{isamarkuptext}%
\noindent Then you should disable the original recursion equation:%
\end{isamarkuptext}%
\isamarkuptrue%
\isacommand{declare}\ find{\isachardot}simps{\isacharbrackleft}simp\ del{\isacharbrackright}\isamarkupfalse%
%
\begin{isamarkuptext}%
Reasoning about such underdefined functions is like that for other
recursive functions.  Here is a simple example of recursion induction:%
\end{isamarkuptext}%
\isamarkuptrue%
\isacommand{lemma}\ {\isachardoublequote}wf{\isacharparenleft}step{\isadigit{1}}\ f{\isacharparenright}\ {\isasymlongrightarrow}\ f{\isacharparenleft}find{\isacharparenleft}f{\isacharcomma}x{\isacharparenright}{\isacharparenright}\ {\isacharequal}\ find{\isacharparenleft}f{\isacharcomma}x{\isacharparenright}{\isachardoublequote}\isanewline
\isamarkupfalse%
\isacommand{apply}{\isacharparenleft}induct{\isacharunderscore}tac\ f\ x\ rule{\isacharcolon}find{\isachardot}induct{\isacharparenright}\isanewline
\isamarkupfalse%
\isacommand{apply}\ simp\isanewline
\isamarkupfalse%
\isacommand{done}\isamarkupfalse%
%
\isamarkupsubsubsection{The {\tt\slshape while} Combinator%
}
\isamarkuptrue%
%
\begin{isamarkuptext}%
If the recursive function happens to be tail recursive, its
definition becomes a triviality if based on the predefined \cdx{while}
combinator.  The latter lives in the Library theory \thydx{While_Combinator}.
% which is not part of {text Main} but needs to
% be included explicitly among the ancestor theories.

Constant \isa{while} is of type \isa{{\isacharparenleft}{\isacharprime}a\ {\isasymRightarrow}\ bool{\isacharparenright}\ {\isasymRightarrow}\ {\isacharparenleft}{\isacharprime}a\ {\isasymRightarrow}\ {\isacharprime}a{\isacharparenright}\ {\isasymRightarrow}\ {\isacharprime}a}
and satisfies the recursion equation \begin{isabelle}%
\ \ \ \ \ while\ b\ c\ s\ {\isacharequal}\ {\isacharparenleft}if\ b\ s\ then\ while\ b\ c\ {\isacharparenleft}c\ s{\isacharparenright}\ else\ s{\isacharparenright}%
\end{isabelle}
That is, \isa{while\ b\ c\ s} is equivalent to the imperative program
\begin{verbatim}
     x := s; while b(x) do x := c(x); return x
\end{verbatim}
In general, \isa{s} will be a tuple or record.  As an example
consider the following definition of function \isa{find}:%
\end{isamarkuptext}%
\isamarkuptrue%
\isacommand{constdefs}\ find{\isadigit{2}}\ {\isacharcolon}{\isacharcolon}\ {\isachardoublequote}{\isacharparenleft}{\isacharprime}a\ {\isasymRightarrow}\ {\isacharprime}a{\isacharparenright}\ {\isasymRightarrow}\ {\isacharprime}a\ {\isasymRightarrow}\ {\isacharprime}a{\isachardoublequote}\isanewline
\ \ {\isachardoublequote}find{\isadigit{2}}\ f\ x\ {\isasymequiv}\isanewline
\ \ \ fst{\isacharparenleft}while\ {\isacharparenleft}{\isasymlambda}{\isacharparenleft}x{\isacharcomma}x{\isacharprime}{\isacharparenright}{\isachardot}\ x{\isacharprime}\ {\isasymnoteq}\ x{\isacharparenright}\ {\isacharparenleft}{\isasymlambda}{\isacharparenleft}x{\isacharcomma}x{\isacharprime}{\isacharparenright}{\isachardot}\ {\isacharparenleft}x{\isacharprime}{\isacharcomma}f\ x{\isacharprime}{\isacharparenright}{\isacharparenright}\ {\isacharparenleft}x{\isacharcomma}f\ x{\isacharparenright}{\isacharparenright}{\isachardoublequote}\isamarkupfalse%
%
\begin{isamarkuptext}%
\noindent
The loop operates on two ``local variables'' \isa{x} and \isa{x{\isacharprime}}
containing the ``current'' and the ``next'' value of function \isa{f}.
They are initialized with the global \isa{x} and \isa{f\ x}. At the
end \isa{fst} selects the local \isa{x}.

Although the definition of tail recursive functions via \isa{while} avoids
termination proofs, there is no free lunch. When proving properties of
functions defined by \isa{while}, termination rears its ugly head
again. Here is \tdx{while_rule}, the well known proof rule for total
correctness of loops expressed with \isa{while}:
\begin{isabelle}%
\ \ \ \ \ {\isasymlbrakk}P\ s{\isacharsemicolon}\ {\isasymAnd}s{\isachardot}\ {\isasymlbrakk}P\ s{\isacharsemicolon}\ b\ s{\isasymrbrakk}\ {\isasymLongrightarrow}\ P\ {\isacharparenleft}c\ s{\isacharparenright}{\isacharsemicolon}\isanewline
\isaindent{\ \ \ \ \ \ \ \ }{\isasymAnd}s{\isachardot}\ {\isasymlbrakk}P\ s{\isacharsemicolon}\ {\isasymnot}\ b\ s{\isasymrbrakk}\ {\isasymLongrightarrow}\ Q\ s{\isacharsemicolon}\ wf\ r{\isacharsemicolon}\isanewline
\isaindent{\ \ \ \ \ \ \ \ }{\isasymAnd}s{\isachardot}\ {\isasymlbrakk}P\ s{\isacharsemicolon}\ b\ s{\isasymrbrakk}\ {\isasymLongrightarrow}\ {\isacharparenleft}c\ s{\isacharcomma}\ s{\isacharparenright}\ {\isasymin}\ r{\isasymrbrakk}\isanewline
\isaindent{\ \ \ \ \ }{\isasymLongrightarrow}\ Q\ {\isacharparenleft}while\ b\ c\ s{\isacharparenright}%
\end{isabelle} \isa{P} needs to be true of
the initial state \isa{s} and invariant under \isa{c} (premises 1
and~2). The post-condition \isa{Q} must become true when leaving the loop
(premise~3). And each loop iteration must descend along a well-founded
relation \isa{r} (premises 4 and~5).

Let us now prove that \isa{find{\isadigit{2}}} does indeed find a fixed point. Instead
of induction we apply the above while rule, suitably instantiated.
Only the final premise of \isa{while{\isacharunderscore}rule} is left unproved
by \isa{auto} but falls to \isa{simp}:%
\end{isamarkuptext}%
\isamarkuptrue%
\isacommand{lemma}\ lem{\isacharcolon}\ {\isachardoublequote}wf{\isacharparenleft}step{\isadigit{1}}\ f{\isacharparenright}\ {\isasymLongrightarrow}\isanewline
\ \ {\isasymexists}y{\isachardot}\ while\ {\isacharparenleft}{\isasymlambda}{\isacharparenleft}x{\isacharcomma}x{\isacharprime}{\isacharparenright}{\isachardot}\ x{\isacharprime}\ {\isasymnoteq}\ x{\isacharparenright}\ {\isacharparenleft}{\isasymlambda}{\isacharparenleft}x{\isacharcomma}x{\isacharprime}{\isacharparenright}{\isachardot}\ {\isacharparenleft}x{\isacharprime}{\isacharcomma}f\ x{\isacharprime}{\isacharparenright}{\isacharparenright}\ {\isacharparenleft}x{\isacharcomma}f\ x{\isacharparenright}\ {\isacharequal}\ {\isacharparenleft}y{\isacharcomma}y{\isacharparenright}\ {\isasymand}\isanewline
\ \ \ \ \ \ \ f\ y\ {\isacharequal}\ y{\isachardoublequote}\isanewline
\isamarkupfalse%
\isacommand{apply}{\isacharparenleft}rule{\isacharunderscore}tac\ P\ {\isacharequal}\ {\isachardoublequote}{\isasymlambda}{\isacharparenleft}x{\isacharcomma}x{\isacharprime}{\isacharparenright}{\isachardot}\ x{\isacharprime}\ {\isacharequal}\ f\ x{\isachardoublequote}\ \isakeyword{and}\isanewline
\ \ \ \ \ \ \ \ \ \ \ \ \ \ \ r\ {\isacharequal}\ {\isachardoublequote}inv{\isacharunderscore}image\ {\isacharparenleft}step{\isadigit{1}}\ f{\isacharparenright}\ fst{\isachardoublequote}\ \isakeyword{in}\ while{\isacharunderscore}rule{\isacharparenright}\isanewline
\isamarkupfalse%
\isacommand{apply}\ auto\isanewline
\isamarkupfalse%
\isacommand{apply}{\isacharparenleft}simp\ add{\isacharcolon}inv{\isacharunderscore}image{\isacharunderscore}def\ step{\isadigit{1}}{\isacharunderscore}def{\isacharparenright}\isanewline
\isamarkupfalse%
\isacommand{done}\isamarkupfalse%
%
\begin{isamarkuptext}%
The theorem itself is a simple consequence of this lemma:%
\end{isamarkuptext}%
\isamarkuptrue%
\isacommand{theorem}\ {\isachardoublequote}wf{\isacharparenleft}step{\isadigit{1}}\ f{\isacharparenright}\ {\isasymLongrightarrow}\ f{\isacharparenleft}find{\isadigit{2}}\ f\ x{\isacharparenright}\ {\isacharequal}\ find{\isadigit{2}}\ f\ x{\isachardoublequote}\isanewline
\isamarkupfalse%
\isacommand{apply}{\isacharparenleft}drule{\isacharunderscore}tac\ x\ {\isacharequal}\ x\ \isakeyword{in}\ lem{\isacharparenright}\isanewline
\isamarkupfalse%
\isacommand{apply}{\isacharparenleft}auto\ simp\ add{\isacharcolon}find{\isadigit{2}}{\isacharunderscore}def{\isacharparenright}\isanewline
\isamarkupfalse%
\isacommand{done}\isamarkupfalse%
%
\begin{isamarkuptext}%
Let us conclude this section on partial functions by a
discussion of the merits of the \isa{while} combinator. We have
already seen that the advantage of not having to
provide a termination argument when defining a function via \isa{while} merely puts off the evil hour. On top of that, tail recursive
functions tend to be more complicated to reason about. So why use
\isa{while} at all? The only reason is executability: the recursion
equation for \isa{while} is a directly executable functional
program. This is in stark contrast to guarded recursion as introduced
above which requires an explicit test \isa{x\ {\isasymin}\ dom\ f} in the
function body.  Unless \isa{dom} is trivial, this leads to a
definition that is impossible to execute or prohibitively slow.
Thus, if you are aiming for an efficiently executable definition
of a partial function, you are likely to need \isa{while}.%
\end{isamarkuptext}%
\isamarkuptrue%
\isamarkupfalse%
\end{isabellebody}%
%%% Local Variables:
%%% mode: latex
%%% TeX-master: "root"
%%% End:


\index{*recdef|)}

\section{Advanced induction techniques}
\label{sec:advanced-ind}
\index{induction|(}
%
\begin{isabellebody}%
\def\isabellecontext{AdvancedInd}%
%
\begin{isamarkuptext}%
\noindent
Now that we have learned about rules and logic, we take another look at the
finer points of induction. The two questions we answer are: what to do if the
proposition to be proved is not directly amenable to induction
(\S\ref{sec:ind-var-in-prems}), and how to utilize (\S\ref{sec:complete-ind})
and even derive (\S\ref{sec:derive-ind}) new induction schemas. We conclude
with an extended example of induction (\S\ref{sec:CTL-revisited}).%
\end{isamarkuptext}%
%
\isamarkupsubsection{Massaging the Proposition%
}
%
\begin{isamarkuptext}%
\label{sec:ind-var-in-prems}
Often we have assumed that the theorem we want to prove is already in a form
that is amenable to induction, but sometimes it isn't.
Here is an example.
Since \isa{hd} and \isa{last} return the first and last element of a
non-empty list, this lemma looks easy to prove:%
\end{isamarkuptext}%
\isacommand{lemma}\ {\isachardoublequote}xs\ {\isasymnoteq}\ {\isacharbrackleft}{\isacharbrackright}\ {\isasymLongrightarrow}\ hd{\isacharparenleft}rev\ xs{\isacharparenright}\ {\isacharequal}\ last\ xs{\isachardoublequote}\isanewline
\isacommand{apply}{\isacharparenleft}induct{\isacharunderscore}tac\ xs{\isacharparenright}%
\begin{isamarkuptxt}%
\noindent
But induction produces the warning
\begin{quote}\tt
Induction variable occurs also among premises!
\end{quote}
and leads to the base case
\begin{isabelle}%
\ {\isadigit{1}}{\isachardot}\ xs\ {\isasymnoteq}\ {\isacharbrackleft}{\isacharbrackright}\ {\isasymLongrightarrow}\ hd\ {\isacharparenleft}rev\ {\isacharbrackleft}{\isacharbrackright}{\isacharparenright}\ {\isacharequal}\ last\ {\isacharbrackleft}{\isacharbrackright}%
\end{isabelle}
After simplification, it becomes
\begin{isabelle}
\ 1.\ xs\ {\isasymnoteq}\ []\ {\isasymLongrightarrow}\ hd\ []\ =\ last\ []
\end{isabelle}
We cannot prove this equality because we do not know what \isa{hd} and
\isa{last} return when applied to \isa{{\isacharbrackleft}{\isacharbrackright}}.

We should not have ignored the warning. Because the induction
formula is only the conclusion, induction does not affect the occurrence of \isa{xs} in the premises.  
Thus the case that should have been trivial
becomes unprovable. Fortunately, the solution is easy:\footnote{A very similar
heuristic applies to rule inductions; see \S\ref{sec:rtc}.}
\begin{quote}
\emph{Pull all occurrences of the induction variable into the conclusion
using \isa{{\isasymlongrightarrow}}.}
\end{quote}
Thus we should state the lemma as an ordinary 
implication~(\isa{{\isasymlongrightarrow}}), letting
\isa{rule{\isacharunderscore}format} (\S\ref{sec:forward}) convert the
result to the usual \isa{{\isasymLongrightarrow}} form:%
\end{isamarkuptxt}%
\isacommand{lemma}\ hd{\isacharunderscore}rev\ {\isacharbrackleft}rule{\isacharunderscore}format{\isacharbrackright}{\isacharcolon}\ {\isachardoublequote}xs\ {\isasymnoteq}\ {\isacharbrackleft}{\isacharbrackright}\ {\isasymlongrightarrow}\ hd{\isacharparenleft}rev\ xs{\isacharparenright}\ {\isacharequal}\ last\ xs{\isachardoublequote}%
\begin{isamarkuptxt}%
\noindent
This time, induction leaves us with a trivial base case:
\begin{isabelle}%
\ {\isadigit{1}}{\isachardot}\ {\isacharbrackleft}{\isacharbrackright}\ {\isasymnoteq}\ {\isacharbrackleft}{\isacharbrackright}\ {\isasymlongrightarrow}\ hd\ {\isacharparenleft}rev\ {\isacharbrackleft}{\isacharbrackright}{\isacharparenright}\ {\isacharequal}\ last\ {\isacharbrackleft}{\isacharbrackright}%
\end{isabelle}
And \isa{auto} completes the proof.

If there are multiple premises $A@1$, \dots, $A@n$ containing the
induction variable, you should turn the conclusion $C$ into
\[ A@1 \longrightarrow \cdots A@n \longrightarrow C \]
Additionally, you may also have to universally quantify some other variables,
which can yield a fairly complex conclusion.  However, \isa{rule{\isacharunderscore}format} 
can remove any number of occurrences of \isa{{\isasymforall}} and
\isa{{\isasymlongrightarrow}}.%
\end{isamarkuptxt}%
%
\begin{isamarkuptext}%
A second reason why your proposition may not be amenable to induction is that
you want to induct on a whole term, rather than an individual variable. In
general, when inducting on some term $t$ you must rephrase the conclusion $C$
as
\[ \forall y@1 \dots y@n.~ x = t \longrightarrow C \]
where $y@1 \dots y@n$ are the free variables in $t$ and $x$ is new, and
perform induction on $x$ afterwards. An example appears in
\S\ref{sec:complete-ind} below.

The very same problem may occur in connection with rule induction. Remember
that it requires a premise of the form $(x@1,\dots,x@k) \in R$, where $R$ is
some inductively defined set and the $x@i$ are variables.  If instead we have
a premise $t \in R$, where $t$ is not just an $n$-tuple of variables, we
replace it with $(x@1,\dots,x@k) \in R$, and rephrase the conclusion $C$ as
\[ \forall y@1 \dots y@n.~ (x@1,\dots,x@k) = t \longrightarrow C \]
For an example see \S\ref{sec:CTL-revisited} below.

Of course, all premises that share free variables with $t$ need to be pulled into
the conclusion as well, under the \isa{{\isasymforall}}, again using \isa{{\isasymlongrightarrow}} as shown above.%
\end{isamarkuptext}%
%
\isamarkupsubsection{Beyond Structural and Recursion Induction%
}
%
\begin{isamarkuptext}%
\label{sec:complete-ind}
So far, inductive proofs were by structural induction for
primitive recursive functions and recursion induction for total recursive
functions. But sometimes structural induction is awkward and there is no
recursive function that could furnish a more appropriate
induction schema. In such cases a general-purpose induction schema can
be helpful. We show how to apply such induction schemas by an example.

Structural induction on \isa{nat} is
usually known as mathematical induction. There is also \emph{complete}
induction, where you must prove $P(n)$ under the assumption that $P(m)$
holds for all $m<n$. In Isabelle, this is the theorem \isa{nat{\isacharunderscore}less{\isacharunderscore}induct}:
\begin{isabelle}%
\ \ \ \ \ {\isacharparenleft}{\isasymAnd}n{\isachardot}\ {\isasymforall}m{\isachardot}\ m\ {\isacharless}\ n\ {\isasymlongrightarrow}\ P\ m\ {\isasymLongrightarrow}\ P\ n{\isacharparenright}\ {\isasymLongrightarrow}\ P\ n%
\end{isabelle}
Here is an example of its application.%
\end{isamarkuptext}%
\isacommand{consts}\ f\ {\isacharcolon}{\isacharcolon}\ {\isachardoublequote}nat\ {\isasymRightarrow}\ nat{\isachardoublequote}\isanewline
\isacommand{axioms}\ f{\isacharunderscore}ax{\isacharcolon}\ {\isachardoublequote}f{\isacharparenleft}f{\isacharparenleft}n{\isacharparenright}{\isacharparenright}\ {\isacharless}\ f{\isacharparenleft}Suc{\isacharparenleft}n{\isacharparenright}{\isacharparenright}{\isachardoublequote}%
\begin{isamarkuptext}%
\noindent
The axiom for \isa{f} implies \isa{n\ {\isasymle}\ f\ n}, which can
be proved by induction on \mbox{\isa{f\ n}}. Following the recipe outlined
above, we have to phrase the proposition as follows to allow induction:%
\end{isamarkuptext}%
\isacommand{lemma}\ f{\isacharunderscore}incr{\isacharunderscore}lem{\isacharcolon}\ {\isachardoublequote}{\isasymforall}i{\isachardot}\ k\ {\isacharequal}\ f\ i\ {\isasymlongrightarrow}\ i\ {\isasymle}\ f\ i{\isachardoublequote}%
\begin{isamarkuptxt}%
\noindent
To perform induction on \isa{k} using \isa{nat{\isacharunderscore}less{\isacharunderscore}induct}, we use
the same general induction method as for recursion induction (see
\S\ref{sec:recdef-induction}):%
\end{isamarkuptxt}%
\isacommand{apply}{\isacharparenleft}induct{\isacharunderscore}tac\ k\ rule{\isacharcolon}\ nat{\isacharunderscore}less{\isacharunderscore}induct{\isacharparenright}%
\begin{isamarkuptxt}%
\noindent
which leaves us with the following proof state:
\begin{isabelle}%
\ {\isadigit{1}}{\isachardot}\ {\isasymAnd}n{\isachardot}\ {\isasymforall}m{\isachardot}\ m\ {\isacharless}\ n\ {\isasymlongrightarrow}\ {\isacharparenleft}{\isasymforall}i{\isachardot}\ m\ {\isacharequal}\ f\ i\ {\isasymlongrightarrow}\ i\ {\isasymle}\ f\ i{\isacharparenright}\ {\isasymLongrightarrow}\isanewline
\isaindent{\ {\isadigit{1}}{\isachardot}\ {\isasymAnd}n{\isachardot}\ }{\isasymforall}i{\isachardot}\ n\ {\isacharequal}\ f\ i\ {\isasymlongrightarrow}\ i\ {\isasymle}\ f\ i%
\end{isabelle}
After stripping the \isa{{\isasymforall}i}, the proof continues with a case
distinction on \isa{i}. The case \isa{i\ {\isacharequal}\ {\isadigit{0}}} is trivial and we focus on
the other case:%
\end{isamarkuptxt}%
\isacommand{apply}{\isacharparenleft}rule\ allI{\isacharparenright}\isanewline
\isacommand{apply}{\isacharparenleft}case{\isacharunderscore}tac\ i{\isacharparenright}\isanewline
\ \isacommand{apply}{\isacharparenleft}simp{\isacharparenright}%
\begin{isamarkuptxt}%
\begin{isabelle}%
\ {\isadigit{1}}{\isachardot}\ {\isasymAnd}n\ i\ nat{\isachardot}\isanewline
\isaindent{\ {\isadigit{1}}{\isachardot}\ \ \ \ }{\isasymlbrakk}{\isasymforall}m{\isachardot}\ m\ {\isacharless}\ n\ {\isasymlongrightarrow}\ {\isacharparenleft}{\isasymforall}i{\isachardot}\ m\ {\isacharequal}\ f\ i\ {\isasymlongrightarrow}\ i\ {\isasymle}\ f\ i{\isacharparenright}{\isacharsemicolon}\ i\ {\isacharequal}\ Suc\ nat{\isasymrbrakk}\isanewline
\isaindent{\ {\isadigit{1}}{\isachardot}\ \ \ \ }{\isasymLongrightarrow}\ n\ {\isacharequal}\ f\ i\ {\isasymlongrightarrow}\ i\ {\isasymle}\ f\ i%
\end{isabelle}%
\end{isamarkuptxt}%
\isacommand{by}{\isacharparenleft}blast\ intro{\isacharbang}{\isacharcolon}\ f{\isacharunderscore}ax\ Suc{\isacharunderscore}leI\ intro{\isacharcolon}\ le{\isacharunderscore}less{\isacharunderscore}trans{\isacharparenright}%
\begin{isamarkuptext}%
\noindent
If you find the last step puzzling, here are the two lemmas it employs:
\begin{isabelle}
\isa{m\ {\isacharless}\ n\ {\isasymLongrightarrow}\ Suc\ m\ {\isasymle}\ n}
\rulename{Suc_leI}\isanewline
\isa{{\isasymlbrakk}i\ {\isasymle}\ j{\isacharsemicolon}\ j\ {\isacharless}\ k{\isasymrbrakk}\ {\isasymLongrightarrow}\ i\ {\isacharless}\ k}
\rulename{le_less_trans}
\end{isabelle}
%
The proof goes like this (writing \isa{j} instead of \isa{nat}).
Since \isa{i\ {\isacharequal}\ Suc\ j} it suffices to show
\hbox{\isa{j\ {\isacharless}\ f\ {\isacharparenleft}Suc\ j{\isacharparenright}}},
by \isa{Suc{\isacharunderscore}leI}\@.  This is
proved as follows. From \isa{f{\isacharunderscore}ax} we have \isa{f\ {\isacharparenleft}f\ j{\isacharparenright}\ {\isacharless}\ f\ {\isacharparenleft}Suc\ j{\isacharparenright}}
(1) which implies \isa{f\ j\ {\isasymle}\ f\ {\isacharparenleft}f\ j{\isacharparenright}} by the induction hypothesis.
Using (1) once more we obtain \isa{f\ j\ {\isacharless}\ f\ {\isacharparenleft}Suc\ j{\isacharparenright}} (2) by the transitivity
rule \isa{le{\isacharunderscore}less{\isacharunderscore}trans}.
Using the induction hypothesis once more we obtain \isa{j\ {\isasymle}\ f\ j}
which, together with (2) yields \isa{j\ {\isacharless}\ f\ {\isacharparenleft}Suc\ j{\isacharparenright}} (again by
\isa{le{\isacharunderscore}less{\isacharunderscore}trans}).

This last step shows both the power and the danger of automatic proofs: they
will usually not tell you how the proof goes, because it can be very hard to
translate the internal proof into a human-readable format. Therefore
Chapter~\ref{sec:part2?} introduces a language for writing readable
proofs.

We can now derive the desired \isa{i\ {\isasymle}\ f\ i} from \isa{f{\isacharunderscore}incr{\isacharunderscore}lem}:%
\end{isamarkuptext}%
\isacommand{lemmas}\ f{\isacharunderscore}incr\ {\isacharequal}\ f{\isacharunderscore}incr{\isacharunderscore}lem{\isacharbrackleft}rule{\isacharunderscore}format{\isacharcomma}\ OF\ refl{\isacharbrackright}%
\begin{isamarkuptext}%
\noindent
The final \isa{refl} gets rid of the premise \isa{{\isacharquery}k\ {\isacharequal}\ f\ {\isacharquery}i}. 
We could have included this derivation in the original statement of the lemma:%
\end{isamarkuptext}%
\isacommand{lemma}\ f{\isacharunderscore}incr{\isacharbrackleft}rule{\isacharunderscore}format{\isacharcomma}\ OF\ refl{\isacharbrackright}{\isacharcolon}\ {\isachardoublequote}{\isasymforall}i{\isachardot}\ k\ {\isacharequal}\ f\ i\ {\isasymlongrightarrow}\ i\ {\isasymle}\ f\ i{\isachardoublequote}%
\begin{isamarkuptext}%
\begin{warn}
We discourage the use of axioms because of the danger of
inconsistencies.  Axiom \isa{f{\isacharunderscore}ax} does
not introduce an inconsistency because, for example, the identity function
satisfies it.  Axioms can be useful in exploratory developments, say when 
you assume some well-known theorems so that you can quickly demonstrate some
point about methodology.  If your example turns into a substantial proof
development, you should replace the axioms by proofs.
\end{warn}

\bigskip
In general, \isa{induct{\isacharunderscore}tac} can be applied with any rule $r$
whose conclusion is of the form ${?}P~?x@1 \dots ?x@n$, in which case the
format is
\begin{quote}
\isacommand{apply}\isa{{\isacharparenleft}induct{\isacharunderscore}tac} $y@1 \dots y@n$ \isa{rule{\isacharcolon}} $r$\isa{{\isacharparenright}}
\end{quote}\index{*induct_tac}%
where $y@1, \dots, y@n$ are variables in the first subgoal.
A further example of a useful induction rule is \isa{length{\isacharunderscore}induct},
induction on the length of a list:\indexbold{*length_induct}
\begin{isabelle}%
\ \ \ \ \ {\isacharparenleft}{\isasymAnd}xs{\isachardot}\ {\isasymforall}ys{\isachardot}\ length\ ys\ {\isacharless}\ length\ xs\ {\isasymlongrightarrow}\ P\ ys\ {\isasymLongrightarrow}\ P\ xs{\isacharparenright}\ {\isasymLongrightarrow}\ P\ xs%
\end{isabelle}

In fact, \isa{induct{\isacharunderscore}tac} even allows the conclusion of
$r$ to be an (iterated) conjunction of formulae of the above form, in
which case the application is
\begin{quote}
\isacommand{apply}\isa{{\isacharparenleft}induct{\isacharunderscore}tac} $y@1 \dots y@n$ \isa{and} \dots\ \isa{and} $z@1 \dots z@m$ \isa{rule{\isacharcolon}} $r$\isa{{\isacharparenright}}
\end{quote}

\begin{exercise}
From the axiom and lemma for \isa{f}, show that \isa{f} is the
identity function.
\end{exercise}%
\end{isamarkuptext}%
%
\isamarkupsubsection{Derivation of New Induction Schemas%
}
%
\begin{isamarkuptext}%
\label{sec:derive-ind}
Induction schemas are ordinary theorems and you can derive new ones
whenever you wish.  This section shows you how to, using the example
of \isa{nat{\isacharunderscore}less{\isacharunderscore}induct}. Assume we only have structural induction
available for \isa{nat} and want to derive complete induction. This
requires us to generalize the statement first:%
\end{isamarkuptext}%
\isacommand{lemma}\ induct{\isacharunderscore}lem{\isacharcolon}\ {\isachardoublequote}{\isacharparenleft}{\isasymAnd}n{\isacharcolon}{\isacharcolon}nat{\isachardot}\ {\isasymforall}m{\isacharless}n{\isachardot}\ P\ m\ {\isasymLongrightarrow}\ P\ n{\isacharparenright}\ {\isasymLongrightarrow}\ {\isasymforall}m{\isacharless}n{\isachardot}\ P\ m{\isachardoublequote}\isanewline
\isacommand{apply}{\isacharparenleft}induct{\isacharunderscore}tac\ n{\isacharparenright}%
\begin{isamarkuptxt}%
\noindent
The base case is vacuously true. For the induction step (\isa{m\ {\isacharless}\ Suc\ n}) we distinguish two cases: case \isa{m\ {\isacharless}\ n} is true by induction
hypothesis and case \isa{m\ {\isacharequal}\ n} follows from the assumption, again using
the induction hypothesis:%
\end{isamarkuptxt}%
\ \isacommand{apply}{\isacharparenleft}blast{\isacharparenright}\isanewline
\isacommand{by}{\isacharparenleft}blast\ elim{\isacharcolon}less{\isacharunderscore}SucE{\isacharparenright}%
\begin{isamarkuptext}%
\noindent
The elimination rule \isa{less{\isacharunderscore}SucE} expresses the case distinction:
\begin{isabelle}%
\ \ \ \ \ {\isasymlbrakk}m\ {\isacharless}\ Suc\ n{\isacharsemicolon}\ m\ {\isacharless}\ n\ {\isasymLongrightarrow}\ P{\isacharsemicolon}\ m\ {\isacharequal}\ n\ {\isasymLongrightarrow}\ P{\isasymrbrakk}\ {\isasymLongrightarrow}\ P%
\end{isabelle}

Now it is straightforward to derive the original version of
\isa{nat{\isacharunderscore}less{\isacharunderscore}induct} by manipulting the conclusion of the above lemma:
instantiate \isa{n} by \isa{Suc\ n} and \isa{m} by \isa{n} and
remove the trivial condition \isa{n\ {\isacharless}\ Suc\ n}. Fortunately, this
happens automatically when we add the lemma as a new premise to the
desired goal:%
\end{isamarkuptext}%
\isacommand{theorem}\ nat{\isacharunderscore}less{\isacharunderscore}induct{\isacharcolon}\ {\isachardoublequote}{\isacharparenleft}{\isasymAnd}n{\isacharcolon}{\isacharcolon}nat{\isachardot}\ {\isasymforall}m{\isacharless}n{\isachardot}\ P\ m\ {\isasymLongrightarrow}\ P\ n{\isacharparenright}\ {\isasymLongrightarrow}\ P\ n{\isachardoublequote}\isanewline
\isacommand{by}{\isacharparenleft}insert\ induct{\isacharunderscore}lem{\isacharcomma}\ blast{\isacharparenright}%
\begin{isamarkuptext}%
Finally we should remind the reader that HOL already provides the mother of
all inductions, well-founded induction (see \S\ref{sec:Well-founded}).  For
example theorem \isa{nat{\isacharunderscore}less{\isacharunderscore}induct} is
a special case of \isa{wf{\isacharunderscore}induct} where \isa{r} is \isa{{\isacharless}} on
\isa{nat}. The details can be found in theory \isa{Wellfounded_Recursion}.%
\end{isamarkuptext}%
\end{isabellebody}%
%%% Local Variables:
%%% mode: latex
%%% TeX-master: "root"
%%% End:

%
\begin{isabellebody}%
\def\isabellecontext{CTLind}%
%
\isamarkupsubsection{CTL revisited%
}
%
\begin{isamarkuptext}%
\label{sec:CTL-revisited}
The purpose of this section is twofold: we want to demonstrate
some of the induction principles and heuristics discussed above and we want to
show how inductive definitions can simplify proofs.
In \S\ref{sec:CTL} we gave a fairly involved proof of the correctness of a
model checker for CTL\@. In particular the proof of the
\isa{infinity{\isacharunderscore}lemma} on the way to \isa{AF{\isacharunderscore}lemma{\isadigit{2}}} is not as
simple as one might intuitively expect, due to the \isa{SOME} operator
involved. Below we give a simpler proof of \isa{AF{\isacharunderscore}lemma{\isadigit{2}}}
based on an auxiliary inductive definition.

Let us call a (finite or infinite) path \emph{\isa{A}-avoiding} if it does
not touch any node in the set \isa{A}. Then \isa{AF{\isacharunderscore}lemma{\isadigit{2}}} says
that if no infinite path from some state \isa{s} is \isa{A}-avoiding,
then \isa{s\ {\isasymin}\ lfp\ {\isacharparenleft}af\ A{\isacharparenright}}. We prove this by inductively defining the set
\isa{Avoid\ s\ A} of states reachable from \isa{s} by a finite \isa{A}-avoiding path:
% Second proof of opposite direction, directly by well-founded induction
% on the initial segment of M that avoids A.%
\end{isamarkuptext}%
\isacommand{consts}\ Avoid\ {\isacharcolon}{\isacharcolon}\ {\isachardoublequote}state\ {\isasymRightarrow}\ state\ set\ {\isasymRightarrow}\ state\ set{\isachardoublequote}\isanewline
\isacommand{inductive}\ {\isachardoublequote}Avoid\ s\ A{\isachardoublequote}\isanewline
\isakeyword{intros}\ {\isachardoublequote}s\ {\isasymin}\ Avoid\ s\ A{\isachardoublequote}\isanewline
\ \ \ \ \ \ \ {\isachardoublequote}{\isasymlbrakk}\ t\ {\isasymin}\ Avoid\ s\ A{\isacharsemicolon}\ t\ {\isasymnotin}\ A{\isacharsemicolon}\ {\isacharparenleft}t{\isacharcomma}u{\isacharparenright}\ {\isasymin}\ M\ {\isasymrbrakk}\ {\isasymLongrightarrow}\ u\ {\isasymin}\ Avoid\ s\ A{\isachardoublequote}%
\begin{isamarkuptext}%
It is easy to see that for any infinite \isa{A}-avoiding path \isa{f}
with \isa{f\ {\isadigit{0}}\ {\isasymin}\ Avoid\ s\ A} there is an infinite \isa{A}-avoiding path
starting with \isa{s} because (by definition of \isa{Avoid}) there is a
finite \isa{A}-avoiding path from \isa{s} to \isa{f\ {\isadigit{0}}}.
The proof is by induction on \isa{f\ {\isadigit{0}}\ {\isasymin}\ Avoid\ s\ A}. However,
this requires the following
reformulation, as explained in \S\ref{sec:ind-var-in-prems} above;
the \isa{rule{\isacharunderscore}format} directive undoes the reformulation after the proof.%
\end{isamarkuptext}%
\isacommand{lemma}\ ex{\isacharunderscore}infinite{\isacharunderscore}path{\isacharbrackleft}rule{\isacharunderscore}format{\isacharbrackright}{\isacharcolon}\isanewline
\ \ {\isachardoublequote}t\ {\isasymin}\ Avoid\ s\ A\ \ {\isasymLongrightarrow}\isanewline
\ \ \ {\isasymforall}f{\isasymin}Paths\ t{\isachardot}\ {\isacharparenleft}{\isasymforall}i{\isachardot}\ f\ i\ {\isasymnotin}\ A{\isacharparenright}\ {\isasymlongrightarrow}\ {\isacharparenleft}{\isasymexists}p{\isasymin}Paths\ s{\isachardot}\ {\isasymforall}i{\isachardot}\ p\ i\ {\isasymnotin}\ A{\isacharparenright}{\isachardoublequote}\isanewline
\isacommand{apply}{\isacharparenleft}erule\ Avoid{\isachardot}induct{\isacharparenright}\isanewline
\ \isacommand{apply}{\isacharparenleft}blast{\isacharparenright}\isanewline
\isacommand{apply}{\isacharparenleft}clarify{\isacharparenright}\isanewline
\isacommand{apply}{\isacharparenleft}drule{\isacharunderscore}tac\ x\ {\isacharequal}\ {\isachardoublequote}{\isasymlambda}i{\isachardot}\ case\ i\ of\ {\isadigit{0}}\ {\isasymRightarrow}\ t\ {\isacharbar}\ Suc\ i\ {\isasymRightarrow}\ f\ i{\isachardoublequote}\ \isakeyword{in}\ bspec{\isacharparenright}\isanewline
\isacommand{apply}{\isacharparenleft}simp{\isacharunderscore}all\ add{\isacharcolon}Paths{\isacharunderscore}def\ split{\isacharcolon}nat{\isachardot}split{\isacharparenright}\isanewline
\isacommand{done}%
\begin{isamarkuptext}%
\noindent
The base case (\isa{t\ {\isacharequal}\ s}) is trivial (\isa{blast}).
In the induction step, we have an infinite \isa{A}-avoiding path \isa{f}
starting from \isa{u}, a successor of \isa{t}. Now we simply instantiate
the \isa{{\isasymforall}f{\isasymin}Paths\ t} in the induction hypothesis by the path starting with
\isa{t} and continuing with \isa{f}. That is what the above $\lambda$-term
expresses. That fact that this is a path starting with \isa{t} and that
the instantiated induction hypothesis implies the conclusion is shown by
simplification.

Now we come to the key lemma. It says that if \isa{t} can be reached by a
finite \isa{A}-avoiding path from \isa{s}, then \isa{t\ {\isasymin}\ lfp\ {\isacharparenleft}af\ A{\isacharparenright}},
provided there is no infinite \isa{A}-avoiding path starting from \isa{s}.%
\end{isamarkuptext}%
\isacommand{lemma}\ Avoid{\isacharunderscore}in{\isacharunderscore}lfp{\isacharbrackleft}rule{\isacharunderscore}format{\isacharparenleft}no{\isacharunderscore}asm{\isacharparenright}{\isacharbrackright}{\isacharcolon}\isanewline
\ \ {\isachardoublequote}{\isasymforall}p{\isasymin}Paths\ s{\isachardot}\ {\isasymexists}i{\isachardot}\ p\ i\ {\isasymin}\ A\ {\isasymLongrightarrow}\ t\ {\isasymin}\ Avoid\ s\ A\ {\isasymlongrightarrow}\ t\ {\isasymin}\ lfp{\isacharparenleft}af\ A{\isacharparenright}{\isachardoublequote}%
\begin{isamarkuptxt}%
\noindent
The trick is not to induct on \isa{t\ {\isasymin}\ Avoid\ s\ A}, as already the base
case would be a problem, but to proceed by well-founded induction \isa{t}. Hence \isa{t\ {\isasymin}\ Avoid\ s\ A} needs to be brought into the conclusion as
well, which the directive \isa{rule{\isacharunderscore}format} undoes at the end (see below).
But induction with respect to which well-founded relation? The restriction
of \isa{M} to \isa{Avoid\ s\ A}:
\begin{isabelle}%
\ \ \ \ \ {\isacharbraceleft}{\isacharparenleft}y{\isacharcomma}\ x{\isacharparenright}{\isachardot}\ {\isacharparenleft}x{\isacharcomma}\ y{\isacharparenright}\ {\isasymin}\ M\ {\isasymand}\ x\ {\isasymin}\ Avoid\ s\ A\ {\isasymand}\ y\ {\isasymin}\ Avoid\ s\ A\ {\isasymand}\ x\ {\isasymnotin}\ A{\isacharbraceright}%
\end{isabelle}
As we shall see in a moment, the absence of infinite \isa{A}-avoiding paths
starting from \isa{s} implies well-foundedness of this relation. For the
moment we assume this and proceed with the induction:%
\end{isamarkuptxt}%
\isacommand{apply}{\isacharparenleft}subgoal{\isacharunderscore}tac\isanewline
\ \ {\isachardoublequote}wf{\isacharbraceleft}{\isacharparenleft}y{\isacharcomma}x{\isacharparenright}{\isachardot}\ {\isacharparenleft}x{\isacharcomma}y{\isacharparenright}{\isasymin}M\ {\isasymand}\ x\ {\isasymin}\ Avoid\ s\ A\ {\isasymand}\ y\ {\isasymin}\ Avoid\ s\ A\ {\isasymand}\ x\ {\isasymnotin}\ A{\isacharbraceright}{\isachardoublequote}{\isacharparenright}\isanewline
\ \isacommand{apply}{\isacharparenleft}erule{\isacharunderscore}tac\ a\ {\isacharequal}\ t\ \isakeyword{in}\ wf{\isacharunderscore}induct{\isacharparenright}\isanewline
\ \isacommand{apply}{\isacharparenleft}clarsimp{\isacharparenright}%
\begin{isamarkuptxt}%
\noindent
Now can assume additionally (induction hypothesis) that if \isa{t\ {\isasymnotin}\ A}
then all successors of \isa{t} that are in \isa{Avoid\ s\ A} are in
\isa{lfp\ {\isacharparenleft}af\ A{\isacharparenright}}. To prove the actual goal we unfold \isa{lfp} once. Now
we have to prove that \isa{t} is in \isa{A} or all successors of \isa{t} are in \isa{lfp\ {\isacharparenleft}af\ A{\isacharparenright}}. If \isa{t} is not in \isa{A}, the second
\isa{Avoid}-rule implies that all successors of \isa{t} are in
\isa{Avoid\ s\ A} (because we also assume \isa{t\ {\isasymin}\ Avoid\ s\ A}), and
hence, by the induction hypothesis, all successors of \isa{t} are indeed in
\isa{lfp\ {\isacharparenleft}af\ A{\isacharparenright}}. Mechanically:%
\end{isamarkuptxt}%
\ \isacommand{apply}{\isacharparenleft}rule\ ssubst\ {\isacharbrackleft}OF\ lfp{\isacharunderscore}unfold{\isacharbrackleft}OF\ mono{\isacharunderscore}af{\isacharbrackright}{\isacharbrackright}{\isacharparenright}\isanewline
\ \isacommand{apply}{\isacharparenleft}simp\ only{\isacharcolon}\ af{\isacharunderscore}def{\isacharparenright}\isanewline
\ \isacommand{apply}{\isacharparenleft}blast\ intro{\isacharcolon}Avoid{\isachardot}intros{\isacharparenright}%
\begin{isamarkuptxt}%
Having proved the main goal we return to the proof obligation that the above
relation is indeed well-founded. This is proved by contraposition: we assume
the relation is not well-founded. Thus there exists an infinite \isa{A}-avoiding path all in \isa{Avoid\ s\ A}, by theorem
\isa{wf{\isacharunderscore}iff{\isacharunderscore}no{\isacharunderscore}infinite{\isacharunderscore}down{\isacharunderscore}chain}:
\begin{isabelle}%
\ \ \ \ \ wf\ r\ {\isacharequal}\ {\isacharparenleft}{\isasymnot}\ {\isacharparenleft}{\isasymexists}f{\isachardot}\ {\isasymforall}i{\isachardot}\ {\isacharparenleft}f\ {\isacharparenleft}Suc\ i{\isacharparenright}{\isacharcomma}\ f\ i{\isacharparenright}\ {\isasymin}\ r{\isacharparenright}{\isacharparenright}%
\end{isabelle}
From lemma \isa{ex{\isacharunderscore}infinite{\isacharunderscore}path} the existence of an infinite
\isa{A}-avoiding path starting in \isa{s} follows, just as required for
the contraposition.%
\end{isamarkuptxt}%
\isacommand{apply}{\isacharparenleft}erule\ contrapos{\isacharunderscore}pp{\isacharparenright}\isanewline
\isacommand{apply}{\isacharparenleft}simp\ add{\isacharcolon}wf{\isacharunderscore}iff{\isacharunderscore}no{\isacharunderscore}infinite{\isacharunderscore}down{\isacharunderscore}chain{\isacharparenright}\isanewline
\isacommand{apply}{\isacharparenleft}erule\ exE{\isacharparenright}\isanewline
\isacommand{apply}{\isacharparenleft}rule\ ex{\isacharunderscore}infinite{\isacharunderscore}path{\isacharparenright}\isanewline
\isacommand{apply}{\isacharparenleft}auto\ simp\ add{\isacharcolon}Paths{\isacharunderscore}def{\isacharparenright}\isanewline
\isacommand{done}%
\begin{isamarkuptext}%
The \isa{{\isacharparenleft}no{\isacharunderscore}asm{\isacharparenright}} modifier of the \isa{rule{\isacharunderscore}format} directive means
that the assumption is left unchanged---otherwise the \isa{{\isasymforall}p} is turned
into a \isa{{\isasymAnd}p}, which would complicate matters below. As it is,
\isa{Avoid{\isacharunderscore}in{\isacharunderscore}lfp} is now
\begin{isabelle}%
\ \ \ \ \ {\isasymlbrakk}{\isasymforall}p{\isasymin}Paths\ s{\isachardot}\ {\isasymexists}i{\isachardot}\ p\ i\ {\isasymin}\ A{\isacharsemicolon}\ t\ {\isasymin}\ Avoid\ s\ A{\isasymrbrakk}\ {\isasymLongrightarrow}\ t\ {\isasymin}\ lfp\ {\isacharparenleft}af\ A{\isacharparenright}%
\end{isabelle}
The main theorem is simply the corollary where \isa{t\ {\isacharequal}\ s},
in which case the assumption \isa{t\ {\isasymin}\ Avoid\ s\ A} is trivially true
by the first \isa{Avoid}-rule). Isabelle confirms this:%
\end{isamarkuptext}%
\isacommand{theorem}\ AF{\isacharunderscore}lemma{\isadigit{2}}{\isacharcolon}\isanewline
\ \ {\isachardoublequote}{\isacharbraceleft}s{\isachardot}\ {\isasymforall}p\ {\isasymin}\ Paths\ s{\isachardot}\ {\isasymexists}\ i{\isachardot}\ p\ i\ {\isasymin}\ A{\isacharbraceright}\ {\isasymsubseteq}\ lfp{\isacharparenleft}af\ A{\isacharparenright}{\isachardoublequote}\isanewline
\isacommand{by}{\isacharparenleft}auto\ elim{\isacharcolon}Avoid{\isacharunderscore}in{\isacharunderscore}lfp\ intro{\isacharcolon}Avoid{\isachardot}intros{\isacharparenright}\isanewline
\isanewline
\end{isabellebody}%
%%% Local Variables:
%%% mode: latex
%%% TeX-master: "root"
%%% End:

\index{induction|)}


\bibliographystyle{plain} \small\raggedright\frenchspacing
\bibliography{string,atp,funprog,general,logicprog,isabelle,theory,crossref}


\chapter{Introduction}

\section{Quick start}

Isar is already part of Isabelle (as of version Isabelle99, or later).  The
\texttt{isabelle} binary provides option \texttt{-I} to run the Isar
interaction loop at startup, rather than the plain ML top-level.  Thus the
quickest way to do anything with Isabelle/Isar is as follows:
\begin{ttbox}
isabelle -I HOL\medskip
\out{> Welcome to Isabelle/HOL (Isabelle99)}\medskip
theory Foo = Main:
constdefs foo :: nat  "foo == 1";
lemma "0 < foo" by (simp add: foo_def);
end
\end{ttbox}
Note that any Isabelle/Isar command may be retracted by \texttt{undo}; the
\texttt{help} command prints a list of available language elements.

Plain TTY-based interaction like this used to be quite feasible with
traditional tactic based theorem proving, but developing Isar documents
demands some better user-interface support.  \emph{Proof~General}\index{Proof
  General} of LFCS Edinburgh \cite{proofgeneral} offers a generic Emacs-based
environment for interactive theorem provers that does all the cut-and-paste
and forward-backward walk through the text in a very neat way.  Note that in
Isabelle/Isar, the current position within a partial proof document is equally
important than the actual proof state.  Thus Proof~General provides the
canonical working environment for Isabelle/Isar, both for getting acquainted
(e.g.\ by replaying existing Isar documents) and real production work.

\medskip

The easiest way to use Proof~General is to make it the default Isabelle user
interface.  Just put something like this into your Isabelle settings file (see
also \cite{isabelle-sys}):
\begin{ttbox}
ISABELLE_INTERFACE=\$ISABELLE_HOME/contrib/ProofGeneral/isar/interface
PROOFGENERAL_OPTIONS="-u false"
\end{ttbox}
You may have to change \texttt{\$ISABELLE_HOME/contrib/ProofGeneral} to the
actual installation directory of Proof~General.  From now on, the capital
\texttt{Isabelle} executable refers to the \texttt{ProofGeneral/isar}
interface.\footnote{There is also a \texttt{ProofGeneral/isa} interface, for
  classic Isabelle tactic scripts.}  Its usage is as follows:
\begin{ttbox}
Usage: interface [OPTIONS] [FILES ...]

  Options are:
    -l NAME      logic image name (default $ISABELLE_LOGIC=HOL)
    -p NAME      Emacs program name (default xemacs)
    -u BOOL      use .emacs file (default true)
    -w BOOL      use window system (default true)

  Starts Proof General for Isabelle/Isar with proof documents FILES
  (default Scratch.thy).

  PROOFGENERAL_OPTIONS=
\end{ttbox} %$
Apart from the command line, the defaults for these options may be overridden
via the \texttt{PROOFGENERAL_OPTIONS} setting as well.  For example, plain GNU
Emacs may be configured as follows:
\begin{ttbox}
PROOFGENERAL_OPTIONS="-u false -p emacs"
\end{ttbox}

Occasionally, a user's \texttt{.emacs} file contains material that is
incompatible with the version of (X)Emacs that Proof~General prefers.  Then
proper startup may be still achieved by using the \texttt{-u false}
option.\footnote{Any Emacs lisp file \texttt{proofgeneral-settings.el}
  occurring in \texttt{\$ISABELLE_HOME/etc} or
  \texttt{\$ISABELLE_HOME_USER/etc} is automatically loaded by the
  Proof~General interface script as well.}

\medskip

With the proper Isabelle interface setup, Isar documents may now be edited by
visiting appropriate theory files, e.g.\ 
\begin{ttbox}
Isabelle \({\langle}isabellehome{\rangle}\)/src/HOL/Isar_examples/BasicLogic.thy
\end{ttbox}
Users of XEmacs may note the tool bar for navigating forward and backward
through the text.  Consult the Proof~General documentation \cite{proofgeneral}
for further basic command sequences, such as ``\texttt{c-c return}'' or
``\texttt{c-c u}''.


\section{Isabelle/Isar theories}

Isabelle/Isar offers two main improvements over classic Isabelle:
\begin{enumerate}
\item A new \emph{theory format}, occasionally referred to as ``new-style
  theories'', supporting interactive development and unlimited undo operation.
\item A \emph{formal proof document language} designed to support intelligible
  semi-automated reasoning.  Instead of putting together unreadable tactic
  scripts, the author is enabled to express the reasoning in way that is close
  to mathematical practice.
\end{enumerate}

The Isar proof language is embedded into the new theory format as a proper
sub-language.  Proof mode is entered by stating some $\THEOREMNAME$ or
$\LEMMANAME$ at the theory level, and left again with the final conclusion
(e.g.\ via $\QEDNAME$).  A few theory extension mechanisms require proof as
well, such as the HOL $\isarkeyword{typedef}$ which demands non-emptiness of
the representing sets.

New-style theory files may still be associated with separate ML files
consisting of plain old tactic scripts.  There is no longer any ML binding
generated for the theory and theorems, though.  ML functions \texttt{theory},
\texttt{thm}, and \texttt{thms} retrieve this information \cite{isabelle-ref}.
Nevertheless, migration between classic Isabelle and Isabelle/Isar is
relatively easy.  Thus users may start to benefit from interactive theory
development even before they have any idea of the Isar proof language at all.

\begin{warn}
  Currently Proof~General does \emph{not} support mixed interactive
  development of classic Isabelle theory files or tactic scripts, together
  with Isar documents.  The ``\texttt{isa}'' and ``\texttt{isar}'' versions of
  Proof~General are handled as two different theorem proving systems, only one
  of these may be active at the same time.
\end{warn}

Porting of existing tactic scripts is best done by running two separate
Proof~General sessions, one for replaying the old script and the other for the
emerging Isabelle/Isar document.


\section{How to write Isar proofs anyway?}

This is one of the key questions, of course.  Isar offers a rather different
approach to formal proof documents than plain old tactic scripts.  Experienced
users of existing interactive theorem proving systems may have to learn
thinking differently in order to make effective use of Isabelle/Isar.  On the
other hand, Isabelle/Isar comes much closer to existing mathematical practice
of formal proof, so users with less experience in old-style tactical proving,
but a good understanding of mathematical proof, might cope with Isar even
better.  See also \cite{Wenzel:1999:TPHOL} for further background information
on Isar.

\medskip This really is a \emph{reference manual}.  Nevertheless, we will also
give some clues of how the concepts introduced here may be put into practice.
Appendix~\ref{ap:refcard} provides a quick reference card of the most common
Isabelle/Isar language elements.  There are several examples distributed with
Isabelle, and available via the Isabelle WWW library:
\begin{center}\small
  \begin{tabular}{l}
    \url{http://www.cl.cam.ac.uk/Research/HVG/Isabelle/library/} \\
    \url{http://isabelle.in.tum.de/library/} \\
  \end{tabular}
\end{center}

See \texttt{HOL/Isar_examples} for a collection of introductory examples, and
\texttt{HOL/HOL-Real/HahnBanach} is a big mathematics application.  Apart from
browsable HTML sources, both sessions provide actual documents (in PDF).

%%% Local Variables: 
%%% mode: latex
%%% TeX-master: "isar-ref"
%%% End: 

\end{document}
