
%% $Id$

\documentclass[12pt,a4paper,fleqn]{report}
\usepackage{latexsym,graphicx}
\usepackage[refpage]{nomencl}
\usepackage{../iman,../extra,../isar,../proof}
\usepackage{Thy/document/isabelle,Thy/document/isabellesym}
\usepackage{style}
\usepackage{../pdfsetup}


\hyphenation{Isabelle}
\hyphenation{Isar}

\isadroptag{theory}
\title{\includegraphics[scale=0.5]{isabelle_isar}
  \\[4ex] The Isabelle/Isar Implementation}
\author{\emph{Makarius M. M. Wenzel}}

\makeglossary
\makeindex


\begin{document}

\maketitle 

\begin{abstract}
  We describe the key concepts underlying the Isabelle/Isar
  implementation, including ML references for the most important
  elements.  The aim is to give some insight into the overall system
  architecture, and provide clues on implementing user extensions.
\end{abstract}

\pagenumbering{roman} \tableofcontents \clearfirst


\chapter{Introduction}

\section{Quick start}

Isar is already part of Isabelle (as of version Isabelle99, or later).  The
\texttt{isabelle} binary provides option \texttt{-I} to run the Isar
interaction loop at startup, rather than the plain ML top-level.  Thus the
quickest way to do anything with Isabelle/Isar is as follows:
\begin{ttbox}
isabelle -I HOL\medskip
\out{> Welcome to Isabelle/HOL (Isabelle99)}\medskip
theory Foo = Main:
constdefs foo :: nat  "foo == 1";
lemma "0 < foo" by (simp add: foo_def);
end
\end{ttbox}
Note that any Isabelle/Isar command may be retracted by \texttt{undo}; the
\texttt{help} command prints a list of available language elements.

Plain TTY-based interaction like this used to be quite feasible with
traditional tactic based theorem proving, but developing Isar documents
demands some better user-interface support.  \emph{Proof~General}\index{Proof
  General} of LFCS Edinburgh \cite{proofgeneral} offers a generic Emacs-based
environment for interactive theorem provers that does all the cut-and-paste
and forward-backward walk through the text in a very neat way.  Note that in
Isabelle/Isar, the current position within a partial proof document is equally
important than the actual proof state.  Thus Proof~General provides the
canonical working environment for Isabelle/Isar, both for getting acquainted
(e.g.\ by replaying existing Isar documents) and real production work.

\medskip

The easiest way to use Proof~General is to make it the default Isabelle user
interface.  Just put something like this into your Isabelle settings file (see
also \cite{isabelle-sys}):
\begin{ttbox}
ISABELLE_INTERFACE=\$ISABELLE_HOME/contrib/ProofGeneral/isar/interface
PROOFGENERAL_OPTIONS="-u false"
\end{ttbox}
You may have to change \texttt{\$ISABELLE_HOME/contrib/ProofGeneral} to the
actual installation directory of Proof~General.  From now on, the capital
\texttt{Isabelle} executable refers to the \texttt{ProofGeneral/isar}
interface.\footnote{There is also a \texttt{ProofGeneral/isa} interface, for
  classic Isabelle tactic scripts.}  Its usage is as follows:
\begin{ttbox}
Usage: interface [OPTIONS] [FILES ...]

  Options are:
    -l NAME      logic image name (default $ISABELLE_LOGIC=HOL)
    -p NAME      Emacs program name (default xemacs)
    -u BOOL      use .emacs file (default true)
    -w BOOL      use window system (default true)

  Starts Proof General for Isabelle/Isar with proof documents FILES
  (default Scratch.thy).

  PROOFGENERAL_OPTIONS=
\end{ttbox} %$
Apart from the command line, the defaults for these options may be overridden
via the \texttt{PROOFGENERAL_OPTIONS} setting as well.  For example, plain GNU
Emacs may be configured as follows:
\begin{ttbox}
PROOFGENERAL_OPTIONS="-u false -p emacs"
\end{ttbox}

Occasionally, a user's \texttt{.emacs} file contains material that is
incompatible with the version of (X)Emacs that Proof~General prefers.  Then
proper startup may be still achieved by using the \texttt{-u false}
option.\footnote{Any Emacs lisp file \texttt{proofgeneral-settings.el}
  occurring in \texttt{\$ISABELLE_HOME/etc} or
  \texttt{\$ISABELLE_HOME_USER/etc} is automatically loaded by the
  Proof~General interface script as well.}

\medskip

With the proper Isabelle interface setup, Isar documents may now be edited by
visiting appropriate theory files, e.g.\ 
\begin{ttbox}
Isabelle \({\langle}isabellehome{\rangle}\)/src/HOL/Isar_examples/BasicLogic.thy
\end{ttbox}
Users of XEmacs may note the tool bar for navigating forward and backward
through the text.  Consult the Proof~General documentation \cite{proofgeneral}
for further basic command sequences, such as ``\texttt{c-c return}'' or
``\texttt{c-c u}''.


\section{Isabelle/Isar theories}

Isabelle/Isar offers two main improvements over classic Isabelle:
\begin{enumerate}
\item A new \emph{theory format}, occasionally referred to as ``new-style
  theories'', supporting interactive development and unlimited undo operation.
\item A \emph{formal proof document language} designed to support intelligible
  semi-automated reasoning.  Instead of putting together unreadable tactic
  scripts, the author is enabled to express the reasoning in way that is close
  to mathematical practice.
\end{enumerate}

The Isar proof language is embedded into the new theory format as a proper
sub-language.  Proof mode is entered by stating some $\THEOREMNAME$ or
$\LEMMANAME$ at the theory level, and left again with the final conclusion
(e.g.\ via $\QEDNAME$).  A few theory extension mechanisms require proof as
well, such as the HOL $\isarkeyword{typedef}$ which demands non-emptiness of
the representing sets.

New-style theory files may still be associated with separate ML files
consisting of plain old tactic scripts.  There is no longer any ML binding
generated for the theory and theorems, though.  ML functions \texttt{theory},
\texttt{thm}, and \texttt{thms} retrieve this information \cite{isabelle-ref}.
Nevertheless, migration between classic Isabelle and Isabelle/Isar is
relatively easy.  Thus users may start to benefit from interactive theory
development even before they have any idea of the Isar proof language at all.

\begin{warn}
  Currently Proof~General does \emph{not} support mixed interactive
  development of classic Isabelle theory files or tactic scripts, together
  with Isar documents.  The ``\texttt{isa}'' and ``\texttt{isar}'' versions of
  Proof~General are handled as two different theorem proving systems, only one
  of these may be active at the same time.
\end{warn}

Porting of existing tactic scripts is best done by running two separate
Proof~General sessions, one for replaying the old script and the other for the
emerging Isabelle/Isar document.


\section{How to write Isar proofs anyway?}

This is one of the key questions, of course.  Isar offers a rather different
approach to formal proof documents than plain old tactic scripts.  Experienced
users of existing interactive theorem proving systems may have to learn
thinking differently in order to make effective use of Isabelle/Isar.  On the
other hand, Isabelle/Isar comes much closer to existing mathematical practice
of formal proof, so users with less experience in old-style tactical proving,
but a good understanding of mathematical proof, might cope with Isar even
better.  See also \cite{Wenzel:1999:TPHOL} for further background information
on Isar.

\medskip This really is a \emph{reference manual}.  Nevertheless, we will also
give some clues of how the concepts introduced here may be put into practice.
Appendix~\ref{ap:refcard} provides a quick reference card of the most common
Isabelle/Isar language elements.  There are several examples distributed with
Isabelle, and available via the Isabelle WWW library:
\begin{center}\small
  \begin{tabular}{l}
    \url{http://www.cl.cam.ac.uk/Research/HVG/Isabelle/library/} \\
    \url{http://isabelle.in.tum.de/library/} \\
  \end{tabular}
\end{center}

See \texttt{HOL/Isar_examples} for a collection of introductory examples, and
\texttt{HOL/HOL-Real/HahnBanach} is a big mathematics application.  Apart from
browsable HTML sources, both sessions provide actual documents (in PDF).

%%% Local Variables: 
%%% mode: latex
%%% TeX-master: "isar-ref"
%%% End: 

%
\begin{isabellebody}%
\def\isabellecontext{prelim}%
%
\isadelimtheory
\isanewline
\isanewline
\isanewline
%
\endisadelimtheory
%
\isatagtheory
\isacommand{theory}\isamarkupfalse%
\ prelim\ \isakeyword{imports}\ base\ \isakeyword{begin}%
\endisatagtheory
{\isafoldtheory}%
%
\isadelimtheory
%
\endisadelimtheory
%
\isamarkupchapter{Preliminaries%
}
\isamarkuptrue%
%
\isamarkupsection{Contexts \label{sec:context}%
}
\isamarkuptrue%
%
\begin{isamarkuptext}%
A logical context represents the background that is required for
  formulating statements and composing proofs.  It acts as a medium to
  produce formal content, depending on earlier material (declarations,
  results etc.).

  For example, derivations within the Isabelle/Pure logic can be
  described as a judgment \isa{{\isasymGamma}\ {\isasymturnstile}\isactrlsub {\isasymTheta}\ {\isasymphi}}, which means that a
  proposition \isa{{\isasymphi}} is derivable from hypotheses \isa{{\isasymGamma}}
  within the theory \isa{{\isasymTheta}}.  There are logical reasons for
  keeping \isa{{\isasymTheta}} and \isa{{\isasymGamma}} separate: theories can be
  liberal about supporting type constructors and schematic
  polymorphism of constants and axioms, while the inner calculus of
  \isa{{\isasymGamma}\ {\isasymturnstile}\ {\isasymphi}} is strictly limited to Simple Type Theory (with
  fixed type variables in the assumptions).

  \medskip Contexts and derivations are linked by the following key
  principles:

  \begin{itemize}

  \item Transfer: monotonicity of derivations admits results to be
  transferred into a \emph{larger} context, i.e.\ \isa{{\isasymGamma}\ {\isasymturnstile}\isactrlsub {\isasymTheta}\ {\isasymphi}} implies \isa{{\isasymGamma}{\isacharprime}\ {\isasymturnstile}\isactrlsub {\isasymTheta}\isactrlsub {\isacharprime}\ {\isasymphi}} for contexts \isa{{\isasymTheta}{\isacharprime}\ {\isasymsupseteq}\ {\isasymTheta}} and \isa{{\isasymGamma}{\isacharprime}\ {\isasymsupseteq}\ {\isasymGamma}}.

  \item Export: discharge of hypotheses admits results to be exported
  into a \emph{smaller} context, i.e.\ \isa{{\isasymGamma}{\isacharprime}\ {\isasymturnstile}\isactrlsub {\isasymTheta}\ {\isasymphi}}
  implies \isa{{\isasymGamma}\ {\isasymturnstile}\isactrlsub {\isasymTheta}\ {\isasymDelta}\ {\isasymLongrightarrow}\ {\isasymphi}} where \isa{{\isasymGamma}{\isacharprime}\ {\isasymsupseteq}\ {\isasymGamma}} and
  \isa{{\isasymDelta}\ {\isacharequal}\ {\isasymGamma}{\isacharprime}\ {\isacharminus}\ {\isasymGamma}}.  Note that \isa{{\isasymTheta}} remains unchanged here,
  only the \isa{{\isasymGamma}} part is affected.

  \end{itemize}

  \medskip By modeling the main characteristics of the primitive
  \isa{{\isasymTheta}} and \isa{{\isasymGamma}} above, and abstracting over any
  particular logical content, we arrive at the fundamental notions of
  \emph{theory context} and \emph{proof context} in Isabelle/Isar.
  These implement a certain policy to manage arbitrary \emph{context
  data}.  There is a strongly-typed mechanism to declare new kinds of
  data at compile time.

  The internal bootstrap process of Isabelle/Pure eventually reaches a
  stage where certain data slots provide the logical content of \isa{{\isasymTheta}} and \isa{{\isasymGamma}} sketched above, but this does not stop there!
  Various additional data slots support all kinds of mechanisms that
  are not necessarily part of the core logic.

  For example, there would be data for canonical introduction and
  elimination rules for arbitrary operators (depending on the
  object-logic and application), which enables users to perform
  standard proof steps implicitly (cf.\ the \isa{rule} method
  \cite{isabelle-isar-ref}).

  \medskip Thus Isabelle/Isar is able to bring forth more and more
  concepts successively.  In particular, an object-logic like
  Isabelle/HOL continues the Isabelle/Pure setup by adding specific
  components for automated reasoning (classical reasoner, tableau
  prover, structured induction etc.) and derived specification
  mechanisms (inductive predicates, recursive functions etc.).  All of
  this is ultimately based on the generic data management by theory
  and proof contexts introduced here.%
\end{isamarkuptext}%
\isamarkuptrue%
%
\isamarkupsubsection{Theory context \label{sec:context-theory}%
}
\isamarkuptrue%
%
\begin{isamarkuptext}%
\glossary{Theory}{FIXME}

  A \emph{theory} is a data container with explicit named and unique
  identifier.  Theories are related by a (nominal) sub-theory
  relation, which corresponds to the dependency graph of the original
  construction; each theory is derived from a certain sub-graph of
  ancestor theories.

  The \isa{merge} operation produces the least upper bound of two
  theories, which actually degenerates into absorption of one theory
  into the other (due to the nominal sub-theory relation).

  The \isa{begin} operation starts a new theory by importing
  several parent theories and entering a special \isa{draft} mode,
  which is sustained until the final \isa{end} operation.  A draft
  theory acts like a linear type, where updates invalidate earlier
  versions.  An invalidated draft is called ``stale''.

  The \isa{checkpoint} operation produces an intermediate stepping
  stone that will survive the next update: both the original and the
  changed theory remain valid and are related by the sub-theory
  relation.  Checkpointing essentially recovers purely functional
  theory values, at the expense of some extra internal bookkeeping.

  The \isa{copy} operation produces an auxiliary version that has
  the same data content, but is unrelated to the original: updates of
  the copy do not affect the original, neither does the sub-theory
  relation hold.

  \medskip The example in \figref{fig:ex-theory} below shows a theory
  graph derived from \isa{Pure}, with theory \isa{Length}
  importing \isa{Nat} and \isa{List}.  The body of \isa{Length} consists of a sequence of updates, working mostly on
  drafts.  Intermediate checkpoints may occur as well, due to the
  history mechanism provided by the Isar top-level, cf.\
  \secref{sec:isar-toplevel}.

  \begin{figure}[htb]
  \begin{center}
  \begin{tabular}{rcccl}
        &            & \isa{Pure} \\
        &            & \isa{{\isasymdown}} \\
        &            & \isa{FOL} \\
        & $\swarrow$ &              & $\searrow$ & \\
  $Nat$ &            &              &            & \isa{List} \\
        & $\searrow$ &              & $\swarrow$ \\
        &            & \isa{Length} \\
        &            & \multicolumn{3}{l}{~~$\isarkeyword{imports}$} \\
        &            & \multicolumn{3}{l}{~~$\isarkeyword{begin}$} \\
        &            & $\vdots$~~ \\
        &            & \isa{{\isasymbullet}}~~ \\
        &            & $\vdots$~~ \\
        &            & \isa{{\isasymbullet}}~~ \\
        &            & $\vdots$~~ \\
        &            & \multicolumn{3}{l}{~~$\isarkeyword{end}$} \\
  \end{tabular}
  \caption{A theory definition depending on ancestors}\label{fig:ex-theory}
  \end{center}
  \end{figure}

  \medskip There is a separate notion of \emph{theory reference} for
  maintaining a live link to an evolving theory context: updates on
  drafts are propagated automatically.  The dynamic stops after an
  explicit \isa{end} only.

  Derived entities may store a theory reference in order to indicate
  the context they belong to.  This implicitly assumes monotonic
  reasoning, because the referenced context may become larger without
  further notice.%
\end{isamarkuptext}%
\isamarkuptrue%
%
\isadelimmlref
%
\endisadelimmlref
%
\isatagmlref
%
\begin{isamarkuptext}%
\begin{mldecls}
  \indexmltype{theory}\verb|type theory| \\
  \indexml{Theory.subthy}\verb|Theory.subthy: theory * theory -> bool| \\
  \indexml{Theory.merge}\verb|Theory.merge: theory * theory -> theory| \\
  \indexml{Theory.checkpoint}\verb|Theory.checkpoint: theory -> theory| \\
  \indexml{Theory.copy}\verb|Theory.copy: theory -> theory| \\[1ex]
  \indexmltype{theory-ref}\verb|type theory_ref| \\
  \indexml{Theory.self-ref}\verb|Theory.self_ref: theory -> theory_ref| \\
  \indexml{Theory.deref}\verb|Theory.deref: theory_ref -> theory| \\
  \end{mldecls}

  \begin{description}

  \item \verb|theory| represents theory contexts.  This is
  essentially a linear type!  Most operations destroy the original
  version, which then becomes ``stale''.

  \item \verb|Theory.subthy|~\isa{{\isacharparenleft}thy\isactrlsub {\isadigit{1}}{\isacharcomma}\ thy\isactrlsub {\isadigit{2}}{\isacharparenright}}
  compares theories according to the inherent graph structure of the
  construction.  This sub-theory relation is a nominal approximation
  of inclusion (\isa{{\isasymsubseteq}}) of the corresponding content.

  \item \verb|Theory.merge|~\isa{{\isacharparenleft}thy\isactrlsub {\isadigit{1}}{\isacharcomma}\ thy\isactrlsub {\isadigit{2}}{\isacharparenright}}
  absorbs one theory into the other.  This fails for unrelated
  theories!

  \item \verb|Theory.checkpoint|~\isa{thy} produces a safe
  stepping stone in the linear development of \isa{thy}.  The next
  update will result in two related, valid theories.

  \item \verb|Theory.copy|~\isa{thy} produces a variant of \isa{thy} that holds a copy of the same data.  The result is not
  related to the original; the original is unchanched.

  \item \verb|theory_ref| represents a sliding reference to an
  always valid theory; updates on the original are propagated
  automatically.

  \item \verb|Theory.self_ref|~\isa{thy} and \verb|Theory.deref|~\isa{thy{\isacharunderscore}ref} convert between \verb|theory| and \verb|theory_ref|.  As the referenced theory
  evolves monotonically over time, later invocations of \verb|Theory.deref| may refer to a larger context.

  \end{description}%
\end{isamarkuptext}%
\isamarkuptrue%
%
\endisatagmlref
{\isafoldmlref}%
%
\isadelimmlref
%
\endisadelimmlref
%
\isamarkupsubsection{Proof context \label{sec:context-proof}%
}
\isamarkuptrue%
%
\begin{isamarkuptext}%
\glossary{Proof context}{The static context of a structured proof,
  acts like a local ``theory'' of the current portion of Isar proof
  text, generalizes the idea of local hypotheses \isa{{\isasymGamma}} in
  judgments \isa{{\isasymGamma}\ {\isasymturnstile}\ {\isasymphi}} of natural deduction calculi.  There is a
  generic notion of introducing and discharging hypotheses.
  Arbritrary auxiliary context data may be adjoined.}

  A proof context is a container for pure data with a back-reference
  to the theory it belongs to.  The \isa{init} operation creates a
  proof context from a given theory.  Modifications to draft theories
  are propagated to the proof context as usual, but there is also an
  explicit \isa{transfer} operation to force resynchronization
  with more substantial updates to the underlying theory.  The actual
  context data does not require any special bookkeeping, thanks to the
  lack of destructive features.

  Entities derived in a proof context need to record inherent logical
  requirements explicitly, since there is no separate context
  identification as for theories.  For example, hypotheses used in
  primitive derivations (cf.\ \secref{sec:thms}) are recorded
  separately within the sequent \isa{{\isasymGamma}\ {\isasymturnstile}\ {\isasymphi}}, just to make double
  sure.  Results could still leak into an alien proof context do to
  programming errors, but Isabelle/Isar includes some extra validity
  checks in critical positions, notably at the end of sub-proof.

  Proof contexts may be manipulated arbitrarily, although the common
  discipline is to follow block structure as a mental model: a given
  context is extended consecutively, and results are exported back
  into the original context.  Note that the Isar proof states model
  block-structured reasoning explicitly, using a stack of proof
  contexts internally, cf.\ \secref{sec:isar-proof-state}.%
\end{isamarkuptext}%
\isamarkuptrue%
%
\isadelimmlref
%
\endisadelimmlref
%
\isatagmlref
%
\begin{isamarkuptext}%
\begin{mldecls}
  \indexmltype{Proof.context}\verb|type Proof.context| \\
  \indexml{ProofContext.init}\verb|ProofContext.init: theory -> Proof.context| \\
  \indexml{ProofContext.theory-of}\verb|ProofContext.theory_of: Proof.context -> theory| \\
  \indexml{ProofContext.transfer}\verb|ProofContext.transfer: theory -> Proof.context -> Proof.context| \\
  \end{mldecls}

  \begin{description}

  \item \verb|Proof.context| represents proof contexts.  Elements
  of this type are essentially pure values, with a sliding reference
  to the background theory.

  \item \verb|ProofContext.init|~\isa{thy} produces a proof context
  derived from \isa{thy}, initializing all data.

  \item \verb|ProofContext.theory_of|~\isa{ctxt} selects the
  background theory from \isa{ctxt}, dereferencing its internal
  \verb|theory_ref|.

  \item \verb|ProofContext.transfer|~\isa{thy\ ctxt} promotes the
  background theory of \isa{ctxt} to the super theory \isa{thy}.

  \end{description}%
\end{isamarkuptext}%
\isamarkuptrue%
%
\endisatagmlref
{\isafoldmlref}%
%
\isadelimmlref
%
\endisadelimmlref
%
\isamarkupsubsection{Generic contexts \label{sec:generic-context}%
}
\isamarkuptrue%
%
\begin{isamarkuptext}%
A generic context is the disjoint sum of either a theory or proof
  context.  Occasionally, this enables uniform treatment of generic
  context data, typically extra-logical information.  Operations on
  generic contexts include the usual injections, partial selections,
  and combinators for lifting operations on either component of the
  disjoint sum.

  Moreover, there are total operations \isa{theory{\isacharunderscore}of} and \isa{proof{\isacharunderscore}of} to convert a generic context into either kind: a theory
  can always be selected from the sum, while a proof context might
  have to be constructed by an ad-hoc \isa{init} operation.%
\end{isamarkuptext}%
\isamarkuptrue%
%
\isadelimmlref
%
\endisadelimmlref
%
\isatagmlref
%
\begin{isamarkuptext}%
\begin{mldecls}
  \indexmltype{Context.generic}\verb|type Context.generic| \\
  \indexml{Context.theory-of}\verb|Context.theory_of: Context.generic -> theory| \\
  \indexml{Context.proof-of}\verb|Context.proof_of: Context.generic -> Proof.context| \\
  \end{mldecls}

  \begin{description}

  \item \verb|Context.generic| is the direct sum of \verb|theory| and \verb|Proof.context|, with the datatype
  constructors \verb|Context.Theory| and \verb|Context.Proof|.

  \item \verb|Context.theory_of|~\isa{context} always produces a
  theory from the generic \isa{context}, using \verb|ProofContext.theory_of| as required.

  \item \verb|Context.proof_of|~\isa{context} always produces a
  proof context from the generic \isa{context}, using \verb|ProofContext.init| as required (note that this re-initializes the
  context data with each invocation).

  \end{description}%
\end{isamarkuptext}%
\isamarkuptrue%
%
\endisatagmlref
{\isafoldmlref}%
%
\isadelimmlref
%
\endisadelimmlref
%
\isamarkupsubsection{Context data%
}
\isamarkuptrue%
%
\begin{isamarkuptext}%
The main purpose of theory and proof contexts is to manage arbitrary
  data.  New data types can be declared incrementally at compile time.
  There are separate declaration mechanisms for any of the three kinds
  of contexts: theory, proof, generic.

  \paragraph{Theory data} may refer to destructive entities, which are
  maintained in direct correspondence to the linear evolution of
  theory values, including explicit copies.\footnote{Most existing
  instances of destructive theory data are merely historical relics
  (e.g.\ the destructive theorem storage, and destructive hints for
  the Simplifier and Classical rules).}  A theory data declaration
  needs to implement the following specification (depending on type
  \isa{T}):

  \medskip
  \begin{tabular}{ll}
  \isa{name{\isacharcolon}\ string} \\
  \isa{empty{\isacharcolon}\ T} & initial value \\
  \isa{copy{\isacharcolon}\ T\ {\isasymrightarrow}\ T} & refresh impure data \\
  \isa{extend{\isacharcolon}\ T\ {\isasymrightarrow}\ T} & re-initialize on import \\
  \isa{merge{\isacharcolon}\ T\ {\isasymtimes}\ T\ {\isasymrightarrow}\ T} & join on import \\
  \isa{print{\isacharcolon}\ T\ {\isasymrightarrow}\ unit} & diagnostic output \\
  \end{tabular}
  \medskip

  \noindent The \isa{name} acts as a comment for diagnostic
  messages; \isa{copy} is just the identity for pure data; \isa{extend} is acts like a unitary version of \isa{merge}, both
  should also include the functionality of \isa{copy} for impure
  data.

  \paragraph{Proof context data} is purely functional.  A declaration
  needs to implement the following specification:

  \medskip
  \begin{tabular}{ll}
  \isa{name{\isacharcolon}\ string} \\
  \isa{init{\isacharcolon}\ theory\ {\isasymrightarrow}\ T} & produce initial value \\
  \isa{print{\isacharcolon}\ T\ {\isasymrightarrow}\ unit} & diagnostic output \\
  \end{tabular}
  \medskip

  \noindent The \isa{init} operation is supposed to produce a pure
  value from the given background theory.  The remainder is analogous
  to theory data.

  \paragraph{Generic data} provides a hybrid interface for both theory
  and proof data.  The declaration is essentially the same as for
  (pure) theory data, without \isa{copy}, though.  The \isa{init} operation for proof contexts merely selects the current data
  value from the background theory.

  \bigskip In any case, a data declaration of type \isa{T} results
  in the following interface:

  \medskip
  \begin{tabular}{ll}
  \isa{init{\isacharcolon}\ theory\ {\isasymrightarrow}\ theory} \\
  \isa{get{\isacharcolon}\ context\ {\isasymrightarrow}\ T} \\
  \isa{put{\isacharcolon}\ T\ {\isasymrightarrow}\ context\ {\isasymrightarrow}\ context} \\
  \isa{map{\isacharcolon}\ {\isacharparenleft}T\ {\isasymrightarrow}\ T{\isacharparenright}\ {\isasymrightarrow}\ context\ {\isasymrightarrow}\ context} \\
  \isa{print{\isacharcolon}\ context\ {\isasymrightarrow}\ unit}
  \end{tabular}
  \medskip

  \noindent Here \isa{init} needs to be applied to the current
  theory context once, in order to register the initial setup.  The
  other operations provide access for the particular kind of context
  (theory, proof, or generic context).  Note that this is a safe
  interface: there is no other way to access the corresponding data
  slot of a context.  By keeping these operations private, a component
  may maintain abstract values authentically, without other components
  interfering.%
\end{isamarkuptext}%
\isamarkuptrue%
%
\isadelimmlref
%
\endisadelimmlref
%
\isatagmlref
%
\begin{isamarkuptext}%
\begin{mldecls}
  \indexmlfunctor{TheoryDataFun}\verb|functor TheoryDataFun| \\
  \indexmlfunctor{ProofDataFun}\verb|functor ProofDataFun| \\
  \indexmlfunctor{GenericDataFun}\verb|functor GenericDataFun| \\
  \end{mldecls}

  \begin{description}

  \item \verb|TheoryDataFun|\isa{{\isacharparenleft}spec{\isacharparenright}} declares data for
  type \verb|theory| according to the specification provided as
  argument structure.  The resulting structure provides data init and
  access operations as described above.

  \item \verb|ProofDataFun|\isa{{\isacharparenleft}spec{\isacharparenright}} is analogous for
  type \verb|Proof.context|.

  \item \verb|GenericDataFun|\isa{{\isacharparenleft}spec{\isacharparenright}} is analogous for
  type \verb|Context.generic|.

  \end{description}%
\end{isamarkuptext}%
\isamarkuptrue%
%
\endisatagmlref
{\isafoldmlref}%
%
\isadelimmlref
%
\endisadelimmlref
%
\isamarkupsection{Name spaces%
}
\isamarkuptrue%
%
\begin{isamarkuptext}%
By general convention, each kind of formal entities (logical
  constant, type, type class, theorem, method etc.) lives in a
  separate name space.  It is usually clear from the syntactic context
  of a name, which kind of entity it refers to.  For example, proof
  method \isa{foo} vs.\ theorem \isa{foo} vs.\ logical
  constant \isa{foo} are easily distinguished thanks to the design
  of the concrete outer syntax.  A notable exception are logical
  identifiers within a term (\secref{sec:terms}): constants, fixed
  variables, and bound variables all share the same identifier syntax,
  but are distinguished by their scope.

  Name spaces are organized uniformly, as a collection of qualified
  names consisting of a sequence of basic name components separated by
  dots: \isa{Bar{\isachardot}bar{\isachardot}foo}, \isa{Bar{\isachardot}foo}, and \isa{foo}
  are examples for qualified names.

  Despite the independence of names of different kinds, certain naming
  conventions may relate them to each other.  For example, a constant
  \isa{foo} could be accompanied with theorems \isa{foo{\isachardot}intro}, \isa{foo{\isachardot}elim}, \isa{foo{\isachardot}simps} etc.  The same
  could happen for a type \isa{foo}, but this is apt to cause
  clashes in the theorem name space!  To avoid this, there is an
  additional convention to add a suffix that determines the original
  kind.  For example, constant \isa{foo} could associated with
  theorem \isa{foo{\isachardot}intro}, type \isa{foo} with theorem \isa{foo{\isacharunderscore}type{\isachardot}intro}, and type class \isa{foo} with \isa{foo{\isacharunderscore}class{\isachardot}intro}.

  \medskip Name components are subdivided into \emph{symbols}, which
  constitute the smallest textual unit in Isabelle --- raw characters
  are normally not encountered.%
\end{isamarkuptext}%
\isamarkuptrue%
%
\isamarkupsubsection{Strings of symbols%
}
\isamarkuptrue%
%
\begin{isamarkuptext}%
Isabelle strings consist of a sequence of
  symbols\glossary{Symbol}{The smallest unit of text in Isabelle,
  subsumes plain ASCII characters as well as an infinite collection of
  named symbols (for greek, math etc.).}, which are either packed as
  an actual \isa{string}, or represented as a list.  Each symbol
  is in itself a small string of the following form:

  \begin{enumerate}

  \item either a singleton ASCII character ``\isa{c}'' (with
  character code 0--127), for example ``\verb,a,'',

  \item or a regular symbol ``\verb,\,\verb,<,\isa{ident}\verb,>,'', for example ``\verb,\,\verb,<alpha>,'',

  \item or a control symbol ``\verb,\,\verb,<^,\isa{ident}\verb,>,'', for example ``\verb,\,\verb,<^bold>,'',

  \item or a raw control symbol ``\verb,\,\verb,<^raw:,\isa{{\isasymdots}}\verb,>,'' where ``\isa{{\isasymdots}}'' refers to any printable ASCII
  character (excluding ``\verb,.,'' and ``\verb,>,'') or non-ASCII
  character, for example ``\verb,\,\verb,<^raw:$\sum_{i = 1}^n$>,'',

  \item or a numbered raw control symbol ``\verb,\,\verb,<^raw,\isa{nnn}\verb,>, where \isa{nnn} are digits, for example
  ``\verb,\,\verb,<^raw42>,''.

  \end{enumerate}

  The \isa{ident} syntax for symbol names is \isa{letter\ {\isacharparenleft}letter\ {\isacharbar}\ digit{\isacharparenright}\isactrlsup {\isacharasterisk}}, where \isa{letter\ {\isacharequal}\ A{\isachardot}{\isachardot}Za{\isachardot}{\isachardot}z} and
  \isa{digit\ {\isacharequal}\ {\isadigit{0}}{\isachardot}{\isachardot}{\isadigit{9}}}.  There are infinitely many regular symbols
  and control symbols available, but a certain collection of standard
  symbols is treated specifically.  For example,
  ``\verb,\,\verb,<alpha>,'' is classified as a (non-ASCII) letter,
  which means it may occur within regular Isabelle identifier syntax.

  Output of symbols depends on the print mode
  (\secref{sec:print-mode}).  For example, the standard {\LaTeX} setup
  of the Isabelle document preparation system would present
  ``\verb,\,\verb,<alpha>,'' as \isa{{\isasymalpha}}, and
  ``\verb,\,\verb,<^bold>,\verb,\,\verb,<alpha>,'' as \isa{\isactrlbold {\isasymalpha}}.

  \medskip It is important to note that the character set underlying
  Isabelle symbols is plain 7-bit ASCII.  Since 8-bit characters are
  passed through transparently, Isabelle may easily process
  Unicode/UCS data as well (using UTF-8 encoding).  Unicode provides
  its own collection of mathematical symbols, but there is no built-in
  link to the ones of Isabelle.%
\end{isamarkuptext}%
\isamarkuptrue%
%
\isadelimmlref
%
\endisadelimmlref
%
\isatagmlref
%
\begin{isamarkuptext}%
\begin{mldecls}
  \indexmltype{Symbol.symbol}\verb|type Symbol.symbol| \\
  \indexml{Symbol.explode}\verb|Symbol.explode: string -> Symbol.symbol list| \\
  \indexml{Symbol.is-letter}\verb|Symbol.is_letter: Symbol.symbol -> bool| \\
  \indexml{Symbol.is-digit}\verb|Symbol.is_digit: Symbol.symbol -> bool| \\
  \indexml{Symbol.is-quasi}\verb|Symbol.is_quasi: Symbol.symbol -> bool| \\
  \indexml{Symbol.is-blank}\verb|Symbol.is_blank: Symbol.symbol -> bool| \\[1ex]
  \indexmltype{Symbol.sym}\verb|type Symbol.sym| \\
  \indexml{Symbol.decode}\verb|Symbol.decode: Symbol.symbol -> Symbol.sym| \\
  \end{mldecls}

  \begin{description}

  \item \verb|Symbol.symbol| represents Isabelle symbols.  This
  type is an alias for \verb|string|, but emphasizes the
  specific format encountered here.

  \item \verb|Symbol.explode|~\isa{s} produces a symbol list from
  the packed form that is encountered in most practical situations.
  This function supercedes \verb|String.explode| for virtually all
  purposes of manipulating text in Isabelle!  Plain \verb|implode|
  may still be used for the reverse operation.

  \item \verb|Symbol.is_letter|, \verb|Symbol.is_digit|, \verb|Symbol.is_quasi|, \verb|Symbol.is_blank| classify certain symbols
  (both ASCII and several named ones) according to fixed syntactic
  conventions of Isabelle, cf.\ \cite{isabelle-isar-ref}.

  \item \verb|Symbol.sym| is a concrete datatype that represents
  the different kinds of symbols explicitly with constructors \verb|Symbol.Char|, \verb|Symbol.Sym|, \verb|Symbol.Ctrl|, or \verb|Symbol.Raw|.

  \item \verb|Symbol.decode| converts the string representation of a
  symbol into the datatype version.

  \end{description}%
\end{isamarkuptext}%
\isamarkuptrue%
%
\endisatagmlref
{\isafoldmlref}%
%
\isadelimmlref
%
\endisadelimmlref
%
\isamarkupsubsection{Qualified names%
}
\isamarkuptrue%
%
\begin{isamarkuptext}%
A \emph{qualified name} essentially consists of a non-empty list of
  basic name components.  The packad notation uses a dot as separator,
  as in \isa{A{\isachardot}b}, for example.  The very last component is called
  \emph{base} name, the remaining prefix \emph{qualifier} (which may
  be empty).

  A \isa{naming} policy tells how to produce fully qualified names
  from a given specification.  The \isa{full} operation applies
  performs naming of a name; the policy is usually taken from the
  context.  For example, a common policy is to attach an implicit
  prefix.

  A \isa{name\ space} manages declarations of fully qualified
  names.  There are separate operations to \isa{declare}, \isa{intern}, and \isa{extern} names.

  FIXME%
\end{isamarkuptext}%
\isamarkuptrue%
%
\isadelimmlref
%
\endisadelimmlref
%
\isatagmlref
%
\begin{isamarkuptext}%
FIXME%
\end{isamarkuptext}%
\isamarkuptrue%
%
\endisatagmlref
{\isafoldmlref}%
%
\isadelimmlref
%
\endisadelimmlref
%
\isamarkupsection{Structured output%
}
\isamarkuptrue%
%
\isamarkupsubsection{Pretty printing%
}
\isamarkuptrue%
%
\begin{isamarkuptext}%
FIXME%
\end{isamarkuptext}%
\isamarkuptrue%
%
\isamarkupsubsection{Output channels%
}
\isamarkuptrue%
%
\begin{isamarkuptext}%
FIXME%
\end{isamarkuptext}%
\isamarkuptrue%
%
\isamarkupsubsection{Print modes \label{sec:print-mode}%
}
\isamarkuptrue%
%
\begin{isamarkuptext}%
FIXME%
\end{isamarkuptext}%
\isamarkuptrue%
%
\isadelimtheory
%
\endisadelimtheory
%
\isatagtheory
\isacommand{end}\isamarkupfalse%
%
\endisatagtheory
{\isafoldtheory}%
%
\isadelimtheory
%
\endisadelimtheory
\isanewline
\end{isabellebody}%
%%% Local Variables:
%%% mode: latex
%%% TeX-master: "root"
%%% End:

%
\begin{isabellebody}%
\def\isabellecontext{logic}%
%
\isadelimtheory
\isanewline
\isanewline
\isanewline
%
\endisadelimtheory
%
\isatagtheory
\isacommand{theory}\isamarkupfalse%
\ logic\ \isakeyword{imports}\ base\ \isakeyword{begin}%
\endisatagtheory
{\isafoldtheory}%
%
\isadelimtheory
%
\endisadelimtheory
%
\isamarkupchapter{Primitive logic \label{ch:logic}%
}
\isamarkuptrue%
%
\begin{isamarkuptext}%
The logical foundations of Isabelle/Isar are that of the Pure logic,
  which has been introduced as a natural-deduction framework in
  \cite{paulson700}.  This is essentially the same logic as ``\isa{{\isasymlambda}HOL}'' in the more abstract setting of Pure Type Systems (PTS)
  \cite{Barendregt-Geuvers:2001}, although there are some key
  differences in the specific treatment of simple types in
  Isabelle/Pure.

  Following type-theoretic parlance, the Pure logic consists of three
  levels of \isa{{\isasymlambda}}-calculus with corresponding arrows: \isa{{\isasymRightarrow}} for syntactic function space (terms depending on terms), \isa{{\isasymAnd}} for universal quantification (proofs depending on terms), and
  \isa{{\isasymLongrightarrow}} for implication (proofs depending on proofs).

  Pure derivations are relative to a logical theory, which declares
  type constructors, term constants, and axioms.  Theory declarations
  support schematic polymorphism, which is strictly speaking outside
  the logic.\footnote{Incidently, this is the main logical reason, why
  the theory context \isa{{\isasymTheta}} is separate from the context \isa{{\isasymGamma}} of the core calculus.}%
\end{isamarkuptext}%
\isamarkuptrue%
%
\isamarkupsection{Types \label{sec:types}%
}
\isamarkuptrue%
%
\begin{isamarkuptext}%
The language of types is an uninterpreted order-sorted first-order
  algebra; types are qualified by ordered type classes.

  \medskip A \emph{type class} is an abstract syntactic entity
  declared in the theory context.  The \emph{subclass relation} \isa{c\isactrlisub {\isadigit{1}}\ {\isasymsubseteq}\ c\isactrlisub {\isadigit{2}}} is specified by stating an acyclic
  generating relation; the transitive closure is maintained
  internally.  The resulting relation is an ordering: reflexive,
  transitive, and antisymmetric.

  A \emph{sort} is a list of type classes written as \isa{{\isacharbraceleft}c\isactrlisub {\isadigit{1}}{\isacharcomma}\ {\isasymdots}{\isacharcomma}\ c\isactrlisub m{\isacharbraceright}}, which represents symbolic
  intersection.  Notationally, the curly braces are omitted for
  singleton intersections, i.e.\ any class \isa{c} may be read as
  a sort \isa{{\isacharbraceleft}c{\isacharbraceright}}.  The ordering on type classes is extended to
  sorts according to the meaning of intersections: \isa{{\isacharbraceleft}c\isactrlisub {\isadigit{1}}{\isacharcomma}\ {\isasymdots}\ c\isactrlisub m{\isacharbraceright}\ {\isasymsubseteq}\ {\isacharbraceleft}d\isactrlisub {\isadigit{1}}{\isacharcomma}\ {\isasymdots}{\isacharcomma}\ d\isactrlisub n{\isacharbraceright}} iff
  \isa{{\isasymforall}j{\isachardot}\ {\isasymexists}i{\isachardot}\ c\isactrlisub i\ {\isasymsubseteq}\ d\isactrlisub j}.  The empty intersection
  \isa{{\isacharbraceleft}{\isacharbraceright}} refers to the universal sort, which is the largest
  element wrt.\ the sort order.  The intersections of all (finitely
  many) classes declared in the current theory are the minimal
  elements wrt.\ the sort order.

  \medskip A \emph{fixed type variable} is a pair of a basic name
  (starting with a \isa{{\isacharprime}} character) and a sort constraint.  For
  example, \isa{{\isacharparenleft}{\isacharprime}a{\isacharcomma}\ s{\isacharparenright}} which is usually printed as \isa{{\isasymalpha}\isactrlisub s}.  A \emph{schematic type variable} is a pair of an
  indexname and a sort constraint.  For example, \isa{{\isacharparenleft}{\isacharparenleft}{\isacharprime}a{\isacharcomma}\ {\isadigit{0}}{\isacharparenright}{\isacharcomma}\ s{\isacharparenright}} which is usually printed as \isa{{\isacharquery}{\isasymalpha}\isactrlisub s}.

  Note that \emph{all} syntactic components contribute to the identity
  of type variables, including the sort constraint.  The core logic
  handles type variables with the same name but different sorts as
  different, although some outer layers of the system make it hard to
  produce anything like this.

  A \emph{type constructor} \isa{{\isasymkappa}} is a \isa{k}-ary operator
  on types declared in the theory.  Type constructor application is
  usually written postfix as \isa{{\isacharparenleft}{\isasymalpha}\isactrlisub {\isadigit{1}}{\isacharcomma}\ {\isasymdots}{\isacharcomma}\ {\isasymalpha}\isactrlisub k{\isacharparenright}{\isasymkappa}}.
  For \isa{k\ {\isacharequal}\ {\isadigit{0}}} the argument tuple is omitted, e.g.\ \isa{prop} instead of \isa{{\isacharparenleft}{\isacharparenright}prop}.  For \isa{k\ {\isacharequal}\ {\isadigit{1}}} the
  parentheses are omitted, e.g.\ \isa{{\isasymalpha}\ list} instead of \isa{{\isacharparenleft}{\isasymalpha}{\isacharparenright}list}.  Further notation is provided for specific constructors,
  notably the right-associative infix \isa{{\isasymalpha}\ {\isasymRightarrow}\ {\isasymbeta}} instead of
  \isa{{\isacharparenleft}{\isasymalpha}{\isacharcomma}\ {\isasymbeta}{\isacharparenright}fun}.
  
  A \emph{type} is defined inductively over type variables and type
  constructors as follows: \isa{{\isasymtau}\ {\isacharequal}\ {\isasymalpha}\isactrlisub s\ {\isacharbar}\ {\isacharquery}{\isasymalpha}\isactrlisub s\ {\isacharbar}\ {\isacharparenleft}{\isasymtau}\isactrlsub {\isadigit{1}}{\isacharcomma}\ {\isasymdots}{\isacharcomma}\ {\isasymtau}\isactrlsub k{\isacharparenright}k}.

  A \emph{type abbreviation} is a syntactic abbreviation \isa{{\isacharparenleft}\isactrlvec {\isasymalpha}{\isacharparenright}{\isasymkappa}\ {\isacharequal}\ {\isasymtau}} of an arbitrary type expression \isa{{\isasymtau}} over
  variables \isa{\isactrlvec {\isasymalpha}}.  Type abbreviations looks like type
  constructors at the surface, but are fully expanded before entering
  the logical core.

  A \emph{type arity} declares the image behavior of a type
  constructor wrt.\ the algebra of sorts: \isa{{\isasymkappa}\ {\isacharcolon}{\isacharcolon}\ {\isacharparenleft}s\isactrlisub {\isadigit{1}}{\isacharcomma}\ {\isasymdots}{\isacharcomma}\ s\isactrlisub k{\isacharparenright}s} means that \isa{{\isacharparenleft}{\isasymtau}\isactrlisub {\isadigit{1}}{\isacharcomma}\ {\isasymdots}{\isacharcomma}\ {\isasymtau}\isactrlisub k{\isacharparenright}{\isasymkappa}} is
  of sort \isa{s} if every argument type \isa{{\isasymtau}\isactrlisub i} is
  of sort \isa{s\isactrlisub i}.  Arity declarations are implicitly
  completed, i.e.\ \isa{{\isasymkappa}\ {\isacharcolon}{\isacharcolon}\ {\isacharparenleft}\isactrlvec s{\isacharparenright}c} entails \isa{{\isasymkappa}\ {\isacharcolon}{\isacharcolon}\ {\isacharparenleft}\isactrlvec s{\isacharparenright}c{\isacharprime}} for any \isa{c{\isacharprime}\ {\isasymsupseteq}\ c}.

  \medskip The sort algebra is always maintained as \emph{coregular},
  which means that type arities are consistent with the subclass
  relation: for each type constructor \isa{{\isasymkappa}} and classes \isa{c\isactrlisub {\isadigit{1}}\ {\isasymsubseteq}\ c\isactrlisub {\isadigit{2}}}, any arity \isa{{\isasymkappa}\ {\isacharcolon}{\isacharcolon}\ {\isacharparenleft}\isactrlvec s\isactrlisub {\isadigit{1}}{\isacharparenright}c\isactrlisub {\isadigit{1}}} has a corresponding arity \isa{{\isasymkappa}\ {\isacharcolon}{\isacharcolon}\ {\isacharparenleft}\isactrlvec s\isactrlisub {\isadigit{2}}{\isacharparenright}c\isactrlisub {\isadigit{2}}} where \isa{\isactrlvec s\isactrlisub {\isadigit{1}}\ {\isasymsubseteq}\ \isactrlvec s\isactrlisub {\isadigit{2}}} holds componentwise.

  The key property of a coregular order-sorted algebra is that sort
  constraints may be always solved in a most general fashion: for each
  type constructor \isa{{\isasymkappa}} and sort \isa{s} there is a most
  general vector of argument sorts \isa{{\isacharparenleft}s\isactrlisub {\isadigit{1}}{\isacharcomma}\ {\isasymdots}{\isacharcomma}\ s\isactrlisub k{\isacharparenright}} such that a type scheme \isa{{\isacharparenleft}{\isasymalpha}\isactrlbsub s\isactrlisub {\isadigit{1}}\isactrlesub {\isacharcomma}\ {\isasymdots}{\isacharcomma}\ {\isasymalpha}\isactrlbsub s\isactrlisub k\isactrlesub {\isacharparenright}{\isasymkappa}} is
  of sort \isa{s}.  Consequently, the unification problem on the
  algebra of types has most general solutions (modulo renaming and
  equivalence of sorts).  Moreover, the usual type-inference algorithm
  will produce primary types as expected \cite{nipkow-prehofer}.%
\end{isamarkuptext}%
\isamarkuptrue%
%
\isadelimmlref
%
\endisadelimmlref
%
\isatagmlref
%
\begin{isamarkuptext}%
\begin{mldecls}
  \indexmltype{class}\verb|type class| \\
  \indexmltype{sort}\verb|type sort| \\
  \indexmltype{arity}\verb|type arity| \\
  \indexmltype{typ}\verb|type typ| \\
  \indexml{fold-atyps}\verb|fold_atyps: (typ -> 'a -> 'a) -> typ -> 'a -> 'a| \\
  \indexml{Sign.subsort}\verb|Sign.subsort: theory -> sort * sort -> bool| \\
  \indexml{Sign.of-sort}\verb|Sign.of_sort: theory -> typ * sort -> bool| \\
  \indexml{Sign.add-types}\verb|Sign.add_types: (bstring * int * mixfix) list -> theory -> theory| \\
  \indexml{Sign.add-tyabbrs-i}\verb|Sign.add_tyabbrs_i: |\isasep\isanewline%
\verb|  (bstring * string list * typ * mixfix) list -> theory -> theory| \\
  \indexml{Sign.primitive-class}\verb|Sign.primitive_class: string * class list -> theory -> theory| \\
  \indexml{Sign.primitive-classrel}\verb|Sign.primitive_classrel: class * class -> theory -> theory| \\
  \indexml{Sign.primitive-arity}\verb|Sign.primitive_arity: arity -> theory -> theory| \\
  \end{mldecls}

  \begin{description}

  \item \verb|class| represents type classes; this is an alias for
  \verb|string|.

  \item \verb|sort| represents sorts; this is an alias for
  \verb|class list|.

  \item \verb|arity| represents type arities; this is an alias for
  triples of the form \isa{{\isacharparenleft}{\isasymkappa}{\isacharcomma}\ \isactrlvec s{\isacharcomma}\ s{\isacharparenright}} for \isa{{\isasymkappa}\ {\isacharcolon}{\isacharcolon}\ {\isacharparenleft}\isactrlvec s{\isacharparenright}s} described above.

  \item \verb|typ| represents types; this is a datatype with
  constructors \verb|TFree|, \verb|TVar|, \verb|Type|.

  \item \verb|fold_atyps|~\isa{f\ {\isasymtau}} iterates function \isa{f}
  over all occurrences of atoms (\verb|TFree| or \verb|TVar|) of \isa{{\isasymtau}}; the type structure is traversed from left to right.

  \item \verb|Sign.subsort|~\isa{thy\ {\isacharparenleft}s\isactrlisub {\isadigit{1}}{\isacharcomma}\ s\isactrlisub {\isadigit{2}}{\isacharparenright}}
  tests the subsort relation \isa{s\isactrlisub {\isadigit{1}}\ {\isasymsubseteq}\ s\isactrlisub {\isadigit{2}}}.

  \item \verb|Sign.of_sort|~\isa{thy\ {\isacharparenleft}{\isasymtau}{\isacharcomma}\ s{\isacharparenright}} tests whether a type
  is of a given sort.

  \item \verb|Sign.add_types|~\isa{{\isacharbrackleft}{\isacharparenleft}{\isasymkappa}{\isacharcomma}\ k{\isacharcomma}\ mx{\isacharparenright}{\isacharcomma}\ {\isasymdots}{\isacharbrackright}} declares new
  type constructors \isa{{\isasymkappa}} with \isa{k} arguments and
  optional mixfix syntax.

  \item \verb|Sign.add_tyabbrs_i|~\isa{{\isacharbrackleft}{\isacharparenleft}{\isasymkappa}{\isacharcomma}\ \isactrlvec {\isasymalpha}{\isacharcomma}\ {\isasymtau}{\isacharcomma}\ mx{\isacharparenright}{\isacharcomma}\ {\isasymdots}{\isacharbrackright}}
  defines a new type abbreviation \isa{{\isacharparenleft}\isactrlvec {\isasymalpha}{\isacharparenright}{\isasymkappa}\ {\isacharequal}\ {\isasymtau}} with
  optional mixfix syntax.

  \item \verb|Sign.primitive_class|~\isa{{\isacharparenleft}c{\isacharcomma}\ {\isacharbrackleft}c\isactrlisub {\isadigit{1}}{\isacharcomma}\ {\isasymdots}{\isacharcomma}\ c\isactrlisub n{\isacharbrackright}{\isacharparenright}} declares new class \isa{c}, together with class
  relations \isa{c\ {\isasymsubseteq}\ c\isactrlisub i}, for \isa{i\ {\isacharequal}\ {\isadigit{1}}{\isacharcomma}\ {\isasymdots}{\isacharcomma}\ n}.

  \item \verb|Sign.primitive_classrel|~\isa{{\isacharparenleft}c\isactrlisub {\isadigit{1}}{\isacharcomma}\ c\isactrlisub {\isadigit{2}}{\isacharparenright}} declares class relation \isa{c\isactrlisub {\isadigit{1}}\ {\isasymsubseteq}\ c\isactrlisub {\isadigit{2}}}.

  \item \verb|Sign.primitive_arity|~\isa{{\isacharparenleft}{\isasymkappa}{\isacharcomma}\ \isactrlvec s{\isacharcomma}\ s{\isacharparenright}} declares
  arity \isa{{\isasymkappa}\ {\isacharcolon}{\isacharcolon}\ {\isacharparenleft}\isactrlvec s{\isacharparenright}s}.

  \end{description}%
\end{isamarkuptext}%
\isamarkuptrue%
%
\endisatagmlref
{\isafoldmlref}%
%
\isadelimmlref
%
\endisadelimmlref
%
\isamarkupsection{Terms \label{sec:terms}%
}
\isamarkuptrue%
%
\begin{isamarkuptext}%
\glossary{Term}{FIXME}

  The language of terms is that of simply-typed \isa{{\isasymlambda}}-calculus
  with de-Bruijn indices for bound variables, and named free
  variables, and constants.  Terms with loose bound variables are
  usually considered malformed.  The types of variables and constants
  is stored explicitly at each occurrence in the term (which is a
  known performance issue).

  FIXME de-Bruijn representation of lambda terms

  Term syntax provides explicit abstraction \isa{{\isasymlambda}x\ {\isacharcolon}{\isacharcolon}\ {\isasymalpha}{\isachardot}\ b{\isacharparenleft}x{\isacharparenright}}
  and application \isa{t\ u}, while types are usually implicit
  thanks to type-inference.

  Terms of type \isa{prop} are called
  propositions.  Logical statements are composed via \isa{{\isasymAnd}x\ {\isacharcolon}{\isacharcolon}\ {\isasymalpha}{\isachardot}\ B{\isacharparenleft}x{\isacharparenright}} and \isa{A\ {\isasymLongrightarrow}\ B}.


  \[
  \infer{\isa{{\isacharparenleft}{\isasymlambda}x\isactrlsub {\isasymtau}{\isachardot}\ t{\isacharparenright}{\isacharcolon}\ {\isasymtau}\ {\isasymRightarrow}\ {\isasymsigma}}}{\isa{t{\isacharcolon}\ {\isasymsigma}}}
  \qquad
  \infer{\isa{{\isacharparenleft}t\ u{\isacharparenright}{\isacharcolon}\ {\isasymsigma}}}{\isa{t{\isacharcolon}\ {\isasymtau}\ {\isasymRightarrow}\ {\isasymsigma}} & \isa{u{\isacharcolon}\ {\isasymtau}}}
  \]%
\end{isamarkuptext}%
\isamarkuptrue%
%
\begin{isamarkuptext}%
FIXME

\glossary{Schematic polymorphism}{FIXME}

\glossary{Type variable}{FIXME}%
\end{isamarkuptext}%
\isamarkuptrue%
%
\isamarkupsection{Theorems \label{sec:thms}%
}
\isamarkuptrue%
%
\begin{isamarkuptext}%
Primitive reasoning operates on judgments of the form \isa{{\isasymGamma}\ {\isasymturnstile}\ {\isasymphi}}, with standard introduction and elimination rules for \isa{{\isasymAnd}} and \isa{{\isasymLongrightarrow}} that refer to fixed parameters \isa{x} and
  hypotheses \isa{A} from the context \isa{{\isasymGamma}}.  The
  corresponding proof terms are left implicit in the classic
  ``LCF-approach'', although they could be exploited separately
  \cite{Berghofer-Nipkow:2000}.

  The framework also provides definitional equality \isa{{\isasymequiv}\ {\isacharcolon}{\isacharcolon}\ {\isasymalpha}\ {\isasymRightarrow}\ {\isasymalpha}\ {\isasymRightarrow}\ prop}, with \isa{{\isasymalpha}{\isasymbeta}{\isasymeta}}-conversion rules.  The internal
  conjunction \isa{{\isacharampersand}\ {\isacharcolon}{\isacharcolon}\ prop\ {\isasymRightarrow}\ prop\ {\isasymRightarrow}\ prop} enables the view of
  assumptions and conclusions emerging uniformly as simultaneous
  statements.



  FIXME

\glossary{Proposition}{A \seeglossary{term} of \seeglossary{type}
\isa{prop}.  Internally, there is nothing special about
propositions apart from their type, but the concrete syntax enforces a
clear distinction.  Propositions are structured via implication \isa{A\ {\isasymLongrightarrow}\ B} or universal quantification \isa{{\isasymAnd}x{\isachardot}\ B\ x} --- anything
else is considered atomic.  The canonical form for propositions is
that of a \seeglossary{Hereditary Harrop Formula}.}

\glossary{Theorem}{A proven proposition within a certain theory and
proof context, formally \isa{{\isasymGamma}\ {\isasymturnstile}\isactrlsub {\isasymTheta}\ {\isasymphi}}; both contexts are
rarely spelled out explicitly.  Theorems are usually normalized
according to the \seeglossary{HHF} format.}

\glossary{Fact}{Sometimes used interchangably for
\seeglossary{theorem}.  Strictly speaking, a list of theorems,
essentially an extra-logical conjunction.  Facts emerge either as
local assumptions, or as results of local goal statements --- both may
be simultaneous, hence the list representation.}

\glossary{Schematic variable}{FIXME}

\glossary{Fixed variable}{A variable that is bound within a certain
proof context; an arbitrary-but-fixed entity within a portion of proof
text.}

\glossary{Free variable}{Synonymous for \seeglossary{fixed variable}.}

\glossary{Bound variable}{FIXME}

\glossary{Variable}{See \seeglossary{schematic variable},
\seeglossary{fixed variable}, \seeglossary{bound variable}, or
\seeglossary{type variable}.  The distinguishing feature of different
variables is their binding scope.}


  \[
  \infer[\isa{{\isacharparenleft}axiom{\isacharparenright}}]{\isa{{\isasymturnstile}\ A}}{\isa{A\ {\isasymin}\ {\isasymTheta}}}
  \qquad
  \infer[\isa{{\isacharparenleft}assume{\isacharparenright}}]{\isa{A\ {\isasymturnstile}\ A}}{}
  \]
  \[
  \infer[\isa{{\isacharparenleft}{\isasymAnd}{\isacharunderscore}intro{\isacharparenright}}]{\isa{{\isasymGamma}\ {\isasymturnstile}\ {\isasymAnd}x{\isachardot}\ b\ x}}{\isa{{\isasymGamma}\ {\isasymturnstile}\ b\ x} & \isa{x\ {\isasymnotin}\ {\isasymGamma}}}
  \qquad
  \infer[\isa{{\isacharparenleft}{\isasymAnd}{\isacharunderscore}elim{\isacharparenright}}]{\isa{{\isasymGamma}\ {\isasymturnstile}\ b\ a}}{\isa{{\isasymGamma}\ {\isasymturnstile}\ {\isasymAnd}x{\isachardot}\ b\ x}}
  \]
  \[
  \infer[\isa{{\isacharparenleft}{\isasymLongrightarrow}{\isacharunderscore}intro{\isacharparenright}}]{\isa{{\isasymGamma}\ {\isacharminus}\ A\ {\isasymturnstile}\ A\ {\isasymLongrightarrow}\ B}}{\isa{{\isasymGamma}\ {\isasymturnstile}\ B}}
  \qquad
  \infer[\isa{{\isacharparenleft}{\isasymLongrightarrow}{\isacharunderscore}elim{\isacharparenright}}]{\isa{{\isasymGamma}\isactrlsub {\isadigit{1}}\ {\isasymunion}\ {\isasymGamma}\isactrlsub {\isadigit{2}}\ {\isasymturnstile}\ B}}{\isa{{\isasymGamma}\isactrlsub {\isadigit{1}}\ {\isasymturnstile}\ A\ {\isasymLongrightarrow}\ B} & \isa{{\isasymGamma}\isactrlsub {\isadigit{2}}\ {\isasymturnstile}\ A}}
  \]


  Admissible rules:
  \[
  \infer[\isa{{\isacharparenleft}generalize{\isacharunderscore}type{\isacharparenright}}]{\isa{{\isasymGamma}\ {\isasymturnstile}\ B{\isacharbrackleft}{\isacharquery}{\isasymalpha}{\isacharbrackright}}}{\isa{{\isasymGamma}\ {\isasymturnstile}\ B{\isacharbrackleft}{\isasymalpha}{\isacharbrackright}} & \isa{{\isasymalpha}\ {\isasymnotin}\ {\isasymGamma}}}
  \qquad
  \infer[\isa{{\isacharparenleft}generalize{\isacharunderscore}term{\isacharparenright}}]{\isa{{\isasymGamma}\ {\isasymturnstile}\ B{\isacharbrackleft}{\isacharquery}x{\isacharbrackright}}}{\isa{{\isasymGamma}\ {\isasymturnstile}\ B{\isacharbrackleft}x{\isacharbrackright}} & \isa{x\ {\isasymnotin}\ {\isasymGamma}}}
  \]
  \[
  \infer[\isa{{\isacharparenleft}instantiate{\isacharunderscore}type{\isacharparenright}}]{\isa{{\isasymGamma}\ {\isasymturnstile}\ B{\isacharbrackleft}{\isasymtau}{\isacharbrackright}}}{\isa{{\isasymGamma}\ {\isasymturnstile}\ B{\isacharbrackleft}{\isacharquery}{\isasymalpha}{\isacharbrackright}}}
  \qquad
  \infer[\isa{{\isacharparenleft}instantiate{\isacharunderscore}term{\isacharparenright}}]{\isa{{\isasymGamma}\ {\isasymturnstile}\ B{\isacharbrackleft}t{\isacharbrackright}}}{\isa{{\isasymGamma}\ {\isasymturnstile}\ B{\isacharbrackleft}{\isacharquery}x{\isacharbrackright}}}
  \]

  Note that \isa{instantiate{\isacharunderscore}term} could be derived using \isa{{\isasymAnd}{\isacharunderscore}intro{\isacharslash}elim}, but this is not how it is implemented.  The type
  instantiation rule is a genuine admissible one, due to the lack of true
  polymorphism in the logic.


  Equality and logical equivalence:

  \smallskip
  \begin{tabular}{ll}
  \isa{{\isasymequiv}\ {\isacharcolon}{\isacharcolon}\ {\isasymalpha}\ {\isasymRightarrow}\ {\isasymalpha}\ {\isasymRightarrow}\ prop} & equality relation (infix) \\
  \isa{{\isasymturnstile}\ x\ {\isasymequiv}\ x} & reflexivity law \\
  \isa{{\isasymturnstile}\ x\ {\isasymequiv}\ y\ {\isasymLongrightarrow}\ P\ x\ {\isasymLongrightarrow}\ P\ y} & substitution law \\
  \isa{{\isasymturnstile}\ {\isacharparenleft}{\isasymAnd}x{\isachardot}\ f\ x\ {\isasymequiv}\ g\ x{\isacharparenright}\ {\isasymLongrightarrow}\ f\ {\isasymequiv}\ g} & extensionality \\
  \isa{{\isasymturnstile}\ {\isacharparenleft}A\ {\isasymLongrightarrow}\ B{\isacharparenright}\ {\isasymLongrightarrow}\ {\isacharparenleft}B\ {\isasymLongrightarrow}\ A{\isacharparenright}\ {\isasymLongrightarrow}\ A\ {\isasymequiv}\ B} & coincidence with equivalence \\
  \end{tabular}
  \smallskip%
\end{isamarkuptext}%
\isamarkuptrue%
%
\isamarkupsection{Rules \label{sec:rules}%
}
\isamarkuptrue%
%
\begin{isamarkuptext}%
FIXME

  A \emph{rule} is any Pure theorem in HHF normal form; there is a
  separate calculus for rule composition, which is modeled after
  Gentzen's Natural Deduction \cite{Gentzen:1935}, but allows
  rules to be nested arbitrarily, similar to \cite{extensions91}.

  Normally, all theorems accessible to the user are proper rules.
  Low-level inferences are occasional required internally, but the
  result should be always presented in canonical form.  The higher
  interfaces of Isabelle/Isar will always produce proper rules.  It is
  important to maintain this invariant in add-on applications!

  There are two main principles of rule composition: \isa{resolution} (i.e.\ backchaining of rules) and \isa{by{\isacharminus}assumption} (i.e.\ closing a branch); both principles are
  combined in the variants of \isa{elim{\isacharminus}resosultion} and \isa{dest{\isacharminus}resolution}.  Raw \isa{composition} is occasionally
  useful as well, also it is strictly speaking outside of the proper
  rule calculus.

  Rules are treated modulo general higher-order unification, which is
  unification modulo the equational theory of \isa{{\isasymalpha}{\isasymbeta}{\isasymeta}}-conversion
  on \isa{{\isasymlambda}}-terms.  Moreover, propositions are understood modulo
  the (derived) equivalence \isa{{\isacharparenleft}A\ {\isasymLongrightarrow}\ {\isacharparenleft}{\isasymAnd}x{\isachardot}\ B\ x{\isacharparenright}{\isacharparenright}\ {\isasymequiv}\ {\isacharparenleft}{\isasymAnd}x{\isachardot}\ A\ {\isasymLongrightarrow}\ B\ x{\isacharparenright}}.

  This means that any operations within the rule calculus may be
  subject to spontaneous \isa{{\isasymalpha}{\isasymbeta}{\isasymeta}}-HHF conversions.  It is common
  practice not to contract or expand unnecessarily.  Some mechanisms
  prefer an one form, others the opposite, so there is a potential
  danger to produce some oscillation!

  Only few operations really work \emph{modulo} HHF conversion, but
  expect a normal form: quantifiers \isa{{\isasymAnd}} before implications
  \isa{{\isasymLongrightarrow}} at each level of nesting.

\glossary{Hereditary Harrop Formula}{The set of propositions in HHF
format is defined inductively as \isa{H\ {\isacharequal}\ {\isacharparenleft}{\isasymAnd}x\isactrlsup {\isacharasterisk}{\isachardot}\ H\isactrlsup {\isacharasterisk}\ {\isasymLongrightarrow}\ A{\isacharparenright}}, for variables \isa{x} and atomic propositions \isa{A}.
Any proposition may be put into HHF form by normalizing with the rule
\isa{{\isacharparenleft}A\ {\isasymLongrightarrow}\ {\isacharparenleft}{\isasymAnd}x{\isachardot}\ B\ x{\isacharparenright}{\isacharparenright}\ {\isasymequiv}\ {\isacharparenleft}{\isasymAnd}x{\isachardot}\ A\ {\isasymLongrightarrow}\ B\ x{\isacharparenright}}.  In Isabelle, the outermost
quantifier prefix is represented via \seeglossary{schematic
variables}, such that the top-level structure is merely that of a
\seeglossary{Horn Clause}}.

\glossary{HHF}{See \seeglossary{Hereditary Harrop Formula}.}


  \[
  \infer[\isa{{\isacharparenleft}assumption{\isacharparenright}}]{\isa{C{\isasymvartheta}}}
  {\isa{{\isacharparenleft}{\isasymAnd}\isactrlvec x{\isachardot}\ \isactrlvec H\ \isactrlvec x\ {\isasymLongrightarrow}\ A\ \isactrlvec x{\isacharparenright}\ {\isasymLongrightarrow}\ C} & \isa{A{\isasymvartheta}\ {\isacharequal}\ H\isactrlsub i{\isasymvartheta}}~~\text{(for some~\isa{i})}}
  \]


  \[
  \infer[\isa{{\isacharparenleft}compose{\isacharparenright}}]{\isa{\isactrlvec A{\isasymvartheta}\ {\isasymLongrightarrow}\ C{\isasymvartheta}}}
  {\isa{\isactrlvec A\ {\isasymLongrightarrow}\ B} & \isa{B{\isacharprime}\ {\isasymLongrightarrow}\ C} & \isa{B{\isasymvartheta}\ {\isacharequal}\ B{\isacharprime}{\isasymvartheta}}}
  \]


  \[
  \infer[\isa{{\isacharparenleft}{\isasymAnd}{\isacharunderscore}lift{\isacharparenright}}]{\isa{{\isacharparenleft}{\isasymAnd}\isactrlvec x{\isachardot}\ \isactrlvec A\ {\isacharparenleft}{\isacharquery}\isactrlvec a\ \isactrlvec x{\isacharparenright}{\isacharparenright}\ {\isasymLongrightarrow}\ {\isacharparenleft}{\isasymAnd}\isactrlvec x{\isachardot}\ B\ {\isacharparenleft}{\isacharquery}\isactrlvec a\ \isactrlvec x{\isacharparenright}{\isacharparenright}}}{\isa{\isactrlvec A\ {\isacharquery}\isactrlvec a\ {\isasymLongrightarrow}\ B\ {\isacharquery}\isactrlvec a}}
  \]
  \[
  \infer[\isa{{\isacharparenleft}{\isasymLongrightarrow}{\isacharunderscore}lift{\isacharparenright}}]{\isa{{\isacharparenleft}\isactrlvec H\ {\isasymLongrightarrow}\ \isactrlvec A{\isacharparenright}\ {\isasymLongrightarrow}\ {\isacharparenleft}\isactrlvec H\ {\isasymLongrightarrow}\ B{\isacharparenright}}}{\isa{\isactrlvec A\ {\isasymLongrightarrow}\ B}}
  \]

  The \isa{resolve} scheme is now acquired from \isa{{\isasymAnd}{\isacharunderscore}lift},
  \isa{{\isasymLongrightarrow}{\isacharunderscore}lift}, and \isa{compose}.

  \[
  \infer[\isa{{\isacharparenleft}resolution{\isacharparenright}}]
  {\isa{{\isacharparenleft}{\isasymAnd}\isactrlvec x{\isachardot}\ \isactrlvec H\ \isactrlvec x\ {\isasymLongrightarrow}\ \isactrlvec A\ {\isacharparenleft}{\isacharquery}\isactrlvec a\ \isactrlvec x{\isacharparenright}{\isacharparenright}{\isasymvartheta}\ {\isasymLongrightarrow}\ C{\isasymvartheta}}}
  {\begin{tabular}{l}
    \isa{\isactrlvec A\ {\isacharquery}\isactrlvec a\ {\isasymLongrightarrow}\ B\ {\isacharquery}\isactrlvec a} \\
    \isa{{\isacharparenleft}{\isasymAnd}\isactrlvec x{\isachardot}\ \isactrlvec H\ \isactrlvec x\ {\isasymLongrightarrow}\ B{\isacharprime}\ \isactrlvec x{\isacharparenright}\ {\isasymLongrightarrow}\ C} \\
    \isa{{\isacharparenleft}{\isasymlambda}\isactrlvec x{\isachardot}\ B\ {\isacharparenleft}{\isacharquery}\isactrlvec a\ \isactrlvec x{\isacharparenright}{\isacharparenright}{\isasymvartheta}\ {\isacharequal}\ B{\isacharprime}{\isasymvartheta}} \\
   \end{tabular}}
  \]


  FIXME \isa{elim{\isacharunderscore}resolution}, \isa{dest{\isacharunderscore}resolution}%
\end{isamarkuptext}%
\isamarkuptrue%
%
\isadelimtheory
%
\endisadelimtheory
%
\isatagtheory
\isacommand{end}\isamarkupfalse%
%
\endisatagtheory
{\isafoldtheory}%
%
\isadelimtheory
%
\endisadelimtheory
\isanewline
\end{isabellebody}%
%%% Local Variables:
%%% mode: latex
%%% TeX-master: "root"
%%% End:


\chapter{Tactics} \label{tactics}
\index{tactics|(}

\section{Other basic tactics}

\subsection{Inserting premises and facts}\label{cut_facts_tac}
\index{tactics!for inserting facts}\index{assumptions!inserting}
\begin{ttbox} 
cut_facts_tac : thm list -> int -> tactic
\end{ttbox}
These tactics add assumptions to a subgoal.
\begin{ttdescription}
\item[\ttindexbold{cut_facts_tac} {\it thms} {\it i}] 
  adds the {\it thms} as new assumptions to subgoal~$i$.  Once they have
  been inserted as assumptions, they become subject to tactics such as {\tt
    eresolve_tac} and {\tt rewrite_goals_tac}.  Only rules with no premises
  are inserted: Isabelle cannot use assumptions that contain $\Imp$
  or~$\Forall$.  Sometimes the theorems are premises of a rule being
  derived, returned by~{\tt goal}; instead of calling this tactic, you
  could state the goal with an outermost meta-quantifier.

\end{ttdescription}


\subsection{Composition: resolution without lifting}
\index{tactics!for composition}
\begin{ttbox}
compose_tac: (bool * thm * int) -> int -> tactic
\end{ttbox}
{\bf Composing} two rules means resolving them without prior lifting or
renaming of unknowns.  This low-level operation, which underlies the
resolution tactics, may occasionally be useful for special effects.
A typical application is \ttindex{res_inst_tac}, which lifts and instantiates a
rule, then passes the result to {\tt compose_tac}.
\begin{ttdescription}
\item[\ttindexbold{compose_tac} ($flag$, $rule$, $m$) $i$] 
refines subgoal~$i$ using $rule$, without lifting.  The $rule$ is taken to
have the form $\List{\psi@1; \ldots; \psi@m} \Imp \psi$, where $\psi$ need
not be atomic; thus $m$ determines the number of new subgoals.  If
$flag$ is {\tt true} then it performs elim-resolution --- it solves the
first premise of~$rule$ by assumption and deletes that assumption.
\end{ttdescription}


\section{*Managing lots of rules}
These operations are not intended for interactive use.  They are concerned
with the processing of large numbers of rules in automatic proof
strategies.  Higher-order resolution involving a long list of rules is
slow.  Filtering techniques can shorten the list of rules given to
resolution, and can also detect whether a subgoal is too flexible,
with too many rules applicable.

\subsection{Combined resolution and elim-resolution} \label{biresolve_tac}
\index{tactics!resolution}
\begin{ttbox} 
biresolve_tac   : (bool*thm)list -> int -> tactic
bimatch_tac     : (bool*thm)list -> int -> tactic
subgoals_of_brl : bool*thm -> int
lessb           : (bool*thm) * (bool*thm) -> bool
\end{ttbox}
{\bf Bi-resolution} takes a list of $\it (flag,rule)$ pairs.  For each
pair, it applies resolution if the flag is~{\tt false} and
elim-resolution if the flag is~{\tt true}.  A single tactic call handles a
mixture of introduction and elimination rules.

\begin{ttdescription}
\item[\ttindexbold{biresolve_tac} {\it brls} {\it i}] 
refines the proof state by resolution or elim-resolution on each rule, as
indicated by its flag.  It affects subgoal~$i$ of the proof state.

\item[\ttindexbold{bimatch_tac}] 
is like {\tt biresolve_tac}, but performs matching: unknowns in the
proof state are never updated (see~{\S}\ref{match_tac}).

\item[\ttindexbold{subgoals_of_brl}({\it flag},{\it rule})] 
returns the number of new subgoals that bi-res\-o\-lu\-tion would yield for the
pair (if applied to a suitable subgoal).  This is $n$ if the flag is
{\tt false} and $n-1$ if the flag is {\tt true}, where $n$ is the number
of premises of the rule.  Elim-resolution yields one fewer subgoal than
ordinary resolution because it solves the major premise by assumption.

\item[\ttindexbold{lessb} ({\it brl1},{\it brl2})] 
returns the result of 
\begin{ttbox}
subgoals_of_brl{\it brl1} < subgoals_of_brl{\it brl2}
\end{ttbox}
\end{ttdescription}
Note that \hbox{\tt sort lessb {\it brls}} sorts a list of $\it
(flag,rule)$ pairs by the number of new subgoals they will yield.  Thus,
those that yield the fewest subgoals should be tried first.


\subsection{Discrimination nets for fast resolution}\label{filt_resolve_tac}
\index{discrimination nets|bold}
\index{tactics!resolution}
\begin{ttbox} 
net_resolve_tac  : thm list -> int -> tactic
net_match_tac    : thm list -> int -> tactic
net_biresolve_tac: (bool*thm) list -> int -> tactic
net_bimatch_tac  : (bool*thm) list -> int -> tactic
filt_resolve_tac : thm list -> int -> int -> tactic
could_unify      : term*term->bool
filter_thms      : (term*term->bool) -> int*term*thm list -> thm{\ts}list
\end{ttbox}
The module {\tt Net} implements a discrimination net data structure for
fast selection of rules \cite[Chapter 14]{charniak80}.  A term is
classified by the symbol list obtained by flattening it in preorder.
The flattening takes account of function applications, constants, and free
and bound variables; it identifies all unknowns and also regards
\index{lambda abs@$\lambda$-abstractions}
$\lambda$-abstractions as unknowns, since they could $\eta$-contract to
anything.  

A discrimination net serves as a polymorphic dictionary indexed by terms.
The module provides various functions for inserting and removing items from
nets.  It provides functions for returning all items whose term could match
or unify with a target term.  The matching and unification tests are
overly lax (due to the identifications mentioned above) but they serve as
useful filters.

A net can store introduction rules indexed by their conclusion, and
elimination rules indexed by their major premise.  Isabelle provides
several functions for `compiling' long lists of rules into fast
resolution tactics.  When supplied with a list of theorems, these functions
build a discrimination net; the net is used when the tactic is applied to a
goal.  To avoid repeatedly constructing the nets, use currying: bind the
resulting tactics to \ML{} identifiers.

\begin{ttdescription}
\item[\ttindexbold{net_resolve_tac} {\it thms}] 
builds a discrimination net to obtain the effect of a similar call to {\tt
resolve_tac}.

\item[\ttindexbold{net_match_tac} {\it thms}] 
builds a discrimination net to obtain the effect of a similar call to {\tt
match_tac}.

\item[\ttindexbold{net_biresolve_tac} {\it brls}] 
builds a discrimination net to obtain the effect of a similar call to {\tt
biresolve_tac}.

\item[\ttindexbold{net_bimatch_tac} {\it brls}] 
builds a discrimination net to obtain the effect of a similar call to {\tt
bimatch_tac}.

\item[\ttindexbold{filt_resolve_tac} {\it thms} {\it maxr} {\it i}] 
uses discrimination nets to extract the {\it thms} that are applicable to
subgoal~$i$.  If more than {\it maxr\/} theorems are applicable then the
tactic fails.  Otherwise it calls {\tt resolve_tac}.  

This tactic helps avoid runaway instantiation of unknowns, for example in
type inference.

\item[\ttindexbold{could_unify} ({\it t},{\it u})] 
returns {\tt false} if~$t$ and~$u$ are `obviously' non-unifiable, and
otherwise returns~{\tt true}.  It assumes all variables are distinct,
reporting that {\tt ?a=?a} may unify with {\tt 0=1}.

\item[\ttindexbold{filter_thms} $could\; (limit,prem,thms)$] 
returns the list of potentially resolvable rules (in {\it thms\/}) for the
subgoal {\it prem}, using the predicate {\it could\/} to compare the
conclusion of the subgoal with the conclusion of each rule.  The resulting list
is no longer than {\it limit}.
\end{ttdescription}

\index{tactics|)}


%%% Local Variables: 
%%% mode: latex
%%% TeX-master: "ref"
%%% End: 

%
\begin{isabellebody}%
\def\isabellecontext{proof}%
%
\isadelimtheory
\isanewline
\isanewline
\isanewline
%
\endisadelimtheory
%
\isatagtheory
\isacommand{theory}\isamarkupfalse%
\ {\isachardoublequoteopen}proof{\isachardoublequoteclose}\ \isakeyword{imports}\ base\ \isakeyword{begin}%
\endisatagtheory
{\isafoldtheory}%
%
\isadelimtheory
%
\endisadelimtheory
%
\isamarkupchapter{Structured reasoning%
}
\isamarkuptrue%
%
\isamarkupsection{Proof context%
}
\isamarkuptrue%
%
\isamarkupsubsection{Local variables%
}
\isamarkuptrue%
%
\begin{isamarkuptext}%
FIXME%
\end{isamarkuptext}%
\isamarkuptrue%
%
\isadelimmlref
%
\endisadelimmlref
%
\isatagmlref
%
\begin{isamarkuptext}%
\begin{mldecls}
  \indexml{Variable.declare-term}\verb|Variable.declare_term: term -> Proof.context -> Proof.context| \\
  \indexml{Variable.add-fixes}\verb|Variable.add_fixes: string list -> Proof.context -> string list * Proof.context| \\
  \indexml{Variable.import}\verb|Variable.import: bool -> thm list -> Proof.context -> ((ctyp list * cterm list) * thm list) * Proof.context| \\
  \indexml{Variable.export}\verb|Variable.export: Proof.context -> Proof.context -> thm list -> thm list| \\
  \indexml{Variable.trade}\verb|Variable.trade: Proof.context -> (thm list -> thm list) -> thm list -> thm list| \\
  \indexml{Variable.polymorphic}\verb|Variable.polymorphic: Proof.context -> term list -> term list| \\
  \end{mldecls}

  \begin{description}

  \item \verb|Variable.declare_term|~\isa{t\ ctxt} declares term
  \isa{t} to belong to the context.  This fixes free type
  variables, but not term variables.  Constraints for type and term
  variables are declared uniformly.

  \item \verb|Variable.add_fixes|~\isa{xs\ ctxt} fixes term
  variables \isa{xs} and returns the internal names of the
  resulting Skolem constants.  Note that term fixes refer to
  \emph{all} type instances that may occur in the future.

  \item \verb|Variable.invent_fixes| is similar to \verb|Variable.add_fixes|, but the given names merely act as hints for
  internal fixes produced here.

  \item \verb|Variable.import|~\isa{open\ ths\ ctxt} augments the
  context by new fixes for the schematic type and term variables
  occurring in \isa{ths}.  The \isa{open} flag indicates
  whether the fixed names should be accessible to the user, otherwise
  internal names are chosen.

  \item \verb|Variable.export|~\isa{inner\ outer\ ths} generalizes
  fixed type and term variables in \isa{ths} according to the
  difference of the \isa{inner} and \isa{outer} context.  Note
  that type variables occurring in term variables are still fixed.

  \verb|Variable.export| essentially reverses the effect of \verb|Variable.import| (up to renaming of schematic variables.

  \item \verb|Variable.trade| composes \verb|Variable.import| and \verb|Variable.export|, i.e.\ it provides a view on facts with all
  variables being fixed in the current context.

  \item \verb|Variable.polymorphic|~\isa{ctxt\ ts} generalizes type
  variables in \isa{ts} as far as possible, even those occurring
  in fixed term variables.  This operation essentially reverses the
  default policy of type-inference to introduce local polymorphism as
  fixed types.

  \end{description}%
\end{isamarkuptext}%
\isamarkuptrue%
%
\endisatagmlref
{\isafoldmlref}%
%
\isadelimmlref
%
\endisadelimmlref
%
\begin{isamarkuptext}%
FIXME%
\end{isamarkuptext}%
\isamarkuptrue%
%
\isamarkupsection{Proof state%
}
\isamarkuptrue%
%
\begin{isamarkuptext}%
FIXME

\glossary{Proof state}{The whole configuration of a structured proof,
consisting of a \seeglossary{proof context} and an optional
\seeglossary{structured goal}.  Internally, an Isar proof state is
organized as a stack to accomodate block structure of proof texts.
For historical reasons, a low-level \seeglossary{tactical goal} is
occasionally called ``proof state'' as well.}

\glossary{Structured goal}{FIXME}

\glossary{Goal}{See \seeglossary{tactical goal} or \seeglossary{structured goal}. \norefpage}%
\end{isamarkuptext}%
\isamarkuptrue%
%
\isamarkupsection{Methods%
}
\isamarkuptrue%
%
\begin{isamarkuptext}%
FIXME%
\end{isamarkuptext}%
\isamarkuptrue%
%
\isamarkupsection{Attributes%
}
\isamarkuptrue%
%
\begin{isamarkuptext}%
FIXME%
\end{isamarkuptext}%
\isamarkuptrue%
%
\isadelimtheory
%
\endisadelimtheory
%
\isatagtheory
\isacommand{end}\isamarkupfalse%
%
\endisatagtheory
{\isafoldtheory}%
%
\isadelimtheory
%
\endisadelimtheory
\isanewline
\end{isabellebody}%
%%% Local Variables:
%%% mode: latex
%%% TeX-master: "root"
%%% End:

%
\begin{isabellebody}%
\def\isabellecontext{locale}%
%
\isadelimtheory
\isanewline
\isanewline
\isanewline
%
\endisadelimtheory
%
\isatagtheory
\isacommand{theory}\isamarkupfalse%
\ {\isachardoublequoteopen}locale{\isachardoublequoteclose}\ \isakeyword{imports}\ base\ \isakeyword{begin}%
\endisatagtheory
{\isafoldtheory}%
%
\isadelimtheory
%
\endisadelimtheory
%
\isamarkupchapter{Structured specifications%
}
\isamarkuptrue%
%
\isamarkupsection{Specification elements%
}
\isamarkuptrue%
%
\begin{isamarkuptext}%
FIXME%
\end{isamarkuptext}%
\isamarkuptrue%
%
\isamarkupsection{Type-inference%
}
\isamarkuptrue%
%
\begin{isamarkuptext}%
FIXME%
\end{isamarkuptext}%
\isamarkuptrue%
%
\isamarkupsection{Local theories%
}
\isamarkuptrue%
%
\begin{isamarkuptext}%
FIXME

  \glossary{Local theory}{FIXME}%
\end{isamarkuptext}%
\isamarkuptrue%
%
\isadelimtheory
%
\endisadelimtheory
%
\isatagtheory
\isacommand{end}\isamarkupfalse%
%
\endisatagtheory
{\isafoldtheory}%
%
\isadelimtheory
%
\endisadelimtheory
\isanewline
\end{isabellebody}%
%%% Local Variables:
%%% mode: latex
%%% TeX-master: "root"
%%% End:

%
\begin{isabellebody}%
\def\isabellecontext{integration}%
%
\isadelimtheory
\isanewline
\isanewline
\isanewline
%
\endisadelimtheory
%
\isatagtheory
\isacommand{theory}\isamarkupfalse%
\ integration\ \isakeyword{imports}\ base\ \isakeyword{begin}%
\endisatagtheory
{\isafoldtheory}%
%
\isadelimtheory
%
\endisadelimtheory
%
\isamarkupchapter{System integration%
}
\isamarkuptrue%
%
\isamarkupsection{Isar toplevel%
}
\isamarkuptrue%
%
\begin{isamarkuptext}%
The Isar toplevel may be considered the centeral hub of the
  Isabelle/Isar system, where all key components and sub-systems are
  integrated into a single read-eval-print loop of Isar commands.
  Here we even incorporate the existing {\ML} toplevel of the compiler
  and run-time system (cf.\ \secref{sec:ML-toplevel}).

  Isabelle/Isar departs from original ``LCF system architecture''
  where {\ML} was really The Meta Language for defining theories and
  conducting proofs.  Instead, {\ML} merely serves as the
  implementation language for the system (and user extensions), while
  our specific Isar toplevel supports particular notions of
  incremental theory and proof development more directly.  This
  includes the graph structure of theories and the block structure of
  proofs, support for unlimited undo, facilities for tracing,
  debugging, timing, profiling.

  \medskip The toplevel maintains an implicit state, which is
  transformed by a sequence of transitions -- either interactively or
  in batch-mode.  In interactive mode, Isar state transitions are
  encapsulated as safe transactions, such that both failure and undo
  are handled conveniently without destroying the underlying draft
  theory (cf.~\secref{sec:context-theory}).  In batch mode,
  transitions operate in a strictly linear (destructive) fashion, such
  that error conditions abort the present attempt to construct a
  theory altogether.

  The toplevel state is a disjoint sum of empty \isa{toplevel}, or
  \isa{theory}, or \isa{proof}.  On entering the main Isar loop we
  start with an empty toplevel.  A theory is commenced by giving a
  \isa{{\isasymTHEORY}} header; within a theory we may issue theory
  commands such as \isa{{\isasymCONSTS}} or \isa{{\isasymDEFS}}, or state a
  \isa{{\isasymTHEOREM}} to be proven.  Now we are within a proof state,
  with a rich collection of Isar proof commands for structured proof
  composition, or unstructured proof scripts.  When the proof is
  concluded we get back to the theory, which is then updated by
  storing the resulting fact.  Further theory declarations or theorem
  statements with proofs may follow, until we eventually conclude the
  theory development by issuing \isa{{\isasymEND}}.  The resulting theory
  is then stored within the theory database and we are back to the
  empty toplevel.

  In addition to these proper state transformations, there are also
  some diagnostic commands for peeking at the toplevel state without
  modifying it (e.g.\ \isakeyword{thm}, \isakeyword{term},
  \isakeyword{print-cases}).%
\end{isamarkuptext}%
\isamarkuptrue%
%
\isadelimmlref
%
\endisadelimmlref
%
\isatagmlref
%
\begin{isamarkuptext}%
\begin{mldecls}
  \indexmltype{Toplevel.state}\verb|type Toplevel.state| \\
  \indexml{Toplevel.UNDEF}\verb|Toplevel.UNDEF: exn| \\
  \indexml{Toplevel.is-toplevel}\verb|Toplevel.is_toplevel: Toplevel.state -> bool| \\
  \indexml{Toplevel.theory-of}\verb|Toplevel.theory_of: Toplevel.state -> theory| \\
  \indexml{Toplevel.proof-of}\verb|Toplevel.proof_of: Toplevel.state -> Proof.state| \\
  \indexml{Toplevel.debug}\verb|Toplevel.debug: bool ref| \\
  \indexml{Toplevel.timing}\verb|Toplevel.timing: bool ref| \\
  \indexml{Toplevel.profiling}\verb|Toplevel.profiling: int ref| \\
  \end{mldecls}

  \begin{description}

  \item \verb|Toplevel.state| represents Isar toplevel states,
  which are normally only manipulated through the toplevel transition
  concept (\secref{sec:toplevel-transition}).  Also note that a
  toplevel state is subject to the same linerarity restrictions as a
  theory context (cf.~\secref{sec:context-theory}).

  \item \verb|Toplevel.UNDEF| is raised for undefined toplevel
  operations: \verb|Toplevel.state| is a sum type, many operations
  work only partially for certain cases.

  \item \verb|Toplevel.is_toplevel| checks for an empty toplevel state.

  \item \verb|Toplevel.theory_of| gets the theory of a theory or proof
  (!), otherwise raises \verb|Toplevel.UNDEF|.

  \item \verb|Toplevel.proof_of| gets the Isar proof state if
  available, otherwise raises \verb|Toplevel.UNDEF|.

  \item \verb|set Toplevel.debug| makes the toplevel print further
  details about internal error conditions, exceptions being raised
  etc.

  \item \verb|set Toplevel.timing| makes the toplevel print timing
  information for each Isar command being executed.

  \item \verb|Toplevel.profiling| controls low-level profiling of the
  underlying {\ML} runtime system.\footnote{For Poly/ML, 1 means time
  and 2 space profiling.}

  \end{description}%
\end{isamarkuptext}%
\isamarkuptrue%
%
\endisatagmlref
{\isafoldmlref}%
%
\isadelimmlref
%
\endisadelimmlref
%
\isamarkupsubsection{Toplevel transitions%
}
\isamarkuptrue%
%
\begin{isamarkuptext}%
An Isar toplevel transition consists of a partial
  function on the toplevel state, with additional information for
  diagnostics and error reporting: there are fields for command name,
  source position, optional source text, as well as flags for
  interactive-only commands (which issue a warning in batch-mode),
  printing of result state, etc.

  The operational part is represented as a sequential union of a list
  of partial functions, which are tried in turn until the first one
  succeeds (i.e.\ does not raise \verb|Toplevel.UNDEF|).  For example,
  a single Isar command like \isacommand{qed} consists of the union of
  some function \verb|Proof.state -> Proof.state| for proofs
  within proofs, plus \verb|Proof.state -> theory| for proofs at
  the outer theory level.

  Toplevel transitions are composed via transition transformers.
  Internally, Isar commands are put together from an empty transition
  extended by name and source position (and optional source text).  It
  is then left to the individual command parser to turn the given
  syntax body into a suitable transition transformer that adjoin
  actual operations on a theory or proof state etc.%
\end{isamarkuptext}%
\isamarkuptrue%
%
\isadelimmlref
%
\endisadelimmlref
%
\isatagmlref
%
\begin{isamarkuptext}%
\begin{mldecls}
  \indexml{Toplevel.print}\verb|Toplevel.print: Toplevel.transition -> Toplevel.transition| \\
  \indexml{Toplevel.no-timing}\verb|Toplevel.no_timing: Toplevel.transition -> Toplevel.transition| \\
  \indexml{Toplevel.keep}\verb|Toplevel.keep: (Toplevel.state -> unit) ->|\isasep\isanewline%
\verb|  Toplevel.transition -> Toplevel.transition| \\
  \indexml{Toplevel.theory}\verb|Toplevel.theory: (theory -> theory) ->|\isasep\isanewline%
\verb|  Toplevel.transition -> Toplevel.transition| \\
  \indexml{Toplevel.theory-to-proof}\verb|Toplevel.theory_to_proof: (theory -> Proof.state) ->|\isasep\isanewline%
\verb|  Toplevel.transition -> Toplevel.transition| \\
  \indexml{Toplevel.proof}\verb|Toplevel.proof: (Proof.state -> Proof.state) ->|\isasep\isanewline%
\verb|  Toplevel.transition -> Toplevel.transition| \\
  \indexml{Toplevel.proofs}\verb|Toplevel.proofs: (Proof.state -> Proof.state Seq.seq) ->|\isasep\isanewline%
\verb|  Toplevel.transition -> Toplevel.transition| \\
  \indexml{Toplevel.proof-to-theory}\verb|Toplevel.proof_to_theory: (Proof.state -> theory) ->|\isasep\isanewline%
\verb|  Toplevel.transition -> Toplevel.transition| \\
  \end{mldecls}

  \begin{description}

  \item \verb|Toplevel.print| sets the print flag, which causes the
  resulting state of the transition to be echoed in interactive mode.

  \item \verb|Toplevel.no_timing| indicates that the transition should
  never show timing information, e.g.\ because it is merely a
  diagnostic command.

  \item \verb|Toplevel.keep| adjoins a diagnostic function.

  \item \verb|Toplevel.theory| adjoins a theory transformer.

  \item \verb|Toplevel.theory_to_proof| adjoins a global goal function,
  which turns a theory into a proof state.  The theory must not be
  changed here!  The generic Isar goal setup includes an argument that
  specifies how to apply the proven result to the theory, when the
  proof is finished.

  \item \verb|Toplevel.proof| adjoins a deterministic proof command,
  with a singleton result state.

  \item \verb|Toplevel.proofs| adjoins a general proof command, with
  zero or more result states (represented as a lazy list).

  \item \verb|Toplevel.proof_to_theory| adjoins a concluding proof
  command, that returns the resulting theory, after storing the
  resulting facts etc.

  \end{description}%
\end{isamarkuptext}%
\isamarkuptrue%
%
\endisatagmlref
{\isafoldmlref}%
%
\isadelimmlref
%
\endisadelimmlref
%
\isamarkupsubsection{Toplevel control%
}
\isamarkuptrue%
%
\begin{isamarkuptext}%
Apart from regular toplevel transactions there are a few
  special control commands that modify the behavior the toplevel
  itself, and only make sense in interactive mode.  Under normal
  circumstances, the user encounters these only implicitly as part of
  the protocol between the Isabelle/Isar system and a user-interface
  such as ProofGeneral.

  \begin{description}

  \item \isacommand{undo} follows the three-level hierarchy of empty
  toplevel vs.\ theory vs.\ proof: undo within a proof reverts to the
  previous proof context, undo after a proof reverts to the theory
  before the initial goal statement, undo of a theory command reverts
  to the previous theory value, undo of a theory header discontinues
  the current theory development and removes it from the theory
  database (\secref{sec:theory-database}).

  \item \isacommand{kill} aborts the current level of development:
  kill in a proof context reverts to the theory before the initial
  goal statement, kill in a theory context aborts the current theory
  development, removing it from the database.

  \item \isacommand{exit} drops out of the Isar toplevel into the
  underlying {\ML} toplevel (\secref{sec:ML-toplevel}).  The Isar
  toplevel state is preserved and may be continued later.

  \item \isacommand{quit} terminates the Isabelle/Isar process without
  saving.

  \end{description}%
\end{isamarkuptext}%
\isamarkuptrue%
%
\isamarkupsection{ML toplevel \label{sec:ML-toplevel}%
}
\isamarkuptrue%
%
\begin{isamarkuptext}%
The {\ML} toplevel provides a read-compile-eval-print loop for
  {\ML} values, types, structures, and functors.  {\ML} declarations
  operate on the global system state, which consists of the compiler
  environment plus the values of {\ML} reference variables.  There is
  no clean way to undo {\ML} declarations, except for reverting to a
  previously saved state of the whole Isabelle process.  {\ML} input
  is either read interactively from a TTY, or from a string (usually
  within a theory text), or from a source file (usually associated
  with a theory).

  Whenever the {\ML} toplevel is active, the current Isabelle theory
  context is passed as an internal reference variable.  Thus {\ML}
  code may access the theory context during compilation, it may even
  change the value of a theory being under construction --- following
  the usual linearity restrictions (cf.~\secref{sec:context-theory}).%
\end{isamarkuptext}%
\isamarkuptrue%
%
\isadelimmlref
%
\endisadelimmlref
%
\isatagmlref
%
\begin{isamarkuptext}%
\begin{mldecls}
  \indexml{context}\verb|context: theory -> unit| \\
  \indexml{the-context}\verb|the_context: unit -> theory| \\
  \indexml{Context.$>$$>$ }\verb|Context.>> : (theory -> theory) -> unit| \\
  \end{mldecls}

  \begin{description}

  \item \verb|context|~\isa{thy} sets the {\ML} theory context to
  \isa{thy}.  This is usually performed automatically by the system,
  when dropping out of the interactive Isar toplevel into {\ML}, or
  when Isar invokes {\ML} to process code from a string or a file.

  \item \verb|the_context ()| refers to the theory context of the
  {\ML} toplevel --- at compile time!  {\ML} code needs to take care
  to refer to \verb|the_context ()| correctly, recall that evaluation
  of a function body is delayed until actual runtime.  Moreover,
  persistent {\ML} toplevel bindings to an unfinished theory should be
  avoided: code should either project out the desired information
  immediately, or produce an explicit \verb|theory_ref| (cf.\
  \secref{sec:context-theory}).

  \item \verb|Context.>>|~\isa{f} applies theory transformation
  \isa{f} to the current theory of the {\ML} toplevel.  In order to
  work as expected, the theory should be still under construction, and
  the Isar language element that invoked the {\ML} compiler in the
  first place shoule be ready to accept the changed theory value
  (e.g.\ \isakeyword{ML-setup}, but not plain \isakeyword{ML}).
  Otherwise the theory may get destroyed!

  \end{description}

  It is very important to note that the above functions are really
  restricted to the compile time, even though the {\ML} compiler is
  invoked at runtime!  The majority of {\ML} code uses explicit
  functional arguments of a theory or proof context, as required.
  Thus it may get run in an arbitrary context later on.

  \bigskip

  \begin{mldecls}
  \indexml{Isar.main}\verb|Isar.main: unit -> unit| \\
  \indexml{Isar.loop}\verb|Isar.loop: unit -> unit| \\
  \indexml{Isar.state}\verb|Isar.state: unit -> Toplevel.state| \\
  \indexml{Isar.exn}\verb|Isar.exn: unit -> (exn * string) option| \\
  \end{mldecls}

  \begin{description}

  \item \verb|Isar.main ()| invokes the Isar toplevel from {\ML},
  initializing the state to empty toplevel state.

  \item \verb|Isar.loop ()| continues the Isar toplevel with the
  current state, after dropping out of the Isar toplevel loop.

  \item \verb|Isar.state ()| and \verb|Isar.exn ()| get current
  toplevel state and optional error condition, respectively.  This
  only works after dropping out of the Isar toplevel loop.

  \end{description}%
\end{isamarkuptext}%
\isamarkuptrue%
%
\endisatagmlref
{\isafoldmlref}%
%
\isadelimmlref
%
\endisadelimmlref
%
\isamarkupsection{Theory database%
}
\isamarkuptrue%
%
\begin{isamarkuptext}%
The theory database maintains a collection of theories,
  together with some administrative information about the original
  sources, which are held in an external store (i.e.\ a collection of
  directories within the regular file system of the underlying
  platform).

  The theory database is organized as a directed acyclic graph, with
  entries referenced by theory name.  Although some external
  interfaces allow to include a directory specification, this is only
  a hint to the underlying theory loader mechanism: the internal
  theory name space is flat.

  Theory \isa{A} is associated with the main theory file \isa{A}\verb,.thy,, which needs to be accessible through the theory
  loader path.  A number of optional {\ML} source files may be
  associated with each theory, by declaring these dependencies in the
  theory header as \isa{{\isasymUSES}}, and loading them consecutively
  within the theory context.  The system keeps track of incoming {\ML}
  sources and associates them with the current theory.  The special
  theory {\ML} file \isa{A}\verb,.ML, is loaded after a theory has
  been concluded, in order to support legacy proof {\ML} proof
  scripts.

  The basic internal actions of the theory database are \isa{update}, \isa{outdate}, and \isa{remove}:

  \begin{itemize}

  \item \isa{update\ A} introduces a link of \isa{A} with a
  \isa{theory} value of the same name; it asserts that the theory
  sources are consistent with that value.

  \item \isa{outdate\ A} invalidates the link of a theory database
  entry to its sources, but retains the present theory value.

  \item \isa{remove\ A} removes entry \isa{A} from the theory
  database.
  
  \end{itemize}

  These actions are propagated to sub- or super-graphs of a theory
  entry in the usual way, in order to preserve global consistency of
  the state of all loaded theories with the sources of the external
  store.  This implies causal dependencies of certain actions: \isa{update} or \isa{outdate} of an entry will \isa{outdate}
  all descendants; \isa{remove} will \isa{remove} all
  descendants.

  \medskip There are separate user-level interfaces to operate on the
  theory database directly or indirectly.  The primitive actions then
  just happen automatically while working with the system.  In
  particular, processing a theory header \isa{{\isasymTHEORY}\ A\ {\isasymIMPORTS}\ B\isactrlsub {\isadigit{1}}\ {\isasymdots}\ B\isactrlsub n\ {\isasymBEGIN}} ensure that the
  sub-graph of the collective imports \isa{B\isactrlsub {\isadigit{1}}\ {\isasymdots}\ B\isactrlsub n}
  is up-to-date.  Earlier theories are reloaded as required, with
  \isa{update} actions proceeding in topological order according to
  theory dependencies.  There may be also a wave of implied \isa{outdate} actions for derived theory nodes until a stable situation
  is achieved eventually.%
\end{isamarkuptext}%
\isamarkuptrue%
%
\isadelimmlref
%
\endisadelimmlref
%
\isatagmlref
%
\begin{isamarkuptext}%
\begin{mldecls}
  \indexml{theory}\verb|theory: string -> theory| \\
  \indexml{use-thy}\verb|use_thy: string -> unit| \\
  \indexml{update-thy}\verb|update_thy: string -> unit| \\
  \indexml{use-thy-only}\verb|use_thy_only: string -> unit| \\
  \indexml{update-thy-only}\verb|update_thy_only: string -> unit| \\
  \indexml{touch-thy}\verb|touch_thy: string -> unit| \\
  \indexml{remove-thy}\verb|remove_thy: string -> unit| \\[1ex]
  \indexml{ThyInfo.begin-theory}\verb|ThyInfo.begin_theory|\verb|: ... -> bool -> theory| \\
  \indexml{ThyInfo.end-theory}\verb|ThyInfo.end_theory: theory -> theory| \\
  \indexml{ThyInfo.register-theory}\verb|ThyInfo.register_theory: theory -> unit| \\[1ex]
  \verb|datatype action = Update |\verb,|,\verb| Outdate |\verb,|,\verb| Remove| \\
  \indexml{ThyInfo.add-hook}\verb|ThyInfo.add_hook: (ThyInfo.action -> string -> unit) -> unit| \\
  \end{mldecls}

  \begin{description}

  \item \verb|theory|~\isa{A} retrieves the theory value presently
  associated with \isa{A}.  The result is not necessarily
  up-to-date!

  \item \verb|use_thy|~\isa{A} loads theory \isa{A} if it is absent
  or out-of-date.  It ensures that all parent theories are available
  as well, but does not reload them if older versions are already
  present.

  \item \verb|update_thy| is similar to \verb|use_thy|, but ensures that
  the \isa{A} and all of its ancestors are fully up-to-date.

  \item \verb|use_thy_only|~\isa{A} is like \verb|use_thy|~\isa{A},
  but refrains from loading the attached \isa{A}\verb,.ML, file.
  This is occasionally useful in replaying legacy {\ML} proof scripts
  by hand.
  
  \item \verb|update_thy_only| is analogous to \verb|use_thy_only|, but
  proceeds like \verb|update_thy| for ancestors.

  \item \verb|touch_thy|~\isa{A} performs \isa{outdate} action on
  theory \isa{A} and all of its descendants.

  \item \verb|remove_thy|~\isa{A} removes \isa{A} and all of its
  descendants from the theory database.

  \item \verb|ThyInfo.begin_theory| is the basic operation behind a
  \isa{{\isasymTHEORY}} header declaration.  The boolean argument
  indicates the strictness of treating ancestors: for \verb|true| (as
  in interactive mode) like \verb|update_thy|, and for \verb|false| (as
  in batch mode) like \verb|use_thy|.  This is {\ML} functions is
  normally not invoked directly.

  \item \verb|ThyInfo.end_theory| concludes the loading of a theory
  proper; an attached theory {\ML} file may be still loaded later on.
  This is {\ML} functions is normally not invoked directly.

  \item \verb|ThyInfo.register_theory|~{text thy} registers an existing
  theory value with the theory loader database.  There is no
  management of associated sources; this is mainly for bootstrapping.

  \item \verb|ThyInfo.add_hook|~\isa{f} registers function \isa{f} as a hook for theory database actions.  The function will be
  invoked with the action and theory name being involved; thus derived
  actions may be performed in associated system components, e.g.\
  maintaining the state of an editor for theory sources.

  The kind and order of actions occurring in practice depends both on
  user interactions and the internal process of resolving theory
  imports.  Hooks should not rely on a particular policy here!  Any
  exceptions raised by the hook are ignored by the theory database.

  \end{description}%
\end{isamarkuptext}%
\isamarkuptrue%
%
\endisatagmlref
{\isafoldmlref}%
%
\isadelimmlref
%
\endisadelimmlref
%
\isadelimtheory
%
\endisadelimtheory
%
\isatagtheory
\isacommand{end}\isamarkupfalse%
%
\endisatagtheory
{\isafoldtheory}%
%
\isadelimtheory
%
\endisadelimtheory
\isanewline
\end{isabellebody}%
%%% Local Variables:
%%% mode: latex
%%% TeX-master: "root"
%%% End:


% Isabelle was not designed; it evolved.
% Not everyone likes this idea. Specification experts rightly abhor trial-and-error programming.
% They suggest that no one should write a program without First writing a complete
% formal specification. But university departments are not software houses. Programs like
% Isabelle are not products: when they have served their purpose, they are discarded.
%
% L.C. Paulson, ``Isabelle: The Next 700 Theorem Provers''

% As I did 20 years ago, I still fervently believe that the only way to
% make software secure, reliable, and fast is to make it small. Fight
% Features.
%
% Andrew S. Tanenbaum

\appendix
%
\begin{isabellebody}%
\def\isabellecontext{ML}%
%
\isadelimtheory
\isanewline
\isanewline
\isanewline
%
\endisadelimtheory
%
\isatagtheory
\isacommand{theory}\isamarkupfalse%
\ {\isachardoublequoteopen}ML{\isachardoublequoteclose}\ \isakeyword{imports}\ base\ \isakeyword{begin}%
\endisatagtheory
{\isafoldtheory}%
%
\isadelimtheory
%
\endisadelimtheory
%
\isamarkupchapter{Aesthetics of ML programming%
}
\isamarkuptrue%
%
\begin{isamarkuptext}%
FIXME style guide, see also
\url{http://caml.inria.fr/resources/doc/guides/guidelines.en.html} and
\url{http://www.cs.cornell.edu/Courses/cs312/2003sp/handouts/style.htm}%
\end{isamarkuptext}%
\isamarkuptrue%
%
\isamarkupchapter{Basic library functions%
}
\isamarkuptrue%
%
\begin{isamarkuptext}%
FIXME beyond the basis library definition%
\end{isamarkuptext}%
\isamarkuptrue%
%
\isadelimtheory
%
\endisadelimtheory
%
\isatagtheory
\isacommand{end}\isamarkupfalse%
%
\endisatagtheory
{\isafoldtheory}%
%
\isadelimtheory
%
\endisadelimtheory
\isanewline
\end{isabellebody}%
%%% Local Variables:
%%% mode: latex
%%% TeX-master: "root"
%%% End:


\begingroup
\tocentry{\bibname}
\bibliographystyle{plain} \small\raggedright\frenchspacing
\bibliography{../manual}
\endgroup

\tocentry{\glossaryname}
\printglossary

\tocentry{\indexname}
\printindex

\end{document}


%%% Local Variables: 
%%% mode: latex
%%% TeX-master: t
%%% End: 
