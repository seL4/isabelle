
\chapter{Theories, Terms and Types} \label{theories}
\index{theories|(}

\section{Basic operations on theories}\label{BasicOperationsOnTheories}

\subsection{*Theory inclusion}
\begin{ttbox}
transfer    : theory -> thm -> thm
\end{ttbox}

Transferring theorems to super theories has no logical significance,
but may affect some operations in subtle ways (e.g.\ implicit merges
of signatures when applying rules, or pretty printing of theorems).

\begin{ttdescription}

\item[\ttindexbold{transfer} $thy$ $thm$] transfers theorem $thm$ to
  theory $thy$, provided the latter includes the theory of $thm$.
  
\end{ttdescription}


\section{*Variable binding}
\begin{ttbox}
loose_bnos     : term -> int list
incr_boundvars : int -> term -> term
abstract_over  : term*term -> term
variant_abs    : string * typ * term -> string * term
\end{ttbox}
These functions are all concerned with the de Bruijn representation of
bound variables.
\begin{ttdescription}
\item[\ttindexbold{loose_bnos} $t$]
  returns the list of all dangling bound variable references.  In
  particular, \texttt{Bound~0} is loose unless it is enclosed in an
  abstraction.  Similarly \texttt{Bound~1} is loose unless it is enclosed in
  at least two abstractions; if enclosed in just one, the list will contain
  the number 0.  A well-formed term does not contain any loose variables.

\item[\ttindexbold{incr_boundvars} $j$]
  increases a term's dangling bound variables by the offset~$j$.  This is
  required when moving a subterm into a context where it is enclosed by a
  different number of abstractions.  Bound variables with a matching
  abstraction are unaffected.

\item[\ttindexbold{abstract_over} $(v,t)$]
  forms the abstraction of~$t$ over~$v$, which may be any well-formed term.
  It replaces every occurrence of \(v\) by a \texttt{Bound} variable with the
  correct index.

\item[\ttindexbold{variant_abs} $(a,T,u)$]
  substitutes into $u$, which should be the body of an abstraction.
  It replaces each occurrence of the outermost bound variable by a free
  variable.  The free variable has type~$T$ and its name is a variant
  of~$a$ chosen to be distinct from all constants and from all variables
  free in~$u$.

\end{ttdescription}

\index{theories|)}


%%% Local Variables: 
%%% mode: latex
%%% TeX-master: "ref"
%%% End: 
