\chapter{Inductively Defined Sets} \label{chap:inductive}
\index{inductive definitions|(}

This chapter is dedicated to the most important definition principle after
recursive functions and datatypes: inductively defined sets.

We start with a simple example: the set of even numbers.  A slightly more
complicated example, the reflexive transitive closure, is the subject of
{\S}\ref{sec:rtc}. In particular, some standard induction heuristics are
discussed. Advanced forms of inductive definitions are discussed in
{\S}\ref{sec:adv-ind-def}. To demonstrate the versatility of inductive
definitions, the chapter closes with a case study from the realm of
context-free grammars. The first two sections are required reading for anybody
interested in mathematical modelling.

\begin{warn}
Predicates can also be defined inductively.
See {\S}\ref{sec:ind-predicates}.
\end{warn}

%
\begin{isabellebody}%
\def\isabellecontext{Even}%
\isanewline
\isacommand{theory}\ Even\ {\isacharequal}\ Main{\isacharcolon}\isanewline
\isanewline
\isanewline
\isamarkupfalse%
\isacommand{consts}\ even\ {\isacharcolon}{\isacharcolon}\ {\isachardoublequote}nat\ set{\isachardoublequote}\isanewline
\isamarkupfalse%
\isacommand{inductive}\ even\isanewline
\isakeyword{intros}\isanewline
zero{\isacharbrackleft}intro{\isacharbang}{\isacharbrackright}{\isacharcolon}\ {\isachardoublequote}{\isadigit{0}}\ {\isasymin}\ even{\isachardoublequote}\isanewline
step{\isacharbrackleft}intro{\isacharbang}{\isacharbrackright}{\isacharcolon}\ {\isachardoublequote}n\ {\isasymin}\ even\ {\isasymLongrightarrow}\ {\isacharparenleft}Suc\ {\isacharparenleft}Suc\ n{\isacharparenright}{\isacharparenright}\ {\isasymin}\ even{\isachardoublequote}\isamarkupfalse%
%
\begin{isamarkuptext}%
An inductive definition consists of introduction rules. 

\begin{isabelle}%
\ \ \ \ \ n\ {\isasymin}\ even\ {\isasymLongrightarrow}\ Suc\ {\isacharparenleft}Suc\ n{\isacharparenright}\ {\isasymin}\ even%
\end{isabelle}
\rulename{even.step}

\begin{isabelle}%
\ \ \ \ \ {\isasymlbrakk}xa\ {\isasymin}\ even{\isacharsemicolon}\ P\ {\isadigit{0}}{\isacharsemicolon}\ {\isasymAnd}n{\isachardot}\ {\isasymlbrakk}n\ {\isasymin}\ even{\isacharsemicolon}\ P\ n{\isasymrbrakk}\ {\isasymLongrightarrow}\ P\ {\isacharparenleft}Suc\ {\isacharparenleft}Suc\ n{\isacharparenright}{\isacharparenright}{\isasymrbrakk}\ {\isasymLongrightarrow}\ P\ xa%
\end{isabelle}
\rulename{even.induct}

Attributes can be given to the introduction rules.  Here both rules are
specified as \isa{intro!}

Our first lemma states that numbers of the form $2\times k$ are even.%
\end{isamarkuptext}%
\isamarkuptrue%
\isacommand{lemma}\ two{\isacharunderscore}times{\isacharunderscore}even{\isacharbrackleft}intro{\isacharbang}{\isacharbrackright}{\isacharcolon}\ {\isachardoublequote}{\isadigit{2}}{\isacharasterisk}k\ {\isasymin}\ even{\isachardoublequote}\isanewline
\isamarkupfalse%
\isacommand{apply}\ {\isacharparenleft}induct{\isacharunderscore}tac\ k{\isacharparenright}\isamarkupfalse%
%
\begin{isamarkuptxt}%
The first step is induction on the natural number \isa{k}, which leaves
two subgoals:
\begin{isabelle}%
\ {\isadigit{1}}{\isachardot}\ {\isadigit{2}}\ {\isacharasterisk}\ {\isadigit{0}}\ {\isasymin}\ even\isanewline
\ {\isadigit{2}}{\isachardot}\ {\isasymAnd}n{\isachardot}\ {\isadigit{2}}\ {\isacharasterisk}\ n\ {\isasymin}\ even\ {\isasymLongrightarrow}\ {\isadigit{2}}\ {\isacharasterisk}\ Suc\ n\ {\isasymin}\ even%
\end{isabelle}
Here \isa{auto} simplifies both subgoals so that they match the introduction
rules, which then are applied automatically.%
\end{isamarkuptxt}%
\ \isamarkuptrue%
\isacommand{apply}\ auto\isanewline
\isamarkupfalse%
\isacommand{done}\isamarkupfalse%
%
\begin{isamarkuptext}%
Our goal is to prove the equivalence between the traditional definition
of even (using the divides relation) and our inductive definition.  Half of
this equivalence is trivial using the lemma just proved, whose \isa{intro!}
attribute ensures it will be applied automatically.%
\end{isamarkuptext}%
\isamarkuptrue%
\isacommand{lemma}\ dvd{\isacharunderscore}imp{\isacharunderscore}even{\isacharcolon}\ {\isachardoublequote}{\isadigit{2}}\ dvd\ n\ {\isasymLongrightarrow}\ n\ {\isasymin}\ even{\isachardoublequote}\isanewline
\isamarkupfalse%
\isacommand{by}\ {\isacharparenleft}auto\ simp\ add{\isacharcolon}\ dvd{\isacharunderscore}def{\isacharparenright}\isamarkupfalse%
%
\begin{isamarkuptext}%
our first rule induction!%
\end{isamarkuptext}%
\isamarkuptrue%
\isacommand{lemma}\ even{\isacharunderscore}imp{\isacharunderscore}dvd{\isacharcolon}\ {\isachardoublequote}n\ {\isasymin}\ even\ {\isasymLongrightarrow}\ {\isadigit{2}}\ dvd\ n{\isachardoublequote}\isanewline
\isamarkupfalse%
\isacommand{apply}\ {\isacharparenleft}erule\ even{\isachardot}induct{\isacharparenright}\isamarkupfalse%
%
\begin{isamarkuptxt}%
\begin{isabelle}%
\ {\isadigit{1}}{\isachardot}\ {\isadigit{2}}\ dvd\ {\isadigit{0}}\isanewline
\ {\isadigit{2}}{\isachardot}\ {\isasymAnd}n{\isachardot}\ {\isasymlbrakk}n\ {\isasymin}\ even{\isacharsemicolon}\ {\isadigit{2}}\ dvd\ n{\isasymrbrakk}\ {\isasymLongrightarrow}\ {\isadigit{2}}\ dvd\ Suc\ {\isacharparenleft}Suc\ n{\isacharparenright}%
\end{isabelle}%
\end{isamarkuptxt}%
\isamarkuptrue%
\isacommand{apply}\ {\isacharparenleft}simp{\isacharunderscore}all\ add{\isacharcolon}\ dvd{\isacharunderscore}def{\isacharparenright}\isamarkupfalse%
%
\begin{isamarkuptxt}%
\begin{isabelle}%
\ {\isadigit{1}}{\isachardot}\ {\isasymAnd}n{\isachardot}\ {\isasymlbrakk}n\ {\isasymin}\ even{\isacharsemicolon}\ {\isasymexists}k{\isachardot}\ n\ {\isacharequal}\ {\isadigit{2}}\ {\isacharasterisk}\ k{\isasymrbrakk}\ {\isasymLongrightarrow}\ {\isasymexists}k{\isachardot}\ Suc\ {\isacharparenleft}Suc\ n{\isacharparenright}\ {\isacharequal}\ {\isadigit{2}}\ {\isacharasterisk}\ k%
\end{isabelle}%
\end{isamarkuptxt}%
\isamarkuptrue%
\isacommand{apply}\ clarify\isamarkupfalse%
%
\begin{isamarkuptxt}%
\begin{isabelle}%
\ {\isadigit{1}}{\isachardot}\ {\isasymAnd}n\ k{\isachardot}\ {\isadigit{2}}\ {\isacharasterisk}\ k\ {\isasymin}\ even\ {\isasymLongrightarrow}\ {\isasymexists}ka{\isachardot}\ Suc\ {\isacharparenleft}Suc\ {\isacharparenleft}{\isadigit{2}}\ {\isacharasterisk}\ k{\isacharparenright}{\isacharparenright}\ {\isacharequal}\ {\isadigit{2}}\ {\isacharasterisk}\ ka%
\end{isabelle}%
\end{isamarkuptxt}%
\isamarkuptrue%
\isacommand{apply}\ {\isacharparenleft}rule{\isacharunderscore}tac\ x\ {\isacharequal}\ {\isachardoublequote}Suc\ k{\isachardoublequote}\ \isakeyword{in}\ exI{\isacharcomma}\ simp{\isacharparenright}\isanewline
\isamarkupfalse%
\isacommand{done}\isamarkupfalse%
%
\begin{isamarkuptext}%
no iff-attribute because we don't always want to use it%
\end{isamarkuptext}%
\isamarkuptrue%
\isacommand{theorem}\ even{\isacharunderscore}iff{\isacharunderscore}dvd{\isacharcolon}\ {\isachardoublequote}{\isacharparenleft}n\ {\isasymin}\ even{\isacharparenright}\ {\isacharequal}\ {\isacharparenleft}{\isadigit{2}}\ dvd\ n{\isacharparenright}{\isachardoublequote}\isanewline
\isamarkupfalse%
\isacommand{by}\ {\isacharparenleft}blast\ intro{\isacharcolon}\ dvd{\isacharunderscore}imp{\isacharunderscore}even\ even{\isacharunderscore}imp{\isacharunderscore}dvd{\isacharparenright}\isamarkupfalse%
%
\begin{isamarkuptext}%
this result ISN'T inductive...%
\end{isamarkuptext}%
\isamarkuptrue%
\isacommand{lemma}\ Suc{\isacharunderscore}Suc{\isacharunderscore}even{\isacharunderscore}imp{\isacharunderscore}even{\isacharcolon}\ {\isachardoublequote}Suc\ {\isacharparenleft}Suc\ n{\isacharparenright}\ {\isasymin}\ even\ {\isasymLongrightarrow}\ n\ {\isasymin}\ even{\isachardoublequote}\isanewline
\isamarkupfalse%
\isacommand{apply}\ {\isacharparenleft}erule\ even{\isachardot}induct{\isacharparenright}\isamarkupfalse%
%
\begin{isamarkuptxt}%
\begin{isabelle}%
\ {\isadigit{1}}{\isachardot}\ n\ {\isasymin}\ even\isanewline
\ {\isadigit{2}}{\isachardot}\ {\isasymAnd}na{\isachardot}\ {\isasymlbrakk}na\ {\isasymin}\ even{\isacharsemicolon}\ n\ {\isasymin}\ even{\isasymrbrakk}\ {\isasymLongrightarrow}\ n\ {\isasymin}\ even%
\end{isabelle}%
\end{isamarkuptxt}%
\isamarkuptrue%
\isacommand{oops}\isamarkupfalse%
%
\begin{isamarkuptext}%
...so we need an inductive lemma...%
\end{isamarkuptext}%
\isamarkuptrue%
\isacommand{lemma}\ even{\isacharunderscore}imp{\isacharunderscore}even{\isacharunderscore}minus{\isacharunderscore}{\isadigit{2}}{\isacharcolon}\ {\isachardoublequote}n\ {\isasymin}\ even\ {\isasymLongrightarrow}\ n\ {\isacharminus}\ {\isadigit{2}}\ {\isasymin}\ even{\isachardoublequote}\isanewline
\isamarkupfalse%
\isacommand{apply}\ {\isacharparenleft}erule\ even{\isachardot}induct{\isacharparenright}\isamarkupfalse%
%
\begin{isamarkuptxt}%
\begin{isabelle}%
\ {\isadigit{1}}{\isachardot}\ {\isadigit{0}}\ {\isacharminus}\ {\isadigit{2}}\ {\isasymin}\ even\isanewline
\ {\isadigit{2}}{\isachardot}\ {\isasymAnd}n{\isachardot}\ {\isasymlbrakk}n\ {\isasymin}\ even{\isacharsemicolon}\ n\ {\isacharminus}\ {\isadigit{2}}\ {\isasymin}\ even{\isasymrbrakk}\ {\isasymLongrightarrow}\ Suc\ {\isacharparenleft}Suc\ n{\isacharparenright}\ {\isacharminus}\ {\isadigit{2}}\ {\isasymin}\ even%
\end{isabelle}%
\end{isamarkuptxt}%
\isamarkuptrue%
\isacommand{apply}\ auto\isanewline
\isamarkupfalse%
\isacommand{done}\isamarkupfalse%
%
\begin{isamarkuptext}%
...and prove it in a separate step%
\end{isamarkuptext}%
\isamarkuptrue%
\isacommand{lemma}\ Suc{\isacharunderscore}Suc{\isacharunderscore}even{\isacharunderscore}imp{\isacharunderscore}even{\isacharcolon}\ {\isachardoublequote}Suc\ {\isacharparenleft}Suc\ n{\isacharparenright}\ {\isasymin}\ even\ {\isasymLongrightarrow}\ n\ {\isasymin}\ even{\isachardoublequote}\isanewline
\isamarkupfalse%
\isacommand{by}\ {\isacharparenleft}drule\ even{\isacharunderscore}imp{\isacharunderscore}even{\isacharunderscore}minus{\isacharunderscore}{\isadigit{2}}{\isacharcomma}\ simp{\isacharparenright}\isanewline
\isanewline
\isanewline
\isamarkupfalse%
\isacommand{lemma}\ {\isacharbrackleft}iff{\isacharbrackright}{\isacharcolon}\ {\isachardoublequote}{\isacharparenleft}{\isacharparenleft}Suc\ {\isacharparenleft}Suc\ n{\isacharparenright}{\isacharparenright}\ {\isasymin}\ even{\isacharparenright}\ {\isacharequal}\ {\isacharparenleft}n\ {\isasymin}\ even{\isacharparenright}{\isachardoublequote}\isanewline
\isamarkupfalse%
\isacommand{by}\ {\isacharparenleft}blast\ dest{\isacharcolon}\ Suc{\isacharunderscore}Suc{\isacharunderscore}even{\isacharunderscore}imp{\isacharunderscore}even{\isacharparenright}\isanewline
\isanewline
\isamarkupfalse%
\isacommand{end}\isanewline
\isanewline
\isamarkupfalse%
\end{isabellebody}%
%%% Local Variables:
%%% mode: latex
%%% TeX-master: "root"
%%% End:

%
\begin{isabellebody}%
\def\isabellecontext{Mutual}%
%
\isamarkupsubsection{Mutual inductive definitions%
}
%
\begin{isamarkuptext}%
Just as there are datatypes defined by mutual recursion, there are sets defined
by mutual induction. As a trivial example we consider the even and odd natural numbers:%
\end{isamarkuptext}%
\isacommand{consts}\ even\ {\isacharcolon}{\isacharcolon}\ {\isachardoublequote}nat\ set{\isachardoublequote}\isanewline
\ \ \ \ \ \ \ odd\ \ {\isacharcolon}{\isacharcolon}\ {\isachardoublequote}nat\ set{\isachardoublequote}\isanewline
\isanewline
\isacommand{inductive}\ even\ odd\isanewline
\isakeyword{intros}\isanewline
zero{\isacharcolon}\ \ {\isachardoublequote}{\isadigit{0}}\ {\isasymin}\ even{\isachardoublequote}\isanewline
evenI{\isacharcolon}\ {\isachardoublequote}n\ {\isasymin}\ odd\ {\isasymLongrightarrow}\ Suc\ n\ {\isasymin}\ even{\isachardoublequote}\isanewline
oddI{\isacharcolon}\ \ {\isachardoublequote}n\ {\isasymin}\ even\ {\isasymLongrightarrow}\ Suc\ n\ {\isasymin}\ odd{\isachardoublequote}%
\begin{isamarkuptext}%
\noindent
The simultaneous inductive definition of multiple sets is no different from that
of a single set, except for induction: just as for mutually recursive datatypes,
induction needs to involve all the simultaneously defined sets. In the above case,
the induction rule is called \isa{even{\isacharunderscore}odd{\isachardot}induct} (simply concenate the names
of the sets involved) and has the conclusion
\begin{isabelle}%
\ \ \ \ \ {\isacharparenleft}{\isacharquery}x\ {\isasymin}\ even\ {\isasymlongrightarrow}\ {\isacharquery}P\ {\isacharquery}x{\isacharparenright}\ {\isasymand}\ {\isacharparenleft}{\isacharquery}y\ {\isasymin}\ odd\ {\isasymlongrightarrow}\ {\isacharquery}Q\ {\isacharquery}y{\isacharparenright}%
\end{isabelle}

If we want to prove that all even numbers are divisible by 2, we have to generalize
the statement as follows:%
\end{isamarkuptext}%
\isacommand{lemma}\ {\isachardoublequote}{\isacharparenleft}m\ {\isasymin}\ even\ {\isasymlongrightarrow}\ {\isadigit{2}}\ dvd\ m{\isacharparenright}\ {\isasymand}\ {\isacharparenleft}n\ {\isasymin}\ odd\ {\isasymlongrightarrow}\ {\isadigit{2}}\ dvd\ {\isacharparenleft}Suc\ n{\isacharparenright}{\isacharparenright}{\isachardoublequote}%
\begin{isamarkuptxt}%
\noindent
The proof is by rule induction. Because of the form of the induction theorem, it is
applied by \isa{rule} rather than \isa{erule} as for ordinary inductive definitions:%
\end{isamarkuptxt}%
\isacommand{apply}{\isacharparenleft}rule\ even{\isacharunderscore}odd{\isachardot}induct{\isacharparenright}%
\begin{isamarkuptxt}%
\begin{isabelle}%
\ {\isadigit{1}}{\isachardot}\ {\isadigit{2}}\ dvd\ {\isadigit{0}}\isanewline
\ {\isadigit{2}}{\isachardot}\ {\isasymAnd}n{\isachardot}\ {\isasymlbrakk}n\ {\isasymin}\ odd{\isacharsemicolon}\ {\isadigit{2}}\ dvd\ Suc\ n{\isasymrbrakk}\ {\isasymLongrightarrow}\ {\isadigit{2}}\ dvd\ Suc\ n\isanewline
\ {\isadigit{3}}{\isachardot}\ {\isasymAnd}n{\isachardot}\ {\isasymlbrakk}n\ {\isasymin}\ even{\isacharsemicolon}\ {\isadigit{2}}\ dvd\ n{\isasymrbrakk}\ {\isasymLongrightarrow}\ {\isadigit{2}}\ dvd\ Suc\ {\isacharparenleft}Suc\ n{\isacharparenright}%
\end{isabelle}
The first two subgoals are proved by simplification and the final one can be
proved in the same manner as in \S\ref{sec:rule-induction}
where the same subgoal was encountered before.
We do not show the proof script.%
\end{isamarkuptxt}%
\end{isabellebody}%
%%% Local Variables:
%%% mode: latex
%%% TeX-master: "root"
%%% End:

%
\begin{isabellebody}%
\def\isabellecontext{Star}%
%
\isadelimtheory
%
\endisadelimtheory
%
\isatagtheory
%
\endisatagtheory
{\isafoldtheory}%
%
\isadelimtheory
%
\endisadelimtheory
%
\isamarkupsection{The Reflexive Transitive Closure%
}
\isamarkuptrue%
%
\begin{isamarkuptext}%
\label{sec:rtc}
\index{reflexive transitive closure!defining inductively|(}%
An inductive definition may accept parameters, so it can express 
functions that yield sets.
Relations too can be defined inductively, since they are just sets of pairs.
A perfect example is the function that maps a relation to its
reflexive transitive closure.  This concept was already
introduced in \S\ref{sec:Relations}, where the operator \isa{\isaliteral{5C3C5E7375703E}{}\isactrlsup {\isaliteral{2A}{\isacharasterisk}}} was
defined as a least fixed point because inductive definitions were not yet
available. But now they are:%
\end{isamarkuptext}%
\isamarkuptrue%
\isacommand{inductive{\isaliteral{5F}{\isacharunderscore}}set}\isamarkupfalse%
\isanewline
\ \ rtc\ {\isaliteral{3A}{\isacharcolon}}{\isaliteral{3A}{\isacharcolon}}\ {\isaliteral{22}{\isachardoublequoteopen}}{\isaliteral{28}{\isacharparenleft}}{\isaliteral{27}{\isacharprime}}a\ {\isaliteral{5C3C74696D65733E}{\isasymtimes}}\ {\isaliteral{27}{\isacharprime}}a{\isaliteral{29}{\isacharparenright}}set\ {\isaliteral{5C3C52696768746172726F773E}{\isasymRightarrow}}\ {\isaliteral{28}{\isacharparenleft}}{\isaliteral{27}{\isacharprime}}a\ {\isaliteral{5C3C74696D65733E}{\isasymtimes}}\ {\isaliteral{27}{\isacharprime}}a{\isaliteral{29}{\isacharparenright}}set{\isaliteral{22}{\isachardoublequoteclose}}\ \ \ {\isaliteral{28}{\isacharparenleft}}{\isaliteral{22}{\isachardoublequoteopen}}{\isaliteral{5F}{\isacharunderscore}}{\isaliteral{2A}{\isacharasterisk}}{\isaliteral{22}{\isachardoublequoteclose}}\ {\isaliteral{5B}{\isacharbrackleft}}{\isadigit{1}}{\isadigit{0}}{\isadigit{0}}{\isadigit{0}}{\isaliteral{5D}{\isacharbrackright}}\ {\isadigit{9}}{\isadigit{9}}{\isadigit{9}}{\isaliteral{29}{\isacharparenright}}\isanewline
\ \ \isakeyword{for}\ r\ {\isaliteral{3A}{\isacharcolon}}{\isaliteral{3A}{\isacharcolon}}\ {\isaliteral{22}{\isachardoublequoteopen}}{\isaliteral{28}{\isacharparenleft}}{\isaliteral{27}{\isacharprime}}a\ {\isaliteral{5C3C74696D65733E}{\isasymtimes}}\ {\isaliteral{27}{\isacharprime}}a{\isaliteral{29}{\isacharparenright}}set{\isaliteral{22}{\isachardoublequoteclose}}\isanewline
\isakeyword{where}\isanewline
\ \ rtc{\isaliteral{5F}{\isacharunderscore}}refl{\isaliteral{5B}{\isacharbrackleft}}iff{\isaliteral{5D}{\isacharbrackright}}{\isaliteral{3A}{\isacharcolon}}\ \ {\isaliteral{22}{\isachardoublequoteopen}}{\isaliteral{28}{\isacharparenleft}}x{\isaliteral{2C}{\isacharcomma}}x{\isaliteral{29}{\isacharparenright}}\ {\isaliteral{5C3C696E3E}{\isasymin}}\ r{\isaliteral{2A}{\isacharasterisk}}{\isaliteral{22}{\isachardoublequoteclose}}\isanewline
{\isaliteral{7C}{\isacharbar}}\ rtc{\isaliteral{5F}{\isacharunderscore}}step{\isaliteral{3A}{\isacharcolon}}\ \ \ \ \ \ \ {\isaliteral{22}{\isachardoublequoteopen}}{\isaliteral{5C3C6C6272616B6B3E}{\isasymlbrakk}}\ {\isaliteral{28}{\isacharparenleft}}x{\isaliteral{2C}{\isacharcomma}}y{\isaliteral{29}{\isacharparenright}}\ {\isaliteral{5C3C696E3E}{\isasymin}}\ r{\isaliteral{3B}{\isacharsemicolon}}\ {\isaliteral{28}{\isacharparenleft}}y{\isaliteral{2C}{\isacharcomma}}z{\isaliteral{29}{\isacharparenright}}\ {\isaliteral{5C3C696E3E}{\isasymin}}\ r{\isaliteral{2A}{\isacharasterisk}}\ {\isaliteral{5C3C726272616B6B3E}{\isasymrbrakk}}\ {\isaliteral{5C3C4C6F6E6772696768746172726F773E}{\isasymLongrightarrow}}\ {\isaliteral{28}{\isacharparenleft}}x{\isaliteral{2C}{\isacharcomma}}z{\isaliteral{29}{\isacharparenright}}\ {\isaliteral{5C3C696E3E}{\isasymin}}\ r{\isaliteral{2A}{\isacharasterisk}}{\isaliteral{22}{\isachardoublequoteclose}}%
\begin{isamarkuptext}%
\noindent
The function \isa{rtc} is annotated with concrete syntax: instead of
\isa{rtc\ r} we can write \isa{r{\isaliteral{2A}{\isacharasterisk}}}. The actual definition
consists of two rules. Reflexivity is obvious and is immediately given the
\isa{iff} attribute to increase automation. The
second rule, \isa{rtc{\isaliteral{5F}{\isacharunderscore}}step}, says that we can always add one more
\isa{r}-step to the left. Although we could make \isa{rtc{\isaliteral{5F}{\isacharunderscore}}step} an
introduction rule, this is dangerous: the recursion in the second premise
slows down and may even kill the automatic tactics.

The above definition of the concept of reflexive transitive closure may
be sufficiently intuitive but it is certainly not the only possible one:
for a start, it does not even mention transitivity.
The rest of this section is devoted to proving that it is equivalent to
the standard definition. We start with a simple lemma:%
\end{isamarkuptext}%
\isamarkuptrue%
\isacommand{lemma}\isamarkupfalse%
\ {\isaliteral{5B}{\isacharbrackleft}}intro{\isaliteral{5D}{\isacharbrackright}}{\isaliteral{3A}{\isacharcolon}}\ {\isaliteral{22}{\isachardoublequoteopen}}{\isaliteral{28}{\isacharparenleft}}x{\isaliteral{2C}{\isacharcomma}}y{\isaliteral{29}{\isacharparenright}}\ {\isaliteral{5C3C696E3E}{\isasymin}}\ r\ {\isaliteral{5C3C4C6F6E6772696768746172726F773E}{\isasymLongrightarrow}}\ {\isaliteral{28}{\isacharparenleft}}x{\isaliteral{2C}{\isacharcomma}}y{\isaliteral{29}{\isacharparenright}}\ {\isaliteral{5C3C696E3E}{\isasymin}}\ r{\isaliteral{2A}{\isacharasterisk}}{\isaliteral{22}{\isachardoublequoteclose}}\isanewline
%
\isadelimproof
%
\endisadelimproof
%
\isatagproof
\isacommand{by}\isamarkupfalse%
{\isaliteral{28}{\isacharparenleft}}blast\ intro{\isaliteral{3A}{\isacharcolon}}\ rtc{\isaliteral{5F}{\isacharunderscore}}step{\isaliteral{29}{\isacharparenright}}%
\endisatagproof
{\isafoldproof}%
%
\isadelimproof
%
\endisadelimproof
%
\begin{isamarkuptext}%
\noindent
Although the lemma itself is an unremarkable consequence of the basic rules,
it has the advantage that it can be declared an introduction rule without the
danger of killing the automatic tactics because \isa{r{\isaliteral{2A}{\isacharasterisk}}} occurs only in
the conclusion and not in the premise. Thus some proofs that would otherwise
need \isa{rtc{\isaliteral{5F}{\isacharunderscore}}step} can now be found automatically. The proof also
shows that \isa{blast} is able to handle \isa{rtc{\isaliteral{5F}{\isacharunderscore}}step}. But
some of the other automatic tactics are more sensitive, and even \isa{blast} can be lead astray in the presence of large numbers of rules.

To prove transitivity, we need rule induction, i.e.\ theorem
\isa{rtc{\isaliteral{2E}{\isachardot}}induct}:
\begin{isabelle}%
\ \ \ \ \ {\isaliteral{5C3C6C6272616B6B3E}{\isasymlbrakk}}{\isaliteral{28}{\isacharparenleft}}{\isaliteral{3F}{\isacharquery}}x{\isadigit{1}}{\isaliteral{2E}{\isachardot}}{\isadigit{0}}{\isaliteral{2C}{\isacharcomma}}\ {\isaliteral{3F}{\isacharquery}}x{\isadigit{2}}{\isaliteral{2E}{\isachardot}}{\isadigit{0}}{\isaliteral{29}{\isacharparenright}}\ {\isaliteral{5C3C696E3E}{\isasymin}}\ {\isaliteral{3F}{\isacharquery}}r{\isaliteral{2A}{\isacharasterisk}}{\isaliteral{3B}{\isacharsemicolon}}\ {\isaliteral{5C3C416E643E}{\isasymAnd}}x{\isaliteral{2E}{\isachardot}}\ {\isaliteral{3F}{\isacharquery}}P\ x\ x{\isaliteral{3B}{\isacharsemicolon}}\isanewline
\isaindent{\ \ \ \ \ \ }{\isaliteral{5C3C416E643E}{\isasymAnd}}x\ y\ z{\isaliteral{2E}{\isachardot}}\ {\isaliteral{5C3C6C6272616B6B3E}{\isasymlbrakk}}{\isaliteral{28}{\isacharparenleft}}x{\isaliteral{2C}{\isacharcomma}}\ y{\isaliteral{29}{\isacharparenright}}\ {\isaliteral{5C3C696E3E}{\isasymin}}\ {\isaliteral{3F}{\isacharquery}}r{\isaliteral{3B}{\isacharsemicolon}}\ {\isaliteral{28}{\isacharparenleft}}y{\isaliteral{2C}{\isacharcomma}}\ z{\isaliteral{29}{\isacharparenright}}\ {\isaliteral{5C3C696E3E}{\isasymin}}\ {\isaliteral{3F}{\isacharquery}}r{\isaliteral{2A}{\isacharasterisk}}{\isaliteral{3B}{\isacharsemicolon}}\ {\isaliteral{3F}{\isacharquery}}P\ y\ z{\isaliteral{5C3C726272616B6B3E}{\isasymrbrakk}}\ {\isaliteral{5C3C4C6F6E6772696768746172726F773E}{\isasymLongrightarrow}}\ {\isaliteral{3F}{\isacharquery}}P\ x\ z{\isaliteral{5C3C726272616B6B3E}{\isasymrbrakk}}\isanewline
\isaindent{\ \ \ \ \ }{\isaliteral{5C3C4C6F6E6772696768746172726F773E}{\isasymLongrightarrow}}\ {\isaliteral{3F}{\isacharquery}}P\ {\isaliteral{3F}{\isacharquery}}x{\isadigit{1}}{\isaliteral{2E}{\isachardot}}{\isadigit{0}}\ {\isaliteral{3F}{\isacharquery}}x{\isadigit{2}}{\isaliteral{2E}{\isachardot}}{\isadigit{0}}%
\end{isabelle}
It says that \isa{{\isaliteral{3F}{\isacharquery}}P} holds for an arbitrary pair \isa{{\isaliteral{28}{\isacharparenleft}}{\isaliteral{3F}{\isacharquery}}x{\isadigit{1}}{\isaliteral{2E}{\isachardot}}{\isadigit{0}}{\isaliteral{2C}{\isacharcomma}}\ {\isaliteral{3F}{\isacharquery}}x{\isadigit{2}}{\isaliteral{2E}{\isachardot}}{\isadigit{0}}{\isaliteral{29}{\isacharparenright}}\ {\isaliteral{5C3C696E3E}{\isasymin}}\ {\isaliteral{3F}{\isacharquery}}r{\isaliteral{2A}{\isacharasterisk}}}
if \isa{{\isaliteral{3F}{\isacharquery}}P} is preserved by all rules of the inductive definition,
i.e.\ if \isa{{\isaliteral{3F}{\isacharquery}}P} holds for the conclusion provided it holds for the
premises. In general, rule induction for an $n$-ary inductive relation $R$
expects a premise of the form $(x@1,\dots,x@n) \in R$.

Now we turn to the inductive proof of transitivity:%
\end{isamarkuptext}%
\isamarkuptrue%
\isacommand{lemma}\isamarkupfalse%
\ rtc{\isaliteral{5F}{\isacharunderscore}}trans{\isaliteral{3A}{\isacharcolon}}\ {\isaliteral{22}{\isachardoublequoteopen}}{\isaliteral{5C3C6C6272616B6B3E}{\isasymlbrakk}}\ {\isaliteral{28}{\isacharparenleft}}x{\isaliteral{2C}{\isacharcomma}}y{\isaliteral{29}{\isacharparenright}}\ {\isaliteral{5C3C696E3E}{\isasymin}}\ r{\isaliteral{2A}{\isacharasterisk}}{\isaliteral{3B}{\isacharsemicolon}}\ {\isaliteral{28}{\isacharparenleft}}y{\isaliteral{2C}{\isacharcomma}}z{\isaliteral{29}{\isacharparenright}}\ {\isaliteral{5C3C696E3E}{\isasymin}}\ r{\isaliteral{2A}{\isacharasterisk}}\ {\isaliteral{5C3C726272616B6B3E}{\isasymrbrakk}}\ {\isaliteral{5C3C4C6F6E6772696768746172726F773E}{\isasymLongrightarrow}}\ {\isaliteral{28}{\isacharparenleft}}x{\isaliteral{2C}{\isacharcomma}}z{\isaliteral{29}{\isacharparenright}}\ {\isaliteral{5C3C696E3E}{\isasymin}}\ r{\isaliteral{2A}{\isacharasterisk}}{\isaliteral{22}{\isachardoublequoteclose}}\isanewline
%
\isadelimproof
%
\endisadelimproof
%
\isatagproof
\isacommand{apply}\isamarkupfalse%
{\isaliteral{28}{\isacharparenleft}}erule\ rtc{\isaliteral{2E}{\isachardot}}induct{\isaliteral{29}{\isacharparenright}}%
\begin{isamarkuptxt}%
\noindent
Unfortunately, even the base case is a problem:
\begin{isabelle}%
\ {\isadigit{1}}{\isaliteral{2E}{\isachardot}}\ {\isaliteral{5C3C416E643E}{\isasymAnd}}x{\isaliteral{2E}{\isachardot}}\ {\isaliteral{28}{\isacharparenleft}}y{\isaliteral{2C}{\isacharcomma}}\ z{\isaliteral{29}{\isacharparenright}}\ {\isaliteral{5C3C696E3E}{\isasymin}}\ r{\isaliteral{2A}{\isacharasterisk}}\ {\isaliteral{5C3C4C6F6E6772696768746172726F773E}{\isasymLongrightarrow}}\ {\isaliteral{28}{\isacharparenleft}}x{\isaliteral{2C}{\isacharcomma}}\ z{\isaliteral{29}{\isacharparenright}}\ {\isaliteral{5C3C696E3E}{\isasymin}}\ r{\isaliteral{2A}{\isacharasterisk}}%
\end{isabelle}
We have to abandon this proof attempt.
To understand what is going on, let us look again at \isa{rtc{\isaliteral{2E}{\isachardot}}induct}.
In the above application of \isa{erule}, the first premise of
\isa{rtc{\isaliteral{2E}{\isachardot}}induct} is unified with the first suitable assumption, which
is \isa{{\isaliteral{28}{\isacharparenleft}}x{\isaliteral{2C}{\isacharcomma}}\ y{\isaliteral{29}{\isacharparenright}}\ {\isaliteral{5C3C696E3E}{\isasymin}}\ r{\isaliteral{2A}{\isacharasterisk}}} rather than \isa{{\isaliteral{28}{\isacharparenleft}}y{\isaliteral{2C}{\isacharcomma}}\ z{\isaliteral{29}{\isacharparenright}}\ {\isaliteral{5C3C696E3E}{\isasymin}}\ r{\isaliteral{2A}{\isacharasterisk}}}. Although that
is what we want, it is merely due to the order in which the assumptions occur
in the subgoal, which it is not good practice to rely on. As a result,
\isa{{\isaliteral{3F}{\isacharquery}}xb} becomes \isa{x}, \isa{{\isaliteral{3F}{\isacharquery}}xa} becomes
\isa{y} and \isa{{\isaliteral{3F}{\isacharquery}}P} becomes \isa{{\isaliteral{5C3C6C616D6264613E}{\isasymlambda}}u\ v{\isaliteral{2E}{\isachardot}}\ {\isaliteral{28}{\isacharparenleft}}u{\isaliteral{2C}{\isacharcomma}}\ z{\isaliteral{29}{\isacharparenright}}\ {\isaliteral{5C3C696E3E}{\isasymin}}\ r{\isaliteral{2A}{\isacharasterisk}}}, thus
yielding the above subgoal. So what went wrong?

When looking at the instantiation of \isa{{\isaliteral{3F}{\isacharquery}}P} we see that it does not
depend on its second parameter at all. The reason is that in our original
goal, of the pair \isa{{\isaliteral{28}{\isacharparenleft}}x{\isaliteral{2C}{\isacharcomma}}\ y{\isaliteral{29}{\isacharparenright}}} only \isa{x} appears also in the
conclusion, but not \isa{y}. Thus our induction statement is too
general. Fortunately, it can easily be specialized:
transfer the additional premise \isa{{\isaliteral{28}{\isacharparenleft}}y{\isaliteral{2C}{\isacharcomma}}\ z{\isaliteral{29}{\isacharparenright}}\ {\isaliteral{5C3C696E3E}{\isasymin}}\ r{\isaliteral{2A}{\isacharasterisk}}} into the conclusion:%
\end{isamarkuptxt}%
\isamarkuptrue%
%
\endisatagproof
{\isafoldproof}%
%
\isadelimproof
%
\endisadelimproof
\isacommand{lemma}\isamarkupfalse%
\ rtc{\isaliteral{5F}{\isacharunderscore}}trans{\isaliteral{5B}{\isacharbrackleft}}rule{\isaliteral{5F}{\isacharunderscore}}format{\isaliteral{5D}{\isacharbrackright}}{\isaliteral{3A}{\isacharcolon}}\isanewline
\ \ {\isaliteral{22}{\isachardoublequoteopen}}{\isaliteral{28}{\isacharparenleft}}x{\isaliteral{2C}{\isacharcomma}}y{\isaliteral{29}{\isacharparenright}}\ {\isaliteral{5C3C696E3E}{\isasymin}}\ r{\isaliteral{2A}{\isacharasterisk}}\ {\isaliteral{5C3C4C6F6E6772696768746172726F773E}{\isasymLongrightarrow}}\ {\isaliteral{28}{\isacharparenleft}}y{\isaliteral{2C}{\isacharcomma}}z{\isaliteral{29}{\isacharparenright}}\ {\isaliteral{5C3C696E3E}{\isasymin}}\ r{\isaliteral{2A}{\isacharasterisk}}\ {\isaliteral{5C3C6C6F6E6772696768746172726F773E}{\isasymlongrightarrow}}\ {\isaliteral{28}{\isacharparenleft}}x{\isaliteral{2C}{\isacharcomma}}z{\isaliteral{29}{\isacharparenright}}\ {\isaliteral{5C3C696E3E}{\isasymin}}\ r{\isaliteral{2A}{\isacharasterisk}}{\isaliteral{22}{\isachardoublequoteclose}}%
\isadelimproof
%
\endisadelimproof
%
\isatagproof
%
\begin{isamarkuptxt}%
\noindent
This is not an obscure trick but a generally applicable heuristic:
\begin{quote}\em
When proving a statement by rule induction on $(x@1,\dots,x@n) \in R$,
pull all other premises containing any of the $x@i$ into the conclusion
using $\longrightarrow$.
\end{quote}
A similar heuristic for other kinds of inductions is formulated in
\S\ref{sec:ind-var-in-prems}. The \isa{rule{\isaliteral{5F}{\isacharunderscore}}format} directive turns
\isa{{\isaliteral{5C3C6C6F6E6772696768746172726F773E}{\isasymlongrightarrow}}} back into \isa{{\isaliteral{5C3C4C6F6E6772696768746172726F773E}{\isasymLongrightarrow}}}: in the end we obtain the original
statement of our lemma.%
\end{isamarkuptxt}%
\isamarkuptrue%
\isacommand{apply}\isamarkupfalse%
{\isaliteral{28}{\isacharparenleft}}erule\ rtc{\isaliteral{2E}{\isachardot}}induct{\isaliteral{29}{\isacharparenright}}%
\begin{isamarkuptxt}%
\noindent
Now induction produces two subgoals which are both proved automatically:
\begin{isabelle}%
\ {\isadigit{1}}{\isaliteral{2E}{\isachardot}}\ {\isaliteral{5C3C416E643E}{\isasymAnd}}x{\isaliteral{2E}{\isachardot}}\ {\isaliteral{28}{\isacharparenleft}}x{\isaliteral{2C}{\isacharcomma}}\ z{\isaliteral{29}{\isacharparenright}}\ {\isaliteral{5C3C696E3E}{\isasymin}}\ r{\isaliteral{2A}{\isacharasterisk}}\ {\isaliteral{5C3C6C6F6E6772696768746172726F773E}{\isasymlongrightarrow}}\ {\isaliteral{28}{\isacharparenleft}}x{\isaliteral{2C}{\isacharcomma}}\ z{\isaliteral{29}{\isacharparenright}}\ {\isaliteral{5C3C696E3E}{\isasymin}}\ r{\isaliteral{2A}{\isacharasterisk}}\isanewline
\ {\isadigit{2}}{\isaliteral{2E}{\isachardot}}\ {\isaliteral{5C3C416E643E}{\isasymAnd}}x\ y\ za{\isaliteral{2E}{\isachardot}}\isanewline
\isaindent{\ {\isadigit{2}}{\isaliteral{2E}{\isachardot}}\ \ \ \ }{\isaliteral{5C3C6C6272616B6B3E}{\isasymlbrakk}}{\isaliteral{28}{\isacharparenleft}}x{\isaliteral{2C}{\isacharcomma}}\ y{\isaliteral{29}{\isacharparenright}}\ {\isaliteral{5C3C696E3E}{\isasymin}}\ r{\isaliteral{3B}{\isacharsemicolon}}\ {\isaliteral{28}{\isacharparenleft}}y{\isaliteral{2C}{\isacharcomma}}\ za{\isaliteral{29}{\isacharparenright}}\ {\isaliteral{5C3C696E3E}{\isasymin}}\ r{\isaliteral{2A}{\isacharasterisk}}{\isaliteral{3B}{\isacharsemicolon}}\ {\isaliteral{28}{\isacharparenleft}}za{\isaliteral{2C}{\isacharcomma}}\ z{\isaliteral{29}{\isacharparenright}}\ {\isaliteral{5C3C696E3E}{\isasymin}}\ r{\isaliteral{2A}{\isacharasterisk}}\ {\isaliteral{5C3C6C6F6E6772696768746172726F773E}{\isasymlongrightarrow}}\ {\isaliteral{28}{\isacharparenleft}}y{\isaliteral{2C}{\isacharcomma}}\ z{\isaliteral{29}{\isacharparenright}}\ {\isaliteral{5C3C696E3E}{\isasymin}}\ r{\isaliteral{2A}{\isacharasterisk}}{\isaliteral{5C3C726272616B6B3E}{\isasymrbrakk}}\isanewline
\isaindent{\ {\isadigit{2}}{\isaliteral{2E}{\isachardot}}\ \ \ \ }{\isaliteral{5C3C4C6F6E6772696768746172726F773E}{\isasymLongrightarrow}}\ {\isaliteral{28}{\isacharparenleft}}za{\isaliteral{2C}{\isacharcomma}}\ z{\isaliteral{29}{\isacharparenright}}\ {\isaliteral{5C3C696E3E}{\isasymin}}\ r{\isaliteral{2A}{\isacharasterisk}}\ {\isaliteral{5C3C6C6F6E6772696768746172726F773E}{\isasymlongrightarrow}}\ {\isaliteral{28}{\isacharparenleft}}x{\isaliteral{2C}{\isacharcomma}}\ z{\isaliteral{29}{\isacharparenright}}\ {\isaliteral{5C3C696E3E}{\isasymin}}\ r{\isaliteral{2A}{\isacharasterisk}}%
\end{isabelle}%
\end{isamarkuptxt}%
\isamarkuptrue%
\ \isacommand{apply}\isamarkupfalse%
{\isaliteral{28}{\isacharparenleft}}blast{\isaliteral{29}{\isacharparenright}}\isanewline
\isacommand{apply}\isamarkupfalse%
{\isaliteral{28}{\isacharparenleft}}blast\ intro{\isaliteral{3A}{\isacharcolon}}\ rtc{\isaliteral{5F}{\isacharunderscore}}step{\isaliteral{29}{\isacharparenright}}\isanewline
\isacommand{done}\isamarkupfalse%
%
\endisatagproof
{\isafoldproof}%
%
\isadelimproof
%
\endisadelimproof
%
\begin{isamarkuptext}%
Let us now prove that \isa{r{\isaliteral{2A}{\isacharasterisk}}} is really the reflexive transitive closure
of \isa{r}, i.e.\ the least reflexive and transitive
relation containing \isa{r}. The latter is easily formalized%
\end{isamarkuptext}%
\isamarkuptrue%
\isacommand{inductive{\isaliteral{5F}{\isacharunderscore}}set}\isamarkupfalse%
\isanewline
\ \ rtc{\isadigit{2}}\ {\isaliteral{3A}{\isacharcolon}}{\isaliteral{3A}{\isacharcolon}}\ {\isaliteral{22}{\isachardoublequoteopen}}{\isaliteral{28}{\isacharparenleft}}{\isaliteral{27}{\isacharprime}}a\ {\isaliteral{5C3C74696D65733E}{\isasymtimes}}\ {\isaliteral{27}{\isacharprime}}a{\isaliteral{29}{\isacharparenright}}set\ {\isaliteral{5C3C52696768746172726F773E}{\isasymRightarrow}}\ {\isaliteral{28}{\isacharparenleft}}{\isaliteral{27}{\isacharprime}}a\ {\isaliteral{5C3C74696D65733E}{\isasymtimes}}\ {\isaliteral{27}{\isacharprime}}a{\isaliteral{29}{\isacharparenright}}set{\isaliteral{22}{\isachardoublequoteclose}}\isanewline
\ \ \isakeyword{for}\ r\ {\isaliteral{3A}{\isacharcolon}}{\isaliteral{3A}{\isacharcolon}}\ {\isaliteral{22}{\isachardoublequoteopen}}{\isaliteral{28}{\isacharparenleft}}{\isaliteral{27}{\isacharprime}}a\ {\isaliteral{5C3C74696D65733E}{\isasymtimes}}\ {\isaliteral{27}{\isacharprime}}a{\isaliteral{29}{\isacharparenright}}set{\isaliteral{22}{\isachardoublequoteclose}}\isanewline
\isakeyword{where}\isanewline
\ \ {\isaliteral{22}{\isachardoublequoteopen}}{\isaliteral{28}{\isacharparenleft}}x{\isaliteral{2C}{\isacharcomma}}y{\isaliteral{29}{\isacharparenright}}\ {\isaliteral{5C3C696E3E}{\isasymin}}\ r\ {\isaliteral{5C3C4C6F6E6772696768746172726F773E}{\isasymLongrightarrow}}\ {\isaliteral{28}{\isacharparenleft}}x{\isaliteral{2C}{\isacharcomma}}y{\isaliteral{29}{\isacharparenright}}\ {\isaliteral{5C3C696E3E}{\isasymin}}\ rtc{\isadigit{2}}\ r{\isaliteral{22}{\isachardoublequoteclose}}\isanewline
{\isaliteral{7C}{\isacharbar}}\ {\isaliteral{22}{\isachardoublequoteopen}}{\isaliteral{28}{\isacharparenleft}}x{\isaliteral{2C}{\isacharcomma}}x{\isaliteral{29}{\isacharparenright}}\ {\isaliteral{5C3C696E3E}{\isasymin}}\ rtc{\isadigit{2}}\ r{\isaliteral{22}{\isachardoublequoteclose}}\isanewline
{\isaliteral{7C}{\isacharbar}}\ {\isaliteral{22}{\isachardoublequoteopen}}{\isaliteral{5C3C6C6272616B6B3E}{\isasymlbrakk}}\ {\isaliteral{28}{\isacharparenleft}}x{\isaliteral{2C}{\isacharcomma}}y{\isaliteral{29}{\isacharparenright}}\ {\isaliteral{5C3C696E3E}{\isasymin}}\ rtc{\isadigit{2}}\ r{\isaliteral{3B}{\isacharsemicolon}}\ {\isaliteral{28}{\isacharparenleft}}y{\isaliteral{2C}{\isacharcomma}}z{\isaliteral{29}{\isacharparenright}}\ {\isaliteral{5C3C696E3E}{\isasymin}}\ rtc{\isadigit{2}}\ r\ {\isaliteral{5C3C726272616B6B3E}{\isasymrbrakk}}\ {\isaliteral{5C3C4C6F6E6772696768746172726F773E}{\isasymLongrightarrow}}\ {\isaliteral{28}{\isacharparenleft}}x{\isaliteral{2C}{\isacharcomma}}z{\isaliteral{29}{\isacharparenright}}\ {\isaliteral{5C3C696E3E}{\isasymin}}\ rtc{\isadigit{2}}\ r{\isaliteral{22}{\isachardoublequoteclose}}%
\begin{isamarkuptext}%
\noindent
and the equivalence of the two definitions is easily shown by the obvious rule
inductions:%
\end{isamarkuptext}%
\isamarkuptrue%
\isacommand{lemma}\isamarkupfalse%
\ {\isaliteral{22}{\isachardoublequoteopen}}{\isaliteral{28}{\isacharparenleft}}x{\isaliteral{2C}{\isacharcomma}}y{\isaliteral{29}{\isacharparenright}}\ {\isaliteral{5C3C696E3E}{\isasymin}}\ rtc{\isadigit{2}}\ r\ {\isaliteral{5C3C4C6F6E6772696768746172726F773E}{\isasymLongrightarrow}}\ {\isaliteral{28}{\isacharparenleft}}x{\isaliteral{2C}{\isacharcomma}}y{\isaliteral{29}{\isacharparenright}}\ {\isaliteral{5C3C696E3E}{\isasymin}}\ r{\isaliteral{2A}{\isacharasterisk}}{\isaliteral{22}{\isachardoublequoteclose}}\isanewline
%
\isadelimproof
%
\endisadelimproof
%
\isatagproof
\isacommand{apply}\isamarkupfalse%
{\isaliteral{28}{\isacharparenleft}}erule\ rtc{\isadigit{2}}{\isaliteral{2E}{\isachardot}}induct{\isaliteral{29}{\isacharparenright}}\isanewline
\ \ \isacommand{apply}\isamarkupfalse%
{\isaliteral{28}{\isacharparenleft}}blast{\isaliteral{29}{\isacharparenright}}\isanewline
\ \isacommand{apply}\isamarkupfalse%
{\isaliteral{28}{\isacharparenleft}}blast{\isaliteral{29}{\isacharparenright}}\isanewline
\isacommand{apply}\isamarkupfalse%
{\isaliteral{28}{\isacharparenleft}}blast\ intro{\isaliteral{3A}{\isacharcolon}}\ rtc{\isaliteral{5F}{\isacharunderscore}}trans{\isaliteral{29}{\isacharparenright}}\isanewline
\isacommand{done}\isamarkupfalse%
%
\endisatagproof
{\isafoldproof}%
%
\isadelimproof
\isanewline
%
\endisadelimproof
\isanewline
\isacommand{lemma}\isamarkupfalse%
\ {\isaliteral{22}{\isachardoublequoteopen}}{\isaliteral{28}{\isacharparenleft}}x{\isaliteral{2C}{\isacharcomma}}y{\isaliteral{29}{\isacharparenright}}\ {\isaliteral{5C3C696E3E}{\isasymin}}\ r{\isaliteral{2A}{\isacharasterisk}}\ {\isaliteral{5C3C4C6F6E6772696768746172726F773E}{\isasymLongrightarrow}}\ {\isaliteral{28}{\isacharparenleft}}x{\isaliteral{2C}{\isacharcomma}}y{\isaliteral{29}{\isacharparenright}}\ {\isaliteral{5C3C696E3E}{\isasymin}}\ rtc{\isadigit{2}}\ r{\isaliteral{22}{\isachardoublequoteclose}}\isanewline
%
\isadelimproof
%
\endisadelimproof
%
\isatagproof
\isacommand{apply}\isamarkupfalse%
{\isaliteral{28}{\isacharparenleft}}erule\ rtc{\isaliteral{2E}{\isachardot}}induct{\isaliteral{29}{\isacharparenright}}\isanewline
\ \isacommand{apply}\isamarkupfalse%
{\isaliteral{28}{\isacharparenleft}}blast\ intro{\isaliteral{3A}{\isacharcolon}}\ rtc{\isadigit{2}}{\isaliteral{2E}{\isachardot}}intros{\isaliteral{29}{\isacharparenright}}\isanewline
\isacommand{apply}\isamarkupfalse%
{\isaliteral{28}{\isacharparenleft}}blast\ intro{\isaliteral{3A}{\isacharcolon}}\ rtc{\isadigit{2}}{\isaliteral{2E}{\isachardot}}intros{\isaliteral{29}{\isacharparenright}}\isanewline
\isacommand{done}\isamarkupfalse%
%
\endisatagproof
{\isafoldproof}%
%
\isadelimproof
%
\endisadelimproof
%
\begin{isamarkuptext}%
So why did we start with the first definition? Because it is simpler. It
contains only two rules, and the single step rule is simpler than
transitivity.  As a consequence, \isa{rtc{\isaliteral{2E}{\isachardot}}induct} is simpler than
\isa{rtc{\isadigit{2}}{\isaliteral{2E}{\isachardot}}induct}. Since inductive proofs are hard enough
anyway, we should always pick the simplest induction schema available.
Hence \isa{rtc} is the definition of choice.
\index{reflexive transitive closure!defining inductively|)}

\begin{exercise}\label{ex:converse-rtc-step}
Show that the converse of \isa{rtc{\isaliteral{5F}{\isacharunderscore}}step} also holds:
\begin{isabelle}%
\ \ \ \ \ {\isaliteral{5C3C6C6272616B6B3E}{\isasymlbrakk}}{\isaliteral{28}{\isacharparenleft}}x{\isaliteral{2C}{\isacharcomma}}\ y{\isaliteral{29}{\isacharparenright}}\ {\isaliteral{5C3C696E3E}{\isasymin}}\ r{\isaliteral{2A}{\isacharasterisk}}{\isaliteral{3B}{\isacharsemicolon}}\ {\isaliteral{28}{\isacharparenleft}}y{\isaliteral{2C}{\isacharcomma}}\ z{\isaliteral{29}{\isacharparenright}}\ {\isaliteral{5C3C696E3E}{\isasymin}}\ r{\isaliteral{5C3C726272616B6B3E}{\isasymrbrakk}}\ {\isaliteral{5C3C4C6F6E6772696768746172726F773E}{\isasymLongrightarrow}}\ {\isaliteral{28}{\isacharparenleft}}x{\isaliteral{2C}{\isacharcomma}}\ z{\isaliteral{29}{\isacharparenright}}\ {\isaliteral{5C3C696E3E}{\isasymin}}\ r{\isaliteral{2A}{\isacharasterisk}}%
\end{isabelle}
\end{exercise}
\begin{exercise}
Repeat the development of this section, but starting with a definition of
\isa{rtc} where \isa{rtc{\isaliteral{5F}{\isacharunderscore}}step} is replaced by its converse as shown
in exercise~\ref{ex:converse-rtc-step}.
\end{exercise}%
\end{isamarkuptext}%
\isamarkuptrue%
%
\isadelimproof
%
\endisadelimproof
%
\isatagproof
%
\endisatagproof
{\isafoldproof}%
%
\isadelimproof
%
\endisadelimproof
%
\isadelimtheory
%
\endisadelimtheory
%
\isatagtheory
%
\endisatagtheory
{\isafoldtheory}%
%
\isadelimtheory
%
\endisadelimtheory
\end{isabellebody}%
%%% Local Variables:
%%% mode: latex
%%% TeX-master: "root"
%%% End:


\section{Advanced Inductive Definitions}
\label{sec:adv-ind-def}
%
\begin{isabellebody}%
\def\isabellecontext{Advanced}%
%
\isadelimtheory
%
\endisadelimtheory
%
\isatagtheory
%
\endisatagtheory
{\isafoldtheory}%
%
\isadelimtheory
%
\endisadelimtheory
%
\begin{isamarkuptext}%
The premises of introduction rules may contain universal quantifiers and
monotone functions.  A universal quantifier lets the rule 
refer to any number of instances of 
the inductively defined set.  A monotone function lets the rule refer
to existing constructions (such as ``list of'') over the inductively defined
set.  The examples below show how to use the additional expressiveness
and how to reason from the resulting definitions.%
\end{isamarkuptext}%
\isamarkuptrue%
%
\isamarkupsubsection{Universal Quantifiers in Introduction Rules \label{sec:gterm-datatype}%
}
\isamarkuptrue%
%
\begin{isamarkuptext}%
\index{ground terms example|(}%
\index{quantifiers!and inductive definitions|(}%
As a running example, this section develops the theory of \textbf{ground
terms}: terms constructed from constant and function 
symbols but not variables. To simplify matters further, we regard a
constant as a function applied to the null argument  list.  Let us declare a
datatype \isa{gterm} for the type of ground  terms. It is a type constructor
whose argument is a type of  function symbols.%
\end{isamarkuptext}%
\isamarkuptrue%
\isacommand{datatype}\isamarkupfalse%
\ {\isacharprime}f\ gterm\ {\isacharequal}\ Apply\ {\isacharprime}f\ {\isachardoublequoteopen}{\isacharprime}f\ gterm\ list{\isachardoublequoteclose}%
\begin{isamarkuptext}%
To try it out, we declare a datatype of some integer operations: 
integer constants, the unary minus operator and the addition 
operator.%
\end{isamarkuptext}%
\isamarkuptrue%
\isacommand{datatype}\isamarkupfalse%
\ integer{\isacharunderscore}op\ {\isacharequal}\ Number\ int\ {\isacharbar}\ UnaryMinus\ {\isacharbar}\ Plus%
\begin{isamarkuptext}%
Now the type \isa{integer{\isacharunderscore}op\ gterm} denotes the ground 
terms built over those symbols.

The type constructor \isa{gterm} can be generalized to a function 
over sets.  It returns 
the set of ground terms that can be formed over a set \isa{F} of function symbols. For
example,  we could consider the set of ground terms formed from the finite 
set \isa{{\isacharbraceleft}Number\ {\isadigit{2}}{\isacharcomma}\ UnaryMinus{\isacharcomma}\ Plus{\isacharbraceright}}.

This concept is inductive. If we have a list \isa{args} of ground terms 
over~\isa{F} and a function symbol \isa{f} in \isa{F}, then we 
can apply \isa{f} to \isa{args} to obtain another ground term. 
The only difficulty is that the argument list may be of any length. Hitherto, 
each rule in an inductive definition referred to the inductively 
defined set a fixed number of times, typically once or twice. 
A universal quantifier in the premise of the introduction rule 
expresses that every element of \isa{args} belongs
to our inductively defined set: is a ground term 
over~\isa{F}.  The function \isa{set} denotes the set of elements in a given 
list.%
\end{isamarkuptext}%
\isamarkuptrue%
\isacommand{inductive{\isacharunderscore}set}\isamarkupfalse%
\isanewline
\ \ gterms\ {\isacharcolon}{\isacharcolon}\ {\isachardoublequoteopen}{\isacharprime}f\ set\ {\isasymRightarrow}\ {\isacharprime}f\ gterm\ set{\isachardoublequoteclose}\isanewline
\ \ \isakeyword{for}\ F\ {\isacharcolon}{\isacharcolon}\ {\isachardoublequoteopen}{\isacharprime}f\ set{\isachardoublequoteclose}\isanewline
\isakeyword{where}\isanewline
step{\isacharbrackleft}intro{\isacharbang}{\isacharbrackright}{\isacharcolon}\ {\isachardoublequoteopen}{\isasymlbrakk}{\isasymforall}t\ {\isasymin}\ set\ args{\isachardot}\ t\ {\isasymin}\ gterms\ F{\isacharsemicolon}\ \ f\ {\isasymin}\ F{\isasymrbrakk}\isanewline
\ \ \ \ \ \ \ \ \ \ \ \ \ \ \ {\isasymLongrightarrow}\ {\isacharparenleft}Apply\ f\ args{\isacharparenright}\ {\isasymin}\ gterms\ F{\isachardoublequoteclose}%
\begin{isamarkuptext}%
To demonstrate a proof from this definition, let us 
show that the function \isa{gterms}
is \textbf{monotone}.  We shall need this concept shortly.%
\end{isamarkuptext}%
\isamarkuptrue%
\isacommand{lemma}\isamarkupfalse%
\ gterms{\isacharunderscore}mono{\isacharcolon}\ {\isachardoublequoteopen}F{\isasymsubseteq}G\ {\isasymLongrightarrow}\ gterms\ F\ {\isasymsubseteq}\ gterms\ G{\isachardoublequoteclose}\isanewline
%
\isadelimproof
%
\endisadelimproof
%
\isatagproof
\isacommand{apply}\isamarkupfalse%
\ clarify\isanewline
\isacommand{apply}\isamarkupfalse%
\ {\isacharparenleft}erule\ gterms{\isachardot}induct{\isacharparenright}\isanewline
\isacommand{apply}\isamarkupfalse%
\ blast\isanewline
\isacommand{done}\isamarkupfalse%
%
\endisatagproof
{\isafoldproof}%
%
\isadelimproof
%
\endisadelimproof
%
\isadelimproof
%
\endisadelimproof
%
\isatagproof
%
\begin{isamarkuptxt}%
Intuitively, this theorem says that
enlarging the set of function symbols enlarges the set of ground 
terms. The proof is a trivial rule induction.
First we use the \isa{clarify} method to assume the existence of an element of
\isa{gterms\ F}.  (We could have used \isa{intro\ subsetI}.)  We then
apply rule induction. Here is the resulting subgoal:
\begin{isabelle}%
\ {\isadigit{1}}{\isachardot}\ {\isasymAnd}x\ args\ f{\isachardot}\isanewline
\isaindent{\ {\isadigit{1}}{\isachardot}\ \ \ \ }{\isasymlbrakk}F\ {\isasymsubseteq}\ G{\isacharsemicolon}\ {\isasymforall}t{\isasymin}set\ args{\isachardot}\ t\ {\isasymin}\ gterms\ F\ {\isasymand}\ t\ {\isasymin}\ gterms\ G{\isacharsemicolon}\ f\ {\isasymin}\ F{\isasymrbrakk}\isanewline
\isaindent{\ {\isadigit{1}}{\isachardot}\ \ \ \ }{\isasymLongrightarrow}\ Apply\ f\ args\ {\isasymin}\ gterms\ G%
\end{isabelle}
The assumptions state that \isa{f} belongs 
to~\isa{F}, which is included in~\isa{G}, and that every element of the list \isa{args} is
a ground term over~\isa{G}.  The \isa{blast} method finds this chain of reasoning easily.%
\end{isamarkuptxt}%
\isamarkuptrue%
%
\endisatagproof
{\isafoldproof}%
%
\isadelimproof
%
\endisadelimproof
%
\begin{isamarkuptext}%
\begin{warn}
Why do we call this function \isa{gterms} instead 
of \isa{gterm}?  A constant may have the same name as a type.  However,
name  clashes could arise in the theorems that Isabelle generates. 
Our choice of names keeps \isa{gterms{\isachardot}induct} separate from 
\isa{gterm{\isachardot}induct}.
\end{warn}

Call a term \textbf{well-formed} if each symbol occurring in it is applied
to the correct number of arguments.  (This number is called the symbol's
\textbf{arity}.)  We can express well-formedness by
generalizing the inductive definition of
\isa{gterms}.
Suppose we are given a function called \isa{arity}, specifying the arities
of all symbols.  In the inductive step, we have a list \isa{args} of such
terms and a function  symbol~\isa{f}. If the length of the list matches the
function's arity  then applying \isa{f} to \isa{args} yields a well-formed
term.%
\end{isamarkuptext}%
\isamarkuptrue%
\isacommand{inductive{\isacharunderscore}set}\isamarkupfalse%
\isanewline
\ \ well{\isacharunderscore}formed{\isacharunderscore}gterm\ {\isacharcolon}{\isacharcolon}\ {\isachardoublequoteopen}{\isacharparenleft}{\isacharprime}f\ {\isasymRightarrow}\ nat{\isacharparenright}\ {\isasymRightarrow}\ {\isacharprime}f\ gterm\ set{\isachardoublequoteclose}\isanewline
\ \ \isakeyword{for}\ arity\ {\isacharcolon}{\isacharcolon}\ {\isachardoublequoteopen}{\isacharprime}f\ {\isasymRightarrow}\ nat{\isachardoublequoteclose}\isanewline
\isakeyword{where}\isanewline
step{\isacharbrackleft}intro{\isacharbang}{\isacharbrackright}{\isacharcolon}\ {\isachardoublequoteopen}{\isasymlbrakk}{\isasymforall}t\ {\isasymin}\ set\ args{\isachardot}\ t\ {\isasymin}\ well{\isacharunderscore}formed{\isacharunderscore}gterm\ arity{\isacharsemicolon}\ \ \isanewline
\ \ \ \ \ \ \ \ \ \ \ \ \ \ \ \ length\ args\ {\isacharequal}\ arity\ f{\isasymrbrakk}\isanewline
\ \ \ \ \ \ \ \ \ \ \ \ \ \ \ {\isasymLongrightarrow}\ {\isacharparenleft}Apply\ f\ args{\isacharparenright}\ {\isasymin}\ well{\isacharunderscore}formed{\isacharunderscore}gterm\ arity{\isachardoublequoteclose}%
\begin{isamarkuptext}%
The inductive definition neatly captures the reasoning above.
The universal quantification over the
\isa{set} of arguments expresses that all of them are well-formed.%
\index{quantifiers!and inductive definitions|)}%
\end{isamarkuptext}%
\isamarkuptrue%
%
\isamarkupsubsection{Alternative Definition Using a Monotone Function%
}
\isamarkuptrue%
%
\begin{isamarkuptext}%
\index{monotone functions!and inductive definitions|(}% 
An inductive definition may refer to the
inductively defined  set through an arbitrary monotone function.  To
demonstrate this powerful feature, let us
change the  inductive definition above, replacing the
quantifier by a use of the function \isa{lists}. This
function, from the Isabelle theory of lists, is analogous to the
function \isa{gterms} declared above: if \isa{A} is a set then
\isa{lists\ A} is the set of lists whose elements belong to
\isa{A}.  

In the inductive definition of well-formed terms, examine the one
introduction rule.  The first premise states that \isa{args} belongs to
the \isa{lists} of well-formed terms.  This formulation is more
direct, if more obscure, than using a universal quantifier.%
\end{isamarkuptext}%
\isamarkuptrue%
\isacommand{inductive{\isacharunderscore}set}\isamarkupfalse%
\isanewline
\ \ well{\isacharunderscore}formed{\isacharunderscore}gterm{\isacharprime}\ {\isacharcolon}{\isacharcolon}\ {\isachardoublequoteopen}{\isacharparenleft}{\isacharprime}f\ {\isasymRightarrow}\ nat{\isacharparenright}\ {\isasymRightarrow}\ {\isacharprime}f\ gterm\ set{\isachardoublequoteclose}\isanewline
\ \ \isakeyword{for}\ arity\ {\isacharcolon}{\isacharcolon}\ {\isachardoublequoteopen}{\isacharprime}f\ {\isasymRightarrow}\ nat{\isachardoublequoteclose}\isanewline
\isakeyword{where}\isanewline
step{\isacharbrackleft}intro{\isacharbang}{\isacharbrackright}{\isacharcolon}\ {\isachardoublequoteopen}{\isasymlbrakk}args\ {\isasymin}\ lists\ {\isacharparenleft}well{\isacharunderscore}formed{\isacharunderscore}gterm{\isacharprime}\ arity{\isacharparenright}{\isacharsemicolon}\ \ \isanewline
\ \ \ \ \ \ \ \ \ \ \ \ \ \ \ \ length\ args\ {\isacharequal}\ arity\ f{\isasymrbrakk}\isanewline
\ \ \ \ \ \ \ \ \ \ \ \ \ \ \ {\isasymLongrightarrow}\ {\isacharparenleft}Apply\ f\ args{\isacharparenright}\ {\isasymin}\ well{\isacharunderscore}formed{\isacharunderscore}gterm{\isacharprime}\ arity{\isachardoublequoteclose}\isanewline
\isakeyword{monos}\ lists{\isacharunderscore}mono%
\begin{isamarkuptext}%
We cite the theorem \isa{lists{\isacharunderscore}mono} to justify 
using the function \isa{lists}.%
\footnote{This particular theorem is installed by default already, but we
include the \isakeyword{monos} declaration in order to illustrate its syntax.}
\begin{isabelle}%
A\ {\isasymsubseteq}\ B\ {\isasymLongrightarrow}\ lists\ A\ {\isasymsubseteq}\ lists\ B\rulename{lists{\isacharunderscore}mono}%
\end{isabelle}
Why must the function be monotone?  An inductive definition describes
an iterative construction: each element of the set is constructed by a
finite number of introduction rule applications.  For example, the
elements of \isa{even} are constructed by finitely many applications of
the rules
\begin{isabelle}%
{\isadigit{0}}\ {\isasymin}\ even\isasep\isanewline%
n\ {\isasymin}\ even\ {\isasymLongrightarrow}\ Suc\ {\isacharparenleft}Suc\ n{\isacharparenright}\ {\isasymin}\ even%
\end{isabelle}
All references to a set in its
inductive definition must be positive.  Applications of an
introduction rule cannot invalidate previous applications, allowing the
construction process to converge.
The following pair of rules do not constitute an inductive definition:
\begin{trivlist}
\item \isa{{\isadigit{0}}\ {\isasymin}\ even}
\item \isa{n\ {\isasymnotin}\ even\ {\isasymLongrightarrow}\ Suc\ n\ {\isasymin}\ even}
\end{trivlist}
Showing that 4 is even using these rules requires showing that 3 is not
even.  It is far from trivial to show that this set of rules
characterizes the even numbers.  

Even with its use of the function \isa{lists}, the premise of our
introduction rule is positive:
\begin{isabelle}%
args\ {\isasymin}\ lists\ {\isacharparenleft}well{\isacharunderscore}formed{\isacharunderscore}gterm{\isacharprime}\ arity{\isacharparenright}%
\end{isabelle}
To apply the rule we construct a list \isa{args} of previously
constructed well-formed terms.  We obtain a
new term, \isa{Apply\ f\ args}.  Because \isa{lists} is monotone,
applications of the rule remain valid as new terms are constructed.
Further lists of well-formed
terms become available and none are taken away.%
\index{monotone functions!and inductive definitions|)}%
\end{isamarkuptext}%
\isamarkuptrue%
%
\isamarkupsubsection{A Proof of Equivalence%
}
\isamarkuptrue%
%
\begin{isamarkuptext}%
We naturally hope that these two inductive definitions of ``well-formed'' 
coincide.  The equality can be proved by separate inclusions in 
each direction.  Each is a trivial rule induction.%
\end{isamarkuptext}%
\isamarkuptrue%
\isacommand{lemma}\isamarkupfalse%
\ {\isachardoublequoteopen}well{\isacharunderscore}formed{\isacharunderscore}gterm\ arity\ {\isasymsubseteq}\ well{\isacharunderscore}formed{\isacharunderscore}gterm{\isacharprime}\ arity{\isachardoublequoteclose}\isanewline
%
\isadelimproof
%
\endisadelimproof
%
\isatagproof
\isacommand{apply}\isamarkupfalse%
\ clarify\isanewline
\isacommand{apply}\isamarkupfalse%
\ {\isacharparenleft}erule\ well{\isacharunderscore}formed{\isacharunderscore}gterm{\isachardot}induct{\isacharparenright}\isanewline
\isacommand{apply}\isamarkupfalse%
\ auto\isanewline
\isacommand{done}\isamarkupfalse%
%
\endisatagproof
{\isafoldproof}%
%
\isadelimproof
%
\endisadelimproof
%
\isadelimproof
%
\endisadelimproof
%
\isatagproof
%
\begin{isamarkuptxt}%
The \isa{clarify} method gives
us an element of \isa{well{\isacharunderscore}formed{\isacharunderscore}gterm\ arity} on which to perform 
induction.  The resulting subgoal can be proved automatically:
\begin{isabelle}%
\ {\isadigit{1}}{\isachardot}\ {\isasymAnd}x\ args\ f{\isachardot}\isanewline
\isaindent{\ {\isadigit{1}}{\isachardot}\ \ \ \ }{\isasymlbrakk}{\isasymforall}t{\isasymin}set\ args{\isachardot}\isanewline
\isaindent{\ {\isadigit{1}}{\isachardot}\ \ \ \ {\isasymlbrakk}\ \ \ }t\ {\isasymin}\ well{\isacharunderscore}formed{\isacharunderscore}gterm\ arity\ {\isasymand}\ t\ {\isasymin}\ well{\isacharunderscore}formed{\isacharunderscore}gterm{\isacharprime}\ arity{\isacharsemicolon}\isanewline
\isaindent{\ {\isadigit{1}}{\isachardot}\ \ \ \ \ }length\ args\ {\isacharequal}\ arity\ f{\isasymrbrakk}\isanewline
\isaindent{\ {\isadigit{1}}{\isachardot}\ \ \ \ }{\isasymLongrightarrow}\ Apply\ f\ args\ {\isasymin}\ well{\isacharunderscore}formed{\isacharunderscore}gterm{\isacharprime}\ arity%
\end{isabelle}
This proof resembles the one given in
{\S}\ref{sec:gterm-datatype} above, especially in the form of the
induction hypothesis.  Next, we consider the opposite inclusion:%
\end{isamarkuptxt}%
\isamarkuptrue%
%
\endisatagproof
{\isafoldproof}%
%
\isadelimproof
%
\endisadelimproof
\isacommand{lemma}\isamarkupfalse%
\ {\isachardoublequoteopen}well{\isacharunderscore}formed{\isacharunderscore}gterm{\isacharprime}\ arity\ {\isasymsubseteq}\ well{\isacharunderscore}formed{\isacharunderscore}gterm\ arity{\isachardoublequoteclose}\isanewline
%
\isadelimproof
%
\endisadelimproof
%
\isatagproof
\isacommand{apply}\isamarkupfalse%
\ clarify\isanewline
\isacommand{apply}\isamarkupfalse%
\ {\isacharparenleft}erule\ well{\isacharunderscore}formed{\isacharunderscore}gterm{\isacharprime}{\isachardot}induct{\isacharparenright}\isanewline
\isacommand{apply}\isamarkupfalse%
\ auto\isanewline
\isacommand{done}\isamarkupfalse%
%
\endisatagproof
{\isafoldproof}%
%
\isadelimproof
%
\endisadelimproof
%
\isadelimproof
%
\endisadelimproof
%
\isatagproof
%
\begin{isamarkuptxt}%
The proof script is virtually identical,
but the subgoal after applying induction may be surprising:
\begin{isabelle}%
\ {\isadigit{1}}{\isachardot}\ {\isasymAnd}x\ args\ f{\isachardot}\isanewline
\isaindent{\ {\isadigit{1}}{\isachardot}\ \ \ \ }{\isasymlbrakk}args\isanewline
\isaindent{\ {\isadigit{1}}{\isachardot}\ \ \ \ {\isasymlbrakk}}{\isasymin}\ lists\isanewline
\isaindent{\ {\isadigit{1}}{\isachardot}\ \ \ \ {\isasymlbrakk}{\isasymin}\ \ }{\isacharparenleft}well{\isacharunderscore}formed{\isacharunderscore}gterm{\isacharprime}\ arity\ {\isasyminter}\isanewline
\isaindent{\ {\isadigit{1}}{\isachardot}\ \ \ \ {\isasymlbrakk}{\isasymin}\ \ {\isacharparenleft}}{\isacharbraceleft}a{\isachardot}\ a\ {\isasymin}\ well{\isacharunderscore}formed{\isacharunderscore}gterm\ arity{\isacharbraceright}{\isacharparenright}{\isacharsemicolon}\isanewline
\isaindent{\ {\isadigit{1}}{\isachardot}\ \ \ \ \ }length\ args\ {\isacharequal}\ arity\ f{\isasymrbrakk}\isanewline
\isaindent{\ {\isadigit{1}}{\isachardot}\ \ \ \ }{\isasymLongrightarrow}\ Apply\ f\ args\ {\isasymin}\ well{\isacharunderscore}formed{\isacharunderscore}gterm\ arity%
\end{isabelle}
The induction hypothesis contains an application of \isa{lists}.  Using a
monotone function in the inductive definition always has this effect.  The
subgoal may look uninviting, but fortunately 
\isa{lists} distributes over intersection:
\begin{isabelle}%
listsp\ {\isacharparenleft}{\isacharparenleft}{\isasymlambda}x{\isachardot}\ x\ {\isasymin}\ A{\isacharparenright}\ {\isasyminter}\ {\isacharparenleft}{\isasymlambda}x{\isachardot}\ x\ {\isasymin}\ B{\isacharparenright}{\isacharparenright}\ {\isacharequal}\ {\isacharparenleft}{\isasymlambda}x{\isachardot}\ x\ {\isasymin}\ lists\ A{\isacharparenright}\ {\isasyminter}\ {\isacharparenleft}{\isasymlambda}x{\isachardot}\ x\ {\isasymin}\ lists\ B{\isacharparenright}\rulename{lists{\isacharunderscore}Int{\isacharunderscore}eq}%
\end{isabelle}
Thanks to this default simplification rule, the induction hypothesis 
is quickly replaced by its two parts:
\begin{trivlist}
\item \isa{args\ {\isasymin}\ lists\ {\isacharparenleft}well{\isacharunderscore}formed{\isacharunderscore}gterm{\isacharprime}\ arity{\isacharparenright}}
\item \isa{args\ {\isasymin}\ lists\ {\isacharparenleft}well{\isacharunderscore}formed{\isacharunderscore}gterm\ arity{\isacharparenright}}
\end{trivlist}
Invoking the rule \isa{well{\isacharunderscore}formed{\isacharunderscore}gterm{\isachardot}step} completes the proof.  The
call to \isa{auto} does all this work.

This example is typical of how monotone functions
\index{monotone functions} can be used.  In particular, many of them
distribute over intersection.  Monotonicity implies one direction of
this set equality; we have this theorem:
\begin{isabelle}%
mono\ f\ {\isasymLongrightarrow}\ f\ {\isacharparenleft}A\ {\isasyminter}\ B{\isacharparenright}\ {\isasymsubseteq}\ f\ A\ {\isasyminter}\ f\ B\rulename{mono{\isacharunderscore}Int}%
\end{isabelle}%
\end{isamarkuptxt}%
\isamarkuptrue%
%
\endisatagproof
{\isafoldproof}%
%
\isadelimproof
%
\endisadelimproof
%
\isamarkupsubsection{Another Example of Rule Inversion%
}
\isamarkuptrue%
%
\begin{isamarkuptext}%
\index{rule inversion|(}%
Does \isa{gterms} distribute over intersection?  We have proved that this
function is monotone, so \isa{mono{\isacharunderscore}Int} gives one of the inclusions.  The
opposite inclusion asserts that if \isa{t} is a ground term over both of the
sets
\isa{F} and~\isa{G} then it is also a ground term over their intersection,
\isa{F\ {\isasyminter}\ G}.%
\end{isamarkuptext}%
\isamarkuptrue%
\isacommand{lemma}\isamarkupfalse%
\ gterms{\isacharunderscore}IntI{\isacharcolon}\isanewline
\ \ \ \ \ {\isachardoublequoteopen}t\ {\isasymin}\ gterms\ F\ {\isasymLongrightarrow}\ t\ {\isasymin}\ gterms\ G\ {\isasymlongrightarrow}\ t\ {\isasymin}\ gterms\ {\isacharparenleft}F{\isasyminter}G{\isacharparenright}{\isachardoublequoteclose}%
\isadelimproof
%
\endisadelimproof
%
\isatagproof
%
\endisatagproof
{\isafoldproof}%
%
\isadelimproof
%
\endisadelimproof
%
\begin{isamarkuptext}%
Attempting this proof, we get the assumption 
\isa{Apply\ f\ args\ {\isasymin}\ gterms\ G}, which cannot be broken down. 
It looks like a job for rule inversion:\cmmdx{inductive\protect\_cases}%
\end{isamarkuptext}%
\isamarkuptrue%
\isacommand{inductive{\isacharunderscore}cases}\isamarkupfalse%
\ gterm{\isacharunderscore}Apply{\isacharunderscore}elim\ {\isacharbrackleft}elim{\isacharbang}{\isacharbrackright}{\isacharcolon}\ {\isachardoublequoteopen}Apply\ f\ args\ {\isasymin}\ gterms\ F{\isachardoublequoteclose}%
\begin{isamarkuptext}%
Here is the result.
\begin{isabelle}%
{\isasymlbrakk}Apply\ f\ args\ {\isasymin}\ gterms\ F{\isacharsemicolon}\isanewline
\isaindent{\ }{\isasymlbrakk}{\isasymforall}t{\isasymin}set\ args{\isachardot}\ t\ {\isasymin}\ gterms\ F{\isacharsemicolon}\ f\ {\isasymin}\ F{\isasymrbrakk}\ {\isasymLongrightarrow}\ P{\isasymrbrakk}\isanewline
{\isasymLongrightarrow}\ P\rulename{gterm{\isacharunderscore}Apply{\isacharunderscore}elim}%
\end{isabelle}
This rule replaces an assumption about \isa{Apply\ f\ args} by 
assumptions about \isa{f} and~\isa{args}.  
No cases are discarded (there was only one to begin
with) but the rule applies specifically to the pattern \isa{Apply\ f\ args}.
It can be applied repeatedly as an elimination rule without looping, so we
have given the \isa{elim{\isacharbang}} attribute. 

Now we can prove the other half of that distributive law.%
\end{isamarkuptext}%
\isamarkuptrue%
\isacommand{lemma}\isamarkupfalse%
\ gterms{\isacharunderscore}IntI\ {\isacharbrackleft}rule{\isacharunderscore}format{\isacharcomma}\ intro{\isacharbang}{\isacharbrackright}{\isacharcolon}\isanewline
\ \ \ \ \ {\isachardoublequoteopen}t\ {\isasymin}\ gterms\ F\ {\isasymLongrightarrow}\ t\ {\isasymin}\ gterms\ G\ {\isasymlongrightarrow}\ t\ {\isasymin}\ gterms\ {\isacharparenleft}F{\isasyminter}G{\isacharparenright}{\isachardoublequoteclose}\isanewline
%
\isadelimproof
%
\endisadelimproof
%
\isatagproof
\isacommand{apply}\isamarkupfalse%
\ {\isacharparenleft}erule\ gterms{\isachardot}induct{\isacharparenright}\isanewline
\isacommand{apply}\isamarkupfalse%
\ blast\isanewline
\isacommand{done}\isamarkupfalse%
%
\endisatagproof
{\isafoldproof}%
%
\isadelimproof
%
\endisadelimproof
%
\isadelimproof
%
\endisadelimproof
%
\isatagproof
%
\begin{isamarkuptxt}%
The proof begins with rule induction over the definition of
\isa{gterms}, which leaves a single subgoal:  
\begin{isabelle}%
\ {\isadigit{1}}{\isachardot}\ {\isasymAnd}args\ f{\isachardot}\isanewline
\isaindent{\ {\isadigit{1}}{\isachardot}\ \ \ \ }{\isasymlbrakk}{\isasymforall}t{\isasymin}set\ args{\isachardot}\isanewline
\isaindent{\ {\isadigit{1}}{\isachardot}\ \ \ \ {\isasymlbrakk}\ \ \ }t\ {\isasymin}\ gterms\ F\ {\isasymand}\ {\isacharparenleft}t\ {\isasymin}\ gterms\ G\ {\isasymlongrightarrow}\ t\ {\isasymin}\ gterms\ {\isacharparenleft}F\ {\isasyminter}\ G{\isacharparenright}{\isacharparenright}{\isacharsemicolon}\isanewline
\isaindent{\ {\isadigit{1}}{\isachardot}\ \ \ \ \ }f\ {\isasymin}\ F{\isasymrbrakk}\isanewline
\isaindent{\ {\isadigit{1}}{\isachardot}\ \ \ \ }{\isasymLongrightarrow}\ Apply\ f\ args\ {\isasymin}\ gterms\ G\ {\isasymlongrightarrow}\isanewline
\isaindent{\ {\isadigit{1}}{\isachardot}\ \ \ \ {\isasymLongrightarrow}\ }Apply\ f\ args\ {\isasymin}\ gterms\ {\isacharparenleft}F\ {\isasyminter}\ G{\isacharparenright}%
\end{isabelle}
To prove this, we assume \isa{Apply\ f\ args\ {\isasymin}\ gterms\ G}.  Rule inversion,
in the form of \isa{gterm{\isacharunderscore}Apply{\isacharunderscore}elim}, infers
that every element of \isa{args} belongs to 
\isa{gterms\ G}; hence (by the induction hypothesis) it belongs
to \isa{gterms\ {\isacharparenleft}F\ {\isasyminter}\ G{\isacharparenright}}.  Rule inversion also yields
\isa{f\ {\isasymin}\ G} and hence \isa{f\ {\isasymin}\ F\ {\isasyminter}\ G}. 
All of this reasoning is done by \isa{blast}.

\smallskip
Our distributive law is a trivial consequence of previously-proved results:%
\end{isamarkuptxt}%
\isamarkuptrue%
%
\endisatagproof
{\isafoldproof}%
%
\isadelimproof
%
\endisadelimproof
\isacommand{lemma}\isamarkupfalse%
\ gterms{\isacharunderscore}Int{\isacharunderscore}eq\ {\isacharbrackleft}simp{\isacharbrackright}{\isacharcolon}\isanewline
\ \ \ \ \ {\isachardoublequoteopen}gterms\ {\isacharparenleft}F\ {\isasyminter}\ G{\isacharparenright}\ {\isacharequal}\ gterms\ F\ {\isasyminter}\ gterms\ G{\isachardoublequoteclose}\isanewline
%
\isadelimproof
%
\endisadelimproof
%
\isatagproof
\isacommand{by}\isamarkupfalse%
\ {\isacharparenleft}blast\ intro{\isacharbang}{\isacharcolon}\ mono{\isacharunderscore}Int\ monoI\ gterms{\isacharunderscore}mono{\isacharparenright}%
\endisatagproof
{\isafoldproof}%
%
\isadelimproof
%
\endisadelimproof
%
\index{rule inversion|)}%
\index{ground terms example|)}


\begin{isamarkuptext}
\begin{exercise}
A function mapping function symbols to their 
types is called a \textbf{signature}.  Given a type 
ranging over type symbols, we can represent a function's type by a
list of argument types paired with the result type. 
Complete this inductive definition:
\begin{isabelle}
\isacommand{inductive{\isacharunderscore}set}\isamarkupfalse%
\isanewline
\ \ well{\isacharunderscore}typed{\isacharunderscore}gterm\ {\isacharcolon}{\isacharcolon}\ {\isachardoublequoteopen}{\isacharparenleft}{\isacharprime}f\ {\isasymRightarrow}\ {\isacharprime}t\ list\ {\isacharasterisk}\ {\isacharprime}t{\isacharparenright}\ {\isasymRightarrow}\ {\isacharparenleft}{\isacharprime}f\ gterm\ {\isacharasterisk}\ {\isacharprime}t{\isacharparenright}set{\isachardoublequoteclose}\isanewline
\ \ \isakeyword{for}\ sig\ {\isacharcolon}{\isacharcolon}\ {\isachardoublequoteopen}{\isacharprime}f\ {\isasymRightarrow}\ {\isacharprime}t\ list\ {\isacharasterisk}\ {\isacharprime}t{\isachardoublequoteclose}%
\end{isabelle}
\end{exercise}
\end{isamarkuptext}
%
\isadelimproof
%
\endisadelimproof
%
\isatagproof
%
\endisatagproof
{\isafoldproof}%
%
\isadelimproof
%
\endisadelimproof
%
\isadelimproof
%
\endisadelimproof
%
\isatagproof
%
\endisatagproof
{\isafoldproof}%
%
\isadelimproof
%
\endisadelimproof
%
\isadelimtheory
%
\endisadelimtheory
%
\isatagtheory
%
\endisatagtheory
{\isafoldtheory}%
%
\isadelimtheory
%
\endisadelimtheory
\end{isabellebody}%
%%% Local Variables:
%%% mode: latex
%%% TeX-master: "root"
%%% End:


%
\begin{isabellebody}%
\def\isabellecontext{AB}%
%
\isamarkupsection{Case Study: A Context Free Grammar%
}
%
\begin{isamarkuptext}%
\label{sec:CFG}
\index{grammars!defining inductively|(}%
Grammars are nothing but shorthands for inductive definitions of nonterminals
which represent sets of strings. For example, the production
$A \to B c$ is short for
\[ w \in B \Longrightarrow wc \in A \]
This section demonstrates this idea with an example
due to Hopcroft and Ullman, a grammar for generating all words with an
equal number of $a$'s and~$b$'s:
\begin{eqnarray}
S &\to& \epsilon \mid b A \mid a B \nonumber\\
A &\to& a S \mid b A A \nonumber\\
B &\to& b S \mid a B B \nonumber
\end{eqnarray}
At the end we say a few words about the relationship between
the original proof \cite[p.\ts81]{HopcroftUllman} and our formal version.

We start by fixing the alphabet, which consists only of \isa{a}'s
and~\isa{b}'s:%
\end{isamarkuptext}%
\isacommand{datatype}\ alfa\ {\isacharequal}\ a\ {\isacharbar}\ b%
\begin{isamarkuptext}%
\noindent
For convenience we include the following easy lemmas as simplification rules:%
\end{isamarkuptext}%
\isacommand{lemma}\ {\isacharbrackleft}simp{\isacharbrackright}{\isacharcolon}\ {\isachardoublequote}{\isacharparenleft}x\ {\isasymnoteq}\ a{\isacharparenright}\ {\isacharequal}\ {\isacharparenleft}x\ {\isacharequal}\ b{\isacharparenright}\ {\isasymand}\ {\isacharparenleft}x\ {\isasymnoteq}\ b{\isacharparenright}\ {\isacharequal}\ {\isacharparenleft}x\ {\isacharequal}\ a{\isacharparenright}{\isachardoublequote}\isanewline
\isacommand{by}\ {\isacharparenleft}case{\isacharunderscore}tac\ x{\isacharcomma}\ auto{\isacharparenright}%
\begin{isamarkuptext}%
\noindent
Words over this alphabet are of type \isa{alfa\ list}, and
the three nonterminals are declared as sets of such words:%
\end{isamarkuptext}%
\isacommand{consts}\ S\ {\isacharcolon}{\isacharcolon}\ {\isachardoublequote}alfa\ list\ set{\isachardoublequote}\isanewline
\ \ \ \ \ \ \ A\ {\isacharcolon}{\isacharcolon}\ {\isachardoublequote}alfa\ list\ set{\isachardoublequote}\isanewline
\ \ \ \ \ \ \ B\ {\isacharcolon}{\isacharcolon}\ {\isachardoublequote}alfa\ list\ set{\isachardoublequote}%
\begin{isamarkuptext}%
\noindent
The productions above are recast as a \emph{mutual} inductive
definition\index{inductive definition!simultaneous}
of \isa{S}, \isa{A} and~\isa{B}:%
\end{isamarkuptext}%
\isacommand{inductive}\ S\ A\ B\isanewline
\isakeyword{intros}\isanewline
\ \ {\isachardoublequote}{\isacharbrackleft}{\isacharbrackright}\ {\isasymin}\ S{\isachardoublequote}\isanewline
\ \ {\isachardoublequote}w\ {\isasymin}\ A\ {\isasymLongrightarrow}\ b{\isacharhash}w\ {\isasymin}\ S{\isachardoublequote}\isanewline
\ \ {\isachardoublequote}w\ {\isasymin}\ B\ {\isasymLongrightarrow}\ a{\isacharhash}w\ {\isasymin}\ S{\isachardoublequote}\isanewline
\isanewline
\ \ {\isachardoublequote}w\ {\isasymin}\ S\ \ \ \ \ \ \ \ {\isasymLongrightarrow}\ a{\isacharhash}w\ \ \ {\isasymin}\ A{\isachardoublequote}\isanewline
\ \ {\isachardoublequote}{\isasymlbrakk}\ v{\isasymin}A{\isacharsemicolon}\ w{\isasymin}A\ {\isasymrbrakk}\ {\isasymLongrightarrow}\ b{\isacharhash}v{\isacharat}w\ {\isasymin}\ A{\isachardoublequote}\isanewline
\isanewline
\ \ {\isachardoublequote}w\ {\isasymin}\ S\ \ \ \ \ \ \ \ \ \ \ \ {\isasymLongrightarrow}\ b{\isacharhash}w\ \ \ {\isasymin}\ B{\isachardoublequote}\isanewline
\ \ {\isachardoublequote}{\isasymlbrakk}\ v\ {\isasymin}\ B{\isacharsemicolon}\ w\ {\isasymin}\ B\ {\isasymrbrakk}\ {\isasymLongrightarrow}\ a{\isacharhash}v{\isacharat}w\ {\isasymin}\ B{\isachardoublequote}%
\begin{isamarkuptext}%
\noindent
First we show that all words in \isa{S} contain the same number of \isa{a}'s and \isa{b}'s. Since the definition of \isa{S} is by mutual
induction, so is the proof: we show at the same time that all words in
\isa{A} contain one more \isa{a} than \isa{b} and all words in \isa{B} contains one more \isa{b} than \isa{a}.%
\end{isamarkuptext}%
\isacommand{lemma}\ correctness{\isacharcolon}\isanewline
\ \ {\isachardoublequote}{\isacharparenleft}w\ {\isasymin}\ S\ {\isasymlongrightarrow}\ size{\isacharbrackleft}x{\isasymin}w{\isachardot}\ x{\isacharequal}a{\isacharbrackright}\ {\isacharequal}\ size{\isacharbrackleft}x{\isasymin}w{\isachardot}\ x{\isacharequal}b{\isacharbrackright}{\isacharparenright}\ \ \ \ \ {\isasymand}\isanewline
\ \ \ {\isacharparenleft}w\ {\isasymin}\ A\ {\isasymlongrightarrow}\ size{\isacharbrackleft}x{\isasymin}w{\isachardot}\ x{\isacharequal}a{\isacharbrackright}\ {\isacharequal}\ size{\isacharbrackleft}x{\isasymin}w{\isachardot}\ x{\isacharequal}b{\isacharbrackright}\ {\isacharplus}\ {\isadigit{1}}{\isacharparenright}\ {\isasymand}\isanewline
\ \ \ {\isacharparenleft}w\ {\isasymin}\ B\ {\isasymlongrightarrow}\ size{\isacharbrackleft}x{\isasymin}w{\isachardot}\ x{\isacharequal}b{\isacharbrackright}\ {\isacharequal}\ size{\isacharbrackleft}x{\isasymin}w{\isachardot}\ x{\isacharequal}a{\isacharbrackright}\ {\isacharplus}\ {\isadigit{1}}{\isacharparenright}{\isachardoublequote}%
\begin{isamarkuptxt}%
\noindent
These propositions are expressed with the help of the predefined \isa{filter} function on lists, which has the convenient syntax \isa{{\isacharbrackleft}x{\isasymin}xs{\isachardot}\ P\ x{\isacharbrackright}}, the list of all elements \isa{x} in \isa{xs} such that \isa{P\ x}
holds. Remember that on lists \isa{size} and \isa{length} are synonymous.

The proof itself is by rule induction and afterwards automatic:%
\end{isamarkuptxt}%
\isacommand{by}\ {\isacharparenleft}rule\ S{\isacharunderscore}A{\isacharunderscore}B{\isachardot}induct{\isacharcomma}\ auto{\isacharparenright}%
\begin{isamarkuptext}%
\noindent
This may seem surprising at first, and is indeed an indication of the power
of inductive definitions. But it is also quite straightforward. For example,
consider the production $A \to b A A$: if $v,w \in A$ and the elements of $A$
contain one more $a$ than~$b$'s, then $bvw$ must again contain one more $a$
than~$b$'s.

As usual, the correctness of syntactic descriptions is easy, but completeness
is hard: does \isa{S} contain \emph{all} words with an equal number of
\isa{a}'s and \isa{b}'s? It turns out that this proof requires the
following lemma: every string with two more \isa{a}'s than \isa{b}'s can be cut somewhere such that each half has one more \isa{a} than
\isa{b}. This is best seen by imagining counting the difference between the
number of \isa{a}'s and \isa{b}'s starting at the left end of the
word. We start with 0 and end (at the right end) with 2. Since each move to the
right increases or decreases the difference by 1, we must have passed through
1 on our way from 0 to 2. Formally, we appeal to the following discrete
intermediate value theorem \isa{nat{\isadigit{0}}{\isacharunderscore}intermed{\isacharunderscore}int{\isacharunderscore}val}
\begin{isabelle}%
\ \ \ \ \ {\isasymlbrakk}{\isasymforall}i{\isachardot}\ i\ {\isacharless}\ n\ {\isasymlongrightarrow}\ {\isasymbar}f\ {\isacharparenleft}i\ {\isacharplus}\ {\isadigit{1}}{\isacharparenright}\ {\isacharminus}\ f\ i{\isasymbar}\ {\isasymle}\ {\isacharhash}{\isadigit{1}}{\isacharsemicolon}\ f\ {\isadigit{0}}\ {\isasymle}\ k{\isacharsemicolon}\ k\ {\isasymle}\ f\ n{\isasymrbrakk}\isanewline
\isaindent{\ \ \ \ \ }{\isasymLongrightarrow}\ {\isasymexists}i{\isachardot}\ i\ {\isasymle}\ n\ {\isasymand}\ f\ i\ {\isacharequal}\ k%
\end{isabelle}
where \isa{f} is of type \isa{nat\ {\isasymRightarrow}\ int}, \isa{int} are the integers,
\isa{{\isasymbar}{\isachardot}{\isasymbar}} is the absolute value function\footnote{See
Table~\ref{tab:ascii} in the Appendix for the correct \textsc{ascii}
syntax.}, and \isa{{\isacharhash}{\isadigit{1}}} is the integer 1 (see \S\ref{sec:numbers}).

First we show that our specific function, the difference between the
numbers of \isa{a}'s and \isa{b}'s, does indeed only change by 1 in every
move to the right. At this point we also start generalizing from \isa{a}'s
and \isa{b}'s to an arbitrary property \isa{P}. Otherwise we would have
to prove the desired lemma twice, once as stated above and once with the
roles of \isa{a}'s and \isa{b}'s interchanged.%
\end{isamarkuptext}%
\isacommand{lemma}\ step{\isadigit{1}}{\isacharcolon}\ {\isachardoublequote}{\isasymforall}i\ {\isacharless}\ size\ w{\isachardot}\isanewline
\ \ {\isasymbar}{\isacharparenleft}int{\isacharparenleft}size{\isacharbrackleft}x{\isasymin}take\ {\isacharparenleft}i{\isacharplus}{\isadigit{1}}{\isacharparenright}\ w{\isachardot}\ P\ x{\isacharbrackright}{\isacharparenright}{\isacharminus}int{\isacharparenleft}size{\isacharbrackleft}x{\isasymin}take\ {\isacharparenleft}i{\isacharplus}{\isadigit{1}}{\isacharparenright}\ w{\isachardot}\ {\isasymnot}P\ x{\isacharbrackright}{\isacharparenright}{\isacharparenright}\isanewline
\ \ \ {\isacharminus}\ {\isacharparenleft}int{\isacharparenleft}size{\isacharbrackleft}x{\isasymin}take\ i\ w{\isachardot}\ P\ x{\isacharbrackright}{\isacharparenright}{\isacharminus}int{\isacharparenleft}size{\isacharbrackleft}x{\isasymin}take\ i\ w{\isachardot}\ {\isasymnot}P\ x{\isacharbrackright}{\isacharparenright}{\isacharparenright}{\isasymbar}\ {\isasymle}\ {\isacharhash}{\isadigit{1}}{\isachardoublequote}%
\begin{isamarkuptxt}%
\noindent
The lemma is a bit hard to read because of the coercion function
\isa{int\ {\isacharcolon}{\isacharcolon}\ nat\ {\isasymRightarrow}\ int}. It is required because \isa{size} returns
a natural number, but subtraction on type~\isa{nat} will do the wrong thing.
Function \isa{take} is predefined and \isa{take\ i\ xs} is the prefix of
length \isa{i} of \isa{xs}; below we also need \isa{drop\ i\ xs}, which
is what remains after that prefix has been dropped from \isa{xs}.

The proof is by induction on \isa{w}, with a trivial base case, and a not
so trivial induction step. Since it is essentially just arithmetic, we do not
discuss it.%
\end{isamarkuptxt}%
\isacommand{apply}{\isacharparenleft}induct\ w{\isacharparenright}\isanewline
\ \isacommand{apply}{\isacharparenleft}simp{\isacharparenright}\isanewline
\isacommand{by}{\isacharparenleft}force\ simp\ add{\isacharcolon}zabs{\isacharunderscore}def\ take{\isacharunderscore}Cons\ split{\isacharcolon}nat{\isachardot}split\ if{\isacharunderscore}splits{\isacharparenright}%
\begin{isamarkuptext}%
Finally we come to the above-mentioned lemma about cutting in half a word with two more elements of one sort than of the other sort:%
\end{isamarkuptext}%
\isacommand{lemma}\ part{\isadigit{1}}{\isacharcolon}\isanewline
\ {\isachardoublequote}size{\isacharbrackleft}x{\isasymin}w{\isachardot}\ P\ x{\isacharbrackright}\ {\isacharequal}\ size{\isacharbrackleft}x{\isasymin}w{\isachardot}\ {\isasymnot}P\ x{\isacharbrackright}{\isacharplus}{\isadigit{2}}\ {\isasymLongrightarrow}\isanewline
\ \ {\isasymexists}i{\isasymle}size\ w{\isachardot}\ size{\isacharbrackleft}x{\isasymin}take\ i\ w{\isachardot}\ P\ x{\isacharbrackright}\ {\isacharequal}\ size{\isacharbrackleft}x{\isasymin}take\ i\ w{\isachardot}\ {\isasymnot}P\ x{\isacharbrackright}{\isacharplus}{\isadigit{1}}{\isachardoublequote}%
\begin{isamarkuptxt}%
\noindent
This is proved by \isa{force} with the help of the intermediate value theorem,
instantiated appropriately and with its first premise disposed of by lemma
\isa{step{\isadigit{1}}}:%
\end{isamarkuptxt}%
\isacommand{apply}{\isacharparenleft}insert\ nat{\isadigit{0}}{\isacharunderscore}intermed{\isacharunderscore}int{\isacharunderscore}val{\isacharbrackleft}OF\ step{\isadigit{1}}{\isacharcomma}\ of\ {\isachardoublequote}P{\isachardoublequote}\ {\isachardoublequote}w{\isachardoublequote}\ {\isachardoublequote}{\isacharhash}{\isadigit{1}}{\isachardoublequote}{\isacharbrackright}{\isacharparenright}\isanewline
\isacommand{by}\ force%
\begin{isamarkuptext}%
\noindent

Lemma \isa{part{\isadigit{1}}} tells us only about the prefix \isa{take\ i\ w}.
An easy lemma deals with the suffix \isa{drop\ i\ w}:%
\end{isamarkuptext}%
\isacommand{lemma}\ part{\isadigit{2}}{\isacharcolon}\isanewline
\ \ {\isachardoublequote}{\isasymlbrakk}size{\isacharbrackleft}x{\isasymin}take\ i\ w\ {\isacharat}\ drop\ i\ w{\isachardot}\ P\ x{\isacharbrackright}\ {\isacharequal}\isanewline
\ \ \ \ size{\isacharbrackleft}x{\isasymin}take\ i\ w\ {\isacharat}\ drop\ i\ w{\isachardot}\ {\isasymnot}P\ x{\isacharbrackright}{\isacharplus}{\isadigit{2}}{\isacharsemicolon}\isanewline
\ \ \ \ size{\isacharbrackleft}x{\isasymin}take\ i\ w{\isachardot}\ P\ x{\isacharbrackright}\ {\isacharequal}\ size{\isacharbrackleft}x{\isasymin}take\ i\ w{\isachardot}\ {\isasymnot}P\ x{\isacharbrackright}{\isacharplus}{\isadigit{1}}{\isasymrbrakk}\isanewline
\ \ \ {\isasymLongrightarrow}\ size{\isacharbrackleft}x{\isasymin}drop\ i\ w{\isachardot}\ P\ x{\isacharbrackright}\ {\isacharequal}\ size{\isacharbrackleft}x{\isasymin}drop\ i\ w{\isachardot}\ {\isasymnot}P\ x{\isacharbrackright}{\isacharplus}{\isadigit{1}}{\isachardoublequote}\isanewline
\isacommand{by}{\isacharparenleft}simp\ del{\isacharcolon}append{\isacharunderscore}take{\isacharunderscore}drop{\isacharunderscore}id{\isacharparenright}%
\begin{isamarkuptext}%
\noindent
In the proof we have disabled the normally useful lemma
\begin{isabelle}
\isa{take\ n\ xs\ {\isacharat}\ drop\ n\ xs\ {\isacharequal}\ xs}
\rulename{append_take_drop_id}
\end{isabelle}
to allow the simplifier to apply the following lemma instead:
\begin{isabelle}%
\ \ \ \ \ {\isacharbrackleft}x{\isasymin}xs{\isacharat}ys{\isachardot}\ P\ x{\isacharbrackright}\ {\isacharequal}\ {\isacharbrackleft}x{\isasymin}xs{\isachardot}\ P\ x{\isacharbrackright}\ {\isacharat}\ {\isacharbrackleft}x{\isasymin}ys{\isachardot}\ P\ x{\isacharbrackright}%
\end{isabelle}

To dispose of trivial cases automatically, the rules of the inductive
definition are declared simplification rules:%
\end{isamarkuptext}%
\isacommand{declare}\ S{\isacharunderscore}A{\isacharunderscore}B{\isachardot}intros{\isacharbrackleft}simp{\isacharbrackright}%
\begin{isamarkuptext}%
\noindent
This could have been done earlier but was not necessary so far.

The completeness theorem tells us that if a word has the same number of
\isa{a}'s and \isa{b}'s, then it is in \isa{S}, and similarly 
for \isa{A} and \isa{B}:%
\end{isamarkuptext}%
\isacommand{theorem}\ completeness{\isacharcolon}\isanewline
\ \ {\isachardoublequote}{\isacharparenleft}size{\isacharbrackleft}x{\isasymin}w{\isachardot}\ x{\isacharequal}a{\isacharbrackright}\ {\isacharequal}\ size{\isacharbrackleft}x{\isasymin}w{\isachardot}\ x{\isacharequal}b{\isacharbrackright}\ \ \ \ \ {\isasymlongrightarrow}\ w\ {\isasymin}\ S{\isacharparenright}\ {\isasymand}\isanewline
\ \ \ {\isacharparenleft}size{\isacharbrackleft}x{\isasymin}w{\isachardot}\ x{\isacharequal}a{\isacharbrackright}\ {\isacharequal}\ size{\isacharbrackleft}x{\isasymin}w{\isachardot}\ x{\isacharequal}b{\isacharbrackright}\ {\isacharplus}\ {\isadigit{1}}\ {\isasymlongrightarrow}\ w\ {\isasymin}\ A{\isacharparenright}\ {\isasymand}\isanewline
\ \ \ {\isacharparenleft}size{\isacharbrackleft}x{\isasymin}w{\isachardot}\ x{\isacharequal}b{\isacharbrackright}\ {\isacharequal}\ size{\isacharbrackleft}x{\isasymin}w{\isachardot}\ x{\isacharequal}a{\isacharbrackright}\ {\isacharplus}\ {\isadigit{1}}\ {\isasymlongrightarrow}\ w\ {\isasymin}\ B{\isacharparenright}{\isachardoublequote}%
\begin{isamarkuptxt}%
\noindent
The proof is by induction on \isa{w}. Structural induction would fail here
because, as we can see from the grammar, we need to make bigger steps than
merely appending a single letter at the front. Hence we induct on the length
of \isa{w}, using the induction rule \isa{length{\isacharunderscore}induct}:%
\end{isamarkuptxt}%
\isacommand{apply}{\isacharparenleft}induct{\isacharunderscore}tac\ w\ rule{\isacharcolon}\ length{\isacharunderscore}induct{\isacharparenright}%
\begin{isamarkuptxt}%
\noindent
The \isa{rule} parameter tells \isa{induct{\isacharunderscore}tac} explicitly which induction
rule to use. For details see \S\ref{sec:complete-ind} below.
In this case the result is that we may assume the lemma already
holds for all words shorter than \isa{w}.

The proof continues with a case distinction on \isa{w},
on whether \isa{w} is empty or not.%
\end{isamarkuptxt}%
\isacommand{apply}{\isacharparenleft}case{\isacharunderscore}tac\ w{\isacharparenright}\isanewline
\ \isacommand{apply}{\isacharparenleft}simp{\isacharunderscore}all{\isacharparenright}%
\begin{isamarkuptxt}%
\noindent
Simplification disposes of the base case and leaves only a conjunction
of two step cases to be proved:
if \isa{w\ {\isacharequal}\ a\ {\isacharhash}\ v} and \begin{isabelle}%
\ \ \ \ \ length\ {\isacharbrackleft}x{\isasymin}v\ {\isachardot}\ x\ {\isacharequal}\ a{\isacharbrackright}\ {\isacharequal}\ length\ {\isacharbrackleft}x{\isasymin}v\ {\isachardot}\ x\ {\isacharequal}\ b{\isacharbrackright}\ {\isacharplus}\ {\isadigit{2}}%
\end{isabelle} then
\isa{b\ {\isacharhash}\ v\ {\isasymin}\ A}, and similarly for \isa{w\ {\isacharequal}\ b\ {\isacharhash}\ v}.
We only consider the first case in detail.

After breaking the conjunction up into two cases, we can apply
\isa{part{\isadigit{1}}} to the assumption that \isa{w} contains two more \isa{a}'s than \isa{b}'s.%
\end{isamarkuptxt}%
\isacommand{apply}{\isacharparenleft}rule\ conjI{\isacharparenright}\isanewline
\ \isacommand{apply}{\isacharparenleft}clarify{\isacharparenright}\isanewline
\ \isacommand{apply}{\isacharparenleft}frule\ part{\isadigit{1}}{\isacharbrackleft}of\ {\isachardoublequote}{\isasymlambda}x{\isachardot}\ x{\isacharequal}a{\isachardoublequote}{\isacharcomma}\ simplified{\isacharbrackright}{\isacharparenright}\isanewline
\ \isacommand{apply}{\isacharparenleft}clarify{\isacharparenright}%
\begin{isamarkuptxt}%
\noindent
This yields an index \isa{i\ {\isasymle}\ length\ v} such that
\begin{isabelle}%
\ \ \ \ \ length\ {\isacharbrackleft}x{\isasymin}take\ i\ v\ {\isachardot}\ x\ {\isacharequal}\ a{\isacharbrackright}\ {\isacharequal}\ length\ {\isacharbrackleft}x{\isasymin}take\ i\ v\ {\isachardot}\ x\ {\isacharequal}\ b{\isacharbrackright}\ {\isacharplus}\ {\isadigit{1}}%
\end{isabelle}
With the help of \isa{part{\isadigit{2}}} it follows that
\begin{isabelle}%
\ \ \ \ \ length\ {\isacharbrackleft}x{\isasymin}drop\ i\ v\ {\isachardot}\ x\ {\isacharequal}\ a{\isacharbrackright}\ {\isacharequal}\ length\ {\isacharbrackleft}x{\isasymin}drop\ i\ v\ {\isachardot}\ x\ {\isacharequal}\ b{\isacharbrackright}\ {\isacharplus}\ {\isadigit{1}}%
\end{isabelle}%
\end{isamarkuptxt}%
\ \isacommand{apply}{\isacharparenleft}drule\ part{\isadigit{2}}{\isacharbrackleft}of\ {\isachardoublequote}{\isasymlambda}x{\isachardot}\ x{\isacharequal}a{\isachardoublequote}{\isacharcomma}\ simplified{\isacharbrackright}{\isacharparenright}\isanewline
\ \ \isacommand{apply}{\isacharparenleft}assumption{\isacharparenright}%
\begin{isamarkuptxt}%
\noindent
Now it is time to decompose \isa{v} in the conclusion \isa{b\ {\isacharhash}\ v\ {\isasymin}\ A}
into \isa{take\ i\ v\ {\isacharat}\ drop\ i\ v},%
\end{isamarkuptxt}%
\ \isacommand{apply}{\isacharparenleft}rule{\isacharunderscore}tac\ n{\isadigit{1}}{\isacharequal}i\ \isakeyword{and}\ t{\isacharequal}v\ \isakeyword{in}\ subst{\isacharbrackleft}OF\ append{\isacharunderscore}take{\isacharunderscore}drop{\isacharunderscore}id{\isacharbrackright}{\isacharparenright}%
\begin{isamarkuptxt}%
\noindent
(the variables \isa{n{\isadigit{1}}} and \isa{t} are the result of composing the
theorems \isa{subst} and \isa{append{\isacharunderscore}take{\isacharunderscore}drop{\isacharunderscore}id})
after which the appropriate rule of the grammar reduces the goal
to the two subgoals \isa{take\ i\ v\ {\isasymin}\ A} and \isa{drop\ i\ v\ {\isasymin}\ A}:%
\end{isamarkuptxt}%
\ \isacommand{apply}{\isacharparenleft}rule\ S{\isacharunderscore}A{\isacharunderscore}B{\isachardot}intros{\isacharparenright}%
\begin{isamarkuptxt}%
Both subgoals follow from the induction hypothesis because both \isa{take\ i\ v} and \isa{drop\ i\ v} are shorter than \isa{w}:%
\end{isamarkuptxt}%
\ \ \isacommand{apply}{\isacharparenleft}force\ simp\ add{\isacharcolon}\ min{\isacharunderscore}less{\isacharunderscore}iff{\isacharunderscore}disj{\isacharparenright}\isanewline
\ \isacommand{apply}{\isacharparenleft}force\ split\ add{\isacharcolon}\ nat{\isacharunderscore}diff{\isacharunderscore}split{\isacharparenright}%
\begin{isamarkuptxt}%
The case \isa{w\ {\isacharequal}\ b\ {\isacharhash}\ v} is proved analogously:%
\end{isamarkuptxt}%
\isacommand{apply}{\isacharparenleft}clarify{\isacharparenright}\isanewline
\isacommand{apply}{\isacharparenleft}frule\ part{\isadigit{1}}{\isacharbrackleft}of\ {\isachardoublequote}{\isasymlambda}x{\isachardot}\ x{\isacharequal}b{\isachardoublequote}{\isacharcomma}\ simplified{\isacharbrackright}{\isacharparenright}\isanewline
\isacommand{apply}{\isacharparenleft}clarify{\isacharparenright}\isanewline
\isacommand{apply}{\isacharparenleft}drule\ part{\isadigit{2}}{\isacharbrackleft}of\ {\isachardoublequote}{\isasymlambda}x{\isachardot}\ x{\isacharequal}b{\isachardoublequote}{\isacharcomma}\ simplified{\isacharbrackright}{\isacharparenright}\isanewline
\ \isacommand{apply}{\isacharparenleft}assumption{\isacharparenright}\isanewline
\isacommand{apply}{\isacharparenleft}rule{\isacharunderscore}tac\ n{\isadigit{1}}{\isacharequal}i\ \isakeyword{and}\ t{\isacharequal}v\ \isakeyword{in}\ subst{\isacharbrackleft}OF\ append{\isacharunderscore}take{\isacharunderscore}drop{\isacharunderscore}id{\isacharbrackright}{\isacharparenright}\isanewline
\isacommand{apply}{\isacharparenleft}rule\ S{\isacharunderscore}A{\isacharunderscore}B{\isachardot}intros{\isacharparenright}\isanewline
\ \isacommand{apply}{\isacharparenleft}force\ simp\ add{\isacharcolon}min{\isacharunderscore}less{\isacharunderscore}iff{\isacharunderscore}disj{\isacharparenright}\isanewline
\isacommand{by}{\isacharparenleft}force\ simp\ add{\isacharcolon}min{\isacharunderscore}less{\isacharunderscore}iff{\isacharunderscore}disj\ split\ add{\isacharcolon}\ nat{\isacharunderscore}diff{\isacharunderscore}split{\isacharparenright}%
\begin{isamarkuptext}%
We conclude this section with a comparison of our proof with 
Hopcroft\index{Hopcroft, J. E.} and Ullman's\index{Ullman, J. D.}
\cite[p.\ts81]{HopcroftUllman}.
For a start, the textbook
grammar, for no good reason, excludes the empty word, thus complicating
matters just a little bit: they have 8 instead of our 7 productions.

More importantly, the proof itself is different: rather than
separating the two directions, they perform one induction on the
length of a word. This deprives them of the beauty of rule induction,
and in the easy direction (correctness) their reasoning is more
detailed than our \isa{auto}. For the hard part (completeness), they
consider just one of the cases that our \isa{simp{\isacharunderscore}all} disposes of
automatically. Then they conclude the proof by saying about the
remaining cases: ``We do this in a manner similar to our method of
proof for part (1); this part is left to the reader''. But this is
precisely the part that requires the intermediate value theorem and
thus is not at all similar to the other cases (which are automatic in
Isabelle). The authors are at least cavalier about this point and may
even have overlooked the slight difficulty lurking in the omitted
cases.  Such errors are found in many pen-and-paper proofs when they
are scrutinized formally.%
\index{grammars!defining inductively|)}%
\end{isamarkuptext}%
\end{isabellebody}%
%%% Local Variables:
%%% mode: latex
%%% TeX-master: "root"
%%% End:


\index{inductive definitions|)}
