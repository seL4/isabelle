
\chapter{Theorems and Forward Proof}

\section{Proof terms}\label{sec:proofObjects}

\subsection{Reconstructing and checking proof terms}\label{sec:reconstruct_proofs}
\index{proof terms!reconstructing}
\index{proof terms!checking}

When looking at the above datatype of proofs more closely, one notices that
some arguments of constructors are {\it optional}. The reason for this is that
keeping a full proof term for each theorem would result in enormous memory
requirements. Fortunately, typical proof terms usually contain quite a lot of
redundant information that can be reconstructed from the context. Therefore,
Isabelle's inference kernel creates only {\em partial} (or {\em implicit})
\index{proof terms!partial} proof terms, in which
all typing information in terms, all term and type labels of abstractions
{\tt AbsP} and {\tt Abst}, and (if possible) some argument terms of
\verb!%! are omitted. The following functions are available for
reconstructing and checking proof terms:
\begin{ttbox}
Reconstruct.reconstruct_proof :
  Sign.sg -> term -> Proofterm.proof -> Proofterm.proof
Reconstruct.expand_proof :
  Sign.sg -> string list -> Proofterm.proof -> Proofterm.proof
ProofChecker.thm_of_proof : theory -> Proofterm.proof -> thm
\end{ttbox}

\begin{ttdescription}
\item[Reconstruct.reconstruct_proof $sg$ $t$ $prf$]
turns the partial proof $prf$ into a full proof of the
proposition denoted by $t$, with respect to signature $sg$.
Reconstruction will fail with an error message if $prf$
is not a proof of $t$, is ill-formed, or does not contain
sufficient information for reconstruction by
{\em higher order pattern unification}
\cite{nipkow-patterns, Berghofer-Nipkow:2000:TPHOL}.
The latter may only happen for proofs
built up ``by hand'' but not for those produced automatically
by Isabelle's inference kernel.
\item[Reconstruct.expand_proof $sg$
  \ttlbrack$name@1$, $\ldots$, $name@n${\ttrbrack} $prf$]
expands and reconstructs the proofs of all theorems with names
$name@1$, $\ldots$, $name@n$ in the (full) proof $prf$.
\item[ProofChecker.thm_of_proof $thy$ $prf$] turns the (full) proof
$prf$ into a theorem with respect to theory $thy$ by replaying
it using only primitive rules from Isabelle's inference kernel.
\end{ttdescription}

\subsection{Parsing and printing proof terms}
\index{proof terms!parsing}
\index{proof terms!printing}

Isabelle offers several functions for parsing and printing
proof terms. The concrete syntax for proof terms is described
in Fig.\ts\ref{fig:proof_gram}.
Implicit term arguments in partial proofs are indicated
by ``{\tt _}''.
Type arguments for theorems and axioms may be specified using
\verb!%! or ``$\cdot$'' with an argument of the form {\tt TYPE($type$)}
(see \S\ref{sec:basic_syntax}).
They must appear before any other term argument of a theorem
or axiom. In contrast to term arguments, type arguments may
be completely omitted.
\begin{ttbox}
ProofSyntax.read_proof : theory -> bool -> string -> Proofterm.proof
ProofSyntax.pretty_proof : Sign.sg -> Proofterm.proof -> Pretty.T
ProofSyntax.pretty_proof_of : bool -> thm -> Pretty.T
ProofSyntax.print_proof_of : bool -> thm -> unit
\end{ttbox}
\begin{figure}
\begin{center}
\begin{tabular}{rcl}
$proof$  & $=$ & {\tt Lam} $params${\tt .} $proof$ ~~$|$~~
                 $\Lambda params${\tt .} $proof$ \\
         & $|$ & $proof$ \verb!%! $any$ ~~$|$~~
                 $proof$ $\cdot$ $any$ \\
         & $|$ & $proof$ \verb!%%! $proof$ ~~$|$~~
                 $proof$ {\boldmath$\cdot$} $proof$ \\
         & $|$ & $id$ ~~$|$~~ $longid$ \\\\
$param$  & $=$ & $idt$ ~~$|$~~ $idt$ {\tt :} $prop$ ~~$|$~~
                 {\tt (} $param$ {\tt )} \\\\
$params$ & $=$ & $param$ ~~$|$~~ $param$ $params$
\end{tabular}
\end{center}
\caption{Proof term syntax}\label{fig:proof_gram}
\end{figure}
The function {\tt read_proof} reads in a proof term with
respect to a given theory. The boolean flag indicates whether
the proof term to be parsed contains explicit typing information
to be taken into account.
Usually, typing information is left implicit and
is inferred during proof reconstruction. The pretty printing
functions operating on theorems take a boolean flag as an
argument which indicates whether the proof term should
be reconstructed before printing.

%%% Local Variables: 
%%% mode: latex
%%% TeX-master: "ref"
%%% End: 
