
\chapter{Tactics} \label{tactics}
\index{tactics|(}

\section{Other basic tactics}

\subsection{Composition: resolution without lifting}
\index{tactics!for composition}
\begin{ttbox}
compose_tac: (bool * thm * int) -> int -> tactic
\end{ttbox}
{\bf Composing} two rules means resolving them without prior lifting or
renaming of unknowns.  This low-level operation, which underlies the
resolution tactics, may occasionally be useful for special effects.
A typical application is \ttindex{res_inst_tac}, which lifts and instantiates a
rule, then passes the result to {\tt compose_tac}.
\begin{ttdescription}
\item[\ttindexbold{compose_tac} ($flag$, $rule$, $m$) $i$] 
refines subgoal~$i$ using $rule$, without lifting.  The $rule$ is taken to
have the form $\List{\psi@1; \ldots; \psi@m} \Imp \psi$, where $\psi$ need
not be atomic; thus $m$ determines the number of new subgoals.  If
$flag$ is {\tt true} then it performs elim-resolution --- it solves the
first premise of~$rule$ by assumption and deletes that assumption.
\end{ttdescription}

%%% Local Variables: 
%%% mode: latex
%%% TeX-master: "ref"
%%% End: 
